
%%%%%%%%%%%%%%%%%%%%%%%%%%%%%%%%%%%%%%%%%%%%%%%%%%%%%%%%%%%%%%%%%%%
\section{Ground state--excited state two-photon transition moments:
\Sec{CCSM}} \label{sec:ccsm}
%%%%%%%%%%%%%%%%%%%%%%%%%%%%%%%%%%%%%%%%%%%%%%%%%%%%%%%%%%%%%%%%%%%
This section describes the calculation of 
%second order 
two-photon transition strengths 
(two-photon dipole is a special case) 
within the Coupled Cluster response code.
The two-photon transition strength is defined
\[
S^{of}_{AB,CD}(\omega) = \frac{1}{2} \{ M^{AB}_{of}(-\omega) M^{CD}_{fo}(\omega)
                         +[M^{CD}_{of}(-\omega) M^{AB}_{fo}(\omega)]^\ast\}
\]
\Sec{CCSM} drives the calculation of the left ($M^{XY}_{of}(\omega)$)
and right ($M^{XY}_{fo}(\omega)$) transition moments, and of the transition 
strength $S^{of}_{AB,CD}(\omega)$.
The methodology is implemented for the CCS, CC2 and CCSD models.
Results obtained using this functionality should cite~\cite{}.
\begin{description}
\item[\Key{OPERAT}] \verb| |\newline
\verb|READ (LUCMD,'(4A)') LABELA, LABELB, LABELC, LABELD|\newline
\verb|DO WHILE (LABELA(1:1).NE.'.' .AND. LABELA(1:1).NE.'*')|\newline
\verb|  READ (LUCMD,'(4A)') LABELA, LABELB, LABELC, LABELD|\newline
\verb|END DO|\newline
Read quadruples of operator labels.
For each of these operator quadruples the first residues of the quadratic response
function will be evaluated at all frequencies.
Operator pairs which do not correspond to symmetry allowed
combinations will be ignored during the calculation.
%
\item[\Key{DIPOLE}] 
Evaluate all symmetry allowed elements of the all--dipole tensor
of the first residues of quadratic response function
(81 components). In other words, all four operator labels 
are set equal to all possible cartesian components of 
the electric dipole moment operator (\verb+DIPLEN+ in Sec.~\Sec{*INTGRL}).
%
\item[\Key{PRINT }] \verb| |\newline
\verb|READ (LUCMD,*) IPRSM|\newline
Read print level. Default is 0.
%
\item[\Key{SELSTA}] \verb| | \newline
\verb|READ (LUCMD,'(A70)') LABHELP|\newline
\verb|DO WHILE (LABHELP(1:1).NE.'.' .AND. LABHELP(1:1).NE.'*')|\newline
\verb|  READ (LUCMD,'(3A)') IXSYM, IXST, SMFREQ|\newline
\verb|END DO| \newline
Select one or more excited states $f$ (among those specified
in \Sec{CCEXCI}), and the laser frequency $\omega$.
The symmetry (\verb+IXSYM+) and state number (\verb+IXST+)
within that symmetry are then given,
one pair (\verb|IXSYM,IXST|) per line, together with the
laser frequency \verb+SMFREQ+ (in atomic units).
%\verb+SMFREQ+ specifies the frequency of the two-photon 
%transition moment (in atomic units).

Default is all states specified in \Sec{CCEXCI}, and for each state 
a laser frequency equal to half the excitation energy.

%
\item[\Key{HALFFR}] 
Set the frequency argument for the two-photon transition moments
equal to  half the excitation energy to the final state $f$. Default,
if \verb+SMFREQ+ is not specified in \Key{SELSTA}.
%
\end{description}
