%%%%%%%%%%%%%%%%%%%%%%%%%%%%%%%%%%%%%%%%%%%%%%%%%%%%%%%%%%%%%%%%%%%
\section{Ground state first-order properties: \Sec{CCFOP}}
\label{sec:ccfop}
%%%%%%%%%%%%%%%%%%%%%%%%%%%%%%%%%%%%%%%%%%%%%%%%%%%%%%%%%%%%%%%%%%%

In this Section, the calculation of ground-state first-order
(one-electron) properties is described. The calculation
is evoked with the \Sec{CCFOP} flag followed by the appropriate
keywords as described in the list below. Note that \Sec{CCFOP}
assumes that the proper integrals are written on the
AOPROPER file, and one therefore has to set the correct property
integral keyword(s) in the \Sec{*INTEGRAL} input Section. For properties
that have both an electronic and a nuclear contribution, these will
be printed separately with a print level of 10 or above.

The calculation of first order properties is implemented for the 
coupled cluster models CCS (which gives SCF first order properties),
CC2, MP2 and CCSD.  
By default, the chosen properties include orbital relaxation contributions,
i.e. they are calculated from the relaxed CC (or MP) densities. 
To disabilitate orbital relaxation for the CC2 and CCSD models, 
see \Key{NONREL} below. Relaxation is always included for MP2. 
Note also that the present implementation does not allow for CC2 relaxed first-order
properties in the frozen core approximation (\Key{FROIMP}, see Sec.~\ref{chap:ccgeneral}).

For details on the implementation, see 
Refs.~\cite{Halkier:CCFOP,HalCor:CC2FOP}
Publications that report results obtained with this module
should cite Ref.\ \cite{Halkier:CCFOP}.

\begin{description}
\item[\Key{DIPMOM}] 
        Calculate the permanent molecular electric dipole moment
        (\verb+DIPLEN+ integrals).
%
\item[\Key{QUADRU}] 
        Calculate the permanent traceless molecular electric
        quadrupole moment (\verb+THETA+ integrals). Note that the
        origin is the origin of the coordinate system specified
        in the MOLECULE.INP file.
%
\item[\Key{NQCC  }] 
        Calculate the electric field gradients at the nuclei
        (\verb+EFGCAR+ integrals).
%
\item[\Key{SECMOM}] 
        Calculate the electronic second moment of charge
        (\verb+SECMOM+ integrals).
%
\item[\Key{RELCOR}] 
        Calculate scalar-relativistic one-electron
        corrections to the ground-state
        energy (\verb+DARWIN+ and \verb+MASSVELO+ integrals).
%
\item[\Key{ALLONE}] 
        Calculate all of the above-mentioned properties (all the
        above-mentioned property integrals are needed).
%
\item[\Key{DAR2EL}] 
        Calculate relativistic two-electron Darwin term.
%
\item[\Key{OPERAT}] \verb| |\newline
\verb|READ (LUCMD,'(1X,A8)') LABPROP|\newline
        Calculate the electronic contribution to the property defined
        by the operator label \verb+LABPROP+ (corresponding 
        \verb+LABPROP+ integrals needed).
%
\item[\Key{NONREL}] 
        Compute the properties using the unrelaxed CC densities instead
        of the default relaxed densities.
%
\item[\Key{TSTDEN}] 
        Calculate the CC energy using the two-electron CC density.
        Programmers keyword used for debugging purposes---Do not use.
%
\end{description}
