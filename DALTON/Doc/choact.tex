~%%%%%%%%%%%%%%%%%%%%%%%%%%%%%%%%%%%%%%%%%%%%%%%%%%%%%%%%%%%%%%%%%%%
\section{Definition of atomic subsystems by Cholesky decomposition of 
the density matrix: \Sec{CHOACT}}
\label{sec:choact}
%%%%%%%%%%%%%%%%%%%%%%%%%%%%%%%%%%%%%%%%%%%%%%%%%%%%%%%%%%%%%%%%%%%
\index{Subsystem}
\index{Cholesky decomposition-based methods}

In this Section, we describe the keywords controlling the algorithm to 
define atomic subspaces by Cholesky decomposing one-electron density 
matrices. In the present implementation, the correlated calculation
is restricted to the active calculation, while the inactive
space is kept frozen in such correlated calculation. We note that 
decomposition of two-electron integrals is not required.


\begin{center}
\fbox{
\parbox[h][\height][l]{12cm}{
\small
\noindent
{\bf Reference literature:}
\begin{list}{}{}
\item {A.~M.~J.~S{\'a}nchez~de~Mer{\'a}s}, H.~Koch, I.~G.~Cuesta, and L.~Boman
 \newblock {\em J.~Chem.~Phys.}, {\bf 132},\hspace{0.25em}204105, (2010).
\end{list}
}}
\end{center}

\begin{description}
\item[\Key{ACTFRE}]\verb| |\newline
\verb|READ (LUCMD,*) NACTFR|\newline
        In addition to the inactive atomic subspace, freeze also
        the NACTFR acitve orbitals with lowest orbital energies.
        (Default: 0).
%
\item[\Key{ATOMIC}] \verb| |\newline
\verb|READ (LUCMD,*) NACINP|\newline
\verb|READ (LUCMD,*) (LACINP(I), I=1,NACINP)|\newline
       Read in the number of active symmetry-independent centers. Next
       line lists the active atoms, which are numbered in the order
       of the standardized input \verb|DALTON.BAS|. A
       negative value of \verb|NACINP| allows for specifying a (long)
       list of active atoms in several lines. In this case, the
       line following the negative \verb|NACINP| gives the number of
       atomic centers listed below. This two lines group is repeated
       as many times as needed to reach the total number of elements
       \verb|-NACINP|. See \verb|choact_mp2_energy| testcase for an example.
%
\item[\Key{DIFADD}]  \verb| |\newline
\verb|DO ISYM = 1,NSYM |\newline
\verb|   READ(LUCMD,*) NEXTBS(ISYM) |\newline
\verb|   READ(LUCMD,*) (IEXTBS(I,ISYM), I=1,NEXTBS(ISYM)) |\newline
\verb|END DO |\newline
        Add to the atomic active space selected (diffuse) basis. 
        (Default: No extra basis is added to the active space).
%
\item[\Key{DOSPRE}]
        Calculate orbital spreads as described in 
        Ref.~\cite{ziolkowski2009}. This option is only
        implemented with no symmetry.
        (Default: \verb|.FALSE.|)
%
\item[\Key{FULDEC}] 
        Full decomposition of occupied and virtual densities as in 
        Ref.~\cite{aquilante2006}. Not suitable to define atomic
        subspaces. (Default: \verb|.FALSE.|)
%
\item[\Key{LIMLOC}]  \verb| |\newline
\verb|READ (LUCMD,*) (MXOCC(I), I=1,NSYM)|\newline
\verb|READ (LUCMD,*) (MXVIR(I), I=1,NSYM)|\newline
       Read maximum number of occupied and virtual Cholesky active
       orbitals. As default, these numbers are determined by the 
       thresholds of the Cholesly decomposition of the correspondent
       density matrices.
%
\item[\Key{MINSPR}] 
       Determine Cholesky orbitals to minimize orbital spread as 
       described in Ref.~\cite{ziolkowski2009}. This option is only
        implemented with no symmetry.
        (Default: \verb|.FALSE.|)
%
\item[\Key{NOSLDR}] 
        Decompose the whole density matrices in an atom-by-atom
        basis scheme and select afterwards according to the values
        given under keyword \verb|.ATOMIC| (See Reference literature
        for details). This option works only without symmetry.
        (Default: \verb|.FALSE.|)
%
\item[\Key{SELDIA}] \verb| |\newline
\verb|READ (LUCMD,*) NABSOC|\newline
\verb|READ (LUCMD,*) (LACBAS(I), I=1,NABSOC)|\newline
\verb|READ (LUCMD,*) NABSVI|\newline
\verb|READ (LUCMD,*) (LACBSV(I), I=1,NABSVI)|\newline
        Read list of specific diagonals (basis) to decompose in 
        occupied and virtual density matrices. It is assumed that
        the list of occupied active diagonals can fit in a single 
        line, but –provided that \verb|NABSVI| is less than zero–
        several lines can be used for virtual diagonals by
        using the same procedure than in option \verb|.ATOMIC|.
%
\item[\Key{THACOC}] \verb| |\newline
\verb|READ (LUCMD,*) THACOC|\newline
       Threshold for the decomposition of the occupied density matrix.
       (Default: 0.2)
%
\item[\Key{THACVI}] \verb| |\newline
\verb|READ (LUCMD,*) THACVI|\newline
       Threshold for the decomposition of the virtual
       pseudo-density matrix.  (Default: 0.2)
%
%
\end{description}
