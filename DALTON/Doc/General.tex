\chapter{General input module}\label{ch:general}

In this chapter the general structure of the input file for \dalton\
is described, as well as the possible input cards that can be entered in
the \Sec{*DALTON} input module. This input section should always begin an
input file for the \dalton\ program.

\section{General input to DALTON : \Sec{*DALTON}}\label{sec:general}

This input module describes the overall type of calculation that is to
be done. It also contains seven submodules describing
the control of the two different geometry optimization routines,
the control of the two different environment models, PCM and QM3,
the control of the reading of the molecule and basis set input
  in the ".mol" file,
tuning of the performance of parallel calculations,
as well as control of the general routines for calculating
numerical geometrical derivatives of molecular energies or selected
first- and second-order properties.

We note that \verb|**DALTON| identifies that what follows is Dalton general input.
This input section is compulsory to do a calculation, by default Dalton does nothing
(will stop with the error message:
{\tt End of file on DALTON.INP, no **DALTON input found}.
Lines before the first line with \verb|**DALTON| are ignored by \dalton .
The general input module stops at the next line stating with two starts: \verb|**anything|.

If more than one node, \dalton\ will automatically perform those modules in parallel
which are implemented in parallel, and Fock matrix calculations will be done AO-direct,
also if you don't specify \verb|.DIRECT|.
In contrast to previous releases of Dalton, it will not quit in the non-parallelized modules
but run them on the master node while the other nodes are idling.
The keyword \verb|.PARALLEL| of previous releases is still recognized for backward compatibility,
but it is not needed any more.
Note that in order to evaluate the parallelization efficiency\index{parallel efficiency},
a print level of at least 2 is needed in the \Sec{PARALLEL} submodule.

\begin{description}
%\item[\Key{ABACUS}] Alternative keyword to \verb|.PROPERTIES|. Obsolete, do not use.

\item[\Key{CHOLES}] The calculation will use Cholesky decomposed two-electron
integrals~\cite{choint}\index{Cholesky decomposition-based methods} where implemented.
At present, the use of decomposed integrals is only implemented for calculation of energy and
first and second order molecular properties for limited models.
In particular, it is possible to carry out Hartree-Fock
\index{SCF}\index{HF}\index{Hartree--Fock},
Density Functional Theory
\index{DFT}\index{Density-Functional Theory},
second-order M{\o}ller-Plesset perturbation theory
(MP2)\index{M{\o}ller-Plesset!second-order}\index{MP2}, and
CC2 coupled cluster calculations\index{CC2}\index{Coupled-Cluster}. Note that
the decomposition of differentiated integrals is not implemented yet and, therefore,
gradients cannot be computed. Moreover, Cholesky CCSD(T)~\index{CCSD(T)}
calculations, where the energy dominators are Cholesky decomposed,
are carried out using standard integrals (or in a direct fashion).


\item[\Key{DIRECT}] The calculation is to be done in a direct
manner\index{direct calculation}, that
is, the two-electron integrals\index{two-electron integral!direct} are to be
constructed ``on the fly''
and not written to disc as is the default. This keyword will only work
for SCF\index{SCF}\index{HF}\index{Hartree--Fock} wave functions,
Density Functional Theory
calculation\index{DFT}\index{Density-Functional Theory} and for
coupled cluster (CC) calculations\index{CC}\index{Coupled-Cluster}. In
HF and DFT calculations the
two-electron integrals (and differentiated
two-electron integrals) will not be written to disc in any part of the
calculation, whereas in the direct CC approach, the two-electron
integrals will be stored in three general-indexed batches, thus
requiring a loop over all basis functions for one of the two-electron
integral indices~\cite{directCC}.

\item[\Key{DOUGLAS-KROLL}] Include scalar relativistic effects by using the
Douglas-Kroll-Hess second-order (DKH2) transformed one-electron potential and kinetic energy Hamiltonian.

%\item[\Key{HERMIT}] Alternative keyword to \verb|.INTEGRALS|. Obsolete, do not use.

\item[\Key{INPTES}] Test the input of the \Sec{*DALTON INPUT} input
module. The program will abort after the completion of the input test,
and no calculations will be executed.

\item[\Key{INTEGRALS}] Invoke the {\her} and/or the {\eri} program module for generating molecular one-
and two-electron integrals.
See Chapter~\ref{ch:hermit} for the \her\ and
Section~\ref{ch:eri}\index{one-electron integral}\index{two-electron integral} for the \eri\ program module, respectively.

\item[\Key{ITERATION}]\verb| |\newline
\verb|READ (LUCMD, '(I5)') ITERNR|

Tells the program at which iteration to start the geometry optimization
using the \Sec{*WALK} module\index{iteration number!geometry, start}.
Note that this will {\em not} affect which molecule
input file that is going to be read, as this is handled by the
job script {\tt dalton}. It only determines what number the
output of the predicted molecular geometry will be, as well as where to start
writing information on the files containing information about an IRC
calculation (\verb|DALTON.IRC|) or a dynamical trajectory walk (\verb|DALTON.TRJ|).

\item[\Key{MAX IT}]\verb| |\newline
\verb|READ (LUCMD, '(I5)') ITERMX|

Change the maximum number of geometry iterations\index{iteration number!geometry, max}\index{geometry iterations!max}\index{geometry optimization!max number of iterations}\index{convergence!geometry, max iterations}
that can be done. Default
is~20.  For numerical
differentiation/vibrational averaging, the number of iterations will
be reset to $6N+1$ (where $N$ is the number of nuclei) as the number
of required iterations for these calculations are well
defined. This number has to be increased in Intrinsic Reaction Coordinate
(IRC)\index{IRC}\index{intrinsic reaction coordinate} or dynamical
trajectory studies\index{dynamics}. However, changing this variable will override this reset
option.

%\item[\Key{MINIMIZE}] Alternative keyword to \verb|.OPTIMIZE|. Obsolete,
%do not use.

\item[\Key{OPTIMIZE}] Perform a geometry optimization\index{geometry optimization!.OPTIMIZE module}.
The default is a geometry minimization\index{geometry minimization!.OPTIMIZE module}:
an optimization of the molecular geometry to a stationary point with no
negative Hessian eigenvalues
(a local minimum) will be done using the default first-order
methods. However, this may be changed using appropriate
keywords in the submodule \Sec{OPTIMIZE}, and we refer to examples in
the chapter on potential energy surfaces
(Chapter~\ref{ch:geometrywalks}), and subsection~\ref{subsec:minimize}
describing the input cards for the \Sec{OPTIMIZE} submodule for a more
detailed description of possible options.

\item[\Key{NEWTRA}]\index{integral transformation!"new"} Requests that the
two-electron integral transformation routines of Bj\o rn
Roos should be used during execution of the program, also if less than
256 orbitals. The "new" transformation is always used if more than 255
orbitals.

\item[\Key{NMDDRV}] Calls for a generalized numerical geometry
differentiation. These routines will take advantage of the full
molecular point group in order to minimize the number of point to be
calculated. What order of derivatives and whether any analytical
derivatives are to be used is determined in the \verb|*NMDDRV| input
module.

%% hjaaj Apr 2011: .PARALLEL is not needed any more, Dalton automatically perform the modules
%% in parallel if more than one MPI node is associated.
%\item[\Key{PARALLEL}] Requests that the calculation of two-electron
%integrals\index{two-electron integral}
%is to be done in parallel\index{parallel calculation}. This also
%implies that the calculation is
%done without writing two-electron integrals to disc. This keyword only
%applies to SCF\index{SCF}\index{HF}\index{Hartree--Fock} wave
%functions and DFT calculations\index{DFT}\index{Density Functional Theory}, but all two-electron integral
%evaluations in an SCF calculation will be done parallel as well as the integration of the exchange-correlation functionals and kernels. More details
%about the parallelization strategy in \dalton\ can be found
%in Ref.~\cite{pndjhapdkrthhkcpl253}.

%The keyword requires that the program has been installed and compiled
%with the appropriate preprocessor directives for an MPI
%installation\index{MPI}.
%
%, or the construction of a slave program for
%the PVM installation\index{PVM}.
%If MPI\index{MPI} is used as message passing\index{message passing}
%interface, no further keywords are
%needed, as the number of nodes will be set equal to the number of
%nodes asked for when submitting the job. However, if PVM\index{PVM} is used as
%message passing interface, the number of nodes needs to be given in the
%\Sec{PARALLEL} submodule input, and this number have to be equal to
%the number of nodes asked for when submitting the job.
%Note that in order to evaluate the parallelization efficiency\index{parallel efficiency},
%a print level of at least 2 is needed in the \Sec{PARALLEL} submodule.

%% hjaaj Apr 2011: .PARNMD activates PARIO, which is not functional in current version.
%\item[\Key{PARNMD}] Option non-functional. Do not use.

%% hjaaj Apr 2011: Dalton now automatically selects the "new" integral transformation,
%% when more than 255 basis functions.
%\item[\Key{PRESORT}]\index{integral sort} Requests that the
%two-electron integrals should be
%sorted and that the integral transformation routines of Bj\o rn
%Roos should be used during execution of the program.
%The keyword is needed if one attempts an MCSCF, CI or MP2 (run through the \sir\ module and not using the Coupled-Cluster module) with more than 255 basis functions.

\item[\Key{PRIERR}]\verb| |\newline
\verb|READ (LUCMD, *) IPRERR|
\index{error print level} \index{print level!error}
Reads in the print level that is to be used in the DALTON.ERR
file. Default print level is \verb|IPRUSR+1|.

\item[\Key{PRINT}]\verb| |\newline
\verb|READ (LUCMD, *) IPRUSR|
\index{print level!general}
Reads in the print level that is to be used the rest of the subsequent
calculations. Default is a print level of 0.

\item[\Key{PROPERTIES}] Invoke the \aba\ program module for the evaluation of static
and dynamic properties, using a previously converged and saved wave function. See Chapter~\ref{ch:abacus}.

\item[\Key{RESPONSE}]

Invoke the \resp\ program module for the evaluation of static and dynamic
properties, using a previously converged and saved wave function. See Chapter~\ref{ch:response}.

\item[\Key{RUN ALL}]

Invoke all the program modules {\her}, {\sir}, {\resp}, and {\aba} for a single point
     calculation.

\item[\Key{RUN PROPERTIES}]

Invoke the program modules {\her}, {\sir}, and {\aba} for a single point
calculation.

\item[\Key{RUN RESPONSE}]

Invoke all the program modules {\her}, {\sir}, and {\resp} for a single point
calculation.

\item[\Key{RUN WAVE FUNCTIONS}]

Invoke the program modules {\her} and {\sir} for a single point energy calculation.

%\item[\Key{RUNABA}] Alternative keyword to \verb|.RUN PROPERTIES|. Obsolete, do not use.

\item[\Key{RUNERI}] Force the use of the vectorized 2-electron integral code {\eri} where possible.

%\item[\Key{RUNRSP}] Alternative keyword to \verb|.RUN RESPONSE|. Obsolete, do not use.

%\item[\Key{RUNSIR}] Alternative keyword to \verb|.RUN WAVE FUNCTIONS|. Obsolete, do not use.

\item[\Key{TOTSYM}] Consider only totally symmetric
perturbations\index{symmetry!only totally symmetric perturbations}.
This option only affects geometric perturbations calculated using the
second-order based \Key{WALK} option and static
electric-field perturbations requested through the keyword \Key{POLARI}.

\item[\Key{VECLEN}]\verb| |
\newline
\verb|READ(LUCMD,*) IVECLN|

Set the number of Fock matrices to be used during Fock-matrix constructions in
direct calculations. This function is only of interest for vector machines.
The default is 128. The larger the number, the more memory will be required
in the calculation.

\item[\Key{WALK}]
Do a geometry walk\index{geometry walk}\index{geometry optimization!.WALK module}.
If no input is given in the
\Sec{WALK} input submodule, an optimization of the molecular
geometry to a stationary point with no negative Hessian eigenvalues (a
local minimum) will be done using a second-order method with
analytical Hessians. However, this may be changed by appropriate
keywords in the submodule \Sec{WALK}, and we refer to examples in
the chapter on potential energy surfaces
(Chapter~\ref{ch:geometrywalks}), and subsection~\ref{sec:abawalk}
describing the input cards for the \Sec{WALK} submodule for a more
detailed description of possible options.

\item[\Key{WAVE FUNCTIONS}]
\index{SCF}\index{HF}\index{Hartree--Fock}\index{MP2}\index{MCSCF}\index{CC}
\index{NEVPT2}\index{GASCI}\index{DFT}
Invoke the {\sir} program module for the evaluation of SCF, MP2, Coupled
Cluster, MCSCF, NEVPT2, GASCI wave functions as well as DFT calculations.
Necessary one-electron integrals are already evaluated and saved on files.
See Chapter~\ref{chap:sirius-inpref}.
\end{description}

%hjaaj Feb 2003: next is redundant, do not mention but keep in code for now
%\subsection{End of General input: \Sec{END OF}}
%
%The last input card for the general input section (\Sec{*DALTON})
%may be \Sec{END OF}.

\subsection{Geometry optimization module 1: \Sec{OPTIMIZE}}\label{subsec:minimize}

This submodule is the usual driver for geometry optimizations\index{geometry optimization!.OPTIMIZE module},
but we note that the \Sec{WALK} module contains another second-order
geometry optimization driver for HF, DFT, and MCSCF.
Only for analytical second-order optimization of HF, DFT, and MCSCF can you choose
any of the two geometry optimization modules.
For the unique capabilities of the other module, see subsection~\ref{sec:abawalk}.

The \Sec{OPTIMIZE} module contains both first and second-order methods
for locating minima and transition states (geometry optimization).
\index{first-order geometry optimization}\index{geometry optimization!first-order}
\index{second-order geometry optimization}\index{geometry optimization!second-order}
Most of the Hessian updating\index{Hessian update} schemes were taken from
ref.\cite{thkrprt95} and \cite{Fletcher}.
The implementation of redundant internal coordinates
\index{redundant internal coordinates}\index{coordinate system!redundant internal coordinates}
follows the work of Peng {\it et al.\/}~\cite{cppyahbsmjfjcc17}.
In addition to this, several keywords for VRML visualization are included~\cite{VRML}.

\begin{description}

\item[\Key{1STORD}]
Use default first-order method\index{first-order geometry optimization}\index{geometry optimization!first-order}.
This means that the BFGS update will\index{BFGS update}\index{Hessian update!BFGS}
be used, and that the optimization is carried out in redundant internal
coordinates\index{redundant internal coordinates}. Same effect as the
combination of the two keywords \Key{BFGS} and \Key{REDINT}. Since the
\Key{BFGS} method ensures a positive definite Hessian, the
\Key{BOFILL} optimization method is used by default in case of
searches for transition states.

\item[\Key{2NDORD}]
Use default second-order method\index{second-order geometry optimization}\index{geometry optimization!second-order}.
Molecular Hessians will be calculated at every
geometry. The level-shifted Newton method and
Cartesian coordinates\index{Cartesian coordinates} are used. Identical
to specifying the keywords \Key{NEWTON} and \Key{CARTES}.

\item[\Key{BAKER}]
Activates the convergence criteria of Baker \cite{Baker}. The minimum
is then said to be found when the largest element of the gradient
vector (in Cartesian or redundant internal coordinates) falls below
$3.0\cdot 10^{-4}$ and either the energy change from the last
iteration is less than $1.0\cdot 10^{-6}$ or the largest element of
the predicted step vector is less $3.0\cdot 10^{-4}$.

\item[\Key{BFGS}]
Specifies the use of a first-order method\index{first-order geometry optimization}
with the Broyden-Fletcher-Goldfarb-Shanno (BFGS)
update\index{BFGS update}\index{Hessian update!BFGS}
formula for optimization. This is the
preferred first-order method for minimizations, as this update is able
to maintain a positive definite Hessian. Note that this also makes it
unsuitable for transitions state optimization (where one negative
eigenvalue is sought).

\item[\Key{BFGSR1}]
Use a linear combination of the BFGS and the symmetric rank one
updating schemes in the same fashion as Bofill's update. Only suitable
for minimizations.

\item[\Key{BOFILL}]
Bofill's update\cite{jmbjcc15}\index{Bofill's update}\index{Hessian update!Bofill's update}
is the default updating scheme for
transition state optimizations. It's a linear combination of the
symmetric rank one and the PSB updating schemes, automatically giving
more weight to PSB whenever the rank one potentially is numerically
unstable.

\item[\Key{CARTES}]
Indicates that Cartesian coordinates\index{Cartesian coordinates}\index{coordinate system!Cartesian coordinates}
should be used in the optimization. This is the default for
second-order methods.

\item[\Key{CMBMOD}]
Uses a combination of the BFGS update and the model Hessian (diagonal
in redundant internal coordinates). The two have equal weight in the
first iteration of the geometry optimization, then for each subsequent
iteration the weight of the model Hessian is halved. Only suitable for
minimizations.

\item[\Key{CONDIT}]\verb| |
\newline
\verb|READ (LUCMD,*) ICONDI|

Set the number of convergence criteria\index{convergence criteria!geometry}
that should be fulfilled before
convergence occurs. There are three different convergence thresholds,
one for the energy\index{geometry convergence criteria!energy change}, one for the gradient
norm\index{geometry convergence criteria!norm of gradient} and one for the step
norm\index{geometry convergence criteria!norm of step}.
The possible values for this variable is therefore between 1 and
3. Default is 2. The three convergence thresholds can be adjusted with
the keywords \Key{ENERGY}, \Key{GRADIE} and \Key{STEP T}.

\item[\Key{CONSTRAINT}]\verb| |\newline
\verb|READ (LUCMD, *) NCON|\newline
\verb|DO I = 1, NCON|\newline
\verb|   READ(LUCMD,*) ICON|\newline
\verb|   ICNSTR(ICON) = 1|\newline
\verb|END DO|

Request a constrained geometry optimization. Only works when using
redundant internal coordinates. The number of primitive coordinates
that should be frozen has to be specified (\verb|NCON|), then a list
follows with the individual coordinate numbers. The coordinate numbers
can be found by first running \dalton\ with the \Key{FINDRE}
keyword. Any number of bonds, angles and dihedral angles may be
frozen. NOTE: Symmetry takes precedence over constraints, if you
{\it e.g.\/} want to freeze just one of several symmetric bonds, symmetry
must be lowered or switched off.

\item[\Key{DELINT}]
Use delocalized internal coordinates\index{delocalized internal coordinates}\index{coordinate system!delocalized internal coordinates}.
These are built up as
non-redundant linear combinations of the redundant internal
coordinates. Performance is more or less the same as for the redundant
internals, but the transformation of displacements (step) is slightly
less stable.

\item[\Key{DFP}]
Specifies that a first-order\index{first-order geometry optimization} method with the
Davidon-Fletcher-Powell (DFP) update\index{DFP update}\index{Hessian update!DFP}
formula should be used for optimization. May be used for both
minimizations and transition state optimizations.

\item[\Key{DISPLA}]\verb| |\newline
\verb|READ (LUCMD,*) DISPLA|

Read one more line containing the norm of the displacement vector to
be used during numerical evaluation of the molecular gradient, as is
needed when doing geometry optimizations with CI or MP2 wave
functions. Default is $1.0\cdot 10^{-3}$ a.u.

\item[\Key{ENERGY}]\verb| |
\newline
\verb|READ(LUCMD,*) THRERG|

Set the convergence threshold for the energy (in a.u.). This is one of the three
convergence thresholds\index{convergence!geometry, criteria} (the keywords
\Key{GRADIE} and \Key{STEP T}
control the other two). Default value is the maximum of $1.0\cdot
10^{-6}$ and two times the threshold for the wave function gradient.

\item[\Key{FINDRE}]
Determines the redundant internal coordinate system then quits without
doing an actual calculation. Useful for setting up constrained
geometry optimizations, where the numbers of individual primitive
internal coordinates are needed.

\item[\Key{GDIIS}]
Use the Geometrical DIIS\cite{pcppjms114}\index{geometrical DIIS}
algorithm to control the
step. Works in much the same way as DIIS for wave functions. However,
the rational function and level-shifted Newton methods are generally
more robust and more efficient. Can only be used for minimizations.

\item[\Key{GEOANA}]
Enables an analysis of the molecular geometry in terms of bond lengths
and bond angles at each new geometry predicted during the optimization
procedure.

\item[\Key{GRADIE}]\verb| |
\newline
\verb|READ(LUCMD,*) THRGRD|

Set the convergence threshold for the norm of the molecular gradient (in
a.u.). This is one of
the three convergence thresholds (the keywords \Key{ENERGY} and
\Key{STEP T} control the other two). Default value is the maximum of
$1.0\cdot 10^{-4}$ and two times the threshold for the wave function
gradient.

\item[\Key{GRDINI}]
Specifies that the Hessian\index{Hessian!reinitialization} should be
reinitialized every time the norm
of the gradient is larger than norm of the gradient two iterations
earlier. This keyword should only be used when it's difficult to
obtain a good approximation to the Hessian during optimization. Only
applies to first-order methods\index{first-order geometry optimization}.

\item[\Key{HELLMA}]
Use gradients and Hessians calculated using the Hellmann-Feynman
approximation. Currently not working properly

\item[\Key{HESFIL}]
Specifies that the initial Hessian
\index{initial Hessian!first-order geometry optimization}\index{Hessian!molecular, initial}
should be read from the file \verb|DALTON.HES|. This applies to first-order
methods, and the Hessian in the file must have the correct
dimensions. This option overrides other options for the initial
Hessian.

Each time a Hessian is calculated or updated\index{Hessian update!geometry optimization},
it's written to this file (in Cartesian coordinates). If an
optimization is interrupted, it can be restarted with the
last geometry and the Hessian in \verb|DALTON.HES|, minimizing the
loss of information. Another useful possibility, is to transfer
the Hessian from a calculation on the same molecule with another
(smaller) basis and/or a cheaper wave function. Finally, one can go in
and edit the file directly to set up a specific force field.

\item[\Key{INIMOD}]
\index{model Hessian}\index{Hessian!model}\index{initial Hessian}\index{Hessian!initial}
Use a simple model Hessian~\cite{rlabgkpamcpl241} diagonal in redundant
internal coordinates as the initial Hessian. All diagonal elements are
determined based on an extremely simplified molecular mechanics model,
yet this model provides Hessians that are good starting points for
most systems, thus avoiding any calculation of the exact Hessian. This
is the default for first-order methods.

\item[\Key{INIRED}]
Specifies that the initial Hessian\index{initial Hessian}\index{Hessian!initial}
should be diagonal in redundant internal coordinates. The different diagonal
elements are set equal to 0.5 for bonds, 0.2 for angles and 0.1 for
dihedral angles, unless \Key{INITEV} has been specified. If the
optimization is run in Cartesian coordinates, the diagonal internal
Hessian is transformed to Cartesians. Only applies to first-order
methods\index{first-order geometry optimization}.

\item[\Key{INITEV}]\verb| |
\newline
\verb|READ(LUCMD,*) EVLINI|
\index{initial Hessian}\index{Hessian!initial}\index{Hessian!diagonal}
The default initial Hessian for first-order
minimizations\index{first-order geometry optimization} is the
identity matrix when Cartesian coordinates are used, and a diagonal
matrix when redundant internal coordinates are used. If \Key{INITEV}
is used, all the diagonal elements (and therefore the eigenvalues) are
set equal to the value EVLINI. This option only has effect when
first-order methods are used and \Key{INITHE} and \Key{HESFIL} are
non-present.

\item[\Key{INITHE}]
\index{initial Hessian}\index{Hessian!initial}
Specifies that the initial Hessian should be
calculated (analytical Hessian), thus yielding a first step that is
identical to that of second-order methods. This provides an excellent starting
point for first-order methods, but should only be used when the
Hessian can be calculated within a reasonable amount of time. It has only
effect for first-order methods and overrides the keywords
\Key{INITEV} and \Key{INIRED}. It has no effect when \Key{HESFIL} has
been specified.

%\item[\Key{INTERA}]
%Specifies a interactive run. The energy, gradient
%norm\index{norm of gradient}, step length\index{norm of step} and
%Hessian index\index{Hessian!index} is then written to the unit LUERR(=0)
%in each iteration, allowing easy monitoring of the optimization.

\item[\Key{LINE S}]
Turns on line searching\index{line search!geometry optimization}\index{geometry optimization!line search},
using a quartic polynomial. By default this
is turned off, as there seems to be no gain in efficiency. Can only be
used for minimizations.

\item[\Key{M-BFGS}]
A list of old geometries and gradients are kept. At each new point,
displacements and gradient difference for the last few steps are
calculated, and all of these are then used to sequentially update the
Hessian, the most weight being given to the last displacement and
gradient difference. Each update is done using the BFGS formula, and
it's thus only suitable for minimizations. Only applies to first-order
methods\index{first-order geometry optimization}.

\item[\Key{M-PSB}]
This identical to \Key{M-BFGS}, except the PSB formula is used for the
updating. Only applies to first-order methods\index{first-order
geometry optimization}, but it can be used for both minimizations and saddle
point optimizations.

\item[\Key{MAX IT}]\verb| |
\newline
\verb|READ(LUCMD,*) ITRMAX|

Read the maximum number of geometry iterations\index{geometry iteration}\index{geometry optimization!iterations}.
Default value is 25.

\item[\Key{MAX RE}]\verb| |
\newline
\verb|READ(LUCMD,*) MAXREJ|

Read maximum number of rejected steps\index{rejected geometry step}\index{geometry optimization!rejected step}
in each iterations, default is 3.

\item[\Key{MODE}]\verb| |
\newline
\verb|READ(LUCMD,*) NSPMOD|

Only has effect when doing saddle point optimizations\index{mode following}\index{geometry optimization!mode following}.
Determines which Hessian eigenmode should be maximized (inverted in the image
method). By default this is the mode corresponding to the lowest
eigenvalue, {\it i.e.\/} mode 1. If an optimization does not end up at
the correct transition state, it may be worthwhile following other modes
(only the lower ones are usually interesting).

\item[\Key{MODHES}]
Determine a new model Hessian (see \Key{INIMOD})
at every geometry without doing any updating. The model is thus used
in much the same manner as an exact Hessian, though it is obviously
only a relatively crude approximation to the analytical Hessian.

\item[\Key{NEWTON}]
Specifies that a second-order Newton method should be used for\index{geometry optimization!second-order}
optimization---that is, the analytical Hessian will be calculated at
every geometry. By default the level-shifted trust region method will
be used, but it is possible to override this by using one of the two
keywords \Key{RF} or \Key{GDIIS}.

\item[\Key{NOAUX}]
Only has effect when using redundant internal coordinates. The default
for minimizations is to add auxiliary bonds between atoms that are up
to two and half times further apart then regular (chemical)
bonds. This increases the redundancy of the coordinate system, but
usually speeds up the geometry optimization slightly. \Key{NOAUX}
turns this off. For saddle point optimizations and constrained
geometry optimization this is off by default (cannot be switched on).

\item[\Key{NOBREA}]
Disables breaking of symmetry\index{symmetry!breaking}\index{geometry optimization!symmetry breaking}.
The geometry will be optimized within the given symmetry, even if a non-zero Hessian
index is found. The default is to let the symmetry be broken until
a minimum is found with a Hessian index\index{Hessian!index} of
zero. This option only has effect when second-order methods are used.

\item[\Key{NODIHE}]
Only has effect when using redundant internal coordinates. Removes all
coordinates that are dihedral angles, leaving only bonds and
angles. Not too useful, but may be used if one wants to limit the
number of internal coordinates. Constrained geometry optimizations can
sometimes benefit from having all dihedral angles removed (assuming no
dihedral angles needs to be frozen).

%\item[\Key{NOREIN}]
%When first-order methods are used, the
%Hessian\index{Hessian!reinitialization} is reinitialized every
%time the Hessian index\index{Hessian!index} becomes non-zero (due to negative
%eigenvalues). This guarantees that the Hessian describes a minimum,
%but valuable information gathered in the Hessian may be
%lost.
%
%\Key{NOREIN}
%Disables this reinitialization, relying on the
%optimization method to restore the Hessian to its correct structure
%(by locating the area near the minimum). This option is particularly
%useful in conjunction with the keyword \Key{INITHE}, as it is usually
%not meaningful to calculate the Hessian at the initial geometry, then
%resetting it to the identity matrix because some negative eigenvalue
%showed up.

\item[\Key{NOTRUS}]
Turns off the trust radius, so that a full Newton step
is taken in each iteration. This should be used with caution, as
global convergence is no longer guaranteed. If long steps are desired,
it is safer to adjust the initial trust radius and the limits for the
actual/predicted energy ratio.

\item[\Key{PREOPT}]\verb| |\newline
\verb|READ (LUCMD,*) NUMPRE|\newline
\verb|DO I = 1, NUMPRE|\newline
\verb|   READ (LUCMD,*) PREBTX(I)|\newline
\verb|END DO|

First we read the number of basis sets\index{basis sets!preoptimization} that should be
used for preoptimization\index{preoptimization!basis sets}, then we read those
basis set names as strings. These
sets will be used for optimization in the order they appear in the
input. One should therefore place the smaller basis at the
top. After the preoptimization, optimization is performed with the
basis specified in the molecule input file.

\item[\Key{PRINT}]\verb| |
\newline
\verb|READ (LUCMD,*) IPRINT|

Set print level for this module.  Read one more line containing print
level. Default value is 0, any value higher than 12 gives debugging
level output.

\item[\Key{PSB}]
Specifies that a first-order method\index{first-order geometry optimization} with the
Powell-Symmetric-Broyden (PSB)\index{PSB update}\index{Hessian update!PSB}
update formula should be used for optimization.

%\item[\Key{QUADSD}]
%Not implemented.

\item[\Key{RANKON}]
Specifies that a first-order method with the rank one update\index{rank one update}\index{Hessian update!rank one}\index{Hessian update!MS}\index{Hessian update!SR1}
formula should be used for optimization. This updating is also
referred to as symmetric rank one (SR1) or Murtagh-Sargent (MS).

\item[\Key{REDINT}]
Specifies that redundant internal coordinates\index{redundant internal coordinates}\index{coordinate system!redundant internal coordinates}
should be used in the optimization. This is the default for
first-order methods\index{first-order optimization}.

\item[\Key{REJINI}]
Specifies that the Hessian should be
reinitialized\index{Hessian!reinitialization} after every
rejected step\index{rejected geometry step}, as a rejected step
indicates that the Hessian models the
true potential surface poorly. Only applies to first-order
methods\index{first-order geometry optimization}.

\item[\Key{REMOVE}]\verb| |\newline
\verb|READ (LUCMD, *) NREM|\newline
\verb|DO I = 1, NREM|\newline
\verb|   READ(LUCMD,*) IREM|\newline
\verb|   ICNSTR(IREM) = 2|\newline
\verb|END DO|

Only has effect when using redundant internal coordinates.
Specifies internal coordinates that should be removed. The input is
identical to the one for \Key{CONSTRAINT}, that is one has to specify
the number of coordinates that should be removed, then the number of
each of those internal coordinates. The coordinate numbers can first
be determined by running with \Key{FINDRE} set.

Removing certain coordinates can sometimes be useful in speeding up
constrained geometry optimization, as certain coordinates sometimes
``struggle'' against the constraints. See also \Key{NODIHE}.

\item[\Key{RF}]
Use the rational function method~\cite{abnajsrsjpc89} instead of
\index{rational function}
level-shifted Newton which is the default. The RF method is often
slightly faster than the level-shifted Newton, but also slightly less
robust.

For saddle point optimizations there's a special partitioned rational
function method (used automatically when both \Key{RF}
and \Key{SADDLE} are set). However, this method is both slower and
less stable than the default trust-region level-shifted image method
(which is the default).

\item[\Key{SADDLE}]
Indicates that a saddle point optimization should be performed rather\index{geometry optimization!transition state}\index{transition state}
than a minimization. The default method is to calculate the Hessian
analytically at the initial geometry, then update it using Bofill's\index{Bofill's update}\index{Hessian update!Bofill's update}
update. The optimization is performed in redundant internal
coordinates and using the trust-region level-shifted image method to
control the step. That is by default all the keywords \Key{INITHE},
\Key{BOFILL} and \Key{REDINT} are already set, but this can of course
be overridden by specifying other keywords. If locating the desired
transition state is difficult, and provided analytical Hessians are
available, it may sometimes be necessary to use the \Key{NEWTON}
keyword so that Hessians are calculated at every geometry.

\item[\Key{SCHLEG}]
Specifies that a first-order method\index{first-order geometry optimization}
with Schlegel's updating scheme\index{Schlegel update!geometry optimization}
\cite{Schlegel} should be used. This makes use of all previous
displacements and gradients, not just the last, to update the
Hessian.

\item[\Key{SP BAS}]\verb| |
\newline
\verb|READ(LUCMD,*) SPBSTX|

Read a string containing the name of a basis set. When the geometry
has converged, a single-point energy will be calculated using this
basis set\index{basis set!converged geometry}.

\item[\Key{STABILIZE}]\verb| |
\newline
\verb|READ(LUCMD,*) ISTBLZ|

Tries to ``stabilize'' the predicted new molecular geometries (and
thus reduce the risk of symmetry breaking) by ignoring all numbers
appearing in the Cartesian coordinates of the atoms beyond digit
number \verb|ISTBLZ|.

\item[\Key{STEEPD}]
Specifies that the first-order steepest descent\index{first-order geometry optimization}
\index{steepest descent} method should be used.
No update is done on the Hessian, so the optimization will be
guided by the gradient alone. The ``pure'' steepest descent method is
obtained when the Hessian is set equal to the identity matrix. Each
step will then be the negative of the gradient vector, and the
convergence towards the minimum will be extremely slow. However, this
option can be combined with other initial Hessians in Cartesian or
redundant internal coordinates\index{Cartesian coordinates}\index{redundant internal coordinates},
giving a method where the main feature is the lack of Hessian updates (static Hessian).

\item[\Key{STEP T}]\verb| |
\newline
\verb|READ(LUCMD,*) THRSTP|

Set the convergence\index{convergence!geometry, criteria} threshold for
the norm of the geometry step (in a.u.).
This is one of the three convergence thresholds (the keywords \Key{ENERGY} and
\Key{GRADIE} control the other two). Default value is $1.0\cdot
10^{-4}$.

\item[\Key{SYMTHR}]\verb| |
\newline
\verb|READ(LUCMD,*) THRSYM|

Determines the gradient threshold (in a.u.) for breaking of the
symmetry\index{symmetry!breaking}. That is, if the index of the
molecular Hessian\index{Hessian!index of molecular} is non-zero when the gradient norm
drops below this value, the symmetry is broken to avoid unnecessary
iterations within the wrong symmetry. This option only applies to
second-order\index{second-order geometry optimization} methods and when the
keyword \Key{NOBREA} is not present. The default value of this
threshold is $5.0\cdot 10^{-3}$.

\item[\Key{TR FAC}]\verb| |
\newline
\verb|READ(LUCMD,*) TRSTIN, TRSTDE|

Read two factors that will be used for increasing and decreasing the
trust radius\index{trust radius!geometry optimization} respectively. Default values are
1.2 and 0.7.

\item[\Key{TR LIM}]\verb| |
\newline
\verb|READ(LUCMD,*) RTENBD, RTENGD, RTRJMN, RTRJMX|

Read four limits for the ratio between the actual and predicted
energies. This ratio indicates how good the step is---that is, how
accurately the quadratic model describes the true energy
surface. If the ratio is below \verb|RTRJMN| or above
\verb|RTRJMX|, the step is rejected. With a ratio between
\verb|RTRJMN| and \verb|RTENBD|, the step is considered bad an the
trust radius\index{trust radius!geometry optimization} decreased to less than the step
length. Ratios between \verb|RTENBD| and \verb|RTENGD| are
considered satisfactory, the trust radius is set equal to the norm
of the step. Finally ratios above \verb|RTENGD| (but below
\verb|RTRJMX|) indicate a good step, and the trust radius is given
a value larger than the step length. The amount the trust radius
is increased or decreased can be adjusted with \Key{TR FAC}. The
default values of \verb|RTENBD|, \verb|RTENGD|, \verb|RTRJMN| and
\verb|RTRJMX| are 0.4, 0.8, -0.1 and 3.0 respectively.

\item[\Key{TRSTRG}]
Specifies that the level-shifted trust region method should be used to
control the step. This is the default, so the keyword is actually
redundant at the moment. Alternative step control methods are \Key{RF}
and \Key{GDIIS}.

\item[\Key{TRUSTR}]\verb| |
\newline
\verb|READ(LUCMD,*) TRSTRA|

Set initial trust radius\index{trust radius!geometry optimization} for calculation. This
will also be the
maximum step length for the first iteration. The trust radius is
updated after each iteration depending on the ratio between predicted
and actual energy change. The default trust radius is 0.5 a.u.

\item[\Key{VISUAL}]
Specifies that the molecule should be
visualized\index{visualization}\index{VRML}\index{geometry!visualization with VRML},
writing a VRML file of the molecular geometry. {\em{No optimization
will be performed when this keyword is given}}. See also related
keywords \Key{VR-BON}, \Key{VR-COR}, \Key{VR-EIG} and \Key{VR-VIB}.

\item[\Key{VRML}]
Specifies that the molecule should be
visualized\index{visualization}\index{VRML}. VRML files describing
both the initial and final geometry
will be written (as \verb|initial.wrl| and \verb|final.wrl|). The file
\verb|final.wrl| is  updated in each iteration, so that it always
reflects the latest geometry. See also related keywords \Key{VR-BON},
\Key{VR-COR}, \Key{VR-EIG} and \Key{VR-VIB}.

\item[\Key{VR-BON}]
Only has effect together with \Key{VRML} or \Key{VISUAL}. Specifies
that the VRML files should include bonds\index{bonded atoms} between nearby
atoms\index{visualization}\index{VRML}. The
bonds are drawn as grey cylinders, making it easier to see the
structure of the molecule. If \Key{VR-BON} is omitted, only the
spheres representing the different atoms will be drawn.

\item[\Key{VR-COR}]
Draws $x$-, $y$- and $z$-axis in the VRML scenes with
geometries. Somewhat useful if one is struggling to build a reasonable
geometry by adjusting coordinates manually.

\item[\Key{VR-EIG}]
Only has effect together with \Key{VRML} or
\Key{VISUAL}\index{visualization!eigenvectors}\index{VRML}.
Specifies that the eigenvectors of the molecule (that is the
eigenvectors of the Hessian, which differs from the normal modes as
they are not mass-scaled) should be visualized. These are written to
the files \verb|eigv_###.wrl|.

\item[\Key{VR-SYM}]
Draws in all symmetry elements of the molecule as vectors (rotational
axes) and semi-transparent planes (mirror planes).

\item[\Key{VR-VIB}]
Similar to \Key{VR-EIG}, but more useful as it draws the actual normal
mode vectors (the mass-weighted eigenvectors). These are written to
the files \verb|norm_###.wrl|. Keyword only has effect when a
vibrational analysis has been requested.

\end{description}


\subsection{Parallel calculations : \Sec{PARALLEL}}

This submodule controls the performance of the parallel
version\index{parallel calculation}
of \dalton . The implementation has been described
elsewhere~\cite{pndjhapdkrthhkcpl253}. \dalton\ only supports MPI as
message passing interface in the current release.
%When PVM\index{PVM} is used as
%message passing interface\index{message passing}, the
%keyword \Key{NODES} is required, otherwise all keywords are
%optional.

\begin{description}

\item[\Key{DEBUG}]\verb| |
Transfers the print level from the master\index{slave!debug}
to the slaves, otherwise the print level on the slaves will always be
zero. Only for debugging purposes.

\item[\Key{DEGREE}]\verb| |\newline
\verb|READ (LUCMD,*) NDEGDI|

Determines the percent of available tasks\index{parallel tasks} that
is to be distributed in
a given distribution of tasks, where a distribution of tasks is defined
as the process of giving batches to {\em all} slaves. The default is
5\% , which ensures that each slave will receive 20 tasks during one
integral evaluation, which will give a reasonable control with the
idle time of each slave.

%\item[\Key{ENCODE}]\verb| |\newline
%\verb|READ (LUCMD, '(A7)') WORD|
%
%This keyword only applies if PVM\index{PVM} has been chosen as message passing
%interface. Three different options exists:
%
%\begin{list}{--}{}
%\item\verb|PVMDEFA|
%\item\verb|PVMRAW|
%\item\verb|PVMINPL|
%\end{list}
%
%The default is \verb|PVMDEFA|. We refer to PVM\index{PVM} manuals for a
%closer description of the different ways of encoding the transfer of
%data between nodes.

\item[\Key{NODES}]\verb| |\newline
\verb|READ (LUCMD,*) NODES|

When MPI\index{MPI} is used as message passing interface, the
default value is the number of nodes that has been assigned to the
job, and these nodes will be partitioned into one master\index{master} and
\verb|NODES-1| slaves\index{slave}. In most cases the program will
find the number of nodes from the run-shell environment and setting
equal to the number of nodes requested when submitting the MPI job minus 1.
%When PVM is used for message
%passing\index{message passing}, the default is 0,  and it is
%therefore required to specify the number of nodes to be used in the
%calculation when PVM is used. \verb|NODES| will in such calculations
%correspond to the number of slaves the user would like to use in the
%calculation. Note that this number may often have to be adjusted in
%accordance with limits in the queuing system of various computers.

%\item[\Key{NTASK}]\verb| |\newline
%\verb|READ (LUCMD,*) NTASK|
%
%The number of tasks to send to each node when distributing the
%calculation of two-electron integrals\index{two-electron
%integral}. The default is 1. A task is
%defined as a shell of atomic integrals\index{basis functions, shell
%of}, a shell being an input
%block. One may therefore increase the number of shells given to each
%node in order to reduce the amount of communication. However, the
%program uses dynamical allocation of work to each
%node, and thus this option should be used with some care, as too large
%tasks may cause the dynamical load balancing to fail\index{load
%balancing}, giving an
%overall decrease in efficiency\index{parallel efficiency}. The
%parallelization is also very
%coarse grained, so that the amount of
%communication\index{communication} seldom represents
%any significant problem. Will be come obsolete, and replaced by \Key{DEGREE}.
%
\item[\Key{PRINT}]\verb| |\newline
\verb|READ (LUCMD, *) IPRPAR|

Read in the print level for the parallel calculation. A print level of
at least 2 is needed in order to be able to evaluate the
parallel efficiency\index{parallel efficiency}. A complete
timing for all nodes will be
given if the print level is 4 or higher.
\end{description}

\subsection{PCM environment model: \Sec{PCM}}
\label{sec:pcm}

Not documented yet. Examples can be found by grepping for "PCM" in the "dalton/test" directory in the distribution.

\subsection{QM/MM environment model: \Sec{QM3}}
\label{sec:qm3}

Not documented yet. Examples can be found by grepping for "QM3" in the "dalton/test" directory in the distribution.


\subsection{Geometry optimization module 2: \Sec{WALK}}
\label{sec:abawalk}

Directives controlling one of the two second-order
geometry\index{second-order geometry optimization}
optimizations  as well as the
execution of dynamical walks and numerical
differentiation\index{numerical differentiation} in
calculations of Raman intensities and optical activity\index{Raman
  intensity}\index{Raman optical activity}\index{ROA},
appear in the \Sec{WALK} section.

\begin{description}
\item[\Key{ANHARM}]
Requests that a determination of the cubic force field is to be determined.
By default this is done calculating numerical derivatives of analytical
Hessians in Cartesian coordinates.

\item[\Key{DISPLA}]\verb| |\newline
\verb|READ (LUCMD, *) DISPLC|

Displacement taken in a numerical differentiation\index{numerical
differentiation!geometry}. This applies both for a numerical molecular
Hessian\index{Hessian!numerical molecular Hessian}, as well as in calculation of Raman
intensities and optical activity\index{Raman intensity}\index{Raman optical activity}\index{ROA}.
Read one more
line specifying value~(*).  Default is~10$^{-4}$ a.u. However,
note that this variable do not determine the displacements used
when evaluating numerical gradient for use in first-order geometry
optimizations with MP2 or CI wave\index{MP2}\index{CI}\index{M{\o}ller-Plesset!second-order}\index{Configuration Interaction}
functions, which is controlled by the \Key{DISPLA} keyword in the
\Sec{OPTIMIZE} module.

\item[\Key{DYNAMI}]
Perform a ``dynamic walk''\index{dynamics}: integrate the
classical equations of motion\index{equation of motion} for the nuclei
analytically on a locally
quadratic surface. The method is discussed in
Ref.~\cite{theuhjajcpl173} as well as in Section~\ref{sec:dynamic}.

\item[\Key{ECKART}]\verb| |\newline
\verb| DO I = 1, NUCDEP|\newline
\verb|    READ (LUCMD,'(7X,F17.10,2F24.10)')|\newline
\verb|&        (ECKGEO(J,I), J = 1, 3)|\newline
\verb| END DO|

During a vibrational averaging, ensure that the properties are
transformed to the appropriate Eckart axis system.  The coordinate
system given should be that of the equilibrium geometry of the
molecule. The coordinates should be given in bohr, and should be given
for all symmetry generated atoms in the order given in the input.

\item[\Key{EIGEN}] Take a step to the boundary of the trust
region\index{trust radius}
along the eigenvector mode\index{eigenvector} specified by \Key{MODE}.

\item[\Key{FRAGME}]\verb| |\newline
\verb|READ (LUCMD, *) NIP|\newline
\verb|READ (LUCMD, *) (IPART(IP), IP = 1, NIP)|

Identify which fragments\index{molecular fragments} atoms belong to in
a dynamic walk.  Read one more line specifying the number of
atoms (the total number of atoms in the molecule), then one more
line identifying which fragment an atom belongs to. The atoms in the
molecule are given a number, different for each fragment. See also the
discussion in Sec.~\ref{sec:dynamic}.

\item[\Key{GRDEXT}] Perform a gradient extremal-based\index{gradient extremal} optimization. The algorithm used in this kind of optimization is
thoroughly described in Ref.\cite{pjhjajthtca73}. This is the default walk
type if the  index of the critical point searched is higher than
1. See also the discussion in Sec.~\ref{sec:gradext}.

\item[\Key{HARMON}]\verb| |\newline
\verb|READ (LUCMD, *) ANHFAC|

Threshold for harmonic dominance.  Read one
more line specifying value. Default is~100. This is  another
way of changing the criterion for changes of the trust
radius\index{trust radius}. See also the keyword \Key{TRUST}.

\item[\Key{IMAGE}] Locate a transition state\index{transition state} using a
trust-region-based image surface\index{image surface} minimization.
Note that only a
point with a Hessian index of~1 can currently be located with this
method, not higher-order stationary points. See also the discussion in
Sec.~\ref{sec:image}.

\item[\Key{INDEX}]\verb| |\newline
\verb|READ (LUCMD,*) IWKIND|

Desired Hessian index\index{Hessian!index} (strictly speaking, of the
totally symmetric block of the Hessian) at the optimized geometry.
Read one more line specifying value.  Default is~0 (minimum).
Note that a stationary point with the wrong Hessian index will not
be accepted as an optimized geometry.

\item[\Key{IRC}]\verb| |\newline
\verb|READ (LUCMD, *) IRCSGN|

Set the geometry walk to be  an
Intrinsic Reaction Coordinate (IRC)\index{intrinsic reaction coordinate}\index{IRC} as described in
Ref.~\cite{kfacr14}. Read one
more line containing the sign (-1 or 1) of the reaction coordinate. It
cannot be decided in advance which reaction pathway a specific sign is
associated with. See also the discussion in Sec.~\ref{sec:irc}.

%\item[\Key{ISOTOP}]\verb| |\newline
%\verb|READ (LUCMD, *) NIS|\newline
%\verb|READ (LUCMD, *) (ISOTPS(IS), IS = 1, NIS)|
%
%Specify the isotopic constitution\index{isotopic constitution} of the
%molecule under investigation.
%This is most interesting in dynamic walks\index{dynamics},
%when using mass-scaled atomic coordinates\index{mass-weighted coordinates}.
%Note that the internal structure of
%\aba\ uses Cartesian coordinates, and for vibrational analysis alone
%the keyword \Key{ISOTOP} in the \Sec{VIBANA} input section is to be
%used. Note also that for defining center of mass
%coordinates or for vibrational averaging, there is a similar \verb|.ISOTOP| keyword in the general
%input module, and this will become the only keyword for specifying
%isotopic substitutions in later versions of the program. The
%\Key{ISOTOP} keyword in this input module (\Sec{WALK}) is to become
%obsolete.

\item[\Key{KEEPSY}] Ensure that the symmetry of the molecule is not
broken. The threshold for determining a mode as breaking symmetry is
controlled by the keyword \Key{ZERGRD}.

\item[\Key{MASSES}] Mass-scale the atomic
coordinates\index{mass-weighted coordinates}.  This is the
default for dynamic walks\index{dynamics}, gradient
extremal\index{gradient extremal} walks and in calculations
of Intrinsic Reaction Coordinates
(IRCs)\index{intrinsic reaction coordinate}\index{IRC}.

\item[\Key{MAXNUC}]\verb| |\newline
\verb|READ (LUCMD, *) XMXNUC|

Maximum displacement\index{displacement of atom} allowed for any one
atom as a result of the geometry update.  Read one more line
specifying value.  Default is~0.5.

\item[\Key{MAXTRU}]\verb| |\newline
\verb|READ (LUCMD, *) TRUMX1|

Set the maximum arc length in an Intrinsic Reaction Coordinate (IRC)
walk\index{intrinsic reaction coordinate}\index{IRC}. Read one more
line containing the maximum arc length.
Default is~0.10. Note that this arc length is also affected by the
\Key{TRUST} keyword, and if both are specified, the arc length will
be set to the minimum value of these to.

\item[\Key{MODE}]\verb| |\newline
\verb|READ (LUCMD,*) IMODE|

Mode to follow in level-shifted Newton optimizations for transition
states\index{transition state}.  Read one more line specifying
mode. Default is to follow
the lowest mode (mode~1).

\item[\Key{MODFOL}] Perform a mode-following (level-shifted
Newton) optimization. This is the default for minimizations and
localization of transition states. See also discussion in
Section~\ref{sec:modfol}.

\item[\Key{MOMENT}]\verb| |\newline
\verb|   READ (LUCMD, *) NSTMOM|\newline
\verb|   DO IP = 1, NSTMOM|\newline
\verb|      READ (LUCMD, *) ISTMOM(IP), STRMOM(IP)|\newline
\verb|   END DO|

Initial momentum for a dynamic walk\index{dynamics}\index{momentum}.
Read one more line specifying
the number of modes to which there is added an initial momentum. Then
read one line for each of these modes, containing first the number of
the mode, and then the momentum. The default is to have no momentum.
See also the section describing how to perform a dynamic
walk, Sec.~\ref{sec:dynamic}.

\item[\Key{NATCON}] Use the natural connection\index{natural connection} when orthogonalizing
the predicted molecular orbitals at the new geometry. By default the
symmetric connection is used.

\item[\Key{NEWTON}] Use a strict
Newton--Raphson\index{Newton-Raphson step} step to update
the geometry. This means that no trust region will be used.

\item[\Key{NO CENTRIFUGAL FORCES}] Do not include contributions from
centrifugal forces when calculating vibrationally averaged geometries
at a finite temperature.

\item[\Key{NOGRAD}]\verb| |\newline
\verb|READ (LUCMD, *) NZEROG|\newline
\verb|READ (LUCMD, *) (IZEROG(I), I = 1, NZEROG)|

 Set some gradient elements  to zero.  Read
one more line specifying how many elements to zero, then one
or more lines listing their sequence numbers.

\item[\Key{NOORTH}] The predicted molecular orbitals at the new
geometry are {\em not} orthogonalized. Default is that the orbitals
are orthogonalized with the symmetric connection. Orthogonalization
can also be done with the natural
connection~\cite{joklbkrthpjtca90}. See the keyword \Key{NATCON}.

\item[\Key{NOPRED}] No prediction of the energy of the wave function
at the updated geometry.

\item[\Key{NORMAL}] Do the calculation of effective (vibrationally
averaged) geometries in normal coordinates. This will restrict the
calculation of the effective geometry to one isotopic species (by
default the most abundant one).

\item[\Key{NUMERI}] Do a numerical
differentiation\index{numerical differentiation}, for instance when
calculating Raman intensities or Raman optical activity, see
Sections~\ref{sec:ramanint} and~\ref{sec:vroa}.

\item[\Key{PRINT}]\verb| |\newline
\verb|READ (LUCMD,*) IPRWLK|

Set the print level in the prediction of new geometry steps.  Read one
more line containing print level. Default value is the value of
\verb|IPRDEF| in the general input module.

\item[\Key{RATLIM}]\verb| |\newline
\verb|READ (LUCMD, *) RTRMIN, RTRGOD, REJMIN, REJMAX|

Limits on ratios between predicted and
observed energy change.  Read one more line specifying four
values~(*).  These are respectively the bad prediction ratio, good
prediction ratio, low rejection ratio and high rejection ratio.
Defaults are~0.4, 0.8, 0.1, and 1.9.

\item[\Key{REJECT}] Signals that the previous geometry step was
rejected\index{rejected geometry step}, and the trust
region\index{trust radius} is
reduced. This keyword is
used in case of restarts to tell the program that when the program
was stopped, the last geometry was in fact rejected.

\item[\Key{REPS}]\verb| |\newline
\verb|READ (LUCMD, *) NREPS|\newline
\verb|READ (LUCMD, *) (IDOREP(I), I = 1, NREPS)|

Consider perturbations of selected
symmetries only.  Read one more line specifying how many
symmetries, then one line listing the desired symmetries. Note that
only those symmetries previously defined to be true with the keyword
\Key{REPS} from the \aba\ input modules will be calculated. This
keyword thus represents a subset of the \Key{REPS} of the general
input module.

\item[\Key{RESTART}] Tells the program that this is a restarted
geometry optimization\index{restart!geometry optimization} and that
information may therefore be available
on the \verb|DALTON.WLK| file.

\item[\Key{REUSE}] Use the property derivatives available on the file
\verb|DALTON.WLK| in a calculation of the harmonic contribution to the
vibrational average. In this case, only a new force field will be calculated.

\item[\Key{SCALE}]\verb| |\newline
\verb| READ (LUCMD, *) NUMNUC|\newline
\verb| DO INUC = 1, NUMNUC|\newline
\verb|    READ (LUCMD, *) IATOM,(SCALCO(J,IATOM), J = 1, 3)|\newline
\verb| END DO|

Scale the atomic coordinates.  Read one more
line specifying how many atoms to scale, then one line for
each of these atoms~(*) specifying the atom number and scale
factors for all three Cartesian coordinates. Default is no scaling of
the atomic coordinates.

%\item[\Key{STRICT}] Strict mode following. Obsolete keyword. Do not use.

\item[\Key{TEMPERATURES}]\verb| |\newline
\verb|  READ (LUCMD, *) NTEMP|\newline
\verb|  READ (LUCMD, *) (TEMP(ITMP), ITMP=1,NTEMP)|

Read a set of temperatures for which the
effective (rovibrationally averaged) geometries are to be
calculated. Read one more line containing the number of different
temperatures, and another line containing the list of temperatures.

\item[\Key{TOLERANCE}]\verb| |\newline
\verb|READ (LUCMD, *) TOLST|

Threshold for convergence of the geometry optimization (on gradient
norm).  Read one more line specifying the threshold~(*).  Default
is~10$^{-5}$.

\item[\Key{TRUST}]\verb| |\newline
\verb|READ (LUCMD, *) TRUSTR, TRUSTI, TRUSTD|

Trust region information\index{trust radius!.WALK geometry optimization}.  Read one more
line specifying three values~(*): initial trust radius, factor by
which radius can be incremented, and factor by which it can be
decremented.  Defaults are~0.5, 1.2 and 0.7, respectively; initial
trust radius default is~0.3 if desired Hessian index is
greater than zero. In dynamic walks\index{dynamics} the trust radius
is by default put
to 0.005, and in walks along an Intrinsic Reaction Coordinate
(IRC)\index{intrinsic reaction coordinate}\index{IRC} the
default trust radius is 0.02. For dynamical walks the default
increment and decrement factor is changed to~2.0 and~0.8
respectively.

\item[\Key{VIBAVE}]
Request the calculation of the harmonic contribution to the
vibrational average of a molecular property.

%\item[\Key{VIBCNV}] Requests a vibrational analysis at the optimized
%geometry. Obsolete keyword. Do not use.

\item[\Key{ZERGRD}]\verb| |\newline
\verb|READ (LUCMD, *) ZERGRD|

Threshold below which gradient elements are
treated as zero.  Read one more line specifying value~(*). Default
is~10$^{-5}$. This keyword is mainly used for judging which modes are
symmetry breaking when using the keyword \Key{KEEPSY} as well as
when deciding what step to take when starting a walk from a transition
state.
\end{description}

\subsection{Molecule geometry and basis sets, \Sec{MOLBAS}}\label{sec:herrdn}

The directives in this section extend or modify the reading of the molecule geometry and basis set
specifications in the ".mol" input file (internally \verb|MOLECULE.INP|).

\begin{description}
\item[\Key{CM FUN}]\verb| |\newline
\verb|READ (LUCMD,*) LCMMAX, CMSTR, CMEND|

Use Rydberg basis functions\index{Rydberg basis function}\index{basis function!Rydberg}
(center of mass functions\index{center of mass function}) as suggested by
Kaufman {\it et al.\/}~\cite{kkwbmjjpbamop22}. \verb|LCMMAX| denoted
the maximum quantum number of the Rydberg functions, basis functions
for all quantum up to and including \verb|LCMMAX| will be generated
(s=0, p=1 etc.) \verb|CMSTR| and \verb|CMEND| are the half-integer
start- and ending quantum number for the Rydberg basis functions. The
basis functions will be placed at the position of a dummy center
indicated as \verb|X| in the \verb|MOLECULE.INP| file. The charge of
the ion-core is determined by the keyword \Key{ZCMVAL} (default: +1).
If no center named \verb|X| is present in the \verb|MOLECULE.INP| file,
this input will be ignored.

\item[\Key{CONTINUUM}]\verb| |\newline
\verb|READ (LUCMD,*) LCNTMAX, CNTSTR, CNTEND|

%
% Introduced by Sonia Coriani, 2012
%
Use continuum-like basis functions\index{Continuum basis function}\index{basis function!Continuum}
(Gaussian continuum-like\index{Gaussian continuum basis}) as suggested by
Kaufman {\it et al.\/}~\cite{kkwbmjjpbamop22} (cf. eq. 20). \verb|LCNTMAX| denoted
the maximum quantum number of the continuum functions, basis functions
for all quantum up to and including \verb|LCNTMAX| will be generated
(s=0, p=1 etc.) \verb|CNTSTR| and \verb|CNTEND| are the integer
start- and ending quantum number for the continuum basis functions. The
basis functions will be placed at the position of a dummy center
indicated as \verb|X| in the \verb|MOLECULE.INP| file. 
No charge of the ion-core is required for these functions.
If no center named \verb|X| is present in the \verb|MOLECULE.INP| file,
this input will be ignored.

\item[\Key{MAXPRI}]\verb| |\newline
\verb|READ (LUCMD,*) MAXPRI|

Set maximum number of primitives\index{primitive orbitals!maximum number} in any
contraction.  Read one more line containing number.  Default
is~25, except for the Cray-T3D/E, where the default is 14.

%\item[\Key{MOLINP}] Indicates that the molecular input comes at the
%end of the current file.
%Default is that the molecular input is to be
%read from unit~9.

\item[\Key{OLDNORM}] Use the normalization scheme for spherical AO's used in {\dalton} 1.0.
From {\dalton} 1.1, all components of spherical AO's are normalized to 1.
This was not the case in {\dalton} 1.0, and this option is needed to read
MO's correctly from an old file generated by {\dalton} 1.0 if spherical AO's were used.
(Cartesian AO's are still not all normalized to 1 for $d$-orbitals and higher $l$'s.)

\item[\Key{PRINT}]\verb| |\newline
\verb|READ (LUCMD,*) IPREAD|

Set print level for input processing.  Read one more line containing
print level. Default is the \verb|IPRDEF| from the \Sec{*INTEGRALS} input
module.

\item[\Key{R12AUX}] An auxiliary basis is used.
Basis sets must be identified as either orbital basis or
auxiliary basis in the \verb|MOLECULE.INP| file (line 5).
See Sec.~\ref{sec:r12aux} on page~\pageref{sec:r12aux}.

\item[\Key{SYMTHR}]\verb| |\newline
\verb|READ (LUCMD,*) TOLLRN|

Read threshold for considering atoms to be related by symmetry. Used
in the automatic symmetry detection routines. Default is $5.0\cdot
10^{-6}$.

\item[\Key{UNCONT}] Force the program to use all input basis sets as primitive (completely decontracted) sets.

\item[\Key{WRTLIN}] Write out the lines read in during the input
  processing of the \verb|MOLECULE.INP| file. Primarily for debugging
  purposes or for analyzing input errors in the \verb|MOLECULE.INP|
  file.

\item[\Key{ZCMVAL}]\verb| |\newline
\verb|READ (LUCMD,*) ZCMVAL|

Read the charge of the ion-core center for the Rydberg basis functions specified
by the \Key{CM FUN} keyword. Default is a charge of one.

\end{description}

\section{Numerical differentiation : \Sec{*NMDDRV}}\label{sec:nmddrv}

This module can calculate any geometrical derivative of the energy
using either as high analytical derivatives as possible, or using the
specified level of analytical derivatives (assuming implemented for
the choice of wave function)~\cite{numder}. Also performs vibrational averaging over
selected first- and second-order molecular properties.

\begin{description}

\item[\Key{DISPLACEMENT}]\verb| |\newline
\verb|READ (LUCMD,*) DISPLC|

Reads in the step lengths (in atomic units) that is to be used in the
numerical differentiation scheme. Default is 1.0D-2.

\item[\Key{DORDR}]\verb| |\newline
\verb|READ (LUCMD,*) NMORDR, NAORDR|

Sets the numerical (NMORDR) and analytical (NAORDR) differentiating
order for calculating force constants. Current implementation has an
artificial boundary at the 5. numerical derivative (independent of the
analytical differentiation order). Notice that if you would like to
calculating 4.derivatives from analytical 2.derivatives the input
would be \verb|2  2|, since you would like to get 2.order numerical
derivatives from 2.order analytical derivatives.

\item[\Key{DRYRUN}]\verb| |\newline
\verb|READ (LUCMD,*) NMREDU|\newline
\verb|READ (LUCMD,*)| (KDRYRN(II),II=1,NMREDU)

The numerical derivatives will not be calculated, and the program will
just set up the required displacement. An optional number of redundant
coordinate displacements can be specified, corresponding to translational and
rotational degrees of freedom.

\item[\Key{HARMONIC FORCE FIELD}]\verb| |\newline

At the end of the calculation, perform a vibrational analysis using
the calculated Hessian matrix.

\item[\Key{MANUAL}]\verb| |\newline

Dump each individual geometry input file to the \verb|DALTON.OUT|
file. Primarily of interest for debugging purposes.

\item[\Key{NORMAL}]
Using this keyword the numerical differentiation will be carried out
with respect to normal coordinates. The program will then do the
necessary numerical differentiation to get the Hessian (and thus
normal coordinates), before it will carry on calculating the
higher-order force field requested with respect to the calculated normal
coordinates.

\item[\Key{PRECALCULATED HESSIAN}]\verb| |\newline

Use a precalculated Hessian available on the \verb|DALTON.HES| file
when defining normal coordinates. Only active in combination with the
keyword \Key{NORMAL}.

\item[\Key{PRINT}]\verb| |\newline
\verb|READ (LUCMD,*) IDRPRI|

Control the print level in the numerical derivative routines. Default
is the same as the general print level \verb|IPRUSR|.

\item[\Key{PROPAV}]\verb| |\newline

Indicates that a averaging over vibrational motions of a molecular
property (including the energy) is to be performed. The input
for what kind of a vibrational analysis is to be performed is
specified in the \Sec{PROPAV} module.

\item[\Key{PROPER}]\verb| |\newline
\verb|READ (LUCMD,*) NMRDRP, NARDRP|

Sets the numerical (NMRDRP) and analytical (NARDRP) differentiating
order for calculating geometrical derivatives of molecular
properties. Currently, the possible presence of analytical property
derivatives cannot be taken advantage of, and the default value of $0$
for \verb|NARDRP| should be used. Currently, dipole transition
strengths and vibrationally averaged spin--spin coupling constants
have been implemented, the former also for Coupled--Cluster wave
functions\index{CC}\index{Coupled-Cluster}. Notice that vibrationally
averaged spin--spin coupling constants also can be calculated using
the \Key{VIBANA} and \Sec{VIBANA} keywords and input section.

\item[\Key{RESTRT}]
Controls the restart procedure. If a calculation crashes during a
force constant calculation, there will be a file in the work directory
called RSTRT.FC. This will contain all of the information necessary to
restart the calculation. If this file is available and the keyword is
used, \dalton\ will attempt to restart the calculation.

\item[\Key{REUSE HESSIAN}]\verb| |\newline

If a (mixed) numerical Hessian has been calculated, it will be saved
in the file \verb|DALTON.HES| for possible future use.

\item[\Key{SDRTST}]\verb| |\newline
If analytical Hessian also has been requested and second-order
numerical derivatives, a comparison of the numerical and analytical
Hessian will be done. Primarily used for testing.

\item[\Key{SPECTRO INTERFACE}]\verb| |\newline

Write an interface file \verb|DALTON.SPC| containing force fields of
different orders suited for analysis with the SPECTRO
program~\cite{spectroref}.

\item[\Key{SYMMET}]\verb| |\newline
\verb|READ (LUCMD,*) FCLASS|

Assigns the molecular point group of the molecule (FCLASS). For
instance for water FCLASS would be equal to ``C2v''. Main rotational
axis needs to be set to the z-axis in the .mol file. Additional
generating elements needs to be the x-axis for a C2 rotation, and the
xy plane for a mirror plane. The current implementation only allows
for symmetry use, when differentiating from energies.

\item[\Key{TEST N}]\verb| |\newline

Test if the normal coordinates used for calculated geometrical
derivatives give rise to force fields with appropriate
symmetries. Mainly for debugging purposes.

\end{description}

\subsection{Vibrational averaging of molecular properties: \Sec{PROPAV}}
\label{sec:nmddrv.propan}

This module sets up overall numerical or mixed-numerical/analytical
geometrical property derivative calculations, as well as performs selected
post-analyses of the calculated energy and property derivatives in
terms of zero-point vibrationally averaged properties.

\begin{description}

\item[\Key{ANHA-P}]\verb| |\newline

Requests the calculation of the anharmonic contribution to a
vibrationally averaged molecular property, requiring the use of the
first-order perturbed vibrational wave function (requiring knowledge
about parts of the cubic force field) and the first derivative of the
molecular property. See also the keyword \Key{HARM-P}. Currently only
implemented for spin--spin coupling constants.

\item[\Key{CUBIC }]\verb| |\newline

Calculate the Hessian and third derivatives and write the files
\verb|DALTON.HES| and \verb|DALTON.WLK| for use in future 
calculation of the vibrationally averaged properties using the
\Key{ONLY-P} keyword.

\item[\Key{EFFECTIVE GEOMETRY}]\verb| |\newline

The effective geometry is calculated. This geometry corresponds to the
zero-point vibrationally averaged geometry and is thus often referred
to as the $r_z$ geometry. For more information on the effective
geometry, see Refs.~\cite{poakrprtjcp112,krpoaprtjcp112}.

\item[\Key{HARM-P}]\verb| |\newline

Requests the calculation of the harmonic contribution to a
vibrationally averaged molecular property, requiring the use of the
unperturbed vibrational wave function and the second derivative of the
molecular property. See also the keyword \Key{ANHA-P}. Currently only
implemented for spin--spin coupling constants.

\item[\Key{MODE ANALYSIS}]\verb| |\newline

Request an analysis of the contribution of the different vibrational modes
to the total zero-point vibrational corrections to a molecular property.

\item[\Key{ONLY-P}]\verb| |\newline

Calculate only the equilibrium value and derivatives of the molecular property
required to calculate the (harmonic and anharmonic) vibrational average. 
In order for this to work, \verb|DALTON.HES| and \verb|DALTON.WLK| files containing the Hessian 
and third derivatives of the molecule must be available from a previous calculation.

\item[\Key{P-BASIS}]\verb| |\newline
\verb|READ (LUCMD,*) PRPBTX|

When calculating vibrationally averaged properties use the basis set in the 
.mol file for calculating the force-constants and \verb|PRPBTX| for 
calculating the property and property derivatives.

\item[\Key{SPIN-SPIN COUPLINGS}]\verb| |\newline

Requests the calculation of indirect spin--spin coupling constants.
If combined with the keywords \Key{HARM-P} and \Key{ANHA-P}, the
zero-point vibrational corrections to these coupling constants will be
calculated.

\end{description}

\subsection{Vibrational analysis: \Sec{VIBANA}}
\label{sec:nmddrv.vibana}

This section is identical to the vibrational analysis module described
in Chapter~\ref{sec:abavib}, but appears in the \Sec{*NMDDRV} module
if the Hessian has been calculated numerically using the \Key{NMDDRV}
keyword. We refer to Chapter~\ref{sec:abavib} for a list of available
keywords.
