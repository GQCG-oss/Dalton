\chapter{Finite field calculations}\label{ch:finite}

Despite the large number of properties that in principle can be
calculated with \dalton , it is often of interest to study the
dependence of these properties under an external electric
perturbation. This can be easily achieved by adding static electric
fields\index{finite field}, and thus increase the number of properties
that can be calculated with \dalton .

In the next section we comment briefly on some important aspects of
finite electric field calculations, and the following section
describes the input for finite field calculations.

\section{General considerations}\label{sec:finitegeneral}

The presence of an external electric field can be modeled by adding a
term to our ordinary field-free, non-relativistic Hamiltonian
corresponding to the interaction between the dipole moment operator
and the external electric field:

\begin{equation}
\mathcal{H} = \mathcal{H}^{0} - \mathbf{Ed}_{e}
\end{equation}
where $\mathbf{d}_{e}$ is the electric dipole moment\index{dipole
moment} operator defined as 

\begin{equation}
\mathbf{d}_{e} = \sum_{i}\mathbf{r}_{i}
\end{equation}
and $\mathcal{H}^{0}$ is our ordinary field-free, non-relativistic
Hamiltonian operator. It is noteworthy that we do not include the
nuclear dipole moment\index{nuclear dipole moment} operator, and the
total electronic energy will 
thus depend on the position of the molecule in the Cartesian
coordinate frame.

The electric field dependence of different molecular properties are
obtainable by adding fields in different directions and with different
signs and then extract the information by numerical
differentiation\index{numerical differentiation}. Note that care has
to be taken to choose a field that 
is weak enough for the numeric differentiation to be valid,
yet large enough to give numerically significant changes in the
molecular properties (see for instance Ref.~\cite{mptskrarjcp121}). Note
also that it may be necessary to increase  
the convergence threshold for the solution of the response equations
if molecular properties are being evaluated.

Whereas the finite field approach may be combined with any property
that can be calculated with the \resp\ module, more care need to be
taken if the finite field method is used with the \aba\
module. Properties that involve perturbation-dependent basis
sets\index{perturbation-dependent basis set},
like nuclear shieldings and molecular Hessians, will often introduce
extra reorthonormalization terms due to the finite field operator, and
care has to be taken to ensure that these terms indeed have been
included in \dalton . 

{\em NOTE: In the current release, the only properties
calculated with perturbation dependent basis sets that may be
numerically differentiated using finite field, are the nuclear shieldings
and magnetizabilities using the implementation described in
Ref.~\cite{arthkrabmjpjjcp102}, and molecular gradients.} 

\section{Input description}\label{sec:finiteinput}

\begin{center}
\fbox{
\parbox[h][\height][l]{12cm}{
\small
\noindent
{\bf Reference literature:}
\begin{list}{}{}
\item Shielding and magnetizability polarizabilities: A.Rizzo, T.Helgaker, K.Ruud, A.Barszczewicz, M.Jaszu\'{n}ski and P.J{\o}rgensen. \newblock {\em J.Chem.Phys.}, {\bf 102},\hspace{0.25em}8953, (1995).
\end{list}
}}
\end{center}

The necessary input for a finite-field\index{finite field} calculation
is given in the 
\Sec{*INTEGRALS} and \Sec{*WAVE FUNCTIONS} input modules. A typical input file
for an finite field SCF calculation of the
magnetizability\index{magnetizability} of a molecule will be:

\begin{verbatim}
**DALTON INPUT
.RUN PROPERTIES
**INTEGRALS
.DIPLEN
**WAVE FUNCTIONS
.HF
*HAMILTONIAN
.FIELD
 0.003
 XDIPLEN
**PROPERTIES
.MAGNET
**END OF DALTON INPUT
\end{verbatim}

In the \Sec{*INTEGRALS} input module we request the evaluation of dipole
length\index{dipole length integral} integrals, as these correspond to
the electric dipole operator\index{dipole moment}, 
and will be used in \sir\ for evaluating the interactions between the
electric dipole and the external electric field. This is achieved in
the \Sec{HAMILTONIAN} input module, where the presence of an external
electric field\index{electric field!external} is signaled by the
keyword \Key{FIELD}. On the next line, the
strength of the electric field (in atomic units) is given, and on the following
line we give the direction of the applied electric field
(\verb|XDIPLEN|, \verb|YDIPLEN|, or \verb|ZDIPLEN|). Several fields may
of course be applied at the same time. In comparison with Dalton~1.2, the present version of \dalton\ can also calculate the nuclear
shielding polarizabilities with respect to an external electric field
gradient using London atomic orbitals, both using the traceless
quadrupole operator \Key{THETA} and the second moment of charge
operator \Key{SECMOM}.
