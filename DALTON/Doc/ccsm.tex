
%%%%%%%%%%%%%%%%%%%%%%%%%%%%%%%%%%%%%%%%%%%%%%%%%%%%%%%%%%%%%%%%%%%
\section{Ground state--excited state two-photon transition moments:
\Sec{CCTPA}} \label{sec:cctpa}
%%%%%%%%%%%%%%%%%%%%%%%%%%%%%%%%%%%%%%%%%%%%%%%%%%%%%%%%%%%%%%%%%%%
\index{transition strength!two-photon}
\index{two-photon!transition moment}
\index{transition moment!two-photon}
\index{transition moment!second-order}

This section describes the calculation of 
two-photon absorption strengths 
through coupled cluster response theory.
The two-photon transition strength is defined as
\[
S^{of}_{AB,CD}(\omega) = \frac{1}{2} \{ M^{AB}_{of}(-\omega) M^{CD}_{fo}(\omega)
                         +[M^{CD}_{of}(-\omega) M^{AB}_{fo}(\omega)]^\ast\}
\]
\Sec{CCTPA} drives the calculation of the left ($M^{XY}_{of}(\omega)$)
and right ($M^{XY}_{fo}(\omega)$) transition moments, and of the 
(diagonal) transition strengths $S^{of}_{AB,AB}(\omega)$.
The methodology is implemented for the CCS, CC2, CCSD and CC3 models.

\begin{center}
\fbox{
\parbox[h][\height][l]{12cm}{
\small
\noindent
{\bf Reference literature:}
\begin{list}{}{}
\item C.~H\"{a}ttig, O.~Christiansen, and P.~J{\o}rgensen \newblock {\em J.~Chem.~Phys.}, {\bf 108},\hspace{0.25em}8331, (1998).
\item C.~H\"{a}ttig, O.~Christiansen, and P.~J{\o}rgensen \newblock {\em J.~Chem.~Phys.}, {\bf 108},\hspace{0.25em}8355, (1998).
\end{list}
}}
\end{center}

\begin{description}
%
\item[\Key{STATES}] \verb| | \newline
\verb|READ (LUCMD,'(A70)') LABHELP|\newline
\verb|DO WHILE (LABHELP(1:1).NE.'.' .AND. LABHELP(1:1).NE.'*')|\newline
\verb|  READ (LUCMD,'(3A)') IXSYM, IXSTATE, SMFREQ|\newline
\verb|END DO| \newline
Select one or more excited states (among those specified
in \Sec{CCEXCI}), and photon energies.
The symmetry class (\verb+IXSYM+) and the number a state $f$
within that symmetry (\verb+IXSTATE+) have to be given
together with a photon energy $\omega$ \verb+SMFREQ+ (in atomic units).
For each state and photon energy the moments 
$M^{AB}_{of}(-\omega)$ and $M^{AB}_{fo}(+\omega)$ and the
strengths $S^{of}_{AB,AB}(\omega)$ are computed for all operator pairs.
Instead of \Key{STATES} the equivalent keywords \Key{TRANSITION},
\Key{SELEXC}, or \Key{SELSTA} may be used.
%
\item[\Key{HALFFR}] 
Set the photon energies (frequencies) for the two-photon transition moments
equal to  half the excitation energy of the final state $f$
as calculated with the present coupled cluster model.
If this option is switched on, the photon energies given with the
specification of the states (\Key{STATES}) will be ignored.
%
\item[\Key{OPERAT}] \verb| |\newline
\verb|READ (LUCMD,'(4A)') LABELA, LABELB|\newline
\verb|DO WHILE (LABELA(1:1).NE.'.' .AND. LABELA(1:1).NE.'*')|\newline
\verb|  READ (LUCMD,'(4A)') LABELA, LABELB|\newline
\verb|END DO|\newline
Read pairs of operator labels.
For each pair the left and right two-photon transition moments
and strengths $S^{of}_{AB,AB}(\omega)$ will be evaluated 
for all states and frequencies.
Operator pairs which do not correspond to symmetry allowed
combinations will be ignored during the calculation.
%
\item[\Key{DIPLEN}] 
Compute all symmetry-allowed elements of the dipole--dipole 
transition moment tensors in the length gauge
and the corresponding transition strengths.
In addition the three averages $\delta_F$, $\delta_G$, and $\delta_H$
are evaluated (see below).
%
\item[\Key{DIPVEL}]
Compute all symmetry-allowed elements of the dipole--dipole 
transition moment tensors in the velocity gauge
and the corresponding transition strengths.
%
\item[\Key{ANGMOM}]
Compute all symmetry-allowed elements of 
the transition moment tensors
with the operators $A$ and $B$ equal to 
a component of the angular momentum operator $\vec{l}$,
which is proportional to the magnetic dipole operator.
%
\item[\Key{PRINT}] \verb| |\newline
\verb|READ (LUCMD,*) IPRSM|\newline
Read print level. Default is 0.
%
\item[\Key{USE X2}] use the second-order vectors $\eta^{AB}$ as intermediates.
   This may save some CPU time with the models CCS, CC2, and CCSD.
   It will be ignored in CC3 calculations, where it cannot be used. 
%
\item[\Key{USE O2}] use the second-order vectors $\xi^{AB}$ as intermediates.
   This may save some CPU time with the models CCS, CC2, and CCSD.
   It will be ignored in CC3 calculations, where it cannot be used. 
\end{description}
If no transitions have been specified using \Key{STATES} or one
of the equivalent keywords, the default is to include all 
states specified in \Sec{CCEXCI} with the photon energies
set to half the excitation energies 
({\it i.e.\/} the \Key{HALFFR} option is implied).

If the full dipole--dipole tensors (in length gauge) have been specified for the
moments ({\it e.g.\/} using \Key{DIPLEN}), the following three isotropic averages will be
evaluated:
\begin{eqnarray*}
  \delta_F & = & \frac{1}{30} 
   \sum_{\alpha\beta=x,y,z} S^{0f}_{\alpha\alpha,\beta\beta}(\omega)
\\
  \delta_G & = & \frac{1}{30} 
   \sum_{\alpha\beta=x,y,z} S^{0f}_{\alpha\beta,\alpha\beta}(\omega)
\\
  \delta_H & = & \frac{1}{30} 
   \sum_{\alpha\beta=x,y,z} S^{0f}_{\alpha\beta,\beta\alpha}(\omega)
\end{eqnarray*}
The option \Key{DIPLEN} will be implied by default if no other
operator pairs have been specified using one of the keywords
\Key{OPERAT}, \Key{DIPVEL}, or \Key{ANGMOM}.

