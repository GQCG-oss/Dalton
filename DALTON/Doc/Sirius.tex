
% checked with "ispell" sep.2003 /hjaaj
\chapter{\label{chap:sirius-inpref}Molecular wave functions, {\sir}}

\section{\label{sec:sirius-ref-notes} General notes for the {\sir} input reference
manual}

{\sir} is the part of the code that computes the wave function/density.

The following sections contain a list of all generally relevant keywords to
{\sir}, only currently inactive keywords and some special debug
options are omitted.

\begin{enumerate}
\item {The input for the wave function section must begin with

\begin{inputex} \begin{verbatim}
**WAVE FUNCTIONS
\end{verbatim} \end{inputex}
   with no leading blanks.  The preceding lines in the input file may
   contain arbitrary information.
}
\item{ Input is directed by keywords written in upper case.
   Only the first 7 characters including the prompt are significant.
   The keywords are divided in a number of main input groups. Each main
   input group is initiated by a {\starkey}. For example

\begin{inputex} \begin{verbatim}
*ORBITAL INPUT
\end{verbatim} \end{inputex}
   marks the beginning of the input group for orbital input.
}
\item { The keywords belonging to one of the main input groups begin with
   the prompt {\dotkey}.
}
\item { Keywords that are necessary to specify are marked by "Required".
   For other keywords the default values can be used in ordinary runs.
}
\item {Any keyword line beginning with a \quotekw{!} or
   \quotekw{\#} will be treated as a
   comment line.  An illegal keyword will cause a dump of all keywords
   for the current input section.
}
\item{A dump of keywords can be obtained in any input section by
specifying the keyword \quotekw{\Key{OPTIONS}}.  For example, the input

\begin{inputex} \begin{verbatim}
**WAVE FUNCTIONS
.OPTIONS
**END OF DALTON INPUT
\end{verbatim} \end{inputex}
   will cause a dump of the labels for the main input groups in {\sir},
   while

\begin{inputex} \begin{verbatim}
**WAVE FUNCTIONS
*ORBITAL INPUT
.OPTIONS
**END OF DALTON INPUT
\end{verbatim} \end{inputex}
   will cause a dump of the labels for the \quotekw{*ORBITAL INPUT} input group
   in {\sir}.
}
\item{ The {\sir} input is finished with a line beginning with two stars,
   {\it e.g.\/}

\begin{inputex} \begin{verbatim}
**END OF DALTON INPUT
\end{verbatim} \end{inputex}
}
\end{enumerate}


\pagebreak[3]
\section{\label{sec:ref-newinp}
   Main input groups in the **WAVE FUNCTIONS input module}

\noindent
The main input groups (those with the {\starkey} prompt) are listed here and
the full descriptions are given in the designated sections.

\noindent
The first input group is always required in order to specify the type of
calculation, and follows immediately after the \Sec{*WAVE FUNCTIONS}
keyword.

%Section~\ref{ref-geninp} \Sec{GENERAL INPUT}

\noindent The remaining input groups may be specified in any
order. In this chapter they are grouped alphabetically, although
the short presentation below gather them according to purpose.

%\ifsolvent
The following two input groups are used to modify the
molecular environment by adding field-dependent
terms in the Hamiltonian and by invoking
the self-consistent reaction field model for solvent
effects, respectively:

Section~\ref{ref-haminp} \Sec{HAMILTONIAN}

Section~\ref{ref-solinp} \Sec{SOLVENT}
%\else
%The following input group describes additional field-dependent
%terms in the Hamiltonian :
%
%Section~\ref{ref-haminp} \Sec{HAMILTONIAN}
%\fi

\noindent
The next input group specifies the configurations included
in the MCSCF and CI wave functions:

Section~\ref{ref-wavinp} \Sec{CONFIGURATION INPUT}

\noindent
The two next groups are used to specify initial orbitals and initial
guess for the CI vector:

Section~\ref{ref-orbinp} \Sec{ORBITAL INPUT}

Section~\ref{ref-civinp} \Sec{CI VECTOR}

\noindent
The two following input groups control the second-order MCSCF
optimization:

Section~\ref{ref-optinp} \Sec{OPTIMIZATION}

Section~\ref{ref-stpinp} \Sec{STEP CONTROL}

\noindent
The next five groups have special input only relevant for the
respective calculation types:

Section~\ref{ref-rhfinp} \Sec{SCF INPUT}

Section~\ref{ref-dftinp} \Sec{DFT INPUT}

Section~\ref{ref-mp2inp} \Sec{MP2 INPUT}

Section~\ref{ref-nevpt2inp} \Sec{NEVPT2 INPUT}

Section~\ref{ref-cicinp} \Sec{CI INPUT}

\noindent
The next section is used to select some types of analysis of the final
Hartree--Fock, DFT, MCSCF, or CI wave function:

Section~\ref{ref-popinp} \Sec{POPULATION ANALYSIS}

\noindent
The next section is used to change the default integral transformation
and specify any final integral transformation after convergence (a
program following {\sir} may need a higher transformation level):

Section~\ref{ref-trainp} \Sec{TRANSFORMATION}

\noindent
The next two input groups control the amount of printed output and
collect options not fitting in any of the other groups:

Section~\ref{ref-priinp} \Sec{PRINT LEVELS}

Section~\ref{ref-auxinp} \Sec{AUXILIARY INPUT}

\noindent
Finally we note that there is an input module controlling the
calculation of coupled cluster wave functions. This is treated in a
separate chapter:

Chapter~\ref{ch:CC} \Sec{CC INPUT}

\bigskip
\noindent
The wave function input is finished when a line is encountered beginning
with two stars, for example

\begin{inputex} \begin{verbatim}
   **END OF DALTON INPUT
\end{verbatim} \end{inputex}
or

\begin{inputex} \begin{verbatim}
   **MOLORB
   ... formatted molecular orbitals coefficients
   **END OF DALTON INPUT
\end{verbatim} \end{inputex}

\noindent
The \Sec{*MOLORB} keyword or the \Sec{*NATORB} keyword
must be somewhere on the input file and be
followed by molecular orbital coefficients if the option for formatted
input of molecular orbitals has been specified.  Apart from this
requirement, arbitrary information can be written to the following lines
of the input file.



\pagebreak[3]
\subsection{\label{ref-geninp}\Sec{*WAVE FUNCTIONS}}

{\bf Purpose:}

Specification of which wave function calculation is to be performed.

{\bf Primary keywords, listed in the order the corresponding modules
 will be executed by the program (if the keyword is set): }

\begin{description}

\item[\Key{HF}]
  Restricted closed-shell or one open-shell 
  Hartree--Fock\index{HF}\index{SCF}\index{Hartree--Fock}\index{open shell!HF} calculation.
  The occupied orbitals, optimization control etc. are specified in the
  \quotekw{*SCF INPUT} submodule.
  Note: you cannot specify both \quotekw{.HF} and \quotekw{.DFT} keywords.

\item[\Key{DFT}]
  \verb"READ (LUINP,'(A80)) LINE" \\ Restricted closed-shell, one
  open-shell or high-spin spin-restricted Kohn-Sham density functional
  theory\index{DFT}\index{Kohn--Sham}\index{open shell!DFT}
  calculation.  On the following line you must specify which
  functional to use.  The occupied orbitals, optimization control
  etc. are specified in the \quotekw{*SCF INPUT} submodule shared with
  the \quotekw{.HF} option.  The DFT specific input options are
  collected in the \quotekw{*DFT INPUT} input submodule.  Note: you
  cannot specify both \quotekw{.HF} and \quotekw{.DFT} keywords.

\item[\Key{MP2}]
 \index{M{\o}ller-Plesset!second-order}\index{MP2}
  M{\o}ller-Plesset second-order perturbation theory calculation.
  Requires \quotekw{.HF} or previously calculated canonical Hartree--Fock orbitals.

\item[(\quotekw{.FC MVO} in \quotekw{*SCF INPUT}]
  Calculation of modified virtual SCF orbitals based on the
  potential determined by the keyword (see comments below).
  The occupied SCF orbitals are not modified.
  Note that this keyword is not located in this module but in the
  \quotekw{*SCF INPUT} submodule. It is mentioned here to make clear
  at what point this transformation will be performed, if requested.

\item[\Key{CI}]
  \index{CI}\index{Configuration Interaction}
  Configuration interaction calculation.

\item[\Key{MCSCF}]
  Multiconfiguration self-consistent field (MCSCF) calculation.\index{MCSCF}

\item[\Key{NEVPT2}]
  Multireference second-order perturbation theory calculation.\index{NEVPT2}

\item[\Key{CC}]
  \index{Coupled Cluster}\index{CC}
  Coupled cluster calculation. Automatically activates Hartree--Fock (\quotekw{.HF}).
  After the Hartree--Fock calculation,
  the \cc\ module is called to do a coupled cluster (response) calculation.
  For further input options for the \cc\ module see Section~\ref{sec:ccgeneral}. 

\item[\Key{CC ONLY}] Skip the calculation of a Hartree--Fock wave
  function, and start directly in the coupled cluster part. Convenient
  for restarts in the coupled cluster module.

\end{description}

{\bf Secondary keywords (in alphabetical order): }

\begin{description}

\item[\Key{FLAGS}]
  \verb"READ (LUINP,NMLSIR)" \\
  Read namelist \verb"$NMLSIR ... $END" \\
  Only for debugging. Set internal control flags directly.
  Usage is not documented.

\item[\Key{INTERFACE}]
  Write the "\verb|SIRIFC|" interface file\index{interface file} for post-processing programs.

%hjaaj oct 2003: obsolete ...
%\item[\Key{NSYM}]
%%  Required, no defaults. \\
%  Default: specified by integral program \\
%  \verb"READ (LUINP,*) NSYM" \\
%  Number of spatial Abelian symmetries (1, 2, 4, or 8), corresponding
%  to $D_{2h}$ or one of its subgroups.

\item[\Key{PRINT}]
  \verb"READ (LUINP,*) IPRSIR" \\
  General {\sir} print level and default for all other print parameters in this module.

\item[\Key{RESTART}]
  restart {\sir}\index{restart!wave function},
  the {\sir} restart file (\verb|SIRIUS.RST|) must be available

%hjaaj Oct 2003: obsolete, we want the specific .HF or .DFT
%\item[\Key{SCF}]
%    Alias for the \quotekw{\Key{HF}} keyword.

\item[\Key{STOP}]
  \kw{READ (LUINP,'(A20)') REWORD} \\
  Terminate {\sir} according to the instruction given on the following line.
  Three stop points are defined:
\begin{enumerate}

\item \hspace{2em} \quotekw{ AFTER INPUT}

\item \hspace{2em} \quotekw{ AFTER MO-ORTHONORMALIZATION}

\item \hspace{2em} \quotekw{ AFTER GRADIENT} (only for MCSCF and 2nd-order HF or DFT)
\end{enumerate}

\item[\Key{TITLE}] 
  \verb"READ (LUINP,'(A)') TITLE(NTIT)" 
  Any number of title lines (until next line beginning with
  \quotekw{.} or \quotekw{*} prompt).
  Up to 6 title lines will be saved and used in the output, additional
  lines will be discarded.

\item[\Key{WESTA}]
  Write the "\verb|SIRIFC|" interface file for the WESTA post-processing program.

%hjaaj Oct 2003: obsolete
%\item[\Key{BASIS SET}]
%   Default: specified by integral program \\
%   \verb"READ (LUINP,*) (NBAS(I), I = 1,NSYM)" \\
%   Number of basis functions per symmetry.
\end{description}

%hj%\noindent{\bf Comments:}

%\ifabacus ABACUS: 
%If the full molecular Hessian is calculated in
%ABACUS and the number of symmetries (\verb|NSYM|) is greater than
%one, then the MCSCF wave function will be automatically calculated
%in determinants\index{determinants} and, if singlet,
%\quotekw{.PLUS COMBINATIONS}).  This is so because the CSFs can
%only have one spatial symmetry, and it is generally necessary to
%solve linear response equations of several symmetries to get the
%full molecular Hessian. 
%\fi

%hjaaj Oct 2003: obsolete
%BASIS SET is provided such that the number of basis functions in each
%symmetry may be specified if {\sir} is modified to interface to an
%integral program which doesn't write this information to the integral
%file.

\pagebreak[3]
\subsection{\label{ref-auxinp}\Sec{AUXILIARY INPUT}}

{\bf Purpose:}

Input which does not naturally fit into any of the other
categories.

\begin{description}
\item[\Key{NOSUPMAT}]
  Do not use P-SUPERMATRIX integrals, but calculate Fock matrices
  from AO integrals (slower, but requires less disk space). The
  default is to use the supermatrix file if it exists. See option
  \Key{NOSUP} in Chapter~\ref{sec:herminp}.

%\item[\Key{ONESUP}]
%  Use same unit for P-SUPERMATRIX\index{supermatrix} and ONE-ELECTRON
%  integrals \index{one-electron integral}
%  (LUSUPM=LUONEL, default is different units).
\end{description}

\pagebreak[3]
\subsection{\label{ref-cicinp}\Sec{CI INPUT}}

{\bf Purpose:}

Options for a CI calculation.

\begin{description}
\item[\Key{CIDENSITY}]
  Calculate CI one-electron density matrix and natural
  orbital\index{natural orbital}
  occupations after convergence.

\item[\Key{CINO}]
  Generate CI\index{CI}\index{Configuration Interaction} natural
  orbitals\index{natural orbital} for CI root
  number \kw{ISTACI},
  clear the \verb|SIRIUS.RST| file and write the orbitals with label \quotekw{NEWORB  }.
  The \quotekw{\Key{STATE}} option must be specified.

\item[\Key{CIROOTS}]
  Default: NROOCI = 1\\
  \kw{READ (LUINP,*) NROOCI} \\
  Converge the lowest \kw{NROOCI} CI roots\index{root!CI} to threshold.

\item[\Key{DISKH2}]
  Active two-electron MO integrals on disk (see comments below).

\item[\Key{MAX ITERATIONS}]
  \kw{READ (LUINP,*) MXCIMA} \\
  Max iterations in iterative diagonalization of CI matrix (default = 20).

\item[\Key{STATE}]
  Default: not specified\\
  \kw{READ (LUINP,*) ISTACI} \\
  Alternative to \quotekw{\Key{CIROOTS}}.  Converge root number \kw{ISTACI}
  to threshold, converge all lower roots only to THQMIN
  (from the \quotekw{\Sec{OPTIMIZATION}} input group, see
  p.~\pageref{ref-optinp}).

\item[\Key{THRESH}]
  Default = 1.D-05\\
  \kw{READ (LUINP,*) THRCI} \\
  Threshold for CI energy gradient.  The CI energy will be converged to
  approximately the square of this number.

\item[\Key{THRPWF}]
  Default = 0.05D0 for electronic ground states, and 0.10D0 for
  excited states\\
  \kw{READ (LUINP,*) THRPWF} \\
  Only CI coefficients greater than threshold are printed
  (PWF: print wave function).

\item[\Key{WEIGHTED RESIDUALS}]
  Use energy weighted residuals\index{residual} (see comments below).

\item[\Key{ZEROELEMENTS}]
  Zero small elements in CI trial vectors (see comments below).
\end{description}


\noindent{\bf Comments:}

DISKH2: By default the CI module will attempt to place the two-electron
integrals with four active indices in memory for more efficient
calculation of CI sigma vectors, if memory is insufficient for this
the
integrals will automatically be placed on disk.  The DISKH2 keyword
forces the integrals always to be on disk.

WEIGHTED RESIDUALS:  Normally the CI states will be converged to a
residual less than the specified threshold, and this will give
approximately the squared number of decimal places in the energy.
Depending on the value of the energy, the eigenvectors will be converged
to different accuracy. If the eigenvectors are wanted with, for instance at
least 6 decimal places for property calculations, specify a threshold of
1.0D-6 and weighted residuals.

ZEROELEMENTS: an experimental option that might save time (if the CI
module can use sparseness) by zeroing all elements less than 1.0D-3
times the largest element in the CI trial vector before
orthonormalization against previous trial vectors.
See also \quotekw{.SYM CHECK} under \quotekw{*OPTIMIZATION}
(p.~\pageref{ref-optinp}).


\pagebreak[3]
\subsection{\label{ref-civinp}\Sec{CI VECTOR}}

{\bf Purpose:}

To obtain initial guess for CI vector(s).

\begin{description}
\item[\Key{PLUS COMBINATIONS}]
  Use with \quotekw{\Key{STARTHDIAGONAL}} to choose plus combination
  of degenerate   diagonal elements ({\bf STRONGLY RECOMMENDED} for
  calculation of singlet states  with \quotekw{\Key{DETERMINANTS}}).

\item[\Key{SELECT}]
  \kw{READ (LUINP,*) ICONF} \\
  Select CSF (or determinant if \quotekw{\Key{DETERMINANTS}}) no.
  ICONF as start configuration.\index{configuration!start}

\item[\Key{STARTHDIAGONAL}]
  Select configurations corresponding to the lowest diagonal elements in
  the configuration part of the Hessian (this is the default option).

\item[\Key{STARTOLDCI}]
  Start from old CI-vector stored saved on the \verb|SIRIUS.RST| file.

%\ifabacus
%\item[.ABACUS]
%  Geometry walk, use CI vector and "GEOSAVE" information saved by
%  ABACUS at previous geometry.  The "GEOSAVE" information is used
%  to decide as early as possible in the wave function optimization
%  if the step should be rejected, thus saving CPU time if the step
%  is rejected.
%\fi

\end{description}

\pagebreak[3]
\subsection{\label{ref-wavinp}\Sec{CONFIGURATION INPUT}}

{\bf Purpose:}

To specify the configuration part in MCSCF and CI calculations.

\begin{description}
\item[\Key{CAS SPACE}]
  \kw{READ (LUINP,*) (NASH(I),I=1,NSYM)} \\
  CAS calculation: Active orbitals\index{active orbital} in each symmetry.

\item[\Key{ELECTRONS}]
  Required.\\
  \kw{READ (LUINP,*) NACTEL} \\
  Number of active electrons\index{active electrons} (the number of
  electrons to be distributed in the active orbitals).
  The total number of electrons is this number
  plus two times the total number of inactive orbitals.

\item[\Key{INACTIVE ORBITALS}]
  Required.\\
  \kw{READ (LUINP,*) (NISH(I),I=1,NSYM)} \\
  Number of inactive orbitals\index{inactive orbital} each symmetry.

\item[\Key{RAS1 ELECTRONS}]
   \kw{READ (LUINP,*) NEL1MN,NEL1MX} \\
   Minimum and maximum number of RAS1 electrons; this can alternatively
   be specified with \quotekw{\Key{RAS1 HOLES}}

\item[\Key{RAS1 HOLES}]
  \kw{READ (LUINP,*) NHL1MN,NHL1MX} \\
  Minimum and maximum number of holes\index{electron hole} in RAS1; alternative
  to \quotekw{\Key{RAS1 ELECTRONS}}

\item[\Key{RAS1 SPACE}]
   \kw{READ (LUINP,*) (NAS1(I),I=1,NSYM)} \\
   RAS calculation: RAS1 orbital space\index{RAS1 orbital space}

\item[\Key{RAS2 SPACE}]
   \kw{READ (LUINP,*) (NAS2(I),I=1,NSYM)} \\
   RAS calculation: RAS2 orbital space\index{RAS2 orbital space}

\item[\Key{RAS3 ELECTRONS}]
   \kw{READ (LUINP,*) NEL3MN, NEL3MX} \\
   Minimum and maximum number of RAS3 electrons

\item[\Key{RAS3 SPACE}]
   \kw{READ (LUINP,*) (NAS3(I),I=1,NSYM)} \\
   RAS calculation: RAS3 orbital space\index{RAS3 orbital space}

\item[\Key{SPIN MULTIPLICITY}]
  Required for MCSCF and CI wave
  functions\index{MCSCF}\index{CI}\index{Configuration Interaction}.\\
  \kw{READ (LUINP,*) ISPIN}\\
  For CSF basis: state spin multiplicity\index{spin multiplicity} = $2S + 1$,
  where $S$ is the spin quantum number. \\
  For determinant basis this option determines the minimum spin
  multiplicity.  The $M_S$ value is determined as (ISPIN-1)/2.

\item[\Key{SYMMETRY}]
  Required for MCSCF and CI wave functions.\\
  \kw{READ (LUINP,*) LSYM} \\
  Spatial symmetry\index{symmetry} of CI and/or MCSCF wave function

\end{description}

\noindent{\bf Comments:}

\noindent SYMMETRY   Specifies total spatial symmetry of the wave
function in $D_{2h}$ symmetry or one of its subgroups: $C_{2v}$, $C_2h$,
$D_2$, $C_s$, $C_i$, $C_2$, $C_1$. The  symmetry number of wave
function follows MOLECULE output ordering of symmetries ($D_{2h}$
subgroup irreps).

\noindent
CAS and RAS\index{RASSCF}\index{CASSCF}\index{MCSCF} are exclusive and
both cannot be specified in the same
MCSCF or CI\index{MCSCF}\index{CI}\index{Configuration Interaction}
calculation. One of them {\em must} be specified.

\pagebreak[3]
\subsection{\label{ref-dftinp}\Sec{DFT INPUT}}

{\bf Purpose:}

To specify the parameters of the DFT integration and the optional use of empirical corrections.

\begin{description}

\item[\Key{DFTAC}]
  \kw{READ (LUINP,*) RTYPE} \\
  \kw{READ (LUINP,*) CTYPE} \\
  \kw{READ (LUINP,*) DFTIPTA, DFTIPTB, DFTBR1, DFTBR2} \\
  Switches on the asymptotic correction of the exchange correlation potential. This correction is a pointwise manipulation of the 
  exchange--correlation potential. This implies activation of the .DFTVXC keyword in the SCF stage. RTYPE defines the
  potential to be used to replace the asymptotic GGA potential, possible options are MULTPOLE (a simple multipole model)
  and LB94 (the potential from the LB94 model potential~\cite{dft:lb94}). CTYPE defines how the potential of the parent
  functional is connected to the asymptotic model, possible options are LINEAR (as used in the Tozer-Handy correction~\cite{dft:th}), 
  TANH (a modified connection by Tozer~\cite{dft:tanh}, which removes discontinuities associated with linear interpolation) and 
  GRAC (the gradient-regulated asymptotic correction of Gr\"uning \emph{et. al.}~\cite{dft:grac}). 
  
  Four numerical input parameters are then input
  the first two are the $\alpha$ and $\beta$ ionization potentials (either calculated or experimental). If GRAC is chosen for the 
  connection type then the last two value specify the parameters $\alpha$ and $\beta$ (see Ref.~\cite{dft:grac} for details). 
  Recommended values are 0.5 and 40.0. Otherwise the last two parameters specify multiples of the 
  Bragg Radii and are used to define the core, interpolation and asymptotic regions. For grid points within DFTBR1 Bragg Radii of 
  each atom the potential is unmodified, for points outside DFTBR2 Bragg Radii the potential is replaced with its asymptotic model.
  In between the interpolation model is used. Recommended values in this case are 3.5 and 4.7. Care should be taken when 
  choosing alternative values for the final two parameters in each scheme, inappropriate values can make SCF convergence difficult.


\item[\Key{DFTD2}]
  switches on Grimmes DFT-D2 empirical dispersion correction~\cite{dft:dftd2}. The code will attempt to assign the correct functional dependent 
  parameters based on the chosen DFT functional. Analytic gradient contributions are available.
 
\item[\Key{D2PAR}]
  \kw{READ (LUINP,*) D2\_s6\_inp, D2\_alp\_inp, D2\_rs6\_inp} \\
  using this keyword user input values of the $s_6$, $\alpha$ and $s_{r,6}$ DFT-D2 parameters may be specified. If supplied these values override
  any values defined within the code.

\item[\Key{DFTD3}]
  switches on Grimmes DFT-D3 empirical dispersion correction~\cite{dft:dftd3}. The code will attempt to assign the correct functional dependent 
  parameters based on the chosen DFT functional. Analytic gradient contributions are available.
  
\item[\Key{DFD3BJ}]
  switches on Grimmes DFT-D3 empirical dispersion correction with Becke-Johnson damping~\cite{dft:dftd3bj}. 
  The code will attempt to assign the correct functional dependent parameters based on the chosen DFT functional. 
  Analytic gradient contributions are available. This is the presently recommended version.
  
\item[\Key{3BODY}]
  keyword for adding 3-body terms to the DFT-D3 dispersion energy. Note that gradients are not implemented for these corrections 

\item[\Key{D3PAR}]
  \kw{READ (LUINP,*) D3\_s6\_inp, D3\_alp\_inp, D3\_rs6\_inp, D3\_rs18\_inp, D3\_s18\_inp} \\
  keyword for specifying the $s_6$, $\alpha$, $s_{r,6}$, $s_{r,8}$ and $s_8$ parameters of the DFT-D3 methods. Note, take care to 
  match the parameter values to the correct version of the DFT-D3 correction.

\item[\Key{DFTELS}]
  \kw{READ (LUINP,*) DFTELS} \\
  safety threshold -- stop if the charge integration gives too large
  error.

\item[\Key{DFTTHR}]
  \kw{READ (LUINP,*) DFTHRO, DFTHRI} \\
  Thresholds determining accuracy of the numerical integration. The
  first number determines the density threshold (contributions to a
  property from places where the density is below the threshold will
  be skipped) and the second one -- orbital threshold (orbitals are
  assumed to be exactly 0 if they are below the threshold). The
  default value for DFTHR0 is $1.0D-9$ and for DFTRHI is $1.0D-13$.

\item[\Key{DFTVXC}]
  keyword to specify explicit construction of the exchange--correlation 
  potential for GGA forms. This is automatically invoked when .DFTAC is selected 
  and not recommended for use otherwise.

\item[\Key{RADINT}]
  \kw{READ (LUINP,*) RADINT} \\

  Determines the quality of the radial part of the grid and
  corresponds to the upper limit of the error in case of an
  integration on an atom. Default value is $1.0D-13$.

\item[\Key{ANGINT}]
  \kw{READ (LUINP,*) ANGINT} \\
  Determines the quality of the angular Lebedev grid -- the angular
  integration of spherical harmonics will be exact up to the specified
  order. Default value is 35.
\item[\Key{GRID TYPE}]
  \kw{READ (LUINP,*) LINE} \\
  Allows specification of different partitioning methods and radial
  schemes. \verb|BECKE| is Becke partitioning scheme with correction
  for atomic sizes using Bragg sizes, \verb|BECKEORIG| is the same
  Becke partitioning scheme but without correction. \verb|SSF| is a
  partitioning scheme for large molecules designed to reduce the grid
  generation time. \verb|LMG| select LMG radial scheme adjusted to
  currently used basis set. Gauss-Chebychev radial scheme of second
  order is provided for reference and can be selected by keyword
  \verb|GC2|.

% The default is \verb|BECKE LMG| which is optimal for an
%  overwhelming number of cases.
\item[\Key{COARSE}]
  Shortcut keyword for radial integration accuracy $10^{-11}$ and
  angular expansion order equal to $35$.
\item[\Key{NORMAL}]
  Shortcut keyword for radial integration accuracy $10^{-13}$ and
  angular expansion order equal to $35$.
\item[\Key{FINE}]
  Shortcut keyword for radial integration accuracy $10^{-13}$ and
  angular expansion order equal to $42$.
\item[\Key{ULTRAF}]
  Shortcut keyword for radial integration accuracy $10^{-15}$ and
  angular expansion order equal to $65$.

\end{description}

\subsection{\label{ref-dft}DFT functionals}

In general, functionals in \dalton\ can be divided into two groups: 
generic exchange and correlation functionals 
and combined functionals. Combined functionals are a linear combination of
generic ones. There are a large number of combined functionals defined below, 
however the user can also create their own combined functionals with 
the \verb|Combine| keyword. 
A number of standalone functionals are also included within \dalton. 
In addition a number of double-hybrid functionals (energies only) are available,
which include a post-SCF second-order perturbation theory contribution. 

It should be noted that the input is not case sensitive, although the notation
employed in this manual makes use of case to emphasize exchange or correlation 
functional properties and reflect the original literature sources.


\subsubsection{Exchange Functionals}
\providecommand\exfn[1]{#1}
\begin{description}

\item[Slater] Dirac-Slater exchange functional
\cite{dft:hohenberg,dft:kohn,dft:slater}.\index{Slater}

\item[Becke] 1988 Becke exchange GGA correction \cite{dft:becke88}. 
  Note that the full Becke88 exchange functional is given as
  \exfn{Slater} + \exfn{Becke}.\index{Becke}

\item[mBecke] 1998 modified \exfn{Becke} exchange correction presented in reference
  \cite{dft:edf1} for use in the EDF1 functional. The $\beta$ value
  of 0.0042 in \exfn{Becke} is changed to 0.0035.\index{mBecke}\index{EDF1}

\item[B86] Becke 1986 exchange functional, a divergence free, semi-empirical 
    gradient-corrected exchange functional~\cite{dft:b86,dft:b86r}.\index{B86} This 
    functional corresponds to the B86R functional of the Molpro program.

\item[B86mx] B86 exchange functional modified with a gradient correction 
  for large density gradients~\cite{dft:b86mgc}.\index{B86mx}

\item[DBx] Double-Becke exchange functional defined in 1998 by 
  Gill et al.\cite{dft:edf1,dft:edf2} for use in the EDF1 functional. 
  The full DBx functional is defined as

  1.030952*\exfn{Slater} - 8.44793*\exfn{Becke} + 10.4017*\exfn{mBecke}
  \index{EDF1}\index{Double-Becke}\index{Becke}

\item[DK87x] DePristo and Kress' 1987 rational function GGA exchange functional 
  (equation 7) from Ref. \cite{dft:dk87}.\index{DK87} The full exchange
  functional is defined as \exfn{Slater} + \exfn{DK87x}.

%\item[FT97ax] Filatov and Thiel 1997 (FT97) exchange functional GGA correction, 
%  variant A.~\cite{dft:ft97}.\index{FT97}
%  The complete exchange functional is given by \exfn{Slater} + \exfn{FT97ax}.
%
%\item[FT97bx] Filatov and Thiel 1997 (FT97) exchange functional GGA correction,
%  variant B.~\cite{dft:ft97}.\index{FT97} In this variant, the $\beta$ parameter 
%  is a switching function dependent on the density gradient, $\nabla n_{\sigma}$ 
%  and only significantly varies from variant A calculated molecular properties 
%  if core electron effects are significant. This is the default exchange functional 
%  in the combined FT97 exchange-correlation functional.
%  The complete exchange functional is given by \exfn{Slater} + \exfn{FT97bx}.

\item[G96x] Gill's 1996 GGA correction exchange functional~\cite{dft:g96}.\index{G96}
  The complete exchange functional is given by \exfn{Slater} + \exfn{G96x}.

\item[LG93x] 1993 GGA exchange functional~\cite{dft:lg93,dft:g961lyp}.\index{LG93}
  The full LG93 exchange functional is given by \exfn{Slater} + \exfn{LG93x} 

\item[LRC95x] 1995 GGA exchange functional with correct asymptotic behavior~\cite{dft:lrc95}.\index{LRC95}
  The LRC95x exchange functional includes the Slater exchange (Eq 6 from original reference).

\item[KTx] Keal and Tozer's 2003 GGA exchange functional. Note that the gradient correction 
  pre-factor constant, $\gamma$, is not included in the KT exchange 
  definition, but rather in the KT1, KT2 and KT3 definitions. The full KT exchange is given by 
  \cite{dft:kt12}\index{KT},

  \exfn{Slater} + $\gamma$\exfn{KTx} ($\gamma$ is -0.006 for KT1,KT2 and -0.004 for KT3). 

\item[OPTX] Handy's 2001 exchange functional correction \cite{dft:optx}.\index{OPTX}
  The full OPTX exchange functional is given by 
  1.05151*\exfn{Slater} - 1.43169*\exfn{OPTX}.

\item[PBEx] Perdew, Burke and Ernzerhof 1996 exchange functional~\cite{dft:pbe}.\index{PBEx}

\item[revPBEx] Zhang and Wang's 1998 revised PBEx exchange functional, with $\kappa$ of 1.245
  \cite{dft:revpbe}.\index{revPBE}\index{PBE}

\item[RPBEx] Hammer, Hansen and N{\o}rskov's 1999 revised PBEx exchange functional 
  \cite{dft:revpbe}.\index{RPBE}\index{PBE}

\item[mPBEx] Adamo and Barone's 2002 modified PBEx exchange functional~\cite{dft:mpbe}.
  \index{mPBE}\index{PBE}

\item[PW86x] Perdew and Wang 1986 exchange functional (the PWGGA-I functional)
  ~\cite{dft:pw86x}.\index{PW86x}

\item[PW91x] Perdew and Wang 1991 exchange functional (the pwGGA-II functional)
  and includes Slater exchange \cite{dft:pw91}.\index{PW86x} This functional is 
  also given in a separate parameterization in Refs.~\cite{dft:g96,dft:mpw},
  which is labeled PW91x2, and is defined as 
  \exfn{PW91x} = \exfn{Slater} + \exfn{PW91x2}.

\item[mPW] Adamo and Barone's 1998 modified PW91x GGA correction exchange functional
  ~\cite{dft:pw91,dft:mpw}. The full exchange functional is given by
  \exfn{Slater} + \exfn{mPW}.\index{mPW}

\end{description}

\subsubsection{Correlation Functionals}
\providecommand\corfn[1]{#1}
\begin{description}

\item[VWN3] correlation functional of Vosko, Wilk and Nusair, 1980 (equation III) 
  \cite{dft:vwn}. This is the form used in the Gaussian program.\index{VWN3}

\item[VWN5] correlation functional of Vosko, Wilk and Nusair, 1980 (equation
  V -- the recommended one). The VWN keyword is a synonym for VWN5~\cite{dft:vwn}.\index{VWN5}

%\item[FT97c] Filatov and Thiel 1997 (FT97) correlation functional
%  \cite{dft:ft97}.\index{FT97}

\item[LYP] correlation functional by Lee, Yang and Parr, 1988
  \cite{dft:lyp1,dft:lyp2}.\index{LYP}

\item[LYPr] 1998 modified \corfn{LYP} functional, which is the re-parameterized EDF1 version 
  with modified parameters (0.055, 0.158, 0.25, 0.3505)
  \cite{dft:lyp1,dft:lyp2,dft:edf1}.\index{LYPr}\index{EDF1}

\item[P86c] non-local part of the correlation functional of the Perdew 1986 correlation functional
  \cite{dft:p86}. PZ81 (1981 Perdew local) is usually used for the local part of the
  functional, with a total corelation functional of \index{P86}\index{PZ81}
  \corfn{P86c} + \corfn{PZ81}.

\item[PBEc] Perdew, Burke and Ernzerhof 1996 correlation functional, 
  defined as PW91c local and PBEc non-local correlation~\cite{dft:pbe}.\index{PBEc}

\item[PW91c] 1991 correlation functional of Perdew and Wang (the pwGGA-II functional)
  ~\cite{dft:pw91}.\index{PW91} This functional includes both the PW91c non-local and 
  PW91c local (ie PW92c) contributions. The non-local PW91c contribution may be determined
  as \corfn{PW91c} - \corfn{PW92c}.

\item[PW92c] local correlation functional of Perdew and Wang, 1992~\cite{dft:pw91,dft:pw92}.\index{PW91}
  This functional is the local contribution to the PW91c correlation functional.

\item[PW92ac] gradient correction to the PW91c correlation functional of Perdew and Wang,
  equation 16 from Ref.~\cite{dft:pw91,dft:pw92}.\index{PW91} The PWGGA-IIA functional
  as defined in the original reference is \corfn{PW91c} + \corfn{PW92ac}. 

\item[PZ81] local correlation functional of Perdew and Zunger, 1981~\cite{dft:pz81}.\index{PZ81}

\item[Wigner] original 1938 spin-polarized correlation functional~\cite{dft:wigner}.\index{Wigner}

\item[WL90c] Wilson and Levy's 1990 non-local spin-dependent correlation functional 
  (equation 15 from Ref.~\cite{dft:wl90}).\index{WL90}

\end{description}

\subsubsection{Standalone Functionals}
\providecommand\onefn[1]{#1}
\begin{description}

\item[LB94] asymptotically correct functional of Leeuwen and
  Baerends 1994~\cite{dft:lb94}. This functional improves description of the
  asymptotic density on the expense of core and inner valence.\index{LB94}

\item[B97] Becke 1997 functional~\cite{dft:b97}.\index{B97}

\item[B97-1] Hamprecht et al.'s 1998 re-parameterization of the 
  B97 functional~\cite{dft:b97-1}.\index{B97}\index{B97-1}

\item[B97-2] Modification of B97 functional in 2001 by Wilson, Bradley and Tozer 
  \cite{dft:b97-2}.\index{B97}\index{B97-2}

\item[B97-D] Grimme's re-parameterization of the B97-1 functional for use with 
empirical dispersion correction~\cite{dft:b97-d}.\index{B97}\index{B97-D}

\item[B97-K] Boese and Martin 2004 re-parameterization of the 
  B97-1 functional for kinetics~\cite{dft:b97-1}.\index{B97}\index{B97-K}

\item[HCTH] is a synonym for the HCTH407 functional (detailed below).
  \cite{dft:hcth407}.\index{HCTH}\index{HCTH407}

\item[1-4] The ``quarter'' functional of Menconi, Wilson and Tozer 
  \cite{dft:14}.\index{1-4}\index{1-4}

\item[HCTH93] Original 1998 HCTH functional, parameterized on a set of 
  93 training systems~\cite{dft:hcth93}.\index{HCTH}\index{HCTH93}

\item[HCTH120] The HCTH functional parameterized on a set of 120 training systems 
  in 2000~\cite{dft:hcth120}.\index{HCTH}\index{HCTH120}

\item[HCTH147] The HCTH functional parameterized on a set of 147 training systems 
  in 2000~\cite{dft:hcth120}.\index{HCTH}\index{HCTH147}

\item[HCTH407] The HCTH functional parameterized on a set of 407 training systems 
  in 2001~\cite{dft:hcth407}.\index{HCTH}\index{HCTH407}

\item[HCTH407p] The HCTH407 functional re-parameterized in 2003 on a set of 407 
  training systems and ammonia dimer to incorporate hydrogen bonding 
  \cite{dft:hcth407p}.\index{HCTH}\index{HCTH407}\index{HCTH407p}

\end{description}

\subsubsection{Combined functionals}
\providecommand\funexample[1]{\\{\tt #1 }}
\begin{description}

\item[Combine] is a universal keyword allowing users to manually
  construct arbitrary linear combinations of exchange and correlation 
  functionals from the list above. Even\index{Combine} fractional 
  Hartree--Fock exchange can be specified. This keyword is to be 
  followed by a string of functionals with associated weights. 
  The syntax is \verb|NAME=WEIGHT ...|. 
  As an example, B3LYP may be constructed as:
\begin{verbatim}
.DFT
 Combine HF=0.2 Slater=0.8 Becke=0.72 LYP=0.81 VWN=0.19
\end{verbatim}

The following GGA and hybrid functional aliases are defined within 
\dalton\ and provide further examples of the Combine keyword.

\item[SVWN5] is a sum of Slater functional and VWN (or VWN5) correlation
  functional. SVWN is a synonym for SVWN5. It is equivalent to
  \funexample{Combine Slater=1 VWN5=1}
  \index{SVWN}

\item[SVWN3] is a sum of the Slater exchange functional and VWN3 correlation
  functional. It is equivalent to the Gaussian program LSDA functional 
  and can alternatively be selected by following set of keywords
  \funexample{Combine Slater=1 VWN3=1}
  \index{SVWN3}

\item[LDA] A synonym for SVWN5 (or SVWN). \index{LDA}

\item[BVWN] is a sum of the \exfn{Slater} functional, \exfn{Becke} correction and 
  \corfn{VWN} correlation functional.  It is equivalent to 
  \funexample{Combine Slater=1 Becke=1 VWN=1}
  \index{BVWN}

\item[BLYP] is a sum of Slater functional, Becke88 correction and LYP
  correlation functional.  It is equivalent to 
  \funexample{Combine Slater=1 Becke=1 LYP=1}
  \index{BLYP}

\item[B3LYP] 3-parameter hybrid functional \cite{dft:b3lyp} equivalent to:
  \funexample{Combine HF=0.2 Slater=0.8 Becke=0.72 LYP=0.81 VWN=0.19}
  \index{B3LYP}

\item[B3LYPg] hybrid functional with VWN3 form used for
  correlation---this is the form used by the Gaussian quantum chemistry
  program. Keyword B3LYPGauss is a synonym for B3LYPg.\index{B3LYPG} 
  This functional can be explicitely set up by
  \funexample{Combine HF=0.2 Slater=0.8 Becke=0.72 LYP=0.81 VWN3=0.19}
  \index{B3LYP, Gaussian version}

\item[B1LYP] 1-parameter hybrid functional with 25\% exact exchange \cite{dft:b1lyp}. 
  Equivalent to: \funexample{Combine HF=0.25 Slater=0.75 Becke=0.75 LYP=1}
  \index{B3LYP}

\item[BP86] Becke88 exchange functional and Perdew86 correlation
  functional (with Perdew81 local correlation). The explicit form is:
  \funexample{Combine Slater=1 Becke=1 PZ81=1 P86c=1}
  \index{BP86}

\item[B3P86] variant of \verb|B3LYP| with VWN used for local
  correlation  and P86 for the nonlocal part.
  \funexample{Combine HF=0.2 Slater=0.8 Becke=0.72 P86c=0.81 VWN=1}
  \index{B3P86}

\item[B3P86g] variant of \verb|B3LYP| with VWN3 used for local
  correlation and P86 for the nonlocal part.
  This is the form used by the Gaussian quantum chemistry program.
  \funexample{Combine HF=0.2 Slater=0.8 Becke=0.72 P86c=0.81 VWN3=1}
  \index{B3P86}\index{B3P86, Gaussian version}

\item[BPW91] Becke88 exchange functional and PW91 correlation
  functional. The explicit form is:
  \funexample{Combine Slater=1 Becke=1 PW91c=1}
  \index{BPW91}

\item[B3PW91] 3-parameter Becke-PW91 functional, with PW91 correlation 
  functional. Note that PW91c includes PW92c local correlation, thus only
  excess PW92c local correlation is required (coefficient of 0.19).
  \funexample{Combine HF=0.2 Slater=0.8 Becke=0.72 PW91c=0.81 PW92c=0.19}
  \index{B3PW91}

\item[B1PW91] 1-parameter hybrid functional \cite{dft:b1lyp} equivalent to:
  \funexample{Combine HF=0.25 Slater=0.75 Becke=0.75 PW91c=1}
  \index{B1PW91}

\item[B86VWN] is a sum of \exfn{Slater} and \exfn{B86x} exchange functionals and 
  the \corfn{VWN} correlation functional. It is equivalent to 
  \funexample{Combine Slater=1 B86x=1 VWN=1}
  \index{B86VWN}

\item[B86LYP] is a sum of \exfn{Slater} and \exfn{B86x} exchange functionals and 
  the \corfn{LYP} correlation functional. It is equivalent to 
  \funexample{Combine Slater=1 B86x=1 LYP=1}
  \index{B86LYP}

\item[B86P86] is a sum of \exfn{Slater} and \exfn{B86x} exchange functionals and 
  the \corfn{P86c} correlation functional. It is equivalent to 
  \funexample{Combine Slater=1 B86x=1 P86c=1}
  \index{B86P86}

\item[B86PW91] is a sum of \exfn{Slater} and \exfn{B86x} exchange functionals and 
  the \corfn{PW91c} correlation functional.  It is equivalent to 
  \funexample{Combine Slater=1 B86x=1 PW91c=1}
  \index{B86PW91}

\item[BHandH] is an simple Half-and-half functional.
  \funexample{Combine HF=0.5 Slater=0.5 LYP=1}
  \index{BHandH}

\item[BHandHLYP] is another simple Half-and-half functional.
  \funexample{Combine HF=0.5 Slater=0.5 Becke=0.5 LYP=1}
  \index{BHandH}

\item[BW] is the sum of the Becke exchange and Wigner correlation
  functionals \cite{dft:wigner,dft:bw}.\index{BW}
  \funexample{Combine Slater=1 Becke=1 Wigner=1}

\item[CAMB3LYP] Coulomb Attenuated Method Functional of Yanai, Tew and
Handy \cite{dft:camb3lyp}. This functional accepts additional arguments
\verb|alpha|, \verb|beta| and \verb|mu| to modify the fraction of HF
exchange for short-range interactions, additional fraction of HF
exchange for long-range interaction and the interaction switching
factor $\mu$. This input can be specified as follows:
\begin{verbatim}
.DFT
 CAMB3LYP alpha=0.190 beta=0.460 mu=0.330
\end{verbatim}
\index{CAMB3LYP}


\item[rCAMB3LYP] Revised version of the CAMB3LYP Functional \cite{dft:rcamb3lyp} 
designed to give near piecewise linear behaviour of the energy vs. particle number. 
This functional accepts additional arguments \verb|alpha|, \verb|beta| and \verb|mu| 
with the same meanings and syntax as for the CAMB3LYP functional.\index{rCAMB3LYP}

\item[DBLYP] is a sum of the Double-Becke exchange functional and
  the LYP correlation functional 
  \cite{dft:becke88,dft:edf1,dft:lyp1,dft:lyp2}.\index{Double-Becke}
 \funexample{Combine Slater=1.030952 Becke=-8.44793 mBecke=10.4017 LYP=1}

\item[DBP86] is the sum of the Double-Becke exchange functional and
  the P86 correlation functional \cite{dft:becke88,dft:edf1,dft:p86}.\index{Double-Becke}
 \funexample{Combine Slater=1.030952 Becke=-8.44793 mBecke=10.4017 P86c=1 PZ81=1}

\item[DBPW91] is a sum of the Double-Becke exchange functional and
  the PW91 correlation functional \cite{dft:becke88,dft:edf1,dft:pw91}.\index{Double-Becke}
 \funexample{Combine Slater=1.030952 Becke=-8.44793 mBecke=10.4017 PW91c=1}

\item[EDF1] is a fitted functional of Adamson, Gill and Pople \cite{dft:edf1}.
  It is a linear combination of the Double-Becke exchange functional and the revised LYP
  functional LYPr.\index{EDF1}
 \funexample{Combine Slater=1.030952 Becke=-8.44793 mBecke=10.4017 LYPr=1}

\item[EDF2] is a linear combination of the Hartree-Fock exchange and the Double-Becke 
  exchange, Slater exchange, LYP correlation, revised LYPr correlation and VWN 
  correlation functionals \cite{dft:edf2}\index{EDF2}.
 \funexample{Combine HF=0.1695 Slater=0.2811 Becke=0.6227 mBecke=-0.0551 VWN=0.3029 LYP=0.5998 LYPr=-0.0053}

\item[G96VWN] is the sum of the G96 exchange functional and the VWN 
  correlation functional \cite{dft:g96}.
 \funexample{Combine Slater=1 G96x=1 VWN=1}

\item[G96LYP] is the sum of the G96 exchange functional and the LYP
  correlation functional \cite{dft:g96}.
 \funexample{Combine Slater=1 G96x=1 LYP=1}

\item[G96P86] is the sum of the G96 exchange functional and the P86
  correlation functional \cite{dft:g96}.
 \funexample{Combine Slater=1 G96x=1 P86c=1}

\item[G96PW91] is the sum of the G96 exchange functional and the PW91
  correlation functional \cite{dft:g96}.
 \funexample{Combine Slater=1 G96x=1 PW91c=1}

\item[G961LYP] is a 1-parameter B1LYP type functional with the exchange gradient 
  correction provided by the G96x functional \cite{dft:g961lyp}.
 \funexample{Combine HF=0.25 Slater=0.75 G96x=0.75 LYP=1}

\item[KMLYP] Kang and Musgrave 2-parameter hybrid functional with a mixture of
  Slater and Hartree--Fock exchange and VWN and LYP correlation functionals.
  \cite{dft:kmlyp}.
 \funexample{Combine HF=0.557 Slater=0.443 VWN=0.552 LYP=0.448}

\item[KT1] Slater-VWN5 functional with the KT GGA exchange correction
  \cite{dft:kt12,dft:kt12a}.\index{KT1}
 \funexample{Combine Slater=1 VWN=1 KT=-0.006}

\item[KT2] differs from KT1 only in that the weights of the Slater and
  VWN5 functionals are from an empirical fit (not equal to 1.0)
  \cite{dft:kt12,dft:kt12a}.\index{KT2}
  \funexample{Combine Slater=1.07173 VWN=0.576727 KT=-0.006}

\item[KT3] a hybrid functional of Slater, OPTX and KT exchange with the
  LYP correlation functional \cite{dft:kt3}. The explicit form is
  \funexample{Combine Slater=1.092 KT=-0.004 LYP=0.864409 OPTX=-0.925452}
  \index{KT3}

\item[LG1LYP] is a 1-parameter B1LYP type functional with the exchange gradient
  correction provided by the LG93x functional \cite{dft:g961lyp}.
 \funexample{Combine HF=0.25 Slater=0.75 LG93x=0.75 LYP=1}

\item[mPWVWN] is the combination of mPW exchange and VWN correlation functionals 
   \cite{dft:mpw,dft:vwn}.\index{mPWVWN}
  \funexample{Combine Slater=1 mPW=1 VWN=1}.

\item[mPWLYP] is the combination of mPW exchange and LYP correlation functionals 
   \cite{dft:mpw,dft:vwn}.\index{mPWLYP}
  \funexample{Combine Slater=1 mPW=1 LYP=1}.

\item[mPWP86] is the combination of mPW exchange and P86 correlation functionals 
   \cite{dft:mpw,dft:vwn}.\index{mPWP86}
  \funexample{Combine Slater=1 mPW=1 P86c=1 PZ81=1}.

\item[mPWPW91] is the combination of mPW exchange and PW91 correlation functionals 
   \cite{dft:mpw,dft:pw91}.\index{mPWPW91}
  \funexample{Combine Slater=1 mPW=1 PW91c=1}.

\item[mPW3PW91] is a 3-parameter combination of mPW exchange and PW91 correlation
  functionals, with the PW91 (PW92c) local correlation~\cite{dft:mpw}.\index{mPW3PW91}
  \funexample{Combine HF=0.2 Slater=0.8 mPW=0.72 PW91c=0.81 PW92c=0.19}.

\item[mPW1PW91] is a 1-parameter combination mPW exchange and PW91 correlation
  functionals with 25\% Hartree--Fock exchange \cite{dft:mpw}.\index{mPW1PW91}
  \funexample{Combine HF=0.25 Slater=0.75 mPW=0.75 PW91c=1}.

\item[mPW1K] optimizes mPW1PW91 for kinetics of H abstractions, with 42.8\% Hartree--Fock
  exchange \cite{dft:mpw1k}.\index{mPW1PW91}
  \funexample{Combine HF=0.428 Slater=0.572 mPW=0.572 PW91c=1}.

\item[mPW1N] optimizes mPW1PW91 for kinetics of H abstractions, with 40.6\% Hartree--Fock
  exchange \cite{dft:mpw1n}.\index{mPW1N}
  \funexample{Combine HF=0.406 Slater=0.594 mPW=0.594 PW91c=1}.

\item[mPW1S] optimizes mPW1PW91 for kinetics of H abstractions, with 6\% Hartree--Fock
  exchange \cite{dft:mpw1s}.\index{mPW1S}
  \funexample{Combine HF=0.06 Slater=0.94 mPW=0.94 PW91c=1}.

\item[OLYP] is the sum of the OPTX exchange functional with the
  LYP correlation functional \cite{dft:optx,dft:lyp1,dft:lyp2}.
  \funexample{Combine Slater=1.05151 OPTX=-1.43169 LYP=1}
  \index{OLYP}

\item[OP86] is the sum of the OPTX exchange functional with the
  P86 correlation functional \cite{dft:optx,dft:p86}.
  \funexample{Combine Slater=1.05151 OPTX=-1.43169 P86c=1 PZ81=1}
  \index{OP86}

\item[OPW91] is the sum of the OPTX exchange functional with the
  PW91 correlation functional \cite{dft:optx,dft:pw91}.
  \funexample{Combine Slater=1.05151 OPTX=-1.43169 PW91c=1}
  \index{OPW91}

\item[PBE0] a hybrid functional of Perdew, Burke and Ernzerhof with
  0.25 weight of exact exchange, 0.75 of \verb|PBEx| exchange functional and
  the \verb|PBEc| correlation functional \cite{dft:pbe0}.
  Alternative aliases are PBE1PBE or PBE0PBE.\index{PBE0}
  \funexample{Combine HF=0.25 PBEx=0.75 PBEc=1}

\item[PBE] same as above but with exchange estimated exclusively by
  \exfn{PBEx} functional \cite{dft:pbe}.\index{PBE} Alias of PBEPBE. 
  This is the form used by CADPAC and NWChem quantum chemistry programs.
  \funexample{Combine PBEx=1 PBEc=1}\\
  Note that the Molpro quantum chemistry program uses the \corfn{PW91c}
  non-local correlation functional instead of \corfn{PBEc}, which is 
  equivalent to the following:
  \funexample{Combine PBEx=1 PW91c=1}.

\item[RPBE] is a revised PBE functional that employs the 
  \exfn{RPBEx} exchange functional.
  \funexample{Combine RPBEx=1 PBEc=1}

\item[revPBE] is a revised PBE functional that employs the 
  \exfn{revPBEx} exchange functional.
  \funexample{Combine revPBEx=1 PBEc=1}

\item[mPBE] is a revised PBE functional that employs the 
  \exfn{mPBEx} exchange functional.
  \funexample{Combine mPBEx=1 PBEc=1}

\item[PW91VWN] is the combination of PW91 exchange and VWN correlation functionals 
   \cite{dft:pw91,dft:vwn}.\index{PW91}
  \funexample{Combine PW91x=1 VWN=1}.

\item[PW91LYP] is the combination of PW91 exchange and LYP correlation functionals 
   \cite{dft:pw91,dft:lyp1,dft:lyp2}.\index{PW91}
  \funexample{Combine PW91x=1 LYP=1}.

\item[PW91P86] is the combination of PW91 exchange and P86 (with Perdew 1981 local) 
  correlation functionals \cite{dft:pw91,dft:pw86,dft:pz81}.
  \funexample{Combine PW91x=1 P86c=1 PZ81=1}.

\item[PW91PW91] is the combination of PW91 exchange and PW91 correlation functionals. 
  Equivalent to PW91 keyword \cite{dft:pw91}.
  \funexample{Combine PW91x=1 PW91c=1}.

\item[XLYP] is a linear combination of \exfn{Slater}, \exfn{Becke} and \exfn{PW91x}
  exchange and \corfn{LYP} correlation functionals \cite{dft:xlyp,dft:x3lyp}.\index{XLYP}
  \funexample{Combine Slater=1 Becke=0.722 PW91x=0.347 LYP=1}.

\item[X3LYP] is a linear combination of Hartree--Fock, \exfn{Slater}, \exfn{Becke} 
  and \exfn{PW91x} exchange and \corfn{VWN} and \corfn{LYP} correlation functionals 
  \cite{dft:xlyp,dft:x3lyp}.\index{X3LYP}
  \funexample{Combine HF=0.218 Slater=0.782 Becke=0.542 PW91x=0.167 VWN=0.129 LYP=0.871}

\end{description}


Note that combinations of local and non-local correlation functionals
can also be generated with the Combine keyword. For example,
\verb|Combine P86c=1 PZ81=1| combines the PZ81 local and P86c non-local 
correlation functional, whereas \verb|Combine VWN=1 P86c=1| 
combines the VWN local and P86 non-local correlation functionals.


Linear combinations of all exchange and correlation functionals listed above
are possible with the \verb|Combine| keyword.

\subsubsection{Double-hybrid functionals}
\begin{description}
\item[B2PLYP] is the double hybrid of Ref.~\cite{dft:b2plyp}

\item[B2TPLYP] is a modification of the B2PLYP functional for thermodynamics~\cite{dft:b2tplyp}

\item[mPW2PLYP] a double hybrid using an alternative GGA exchange contribution and tested on the G3/05 benchmark dataset~\cite{dft:mpw2plyp}

\item[B2GPLYP] is a modification of the B2PLYP functional for general purpose calculations~\cite{dft:b2tplyp}

\item[B2PIPLYP] is a form related to B2PLYP but designed to give better performance for sterically crowed or stacked aromatic ring systems~\cite{dft:b2piplyp}

\item[PBE0DH] a theoretically derived double-hybrid parameterization~\cite{dft:pbe0dh}

\end{description}

Note that at present double-hybrid functionals are implemented for energies only, analytic gradient contributions are not implemented.

\pagebreak[3]
\subsection{\label{ref-haminp}\Sec{HAMILTONIAN}}

{\bf Purpose:}

Add extra terms to the Hamiltonian (for finite field\index{finite field} calculations).

\begin{description}
\item[\Key{FIELD TERM}]
  Default = no fields. \\
  \verb"READ (LUINP,  *  ) EFIELD(NFIELD)" \\
  \verb"READ (LUINP,'(A)') LFIELD(NFIELD)" \\
  Enter field strength (in atomic units) and property label on separate lines
  where label is a \molecule-style property label on file \verb|AOPROPER|
  produced by the property module, see Chapter~\ref{ch:hermit}.
  The calculation of the necessary property integral(s) must be requested
  in the \quotekw{**INTEGRALS} input module. \\
  NOTE: This keyword may be repeated several times for adding more than
  one finite field (max 20 fields).

\item[\Key{PRINT}]
  Default = 0.\\
  \verb"READ (LUINP,*) IPRH1" \\
  If greater than zero:
  print the one-electron Hamiltonian matrix, including
  specified field-dependent terms, in AO basis.
\end{description}

\noindent{\bf Comments:}

Up to \mxfelt simultaneous fields may be specified by repeating the
\quotekw{\Key{FIELD TERM}} keyword.
The field integrals are read from file \verb|AOPROPER| with the specified label.

\pagebreak[3]
\subsection{\label{ref-mp2inp}\Sec{MP2 INPUT}}

{\bf Purpose:}

\index{MP2}\index{M{\o}ller-Plesset!second-order}
To direct MP2 calculation. Note that MP2 energies as well as
properties also are available through the coupled cluster module, see
Chapter~\ref{ch:CC}.

\begin{description}
\item[\Key{MP2 FROZEN}]
  Default = no frozen orbitals\\
  \kw{READ (LUINP,*) (NFRMP2(I),I=1,NSYM)} \\
  Orbitals frozen in MP2 calculation

\item[\Key{PRINT}]
  \kw{READ (LUINP,*) IPRMP2} \\
  Print level for MP2 calculation
\end{description}


\noindent{\bf Comments:}


The MP2 module expects canonical Hartree--Fock orbitals. The MP2 module will
check the orbitals and exits if the Fock matrix has off-diagonal non-negligible
elements.
If starting from saved canonical Hartree--Fock orbitals from a previous calculations,
although no Hartree--Fock calculation will be done
the number of occupied Hartree--Fock orbitals in each symmetry must anyway be
specified with the \quotekw{.DOUBLY OCCUPIED} under \quotekw{*SCF INPUT}.

The MP2 calculation will produce the MP2 energy and the natural orbitals
for the density matrix through second order.  The primary purpose of
this option is to generate good starting orbitals for CI or MCSCF wave
functions, but it
may of course also be used to obtain the MP2 energy, perhaps with frozen
core orbitals. {\em For valence MCSCF calculations it is recommended that the
\quotekw{\Key{MP2 FROZEN}} option is used in order to obtain the appropriate
correlating orbitals\index{correlating orbitals}\index{MCSCF} as start
for an MCSCF calculation.\/}  As the commonly
used basis sets do not contain correlating orbitals for the core
orbitals and as the core correlation energy therefore becomes arbitrary,
the \quotekw{\Key{MP2 FROZEN}} option can also be of benefit in MP2 energy
calculations.

\pagebreak[3]
\subsection{\label{ref-nevpt2inp}\Sec{NEVPT2 INPUT}}

{\bf Purpose:}

\index{NEVPT2}\index{multireference PT!second-order}
Calculation of the second order correction to the energy for a
CAS--SCF or CAS--CI zero order wavefunction.
The user is referred to Chapter~\ref{ch:nevpt2} on
page~\pageref{ch:nevpt2}  for a brief
introduction to the $n$--electron valence state second order
perturbation theory (NEVPT2).

\begin{description}
\item[\Key{THRESH}]
 Default = 0.0D0\\
  \kw{READ (LUINP,*) THRNEVPT} \\
  Threshold to discard small coefficients in the CAS wavefunction


\item[\Key{FROZEN}]
  Default = no frozen orbitals\\
  \kw{READ (LUINP,*) (NFRNEVPT2(I),I=1,NSYM)} \\
  Orbitals frozen in NEVPT2 calculation

\item[\Key{STATE}]
 No default provided\\
\kw{READ (LUINP,*) ISTNEVCI} \\
Root number in a CASCI calculation. This keyword is unnecessary
(ignored) in the CASSCF case.
\end{description}


\noindent{\bf Comments:}


%The present version of the NEVPT2 module requires the
%\quotekw{\Key{DETERMINANTS}} option  to be set.

The use of canonical orbitals for the core and virtual orbitals is
strongly recommended since this choice guarantees compliance of the
results with a totally invariant form of NEVPT2 (see page~\pageref{ch:nevpt2}).

At present the NEVPT2 module can deal with active spaces of dimension
not higher than 14.

\pagebreak[3]
\subsection{\label{ref-optinp}\Sec{OPTIMIZATION}}

{\bf Purpose:}

To change defaults for optimization of an MCSCF\index{MCSCF} wave function.
Some of the options also affect a QC-HF optimization.

\begin{description}
\item[\Key{ABSORPTION}]
  \kw{READ (LUINP,'(A8)') RWORD} \\
  RWORD = ` LEVEL 1', ` LEVEL 2', or ` LEVEL 3'\\
  Orbital absorption\index{orbital absorption} in MCSCF optimization
  at level 1, 2, or 3, as specified
  (normally level 3, see comments below).  This keyword may be repeated to
  specify more than one absorption level, the program will then begin with
  the lowest level requested and, when that level is converged,
  disable the lower level and shift to the next level.
% 940816-hjaaj: The following may not be true for RAS ????
% Absorption at several levels are only useful in
% first macro iteration, therefore the lower levels are disabled after
% convergence.

\item[\Key{ACTROT}]
  include specified active-active rotations
\begin{verbatim}
  READ (LUINP,*) NWOPT
  DO I = 1,NWOPT
    READ (LUINP,*) JWOP(1,I),JWOP(2,I)
  END DO
\end{verbatim}
  JWOP(1:2,I) denotes normal molecular orbital numbers (not the active
  orbital numbers).

\item[\Key{ALWAYS ABSORPTION}]
  Absorption\index{orbital absorption} in all MCSCF macro iterations
  (default is to disable absorption in
  local region or after \quotekw{\Key{MAXABS}} macro iterations, whichever comes first).
  Absorption is always disabled after Newton-Raphson algorithm has been used,
  and thus also when doing \quotekw{\Key{CORERELAX}},
  because absorption may cause variational collapse if the desired state is excited.

\item[\Key{CI PHP MATRIX}]
  Default : MAXPHP = 1 (Davidson's algorithm)\\
  \kw{READ (LUINP,*) MAXPHP} \\
  PHP is a subblock of the CI matrix which is calculated explicitly
  in order to obtain improved CI trial vectors compared to the
  straight Davidson\index{Davidson algorithm} algorithm.  The
  configurations corresponding to
  the lowest diagonal elements are selected, unless
  \quotekw{\Key{PHPRESIDUAL}} is specified.
  \kw{MAXPHP} is the maximum dimension of PHP, the actual dimension
  will be less if \kw{MAXPHP} will split degenerate configurations.

\item[\Key{COREHOLE}]
  \kw{READ (LUINP,*) JCHSYM,JCHORB} \\
  JCHSYM = symmetry of core orbital\\
  JCHORB = the orbital in symmetry JCHSYM with a single core hole\\
  Single core hole\index{core hole} MCSCF calculation. The calculation must be of RAS type
  with only the single core-hole orbital in RAS1, the state specified with
  \quotekw{\Key{STATE}} is optimized with the core-hole orbital
  frozen\index{frozen core hole}.
  The specified core hole orbital must be either inactive or
  the one RAS1 orbital, if it is inactive then it will switch places with
  the RAS1 orbital and it will not be possible to also
  specify \quotekw{\Key{REORDER}}. If explicit reordering is required you must reorder
  the core orbital yourself and let \kw{JCHORB} point to the one RAS1 orbital.
  Orbital absorption is activated at level 2. See comments below for more information.

\item[\Key{CORERELAX}]
  (ignored if \quotekw{\Key{COREHOLE}} isn't also specified)\\
  Optimize state with relaxed core orbital\index{relaxed core hole} (using Newton-Raphson algorithm,
  it is not necessary to explicitly specify \quotekw{\Key{NR ALWAYS}}).
  It is assumed that this calculation follows an optimization
  with frozen core orbital and that the orbital has already been
  moved to the RAS1 space ({\it i.e.\/}, the specific value of
  \quotekw{JCHORB} under \quotekw{\Key{COREHOLE}} is ignored). Any
  orbital absorption   will be ignored.

\item[\Key{DETERMINANTS}]
  Use determinant\index{determinants} basis instead of CSF basis (see comments).

\item[\Key{EXACTDIAGONAL}]
  Default for RAS calculations.\\
  Use the exact orbital Hessian\index{orbital Hessian} diagonal.

\item[\Key{FOCKDIAGONAL}]
  Default for CAS calculations.\\
  Use an approximate orbital Hessian diagonal which only uses Fock
  contributions.

\item[\Key{FOCKONLY}]
  Activate TRACI option (default : program decides).\\
  Modified TRACI option where all orbitals, also active orbitals, are
  transformed to Fock type orbitals in each iteration.

\item[\Key{FROZEN CORE ORBITALS}]
  \kw{READ (LUINP,*) (NFRO(I),I=1,NSYM)} \\
  Frozen orbitals : Number of inactive (doubly occupied) orbitals to be frozen
  in each symmetry (the first NFRO(I) in symmetry I) in MCSCF.\index{frozen orbitals!MCSCF}
  Active orbitals and specific inactive orbitals can be frozen with \quotekw{.FREEZE}
  under \Sec{ORBITAL INPUT}.
  Frozen orbitals in SCF are specified in the \Sec{SCF INPUT} input module.

\item[\Key{MAX CI}]
  \kw{READ (LUINP,*) MAXCIT} \\
  maximum number of CI iterations before MCSCF (default = 3).

\item[\Key{MAX MACRO ITERATIONS}]
  \kw{READ (LUINP,*) MAXMAC} \\
  maximum number of macro iterations in MCSCF optimization (default = 25).
\index{iteration number!MCSCF macro, max}

\item[\Key{MAX MICRO ITERATIONS}]
  \kw{READ (LUINP,*) MAXJT} \\
  maximum number of micro iterations per macro iteration in MCSCF optimization
  (default = 24).

\item[\Key{MAXABS}]
  \kw{READ (LUINP,*) MAXABS} \\
  maximum number of macro iterations with 
  absorption\index{orbital absorption} (default = 3).

\item[\Key{MAXAPM}]
  \kw{READ (LUINP,*) MAXAPM} \\
  maximum number orbital absorptions\index{orbital absorption} within
  a macro iteration
  (APM : Absorptions Per Macro iteration; default = 5)

\item[\Key{NATONLY}]
  Activate TRACI option (default : program decides).\\
  Modified TRACI option where the inactive and secondary orbitals are not
  touched (these two types of orbitals are already natural orbitals).

\item[\Key{NEO ALWAYS}]
  Always norm-extended optimization (never switch to New\-ton-Raph\-son).
  Note: \quotekw{\Key{NR ALWAYS}} and \quotekw{\Key{CORERELAX}}
  takes precedence over \quotekw{\Key{NEO ALWAYS}}.

\item[\Key{NO ABSORPTION}]
  Never orbital absorption\index{orbital absorption} (default settings removed)

\item[\Key{NO ACTIVE-ACTIVE ROTATIONS}]
  No active-active rotations in RAS optimization.

\item[\Key{NOTRACI}]
  Disable TRACI option (default : program decides).

\item[\Key{NR ALWAYS}]
  Always Newton-Raphson optimization (never NEO optimization).
  Note: \quotekw{\Key{NR ALWAYS}} takes precedence over
  \quotekw{\Key{NEO ALWAYS}}.

\item[\Key{OLSEN}]
  Use Jeppe Olsen's generalization of the Davidson
  algorithm\index{Davidson algorithm}.

\item[\Key{OPTIMAL ORBITAL TRIAL VECTORS}]
  Generate "optimal" orbital trial
  vectors~\cite{hjajpjhajcp87}.\index{optimal orbital trial vector} 

\item[\Key{ORB\_TRIAL VECTORS}]
  Use also orbital trial vectors as start vectors for auxiliary roots
  in each macro iteration (CI trial vectors are always generated).

\item[\Key{PHPRESIDUAL}]
  Select configurations for PHP matrix based on largest residual
  rather than lowest diagonal elements.

\item[\Key{SIMULTANEOUS ROOTS}]
  Default : NROOTS = ISTATE, LROOTS = NROOTS\\
  \kw{READ (LUINP,*) NROOTS, LROOTS} \\
  NROOTS = Number of simultaneous roots in NEO\\
  LROOTS = Number of simultaneous roots in NEO at start

\item[\Key{STATE}]
  Default = 1\\
  \kw{READ (LUINP,*) ISTATE} \\
  Index of MCSCF Hessian\index{MCSCF Hessian} at convergence (1 for
  lowest state, 2 for first
  excited state, etc. within the spatial symmetry\index{symmetry} and
  spin symmetry\index{spin symmetry}
  specified under \Sec{CONFIGURATION INPUT}).

\item[\Key{SYM CHECK}]
  Default: ICHECK = 2 when NROOTS $>$ 1, else ICHECK = -1.\\
  \kw{READ (LUINP,*) ICHECK} \\
  Check symmetry of the LROOTS start CI-vectors and remove those which
  have wrong symmetry ({\it e.g.\/} vectors of delta symmetry in a sigma
  symmetry calculation).
\begin{verbatim}
  ICHECK < 0  : No symmetry check.
  ICHECK = 1  : Remove those vectors which do not have the same
                symmetry as the ISTATE vector, reassign ISTATE.
  ICHECK = 2  : Remove those vectors which do not have the same
                symmetry as the lowest state vector before selecting
                the ISTATE vector.
  other values: check symmetry, do not remove any CI vectors.
\end{verbatim}
  The \quotekw{\Key{SIMULTANEOUS ROOTS}} input will automatically be
  updated if CI vectors are removed.

\item[\Key{THRCGR}]
  \kw{READ (LUINP,*) THRCGR} \\
  Threshold for print of CI gradient. Default is 0.1D0.

\item[\Key{THRESH}]
  Default = 1.0D-05\\
  \kw{READ (LUINP,*) THRMC} \\
  Convergence threshold for energy gradient in MCSCF optimization.
  The convergence of the energy will be approximately the square of this
  number.

\item[\Key{TRACI}]
  Activate TRACI option (default : program decides).\\
  Active orbitals are transformed to natural orbitals and the CI-vectors
  are counter-rotated such that the CI states do not change.  The
  inactive and secondary orbitals are transformed to Fock type orbitals
  (corresponding to canonical orbitals for closed shell Hartree--Fock).
  For RAS wave functions the active orbitals are only transformed
  within their own class (RAS1, RAS2, or RAS3) as the wave function is
  not invariant to orbital rotations between the classes.  For RAS, the
  orbitals are thus not true natural orbitals, the density matrix is
  only block diagonalized.  Use \quotekw{\Key{IPRCNO}} (see
  p.~\pageref{ref-priinp})   to control output from this
  transformation.

\end{description}


\noindent{\bf Comments:}

COREHOLE: Single core-hole\index{core hole} calculations are
performed as RAS calculations where the opened core orbital is in
the RAS1 space.  The RAS1 space must therefore contain one and
only one orbital when the COREHOLE option is used, and the
occupation must be restricted to be exactly one electron. The
orbital identified as the core orbital must be either inactive or
the one RAS1 orbital, if it is inactive it will switch places with
the one RAS1 orbital. The core orbital (now in RAS1) will be
frozen in the following optimization. After this calculation has
converged, the CORERELAX option may be added and the core orbital
will be relaxed\index{relaxed core hole}.  When CORERELAX is
specified it is assumed that the calculation was preceded by a
frozen core\index{frozen core hole} calculation, and that the
orbital has already been moved to the RAS1 space. Default
corresponds to the main peak, shake-up energies may be obtained by
specifying \quotekw{\Key{STATE}} larger than one. Absorption is
very beneficial in core hole calculations because of the large
orbital relaxation following the opening of the core hole.

ABSORPTION: Absorption\index{orbital absorption} level 1 includes occupied - occupied rotations
only (including active-active rotations); level 2 adds inactive -
secondary rotations and only active - secondary rotations are excluded
at this level; and finally level 3 includes all non-redundant rotation
for the frozen CI vector.  Levels 1 and 2 require the same integral
transformation (because the inactive - secondary rotations are
performed using the P-supermatrix integrals) and level 1 is therefore
usually not used. Level 3 is the normal and full level, but it can be
advantageous to activate level 2 together with level 3 if big
inactive-active or occupied-occupied rotations are expected.

ORB\_TRIAL: Orbital trial\index{orbital trial vector}\ vectors as
start vectors can be used for
excited states and other calculations with more than one simultaneous
roots.  The orbital start trial vectors are based on the eigenvectors of
the NEO matrix in the previous macro iterations.  However, they are
probably not cost-effective for multiconfiguration calculations where
optimal orbital trial\index{optimal orbital trial vector} vectors are
used and they are therefore not used
by default.

SYM CHECK: The symmetry check is performed on the matrix element
$\langle VEC1 \mid oper \mid VEC2\rangle$, where "oper" is
the CI-diagonal.
It is recommended and the default to use \quotekw{\Key{SYM CHECK}}
for excited states, including
CI vectors of undesired symmetries is a waste of CPU time.

DETERMINANTS: The kernels of the CI sigma routines and density matrix
routines are always performed in determinant\index{determinants}
basis.  However, this
keyword specifies that the external representation is Slater
determinants as well.  The default is that the external representation
is in CSF\index{CSF}\index{configuration state function} basis as
described in chapter 8 of MOTECC-90.  The external
CSF\index{CSF}\index{configuration state function} basis is
generally to be preferred to be sure that the converged
state(s) have pure and correct spin symmetry\index{spin symmetry}, and
to save disk space.
It is recommended to specify \quotekw{\Key{PLUS COMBINATIONS}} under
\quotekw{\Sec{CI VECTOR}} for
calculations on singlet states\index{singlet state} with
determinants\index{determinants},
in particular for
excited singlet\index{excited state} states which often have lower lying triplet states.


\pagebreak[3]
\subsection{\label{ref-orbinp}\Sec{ORBITAL INPUT}}

{\bf Purpose:}

To define an initial set of molecular orbitals\index{molecular orbital!initial set}
and to control the use of super symmetry\index{supersymmetry}, frozen
orbitals\index{frozen orbitals}, deletion of orbitals\index{delete orbitals},
reordering and punching of orbitals.

\begin{description}
\item[\Key{5D7F9G}]
  Delete unwanted components in Cartesian d, f, and g orbitals.
  (s in d; p in f; s and d in g). By default, \her\ provides atomic
  integrals in spherical basis, and this option should therefore not
  be needed.

\item[\Key{AO DELETE}]
  \kw{READ (LUINP,*) THROVL } \\
  Delete MO's based on canonical orthonormalization using eigenvalues
  and eigenvectors of the AO overlap matrix.\index{linear dependence} \\
  THROVL: limit for basis
  set numerical linear dependence (eigenvectors with eigenvalue less
  than THROVL are excluded). Default is 1.0$\cdot$10$^{-6}$.

\item[\Key{CMOMAX}]
  \kw{READ (LUINP,*) CMAXMO} \\
  Abort calculation if the absolute value of any initial MO coefficient is
  greater than CMAXMO (default : CMAXMO = $10^3$).  Large MO coefficients
  can cause significant loss of accuracy in the two-electron integral
  transformation.

\item[\Key{DELETE}]
  \kw{READ (LUINP,*) (NDEL(I),I = 1,NSYM) } \\
  Delete orbitals\index{deleted orbitals}, {\it i.e.\/} number of molecular orbitals
  in symmetry \quotekw{I} is number of basis functions in symmetry \quotekw{I} minus
  \quotekw{NDEL(I)}. \\
  Only for use with \quotekw{.MOSTART} options \quotekw{FORM12} or \quotekw{FORM18}, 
  it cannot be used with \quotekw{H1DIAG}, \quotekw{EWMO}, or \quotekw{HUCKEL},
  and the other restart options as \quotekw{NEWORB} reads this information from file
  and this will overwrite what ever was specified here.

\item[\Key{FREEZE}]
  Default: no frozen orbitals.
\begin{verbatim}
  READ (LUINP,*) (NNOR(ISYM), ISYM = 1,NSYM)
  DO ISYM = 1,NSYM
    IF (NNOR(ISYM) .GT. 0) THEN
      READ (LUINP,*) (INOROT(I), I = 1,NNOR(ISYM))
      ...
    END IF
  END DO
\end{verbatim}
  where \kw{INOROT} = orbital numbers of the orbitals to be
          frozen\index{frozen orbitals!MCSCF and SCF} (not rotated)
          in symmetry \quotekw{ISYM} both in SCF and MCSCF
          after any reordering (counting from 1 in each symmetry).\\
  Must be specified after all options reducing the number of orbitals.
  Frozen occupied orbitals in SCF can only be specified in the \Sec{SCF INPUT} input module
  and frozen inactive orbitals in MCSCF can only be specified in the \Sec{OPTIMIZATION}
  input module.

\item[\Key{GRAM-SCHMIDT ORTHONORMALIZATION}]
  Default.\\
  Gram--Schmidt orthonormalization\index{orthonormalization!Gram--Schmidt} of input orbitals.

\item[\Key{LOCALIZATION}]
  \kw{READ (LUINP,*) REWORD} \\
  Specify that the doubly occupied (inactive) orbitals should be localized after SCF 
  or MCSCF is converged.
  Two options for localization of the orbitals are currently available:
  \begin{description}
  \item[{\tt BOYLOC\ }] Use the Boys localization scheme~\cite{Boyloc}.
  %\item[{\tt PIPLOC\ }] Use the Pipek-Mezey localization scheme~\cite{}.
  % aug 04: PIPLOC is not implemented yet.
  \item[{\tt SELECT\ }] Select a subset of the orbitals to be localized. The first
  line following this option contains the number orbitals to localize per symmetry,
  and the following lines contain which orbitals to localize within each symmetry,
  one line per symmetry with orbitals to localize.
  This option is typically used for localizing degenerate 
  core orbitals, leaving all other orbitals intact.
  \begin{verbatim}
         READ(LUCMD,*)(NBOYS(I),I=1,NSYM)
         DO I=1,NSYM
            IF (NBOYS(I).GT.0) THEN
               READ(LUCMD,*)(BOYSORB(J,I),J=1,NBOYS(I))
            END IF
         END DO
   \end{verbatim}
  \end{description}

\item[\Key{MOSTART}]
   Molecular orbital input\index{molecular orbital}\\
   \kw{READ (LUINP,'(1X,A6)') RWORD} \\
   where RWORD is one of the following:
   \begin{description}
   \item[{\tt FORM12\ }] Formatted input (6F12.8)  supplied after
        \Sec{*MOLORB} or \Sec{*NATORB} keyword. Use also \quotekw{.DELETE}
        if orbitals were deleted.
   \item[{\tt FORM18\ }] Formatted input (4F18.14) supplied after
        \Sec{*MOLORB} or \Sec{*NATORB} keyword. Use also \quotekw{.DELETE}
        if orbitals were deleted.
   \item[{\tt EWMO\ }] Start orbitals generated by projecting the EWMO
        H{\"u}ckel eigenvectors in a good generally contracted ANO basis set
        onto the present basis set.
        The EWMO model generally works better than the EHT model.
        Default initial guess for molecules in which all atoms have a nuclear charge
        less than or equal to 36.
        Note: EWMO/HUCKEL is not implemented yet if any element has a
        charge larger than 36).
        The start density will thus be close to one generated from atomic densities,
        but with molecular valence interaction in the EWMO model.
        This works a lot better than using a minimal basis set for EWMO.
   \item[{\tt HUCKEL\ }] Start orbitals generated by projecting the EHT
        H{\"u}ckel eigenvectors in a good generally contracted ANO basis set
        onto the present basis set.
        Note: EWMO/HUCKEL is not implemented yet if any element has a
        charge larger than 36.
        The start density will thus be close to one generated from atomic densities,
        but with molecular valence interaction in the H{\"u}ckel model.
        This works a lot better than using a minimal basis set for H{\"u}ckel.
   \item[{\tt H1DIAG\ }] Start orbitals that diagonalize
        one-electron Hamiltonian matrix (default
        for molecules containing elements with a nuclear larger than 36).
   \item[{\tt NEWORB\ }] Input from {\sir} restart file
                            (\verb|SIRIUS.RST| file) with label \quotekw{NEWORB  }
   \item[{\tt OLDORB\ }] Input from {\sir} restart file
                            (\verb|SIRIUS.RST| file) with label \quotekw{OLDORB  }
   \item[{\tt SIRIFC\ }] Input from {\sir} interface file ("\verb|SIRIFC|")
\end{description}

%Oct 2003 hjaaj: now NOSUPSYM always default,
% SUPSYM must be activated explicitly with .SUPSYM
%\item[\Key{NOSUPSYM}]
%  Deactivate automatic identification of "super
%  symmetry"\index{supersymmetry} (see
%  comments). This is automatically enforced in case of \aba\ or
%  \resp\ calculations.

\item[\Key{PUNCHINPUTORBITALS}]
  Punch input orbitals with label \Sec{*MOLORB}, Format (4F18.14).
  These orbitals may {\it e.g.\/} be transferred to another computer and
  read there with \quotekw{.MOSTART} followed by \quotekw{ FORM18} on
  next line from this input section.

\item[\Key{PUNCHOUTPUTORBITALS}]
  Punch final orbitals with label \Sec{*MOLORB}, Format (4F18.14).
  These orbitals may {\it e.g.\/} be transferred to another computer and
  read there with \quotekw{.MOSTART} followed by \quotekw{ FORM18} on
  next line from this input section.

\item[\Key{REORDER}]
Default: no reordering.
\begin{verbatim}
  READ (LUINP,*) (NREOR(I), I = 1,NSYM)
  DO I = 1,NSYM
     IF (NREOR(I) .GT. 0) THEN
        READ (LUINP,*) (IMONEW(J,I), IMOOLD(J,I), J = 1,NREOR(I))
     END IF
  END DO
  NREOR(I) = number of orbitals to be reordered in symmetry I
  IMONEW(J,I), IMOOLD(J,I) are orbital numbers in symmetry I.

For example if orbitals 1 and 5 in symmetry 1 should change place, specify
.REORDER
 2 0 0 0
 1 5 5 1
\end{verbatim}
  Reordering of molecular orbitals (see comments).

\item[\Key{SUPSYM}]
  Default is NOSUPSYM.\\
  Enforce automatic identification of "super
  symmetry"\index{supersymmetry} (see comments).

\item[\Key{SYMMETRIC ORTHONORMALIZATION}]
  Default: Gram-Schmidt orthonormalization\\
  Symmetric orthonormalization of input
  orbitals\index{orthonormalization!symmetric}.

\item[\Key{THRSSY}]
  \kw{READ (LUINP,*) THRSSY} \\
  Threshold for identification of "super
  symmetry"\index{supersymmetry} and degeneracies among
  "super symmetries" from matrix elements of the kinetic energy matrix
  (default: 5.0D-8).

\end{description}


\noindent{\bf Comments:}

If \quotekw{\Key{SUPSYM}} is specified, then
{\sir} automatically identifies "super symmetry"\index{super symmetry!orbitals},
{\it i.e.\/} it assigns orbitals to the irreps of the true point
group of the molecule\index{symmetry!group} which is a
"super group" of the Abelian group used in the calculation.
Degenerate orbitals will be averaged and the "super symmetry"
will be enforced in the orbitals.
Note that "super symmetry" can only be used
in the RHF, MP2, MCSCF, and RESPONS modules, and should
not be invoked if other modules are used,
for example, if \Sec{*PROPERTIES} (\aba) is invoked.
%hj aug 04: it should be OK for closed shell cases, also for CC ???
% it is only for spatially degenerate states that elements
% of the orbital gradient may be non-zero, right ???
Also, it cannot be used
in finite field calculations where the field lowers the symmetry.
The initial orbitals must be symmetry orbitals, and the super symmetry
analysis is performed on the kinetic energy matrix in this basis.
The \quotekw{.THRSSY} option is used to define when the kinetic
energy matrix element between two orbitals is considered to be
zero and when two diagonal matrix elements are degenerate. In the
first case the orbitals can belong to different irreps of the
supergroup and in the second case the two orbitals are considered
to be degenerate. The analysis will fail if there are accidental
degeneracies in diagonal elements.  This can happen if the nuclear
geometry deviates slightly from a higher symmetry point group, for
example because too few digits has been used in the input of the
nuclear geometry. If the program stops because the super symmetry
analysis fails with a degeneracy error, you might consider to use
more digits in the nuclear coordinates, to change \kw{THRSSY}, or
to disable super symmetry by not using \quotekw{.SUPSYM}.  The value of
\kw{THRSSY} should be sufficiently small to avoid accidental
degeneracies and sufficiently large to ignore small errors in
geometry and numerical round-off errors.


\Key{REORDER}\index{orbital reordering} can for instance be used for
linear molecules to interchange
undesired delta orbitals among the active orbitals in symmetry 1 with
sigma orbitals.  Another example is movement of the core orbital to the
RAS1 space for core hole calculation.  In general, use of this option
necessitates a pre-calculation with STOP AFTER MO-ORTHONORMALIZATION and
identification of the various orbitals by inspection of the output.


\pagebreak[3]
\subsection{\label{ref-popinp}\Sec{POPULATION ANALYSIS}}

{\bf Purpose:}

To direct population analysis\index{population analysis} of the wave function.
Requires a set of natural orbitals\index{natural orbital} and their occupation.

\begin{description}
\item[\Key{ALL}]
  Do all options.

%\item[\Key{DIPMOM}]
%  Calculate dipole moments. Note that this requires that the dipole
%  length integrals are available on the file \verb|AOONEINT|.\index{dipole moment}
%Aug 04: not working as far as I know /hjaaj

\item[\Key{GROSSALL}]
  Do all gross population analysis. Note that this requires that the dipole
  length integrals are available on the file \verb|AOPROPER|\index{population analysis}

\item[\Key{GROSSMO}]
  Do gross MO population analysis.\index{population analysis}

\item[\Key{MULLIKEN}]
  Do Mulliken population analysis\index{population analysis}\index{population analysis!Mulliken}\index{Mulliken population analysis}

\item[\Key{NETALL}]
  Do all net population analysis.\index{population analysis}

\item[\Key{NETMO}]
  Do net MO population analysis.\index{population analysis}

\item[\Key{PRINT}]
  Default = 1\\
  \kw{READ (LUINP,*) IPRMUL} \\
  Print level for population analysis.

%\item[\Key{QUADRP}]
%  Calculate quadrupole moments. Note that this requires that the quadrupole
%  integrals are available on the file \verb|AOONEINT|\index{quadrupole moment}
%Aug 04: not working as far as I know /hjaaj

\item[\Key{VIRIAL}]\index{virial analysis}
  Do virial analysis.
\end{description}

\pagebreak[3]
\subsection{\label{ref-priinp}\Sec{PRINT LEVELS}}

{\bf Purpose:}

To control the printing of output.

\begin{description}
\item[\Key{CANONI}]
  Generate canonical/natural orbitals if the wave function has
  converged\index{canonical orbital}\index{natural orbital}.

\item[\Key{IPRAVE}]
  \kw{READ (LUINP,*) IPRAVE} \\
  Sets print level for routines used in "super symmetry" averaging
  (default = 0).

\item[\Key{IPRCIX}]
  \kw{READ (LUINP,*) IPRCIX} \\
  Sets print level for setup of determinant/CSF index information (default = 0).

\item[\Key{IPRCNO}]
  \kw{READ (LUINP,*) IPRCNO} \\
  Sets print level for \quotekw{.TRACI} option (default = 1,
  to print the natural orbital occupations in each iteration set
  IPRCNO = 1, higher values will give more print).

\item[\Key{IPRDIA}]
  \kw{READ (LUINP,*) IPRDIA} \\
  Sets print level for calculation of CI diagonal (default = 0)

\item[\Key{IPRDNS}]
  \kw{READ (LUINP,*) IPRDNS} \\
  Sets print level for calculation of CI density matrices (default = 0)

%\item[\Key{IPRERR}]
%  \kw{READ (LUINP,*) IPRERR} \\
%  Sets print level for statistics in error file, LUERR (default = 1)

\item[\Key{IPRFCK}]
  \kw{READ (LUINP,*) IPRFCK} \\
  Sets print level in the supersymmetry section (default=0).

\item[\Key{IPRKAP}]
  \kw{READ (LUINP,*) IPRKAP} \\
  Sets print level in routines for calculation of optimal orbital trial
  vectors (default = 0)

\item[\Key{IPRSIG}]
  \kw{READ (LUINP,*) IPRSIG} \\
  Sets print level for calculation of CI sigma vectors (default = 0)

\item[\Key{IPRSOL}]
  \kw{READ (LUINP,*) IPRSOL} \\
  Sets print level in the solvent contribution parts of the
  calculation (default = 5).

\item[\Key{NOSUMMARY}]
  No final summary of calculation.

\item[\Key{POPPRI}]
  \kw{READ (LUINP,*) LIM\_POPPRI} \\
  Print Mulliken occupation of the first LIM\_POPPRI atoms in
  each SCF iteration. Useful for understanding convergence.
  (Default = 16, corresponding to two lines of output).

\item[\Key{PRINTFLAGS}]
 Default: flags set by general levels in \quotekw{\Key{PRINTLEVELS}}
\begin{verbatim}
  READ (LUINP,*) NUM6, NUM4
  IF (NUM6 .GT. 0) READ (LUINP,*) (NP6PTH(I), I=1,NUM6)
  IF (NUM4 .GT. 0) READ (LUINP,*) (NP4PTH(I), I=1,NUM4)
\end{verbatim}
  Individual print flag settings (debug option).

\item[\Key{PRINTLEVELS}]
  Default: IPRI6 = 0 and IPRI4 = 5 \\
  \kw{READ (LUINP,*) IPRI6,IPRI4 } \\
  Print levels on units LUW6 and LUW4, respectively.
%
%\item[\Key{PRINTUNITS}]
%  \kw{READ (LUINP,*) LUW6,LUW4 } \\
%  Unit numbers for general output and summary output, respectively
%  (default: LUW4 = 6 and LUW6 = 6).
%
\item[\Key{THRPWF}]
  \kw{READ (LUINP,*) THRPWF} \\
  Threshold for printout of wave function CI coefficients (default = 0.05).
 \end{description}



%\ifsolvent
\pagebreak[3]
\subsection{\label{ref-rhfinp}\Sec{SCF INPUT}}

{\bf Purpose:}

This section deals with the closed shell, one open shell and
high--spin spin-restricted 
Hartree--Fock cases\index{SCF}\index{HF}\index{Hartree--Fock}
and Kohn-Sham DFT\index{DFT}. 
The input here will usually only be used if either
\quotekw{\Key{DFT}} or \quotekw{\Key{HF}}
has been specified under \quotekw{\Sec{*WAVE FUNCTIONS}}
(though it is also needed for MP2 calculations based on saved closed-shell HF
orbitals).
High--spin spin-restricted open-shell Hartree--Fock or Kohn--Sham DFT calculations are activated by
using the \quotekw{.SINGLY OCCUPIED} described here.
Other single configuration cases with more than one open shell\index{open shell!SCF}
can be handled by the general \quotekw{\Key{MCSCF}} option, by appropriate specifications
in the \Sec{CONFIGURATION INPUT} section.

\begin{description}
\item[\Key{AUTOCCUPATION}]
  Default for SCF calculations starting from extended H\"{u}ckel, EWMO, or H1DIAG
  starting orbitals.

  Allow the distribution of the Hartree--Fock/DFT occupation numbers over
  symmetries\index{Hartree--Fock occupation}\index{HF occupation} to
  change based on changes in orbital ordering during DIIS\index{DIIS} optimization.
  This keyword is incompatible with \quotekw{.SINGLY OCCUPIED} and \quotekw{.COREHOLE}, or
  if the HF calculation is followed by CI or MCSCF.

%\item[\Key{EDIIS}]
%  Use a from Kudin {\it et al.} slightly modified (E)DIIS-scheme.
%  Keys associated with DIIS are also valid for EDIIS (e.g. MXDIIS etc.)

\item[\Key{C2DIIS}]
  Use Harrell Sellers' C2-DIIS algorithm instead of Pulay's C1-DIIS algorithm
  (see comments).

\item[\Key{COREHOLE}]
  \kw{READ (LUINP,*) JCHSYM,JCHORB} \\
  JCHSYM = symmetry of core orbital\\
  JCHORB = the orbital in symmetry JCHSYM with a single core hole\\
  Single core hole\index{core hole} open shell RHF calculation, \quotekw{\Key{OPEN
  SHELL}} must not
  be specified.  The specified core orbital must be
  inactive\index{inactive orbital}.
  The number of doubly occupied orbitals in symmetry \kw{JCHSYM} will be reduced with one
  and instead an open shell orbital will be added for the core hole orbital.
  If the specified core orbital is not the last occupied orbital in symmetry
  \kw{JCHSYM} it will switch places with that orbital and user-defined reordering
  is not possible.
  If explicit reordering is required you must also reorder
  the core orbital yourself and let \kw{JCHORB} point to the last occupied orbital
  of symmetry \kw{JCHSYM}.  See comments below.

\item[\Key{CORERELAX}]
  (ignored if \quotekw{\Key{COREHOLE}} isn't also specified)\\
  Optimize core hole\index{core hole} state with relaxed
  core\index{relaxed core} orbital using Newton-Raphson algorithm.
  It is assumed that this calculation follows an optimization
  with frozen core orbital and the specific value of
  \quotekw{JCHORB} under \quotekw{\Key{COREHOLE}} is ignored (no
  reordering will take place).

%\item[\Key{DIRFOCK}]
%  Direct Fock matrix constructions (recalculate integrals when needed).
%  Default: AO integrals or P-supermatrix integrals read from disk.
%\fi

\item[\Key{DOUBLY OCCUPIED}]
    \kw{READ (LUINP,*) (NRHF(I),I=1,NSYM)} \\
  \index{HF}\index{SCF}\index{Hartree--Fock}\index{MP2}\index{M{\o}ller-Plesset!second-order}
  Explicit specification of number of doubly occupied orbitals in each symmetry
  for DFT, RHF and MP2 calculations. This keyword
  is required when Hartree--Fock or MP2 is part of a multistep
  calculation which includes an MCSCF wave function. 
  Otherwise the program by default will try to guess the occupation,
  corresponding to the  \quotekw{.AUTOCC} keyword.

\item[\Key{ELECTRONS}]
  \kw{READ (LUINP,*) NRHFEL} \\
  Number of electrons in the molecule\index{electrons in molecule}.
  By default, this number will be determined on the basis of the nuclear
  charges and the total charge of the molecule\index{charge of molecule}
  as specified in the \molinp\ file.
  The keyword is incompatible with the keywords \quotekw{.DOUBLY OCCUPIED},
  \quotekw{.OPEN SHELL}, and \quotekw{.SINGLY OCCUPIED}.

\item[\Key{FC MVO}]
  \kw{READ (LUINP,*) (NMVO(I), I = 1,NSYM)} \\
  Modified virtual orbitals using Bauschlichers suggestion
  (see Ref.~\cite{cwbjcp72})
  for CI or for start guess for MCSCF. The modified virtual orbitals
  are obtained by  diagonalizing the virtual-virtual
  block of the Fock matrix constructed from NMVO(1:NSYM) doubly
  occupied orbitals.
  The occupied SCF orbitals (i.e those specified with
  \quotekw{.DOUBLY OCCUPIED} and \quotekw{.OPEN SHELL}
  or by automatic occupation) are not modified.
  The construction of modified virtual orbitals
  will follow any SCF and MP2 calculations.
  See comments below.

\item[\Key{FOCK ITERATIONS}]
  \kw{READ (LUINP,*) MAXFCK} \\
  Maximum number of closed-shell Roothaan\index{Roothaan iteration}
  Fock iterations (default = 0).

\item[\Key{FROZEN CORE ORBITALS}]
  \kw{READ (LUINP,*) (NFRRHF(I),I=1,NSYM)} \\
  Frozen orbitals per symmetry (if MP2 follows then at least these orbitals
  must be frozen in the MP2 calculation).
  NOTE: no Roothaan Fock iterations allowed if frozen orbitals.

\item[\Key{H1VIRT}] Use the virtual orbitals that diagonalize the
  one-electron Hamiltonian operator.

\item[\Key{MAX DIIS ITERATIONS}]
  \kw{READ (LUINP,*) MXDIIS} \\
  Maximum number of DIIS iterations\index{iteration number!DIIS, max}\index{DIIS!max iterations} (default = 60).

\item[\Key{MAX ERROR VECTORS}]
  \kw{READ (LUINP,*) MXEVC} \\
  Maximum number of DIIS error vectors\index{DIIS!error vectors, max}
  (default = 10, if there is sufficient memory available to hold these
  vectors in memory).

\item[\Key{MAX MACRO ITERATIONS}]
  \kw{READ (LUINP,*) MXHFMA} \\
  Maximum number of QCSCF macro\index{iteration number!QCSCF macro, max}
 iterations (default = 15).


\item[\Key{MAX MICRO ITERATIONS}]
  \kw{READ (LUINP,*) MXHFMI} \\
  Maximum number of QCSCF\index{SCF!quadratic convergent} micro iterations per macro iteration (default = 12).

\item[\Key{NODIIS}]
  Do not use DIIS algorithms\index{DIIS} (default: use DIIS algorithm).

\item[\Key{NONCANONICAL}]
  No transformation to canonical orbitals\index{canonical orbital}.

\item[\Key{NOQCSCF}]
  No quadratically convergent SCF\index{SCF!no quadratically convergent} iterations.
  Default is to switch to QCSCF if DIIS doesn't converge.

\item[\Key{OPEN SHELL}]
  Default = no open shell\\
  \kw{READ (LUINP,*) IOPRHF} \\
  Symmetry of the open shell in a one open shell\index{open shell!HF}\index{HF!open shell}
  calculation. See also \quotekw{.SINGLY OCCUPIED} for high-spin ROHF with more than one
  singly occupied orbital.

\item[\Key{PRINT}]
  \kw{READ (LUINP,*) IPRRHF} \\
  Resets general print level to \verb|IPRRHF| in Hartree--Fock/DFT calculations
  (if not specified, global print levels will be used).

\item[\Key{SHIFT}] 
  \kw{READ (LUINP,*) SHFTLVL} \\
  Initial value of level-shift parameter in DIIS iterations. 
  The default value is 0.0D0 (no level shift).
  May be tried if convergence problems in DIIS. The value is added
  to the diagonal of the occupied part of the Fock matrix before
  Roothaan diagonalization, reducing the mixing of occupied and
  virtual orbitals (step restriction).
  NOTE that the value should thus be negative.  The DIIS routines
  will automatically invoke level-shifting (step restriction) if
  DIIS seems to be stalling.

\item[\Key{SINGLY OCCUPIED}] Default = no singly occupied orbitals \\
    \kw{READ (LUINP,*) (NROHF(I),I=1,NSYM)} \\
  High--spin spin-restricted open-shell Hartree--Fock or Kohn-Sham DFT.
 \index{HSROHF}\index{HSRODFT}\index{ROHF!high spin}\index{RODFT!high spin}\index{ROKS!high spin}
  Specify the number of singly occupied orbitals in each irreducible representation
  of the molecular point group. Only the high-spin state of these
  singly-occupied orbitals will be made and used in the calculations.
  We recommend to always run high-spin open-shell geometry optimizations as direct calculations
  (\quotekw{\Key{DIRECT}} under \quotekw{\Sec{DALTON}}),
  because analytical molecular gradients are only implemented for direct calculations
  (numerical gradients will be used for non-direct calculations).

\item[\Key{THRESH}]
  Default = 1.0D-05 (1.0D-06 if MP2)\\
  \kw{READ (LUINP,*) THRRHF} \\
  Hartree--Fock/DFT convergence threshold for energy gradient.  The convergence
  of the energy will be approximately the square of this number.

\end{description}


\noindent{\bf Comments:}

By default, the RHF/DFT part of a calculation will consist of :
\begin{enumerate}
\item {MAXFCK Roothaan Fock iterations (early exit if convergence
    or oscillations). However, the default is that no Roothaan Fock
iterations are done unless explicitly requested through the keyword
\quotekw{.FOCK I}.
}
\item {MXDIIS DIIS iterations (exit if convergence, {\it i.e.\/} gradient norm
    less than THRRHF, and if convergence rate too slow or even diverging).
}
\item {Unless NOQCSCF, quadratically convergent Hartree--Fock/DFT until
    gradient norm less than THRRHF.
}
\item{If \quotekw{.FC MVO} has been specified
    then the virtual SCF orbitals will be modified by diagonalizing
    the virtual-virtual block of
    a modified Fock matrix: the Fock matrix
    based on the occupied orbitals specified after the keyword, a
    good choice is the inactive (doubly occupied) orbitals in the
    following CI or MCSCF.
    The occupied SCF orbitals will not be modified.
    If the RHF calculation is followed by a CI or an MCSCF calculation,
    \quotekw{.FC MVO} will usually provide much
    better start orbitals than the canonical orbitals (canonical
    orbitals will usually put diffuse, non-correlating orbitals in the
    active space). \\
    WARNING: if both \quotekw{.MP2} and \quotekw{.FC MVO} are specified,
    then the MP2 orbitals will be destroyed and replaced with \quotekw{.FC MVO}
    orbitals.
}
\end{enumerate}

In general \quotekw{.DOUBLY OCCUPIED} should be specified for CI or MCSCF
\index{HF occupation}\index{Hartree--Fock occupation}\index{CI}\index{MCSCF}
\index{Configuration Interaction}
wave function calculations -- you anyway need to know the distribution
of orbitals over symmetries to specify the \quotekw{*CI INPUT} input.
For RHF\index{RHF}\index{SCF}\index{Hartree--Fock}
or MP2\index{MP2}\index{M{\o}ller-Plesset!second-order}
calculations the orbital occupation will be determined on the
basis of the nuclear charges and molecular charge of the molecule as
specified in the \molinp\ file.

By default, starting orbitals and initial orbital occupation will
be determined automatically on the basis of a H\"{u}ckel\index{H\"{u}ckel}
calculation (for molecules where all nuclear charges are
less than or equal to 36), corresponding to the \quotekw{.AUTOCC} keyword.
\index{starting orbitals!SCF}\index{H\"{u}ckel!starting orbitals}.
If problems is experienced due to the
H\"{u}ckel starting guess, it can be avoided by requiring another set of
starting orbitals ({\it e.g.\/} \verb|H1DIAG|).

%The default convergence threshold is quite sharp (compare with the
%default for MCSCF), this is done in order to have good orbitals
%for MP2 calculations.  For Hartree--Fock
%calculations with many basis functions
%which are not to be followed by MP2 or used for finite difference
%property calculations, some CPU time may be save by lowering the
%threshold to the minimum acceptable accuracy.

It is our experience that
it is usually most efficient not to perform any Roothaan Fock iterations
before DIIS is activated, therefore, MAXFCK = 0 as default.
The algorithm described in
Harrell Sellers, Int. J. Quant. Chem. {\bf 45}, 31-41 (1993) is
also implemented, and may be selected with \quotekw{\Key{C2DIIS}}.


FC MVO: This option can be used without a Hartree--Fock calculation
to obtain compact virtual orbitals, but \quotekw{.DOUBLY OCCUPIED} must be
specified anyway in order to identify the virtual orbitals to be transformed.

COREHOLE: Enable SCF
single core-hole\index{core hole} calculations. To perform
an SCF core hole calculation just add the \quotekw{\Key{COREHOLE}}
keyword to the input for the closed-shell RHF ground state
calculation, specifying from which orbital to remove an electron,
and provide the program with the ground state orbitals using the
appropriate \quotekw{\Key{MOSTART}} option (normally \kw{NEWORB}).
Note that this is different from the MCSCF version of
\quotekw{\Key{COREHOLE}} under \quotekw{\Sec{OPTIMIZATION}}
(p.~\pageref{ref-optinp}); in the MCSCF case the user must
explicitly move the core hole orbital from the inactive class to
RAS1 by modifying the \quotekw{\Sec{CONFIGURATION INPUT}}
(p.~\pageref{ref-wavinp}) specifications between the initial
calculation with filled core orbitals and the core hole
calculation. The core hole\index{core hole} orbital will be
frozen\index{frozen core hole} in the following optimization.
After this calculation has converged, the CORERELAX option may be
added and the core orbital will be relaxed\index{relaxed core hole}.  
When CORERELAX is specified it is assumed that the
calculation was preceded by a frozen core calculation, and that
the orbital has already been moved to the open shell orbital. Only
the main peak can be obtained in SCF calculations, for shake-up
energies MCSCF must be used.

\pagebreak[3]
\subsection{\label{ref-solinp}\Sec{SOLVENT}}

{\bf Purpose:}

Model solvent effects with the self-consistent
reaction\index{reaction field} field model.
Any specification of dielectric constant(s)\index{dielectric constant}
will activate this model.

\begin{description}
\item[\Key{CAVITY}]
  Required, no defaults.\\
  \kw{READ (LUINP,*) RSOLAV}\\
  Enter radius of spherical cavity\index{cavity!radius} in atomic units (\bohr{}).

\item[\Key{DIELECTRIC CONSTANT}]
  \kw{READ (LUINP,*) EPSOL}\\
  Enter relevant dielectric constant\index{dielectric constant} of solvent.

\item[\Key{INERSINITIAL}]
  \kw{READ (LUINP,*) EPSOL, EPPN}\\
  Enter static and optical dielectric constant\index{dielectric constant} of solvent for calculation
  of the initial state defining inertial polarization\index{inertial
  polarization}. Note that the optical dielectric constant specified here
  only will be used in case there is a calculation of response
  properties, for which this is an alternative input to the use of the
  keyword \Key{INERSFINAL}.

\item[\Key{INERSFINAL}]
  \kw{READ (LUINP,*) EPSTAT,EPSOL}\\
  Enter static and optical dielectric\index{dielectric constant} constants of solvent
  for calculation of the final state with inertial polarization
  from a previous calculation with \quotekw{\Key{INERSINITIAL}}\index{final polarization}. 
  This can for example be used to optimize an excited electronic state
  with inertial polarization from a previous ground state calculation.
  This keyword an  also be used to specify the static and optical dielectric constants
  for non-equilibrium solvation linear, quadratic, or cubic response functions,
  see also Sec.~\ref{sec:solvnoneqrsp}, but this is usually easier done with
  \quotekw{.INERSINITIAL} (requires only one \dalton\ calculation instead of two).

\item[\Key{MAX L}]
  Required, no defaults.\\
  \kw{READ (LUINP,*) LSOLMX}\\
  Enter maximum L quantum number in multipole expansion of charge
  distribution in cavity.

\item[\Key{PRINT}]
  \kw{READ (LUINP,*) IPRSOL} \\
  Print level in solvent module routines (default = 0).
\end{description}

\noindent{\bf Comments:}

One and only one of \quotekw{\Key{DIELECTRIC CONSTANT}},
\quotekw{\Key{INERSINITIAL}}, and \quotekw{\Key{INERSFINAL}} must be
specified.
%\fi % end of \ifsolvent



\pagebreak[3]
\subsection{\label{ref-stpinp}\Sec{STEP CONTROL}}

{\bf Purpose:}

User control of the NEO restricted step optimization.

\begin{description}
\item[\Key{DAMPING FACTOR}]
  Default = 1.0D0\\
  \kw{READ (LUINP,*) BETA} \\
  Initial value of damping (BETA).\index{damping}

\item[\Key{DECREMENT FACTOR}]
  Default = 0.67D0\\
  \kw{READ (LUINP,*) STPRED} \\
  Decrement factor on trust radius\index{trust radius}

\item[\Key{GOOD RATIO}]
  Default = 0.8D0 \\
  \kw{READ (LUINP,*) RATGOD} \\
  Threshold ratio for good second order agreement: the trust radius can
  be increased if ratio is better than RATGOD.

\item[\Key{INCREMENT FACTOR}]
  Default = 1.2D0\\
  \kw{READ (LUINP,*) STPINC} \\
  Increment factor on trust radius.\index{trust radius}

\item[\Key{MAX DAMPING}]
  Default = 1.0D6\\
  \kw{READ (LUINP,*) BETMAX} \\
  Maximum damping value.\index{damping}

\item[\Key{MAX STEP LENGTH}]
  Default = 0.7D0\\
  \kw{READ (LUINP,*) STPMAX} \\
  Maximum acceptable step length, trust radius will never be larger than
  STPMAX even if the ratio is good as defined by GOOD RATIO.

\item[\Key{MIN DAMPING}]
  Default = 0.2D0\\
  \kw{READ (LUINP,*) BETMIN} \\
  Minimum damping value

\item[\Key{MIN RATIO}]
  Default = 0.4D0 for ground state, 0.6 for excited states\\
  \kw{READ (LUINP,*) RATMIN} \\
  Threshold ratio for bad second order agreement: the trust radius is
  to be decreased if ratio is worse than RATMIN.

\item[\Key{NO EXTRA TERMINATION TESTS}]
  Skip extra termination tests and converge micro iterations to
  threshold.   Normally the micro iterations are terminated if the
  reduced NEO matrix has more negative eigenvalues than corresponding
  to the desired state, because then we are in a "superglobal" region
  and we just want to step as quickly as possible to the region where
  the Hessian (and NEO matrix) has the correct structure.  Further
  convergence is usually wasted.

\item[\Key{REJECT THRESHOLD}]
  Default = 0.25 for ground state, 0.4 for excited states\\
  \kw{READ (LUINP,*) RATREJ} \\
  Threshold ratio for unacceptable second order agreement: the step
  must be rejected if ratio is worse than RATREJ.

\item[\Key{THQKVA}]
  Default: 8.0 for MCSCF; 0.8 for QCSCF\\
  \kw{READ (LUINP,*) THQKVA} \\
  Convergence factor for micro iterations in local (quadratic) region:
  THQKVA*(norm of gradient)**2

\item[\Key{THQLIN}]
  Default: 0.2D0\\
  \kw{READ (LUINP,*) THQLIN} \\
  Convergence factor for micro iterations in global (linear) region: \\
  THQLIN*(norm of gradient)

\item[\Key{THQMIN}]
  Default: 0.1D0\\
  \kw{READ (LUINP,*) THQMIN} \\
  Convergence threshold for auxiliary roots in NEO MCSCF optimization.

\item[\Key{TIGHT STEP CONTROL}]
  Tight step control also for ground state calculations
  (tight step control is always enforced for excited states)

\item[\Key{TOLERANCE}]
  Default = 1.1D0\\
  \kw{READ (LUINP,*) RTTOL} \\
  Acceptable tolerance in deviation of actual step from trust radius
  (the default value of 1.1 corresponds to a maximum of 10\% deviation).

\item[\Key{TRUST RADIUS}]
  Default = STPMAX=0.7D0 or, if restart, trust radius determined by previous
            iteration.\index{trust radius}\\
  \kw{READ (LUINP,*) RTRUST} \\
  Initial trust radius.

\end{description}


\pagebreak[3]
\subsection{\label{ref-trainp}\Sec{TRANSFORMATION}}

{\bf Purpose:}

Transformation\index{integral transformation} of two-electron
integrals\index{two-electron integral} to molecular orbital
basis\index{molecular orbital}.

\begin{description}
\item[\Key{FINAL LEVEL}]
  \kw{READ (LUINP,*) ITRFIN} \\
  Final integral transformation\index{integral transformation} level (only active if the keyword
  \quotekw{\Key{INTERFACE}} has been specified, or this is an \aba\ or
  \resp\ calculation.

\item[\Key{LEVEL}]
  \kw{READ (LUINP,*) ITRLVL} \\
  Integral transformation level (see comments).

\item[\Key{OLD TRANSFORMATION}]
  Use existing transformed integrals

\item[\Key{PRINT}]
  \kw{READ (LUINP,*) IPRTRA} \\
  Print level in integral transformation module

\item[\Key{RESIDENT MEMORY}]
  \kw{READ (LUINP,*) MWORK} \\
  On virtual memory computers, the transformation will run more
  efficiently if it can be kept within the possible resident memory
  size: the real memory size.  {\sir} will attempt to only use MWORK
  double precision words in the transformation.
\end{description}


\noindent{\bf Comments:}

There are several types of integral transformations which may be
specified by the two transformation level keywords.
\begin{itemize}
   \item[0:] CI calculations, MCSCF gradient (default if CI, but
             no MCSCF specified).
             One index all orbitals, three indices only active
             orbitals.

   \item[1:] Obsolete, do not use.

   \item[2:] Default for MCSCF optimization. All integrals needed for {\sir}
             second-order MCSCF optimization, including the integrals
             needed to explicitly construct the diagonal of the orbital
             Hessian. Two indices occupied orbitals, two indices all
             orbitals, with some reduction for inactive indices.
             Both (cd/ab) and (ab/cd) are stored.

   \item[3:] Same integrals as 2, including also the (ii/aa) and
             (ia/ia) integrals for exact inactive-secondary diagonal elements
             of the orbitals Hessian.

   \item[4:] All integrals with minimum two occupied indices.

   \item[5:] 3 general indices, one occupied index.  Required for MP2
             natural orbital analysis (the MP2 module automatically
             performs an integral transformation of this level).

  \item[10:] Full transformation.
\end{itemize}


\pagebreak[3]
\subsection{\label{ref-cube}\Sec{CUBE}}

{\bf Purpose:}

Generates cube file\index{cube file} of total SCF electron density and/or
molecular orbitals after SCF calculations. The keyword \quotekw{\Key{INTERFACE}}
must be specified.

\begin{description}
\item[\Key{DENSITY}]
  Generates cube file ``\kw{density.cube}'' with total SCF electron density.

\item[\Key{HOMO}]
  Generates cube file ``\kw{homo.cube}'' with the information of the highest
occupied molecular orbitals.

\item[\Key{LUMO}]
  Generates cube file ``\kw{lumo.cube}'' with the information of the lowest
unoccupied molecular orbitals.

\item[\Key{MO}]
  \kw{READ (LUINP,*) IDX\_MO} \\
  Generates cube file ``\kw{mo.cube}'' with specified indices of molecular orbitals
by ``\kw{IDX\_MO}''. For instance, valid format is like ``1-6,7,10-12'' only including
digits, minus sign and comma.

\item[\Key{FORMAT}]
  \kw{READ (LUINP,*) CUBE\_FORMAT} \\
  Specifies cube file format, only ``\kw{GAUSSIAN}'' (Gaussian cube file format,
see\linebreak \verb|http://www.gaussian.com/g_tech/g_ur/u_cubegen.htm|) for the
time being.

\item[\Key{ORIGIN}]
  \kw{READ (LUINP,*) CUBE\_ORIGIN} \\
  Reads the coordinates (a.u.) of origin/initial point.

\item[\Key{INCREMENT}]
  \kw{READ (LUINP,*) N1, X1, Y1, Z1} \\
  \kw{READ (LUINP,*) N2, X2, Y2, Z2} \\
  \kw{READ (LUINP,*) N3, X3, Y3, Z3} \\
  Reads the number of increments and increments (a.u.) along three running directions,
in which ``\kw{(X1,Y1,Z1)}'' is the slowest running direction, and ``\kw{(X3,Y3,Z3)}''
is the fastest running direction.

As described at \verb|http://www.gaussian.com/g_tech/g_ur/u_cubegen.htm|, if the
origin/initial point is (X0,Y0,Z0), then the point at (I1,I2,I3) has coordinates:

X-coordinate: X0+(I1-1)*X1+(I2-1)*X2+(I3-1)*X3\\
Y-coordinate: Y0+(I1-1)*Y1+(I2-1)*Y2+(I3-1)*Y3\\
Z-coordinate: Z0+(I1-1)*Z1+(I2-1)*Z2+(I3-1)*Z3
\end{description}


\pagebreak[3]
\section{\label{sec:ref-molorbinp} \Sec{*MOLORB} input module}

If formatted input of the molecular orbitals has been specified in
the \Sec{ORBITAL INPUT} section, then {\sir} will attempt to find
the two-star label "\verb|**MOLORB|" in the input file and read
the orbital coefficients from the lines following this label.
