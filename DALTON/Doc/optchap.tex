\chapter{Calculation of optical and Raman properties}\label{ch:optchap}

This chapter describes the calculation of different optical properties
which have been implemented in the \dalton\ program. This
includes electronic excitation energies and corresponding oscillator
strengths as well as properties related to different kinds of circular
dichroism, more specifically vibrational circular dichroism
(VCD)\index{VCD}\index{vibrational circular dichroism} as described in
Ref.~\cite{klbpjthkrhjajjcp98}, electronic circular dichroism
(ECD)\index{ECD}\index{electronic circular dichroism} as described in
Refs.~\cite{klbaehkrthjopjtca90,tbpaehcpl246}, Raman Optical Activity
(ROA)\index{ROA}\index{Raman optical activity} as described in
Ref.~\cite{thkrklbpjjofd99}, and optical
rotation~\cite{plpmp91,plpdkckrcpl319}.

By default, all calculations of optical properties are done with the
use of London atomic orbitals,\index{London orbitals} if possible for
the chosen wave function, in order to enhance the basis set convergence
as well as to give the correct physical dependence on the gauge
origin\index{gauge origin}.


\section{Electronic excitation energies and oscillator strengths}\label{sec:uvvis}

\begin{center}
\fbox{
\parbox[h][\height][l]{12cm}{
\small \noindent {\bf Reference literature:}
\begin{list}{}{}
\item MCSCF: K.L.Bak, Aa.E.Hansen, K.Ruud, T.Helgaker, J.Olsen, and
P.J{\o}rgensen. \newblock {\em Theor. Chim. Acta.}, {\bf
90},\hspace{0.25em}441, (1995).
\item SOPPA: M.~J.~Packer, E.~K.~Dalskov, T.~Enevoldsen,
H.~J.~Aa.~Jensen and J.~Oddershede,
\newblock {\em J. Chem. Phys.}, {\bf 105}, \hspace{0.25em}5886, (1996).
\item AO direct SOPPA:
K.~L.~Bak, H.~Koch, J.~Oddershede, O.~Christiansen and S.~P.~A.~Sauer,
\newblock {\em J. Chem. Phys.}, {\bf 112}, \hspace{0.25em} 4173,
(2000).
\item RPA(D):
O.~Christiansen, K.~L.~Bak, H.~Koch and S.~P.~A.~Sauer,  Chem. Phys.
\newblock {\em Chem. Phys.}, {\bf 284}, \hspace{0.25em} 47, (1998).
\item SOPPA(CCSD):
H.~H.~Falden, K.~R.~Falster-Hansen, K.~L.~Bak, S.~Rettrup and
S.~P.~A.~Sauer,
\newblock {\em J. Phys. Chem. A}, {\bf 113}, \hspace{0.25em} 11995,
(2009).
\end{list}
}}
\end{center}

The calculation of electronic singlet and triplet excitation energies
\index{electronic excitation} is invoked by the keyword \Key{EXCITA} in
the \Sec{*PROPERTIES} input module. However, it is also necessary to
specify the number of electronic excitations\index{electronic
excitation} in each symmetry with the keyword \Key{NEXCIT} in the
\Sec{EXCITA} section. The corresponding dipole oscillator
strengths\index{transition moment} can conveniently be calculated at
the same time by adding the keyword \Key{DIPSTR} in the \Sec{EXCITA}
section.

A typical input for the calculation of electronic singlet excitation
energies and corresponding dipole oscillator strengths for a molecule
with C$_{2v}$ symmetry would look like:

\begin{verbatim}
**DALTON INPUT
.RUN PROPERTIES
**WAVE FUNCTIONS
.HF
**PROPERTIES
.EXCITA
*EXCITA
.DIPSTR
.NEXCIT
    3    2    1    0
**END OF DALTON INPUT
\end{verbatim}
This input will calculate the oscillator strength (\Key{DIPSTR}) of the
6 lowest electronic excitations distributed in a total of 4 irreducible
representations (as in C$_{2v}$). The oscillator strength will be
calculated both in length and velocity forms.

A typical input for the calculation of electronic triplet excitation
energies would look like:
\begin{verbatim}
**DALTON INPUT
.RUN PROPERTIES
**WAVE FUNCTIONS
.HF
**PROPERTIES
.EXCITA
*EXCITA
.TRIPLE
.NEXCIT
    3    2    1    0
**END OF DALTON INPUT
\end{verbatim}



For a SOPPA\index{SOPPA}, SOPPA(CC2)\index{SOPPA(CC2)} or
SOPPA(CCSD)\index{SOPPA(CCSD)} calculation of electronic excitation
energies and corresponding oscillator strengths the additional keywords
\Key{SOPPA} or \Key{SOPPA(CCSD)} have to be specified in the
\Sec{*PROPERTIES} input module. For SOPPA an MP2 calculation has to be
requested by the keyword \Key{MP2} in the \Sec{*WAVE FUNCTIONS} input
module, whereas for SOPPA(CC2) or SOPPA(CCSD) a CC2 or CCSD calculation
has to be requested by the keyword \Key{CC} in the \Sec{*WAVE
FUNCTIONS} input module with the \Sec{CC INPUT} option \Key{SOPPA2} or
\Key{SOPPA(CCSD)}. Details on how to invoke an atomic integral direct
RPA(D), SOPPA or SOPPA(CCSD) calculation of the oscillator strengths
are given in Chapters \ref{sec:AOsoppa} and {sec:soppa}.


We also note that excitation energies also can be obtained using the
\resp\ program (see Chapter~\ref{ch:rspchap}).

For a more detailed control of the individual parts of the
calculation of properties related to electronic excitation energies,
we refer to the input modules affecting the different parts of such
calculations:

\begin{description}
\item[\Sec{EXCITA}] Controls the calculation of electronic excitation
energies and the evaluation of all terms contributing to for instance
dipole strength.

\item[\Sec{GETSGY}] Controls the setup of the necessary right-hand
sides.

\item[\Sec{SOPPA}] Controls the details of a SOPPA calculation.

\end{description}



\section{Vibrational Circular Dichroism calculations}

\begin{center}
\fbox{
\parbox[h][\height][l]{12cm}{
\small
\noindent
{\bf Reference literature:}
\begin{list}{}{}
\item K.L.Bak, P.J{\o}rgensen, T.Helgaker, K.Ruud, and H.J.Aa.Jensen. \newblock {\em J.Chem.Phys.}, {\bf 98},\hspace{0.25em}8873, (1993).
\item K.L.Bak, P.J{\o}rgensen, T.Helgaker, K.Ruud, and H.J.Aa.Jensen. \newblock {\em J.Chem.Phys.}, {\bf 100},\hspace{0.25em}6620, (1994).
\end{list}
}}
\end{center}

The calculation of vibrational circular
dichroism\index{VCD}\index{vibrational circular dichroism} is invoked
by the
keyword \Key{VCD} in the \Sec{*PROPERTIES} input module. Thus a complete
input file for the calculation of vibrational circular dichroism will
look like:

\begin{verbatim}
**DALTON INPUT
.RUN PROPERTIES
**WAVE FUNCTIONS
.HF
**PROPERTIES
.VCD
**END OF DALTON INPUT
\end{verbatim}

This will invoke the calculation of vibrational circular dichroism
using London atomic orbitals\index{London orbitals} to ensure fast
basis set convergence as
well as gauge origin\index{gauge origin} independent results. By
default the natural
connection\index{natural connection} is used in order to get
numerically accurate
results~\cite{joklbkrthpjtca90,krthjopjklbcpl235}.

We notice, however, that vibrational circular dichroism only arises in
vibrationally chiral molecules. An easy way of introducing
vibrational chirality into small molecular systems is by isotopic
substitution. This is in
\dalton\ controlled in the \Sec{VIBANA} submodule, and the reader is
referred to that section for an exemplification of how this is done.

There has only  been a few investigation of basis set requirement for
the calculation of VCD given in
Ref.~\cite{klbpjthkrhjajjcp100,klbpjthkrfd99},
and the reader is referred to these references when choosing basis set
for the calculations of VCD.

In the current implementation, the \Key{NOCMC} option is automatically
turned on in VCD calculations, that is, the coordinate system origin
is always used as gauge origin\index{gauge origin}.

We note that if a different force field is wanted in the calculation
of the VCD parameters, this can be obtained by reading in an
alternative Hessian\index{Hessian} matrix with the input

\begin{verbatim}
**DALTON INPUT
.RUN PROPERTIES
**WAVE FUNCTIONS
.HF
**PROPERTIES
.VCD
*VIBANA
.HESFIL
**END OF DALTON INPUT
\end{verbatim}

We note that in the current Dalton release, Vibrational Circular
Dichroism can not be calculated using density functional theory, and if
requested, the program will stop.

If more close control of the different parts of the calculation of
vibrational circular dichroism is wanted, we refer the reader to the
sections describing the options available. The input sections that control
the calculation of vibrational circular dichroism are:

\begin{description}
\item[\Sec{AAT}] Controls the final calculation of the different
contributions to the Atomic Axial Tensors.
\item[\Sec{GETSGY}] Controls the set up of both the magnetic and
geometric right hand sides (gradient terms).
\item[\Sec{LINRES}] Controls the solution of the magnetic response
equations.
\item[\Sec{RELAX}] Controls the multiplication of solution and right hand
side vectors into relaxation contributions.
\item[\Sec{NUCREP}] Controls the calculation of the nuclear
contribution to the geometric Hessian.
\item[\Sec{TROINV}] Controls the use of translation and rotational invariance.
\item[\Sec{ONEINT}] Controls the calculation of one-electron
contributions to the geometric Hessian.
\item[\Sec{TWOEXP}] Controls the calculation of two-electron
expectation values to the geometric Hessian.
\item[\Sec{REORT}] Controls the calculation of reorthonormalization
terms to the geometric Hessian.
\item[\Sec{RESPON}] Controls the solution of the geometric response equations.
\item[\Sec{GEOANA}] Describes what analysis of the molecular geometry
is to be printed.
\item[\Sec{VIBANA}] Sets up the vibrational and rotational analysis of the
molecule, for instance its isotopic substitution.
\item[\Sec{DIPCTL}] Controls the calculation of the Atomic Polar
Tensors (dipole gradient).
\end{description}

\section{Electronic circular dichroism (ECD)}\label{sec:ecd}

\begin{center}
\fbox{
\parbox[h][\height][l]{12cm}{
\small
\noindent
{\bf Reference literature:}
\begin{list}{}{}
\item MCSCF: K.L.Bak, Aa.E.Hansen, K.Ruud, T.Helgaker, J.Olsen, and
P.J{\o}rgensen. \newblock {\em Theor. Chim. Acta.}, {\bf 90},\hspace{0.25em}441, (1995).
\item  DFT: M.Pecul, K.Ruud, and T.Helgaker.
\newblock {\em Chem. Phys. Lett.}, {\bf 388},\hspace{0.25em}110, (2004).
\end{list}
}}
\end{center}

The calculation of Electronic Circular Dichroism
(ECD)\index{ECD}\index{electronic circular dichroism} is invoked by the
keyword \Key{ECD} in the \Sec{*PROPERTIES} input module. However, it is
also necessary to specify the number of electronic
excitations\index{electronic excitation} in each symmetry. As ECD only
is observed for chiral molecules, such calculations will in general not
employ any symmetry (although the implementation does make use of it,
if present), a complete input for a molecule without symmetry will thus
look like:

\begin{verbatim}
**DALTON INPUT
.RUN PROPERTIES
**WAVE FUNCTIONS
.HF
**PROPERTIES
.ECD
*EXCITA
.NEXCIT
    3
**END OF DALTON INPUT
\end{verbatim}

In this run we will calculate the rotatory strength
corresponding to
the three lowest electronic excitations\index{electronic excitation}
(the \Key{NEXCIT} keyword)
using London atomic orbitals\index{London orbitals}.
If rotatory strengths obtained without London atomic orbitals is also
wanted, this is easily accomplished by adding the keyword
\Key{ROTVEL} in the \Sec{EXCITA} input module.

The rotatory strength tensors~\cite{tbpaehcpl246}
that govern the ECD of oriented samples
(OECD)\index{ECD!oriented}\index{electronic circular dichroism!oriented}\index{OECD}\index{oriented electronic circular dichroism} may additionally be
calculated by specifying the \Key{OECD} keyword. As an example,
the input

\begin{verbatim}
**DALTON INPUT
.RUN PROPERTIES
**WAVE FUNCTIONS
.HF
**PROPERTIES
.OECD
*EXCITA
.NEXCIT
    3
**END OF DALTON INPUT
\end{verbatim}
requests calculation of OECD as well as ECD. Note that London orbitals
are not implemented for OECD and the rotatory strength tensors are
calculated using both the length and velocity forms (the latter being
origin invariant). Since the rotatory strength tensor is composed of
electric quadrupole and magnetic dipole
contributions~\cite{tbpaehcpl246}, these parts must be computed at the
same origin. Therefore, the \Key{NOCMC} option is automatically turned
on for OECD calculations.

There are only a few studies of Electronic Circular
Dichroism using London atomic
orbitals~\cite{klbaehkrthjopjtca90,mpkrthcpl388}, and the results of
these investigations indicate that the aug-cc-pVDZ or the d-aug-cc-pVDZ
basis set, which is supplied with the \dalton\ basis set library, is
reasonable for such calculations, the double augmentation being
important in the case of diffuse/Rydberg-like excited states.

For a SOPPA\index{SOPPA}, SOPPA(CC2)\index{SOPPA(CC2)} or
SOPPA(CCSD)\index{SOPPA(CCSD)} calculation of rotatory strengths the
additional keywords \Key{SOPPA} or \Key{SOPPA(CCSD)} have to be
specified in the \Sec{*PROPERTIES} input module. For SOPPA an MP2
calculation has to be requested by the keyword \Key{MP2} in the
\Sec{*WAVE FUNCTIONS} input module, whereas for SOPPA(CC2) or
SOPPA(CCSD) a CC2 or CCSD calculation has to be requested by the
keyword \Key{CC} in the \Sec{*WAVE FUNCTIONS} input module with the
\Sec{CC INPUT} option \Key{SOPPA2} or \Key{SOPPA(CCSD)}. Details on how
to invoke an atomic integral direct RPA(D), SOPPA or SOPPA(CCSD)
calculation of rotatory strengths are give in chapters
\ref{sec:AOsoppa} and {sec:soppa}.


The calculation of rotatory strengths may of course be combined with
the calculation of oscillator strengths (chapter \ref{sec:uvvis}) in a
single run with an input that would then look like (where we also
request the rotatory strength to be calculated without the use of
London orbitals):

\begin{verbatim}
**DALTON INPUT
.RUN PROPERITES
**WAVE FUNCTIONS
.HF
**PROPERTIES
.ECD
.EXCITA
*EXCITA
.DIPSTR
.ROTVEL
.NEXCIT
    3
**END OF DALTON INPUT
\end{verbatim}


For a more detailed control of the individual parts of the
calculation of properties related to electronic excitation energies,
we refer to the input modules affecting the different parts of such
calculations:

\begin{description}
\item[\Sec{EXCITA}] Controls the calculation of electronic excitation
energies and the evaluation of all terms contributing to for instance
dipole strength or electronic circular dichroism.

\item[\Sec{GETSGY}] Controls the setup of the necessary right-hand
sides.

\item[\Sec{SOPPA}] Controls the details of a SOPPA calculation.
\end{description}

\section{Optical Rotation}\label{sec:optrot}

\begin{center}
\fbox{
\parbox[h][\height][l]{12cm}{
\small
\noindent
{\bf Reference literature:}
\begin{list}{}{}
\item MCSCF: T.Helgaker, K.Ruud, K.L.Bak, P.J{\o}rgensen, and
J.Olsen. \newblock {\em Faraday Discuss.}, {\bf 99},\hspace{0.25em}165, (1994).
\item P.~L.~Polavarapu \newblock {\em Mol.~Phys.}, {\bf 91},\hspace{0.25em}551, (1997).
\item DFT: K.Ruud and T.Helgaker. {\em Chem. Phys. Lett.}, {\bf 352},\hspace{0.25em}533, (2002).
\end{list}
}}
\end{center}
\index{optical rotation}

The calculation of optical rotation is a special case of the
calculation of Vibrational Raman Optical Activity (see
Sec.~\ref{sec:vroa}), as the tensor determining the optical rotation,
the mixed electric-magnetic dipole polarizability, also contributes to
vibrational Raman optical activity, although in the latter case it is the
geometrical derivatives of the tensor which are the central quantities.

Many of the comments made regarding basis set requirements for VROA
calculations will thus be applicable to the calculation of
optical rotation, too. It should be noted that a very extensive basis set
investigation of optical rotation has been
reported~\cite{jrcmjffjdpjsjpca104}.
A typical input for the calculation of optical rotation at
355 and 589.3 nm as well as close to the static limit would
be:

\begin{verbatim}
**DALTON INPUT
.RUN PROPERTIES
**WAVE FUNCTIONS
.HF
**PROPERTIES
.OPTROT
*ABALNR
.FREQUENCY
 1
 0.001
.WAVELENGTH
 2
 355.0 589.3
.THRESH
 1.0D-4
**END OF DALTON INPUT
\end{verbatim}

\dalton\ will always calculate the optical rotation both with and
without London atomic orbitals (length gauge form)
as this has a negligible computational
cost compared to the calculation using London atomic orbitals only,
since we will anyway have to solve only three response equations
corresponding to the perturbing electric field. The optical rotation
will only be observed for chiral molecules,
and by definition the optical
rotation will be zero in the static limit. One can approximate the
static limit by supplying the program with a very large wavelength
(small frequency), as in the example above, in order to be able to
compare with approximations that are only valid in the static
limit~\cite{rdacpl87,jrcmjffjdpjsjpca104}. Note that while the
frequency input must be in atomic units (hartree),
wavelengths must be supplied in nanometers (nm).

Origin invariance may also be guaranteed using the
``modified'' velocity
gauge\index{Modified velocity gauge}
formulation~\cite{Pedersen:ORMVE}.
This is invoked with the \Key{OR}
keyword which automatically
activates the \Key{OPTROT}, too. As additional response equations
must be solved, the \Key{OR} option is computationally more demanding than
specifying the \Key{OPTROT} keyword alone.
The modified velocity gauge formulation is used to ensure origin
invariance in Coupled Cluster calculations of optical
rotation~\cite{Pedersen:ORMVE} and the \Key{OR} option thus
allows direct comparisons of Coupled Cluster and
SCF, MCSCF, or DFT results.
An input that invokes calculation of both London and modified velocity gauge
optical rotation would be


\begin{verbatim}
**DALTON INPUT
.RUN PROPERTIES
**WAVE FUNCTIONS
.HF
**PROPERTIES
.OR
*ABALNR
.FREQUENCY
 1
 0.001
.WAVELENGTH
 2
 355.0 589.3
.THRESH
 1.0D-4
**END OF DALTON INPUT
\end{verbatim}

\section{Vibrational Raman Optical Activity (VROA)}\label{sec:vroa}

\begin{center}
\fbox{
\parbox[h][\height][l]{12cm}{
\small
\noindent
{\bf Reference literature:}
\begin{list}{}{}
\item MCSCF: T.Helgaker, K.Ruud, K.L.Bak, P.J{\o}rgensen, and
J.Olsen. \newblock {\em Faraday Discuss.}, {\bf 99},\hspace{0.25em}165, (1994).
\item DFT: K.Ruud, T.Helgaker, and P.Bour.
\newblock {\em J. Phys. Chem. A}, {\bf 106},\hspace{0.25em}7448, (2002).
\end{list}
}}
\end{center}

The calculation of vibrational Raman intensities and vibrational Raman optical
activity (VROA)\index{ROA}\index{Raman intensity}\index{Raman
optical activity}
is one of the more computationally expensive properties that can be
evaluated with \dalton .

Due to the time spent in the numerical differentiation\index{numerical
differentiation}, we have chosen
to calculate ROA both with and without London atomic
orbitals\index{London orbitals} in the
same calculation, because the time used in the set-up of the right-hand
sides differentiated
with respect to the external magnetic field is negligible compared to
the time used in the solution of the time-dependent response
equations~\cite{thkrklbpjjofd99}. Because of this,  all relevant Raman
properties (intensities and depolarization ratios)\index{Raman
intensity}\index{depolarization ratio} are also calculated
at the same time as ROA.


A very central part in the evaluation of Raman Optical Activity is the
evaluation the electric dipole-electric dipole, the electric
dipole-magnetic dipole, and the electric dipole-electric quadrupole
polarizabilities, and we refer to Section~\ref{sec:polari} for a more
detailed description of the input for such calculations.


When calculating Raman intensities\index{Raman intensity} and
ROA\index{ROA}\index{Raman optical activity} we need to do a numerical
differentiation\index{numerical differentiation} of the electric
dipole-electric dipole, the electric
dipole-magnetic dipole, and the electric dipole-electric quadrupole
polarizabilities along the normal modes\index{normal mode} of the
molecule. The procedure
is described in Ref.~\cite{thkrklbpjjofd99}. We thus need to do a
geometry walk of the type numerical differentiation. In each geometry
we need to evaluate the electric dipole-electric dipole, the electric
dipole-magnetic dipole, and the electric dipole-electric quadrupole
polarizabilities. This may be achieved by the following input:

\begin{verbatim}
**DALTON INPUT
.WALK
*WALK
.NUMERI
**WAVE FUNCTIONS
.HF
*HF INPUT
.THRESH
1.0D-8
**START
.VROA
*ABALNR
.THRESH
1.0D-7
.FREQUE
     2
0.0 0.09321471
**EACH STEP
.VROA
*ABALNR
.THRESH
1.0D-7
.FREQUE
     2
0.0 0.09321471
**PROPERTIES
.VROA
.VIBANA
*ABALNR
.THRESH
1.0D-7
.FREQUE
     2
0.0 0.09321471
*RESPONSE
.THRESH
1.0D-6
*VIBANA
.PRINT
 2
.ISOTOP
 1 5
 1 1 1 2 3
**END OF DALTON INPUT
\end{verbatim}

This is the complete input for a calculation of
VROA\index{ROA}\index{Raman optical activity} on the CFHDT
molecule\index{fluoromethane}. In addition to the keyword \Key{VROA} in
the different \Sec{*PROPERTIES} input modules, we still need to tell
the program that frequencies of the laser field are to be read in the
\Sec{ABALNR} section.

The only isotopic substitution of this molecule that shows vibrational
optical activity is the one containing one hydrogen, one deuterium and
one tritium nucleus. If we want the center-of-mass\index{center of
mass} to be the gauge origin\index{gauge origin} for the VROA
calculation not employing London atomic orbitals, this have to be
reflected in the specification of the isotopic constitution of the
molecule, see Chapter~\ref{ch:molinp}. We note that a user specified
gauge origin can be supplied with the keyword \Key{GAUGEO} in the
\Sec{*PROPERTIES} input modules. The gauge origin can also be chosen as
the origin of the Cartesian Coordinate system~(0,0,0) by using the
keyword \Key{NOCMC}. Note that neither of these options will affect the
results obtained with London orbitals.

The input in the \Sec{ABALNR} input section should be
self-explanatory from the discussion of the frequency dependent
polarizability\index{polarizability} in Sec.~\ref{sec:polari}. Note
that because of the numerical differentiation\index{numerical
differentiation} the response equations need to be converged
rather tightly (1.0$\cdot$10$^{-7}$). Remember also that this will
require you to converge your wave function\index{wave
function}\index{convergence!threshold} more
tightly than is the default.

The numerical differentiation\index{numerical differentiation} is
invoked through the keyword
\Key{NUMERI} in the \Sec{WALK} submodule. Note that this will
automatically
turn off the calculation of the molecular Hessian\index{Hessian},
putting limitations
on what properties may be calculated during a ROA calculation. Because
of this there will not be any prediction of the energy at the new
point.

It should also be noted that the program in a numerical differentiation will
step plus and minus one displacement along each Cartesian
coordinate
of all nuclei, as well as calculating the property in the reference
geometry. Thus, for a molecule with $N$ atoms the properties will be
calculated in a total of $2*3*N + 1$ points, which for a 5 atom
molecule will amount to 31 points. The default maximum number of steps
of the program is 20. By default the program will for numerical
differentiation calculations reset the maximum number of iterations to
6$N$+1. However, it is also possible to set the number of iterations
explicitly  in the general input module using the keyword \Key{MAX IT}
described in Section~\ref{sec:general}.

The default step length in the numerical integration is 10$^{-4}$
a.u., and this step length may be adjusted by the keyword
\Key{DISPLA} in the \Sec{WALK} module. The steps are taken in
the Cartesian directions\index{Cartesian coordinates} and
not along normal modes\index{normal mode}. This enables us to study a
large number of
isotopically substituted\index{isotopic constitution} molecules at
once, as the London orbital \index{London orbitals}
results for ROA does not depend on the choice of gauge origin. This is
done in the \Sec{*PROPERTIES} input module, but as only one isotopic
substituted species show optical activity, we have only requested a
vibrational analysis for this species.

We note that as in the case of Vibrational Circular Dichroism, a different
force field may be used in the estimation of the VROA intensity
parameters. Indeed, a number of force fields can be used to estimate
the VROA parameters obtained with a given basis set through the input:

\begin{verbatim}
**DALTON INPUT
.WALK
.ITERATION
 31
*WALK
.NUMERI
**PROPERTIES
.VROA
.VIBANA
*ABALNR
.THRESH
1.0D-7
.FREQUE
     2
0.0 0.09321471
*RESPONSE
.THRESH
1.0D-6
*VIBANA
.HESFIL
.PRINT
 2
.ISOTOP
 1 5
 1 1 1 2 3
**END OF DALTON INPUT
\end{verbatim}
by copying different \verb|DALTON.HES| files to the scratch
directory, which in turn is read through the keyword \Key{HESFIL}. By
choosing the start iteration to be 31 through the keyword
\Key{ITERAT}, we tell the program that the walk has finished (for
CHFDT with 31 points that need to be calculated). However, this
requires that all information is available in the \verb|DALTON.WLK|
file.

Concerning basis sets requirement for Raman Optical Activity, a
thorough investigation of the basis set requirements for the circular
intensity differences (CIDs) in VROA was presented by Zuber and
Hug~\cite{gzwhjpca108}. They also presented a close-to-minimal basis
set that yields high-quality CIDs. The force fields do, however, have
to be determined using larger basis (aug-cc-pVTZ) and including
electron correlation for a reliable prediction of VROA spectra.
