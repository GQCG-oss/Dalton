\chapter{Coupled cluster calculations, CC}\label{ch:CC}
\index{Coupled Cluster}
\index{CCS}
\index{CC2}
\index{CCD}
\index{CCSD}
\index{CCSD(T)}
\index{CC3}

The coupled cluster module {\cc} is designed for large-scale
correlated calculations of energies and properties using a
hierarchy of coupled cluster models: CCS, CC2, CCSD, and CC3, as well as standard methods such as MP2 and CCSD(T).
%\typeout{SONIA: Mention CCSD(T) more explicitly as well???}
The module offers almost the same options as the other
parts of the \dalton\ program for SCF and MCSCF wave functions.
It thus contains a wave function optimization section and 
a response function section
where linear, quadratic and cubic response functions and electronic
transition properties are calculated.
At the moment, molecular gradients are available 
up to the CCSD(T) level\index{molecular gradient!Coupled Cluster} for ground states.  
London orbitals have so far not been implemented 
for the calculation of magnetic
properties\index{magnetic properties!Coupled Cluster}\index{London orbitals!Coupled Cluster}.
However, gauge-invariant magnetic properties can be calculated using the
CTOCD-DZ method~\cite{ctocd,pccpcctocd}.
In the manual, more details can be found about the specific implementations.

An additional feature of the module is that all levels of correlation
treatment have been implemented using integral-direct techniques making
it possible to run calculations using large basis
sets~\cite{directCC}\index{integral direct!Coupled Cluster}.

In this chapter the general structure of the input for the
coupled cluster module is described.
The complete input for the coupled cluster module appears as
sections in the input for the \sir\ module, with the general
input in the input section \Sec{CC INPUT}. In order to get into the \cc\ module
one has to specify the \Key{CC} keyword in the general input
section of \sir\ \Sec{*WAVE FUNCTIONS}. For instance, a minimal
input file for a CCSD(T) energy calculation would be:
\begin{verbatim}
**DALTON INPUT
.RUN WAVE FUNCTIONS
**WAVE FUNCTIONS
.CC
*CC INPUT
.CC(T)
**END OF DALTON INPUT
\end{verbatim}

Several models can be calculated at the same time by specifying more models
in the CC input section. 
The models supported in the \cc\ module are 
CCS\cite{Christiansen:CPL243},
MP2\cite{Moller34},
CC2\cite{Christiansen:CPL243},
the CIS(D) excitation energy approximation \cite{Head-Gordon:94},
CCSD\cite{Purvis82},
the CCSDR(3) excitation energy approximation\cite{Christiansen:PERTURBATIVE_TRIPLES}, 
CCSD(T)\cite{Raghavachari89}, and CC3\cite{Christiansen:JCP103,Koch:JCP106}.
Several electronic properties can also 
be calculated in one calculation by specifying simultaneously 
the various input sections in the *CC INPUT section
as detailed in the following sections.

There is also a possibility for performing cavity coupled cluster
self-consistent-reaction field calculations for solvent modeling.
