\chapter{Generalized Active Space CI calculation with {\lucita}}\label{ch:lucita}

\section{General}\label{sec:lucita-general}

{\lucita} is a Generalized Active Space CI module written by Jeppe Olsen, 
which has been adapted to \dalton\ as well as fully parallelized 
based on the message passing interface (MPI) paradigm 
by Stefan Knecht and Hans J{\o}rgen Aagaard Jensen \cite{knecht08}.
It is integrated in \dalton\ as a part of the wave function/density computation code 
and can be activated via the keyword {\Key{GASCI}} in the **WAVE FUNCTIONS 
input section:\\[2ex]
\begin{inputex} \begin{verbatim}
**WAVE FUNCTIONS
.GASCI
\end{verbatim} \end{inputex}
\vspace{2ex}

{\lucita} is a string-based Hamiltonian-direct configuration interaction (CI) program, based on the LUCIA code \cite{olsen90}. 
For high efficiency the code takes advantage of point group symmetry of D$_{2h}$\ and it's subgroups.

A central feature of the program is the Generalized Active Space (GAS) concept \cite{fleig_gasci2}, 
in which the underlying total orbital space is subdivided 
into a basically arbitrary number (save for an upper limit) 
of subspaces with arbitrary occupation constraints. 
This is the most general approach to orbital space subdivisions and allows one to do 
efficient CI computations at arbitrary excitation level, e.g. FCI, SDCI, RASCI, 
and MRCI. The program uses orbitals from either a closed- or an open-shell calculation.

The technical limitations are roughly set by several 100 million determinants 
in the CI expansion on desktop PCs and several billions of determinants 
on common computing clusters and supercomputers with ample memory.

If desired, the program also computes 1- and 2-particle densities from optimized CI wave functions. 
The density matrices may be printed along with natural orbital occupations 
and the corresponding eigenvectors (NOs).

\section{\Sec{LUCITA} directives}\label{sec:lucita-inp}

          The following \Sec{LUCITA}\ directives for a succesful GAS-CI calculation 
are subdivided into {\bf{compulsory}} (Section \ref{sec:lucita-inp-mand}) 
and {\bf{optional keywords}} (Section \ref{sec:lucita-inp-opt}). 

\subsection{Mandatory keywords of \Sec{LUCITA}}\label{sec:lucita-inp-mand}

\begin{description}
\item[\Key{INIWFC}] Initial wave function type, either closed-shell (``HF\_SCF'') or open-shell (``RASSCF'') 
\verb|READ(LUCMD,*) lucita_cfg_ini_wavef |

\item[\Key{CITYPE}] Type of CI calculation. Allowed values are ``GASCI'' and ``RASCI'' \verb| |\newline
\verb|READ(LUCMD,*) lucita_cfg_ci_type |

\item[\Key{MULTIP}] Spin state multiplicity (singlett: 1, triplett: 3, etc.) \verb| |\newline
\verb|READ(LUCMD,*) lucita_cfg_is_spin_multiplett |


\vspace{0.75cm}

\noindent {\bf{\large{``RASCI''\ specific input (i.e. if \Key{CITYPE} == ``RASCI''):}}}

\item[\Key{NACTEL}] Number of active electrons \verb| |\newline
\verb|READ(LUCMD,*) lucita_cfg_nr_active_e |

\item[\Key{RAS1}] RAS1 specification and maximum number of holes. 
First line with orbitals per point group irrep separated by white spaces, 
followed by a line with the maximum number of holes in RAS1 \verb| |\newline
\verb|READ(LUCMD,*) (nas1_lucita(i), i = 1, #point_group_irreps) |\\
\verb|READ(LUCMD,*) lucita_cfg_max_holes_ras1 |

\item[\Key{RAS2}] RAS2 specification. A line with orbitals per point group irrep separated by white spaces \verb| |\newline
\verb|READ(LUCMD,*) (nas2_lucita(i), i = 1, #point_group_irreps) |

\item[\Key{RAS3}] RAS3 specification and maximum number of electrons. 
First line with orbitals per point group irrep separated by white spaces, 
followed by a line with the maximum number of electrons in RAS3 \verb| |\newline
\verb|READ(LUCMD,*) (nas3_lucita(i), i = 1, #point_group_irreps) |\\
\verb|READ(LUCMD,*) lucita_cfg_max_e_ras3 |

\newpage
\noindent {\bf{\large{``GASCI''\ specific input (i.e. if \Key{CITYPE} == ``GASCI''):}}}

\item[\Key{GAS SHELLS}] Number and specification of GAS shell excitation constraints and orbital occupations. 
A integer value (NGAS) in the first line specifies the number of GA spaces to be used (minimum 1; maximum 6). 
The next NGAS lines comprise one line per GAS with the minimum resp. maximum number of 
accumulated electrons after this GAS followed by (after the ``/'') the number 
of orbitals per point group irrep, separated by white spaces \verb| |\newline
\verb|READ(LUCMD,*) NGAS |\\
\verb|READ(LUCMD,*) (ngso_lucita(1,i), i=1,2)/(ngsh_lucita(1,i), i=1,#point_group_irreps) |\\
\verb|READ(LUCMD,*) (ngso_lucita(2,i), i=1,2)/(ngsh_lucita(2,i), i=1,#point_group_irreps) |\\
\verb|READ(LUCMD,*) (ngso_lucita(3,i), i=1,2)/(ngsh_lucita(3,i), i=1,#point_group_irreps) |\\
\verb|READ(LUCMD,*) until ``j'' = NGAS for ngso_lucita(j,*) and ngsh_lucita(j,*) |

\end{description}
\subsection{Optional keywords of \Sec{LUCITA}}\label{sec:lucita-inp-opt}

\begin{description}
\item[\Key{TITLE}] One line with a title of the CI calculation \verb| |\newline
\verb|READ(LUCMD,*) lucita_cfg_run_title |

\item[\Key{SYMMET}] State symmetry. This is the point-group irrep label 
referring to an irrep ordering as defined by the group generators (default: 1, e.g. totally symmetric irrep)  \verb| |\newline
\verb|READ(LUCMD,*) lucita_cfg_ptg_symmetry |

\item[\Key{INACTI}] Inactive doubly orbitals per point group irrep, 
separated by white spaces. (default: all orbitals active) \verb| |\newline
\verb|READ(LUCMD,*) (nish_lucita(i), i = 1, #point_group_irreps) |

\item[\Key{NROOTS}] number of eigenstates to be calculated (default: 1) \verb| |\newline
\verb|READ(LUCMD,*) lucita_cfg_nr_roots |

\item[\Key{MAXITR}] maximum number of Davidson CI iterations (default: 30) \verb| |\newline
\verb|READ(LUCMD,*) lucita_cfg_max_nr_dav_ci_iter |

\item[\Key{MXCIVE}] maximum dimension of Davidson subspace Hamiltonian (default: 3 $\times$\ \Key{NROOTS})\newline
\verb|READ(LUCMD,*)  lucita_cfg_max_dav_subspace_dim|

If the number of eigenstates to be calculated (\Key{NROOTS}) is ``5''\ 
the default will be\\ ``3 $\times$ 5'' == ``15''. A typical value will 
be given as a multiple $m$\ of the number of eigenstates where $m \geq 3$.

\item[\Key{ANALYZ}] analyze the composition of the final CI wave function \newline

\item[\Key{DENSI}]  Level of computed density matrices, ``1''\ means one-particle density only, ``2'' computes one- and two-particle density matrices (default: 1) \verb| |\newline
\verb|READ(LUCMD,*)  lucita_cfg_density_calc_lvl |

If set to ``1'', \lucita\ will calculate and print natural orbital occupation numbers which are a useful tool to further analyze 
your final CI wave function together with the composition of leading determinants (obtained via the keyword \Key{ANALYZ}).

\item[\Key{NATORB}] \lucita\ will calculate and print natural orbital occupation numbers (alias for \Key{DENSI} with input ``1'').\verb| |\newline

\item[\Key{RSTART}] restart option if set to ``1'' (default: 0, no restart) \verb| |\newline
\verb|READ(LUCMD,*) lucita_cfg_restart_ci |

Allows the restart of the CI calculation from coefficients obtained 
in a preceding calculation which are read from the file LUCITA\_CIVECS.``SYM''. 
The file ending ``SYM''\ is a single character in the range [a-h] and 
corresponds to the symmetry as specified via the input keyword \Key{SYMMET}, e.g., 
for symmetry 1\ the correct ending is ``a'', for symmetry 2\ we have to specify ``b'', etc.

\item[\Key{DISTRT}] Specifies vector block distribution routine to be used in a {\bf{parallel calculation}}; the keyword will be ignored in a serial run. (default: 2) \verb| |\newline
\verb|READ(LUCMD,*) lucipar_cfg_ttss_dist_strategy |

A simple distribution routine useful in small calculations 
can be activated by setting the variable to ``1''. A more advanced and in particular for large CI expansions most 
efficient distribution is activated by setting the parameter to ``2'' (default).

\end{description}
