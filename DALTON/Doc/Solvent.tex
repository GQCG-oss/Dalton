\chapter{Continuum solvation calculations}\label{ch:solvent}

This chapter describes the 
Multiconfigurational Self-Consistent Reaction Field (MCSCRF) model
as implemented in {\dalton}. The first section describes some
considerations about the implementation  and the range of
properties that may be evaluated with the present MCSCRF
implementation. The second section gives two input examples for
MCSCRF calculations.

\section{General considerations}\label{sec:solventimpl}

{\dalton} has the possibility of modeling the effect of a
surrounding linear, homogeneous dielectric
medium\index{dielectric medium} on a variety of molecular properties using
SCF\index{SCF}\index{HF}\index{Hartree--Fock}\index{MCSCF} or MCSCF
wave functions. This is achieved by the Multiconfigurational
Self-Consistent Reaction Field
(MCSCRF)\index{reaction field}\index{MCSCRF}
approach~\cite{kvmedpsjpc91,kvmhahjajthjcp89}, where the solute is
placed in a spherical cavity\index{cavity} and surrounded by the
dielectric medium. The solvent response to the presence of the
solute is modeled by a multipole 
expansion\index{multipole expansion}, in {\dalton} in principle to infinite order, but
practical applications show that the multipole expansion is
usually converged at order $L=6$.

In {\dalton} the solvent model is implemented both for SCF, DFT and MCSCF wave
functions in a self-consistent manner as describes in
Ref.~\cite{kvmedpsjpc91,kvmhahjajthjcp89}. In MCSCF calculations where
MP2 orbitals is requested as starting orbitals for the MCSCF
optimization, the solvent model will not be added before entering the
MCSCF optimization stage, so MP2 gas-phase orbitals can be used as
starting guess even though the solvent model has not been implemented
for this wave function model. Note also that differential densities will be
disabled in direct calculations when the solvent model is employed.

As regards molecular properties, the solvent model has so far been
extended to singlet linear, quadratic and cubic response, and  triplet
linear response\index{linear response}\index{quadratic response}\index{cubic response}\index{triplet response}\index{response!linear}\index{response!quadratic}\index{response!cubic}\index{response!triplet}
in the {\resp} program, both using equilibrium and non-equilibrium
solvation. A number of properties and excitation energies can
be calculated with the (MC)SCRF model, and several studies of such
properties have been presented, and we refer to these papers for an
overview of what can currently be calculated with the
approach~\cite{kvmpjhjajjcp100,kvmylhapjjcp100}, including ESR
hyperfine coupling
constants~\cite{bfocobpjkvmjcp104}\index{hyperfine coupling}.

In addition, a non-equilibrium
solvation\index{non-equilibrium solvation} model has been implemented
for molecular energies~\cite{kvmachahjajjcp103}. This model in needed
when studying processes where the charge distribution of the solute
cannot be expected to be in equilibrium with the charge distribution
of the solvent, {\it e.g.\/} when comparing with experiments where light has
been used as a perturbation.

In the \aba\ program, the solvent model has been implemented for
geometric distortions and nuclear shieldings and
magnetizabilities\index{nuclear shielding}\index{magnetizability},
and of course all the
properties that do not use perturbation-dependent basis
sets\index{perturbation-dependent basis set}, such as for
instance indirect spin-spin coupling 
constants\index{spin-spin coupling}. This is noteworthy, as
although the program will probably give results for most results
calculated using the solvent  model, these results will not
necessarily be theoretically correct, due to lack of reorthonormalization
contributions that have not been considered in the program. We
therefore give a fairly complete literature reference of works that
have been done with the
program~\cite{kvmpjkrthjcp106,poakvmkrthjpc100}. Properties not
included in this list are thus not trustworthy with the current
version of {\dalton}.

\section{Input description}\label{sec:solventinp}

\begin{center}
\fbox{
\parbox[h][\height][l]{12cm}{
\small
\noindent
{\bf Reference literature:}
\begin{list}{}{}
\item General reference: K.V.Mikkelsen, E.Dalgaard,
P.Svanstr{\o}m. \newblock {\em J.Phys.Chem}, {\bf
91},\hspace{0.25em}3081, (1987).
\item General reference: K.V.Mikkelsen, H.{\AA}gren, H.J.Aa.Jensen,
and T.Helgaker. \newblock {\em J.Chem.Phys.}, {\bf
89},\hspace{0.25em}3086, (1988).
\item Non-equilibrium solvation: K.V.Mikkelsen, A.Cesar, H.{\AA}gren,
H.J.Aa.Jensen.\newblock {\em J.Chem.Phys.}, {\bf
103},\hspace{0.25em}9010, (1995).
\item Linear singlet response: K.V.Mikkelsen, P.J{\o}rgensen,
H.J.Aa.Jensen.\newblock {\em J.Chem.Phys.}, {\bf
100},\hspace{0.25em}6597, (1994).
\item Linear triplet response: P.-O.\AA strand, K.V.Mikkelsen, P.J{\o}rgensen,
K.Ruud and T.Helgaker. \newblock {\em J.~Chem.~Phys.}, {\bf 108},\hspace{0.25em} 2528 (1998).
\item Hyperfine couplings: B.Fernandez, O.Christensen, O.Bludsky,
P.J{\o}rgensen,
K.V.Mikkelsen. \newblock {\em J.Chem.Phys.}, {\bf
104},\hspace{0.25em}629, (1996).
\item Magnetizabilities and nuclear shieldings: K.V.Mikkelsen,
P.J{\o}rgensen, K.Ruud, and T.Helgaker. \newblock {\em J.Chem.Phys.}, {\bf
106},\hspace{0.25em}1170, (1997).
\item Molecular Hessian: P.-O.\AA strand, K.V.Mikkelsen, K.Ruud and
T.Helgaker. \newblock {\em J.Phys.Chem.}, {\bf
100},\hspace{0.25em}19771, (1996).
\item Spin-spin couplings: P.-O.\AA strand, K.V.Mikkelsen, P.J{\o}rgensen,
K.Ruud and T.Helgaker. \newblock {\em J.~Chem.~Phys.}, {\bf 108},\hspace{0.25em} 2528 (1998).
\end{list}
}}
\end{center}


The necessary input for a solvent calculation is given in the
\Sec{*INTEGRALS} and \Sec{*WAVE FUNCTIONS} input modules. A typical input file
for an SCF calculation of the nuclear shielding constants of a
molecule in a dielectric medium will look
like\index{nuclear shielding}\index{dielectric medium}:

\begin{verbatim}
**DALTON INPUT
.RUN PROPERTIES
**INTEGRALS
*ONEINT
.SOLVENT
 10
**WAVE FUNCTIONS
.HF
*SOLVENT
.DIELECTRIC CONSTANT
 78.5
.MAX L
 10
.CAVITY
 3.98
**PROPERTIES
.SHIELD
**END OF DALTON INPUT
\end{verbatim}

In \Sec{*INTEGRALS} we request the evaluation of the undifferentiated solvent
multipole integrals\index{multipole integral} as given in for instance
Ref.~\cite{kvmhahjajthjcp89} by the keyword \Key{MAX L} in the
\Sec{ONEINT} submodule. We request all
integrals up to $L=10$ to be evaluated. This is needed if static or
dynamic (response) properties
calculations are to be done, but is not needed for a run of the
wave function only (\Sec{*WAVE FUNCTIONS}).

In \Sec{*WAVE FUNCTIONS} there is a separate input module for the
solvent input,
headed by the name \Sec{SOLVENT}. We refer to Sec.~\ref{ref-solinp}
for a presentation of all possible keywords in this submodule. The
interaction between the solute and the dielectric
medium\index{dielectric medium} is
characterized by three parameters; the dielectric
constant\index{dielectric constant}, the cavity\index{cavity}
size, and the order of the multipole
expansion\index{multipole expansion}. In the above input we have
requested a dielectric constant of 78.5 (corresponding to water)
through the keyword \Key{DIELECTRIC CONSTANT}, a cavity radius of 3.98 atomic
units with the keyword \Key{CAVITY}, and the multipole expansion is to
include all terms up to $L=10$, as can be seen from the keyword
\Key{MAX L}. Note that this number cannot be larger than the number given for
\Key{SOLVENT} in the \Sec{ONEINT} input module.

\subsection{Geometry optimization}\label{sec:solventgeoopt}

In the present release of the {\dalton} program, there are certain
limitations imposed on the optimizing geometries using the solvent
model. Only second-order geometry optimizations\index{geometry optimization}\index{geometry optimization!solvation}\index{solvation!geometry optimization}
are available, and only through the general \Sec{WALK} module. 
%Furthermore, symmetry
%cannot be used during the geometry optimization, and care must be
%exercised in order to turn off automatic symmetry detection in case of
%Hartree--Fock calculations. 
Thus the input for an SCF geometry optimization with the solvent model
would look like:

\begin{verbatim}
**DALTON INPUT
.WALK
**INTEGRALS
*ONEINT
.SOLVENT
 10
**WAVE FUNCTIONS
.HF
*SOLVENT
.DIELECTRIC CONSTANT
 78.5
.MAX L
 10
.CAVITY
 3.98
**PROPERTIES
.VIBANA
.SHIELD
**END OF DALTON INPUT
\end{verbatim}


\subsection{Non-equilibrium solvation}\label{sec:solvnoneqrsp}

This example describes calculations for non-equilibrium\index{non-equilibrium solvation}\index{solvation!non-equilibrium}
solvation. Usually one starts with a calculation of a reference state
(most often the ground state) with equilibrium solvation, using
keyword \Key{INERSFINAL}. The interface file is then
used (without user interference) for
a non-equilibrium excited state calculation; keyword
\Key{INERSINITIAL}.

\begin{verbatim}
**DALTON INPUT
.RUN WAVE FUNCTIONS
**INTEGRALS
*ONEINT
.SOLVENT
 10
**WAVE FUNCTIONS
.TITLE
 2-RAS(2p2p') : on F+ (1^D) in Glycol
 Widmark (5432)-ANO Basis set
.MCSCF
*CONFIGURATION INPUT
.SPIN MULTIPLICITY
 1
.SYMMETRY
 1
.INACTIVE ORBITALS
 1  0  0  0  0  0  0  0
.ELECTRONS
 6
.RAS1 SPACE
 0  0  0  0  0  0  0  0
.RAS2 SPACE
 1  2  2  0  2  0  0  0
.RAS3 SPACE
 8  4  4  3  4  3  3  1
.RAS1 ELECTRONS
 0  0
.RAS3 ELECTRONS
 0  2
*OPTIMIZATION
.NEO ALWAYS
.OPTIMAL ORBITAL TRIAL VECTORS
.MAX CI
 30
*ORBITAL INPUT
.MOSTART       | Note, we assume the existence of an SIRIUS.RST file
 NEWORB
*CI VECTOR
.STARTOLDCI    | Note, we assume the existence of an SIRIUS.RST file
*SOLVENT
.CAVITY
 2.5133D0
.INERSINITIAL     | initial state inertial polarization
 37.7D0  2.050D0  | static and optic dielectric constants for Glycol
.MAX L
 10
**END OF DALTON INPUT
\end{verbatim}
