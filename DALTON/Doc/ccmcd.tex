
%%%%%%%%%%%%%%%%%%%%%%%%%%%%%%%%%%%%%%%%%%%%%%%%%%%%%%%%%%%%%%%%%%%
\section{Magnetic circular dichroism: \Sec{CCMCD}}\label{sec:ccmcd}
%%%%%%%%%%%%%%%%%%%%%%%%%%%%%%%%%%%%%%%%%%%%%%%%%%%%%%%%%%%%%%%%%%%
\index{magnetic circular dichroism}
\index{MCD}
\index{transition strengths}
\index{transition moment!one-photon}
\index{transition moment!two-photon}
\index{B-term}

This section deals with the calculation of the 
${\cal{B}}(o\!\to\!f)$ term of magnetic circular dichroism
for a dipole-allowed transition between the ground state $o$ and 
the excited state $f$.
The ${\cal{B}}(o\to f)$ term results from an appropriate combination of
products of electric dipole one-photon transition moments
and mixed electric dipole-magnetic dipole
two-photon transition moments---
or, more precisely, a combination of transition strengths
$S_{AB,C}^{of}(0.0)$---
which are related to the single residues of the (linear and) 
quadratic response functions~\cite{Coriani:MCDRSP,Coriani:MOACC,Coriani:PHD}:
\[
{\cal{B}}(o\to f) \propto \varepsilon_{\alpha\beta\gamma} 
        M^{\mu_\gamma m_\beta}_{of}(0.0)M^{\mu_\alpha}_{fo}
\]
%
%re corresponds to an appropriate 
%combination of single residues of quadratic response functions, 
%or, more precisely, a combination of transition strengths
%$S_{AB,C}^{of}(0.0)$.
\noindent
Within \Sec{CCMCD} we specify the keywords needed in order
to compute the individual 
contributions for each chosen transition $(o\to f)$ to the
${\cal{B}}(o\to f)$ term. \Sec{CCMCD} has to be used in 
connection with \Sec{CCEXCI} for the calculation of 
excited states.
The calculation is implemented for the models CCS, CC2 and CCSD.
Note that the output results refer to the magnetic moment operator.

%\noindent Publications that report results obtained by this module
%should cite Ref.\ \cite{Coriani:MOACC}.

\begin{center}
\fbox{
\parbox[h][\height][l]{12cm}{
\small
\noindent
{\bf Reference literature:}
\begin{list}{}{}
\item S.~Coriani, C.~H\"{a}ttig, P.~J{\o}rgensen, and T.~Helgaker \newblock {\em J.~Chem.~Phys.}, {\bf 113},\hspace{0.25em}3561, (2000).
\end{list}
}}
\end{center}

\begin{description}
\item[\Key{MCD}] 
All six triples of operators obtained from the
permutations of 3 simultaneously different 
Cartesian components of the 3 operators 
\verb+LABELA,LABELB,LABELC+ are automatically specified.
Operator triples which do not correspond to symmetry-allowed
combinations will be ignored during the calculation.
This is the default.

\item[\Key{OPERAT}] \verb| |\newline
\verb|READ (LUCMD,'(3A)') LABELA, LABELB, LABELC|\newline
\verb|DO WHILE (LABELA(1:1).NE.'.' .AND. LABELA(1:1).NE.'*')|\newline
\verb|  READ (LUCMD,'(3A)') LABELA, LABELB, LABELC|\newline
\verb|END DO|
%

Manually select the operator label triplets defining the operator
Cartesian components involved in the ${\cal B}$ term. 
%, or, in general, in the $S^of_AB,C(0)$ strength.
The first two operators \verb+(LABELA,LABELB)+
are those that enter the two-photon moment.
The third operator \verb+(LABELC)+ 
is the one that enters the one-photon 
moment. They could be any of the (one-electron)
operators for which integrals are available in 
\Sec{*INTEGRALS}.
Specifically for the ${\cal{B}}$ term, \verb+LABELA+
corresponds to any component of the electric dipole 
moment operator, \verb+LABELB+ a component of the
angular momentum (magnetic dipole) and
\verb+LABELC+ to a component of the electric dipole.
Operator triples which do not correspond to symmetry allowed
combination will be ignored during the calculation.

%In case we need to specify relaxed/derivative operators
%the \Key{OPERAT} input is slightly different:
%\item[\Key{OPERAT}] \verb| |\newline
%\verb|READ (LUCMD'(3A)') LABELA, LABELB, LABELC|\newline
%\verb|DO WHILE (LABELA(1:1).NE.'.' .AND. LABELA(1:1).NE.'*')|\newline
%\verb|  READ (LUCMD'(3A)') (UNREL), (RELAX), (UNREL)|\newline
%\verb|  READ (LUCMD'(3A)') LABELA,  LABELB,  LABELC|\newline
%\verb|END DO|
%where \verb+(UNREL)+ refers to unrelaxed operators and 
%      \verb+(RELAX)+to relaxed operators.
%
%\item[\Key{MCDLAO}] 
%
% This keyword turns on the calculation of the B term components
% using London atomic orbitals or the relaxed angular momentum 
% operator. Unfinished.
%
%\item[\Key{NO2N+1}] 
%
%Disable the use of the $\bar{M}^f$ vectors in the one-photon transition
%moment part (does not use $2n+1$ rule).
%Disabled
\item[\Key{PRINT}] \verb| |\newline
\verb|READ (LUCMD,*) IPRINT |\newline
Sets the print level in the \Sec{MCDCAL} section. Default is \verb+IPRINT=0+.
%
\item[\Key{SELSTA}] \verb| |\newline 
\verb|READ (LUCMD,'(A70)') LABHELP|\newline
\verb|DO WHILE (LABHELP(1:1).NE.'.' .AND. LABHELP(1:1).NE.'*')|\newline
\verb|   READ(LABHELP,*) IXSYM,IXST|\newline
\verb|END DO|

Manually select one or more given excited states $f$ among those specified
in \Sec{CCEXCI}. 
%The maximum number of states that can be selected is
%set in the ccmcd.h common block.
The symmetry (\verb+IXSYM+) and state number (\verb+IXST+)
within that symmetry are then given,
one pair (\verb|IXSYM,IXST|) per line.
Default is all dipole allowed states for all symmetries 
specified in \Sec{CCEXCI}.
%
%\item[\Key{RELAXE}] 
%
% Disabled
%
%\item[\Key{UNRELA}] 
%
% Disabled
%
%\item[\Key{USEPL1}] 
%
%To be used in connection with LAO-relaxed angular momentum operators.
%
\end{description}
