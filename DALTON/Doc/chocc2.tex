~%%%%%%%%%%%%%%%%%%%%%%%%%%%%%%%%%%%%%%%%%%%%%%%%%%%%%%%%%%%%%%%%%%%
\section{Cholesky based CC2: \Sec{CHOCC2}}
\label{sec:chocc2}
%%%%%%%%%%%%%%%%%%%%%%%%%%%%%%%%%%%%%%%%%%%%%%%%%%%%%%%%%%%%%%%%%%%
\index{CC2}
\index{Cholesky decomposition-based methods}

In this Section, we describe the keywords controlling the algorithm to 
calculate CC2 energies as well as first  and  second order properties using Cholesky 
decomposed two-electron integrals. The Cholesky CC2 algorithm will be
automatically employed if the keyword \Key{CHOLES} was activated in the
\Sec{*DALTON} input module.  Calculation details such as, for instance, the
frequency at which the dipole polarizability is computed, are specified
in the \Sec{CCLR} section.


\begin{center}
\fbox{
\parbox[h][\height][l]{12cm}{
\small
\noindent
{\bf Reference literature:}
\begin{list}{}{}
\item H.~Koch, {A.~M.~J.~S{\'a}nchez~de~Mer{\'a}s}, and T.~B. Pedersen \newblock {\em J.~Chem.~Phys.}, {\bf 118},\hspace{0.25em}9481, (2003).
\item T.~B. Pedersen, {A.~M.~J.~S{\'a}nchez~de~Mer{\'a}s}, and H.~Koch 
\newblock {\em J.~Chem.~Phys.}, {\bf 120},\hspace{0.25em}8887, (2004).
`
\end{list}
}}
\end{center}

\begin{description}
\item[\Key{ALGORI}]\verb| |\newline
\verb|READ (LUCMD,*) IALGO|\newline
        Set algorithm: 1 for single virtual batch or 2 for 
        double. (Default: 2).
%
\item[\Key{CHOMO}] 
        Decompose the (ia$\mid$jb) integrals. Implies no decomposition
        of the CC2 $t_{2}$ amplitudes.
%
\item[\Key{CHOT2}] 
        Decompose the CC2 $t_{2}$ amplitudes. Implies no decomposition
        of the (ia$\mid$jb) integrals.
%
\item[\Key{MXDECM}] \verb| |\newline
\verb|READ (LUCMD,*) MXDECM|\newline
        Read in the maximum number of qualified diagonals in a 
        batch of the decomposition. (Default: 50).
%
\item[\Key{NCHORD}] \verb| |\newline
\verb|READ (LUCMD,*) NCHORD|\newline
        Read in the the maximum number of previous vectors
        to be read in each batch. (Default: 200).
%
\item[\Key{NOCHOM}] 
        No decompositions in CC2 section. But, of course,
        the previously calculated decomposed two-electron integrals
        will be used. This is the default.
%
\item[\Key{SPACC2}] \verb| |\newline
\verb|READ (LUCMD,*) SPACC2|\newline
        Span factor for decomposition. (Default: 
        The span factor used in the decomposition of
        the two-electron integrals).
%
\item[\Key{SPACCC}] \verb| |\newline
\verb|READ (LUCMD,*) SPACCC|\newline
        Span factor to use in amplitude decomposition for
        response intermediates and right-hand sides. (Default: 
        The span factor used in the decomposition of
        the two-electron integrals).
%
\item[\Key{SPLITM}] \verb| |\newline
\verb|READ (LUCMD,*) SPLITM|\newline
        Weight factor for Cholesky part in memory split for
        batching over virtuals. (Default: 1.0D0).
%
\item[\Key{THRCC2}] \verb| |\newline
\verb|READ (LUCMD,*) THRCC2|\newline
        Threshold for decomposition. (Default: 
        The threshold used in the decomposition of
        the two-electron integrals).
%
\item[\Key{THRCCC}] \verb| |\newline
\verb|READ (LUCMD,*) THRCCC|\newline
        Threshold to use in amplitude decomposition for
        response intermediates and right-hand sides. (Default: 
        The threshold used in the decomposition of
        the two-electron integrals).
%
\item[\Key{ZERO}] \verb| |\newline
\verb|READ (LUCMD,*) THZCC2|\newline
        Threshold for diagonal zeroing in decompositions. (Default: 
        The same used in the decomposition of
        the two-electron integrals).
%
\end{description}
