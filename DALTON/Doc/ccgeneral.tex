%%%%%%%%%%%%%%%%%%%%%%%%%%%%%%%%%%%%%%%%%%%%%%%%%%%%%%%%%%%%%%%%%%%
\section{General input for CC: \Sec{CC INPUT}}\label{sec:ccgeneral}
%%%%%%%%%%%%%%%%%%%%%%%%%%%%%%%%%%%%%%%%%%%%%%%%%%%%%%%%%%%%%%%%%%%

In this section keywords for the spin-restricted coupled
cluster options in \dalton are defined. In particular the coupled cluster 
model(s) and other parameters common to all submodules are specified.
The molecular orbitals must be canonical orbitals from a HF or HFsrDFT calculation.

\begin{description}
\item[\Key{B0SKIP}] 
   Skip the calculation of the $B0$-vectors 
   in a restarted run. See comments under \verb|.IMSKIP|.
%
\item[\Key{CC(2)}]  
        Run calculation for non-iterative CC2 excitation energy model 
        (which is actually the well-known CIS(D) model).
 
\item[\Key{CC(3)}]  
       Run calculation for CC(3) ground state energy model.

\item[\Key{CC(T)}]  
        Run calculation for CCSD(T) (Coupled Cluster Singles and 
        Doubles with perturbational treatment of triples) model.
        \index{CCSD(T)}
%
\item[\Key{CC2}]    
        Run calculation for CC2 model. 
        \index{CC2}
 
\item[\Key{CC3}]     
        Run calculation for CC3 model.
        \index{CC3}
%
\item[\Key{CCD}]    
        Run calculation for Coupled Cluster Doubles
        (CCD) model. 
        \index{Coupled Cluster!Doubles}
        \index{CCD}
%
\item[\Key{CIS}]    
        Run calculation for CI Singles (CIS) method. 
        \index{Configuration Interaction!Singles}
        \index{CI!Singles}
        \index{CIS}
%
\item[\Key{CHO(T)}]  
        Run calculation for CCSD(T) (Coupled Cluster Singles and 
        Doubles with perturbational treatment of triples) model
        using Cholesky-decomposed orbital energies denominators.
        Directives to control the calculation can be specifed in the
        \Sec{CHO(T)} input module.
        \index{CCSD(T)}
%
\item[\Key{CCR(3)}] 
        Run calculation for CCSDR(3) excitation energy model.
%
\item[\Key{CCS}] 
        Run calculation for Coupled Cluster Singles
        (CCS) model. 
        \index{Coupled Cluster!Singles}
        \index{CCS}
%
\item[\Key{CCSD}]   
        Run calculation for Coupled Cluster Singles and Doubles
        (CCSD) model. 
        \index{Coupled Cluster!Singles and Doubles}
        \index{CCSD}
%
%
\item[\Key{MTRIP}]   
        Run calculation of modified triples corrections for (MCCSD(T) for CCSD(T) and MCC(3) for CC(3))
        meaning that if the ground state lagrangian multipliers are calculated 
        they are used instead of amplitudes. Specialists option. 
        Do not use with analytical gradients. 
%
\item[\Key{CCSTST}] 
   Test option which runs the CCS finite field calculation as a pseudo CC2
   calculation. \verb+CCSTST+ disables all terms which depend on the
   double excitation amplitudes or multipliers. This flag must be
   set for CCS finite field calculations.
   \index{CCS}
   \index{finite field}
%
\item[\Key{DEBUG}]  
   Test option: print additional debug output.
%
\item[\Key{E0SKIP}] 
   Skip the calculation of the $E0$-vectors 
   in a restarted run. See comment under \verb|.IMSKIP|.
%
\item[\Key{F1SKIP}] 
   Skip the calculation of F-matrix transformed first-order
   cluster amplitude responses in a restarted run. See comment under \verb|.IMSKIP|.
%
\item[\Key{F2SKIP}] 
   Skip the calculation of F-matrix transformed second-order
   cluster amplitude responses in a restarted run. See comment under \verb|.IMSKIP|.
%
\item[\Key{FIELD}] \verb| |\newline
    \verb|READ (LUCMD,*) EFIELD|\newline
    \verb|READ (LUCMD,'(1X,A8)') LFIELD|

    Add an external finite field (operator label \verb+LFIELD+)
    of strength \verb+EFIELD+ to the Hamiltonian
    (same input format as in Sec.~\ref{ref-haminp}).
    These fields are only included in the CC calculation and 
    not in the calculation of the SCF reference state 
    ({\it i.e.\/} the orbitals). 
    Using this option in the calculation of numerical derivatives 
    thus gives so-called orbital-unrelaxed energy derivatives, 
    which is standard in coupled cluster response function theory.
    Note that this way of adding an external field does not work for 
    models including triples excitations (most notably CCSD(T) and CC3).
    For finite field calculations of orbital-relaxed energy 
    derivatives the field must be included in the SCF calculation.
    For such calculations, use the input section \Sec{HAMILTONIAN} (Sec.~\ref{ref-haminp}),
    and do \emph{not} specify the finite field here.
    \index{finite field}
    \index{external field}
 
\item[\Key{FREEZE}] \verb| |\newline
      \verb|READ (LUCMD,*) NFC, NFV|

     Specify the number of frozen core orbitals and number of frozen virtuals (0 is recommended for the latter).
     The program will automatically freeze the NFC(NFV) orbitals of lowest (highest) orbital energy
     among the canonical Hartree--Fock orbitals.
     For benzene a relevant choice is for example 6 0.

\item[\Key{FROIMP}] \verb| |\newline
      \verb|READ (LUCMD,*) (NRHFFR(I),I=1,MSYM)|\newline
      \verb|READ (LUCMD,*) (NVIRFR(I),I=1,MSYM)|

      Specify for each irreducible representation how
      many orbitals should be frozen (deleted) for the coupled
      cluster calculation. In calculations, the first \verb+NRHFFR(I)+
      orbitals will be kept frozen in symmetry class \verb+I+ and
      the last \verb+NVIRFR(I)+ orbitals will be deleted from the 
      orbital list.
 
\item[\Key{FROEXP}]  \verb| |\newline
    \verb|READ (LUCMD,*) (NRHFFR(I),I=1,MSYM)|\newline
    \verb|DO ISYM = 1, MSYM|\newline
    \verb|  IF (NRHFFR(ISYM.NE.0) THEN|\newline
    \verb|    READ (LUCMD,*) (KFRRHF(J,ISYM),J=1,NRHFFR(ISYM))|\newline
    \verb|  END IF|\newline
    \verb|END DO|\newline
    \verb|READ (LUCMD,*) (NVIRFR(I),I=1,MSYM)|\newline
    \verb|DO ISYM = 1, MSYM|\newline
    \verb|  IF (NVIRFR(ISYM.NE.0) THEN|\newline
    \verb|    READ (LUCMD,*) (KFRVIR(J,ISYM),J=1,NVIRFR(ISYM))|\newline
    \verb|  END IF|\newline
    \verb|END DO|

    Specify explicitly for each irreducible representation the
    orbitals that should be frozen (deleted) in the coupled cluster
    calculation.
 
\item[\Key{FRSKIP}] 
   Skip the calculation of the F-matrix transformed right eigenvectors
   in a restarted run. See comment under \verb|.IMSKIP|.
%
\item[\Key{HERDIR}] 
       Run coupled cluster program AO-direct\index{integral direct}
       using {\her}
       (Default is to run the program only AO integral direct
       if the \verb+DIRECT+ keyword was set in the general
       input section of \dalton . If \verb+HERDIR+ is not specified the \eri\
       program is used as integral generator.) 
%
\item[\Key{IMSKIP}] 
   Skip the calculation of some response intermediates in a restarted run.
   This options and the following skip options is primarily for very experienced users 
   and programmers who want to save a little bit of CPU time in restarts on very large 
   calculations. 
   {\em It is assumed that the user knows what he is doing when using these options, and
   the program does not test if the intermediates are there or are correct. 
   The relevant quantities are assumed to be at the correct directory and files.
   These comments also applies for all the other skip options following! }
%
\item[\Key{L0SKIP}]  
   Skip the calculation of the zeroth-order ground-state Lagrange
   multipliers in a restarted run. See comment under \verb|.IMSKIP|.
%
\item[\Key{L1SKIP}] 
   Skip the calculation of the first-order responses of the 
   ground-state Lagrange multipliers in a restarted run. See comment under \verb|.IMSKIP|.
%
\item[\Key{L2SKIP}]  
   Skip the calculation of the second-order responses of the 
   ground-state Lagrange multipliers in a restarted run. See comment under \verb|.IMSKIP|.
%
\item[\Key{LESKIP}]  
   Skip the calculation of left eigenvectors
   in a restarted run. See comment under \verb|.IMSKIP|.
%
\item[\Key{LISKIP}] 
   Skip the calculation of (ia$\mid$jb) integrals in a restarted run. See comment under \verb|.IMSKIP|.
%
\item[\Key{M1SKIP}] 
   Skip the calculation of the special zeroth-order Lagrange 
   multipliers for ground-excited state transition moments,
   the so-called $M$-vectors, in a restarted run. See comment under \verb|.IMSKIP|.
%
\item[\Key{MAX IT}] \verb| |\newline
  \verb|READ (LUCMD,'(I5)') MAXITE|

  Maximum number of iterations for wave function optimization 
  (default is \verb+MAXITE = 40+).
 
\item[\Key{MAXRED}] \verb| |\newline 
  \verb|READ (LUCMD,*) MAXRED|

  Maximum dimension of the reduced space for the 
  solution of linear equations (default is \verb+MAXRED = 200+).
 
%\item[\Key{CC1A}]   
%        Run calculation for CCSDT-1A model.  Not recommended for general use.
%
%\item[\Key{CC1B}]    
%        Run calculation for CCSDT-1B model.  Not recommended for general use.
%
\item[\Key{MP2}]    
Run calculation for second-order M{\o}ller-Plesset perturbation theory
(MP2)\index{M{\o}ller-Plesset!second-order}\index{MP2} method.

\item[\Key{MXDIIS}] \verb| |\newline
  \verb|READ (LUCMD,*) MXDIIS|

  Maximum number of iterations for DIIS algorithm
  before wave function information is discarded and a new DIIS 
  sequence is started
  (default is \verb+MXDIIS = 8+).
 
\item[\Key{MXLRV}] \verb| |\newline
  \verb|READ (LUCMD, *) MXLRV|

  Maximum number of trial vectors in the solution of 
  linear equations. If the number of trial vectors reaches this
  value, all trial vectors and their transformations with the
  Jacobian are skipped and the iterative procedure for the solution of the
  linear (i.e. the response) equations is restarted from the current 
  optimal solution. 
 
\item[\Key{NOCCIT}]
   No iterations in the wave function optimization is carried out.

\item[\Key{NSIMLE}] \verb| |\newline
  \verb|READ (LUCMD, *) NSIMLE|

  Set the maximum number of response equations that should be 
  solved simultaneously. Default is 0 which means that all
  compatible equations (same equation type and symmetry class) 
  are solved simultaneously.
 
\item[\Key{NSYM}] \verb| |\newline
       \verb|READ (LUCMD,*) MSYM2|

       Set number of irreducible representations. 
%       Due to difficulties to parse the number of irrep's from
%       the integral program to the coupled cluster input section,
%       this information must be specified in the coupled cluster
%       input section, if one of the options \verb+.FROIMP+ or
%       \verb+.NCCEXCI+ (see \verb+*CCEXCI+ subsection) are to be used.
%       Note that \verb+.NSYM+ has to be set before these keywords.
 
\item[\Key{O2SKIP}] 
   Skip the calculation of right-hand side vectors for the 
   second-order cluster amplitude equations
   in a restarted run. See comment under \verb|.IMSKIP|.
%
%\item[\Key{CCR(A)}] 
%        Run calculation for CCSDR(A) model. Not recommended for general use.
%
%\item[\Key{CCR(B)}]  
%        Run calculation for CCSDR(B) model. Not recommended for general use.

%
%\item[\Key{CCR(T)}] 
%        Run calculation for CCSDR(T) model. Not recommended for general use.
%
\item[\Key{PAIRS}]
         Decompose the singles and/or doubles coupled cluster correlation energy
         into contributions from singlet- and triplet-coupled
         pairs of occupied orbitals. Print those pair energies.

\item[\Key{PRINT}]  \verb| |\newline
\verb|READ (LUCMD,'(I5)') IPRINT|

       Set print parameter for coupled cluster program
       (default is to take the value \verb+IPRUSR+, set in the general
       input section of \dalton).
%
\item[\Key{R1SKIP}] 
   Skip the calculation of the first-order amplitude responses
   in a restarted run. See comment under \verb|.IMSKIP|.
%
\item[\Key{R2SKIP}] 
   Skip the calculation of the 
   second-order cluster amplitude equations
   in a restarted run. See comment under \verb|.IMSKIP|.
%
\item[\Key{RESKIP}] 
   Skip the calculation of the first-order responses of the 
   ground-state Lagrange multipliers in a restarted run. See comment under \verb|.IMSKIP|.
%
\item[\Key{RESTART}] 
       Try to restart the calculation from the cluster amplitudes,
       Lagrange multipliers, response amplitudes etc.\ stored on
       disk.
%
\item[\Key{SOPPA(CCSD)}] 
       Write the CCSD singles and doubles amplitudes on the Sirius interface
       for later use in a SOPPA(CCSD)\index{SOPPA(CCSD)} calculation. This
       chooses automatically the Coupled Cluster Singles and Doubles (CCSD) model.
%
\item[\Key{THRENR}] \verb| |\newline
       \verb|READ (LUCMD,*) THRENR|

       Set threshold for convergence of the ground state energy.
 
\item[\Key{THRLEQ}] \verb| |\newline
       \verb|READ (LUCMD,*) THRLEQ|

       Set threshold for convergence of the response equations.
 
\item[\Key{THRVEC}] \verb| |\newline
       \verb|READ (LUCMD,*) THRVEC|

       Set threshold for convergence of the ground state CC vector function.
 
%
%\item[\Key{NEWCAU}] % unfinished  option !
%   Solve Cauchy equations with different Cauchy order simultaneous. 
%   (The algorithm behind this is still in the test phase...)
%
\item[\Key{X2SKIP}] 
   Skip the calculation of the $\eta^{(2)}$ intermediates (needed
   to build the right-hand side vectors for the second-order 
   ground state Lagrange multiplier response equations) 
   in a restarted run. See comment under \verb|.IMSKIP|.
%
\end{description}

%All the special skip options for the restart of response calculations
%would better be placed in a separate input subsection.
%(And the corresponding logical variables on a separate 
%common block.)
