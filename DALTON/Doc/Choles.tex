~%%%%%%%%%%%%%%%%%%%%%%%%%%%%%%%%%%%%%%%%%%%%%%%%%%%%%%%%%%%%%%%%%%%
\section{Decomposition of two-electron integrals : \Sec{*CHOLES}}\label{sec:choles}
%
% Should this be a chapter after HERMIT?
%
%\chapter{\label{sec:choles} Decomposition of two-electron integrals, {\choles}}
%
%%%%%%%%%%%%%%%%%%%%%%%%%%%%%%%%%%%%%%%%%%%%%%%%%%%%%%%%%%%%%%%%%%%
\index{Cholesky decomposition-based integrals}

In this Section, we describe the keywords controlling the algorithm to 
decompose the two-electron integrals, which is activated by the 
keyword \Key{CHOLES} in the \Sec{*DALTON} input module.
Note that apart from
the options related to restart and, sometimes, to the decomposition
threshold, the following keywords are very seldom needed,
except for pathological cases. The same is the case for the sections
\Sec{CHOMP2}~\ref{sec:chomp2} and \Sec{CHOCC2}~\ref{sec:chocc2} 
of the coupled cluster input.

Note that two types of Cholesky decomposition are implemented in \dalton.
The other type, the Cholesky decomposition of energy denominators in CCSD(T) \index{CCSD(T)} is
invoked with keyword \Key{CHO(T)} (or \Sec{CHO(T)}~\ref{sec:chopt} if 
closer control is needed) in \Sec{CC INP}.


\begin{center}
\fbox{
\parbox[h][\height][l]{12cm}{
\small
\noindent
{\bf Reference literature:}
\begin{list}{}{}
\item H.~Koch, {A.~M.~J.~S{\'a}nchez~de~Mer{\'a}s}, and T.~B. Pedersen \newblock {\em J.~Chem.~Phys.}, {\bf 118},\hspace{0.25em}9481, (2003).
\end{list}
}}
\end{center}

\begin{description}
%
\item[\Key{COMPLE}] 
        Do not use the reduced set to store the vectors, but keep them in
        full dimension. Mainly for debugging purposes.
%
\item[\Key{DENDEC}] \verb| |\newline
\verb|READ (LUCMD,*) THRDC|\newline
        Decompose the density matrix in DIIS SCF. In the next line, the
        threshold for that decomposition is read. The default is not to
        decompose the density matrix.
%
\item[\Key{LENBUF}] \verb| |\newline
\verb|READ (LUCMD,*) LENBUF|\newline
        Buffer length used in the decomposition. Mainly for debugging purposes.
%
\item[\Key{NOSCDI}] 
        Do not screen  the initial diagonal 
        $D^{(0)}_{\alpha \beta}=
         M^{(0)}_{\alpha \beta,\alpha \beta}=(\alpha \beta \mid \alpha \beta)$
        of the two-electron integral matrix $(\alpha \beta \mid \gamma \delta)$.
        (Default: screen the diagonal).
%
\item[\Key{REDUCE}] 
        Use the reduced set to store the Cholesky vectors in disk, that is, keep
        only the elements corresponding to those not screened in the initial
        diagonal. This is the default.
%
\item[\Key{RSTCHO}]
        Restart the decomposition. File CHOLESKY.RST must be available.
%
\item[\Key{RSTDIA}]
        Use the diagonal computed in a previous calculation. File CHODIAG
        must be available.
%
\item[\Key{SPANDI}] \verb| |\newline
\verb|READ (LUCMD,*) SPAN|\newline
        Span factor ($s$) used in the decomposition. Only diagonals elements 
        \newline 
        \begin{center}
        $D^{(J)}_{\alpha \beta} \geq s \times (D^{(J)}_{\alpha \beta})_{max}$
        \end{center}
        are decomposed. (Default: 1.0D-3).
%
\item[\Key{THRCOM}] \verb| |\newline
\verb|READ (LUCMD,*) THRCOM|\newline
        Threshold ($\Delta$) for the decomposition. Convergence is achieved 
        when all the diagonal elements $D^{(J)}_{\alpha \beta} < \Delta$.
        (Default: 1.0D-8)
%
\item[\Key{THINDI}] \verb| |\newline
\verb|READ (LUCMD,*) THINDI|\newline
       Threshold ($\tau_0$) to zero out elements in the initial diagonal. 
        Diagonal elements are zeroed when 
       $D^{(0)}_{\alpha \beta} \times (D^{(0)}_{\alpha \beta})_{max} < \Delta^2/\tau_0$.
        (Default: 1.0D0).
%
\item[\Key{THSUDI}] \verb| |\newline
\verb|READ (LUCMD,*) THRCC2|\newline
       Threshold ($\tau_1$) to zero out elements in the subsequent diagonals along 
       the decomposition. Diagonal elements are zeroed when 
       $D^{(J)}_{\alpha \beta} \times (D^{(J)}_{\alpha \beta})_{max} < \Delta^2/\tau_1$.
        (Default: 1.0D3).
%
\end{description}
