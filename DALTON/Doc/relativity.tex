\chapter{\label{chap:Relativity}Relativistic Effects}

The following approaches to treat relativistic effects are available in DALTON:

\begin{description}

\item[ECP]\index{Effective core potentials}\index{ECP}\index{basis set!ECP}
The Effective Core Potential approach of Pitzer and
Winter~\cite{rmpnmwijqc40} is available for single-point calculations
by asking for ECP as the basis set for the chosen element. So far,
only a limited set of elements is covered by the basis set library. See
the \verb|rsp_ecp| example in the test-suite. The corresponding spin-orbit
operators are not implemented.

\begin{center}
\fbox{
\parbox[h][\height][l]{12cm}{
\small
\noindent
{\bf Reference literature:}
\begin{list}{}{}
\item R.~M.~Pitzer and N.~M.~Winter. \newblock {\em Int.~J. of Quantum Chem.}, {\bf 40}, \hspace{0.25em} 773 (1991)
\item L.~E.~McMurchie and E.~R.~Davidson.  \newblock {\em J.~Comp.~Phys.}, {\bf 44},\hspace{0.25em} 289 (1981)
\end{list}
}}
\end{center}


\item[Douglas-Kroll]\index{Douglas--Kroll} The Douglas--Kroll scalar relativistic one-electron integrals  
are available by adding the \verb|.DOUGLAS-KROLL|   keyword
\begin{verbatim}
**DALTON INPUT
.DOUGLAS-KROLL
.RUN WAVE FUNCTIONS
....
\end{verbatim}

See also the \verb|energy_douglaskroll| example in the test suite.

  NOTE: Exact analytical gradients and Hessians are not available 
at the moment, the approximate gradient and Hessians does, however, give fairly accurate geometries.  For this approach, only basis sets should be used 
where the contraction coefficients were optimized including the Douglas-Kroll 
operators. DALTON currently provides:  DK-Pol (relativistic version of Sadlej's POL basis sets), raf-r for some heavy elements, and the relativistically recontracted correlation-consistent basis sets of Dunning (cc-pVXZ-DK, X=D,T,Q,5). The combination with 
property operators should be done with care, {\it e.g.\/} the standard
magnetic property operators are not suitable in this case.

\begin{center}
\fbox{
\parbox[h][\height][l]{12cm}{
\small
\noindent
{\bf Reference literature:}
\begin{list}{}{}
\item M.~Douglas and N.~M.~Kroll. \newblock {\em Ann.~Phys.~(N.Y.)}, {\bf 82},\hspace{0.25em} 89 (1974)
\item B.~A.~Hess. \newblock {\em Phys.~Rev.~{\bf A}}, {\bf 33},\hspace{0.25em} 3742 (1986)
\end{list}
}}
\end{center}

\item[Spin-orbit Mean-Field]\index{AMFI}\index{spin-orbit mean-field} The spin-orbit mean-field approach can be used 
for either replacing the Breit-Pauli spin-orbit operator, or as an operator 
with suitable relativistic corrections in combination with the Douglas-Kroll 
approach. It is based on an effective one-electron operator, where the two-electron
terms are summed in a way comparable to the Fock operator~\cite{bahcmmuwogcpl251}. As all multi-center 
integrals are neglected, this scheme is very fast, avoids the storage of the 
two-electron spin-orbit integrals, and can therefore be used for large systems. 

\begin{verbatim}
.....
**INTEGRALS    
.MNF-SO    replaces     .SPIN-ORBIT             
.....
\end{verbatim}

For properties, the same substitution should be made, in the case of special 
components, \verb|X1SPNORB| labels are replaced by \verb|X1MNF-SO| and so on, whereas the 
two-electron terms will be skipped completely. For calculating phosphorescence with 
the quadratic response scheme, \verb|.PHOSPHORESENCE| should be just replaced by 
\verb|.MNFPHO| which takes care of choosing the appropriate integrals. 


\begin{center}
\fbox{
\parbox[h][\height][l]{12cm}{
\small
\noindent
{\bf Reference literature:}
\begin{list}{}{}
\item B.~A.~Hess, C.~M.~Marian, U.~Wahlgren and O.~Gropen. \newblock {\em Chem.~Phys.~Lett.}, {\bf 251},\hspace{0.25em} 365 (1996)
\end{list}
}}
\end{center}


NOTE:  

The choice between the Breit-Pauli or Douglas-Kroll mean-field operator 
is done by (not) providing the .DOUGLAS-KROLL keyword. It is therefore 
not possible to combine {\it e.g.\/} non-relativistic wave-functions with the 
Douglas-Kroll spin-orbit integrals. 


In the present implementation, the mean-field approach works only for basis sets 
with a generalized contraction scheme such as the ANO basis sets, raf-r, or cc-pVXZ(-DK).
For other types of basis sets, the program might work without a crash, but it
will most likely provide erroneous results.  



\end{description}



