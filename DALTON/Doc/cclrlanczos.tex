
%%%%%%%%%%%%%%%%%%%%%%%%%%%%%%%%%%%%%%%%%%%%%%%%%%%%%%%%%%%%%%%%%%%
\section{Calculation of linear response properties using an aymmetric Lanczos algorithm: \Sec{CCLRLCZ}}\label{sec:cclrlanczos}
%%%%%%%%%%%%%%%%%%%%%%%%%%%%%%%%%%%%%%%%%%%%%%%%%%%%%%%%%%%%%%%%%%%
%\index{linear response}
%\index{response!linear}
%\index{excitation energy!Coupled Cluster}
%\index{excited state!Coupled Cluster}

This section describes how to perform a calculation of (damped) 
linear response properties using an asymmetric Lanczos algorithm, 
instead of the standard iterative algorithm used in all other CC response sections.

{\bf Note that this is still highly experimental code, by no means optimized or efficient.}

For understanding the theoretical background for some
aspects of the method see Refs.\ \cite{Coriani:JCTC8,Coriani:PRA85}.

The properties that can be addressed with the current implementation are
\begin{itemize}
\item Global excitation energy spectrum (excitation energies and oscillator strengths) for a 
specific Cartesian component of the electric dipole operator;
\item Diagonal ($xx$, $yy$ or $zz$) component of the Damped 
Real dipole polarizability (for a given damping parameter);
\item Diagonal ($xx$, $yy$ or $zz$) component of the Damped 
Imaginary  dipole polarizability (for a given damping parameter);
\item Absorption Cross section spectrum / Dispersion profile for a selected 
laser frequency interval
\item Sum rules $S(n)$, $L(n)$ and $I(n)$ for $n=2,0,...,-6$
\end{itemize}

\begin{center}
\fbox{
\parbox[h][\height][l]{12cm}{
\small
\noindent
{\bf Reference literature:}
\begin{list}{}{}
\item S.~Coriani, O.~Christiansen, T.~Fransson and P.~Norman \newblock {\em Phys. Rev. A}, {\bf 85},\hspace{0.25em}022507, (2012).
\item S. Coriani, T. Fransson, O. Christiansen and P.~Norman, \newblock {\em J. Chem. Theory Comp.}, {\bf 8},\hspace{0.25em}1616, (2012).
\end{list}
}}
\end{center}

The method is  implemented for the iterative CC models CCS, CC2 and CCSD (for 
singlet states only).

\begin{description}

\item[\Key{OPERAT}] \verb| |\newline
   \verb|READ (LUCMD,'(2A)') LABELA, LABELB|\newline
%   \verb|DO WHILE (LABELA(1:1).NE.'.' .AND. LABELA(1:1).NE.'*')|\newline
 %  \verb|  READ (LUCMD,'(2A)') LABELA, LABELB|\newline
  % \verb|END DO|

Read the wished pair of operator labels. Note that ONLY \verb|LABELA=LABELB| 
works in current implementation, and only for one pair at a time
(i.e. you need to run 3 jobs if you want both $xx$, $yy$ and $zz$ components.

\item[\Key{CHAINL}] \verb| |\newline
      \verb|READ (LUCMD,'(4I)') JCHAIN |\newline
      
Read the wished maximum dimension of the Lanczos chain (the ``chain length''). 
This length is reset if it is found larger than the 
excitation space for the considered system.      

\item[\Key{FREQ I}] \verb| |\newline
      \verb|READ(LUCMD,*) (FREQ_RANGE(I), I=1,3) |\newline
      
Read the  frequency interval within which to generate the absorption cross section spectrum.
\verb|FREQ_RANGE(1)| = start frequency;  \verb|FREQ_RANGE(2)| = end frequency, 
\verb|FREQ_RANGE(3)| = frequency step.

\item[\Key{DAMPING}] \verb| |\newline
      \verb| READ (LUCMD,*) NDAMP |\newline
      \verb| READ (LUCMD,*) (DAMPING(I),I=1,NDAMP) |\newline
      
Read the wished number of damping parameters (\verb|NDAMP|) and 
the their values (\verb|DAMPING(I)|). 
If nothing is specified, a default value of 1000 cm$^{-1}$ (0.004556 a.u.) 
will be used.

\item[\Key{SUMRULES}] \verb| |\newline
%      \verb| READ (LUCMD,*) NDAMP |\newline
%      \verb| READ (LUCMD,*) (DAMPING(I),I=1,NDAMP) |\newline


\end{description}
