\chapter{Molecular vibrations}\label{ch:vibrot}

In this chapter we discuss properties related  to the
vibrational motions of a molecule. This includes
vibrational frequencies and the associated infrared
(IR)\index{IR intensity} and Raman intensities\index{Raman intensity}.

\section{Vibrational frequencies}\label{sec:vibfreq}

The calculation of vibrational frequencies\index{vibrational
frequency} are
controlled by the keyword \Key{VIBANA}. Thus, in order to
calculate the vibrational frequencies
of a molecule, all that is is needed is the input:

\begin{verbatim}
**DALTON INPUT
.RUN PROPERTIES
**WAVE FUNCTIONS
.HF
**PROPERTIES
.VIBANA
**END OF DALTON INPUT
\end{verbatim}

This keyword will, in addition to calculating the molecular
frequencies, also calculate the zero-point vibrational
energy corrections\index{zero-point vibrational energy} and vibrational
and rotational partition functions\index{partition functions}
at selected temperatures.

\dalton\ evaluates the molecular Hessian\index{Hessian} in Cartesian
coordinates\index{Cartesian coordinates}, and
the vibrational frequencies of any isotopically substituted  species
may therefore easily be obtained on the basis of the full
Hessian. Thus, if we would like to calculate the vibrational
frequencies of isotopically substituted molecules\index{isotopic
constitution}, this may be obtained through an input like:

\begin{verbatim}
**DALTON INPUT
.RUN PROPERTIES
**WAVE FUNCTIONS
.HF
**PROPERITES
.VIBANA
*VIBANA
.ISOTOP
   2   5
 1 2 1 1 1
 2 1 1 1 1
**END OF DALTON INPUT
\end{verbatim}

The keyword \Key{ISOTOP} in the \Sec{VIBANA} input module
indicates that more than only the isotopic species containing the most
abundant isotopes are to be calculated, which will always be
calculated. The numbers on the second line denote the number of
isotopically substituted species that are requested and the number of
atoms in the system. The following lines then
list the isotopic constitution of each of these species. 1
corresponds to the most abundant isotope, 2 corresponds to the second
most abundant isotope and so on. The isotopic substitution have to be
given for all atoms in the molecule (not only the symmetry
independent), and the above input could for instance correspond to a
methane\index{methane} molecule, with the isotopic species CH$_3$D
and $^{13}$CH$_4$.

As the isotopic substitution of all atoms in the molecule have to be
specified, let us mention the way symmetry-dependent atoms will be
generated. The atoms will be grouped in symmetry-dependent atom
blocks. The specified symmetry-independent atom will be the first of
this block, and the symmetry-dependent atoms will be generated
according to the order of the symmetry elements. Thus, assuming
D$_{2h}$ symmetry with symmetry generating elements \verb|X  Y  Z|,
the atoms generated will come in the order \verb|X|, \verb|Y|,
\verb|XY|, \verb|Z|, \verb|XZ|, \verb|YZ|, and \verb|XYZ|.

\section{Infrared (IR) intensities}\label{sec:irint}

\begin{center}
\fbox{
\parbox[h][\height][l]{12cm}{
\small
\noindent
{\bf Reference literature:}
\begin{list}{}{}
\item R.D.Amos. \newblock {\em Chem.Phys.Lett.}, {\bf
108},\hspace{0.25em}185, (1984).
\item T.U.Helgaker, H.J.Aa.Jensen, and P.J{\o}rgensen. \newblock {\em
J.Chem.Phys.}, {\bf 84},\hspace{0.25em}6280, (1986).
\end{list}
}}
\end{center}

\index{IR intensity} The evaluation of infrared intensities
requires the calculation of the
dipole gradients\index{dipole gradient}\index{APT}\index{atomic polar
tensor} (also known as Atomic Polar Tensors (APTs)). Thus, by
combining the calculation of vibrational frequencies with the
calculation of dipole gradients, IR intensities will be obtained. Such
an input may look like:

\begin{verbatim}
**DALTON INPUT
.RUN PROPERTIES
**WAVE FUNCTIONS
.HF
**PROPERTIES
.VIBANA
.DIPGRA
*VIBANA
.ISOTOP
   2   5
 1 2 1 1 1
 2 1 1 1 1
**END OF DALTON INPUT
\end{verbatim}

\noindent The keyword \Key{DIPGRA} invokes the calculation of the dipole
gradients.

\section{Dipole-gradient based population analysis}

\begin{center}
\fbox{
\parbox[h][\height][l]{12cm}{
\small
\noindent
{\bf Reference literature:}
\begin{list}{}{}
\item J.Cioslowski. \newblock {\em J.Am.Chem.Soc.}, {\bf
111},\hspace{0.25em}8333, (1989).
\item T.U.Helgaker, H.J.Aa.Jensen, and P.J{\o}rgensen. \newblock {\em
J.Chem.Phys.}, {\bf 84},\hspace{0.25em}6280, (1986).
\end{list}
}}
\end{center}

\index{population analysis} As dipole gradients\index{dipole
gradient}\index{APT}\index{atomic polar tensor} are readily available in the
\dalton\ program, the
population analysis basis on the Atomic Polar Tensor as suggested by
Cioslowski~\cite{jcjacs111,poakrkvmthjpca102} can be obtained from an input like

\begin{verbatim}
**DALTON INPUT
.RUN PROPERTIES
**WAVE FUNCTIONS
.HF
**PROPERTIES
.POPANA
*END OF DALTON
\end{verbatim}

This population analysis is of course significantly more expensive
than the ordinary Mulliken population analysis\index{population analysis!Mulliken}\index{Mulliken population analysis} obtainable directly
from the molecular wave functions through an input like

\begin{verbatim}
**DALTON INPUT
.RUN WAVE FUNCTIONS
**WAVE FUNCTIONS
.HF
*POPULATION ANALYSIS
.MULLIKEN
*END OF DALTON
\end{verbatim}


\section{Raman intensities}\label{sec:ramanint}

\begin{center}
\fbox{
\parbox[h][\height][l]{12cm}{
\small
\noindent
{\bf Reference literature:}
\begin{list}{}{}
\item T.Helgaker, K.Ruud, K.L.Bak, P.J{\o}rgensen, and J.Olsen. \newblock {\em
Faraday Discuss.}, {\bf 99},\hspace{0.25em}165, (1994).
\end{list}
}}
\end{center}

\index{Raman intensity} Calculating Raman intensities is by no means
a trivial task, and
because of the computational cost of such calculations, there are
therefore few theoretical investigations of basis set requirements and
correlation effects on calculated Raman intensities. The Raman
intensities calculated are the ones obtained within the Placzek
approximation~\cite{placzek}\index{Placzek approximation},
and the implementation is described in Ref.~\cite{thkrklbpjjofd99}.

The Raman intensity is the differentiated frequency-dependent
polarizability\index{polarizability} with respect to nuclear displacements.
As it is a third derivative depending on the nuclear positions through
the basis set, numerical differentiation of
the polarizability with respect to nuclear coordinates is
necessary.

The input looks very similar to the input needed for the calculation
of Raman optical activity\index{ROA}\index{Raman optical activity}  described
in Section~\ref{sec:vroa}

\begin{verbatim}
**DALTON INPUT
.WALK
*WALK
.NUMERI
**WAVE FUNCTIONS
.HF
*SCF INPUT
.THRESH
1.0D-8
**START
.RAMAN
*ABALNR
.THRESH
1.0D-7
.FREQUE
     2
0.0 0.09321471
**EACH STEP
.RAMAN
*ABALNR
.THRESH
1.0D-7
.FREQUE
     2
0.0 0.09321471
**PROPERTIES
.RAMAN
.VIBANA
*RESPONSE
.THRESH
1.0D-6
*ABALNR
.THRESH
1.0D-7
.FREQUE
     2
0.0 0.09321471
*VIBANA
.PRINT
 1
.ISOTOP
   1   5
 1 1 1 2 3
**END OF DALTON INPUT
\end{verbatim}

The keyword \Key{RAMAN} in the general input module indicates that
a frequency-dependent polarizability\index{polarizability}\index{response!linear}
calculation is to be done. The keyword \Key{RAMAN} indicates that we are only
interested in the Raman intensities\index{Raman intensity} and
depolarization ratios\index{depolarization ratio}. Note
that these parameters are also obtainable by using the keyword
\Key{VROA}. In this calculation we calculate the Raman intensities for two
frequencies, the static case and a frequency of the incident light
corresponding to a laser of wavelength 488.8 nm.

Due to the numerical differentiation\index{numerical differentiation}
that is done, the threshold for
the iterative solution of the response equations are by default
10$^{-7}$, in order to get Raman intensities that are numerically
stable to one decimal digit.

In the \Sec{WALK} input module we have specified
that the walk is a numerical differentiation. This will automatically
turn off the calculation of the geometric Hessian\index{Hessian},
putting limitations
on what kind of properties that may be calculated at the same time as
Raman intensities. Because the Hessian is not calculated,
there will not be any prediction of the energy at the new
point.

It should also be noted that  in a numerical
differentiation\index{numerical differentiation}, the
program will
step plus and minus one displacement unit along each Cartesian coordinate
of all nuclei, as well as calculating the property at the reference
geometry. Thus, for a molecule with $N$ atoms the properties will need
to be calculated in a total of 2*3*$N$ + 1 points, which for a
molecule with five atoms will amount to 31 points. The default maximum number of
steps in \dalton\ is 20. However, in numerical differentiation
calculations, the number of iterations will always be reset (if there
are more than 20 steps that need to be taken) to 6$N$+1, as it is
assumed that the user always wants the calculation to complete
correctly. The maximum number of allowed iterations\index{geometry
iteration} can be manually set by adding the keyword
\Key{MAX IT} in the \Sec{*DALTON} input module.

The default step length in the numerical
differentiation\index{numerical differentiation} is $1.0\cdot 10^{-4}$
a.u., and this step length may be adjusted by the keyword
\Key{DISPLA} in the \Sec{WALK} input module. The steps are taken
in the Cartesian directions and
not along normal modes. This enables us to study the Raman intensities
of a large number of isotopically substituted molecules at once. This
is done in the \Sec{*PROPERTIES} input section, where we
have requested one isotopically substituted species in addition to the
isotopic species containing the most abundant isotope of each element.


\section{Vibrational g factor}




\begin{center}
\fbox{
\parbox[h][\height][l]{12cm}{
\small \noindent {\bf Reference literature:}
\begin{list}{}{}
\item K.L.Bak, S.P.A.Sauer, J.Oddershede and J.F.Ogilvie.
\newblock {\em Phys.Chem.Chem.Phys.},
{\bf 7},\hspace{0.25em}1747, (2005).
\item H.Kj{\ae}r and S.P.A.Sauer.
\newblock {\em Theor.Chem.Acc.}, {\bf 122},\hspace{0.25em}137, (2009).
\end{list}
}}
\end{center}

Non-adiabatic corrections to the moment of inertia tensor for molecular
vibrations can be calculated with the \dalton\ program with the keyword
\Key{VIB\_G}.\index{non-adiabatic corrections}\index{moment of inertia
tensor!non-adiabatic corrections} For diatomic molecules this is known
as the vibrational g factor \cite{rv66ha,jfo98ho,spas063}.
\index{vibrational g factor} The response functions necessary for the
electronic contribution to the vibrational g factor can be obtained
from an input like


\begin{verbatim}
**DALTON INPUT
.RUN PROPERTIES
**WAVE FUNCTIONS
.HF
**PROPERTIES
.VIB_G
*TROINV
.SKIP
**END OF DALTON INPUT
\end{verbatim}

SCF and MCSCF wavefunctions or DFT can be employed in such a
calculation. The nuclear contribution, on the other hand, can trivially
be calculated from the nuclear charges and coordinates
\cite{spas063,spas084}. The program generates thus a series of linear
response functions for components of the nuclear linear momentum
operators, i.e.\ derivatives with respect to the nuclear (symmetry)
coordinates. The user has then to select the pair of coordinates
relevant for the vibrational mode of interest and multiply the
corresponding response function with the appropriate masses and natural
constants \cite{spas063,spas084}. In the case, that the calculation has
made use of the molecular point group symmetry, one has to remove the
symmetry adaptation of then nuclear coordinates and thus derivative
operators as well.

Mass-independent contributions to the vibrational g factor of diatomic
molecules can analogously be obtained by combining the linear response
function of two nuclear momentum operators with response functions
involving one nuclear momentum operator and components of the total
electronic moment operator according to Eq.~(11) in Ref.~\cite{spas084}.
 These response functions are also printed in the
output.

The convergence thresholds for the calculation of the molecular gradient as
well as the linear response functions of the nuclear linear momentum operators
can be controlled with the \Key{THRESH} keyword in the \Sec{RESPON} and
\Sec{ABALNR} sections, respectively. The input file for a calculation with
significantly smaller thresholds than the default values would like like the
following

\begin{verbatim}
**DALTON INPUT
.RUN PROPERTIES
**WAVE FUNCTIONS
.HF
**PROPERTIES
.VIB_G
*TROINV
.SKIP
*RESPON
.THRESH
 1.0D-07
*ABALNR
.THRESH
1.0D-7
**END OF DALTON INPUT
\end{verbatim}
