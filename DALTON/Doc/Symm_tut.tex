\chapter{Aspects of symmetry in Dalton}

% radovan: for the moment taken out until we have a decision whether
%          we want explicit author information and acknowledgements
%\begin{center}
%\textbf{Peter R. Taylor} \\
%Victorian Life Sciences Computation Initiative and
%School of Chemistry, \\
%University of Melbourne, Vic~3010, Australia.
%\end{center}
%
%\vspace{5mm}

It is not always straightforward for new Dalton users to understand
why some aspects of symmetry handling are the way they are.  The order
of ``symmetries'', that is, irreducible representations, is for example often so
different from what a user might see looking at character tables in
books or online that it may be confusing, and the
question ``why do symmetries come out in this bizarre order?'' is not
an unfamiliar one.  This section is an attempt to explain to users how
the symmetry handling in Dalton ''works'', why things are the way they
are, and how to exploit the considerable potential for computational
saving that the symmetry handling presents, both for the point groups
that Dalton uses explicitly ($D_{2h}$ and its subgroups), and for
higher symmetry systems.

\section{Specifying symmetry by generators}

Dalton follows the lead established by Alml{\"o}f~\cite{Alm72} more
than forty
years ago, by describing, and internally treating, the system symmetry
by means of group \emph{generators}.  In the mathematics of group
theory, the operations of groups of very high order (much higher order
than the point groups we encounter in chemistry!) are specified in
terms of a set of elementary operations called ``generators'',
together with the complete set of relations these operators obey, so
that by constructing appropriate products of the generators according
to these relations, all of the operations in the group can be
obtained.  Dalton internally uses the great simplification of treating
at most $D_{2h}$ symmetry explicitly, and this has a consequent
simplification for the use of generators.  To be specific, Dalton
explicitly handles the following groups.
\begin{itemize}
\item Group of order~1: $C_1$.
\item Groups of order~2: $C_s$, $C_2$, $C_i$ (also referred to as
$S_2$).
\item Groups of order~4: $C_{2v}$, $D_2$, $C_{2h}$.
\item Group of order~8: $D_{2h}$.
\end{itemize}
These groups have generators the number of which behaves as $\log_2
g$, where $g$ is
the order of the group, so respectively 0, 1, 2, 3.

Obviously the only operator in $C_1$ is $E$, the identity: end of
story.  For the groups of order~2, there is an additional
generator~$G_1$: these groups are isomorphic (that is, they have the
same group multiplication table).  The groups of order~4 have two
generators that we denote $G_1$ and $G_2$: these groups are also
isomorphic to one another.  We note at this point that there is
another group of order~4: the
group isomorphic to the group $C_4$, the cyclic group of order~4.
This group cannot be treated by the methods coded
in Dalton.  Finally, $D_{2h}$ requires the specification of three
generators, $G_1$, $G_2$, and $G_3$.  There are two other point groups
of order~8: groups isomorphic to $D_{2d}$ (or equivalently to
$D_4$ or to $C_{4v}$) and to $C_8$.
Again, none of the latter groups can be treated explicitly within
Dalton.

Because all of the groups that Dalton handles internally are Abelian,
that is,
\begin{displaymath}
  G_i G_j = G_j G_i, \forall\ i,j,
\end{displaymath}
the situation with the generators and the group operations their
products generate is relatively simple.  The \emph{presentation} (that
is, the complete spcification of the groups in terms of generators),
is given by just two rules: the commutation relation given immediately
above and
\begin{displaymath}
  G_i^2 = E, \forall\ i.
\end{displaymath}
For the
groups of order~2 the operations are trivially $E$ and $G_1$.  For the
groups of order~4 they are $E$, $G_1$, $G_2$, and $G_1G_2$.  For the
group of order~8 the latter are augmented with $G_3$, $G_1G_3$,
$G_2G_3$, and $G_1G_2G_3$.  We emphasize here that this ordering is
important: this is how Dalton assembles the symmetry operations when
specifying internally the point group.  In particular,
Table~\ref{tab:ops} shows the specific ordering of the elements given
a particular set of generators, as well as the symmetry properties ---
symmetric~(+) or antisymmetric~($-$) --- under each generator.  It is
this ordering that determines the ordering of the irreducible
representations in a calculation.  This shows that (for example) the
eighth symmetry in Dalton comprises functions that are antisymmetric
with respect to all three generators.

\begin{table}[hbt]
  \caption{Symmetry/antisymmetry behaviour under generator products}
  \label{tab:ops}
%  \vspace*{10pt}
\begin{center}
  \begin{tabular}{c|c|ccc}
      & Generator & \multicolumn{3}{c}{Behaviour under}\\
      &  Product        & ~~~$G_1$~~~ & ~~~$G_2$~~~ & ~~~$G_3$~~~ \\
    \hline\\[-7pt]
    1 &      $E$        &  +  &  +  &  + \\
    2 &     $G_1$       & $-$ &  +  &  + \\
    3 &     $G_2$       &  +  & $-$ &  + \\
    4 &   $G_1$$G_2$    & $-$ & $-$ &  + \\
    5 &     $G_3$       &  +  &  +  & $-$\\
    6 &   $G_1$$G_3$    & $-$ &  +  & $-$\\
    7 &   $G_2$$G_3$    &  +  & $-$ & $-$\\
    8 & $G_1$$G_2$$G_3$ & $-$ & $-$ & $-$
  \end{tabular}
\end{center}
%% \noindent
%% This shows the behaviour of a function (symmetric or antisymmetric)
%% with respect to the possible point-group operations.
\end{table}

  If
the sequence numbers given in Table~\ref{tab:ops} are used to label
each symmetry operation, then the group multiplication table is as
given in Table~\ref{tab:mult}.

\begin{table}[hbt]
  \caption{Group multiplication table (products of operations)}
  \label{tab:mult}
%  \vspace*{10pt}
\begin{center}
  \begin{tabular}{r|rrrrrrrr}
    & 1 & 2 & 3 & 4 & 5 & 6 & 7 & 8\\
    \hline\\[-7pt]
    1 & 1 & 2 & 3 & 4 & 5 & 6 & 7 & 8\\
    2 & 2 & 1 & 4 & 3 & 6 & 5 & 8 & 7\\
    3 & 3 & 4 & 1 & 2 & 7 & 8 & 5 & 6\\
    4 & 4 & 3 & 2 & 1 & 8 & 7 & 6 & 5\\
    5 & 5 & 6 & 7 & 8 & 1 & 2 & 3 & 4\\
    6 & 6 & 5 & 8 & 7 & 2 & 1 & 4 & 3\\
    7 & 7 & 8 & 5 & 6 & 3 & 4 & 1 & 2\\
    8 & 8 & 7 & 6 & 5 & 4 & 3 & 2 & 1
  \end{tabular}
\end{center}
\end{table}

It is significant to note here that with this ordering of generators,
the multiplication table of the groups of order~4 comprise the first
four rows/columns of the multiplication table of the group of order~8,
as is clear from inspection of Table~\ref{tab:mult},
and the multiplication table for the groups of order~2 form the first
2-by-2 subarray.  This is an important feature of symmetry handling
internal to Dalton: it is not significant to a user, but it is
exploited heavily inside the code.

We can now see how Dalton handles the symmetry input, including the
case of automatic symmetry detection.  First, we look for planes of
symmetry.  In part this is historical: the original MOLECULE code
benefitted most, in terms of reduction of numerical operations, from
reflection planes, followed by rotational axes, followed by the
``least useful'' case of the inversion.  If we, as a user, know that
the molecule we are interested in has $D_{2h}$ symmetry, we can
specify this explicitly.  The most straightfoward way to do this is in
terms of the generators that correspond to reflections: in particular,
we specify \verb|X Y Z| as our generators.  That is, we assert that
reflection in the $yz$~plane, followed by reflection in the
$xz$~plane, followed by reflection in the $xy$~plane are our
generators $G_1$, $G_2$, $G_3$.  This also illustrates how
generators/symmetry operations are specified in the input to Dalton:
they indicate \emph{which coordinates change sign} under a particular
operation.  Thus \verb|X| is the operation that changes the sign of
the $x$ coordinate (only): that is, reflection in the $yz$-plane.
Similarly, \verb|XY| changes the sign of both the $x$ and $y$
coordinates: this thus denotes two-fold rotation around the $z$~axis.  Finally,
\verb|XYZ| changes the sign of all coordinates, corresponding to
inversion through the coordinate origin.

It is important to emphasize two key points here.  First, the user
must specify the \emph{minimal} number of generators required to
define the group.  That is, it is an error to specify three
operations, \verb|X Y XY|, say, to try to use the group $C_{2v}$,
because the third operation is redundant.  Second, the ordering of the
generators will affect the order in which the irreducible
representations appear.  It is perfectly legitimate in principle to
define $D_{2h}$ using the generators \verb|Z Y X|, but as we now
discuss, there is a set of conventions both for orienting molecules
that have symmetry, and for defining appropriate generators for their
point groups.  We repeat that it is not necessary to follow these
conventions to obtain correct results, but it is essential then to
take great care in comparing with other calculations and with
experimental results.

There is a set of conventions for the various point groups,
largely consistent with those proposed many years ago by
Mulliken~\cite{unk55} (note that this article was published
anonymously but is widely accepted to be Mulliken's work).  See also
Hurley~\cite{Hur76}.  In
particular, for $C_s$ the symmetry plane is conventionally taken as
the $xy$~plane, as it is for $C_{2h}$, so in the latter the $C_2$ axis
is the $z$-axis, and the $z$-axis is also the usual choice for the
symmetry axis in
$C_{2v}$.  We can therefore list the ``conventional'' choices of
generators, and their order, for the eight point groups that Dalton
treats explicitly: again, this is what the program uses internally in
automatic symmetry detection.
\begin{itemize}
  \item Group of order~1
  \begin{itemize}
    \item $C_1$ (no operations)
  \end{itemize}
  \item Groups of order~2
  \begin{itemize}
    \item $C_s$: \verb|Z|
    \item $C_2$: \verb|XY|
    \item $C_i$: \verb|XYZ|
  \end{itemize}
  \item Groups of order~4
    \begin{itemize}
    \item $C_{2v}$: \verb|X Y|
    \item $C_{2h}$: \verb|Z XY|
    \item $D_2$: \verb|XZ YZ|
    \end{itemize}
  \item Group of order~8
    \begin{itemize}
    \item $D_{2h}$: \verb|X Y Z|
    \end{itemize}
\end{itemize}
The conventions in this paragraph specify fully how the
program chooses planes and axes, at least for molecules that have at
most $D_{2h}$ symmetry themselves.  Molecules that have higher
intrinsic symmetry may require special consideration, and this is
taken up in a later section.

As noted above, there is also a historical aspect here: in
Alml{\"o}f's original scheme~\cite{Alm72} reflections were preferred
over rotational axes because they led to fewer nonvanishing
integrals.  This is less of an issue in Dalton, but for many reasons
it is convenient to prefer planes to rotational axes, and in turn over
the inversion, as the symmetry specification.  This is also
the order of preference that Dalton exercises internally when
determining symmetry automatically.

\section{Labelling of irreducible representations}

We can now see why Dalton produces an ordering of irreducible
representations that appears counterintuitive when compared to many
standard character tables.  Let us consider as an example the case of
$C_{2v}$, specified (or determined automatically by the program) by
the operations \verb|X Y|, reflection respectively in the $yz$ and
$xz$ planes.  Using these generators as $G_1$ and $G_2$, the
operations of the point group will be $E$, $G_1$, $G_2$, and $G_1G_2$,
and following the conventions of Table~\ref{tab:ops} this means that
the irreducible representations will appear in the order $A_1$, $B_2$,
$B_1$, and $A_2$, using the Mulliken convention that for a planar
$C_{2v}$ molecule the irreducible representations $A_1$ and $B_2$ are
symmetric with respect with respect to the molecular plane (and $B_1$
and $A_2$ are antisymmetric).  This is why Dalton lists
irreducible representations in the order that it does: the ordering is
derived from the behaviour under the group generators as given in
Table~\ref{tab:ops}.

We emphasize again that it is important to understand that the labelling of
the irreducible representations within Dalton is determined by a
particular choice of molecular orientation.  For example, although
this would be considered unconventional, there is nothing to prevent
someone from labelling the $B_1$ and $B_2$ irreducible representations
opposite from the Mulliken convention, and thus opposite from the
Dalton internal choice.  Similarly, it cannot necessarily be assumed
that Dalton's labelling of the irreducible representations in $D_{2h}$
symmetry is consistent with some paper in the literature --- it is
important to establish what conventions are being used in both cases.

\section{Nuclear coordinates; symmetry-lowering}

If the user specifies the symmetry, it is only necessary to include
atoms that are symmetry-distinct.  That is, if the generators
\verb|X Y| are input to specify the $C_{2v}$ symmetry of the water
molecule, only the coordinates of the oxygen atom and one hydrogen
atom are required.  Indeed, the user should not enter the coordinates
for the second atom, because the program will already have generated
an atom at that point, and the explicit input of the second one would
be treated as an error (two nuclei trying to occupy the same point).

It is obvious that isotopic substitution will lower or even destroy
the symmetry of some systems: HOD has only $C_s$~symmetry.  If the
main interest is, however, the effect of isotopic substitution on the
harmonic vibrational frequencies, it is not necessary to explicitly
lower the symmetry, because the harmonic force constant matrix (in the
Born-Oppenheimer approximation) displays the full symmetry of the
system as it does not depend on the nuclear masses.  Dalton can thus
use the full symmetry of H$_2$O in calculating the force constants,
and the lowering of symmetry by isotopic substitution is handled
internally when computing the vibrational frequencies.  The user
should note that the calculation of vibrationally averaged
\emph{properties} in Dalton is done using a modified geometry, and the
treatment of symmetry here can be considerably more complicated.
In fact, at present such calculations can only be run in $C_1$
symmetry.  Other properties such as the rotational \mbox{$g$-tensor}
are defined with respect to the centre of mass of the molecule, and
again the symmetry of the actual nuclear framework for a given case
may have to be considered.

Another situation which may yield lower symmetry than the nuclear
framework displays is when an external perturbation, such as an
electric field or field gradient, is included.  For example, imposing
an electric field in the $z$~direction will necessarily destroy
symmetries such as inversion, reflection in the $xy$~plane, and $C_2$
operations around the $x$ and/or $y$~axes.  Such symmetry lowering is
not accounted for automatically in Dalton: the program uses the
symmetry as specified by the user.  It is therefore necessary for the
user to analyze what symmetry remains, and to specify that using the
group generators, when including external perturbations.  Perhaps a
more elegant way to state this is that one should look at the symmetry
of the total Hamiltonian, that is, the original Born-Oppenheimer
Hamiltonian plus the various perturbations.

\section{Treatment of higher symmetries}

It is not uncommon that symmetrical molecules have a higher symmetry
than that treated explicitly by Dalton.  Common examples include
linear symmetry and higher-order (than~2) axes; the cubic groups
(tetrahedral symmetry, etc.) may also be encountered.  We give here a
few conventions for treating these systems, as well as the behaviour
of different one-electron basis functions and many-electron states
within $D_{2h}$ and its subgroups, and some tactics for treating these
higher symmetries.

In general, it is preferable to use the $z$-axis as the principal axis
for molecules belonging to groups with a single higher-order axis.
This is consistent with most group theory texts, for example.  Most
users will already be aware that in order to obtain the desired
symmetry among molecular orbitals, property values, etc., it may be
necessary both to specify the geometry to very high precision (in
order to ensure factors of, say, $\sqrt{3}$ or other irrationals are
handled properly), and to ensure that the calculation is converged
adequately --- the default convergence thresholds in Dalton are
tighter than most programs, but are still not necessarily stringent
enough to recover the full symmetry in some cases.  There is a
difference of opinion about the positioning of atomic centres in some
situations: for example, when there is a four-fold axis.  There are
advantages in the symmetry analysis both for positioning atoms on the
$xz$ and $yz$ planes, and for positioning them between these planes.
The latter convention seems more common, but this is up to the user.

For linear molecules the use of the $z$-axis as the principal axis is
universal, and if (and only if!) the conventions of the previous
section are followed for specifying $C_{2v}$ symmetry, or $D_{2h}$
symmetry for a centrosymmetric system, we can list the properties of
various linear system irreducible representations within the Dalton
symmetries.  Note that we give here the Dalton symmetry
\emph{ordering}: we do not use the (say) $D_{2h}$ labels.  In the case
of $C_{\infty v}$ the correspondence is that the $D_{2h}$ symmetries
that appear as 5, 6, 7, and 8 map respectively onto
symmetries 1 through 4 in $C_{2v}$, so our tabulation below can be
used for both cases.  For $D_{\infty h}$ we have the
following behaviour of one-electron functions:
\begin{itemize}
\item $\sigma_g$: 1
\item $\sigma_u$: 5
\item $\pi_g$: 6, 7
\item $\pi_u$: 2, 3
\item $\delta_g$: 1, 4
\item $\delta_u$: 5, 8
\item $\phi_g$: 6, 7
\item $\phi_u$: 2, 3
\end{itemize}
and the behaviour of higher angular momentum functions should be
obvious from this.  In the linear symmetry case there are also the
$\Sigma^-$ irreducible representations, for which there are no
one-electron basis functions (basis functions for these symmetries
cannot be represented using the coordinates of only one electron).
These fall in the Dalton symmetries 4 and 8, depending on whether,
given a centrosymmetric system, the desired $\Sigma^-$ state is even
or odd ($g$ or $u$) under inversion.

Atoms present additional problems, despite their geometrical
simplicity!  An atomic state can be even or odd under inversion, and
this creates some complications.  Specifically, for a
single-configuration orbital occupation, the parity is determined by
$\sum_i l_i$, where the sum is over all unpaired electrons and $l_i$
is the orbital angular momentum of the $i$th electron.  Hence the
$2s^22p^3$ occupation of the nitrogen atom gives rise to three
\emph{odd parity} states ${}^4S_u$, ${}^2D_u$, and ${}^2P_u$, whereas
the $2s^22p^2$ occupation of carbon yields even parity ${}^3P_g$,
${}^1D_g$, and ${}^1S_g$ states.  The parity is usually not indicated
in atomic state labels, but it is important here: the ground state of
the nitrogen atom appears in symmetry~8 for the usual specification of
$D_{2h}$ using the generators \verb|X Y Z|,
and will never be in symmetry~1, irrespective of how the
generators are specified, unless the calculation is explicitly
directed to use no symmetry!  A list of the various angular momenta and
parities is given here for the ``standard'' $D_{2h}$ specification
with generators \verb|X Y Z|.
\begin{itemize}
\item $S_g$: 1
\item $S_u$: 8
\item $P_g$: 4, 6, 7
\item $P_u$: 2, 3, 5
\item $D_g$: 1, 1, 4, 6, 7
\item $D_u$: 2, 3, 5, 5, 8
\item $F_g$: 1, 4, 4, 6, 6, 7, 7
\item $F_u$: 2, 2, 3, 3, 5, 5, 8
\item $G_g$: 1, 1, 1, 4, 4, 6, 6, 7, 7
\item $G_u$: 2, 2, 3, 3, 5, 5, 5, 8, 8
\end{itemize}
The pattern here should again be obvious for extension to higher total
angular momenta.  There are reasons (computational, not theoretical),
by the way, for using symmetry~1, when possible, for states derived
from $d^n$ occupations, as discussed by Hay~\cite{Hay77} and by Walch
and Bauschlicher~\cite{Wal83}.

The cubic groups create interesting issues of their own.  The naive
tendency for treating, e.g., methane is often to use the $C_{2v}$
subgroup of $T_d$.  In order to ensure exact tetrahedral symmetry this
requires the user to input the geometry to very high accuracy because
of the factors of $\sqrt{3}$.  A better solution when exact
tetrahedral symmetry is required is to use $D_2$, with generators
\verb|XZ YZ|, as this generates four equivalent hydrogens, and thus
exact tetrahedral symmetry, irrespective of how accurately the
position of the symmetry-distinct hydrogen is specified as long as the
(absolute values of) the $x$, $y$, and $z$~coordinates of that
distinct hydrogen are exactly the same.

The \verb|SIRIUS| code within Dalton can recognize and exploit higher
symmetries automatically for SCF, MCSCF, and MP2 calculations, and for
response calculations invoked through the \verb|.RUN RESPONSE| command
in the general input section: one supplies the keyword \verb|.SUPSYM|
in the \verb|*ORBITAL INPUT| section.  Automatic detection of higher
symmetry requires that the geometry displays that symmetry to within a
threshold that can be adjusted (see the manual section on this
keyword): as noted above this may require care in specifying the
coordinate input.  Note that calculations run via the \verb|ABACUS|
code within Dalton cannot utilize higher symmetries, and thus the
\verb|.SUPSYM| keyword is not compatible with runs in which the
general input \verb|**PROPERTIES| is used, thereby invoking
\verb|ABACUS|.

This by no means exhausts the complexities and issues associated with
symmetry that may confront a new user when using the Dalton program.
For example, the use in some situations of the point group $D_2$ may
cause difficulties, typically when the point group element $G_1G_2$
leaves an atom fixed that was transformed into a new centre under
$G_1$ and/or $G_2$.
These caveats notwithstanding, the capability of using symmetry to
reduce the computational labour in a given calculation (a reduction by
a factor of between $g$ and $g^2$, where $g$ is the order of the point
group Dalton uses, and where the exponent varies according to the type
of calculation) and to ensure that Dalton finds a wave function that
has the particular desired symmetry properties, is a very powerful
one.  Complications and issues with using symmetry will normally find
a ready response on the Dalton mailing list.  And it is appropriate to
close this little exposition by again referring to the
analysis~\cite{Alm72} of the
late Jan Alml{\"o}f, who laid the foundations for the implementation
of symmetry in Dalton and other programs in terms of elementary bit
operations such as \verb|AND|, \verb|OR|, and \verb|XOR|.  The
underlying mathematics is treated at length in the language of group
theory in the paper by Davidson~\cite{Dav75a}, although a more pedagogical and
accessible treatment, including the handling of property operators,
gradients, etc., is perhaps available in the work of
Taylor~\cite{Tay92}.  The latter article is reprinted every two years
in the lecture notes for the European Summerschool in Quantum
Chemistry.

% radovan: for the moment taken out until we have a decision whether
%          we want explicit author information and acknowledgements
%J{\'o}gvan Magnus Haugaard Olsen and Kenneth Ruud
%read an earlier
%draft of this document and made a number of helpful suggestions that
%have been incorporated.
