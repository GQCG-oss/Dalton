\section*{Appendix: DALTON Tool box}

This appendix describes the pre- and postprocessing programs
supplied to us by various users and authors. The programs are designed
to create files directly usable by the \dalton\ program, or do
various final analyses of one or more \dalton\ output files. We
strongly encourage users to supply us with any such programs they have
written in connection with their own use of the \dalton\ program,
which we will be glad to distribute along with the \dalton\
program.

This appendix gives a short description of the programs supplied with
the current distribution of the program, with proper references and a
short description of the use of the program.

Executable versions of the programs in this directory will
automatically be produced during installation of the \dalton\ program
and placed in the
\verb|build/tools| directory.

\subsection*{A1: FChk2HES}

\noindent
{\large\bf Author: \normalsize\large Gilbert Hangartner,
Universit\"{a}t Freiburg, Switzerland}

\smallskip

\noindent
{\bf Purpose:} Reads the formatted checkpoint file of Gaussian
     and creates a \verb|DALTON.HES| file with Hessian and atomic coordinates
     as well as a dummy \verb|MOLECULE.INP| file.

\smallskip
\noindent
{\bf Commandline:} FChk2HES [-n] [filename]

-n does not read and write geometry; filename is optional, default is
"\verb|Test.FChk|" .

\smallskip
\noindent
{\bf Comments:}     The Gaussian checkpoint file "Test.FChk" is created
     with the keyword "FormCheck=ForceCart", and where the keyword "NoSymm"
     has  been specified

If symmetry is used: Hessian\index{Hessian} will be in standard(Gaussian) orientation,
        but geometry will be in input orientation. So do not use the
        \verb|MOLECULE.INP| file from this utility,
        neither the coordinates in the end of \verb|DALTON.HES|.
        Use either the standard-orientation from Gaussian output file, or
        use the converted checkpoint file (created with the command
        "formchk checkpointfilename convertedfilename") as an input for
        this program. Anyway, this will not lead to the geometry you
        specified in the input, and the Hessian will be incompatible with the
        calculation you want to do in \dalton. (If this was already carried
        out at the input orientation ...)

The only way to get the geometry orientation you want is to turn
        off symmetry.
        Then, both the Hessian and the geometry in the Test.FChk file are
        in the input orientation, and everything is fine.

The program has been made for the purpose of being used in VCD or
VROA\index{VCD}\index{ROA}\index{vibrational circular
dichroism}\index{Raman optical activity}
calculations where it may be of interest to compare predicted spectra
obtained from SCF London orbital invariants/AATs\index{atomic axial
tensor}\index{AAT} with a MP2 or DFT force field.

\subsection*{A2: labread}

\noindent
{\large\bf Author: \normalsize\large Hans J\o rgen Aa. Jensen,
Odense University, Denmark}

\smallskip

\noindent
{\bf Purpose:} Reads unformatted AO-integral files and prints the
INTEGRAL labels found on the file

\smallskip
\noindent
{\bf Commandline:} labread $<$ infile $>$ outfile

\smallskip
\noindent
{\bf Comments:}  A convenient tool in connection with \dalton\ errors
connected to missing labels on a given file in order to check that the
given integrals do indeed, or do not, exist on a given file (usually
\verb|AOPROPER|).

\subsection*{A3: ODCPRG}

\noindent
{\large\bf Author: \normalsize\large Antonio Rizzo (Istituto di
Chimica Quantistica ed Energetica Molecolare, Pisa, Italy), and
Sonia Coriani (University of Trieste, Italy)}

\smallskip

\noindent
{\bf Purpose:} Analyze the magnetizability and nuclear shielding
polarizabilities from a set of finite-field magnetizability and
nuclear shielding calculations.

\smallskip
\noindent
{\bf Commandline:} odcprog

Requires the existence of a readmg.dat file, and $n$ \verb|DALTON.CM| files,
where $n$ is the number of finite field output files.

\smallskip
\noindent
{\bf readmg.dat:}

\begin{verbatim}
TITLE (1 line)
N ZPRT IPG ISSYM NIST NLAST NSTEP
FILE1
FILE2
...
FILEN
\end{verbatim}
where
\begin{itemize}
\item[N] Number of output DALTON.CM files for the FF calculations
\item[ZPRT]  Sets the print level (T for maximum print level, F otherwise)
\item[IPG] Point group (1=Td, 2=Civ, 3=D2h, 4=C2v, 5=C3v, 6=Dih)
\item[ISSYM] Site symmetry of the atom for which the shielding
               polarizabilities are required
\item[NIST] First atom of which shielding polarizabilities must be computed
               (of those in  the \verb|DALTON.CM| files)
\item[NLAST] Last atom of which shielding polarizabilities must be computed
               (of those in  the \verb|DALTON.CM| files)
\item[NSTEP] Step to go from nist to nlast in the do-loop for
               shielding polarizabilities
\item[FILEn] $n$th \verb|DALTON.CM| file (current name and location)
\end{itemize}

\smallskip
\noindent
{\bf Comments:}
    Only 6 point groups are presently implemented, of which
    C3v only for shielding-polarizabilities and
    Dih only for hyperpolarizabilities

    When \verb|issym.ne.ipg| and both
    hypermagnetizabilities\index{magnetizability polarizability}\index{shielding polarizability}
    and shielding
    polarizabilities are required, the number of field set-ups should
    be equal to the one for shieldings

    Different field set-ups are needed according to
    molecular  and/or nuclear site symmetries
    See for reference Raynes and Ratcliffe~\cite{wtrrrmp37}.
    Linear molecules along the Z axis, planar molecules
    on XZ plane. In general, follow standard point
    symmetry arrangements

\begin{itemize}
\item[T$_d$]   Eg. CH$_4$ - No symmetry - 3 calculations

\begin{verbatim}
Need     0  0  0
Need     0  0  z
Need     x  0  z
\end{verbatim}

\item[C$_{\infty v}$] Eg. CO - No symmetry - 5 calculations

\begin{verbatim}
Need     0  0  0
Need     x  0  0
Need     0  0  z
Need     0  0 -z
Need     x  0  z
\end{verbatim}

\item[D$_{2h}$] Eg. C$_2$H$_4$ -  No symmetry - 7 calculations

\begin{verbatim}
Need     0  0  0
Need     0  0  z
Need     x  0  0
Need     0  y  0
Need     x  y  0
Need     x  0  z
Need     0  y  z
\end{verbatim}

\item[C$_{2v}$] Eg. H$_2$O - No symmetry - 8 calculations

\begin{verbatim}
Need     0  0  0
Need     0  0  z
Need     0  0 -z
Need     x  0  0
Need     0  y  0
Need     x  y  0
Need     x  0  z
Need     0  y  z
\end{verbatim}

\item[C$_{3v}$] Eq. Shielding H1 in CH$_4$

\begin{verbatim}
Need     0  0  0
Need     0  0  z
Need     0  0 -z
Need     x  0  0
Need    -x  0  0
Need     x  0  z
\end{verbatim}

\item[D$_{\infty h}$] Eg. N$_2$ -  4 calculations

\begin{verbatim}
Need     0  0  0
Need     0  0  z
Need     x  0  0
Need     x  0  z
\end{verbatim}
\end{itemize}

\subsection*{A4: xyz2dalton, distances, and aces2dalton}

This documents the use of three small utilities (\verb|xyz2dalton|,
\verb|distances|, and \verb|aces2dalton|) that are written by
Peter Robert Taylor and distributed with
Dalton2013.  Two are designed to assist in the preparation of Dalton
input files; one is less specifically Dalton-oriented but may be
useful in a variety of 
contexts when setting up calculations.  All three programs run as
filters in traditional UNIX style, that is, they are invoked as

\noindent
\verb|program <input_file >output_file|

\noindent
and by default would read from standard input (the keyboard, which is
almost certainly not what you want!) and write to standard output (the
screen, which is probably not what you want either\ldots).  Error
messages are written to standard error.  The 
first few source lines of each program are fairly detailed explanations of
how to use them.  

\verb|xyz2dalton| reads an XYZ coordinate file from, e.g., the PDB,
and produces a complete-except-for-the-basis 
Dalton mol file (with coordinates in atomic units).  There is no
default basis choice and so the mol file generated is not ``correct''
until the user edits it to provide a basis choice of his/her own.
The program currently writes "Choose basis here!" in \verb|output_file|
where the basis should be specified.  This also implicitly assumes the
user will make a global basis choice: for \verb|ATOMBASIS| inputs the
user will need to edit the mol file further.  The program is invoked
as

\noindent
\verb|xyz2dalton [-c] <input_file >output_file|

\noindent
where including the optional argument \verb|-c| causes the geometry
origin to be translated to the molecular centre of mass (using IUPAC
atomic weights for the most abundant isotope of each element).

\verb|distances| reads an XYZ file and calculates a list of
internuclear distances. 
This is probably of limited interest to users but can be helpful to
identify neighbouring atoms: something that is not always easy with
XYZ files.  Command-line arguments can be specified to restrict both the
minimum distances and maximum distances that appear in the output
file, so the program can be invoked (there should be no line break in
the command line here: the break shown is purely for typesetting
convenience) as 

\noindent
\verb|distances [--min lower_bound] [--max upper_bound]|

\noindent
\verb|         < XYZinputfile > Distancesfile|

\noindent
where only distances between \verb|lower_bound| and \verb|upper_bound| will
appear in the output file.  The default values for the bounds are
respectively zero and infinity, so by default all distances are printed.

The limitations on dimensioning for the numbers of nuclear centres in
the codes (\verb|xyz2dalton| and \verb|distances|) are determined in large part by 
the desire to ensure that every atom is numbered differently.  This means
single letter atoms can go to 999 (the first carbon, say, will be labelled
C001) but two-letter atoms only to 99 (the first helium will be He01).

There are some idiosyncrasies that result from handling XYZ files.
There is no convention/require\-ment that all atoms of a given
type will be grouped together in the XYZ file.  For many reasons,
however, it is desirable to have a Dalton
mol file in which each atom type occurs only once, and this is how the
coordinate information is eventually
written to stdout.  The atom types are listed by decreasing atomic number.

\verb|aces2dalton| reads a file output in ACES/C4 library format from
the EMSL 
basis set library server.  The EMSL server can
export a variety of different formats, including a Dalton format.
However, the EMSL developers insist on only one output format for each
program they export for, and there are several different Dalton
formats, related to the need to handle different exponent sizes, and
the Dalton format EMSL produces is not always convenient/ideal.  With
\verb|aces2dalton| a user can produce a library file in Dalton format
for a desired basis set, for any or all atoms that set is available for.

Specify the desired basis sets to be displayed by the EMSL server in
ACES/C4 format, remember to ensure "Optimize general contractions" is
checked, and save the page that EMSL generates as a text file from
within the browser window, Such a file can contain basis sets for as
many elements as you like.  After running

\noindent
\verb|aces2dalton <input_file >output_file|

\noindent
where \verb|input_file| is the text file you just saved from the EMSL
webpage, 
move \verb|output_file| to a directory where Dalton will find it
(see the Dalton documentation).
The code will work for angular momenta up to $l=7$ and for all
elements up to copernicium ($Z=112$).
It is assumed that no basis set will contain more than 100 exponents
of a given angular type.
No more than 7 decimal places are possible for any exponent, and
no more than 6 for a contraction coefficient.  For exponents larger
than 1,000,000 only 3 decimal places are available.  It should be
straightforward for a user who wants to change any of these limits by
editing the source: most limits are set in \verb|PARAMETER| statements
in the various \verb|MODULES|.
