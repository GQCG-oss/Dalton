
%%%%%%%%%%%%%%%%%%%%%%%%%%%%%%%%%%%%%%%%%%%%%%%%%%%%%%%%%%%%%%%%%%%
\section{Ground state--excited state three-photon 
transition moments: \Sec{CCTM}}\label{sec:cctm}
%%%%%%%%%%%%%%%%%%%%%%%%%%%%%%%%%%%%%%%%%%%%%%%%%%%%%%%%%%%%%%%%%%%
\index{transition moment!three-photon}
\index{transition moment!third-order}
\index{transition strength!three-photon}
\index{transition strength!third-order}
\index{three-photon!transition moment}

This section describes the calculation of third-order transition
moments and strengths. Three photon transition strengths are defined as
{\small \[
S^{of}_{ABC,DEF}(\omega_1,\omega_2) = \frac{1}{2} 
       \{ M^{ABC}_{of}(-\omega_1,-\omega_2) M^{DEF}_{fo}(\omega_1,\omega_2)
        +[M^{DEF}_{of}(-\omega_1,-\omega_2) M^{ABC}_{fo}(\omega_1,\omega_2)]^\ast\}
\] }
where $M^{ABC}_{of}(\omega_1,\omega_2)$ and $M^{ABC}_{fo}(\omega_1,\omega_2)$
are the left and right three photon transition moments, respectively.
The methodology is implemented for the CC models CCS, CC2 and CCSD.
%Publications reporting results obtained with this module should 
%cite~\cite{Haettig:MULTIPHOTON}.

\begin{center}
\fbox{
\parbox[h][\height][l]{12cm}{
\small
\noindent
{\bf Reference literature:}
\begin{list}{}{}
\item C.~H\"{a}ttig, O.~Christiansen, and P.~J{\o}rgensen \newblock {\em J.~Chem.~Phys.}, {\bf 108},\hspace{0.25em}8331, (1998).
\end{list}
}}
\end{center}

\begin{description}
\item[\Key{DIPOLE}] \verb| |\newline
Calculate the three-photon moments and strengths for all possible
combinations of Cartesian 
components of the electric dipole moment operator (729 combinations).
\item[\Key{OPERAT}] \verb| |\newline
\verb|READ (LUCMD,'(6A)') LABELA, LABELB, LABELC, LABELD, LABELE, LABELF|\newline
\verb|DO WHILE (LABELA(1:1).NE.'.' .AND. LABELA(1:1).NE.'*')|\newline
\verb|   READ (LUCMD,'(6A)') LABELA, LABELB, LABELC, LABELD, LABELE, LABELF|\newline
\verb|ENDDO|\newline   
Select the sextuples of operator labels for which to calculate the three-photon transition
moments. Operator sextuples which do not correspond to symmetry-allowed combinations will be
ignored during the calculation.
\item[\Key{PRINT}] \verb| |\newline
\verb|READ (LUCMD,*) IPRTM|\newline
Read print level. Default is 0.
\item[\Key{SELSTA}] \verb| | \newline
\verb|READ (LUCMD,'(A70)') LABHELP|\newline
\verb|DO WHILE (LABHELP(1:1).NE.'.' .AND. LABHELP(1:1).NE.'*')|\newline
\verb|  READ (LUCMD,*) IXSYM, IXST, FREQB, FREQC|\newline
\verb|END DO| \newline
Select one or more excited states $f$ (among those specified
in \Sec{CCEXCI}), and the laser frequencies.
The symmetry (\verb+IXSYM+) and state number (\verb+IXST+)
within that symmetry are then given,
one pair (\verb|IXSYM,IXST|) per line.
\verb+FREQB+ and \verb+FREQC+ specify the 
laser frequencies $\omega_1$ and $\omega_2$ (in atomic units).


Default is all states specified in \Sec{CCEXCI}, and for each state
both laser frequencies equal to one third of the excitation energy.

%
\item[\Key{THIRDF}]
Set the frequency arguments for the three-photon transition moments
equal to one third of the excitation energy of the (chosen) 
final state $f$. Default, if \verb+FREQB,FREQC+ are not specified 
in \Key{SELSTA}.
%
\end{description}
