%
%%%%%%%%%%%%%%%%%%%%%%%%%%%%%%%%%%%%%%%%%%%%%%%%%%%%%%%%%%%%%%%%%%
\section{R12 methods: \Sec{R12}}\label{sec:r12}
%%%%%%%%%%%%%%%%%%%%%%%%%%%%%%%%%%%%%%%%%%%%%%%%%%%%%%%%%%%%%%%%%%%
%
\index{R12 theory!Coupled Cluster}
\index{R12 theory!MP2-R12 method}
\index{Coupled Cluster!R12 theory}
\index{MP2-R12 method}
\index{cusp correction!R12 theory}

The calculation of MP2-R12 energy corrections is requested. Note that an
integral-direct calculation must be carried out and that the
key \Key{R12} must be specified in the \Sec{*INTEGRALS} section.

\begin{center}
\fbox{ \parbox[h][\height][l]{12cm}{\small\noindent
{\bf Reference literature:}
\begin{list}{}{}
\item W.~Klopper and C.~C.~M.~Samson,
\newblock {\em J.~Chem.~Phys.\/} {\bf 116}, 6397 (2002).
\item C.~C.~M.~Samson, W.~Klopper and T.~Helgaker,
\newblock {\em Comp.~Phys.~Commun.\/} {\bf 149}, 1 (2002).
\end{list} }} \end{center}

\begin{description}

\item[\Key{NO 1}]

Results for Ansatz 1 of the MP2-R12 method are not computed.

\item[\Key{NO 2}]

Results for Ansatz 2 of the MP2-R12 method are not computed.

\item[\Key{NO A}]

Results for approximation A of the MP2-R12 method are not printed.

\item[\Key{NO A'}]

Results for approximation A$^\prime$ of the MP2-R12 method are not printed.

\item[\Key{NO B}]

Results for approximation B are not computed and not printed.

\item[\Key{NO RXR}]

Those extra terms are ignored, which occur in approximation B when an auxiliary basis set
is invoked for the resolution-of-identity (RI) approximation. If they are included, 
the results are marked as B$^\prime$.

\item[\Key{NO HYB}]

The default MP2-R12 calculation implemented in the \dalton\ program
avoids two-electron integrals that involve
two or more basis functions of the auxiliary basis (of course, only
if such a basis is employed). This approach is denoted
as hybrid scheme between approximations A and B. To obtain the
full MP2-R12 energy in approximation B, the keyword \verb|.NO HYB| must
be specified. Then, two-electron integrals with up to two
auxiliary basis functions are calculated but the calculation becomes 
more time-consuming.

\item[\Key{R12DIA}]
The MP2-R12 equations are solved by diagonalizing the matrix representation 
of the Fock operator in the basis of R12 double replacements. If 
negative eigenvalues occur,
a warning is issued. When this happens, the results should not be trusted,
since the RI approximation appears to be insufficiently accurate.
This diagonalizing is the default.

\item[\Key{R12SVD}]
The MP2-R12 equations are solved by single value decomposition
(the use of this keyword is not recommended).

\item[\Key{R12XXL}]

All possible output from the MP2-R12 approach is generated (the 
use of this keyword is not recommended).

\item[\Key{SVDTHR}]  \verb| | \newline
\verb|READ (LUCMD,*) SVDTHR|\newline
Threshold for singular value decomposition (default = $10^{-12}$).

\item[\Key{VCLTHR}]  \verb| | \newline
\verb|READ (LUCMD,*) SVDTHR|\newline
Threshold for neglect of R12 terms (default = 0, neglecting nothing).

\end{description}
