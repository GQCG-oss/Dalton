% checked with "ispell" 18.Sep.2003 /hjaaj
\chapter{\label{ch:wf-guide}Getting the wave function you want}

Currently the following wave function types are implemented in DALTON:

\begin{description}

\item[RHF] Closed-shell singlet and spin-restricted high-spin
Hartree--Fock (SCF, self-consistent
field)\index{HF}\index{SCF}\index{Hartree--Fock}.

\item[DFT] Density functional theory, closed shell and spin-restricted high-spin Kohn--Sham DFT, allowing for both local density functionals (LDA), generalized gradient approximation (GGA) functionals and hybrid functionals.\index{DFT}\index{Density Functional Theory}.

\item[MP2]
\index{MP2}\index{M{\o}ller-Plesset!second-order} 
M{\o}ller-Plesset second-order perturbation theory which is based on
closed-shell restricted Hartree--Fock.  Energy, second-order one-electron
density matrix and first-order wave function (for SOPPA) may be
calculated, as well as first-order geometry optimizations using
numerical gradients. MP2 is also part of CC (see Chapter~\ref{ch:CC}).
First-order properties and geometry optimization are available through
this module.  

\item[CI] Configuration
interaction\index{CI}\index{Configuration Interaction}. Two types of
configuration selection schemes are implemented: the complete active
space model (CAS) and the restricted active space model (RAS).

\item[GASCI] Generalized Active Spaces Configuration
interaction\index{GASCI}\index{Generalized Active Spaces Configuration Interaction}. 
The most general scheme of defining configuration selection schemes (including CAS and RAS).
For more information on how to run this kind of calculation, we refer to
Chapters~\ref{ch:lucitaexamples} and~\ref{ch:lucita}.

\item[CC] Coupled Cluster\index{CC}\index{Coupled Cluster}.
For information on how to run this kind of calculation, we refer to 
Chapters~\ref{ch:ccexamples} and~\ref{ch:CC}.

\item[MCSCF] Multiconfiguration self-consistent field based on
the complete active space model (CASSCF) and
the restricted active space model
(RASSCF)\index{MCSCF}\index{CASSCF}\index{RASSCF}.

\end{description}

SIRIUS is the part of the code that computes all of these
wave functions, with the exception of the coupled cluster wave functions.
This part of the DALTON manual discusses aspects of converging the
wave function you want\index{convergence!wave function}.

\section{\label{sec:ig_necinp} Necessary input to SIRIUS}

In order for DALTON to perform a calculation for a given wave function a
minimum amount of
information is required. This information is collected in the
\Sec{*WAVE FUNCTIONS} input module. Fig.~\ref{fig-nec.inp.} collects the most
essential information.

\begin{figure}
    \newlength{\mpwidth}
    \settowidth{\mpwidth}{\tt M}
    \addtolength{\mpwidth}{65\mpwidth}
\begin{tabular}{|c|}
\hline
\begin{minipage}{\mpwidth}
\begin{verbatim}

   Calculation type(s): RHF, MP2, CI, MCSCF, CC.

   Number of symmetries (needed only for MCSCF wave functions)
   If MCSCF or CI calculation then
      Symmetry of wave function.
      Spin multiplicity of wave function.
   else if one open shell RHF
      Symmetry of open shell, doublet spin multiplicity
   else
      Totally symmetric, singlet spin multiplicity
   end if

   Number of basis orbitals.
   Number of molecular orbitals.

   Number of inactive orbitals in each symmetry.
   If high-spin open shell RHF then
      Number of singly occupied orbitals in each symmetry
   else if MCSCF or CI calculation then
      Number of active electrons.
      If CAS calculation desired then
         Number of active orbitals in each symmetry.
      else if RAS calculation desired then
         Number of active orbitals in RAS1 space in each symmetry.
         Number of active orbitals in RAS2 space in each symmetry.
         Number of active orbitals in RAS3 space in each symmetry.
         Limits on number of electrons in RAS1 and RAS3.
      end if
   end if

   if not MP2 then convergence threshold.

   how to begin the calculation (initial guess).

\end{verbatim}
\end{minipage} \\ \hline
\end{tabular}
\vspace{0.5cm}
\caption{Necessary input information for determining the wave
function.}\label{fig-nec.inp.}
\end{figure}


\section{\label{sec:si-ex} An input example for SIRIUS}

\noindent
The number of symmetries and the number of basis orbitals are read from
the one-electron integral file.  The number of symmetries may also be
specified in the \Sec{*WAVE FUNCTIONS} input module as a way of
obtaining a consistency check between the wave function input section
and available integral files.

\noindent
The following sample input specifies a CASSCF\index{CASSCF}
calculation on water with maximum use of defaults:

\begin{verbatim}
**WAVE FUNCTIONS
.MCSCF              -- specifies that this is an MCSCF calculation
*CONFIGURATION INPUT
.SPIN MULTIPLICITY  -- converge to singlet state
    1
.SYMMETRY           -- converge to state of symmetry 1 (i.e. A1)
    1
.INACTIVE ORBITALS  -- number of doubly occupied orbitals each symmetry
    1    0    0    0
.CAS SPACE          -- selects CAS and defines the active orbitals.
    4    2    2    0
.ELECTRONS          -- number of electrons distributed in the active
    8                  orbitals.
*ORBITAL INPUT
.MOSTART            -- initial orbitals are orbitals from a previous run
 NEWORB                (for example MP2 orbitals)
**(name of next input module)
\end{verbatim}

\noindent
Comparing to the list of necessary input information in
Fig.~\ref{fig-nec.inp.},
the following points may be noted:
\noindent
The number of basis functions are read from the one-electron integral
file. Because no orbitals are going to be deleted, the number of
molecular orbitals will be the same as the number of basis functions.
\noindent
No state has been specified as ground state calculation is the default.
\noindent
The default convergence threshold\index{convergence!threshold} of
1.0D-5 is used.


\clearpage
\section{\label{sec:ig_hints} Hints on the structure of the \Sec{*WAVE FUNCTIONS} input}


The following list attempts to show the interdependence of the various
input modules through indentation
(\kw{*R*}: required input) :

\begin{verbatim}
**WAVE FUNCTIONS
  .TITLE
  .HF
     *SCF INPUT
       .DOUBLY OCCUPIED
       ...
     *ORBITAL INPUT        (2)
     *STEP CONTROL         (if QCSCF)
  .DFT
     Functional name
     *SCF INPUT
       .DOUBLY OCCUPIED
     *DFT INPUT
       ...
     *ORBITAL INPUT        (2)
     *STEP CONTROL         (if QCSCF)
  .MP2
     *MP2 INPUT
     *ORBITAL INPUT        (2)
     *SCF INPUT
       .DOUBLY OCCUPIED
  .CI
     *CONFIGURATION INPUT  (1)
     *ORBITAL INPUT        (2)
     *CI VECTOR            (3)
     *CI INPUT
  .MCSCF
     *CONFIGURATION INPUT  (1)
     *ORBITAL INPUT        (2)
     *CI VECTOR            (3)
     *OPTIMIZATION
     *STEP CONTROL
  .CC
     *CC INPUT
  .STOP
  .RESTART
  .INTERFACE
  .PRINT
*HAMILTONIAN
*TRANSFORMATION
*POPULATION ANALYSIS
*PRINT LEVELS
*AUXILIARY INPUT
\end{verbatim}

%\clearpage
{\bf Notes:}

\begin{verbatim}

(1) *CONFIGURATION INPUT
      .SPIN MULTIPLICITY (*R*)
      .SYMMETRY (*R*)
      .INACTIVE ORBITALS (*R*)

      .ELECTRONS (*R*)
      Either
        .CAS SPACE
      or some or all of
        .RAS1 SPACE
        .RAS2 SPACE
        .RAS3 SPACE
        .RAS1 HOLES or .RAS1 ELECTRONS
        .RAS3 ELECTRONS
      are required.

(2) *ORBITAL INPUT
      .MOSTART   ! default Huckel guess if available, otherwise H1DIAG

      .SYMMETRIC ORTHONORMALIZATION
      .GRAM-SCHMIDT ORTHONORMALIZATION

      .FROZEN CORE ORBITALS
      .FREEZE ORBITALS

      .5D7F9G
      .AO DELETE
      .DELETE

      .REORDER

      .PUNCHINPUTORBITALS
      .PUNCHOUTPUTORBITALS

(3) *CI VECTOR
      One and only one of the following three
        .STARTHDIAGONAL  ! default
        .STARTOLDCI
        .SELECT CONFIGURATION

      .CNO START
\end{verbatim}

\pagebreak[3]
\section{\label{sec:ig_restart} How to restart a wave function calculation}

It is possible to restart\index{restart} after the last finished
macro iteration in (MC)SCF, provided the \verb|SIRIUS.RST|\index{SIRIUS.RST}
file was saved.  You will then at most
loose part of one macro iteration in case of system crashes, disks
running full, or other irregularities. \noindent In general the
only change needed to the wave function input is to add the
\Key{RESTART} keyword under \Sec{*WAVE FUNCTIONS}. However, if the
original job was a multistep calculation involving {\it e.g.\/} SCF,
MP2 and MCSCF wave functions, restarting the MCSCF part requires the
removal of the \quotekw{HF} and \quotekw{MP2} keywords in the input
file. If the correct
two-electron integrals over molecular orbitals are available, you
can skip the integral transformation\index{integral transformation} 
in the beginning of the macro iteration by means
of the \Key{OLD TRANSFORMATION} keyword under
\Sec{TRANSFORMATION}. When resubmitting the job, you must make
sure that the correct \verb|SIRIUS.RST| file is
made available to the job.
%(specify the old "f21" file using
%the format described in Section~\ref{VM-runs} for VM/CMS
%and in Section~\ref{AIX-runs} for AIX).

\noindent
%As done in Example~\ref{ex-6.1.4}, you can also use the restart feature if you
You can also use the restart feature if you find it desirable to
converge\index{convergence!threshold} the wave function sharper
than specified in the original input. By default, SCF wave
functions are converged to 1.0D-6 in the gradient norm, whereas MCSCF
and CI wave functions are converged to 1.0D-5 in the gradient norm. 

\pagebreak[2]
\section{\label{sec:ig_orbtransfer}
Transfer of molecular orbitals between different computers}

In order to be able to transfer molecular orbital coefficients between
different computer systems,
an option is provided for
formatted punching and formatted reading of the molecular orbital
coefficients.  The options are
\begin{center}
   \Key{PUNCHOUTPUTORBITALS} or \Key{PUNCHINPUTORBITALS}
\end{center}
in input group \Sec{ORBITAL INPUT}. The first option is used to
punch the orbitals at the end of the optimization, while the
second option is used to punch some molecular
orbitals\index{molecular orbital} already available on the \verb|SIRIUS.RST|
file, for instance the converged
orbitals from a previous calculation.  The orbitals can then be
transferred to another computer and appended to the SIRIUS input
there. The orbitals may then be read by \dalton\ using the \Key{MOSTART}
followed by \Key{FORM18} option on the next line (with this option
\dalton\ will, after having finished the \Sec{*WAVE FUNCTIONS}
input module, search the input file for either the \Sec{*MOLORB}
keyword or the \Sec{*NATORB} keyword and expect the orbital
coefficients to follow immediately afterwards).


\section{\label{examples} Wave function input examples}

Those who are used to the old {\sc sirius} input will not
experience any dramatic changes with respect to the DALTON input,
although a few minor differences exist in order to create a more
user-friendly environment. The input sections are the same as
before starting with a wave function specification, currently,
including options for SCF, DFT, MP2, MCSCF, and CC reference states.
Depending on this choice of reference state the following input
sections take different form, from the simplest SCF input to the
more complex MCSCF input. Minimal SCF and SCF+MP2\index{HF}\index{Hartree--Fock}\index{SCF}\index{MP2}\index{M{\o}ller-Plesset!second-order}
using the new feature of generating the HF
occupation on the basis of an initial H\"{u}ckel
calculation\index{H\"{u}ckel} and then possibly change the
occupation during the first DIIS\index{DIIS} iterations will be
shown. There will also be an example showing how an old
\verb|SIRIUS.RST| file is used for restart in the
input. \index{SIRIUS.RST}\index{restart}
In this case the Hartree--Fock occupation will be read from
the file \verb|SIRIUS.RST| and used as initial Hartree--Fock
occupation\index{HF occupation}\index{Hartree--Fock occupation}.
The HF occupation is
needed for an MCSCF calculation\index{MCSCF}, as this is anyway
determined when establishing a suitably chosen active
space\index{active space}. An example of an
MCSCF-CAS\index{CASSCF} input without starting
orbitals\index{starting orbitals} will be given, as well as an
MCSCF-RAS\index{RASSCF} with starting orbitals. We note that an
MCSCF wave function by default will be optimized using
spin-adapted configurations
(CSFs)\index{CSF}\index{configuration state function}.
%unless the wave function optimization is followed
%by a calculation of molecular properties or geometry optimization,
%when determinants\index{determinants} will be used. 
The three
following examples illustrate calculations on and excited
state\index{excited state}, on a core-hole\index{core hole} state
with frozen core orbital\index{frozen core} and on a core-hole
state with relaxed core\index{relaxed core} orbital. The last
example shows an MCSCF calculation of non-equilibrium solvation
energy\index{non-equilibrium solvation}. For examples on how to run CC
wave functions, we refer to Chapter~\ref{ch:ccexamples}.

\bigskip

The following input example gives a minimal
SCF\index{SCF}\index{HF}\index{Hartree--Fock} input with default
starting  orbitals (that is, H\"{u}ckel guess), and automatic Hartree--Fock
occupation, first based on the H\"{u}ckel\index{H\"{u}ckel} guess, and
then updated during the DIIS\index{DIIS} iterations:

\begin{verbatim}
**DALTON INPUT
.RUN WAVE FUNCTIONS
**WAVE FUNCTIONS
.HF
**END OF DALTON INPUT
\end{verbatim}
\label{sirius_ex1a}

The following input give a minimal input for an
MP2\index{MP2}\index{M{\o}ller-Plesset!second-order}
calculation using all default settings for the Hartree--Fock
calculation (see previous example):

\begin{verbatim}
**DALTON INPUT
.RUN WAVE FUNCTIONS
**WAVE FUNCTIONS
.HF
.MP2
**END OF DALTON INPUT
\end{verbatim}
\label{sirius_ex1b}

If we would like to calculate molecular properties in several
geometries, we may take advantage of the fact the molecular
orbitals at the previous geometry probably is quite close to the
optimized MOs at the new geometry, and thus restart\index{restart}
from the MOs contained in the \verb|SIRIUS.RST|\index{SIRIUS.RST}
file, as indicated in the following example:

\begin{verbatim}
**DALTON INPUT
.RUN WAVE FUNCTIONS
**WAVE FUNCTIONS
.HF
*ORBITAL INPUT
.MOSTART
 NEWORB
**END OF DALTON INPUT
\end{verbatim}
\label{sirius_ex2}

Running DFT~\index{DFT} calculations is very similar to Hartree-Fock, except for
the specification of the functional: The \Key{DFT} keyword is followed by a
line describing the functional to be used. The simple input below will
start a B3LYP geometry optimization:
\begin{verbatim}
**DALTON INPUT
.OPTIMIZE
**WAVE FUNCTIONS
.DFT
 B3LYP
**END OF DALTON INPUT
\end{verbatim}

We finish this list of examples with a high-spin open-shell DFT input.
We need to specify the number of singly as well as doubly occupied orbitals 
in each symmetry in the \Sec{*SCF INPUT} section using the \Key{SINGLY OCCUPIED}
and \Key{DOUBLY OCCUPIED} keywords respectively
(here assuming $D_{2h}$ symmetry):
\begin{verbatim}
**DALTON
.RUN WAVE FUNCTIONS
**WAVE FUNCTION
.DFT
 B3LYP
*SCF INPUT
.DOUBLY OCCUPIED
 3 1 1 0 2 0 0 0
.SINGLY OCCUPIED
 0 0 0 0 0 1 1 0
**END OF DALTON INPUT
\end{verbatim}

A complete list of available functionals can be found in the Reference Guide, see Sec.~\ref{ref-dft}

\begin{center}
\fbox{
\parbox[h][\height][l]{12cm}{
\small
\noindent
{\bf Reference literature:}
\begin{list}{}{}
\item Restricted-step second-order MCSCF: H.J.Aa.Jensen, and H.\AA
gren. \newblock {\em Chem.Phys.Lett.} {\bf 110},\hspace{0.25em}140,
(1984).
\item Restricted-step second-order MCSCF: H.J.Aa.Jensen, and H.\AA
gren. \newblock {\em Chem.Phys.} {\bf 104},\hspace{0.25em}229,
(1986).
\item MP2 natural orbital occupation analysis:
H.J.Aa.Jensen, P.J\o rgensen, H.\AA gren, and J.Olsen. \newblock
{\em J.Chem.Phys.} {\bf 88},\hspace{0.25em}3834, (1988);
{\bf 89},\hspace{0.25em}5354, (1988).
\end{list}
}}
\end{center}

The next input example gives the necessary input for a Complete
Active Space SCF (CASSCF)\index{CASSCF} calculation where we use MP2 to provide
starting orbitals\index{MP2}\index{M{\o}ller-Plesset!second-order}\index{starting orbitals}
for our MCSCF. The
active space may for instance be
chosen on the basis of an MP2 natural orbital occupation
analysis\index{active space} as
described in Ref.~\cite{hjajpjhajojcp88}. The input would be:

\begin{verbatim}
**DALTON INPUT
.RUN WAVE FUNCTIONS
**WAVE FUNCTIONS
.HF
.MP2
.MCSCF
*SCF INPUT
.DOUBLY OCCUPIED
 2 0 0 0
*CONFIGURATION INPUT
.SYMMETRY
 1
.SPIN MULTIPLICITY
 1
.INACTIVE ORBITALS
 0 0 0 0
.ELECTRONS (active)
 4
.CAS SPACE
 6 4 4 0
**END OF DALTON INPUT
\end{verbatim}
\label{sirius_ex3}

As for Hartree--Fock calculation, we may want to use available
molecular orbitals on the \verb|SIRIUS.RST|\index{SIRIUS.RST} file
from previous 
calculations as starting orbitals\index{starting orbitals} for our
MCSCF\index{RASSCF} as indicated in this input example:

\begin{verbatim}
**DALTON INPUT
.RUN WAVE FUNCTIONS
**WAVE FUNCTIONS
.MCSCF
*ORBITAL INPUT
.MOSTART
 NEWORB
*CONFIGURATION INPUT
.SYMMETRY
 1
.SPIN MULTIPLICITY
 1
.INACTIVE ORBITALS
 0 0 0 0
.ELECTRONS (active)
 4
.RAS1 SPACE
 2 1 1 0
.RAS2 SPACE
 2 2 2 0
.RAS3 SPACE
 6 4 4 2
.RAS1 ELECTRONS
 0 2
.RAS3 ELECTRONS
 0 2
*OPTIMIZATION
.TRACI
.FOCKONLY
**END OF DALTON INPUT
\end{verbatim}
\label{sirius_ex4}

\begin{center}
\fbox{
\parbox[h][\height][l]{12cm}{
\small
\noindent
{\bf Reference literature:}
\begin{list}{}{}
\item Optimal orbital trial vectors: H.J.Aa.Jensen, P.J\o rgensen,
and H.\AA gren. \newblock {\em J.Chem.Phys.}, {\bf
87},\hspace{0.25em}451, (1987).
\item Excited state geometry optimizations: A.Cesar, H.\AA gren,
T.Helgaker, P.J\o rgensen, and H.J.Aa.Jensen. \newblock {\em
J.Chem.Phys.}, {\bf 95},\hspace{0.25em}5906, (1991).
\end{list}
}}
\end{center}

The next input describes the optimization of the first excited
state\index{excited state}
of the same symmetry as the ground state. To speed up convergence, we
employ optimal orbital\index{optimal orbital trial vector} trial
vectors as described in
Ref.~\cite{hjajpjhajcp87}. Such an input would look like:

\begin{verbatim}
**DALTON INPUT
.RUN WAVE FUNCTIONS
**WAVE FUNCTIONS
.TITLE
4 2 2 0 CAS on first excited 1A_1 state, converging to 1.D-07
.MCSCF
*CONFIGURATION INPUT
.SYMMETRY
  1                 | same symmetry as ground state
.SPIN MULTIPLICITY
  1
.INACTIVE ORBITALS
 1 0 0 0
.ELECTRONS (active)
 8
.CAS SPACE
 4 2 2 0
*ORBITAL INPUT
.MOSTART
 NEWORB
*CI VECTOR
.STARTHDIAGONAL    | compute start vector from Hessian CI-diagonal
*OPTIMIZATION
.THRESH
 1.D-07
.SIMULTANEOUS ROOTS
  2 2
.STATE
  2                 | 2 since the first exited state has the same symmetry
!                     as the ground state
.OPTIMAL ORBITAL TRIAL VECTORS
*PRINT LEVELS
.PRINTUNITS
 6 6
.PRINTLEVELS
 5 5
**END OF DALTON INPUT
\end{verbatim}
\label{sirius_ex5}

\begin{center}
\fbox{
\parbox[h][\height][l]{12cm}{
\small
\noindent
{\bf Reference literature:}
\begin{list}{}{}
\item Core hole: H.\AA gren, and H.J.Aa.Jensen. \newblock {\em Chem.Phys.Lett.}, {\bf 137},\hspace{0.25em}431, (1987).
\item Core hole: H.\AA gren, and H.J.Aa.Jensen. \newblock {\em
Chem.Phys.}, {\bf 172},\hspace{0.25em}45, (1993).
\end{list}
}}
\end{center}

The next input describes the calculation of a core-hole\index{core hole}
state of the carbon 1s orbital in carbon monoxide, the first
example employing a frozen core\index{frozen core}:

\begin{verbatim}
**DALTON INPUT
.RUN WAVE FUNCTIONS
**WAVE FUNCTIONS
.TITLE
C1s core hole state of CO, 4 2 2 0 valence CAS + C1s.
Frozen core orbital calculation.
.MCSCF
*CONFIGURATION INPUT
.SYMMETRY
  1
.SPIN MULTIPLICITY
  2                | Doublet spin symmetry because of opened core orbital
.INACTIVE ORBITALS
 2 0 0 0           | O1s and O2s orbitals are inactive while the
!                    opened core orbital, C1s, must be active
.ELECTRONS (active)
 9                 | All valence electrons plus the core hole electron
!                    are active
.RAS1 SPACE
 1 0 0 0           | The opened core orbital (NOTE: always only this orb.)
.RAS1 ELECTRONS
 1 1               | We impose single occupancy for the opened core orbital
.RAS2 SPACE
 4 2 2 0           | Same as the CAS space in the ground state calculation
*OPTIMIZATION
.COREHOLE
 1 2               | Symmetry of the core orbital and the orbital in this
!                    symmetry with the core hole according to list of input
!                    orbitals. The same thing could be obtained by
!                    reordering the core orbital to the first active orb.
!                    (by .REORDER), and specifying .FREEZE and .NEO ALWAYS
.TRACI
*ORBITAL INPUT
.MOSTART
 NEWORB            | Start from corresponding MCSCF ground state
*CI VECTOR
.STARTHDIAGONAL
**END OF DALTON INPUT
\end{verbatim}
\label{sirius_ex6}


whereas we in a calculation where we would allow the core to
relax\index{relaxed core} only would require the following changes
compared to the previous input, assuming that we start out from
orbitals and CI vectors generated by the previous calculation
\Key{STARTHDIAGONAL} is therefore replaced by \Key{STARTOLDCI}, and
the core orbital has number 3 (instead of 1) in the list of
orbitals in the first symmetry. "\Key{CORERELAX}" is specified for
relaxation of the core orbital using the NR algorithm.

\begin{verbatim}
*OPTIMIZATION
.COREHOLE
 1 3
.CORERELAX
*CI VEC
.STARTOLDCI
\end{verbatim}
\label{sirius_ex7}

\begin{center}
\fbox{
\parbox[h][\height][l]{12cm}{
\small
\noindent
{\bf Reference literature:}
\begin{list}{}{}
\item General reference: K.V.Mikkelsen, E.Dalgaard,
P.Svanstr{\o}m. \newblock {\em J.Phys.Chem}, {\bf
91},\hspace{0.25em}3081, (1987).
\item General reference: K.V.Mikkelsen, H.{\AA}gren, H.J.Aa.Jensen,
and T.Helgaker. \newblock {\em J.Chem.Phys.}, {\bf
89},\hspace{0.25em}3086, (1988).
\item Non-equilibrium solvation: K.V.Mikkelsen, A.Cesar, H.{\AA}gren,
H.J.Aa.Jensen.\newblock {\em J.Chem.Phys.}, {\bf
103},\hspace{0.25em}9010, (1995).
\end{list}
}}
\end{center}

This example describes calculations for non-equilibrium
solvation\index{non-equilibrium solvation}, where the mole\-cule will
be enclosed in a spherical cavity. Usually one starts
with a calculation of a reference state (most often the ground
state) with equilibrium solvation, using keyword \Key{INERSFINAL}. The
interface file is then used (without user interference) for a
non-equilibrium excited state calculation; keyword \Key{INERSINITIAL}.

\begin{verbatim}
**DALTON INPUT
.RUN WAVE FUNCTIONS
**WAVE FUNCTIONS
.TITLE
 2-RAS(2p2p') : on F+ (1^D) in Glycol
.MCSCF
*CONFIGURATION INPUT
.SPIN MULTIPLICITY
 1
.SYMMETRY
 1
.INACTIVE ORBITALS
 1  0  0  0  0  0  0  0
.ELECTRONS (active)
 6
.RAS1 SPACE
 0  0  0  0  0  0  0  0
.RAS2 SPACE
 1  2  2  0  2  0  0  0
.RAS3 SPACE
 8  4  4  3  4  3  3  1
.RAS1 ELECTRONS
 0  0
.RAS3 ELECTRONS
 0  2
*OPTIMIZATION
.NEO ALWAYS
.OPTIMAL ORBITAL TRIAL VECTORS
*ORBITAL INPUT
.MOSTART
 NEWORB
*CI VECTOR
.STARTOLDCI
*SOLVENT
.CAVITY
 2.5133D0
.INERSINITIAL     | initial state inertial polarization
 37.7D0  2.050D0  | static and optic dielectric constants for Glycol
.MAX L
 10
.PRINT
 6
**END OF DALTON INPUT
\end{verbatim}
\label{sirius_ex8}
