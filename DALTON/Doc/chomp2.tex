~%%%%%%%%%%%%%%%%%%%%%%%%%%%%%%%%%%%%%%%%%%%%%%%%%%%%%%%%%%%%%%%%%%%
\section{Cholesky based MP2: \Sec{CHOMP2}}
\label{sec:chomp2}
%%%%%%%%%%%%%%%%%%%%%%%%%%%%%%%%%%%%%%%%%%%%%%%%%%%%%%%%%%%%%%%%%%%
\index{MP2}
\index{Cholesky decomposition-based methods}

In this Section, we describe the keywords controlling the algorithm to 
calculate MP2 energies and  second order properties using Cholesky 
decomposed two-electron integrals. The Cholesky MP2 algorithm will be
automatically employed if the keyword \Key{CHOLES} was activated in the
\Sec{*DALTON} input module.  


\begin{center}
\fbox{
\parbox[h][\height][l]{12cm}{
\small
\noindent
{\bf Reference literature:}
\begin{list}{}{}
\item H.~Koch, {A.~M.~J.~S{\'a}nchez~de~Mer{\'a}s}, and T.~B. Pedersen \newblock {\em J.~Chem.~Phys.}, {\bf 118},\hspace{0.25em}9481, (2003).
\end{list}
}}
\end{center}

\begin{description}
\item[\Key{ALGORI}]\verb| |\newline
\verb|READ (LUCMD,*) IALGO|\newline
        Set algorithm: 0 to let the CHOMP2 routine decide according
        to available memory,
        1 to force storage of full-square (ia$\mid$jb) integrals 
        in core, 2 to batch over one virtual index, 3 to batch
        over two virtual indices. (Default: 0).
%
\item[\Key{CHOMO}] 
        Decompose the (ai$\mid$bj) integrals. 
%
\item[\Key{MP2SAV}] 
        Save the MP2 amplitudes on disk. (Default: 
        \verb| .FALSE.|)
%
\item[\Key{MXDECM}] \verb| |\newline
\verb|READ (LUCMD,*) MXDECM|\newline
        Read in the maximum number of qualified diagonals in
        the decomposition of the (ai$\mid$bj) integrals. (Default: 50).
%
\item[\Key{NCHORD}] \verb| |\newline
\verb|READ (LUCMD,*) NCHORD|\newline
        Read in the maximum number of previous vectors
        to be read in each batch. (Default: 200).
%
\item[\Key{NOCHOM}] 
        No decompositions of the (ai$\mid$bj) integrals. But, of course,
        the previously calculated decomposed two-electron AO integrals
        will be used. This is the default.
%
\item[\Key{OLDEN2}] \verb| |\newline
\verb|READ (LUCMD,*) OLDEN2|\newline
        In a restarted calculation, read in the contribution
        to the MP2 energy from the virtual orbitals treated in
        a previous calculation. This information can be found in
        the file CHOMP2\_RST from the previous 
        calculation. (Default: 0.0D0).
%
\item[\Key{RSTMP2}] \verb| |\newline
\verb|READ (LUCMD,*) IFSYMB, IFVIRB|\newline
        Restart Cholesky MP2 calculation. The next line reads
        the symmetry and the number of the first virtual orbital 
        to be included in the calculation.This information can 
        be gotten in the file CHOMP2\_RST from the previous 
        calculation. 
%
\item[\Key{SKIPCH}] 
        Skip (ai$\mid$bj) decomposition; read info from disk
%
\item[\Key{SKIPTR}] 
        Skip MO transformation; use old vectors.
%
\item[\Key{SPAMP2}] \verb| |\newline
\verb|READ (LUCMD,*) SPAMP2|\newline
        Span factor for (ai$\mid$bj) decomposition. (Default: 
        The span factor used in the decomposition of
        the two-electron integrals in tha AO basis).
%
\item[\Key{SPLITM}] \verb| |\newline
\verb|READ (LUCMD,*) SPLITM|\newline
        Weight factor for Cholesky part in memory split for
        batching over virtuals. (Default: 1.0D0).
%
\item[\Key{THRMP2}] \verb| |\newline
\verb|READ (LUCMD,*) THRMP2|\newline
        Threshold for decomposition. (Default: 
        The threshold used in the decomposition of
        the two-electron integrals).
%
\item[\Key{ZERO}] \verb| |\newline
\verb|READ (LUCMD,*) THZMP2|\newline
        Threshold for diagonal zeroing in decompositions. (Default: 
        The same used in the decomposition of
        the two-electron integrals).
%
\end{description}
