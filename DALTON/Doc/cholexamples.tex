%%%%%%%%%%%%%%%%%%%%%%%%%%%%%%%%%%%%%%%%%%%%%%%%%%%%%%%%%%%%%%%%%%%
\chapter{Examples of Cholesky decomposition-based calculations}
\label{ch:choexamples}
%%%%%%%%%%%%%%%%%%%%%%%%%%%%%%%%%%%%%%%%%%%%%%%%%%%%%%%%%%%%%%%%%%%

We present in this Chapter some examples of input files 
\verb|DALTON.INP| for calculations that can be done
with the Dalton program using either decomposed two-electron 
integrals~\cite{choint,chorsp} either decomposed orbital 
energy denominators~\cite{jcp_chopt,ijqc_chopt}.
A few more examples may be found in the test suite. 


\section{Hartree-Fock energy and polarizability}
%
{\choles}  allows for the calculation of Hartree-Fock energy
and properties. The use of decomposed integrals is invoked by 
the keyword \Key{CHOLES} in the \Sec{*DALTON} input module. The
next example shows how to compute the dipole polarizabilty in this
way. 
%
\begin{verbatim}
**DALTON INPUT
.RUN PROPER
.CHOLES
**CHOLES
.REDUCE
.THRCOM
 1.0D-6
**WAVE FUNCTIONS
.HF
**PROPERTIES
.ALPHA
*ABALNR
.FREQUE
  2
 0.0   0.0932
*END OF
\end{verbatim}
Observe, that actually the keyword \Key{REDUCE} it is not needed, since it is
the default. The threshold of the decompostion is $10^{-6}$, which is enough to 
converge the wave function until the default value of $10^{-5}$. Furthermore,
note that the full section \Sec{*CHOLES} could have been omitted if the
decomposition would have been carried out until its default value of $10^{-8}$.

\section{KT3 magnetic properties using London orbitals}

The following input exemplifies the calculation of magnetic properties 
(magnetizabilities and nuclear shielding constants) using Cholesky
decomposed two-electron integrals, as required by keyword \Key{CHOLES}.
The decomposition uses default values since no \Sec{*CHOLES} appears.
Just to show how to do it, it has been required a tighter convergence
of the linear equations.  Note that otherwise the input is basically identical 
to a standard (or direct) calculation of magnetic properties with {\dalton}.
\begin{verbatim}
**DALTON INPUT
.RUN PROPERTIES
.CHOLES
**WAVE FUNCTIONS
.DFT
 KT3
*HF INPUT
.THRESHOLD
 1.0D-8
**PROPERTIES
.MAGNET
.SHIELD
*LINRES
.THRESH
 1.0D-6
*END OF INPUT
\end{verbatim}


\section{MP2 energy}
This example shows how to compute the MP2 energy using decomposed integrals. 
Observe that, since default values for the decompostions all used throughout
the whole calculation, the only difference with respect to a standard 
MP2 calculation using the {\cc} code is the presence of the keyword 
\Key{CHOLES} in the \Sec{*DALTON} general input section.

\begin{verbatim}
**DALTON INPUT
.CHOLES
.RUN WAVE
**WAVE FUNCTIONS
.CC
*SCF IN
.THRESHOLD
 5.0D-8
.MAX DIIS
 100
*CC INP
.PRINT
 3
.FREEZE
  54 0
.MP2
*END OF DALTON INPUT
\end{verbatim}

\section{Restart of MP2 energy}
Let's assume now that the above example did not finish because, say,
it exceeded the CPU time limit of the queue system. In addition to
the standard output, one can find in the scratch directory the
file \verb|CHOMP2_RST| with, for instance, the following
contents:

\begin{verbatim}

          Cholesky MP2 restart information
          --------------------------------


  FstSy   FstOr   Bfrst        LstSy   LstOr   Blast              Energy
    1        1       1           1       37      37           -0.7588448137
    1       38      38           1       74      74           -1.1690778421
    1       75      75           1      111     111           -1.4200963373
    1      112     112           1      148     148           -1.5553412292
    1      149     149           1      185     185           -1.6280847066
    1      186     186           2       35     222           -2.3350317768
    2       36     223           2       72     259           -2.7682043113
    2       73     260           2      109     296           -3.0257337904
    2      110     297           2      146     333           -3.1460304016
    2      147     334           2      183     370           -3.2146165324
    2      184     371           3       35     407           -3.8731562262
    3       36     408           3       72     444           -4.2604843530
    3       73     445           3      109     481           -4.4682304682
    3      110     482           3      146     518           -4.5838011660
\end{verbatim}

Looking at the last line, we can know that the calculation died after having
finished the (outermost) virtual orbital batch comprising from virtual
orbital nr. 482, i.e. \verb|Bfrst|, to virtual orbital nr. 518, i.e.
\verb|Blast|. This last orbital is nr. 146 (\verb|LstOr|) of symmetry
3 (\verb|LstSy|), the MP2 contribution until this point being -4.5838 a.u.
With this information, it is possible to restart the calculation by means
of the following input:

\begin{verbatim}
**DALTON INPUT
.CHOLES
.RUN WAVE FUNCTIONS
*CHOLES
.RSTDIA          ! Requires file CHODIAG
.RSTCHO          ! Requires files CHOLESKY.RST and CHOLES_*
**SIRIUS
.CC
*SCF INPUT
.THRESHOLD
 5.0D-8
.MAX DIIS
 100
*ORBITAL
.MOSTART
 NEWORB
*CC INP
.PRINT
 3
.FREEZE
  54 0
.MP2
*CHOMP2
.SKIPTR          ! Requires files CHOIA_*
.RSTMP2
   3  147        ! The virtual loop begins in orbital 147 of symmetry 3.
.OLDEN2
  -4.5838011660  ! The contribution of previous orbitals.
**END OF DALTON INPUT
\end{verbatim}

Of course, the comments (marked by the ! sign) are not needed. In
addition, keep in mind that if the default value \Key{REDUCE} is
used in the original AO decomposition, it is not possible to modify the
decomposition threshold later on.


\section{CC2 magnetic properties using the CTOCD/DZ method}
This example shows how to compute CTOCD/DZ magnetic properties
at the CC2 level with Cholesky decomposed integrals. Again, 
default values are used in the decomposition and, therefore, the
same input can be used in standard calculations provided that the 
keyword \Key{CHOLES} is deleted. In that case, CCSD calculations 
are also possible. Note that the needed one-electron integrals
are explicitly required by the keywords \Key{DIPVEL} and so on.
The code does not check if they are not available, but simply
takes them as zero!

\begin{verbatim}
**DALTON INPUT
.RUN WAVE
.CHOLES
**INTEGRAL
.DIPVEL             ! Susceptibilities and shiledings
.RANGMO             ! Susceptibilities
.ANGMOM             ! Susceptibilities and shiledings
.RPSO               ! Shieldings
.PSO                ! Shieldings
**WAVE FUN
.CC
*SCF INPUT
.THRESHOLD
 1.0D-8
*CC INP
.CC2
.MAX IT
 150
.THRENR
 1.0D-8
.THRLEQ
 1.0D-6
*CCLR
.CTOSUS             ! Compute susceptibilities (magnetizabilities)
.CTOSHI             ! Compute nuclear magnetic shieldings.
*END OF DALTON INPUT
\end{verbatim}

Obviously, it is not needed to compute simultaneously the two
magnetic properties presented here.

\section{CCSD(T) energy calculation using decomposed 
         energy denominators}

We now illustrate the use of Cholesky-based techniques to calculate
the CCSD(T) energy correction. The calculation of molecular 
properties has not been implemented yet with this method. Finally,
note that standard (not decomposed) two-electron integrals are 
used. More detailed information can be found at section~\ref{sec:chopt}.

\begin{verbatim}
**DALTON INPUT
.RUN WAVE
**WAVE FUNCTION
.CC
*SCF INPUT
.THRESH
 1.0d-8
*CC INPUT
.THRENR
 1.0d-08
.PRINT
 3
.FREEZE
 24   0
.CHO(T)
*CHO(T)
.THRCHO     ! Converge each term only to 1.0D-6 a.u.
 1.0D-6
.MXCHVE     ! Trunk the expansion of the denominators after the 6th term.
  6
*END OF DALTON INPUT
\end{verbatim}
