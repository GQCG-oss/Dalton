
%%%%%%%%%%%%%%%%%%%%%%%%%%%%%%%%%%%%%%%%%%%%%%%%%%%%%%%%%%%%%%%%%%%
\section{Ground state--excited state transition moments: \Sec{CCLRSD}}
\label{sec:cclrsd}
%%%%%%%%%%%%%%%%%%%%%%%%%%%%%%%%%%%%%%%%%%%%%%%%%%%%%%%%%%%%%%%%%%%
\index{linear response!Coupled Cluster}
\index{response!linear}
\index{oscillator strength!Coupled Cluster}
\index{rotatory strength!Coupled Cluster}
\index{transition moment!Coupled Cluster}
\index{transition strength!Coupled Cluster}
\index{absorption strength!Coupled Cluster}
\index{ECD!Coupled Cluster}
\index{electronic circular dichroism!Coupled Cluster}
\index{OECD!Coupled Cluster}
\index{oriented electronic circular dichroism!Coupled Cluster}


In the \Sec{CCLRSD} section the input that is
specific for coupled cluster response calculation of ground state--excited state
electronic transition properties is read in.
This section includes for example calculation of oscillator strength etc.
The transition properties are implemented for the models CCS, CC2, CCSD, and CC3
(singlet states only).
%Publications that report results obtained with this CC module should cite 
%Ref.\ \cite{Christiansen:CCLR}.
The theoretical background for the implementation is detailed in Ref.\ \cite{Christiansen:CCLR,Christiansen:QEL}.
This section \Sec{CCLRSD} has to be used in connection with \Sec{CCEXCI} 
for the calculation of excited states.

\begin{center}
\fbox{
\parbox[h][\height][l]{12cm}{
\small
\noindent
{\bf Reference literature:}
\begin{list}{}{}
\item Ground-state transition moments: O.~Christiansen, A.~Halkier, H.~Koch, P.~J{\o}rgensen, and T.~Helgaker \newblock {\em J.~Chem.~Phys.}, {\bf 108},\hspace{0.25em}2801, (1998).
\item CC3: F.~Pawlowski, P.~J{\o}rgensen, and Ch.~H{\"a}ttig \newblock {\em Chem.~Phys.~Lett.}, {\bf 389},\hspace{0.25em}413, (2004).
\end{list}
}}
\end{center}

\begin{description}
\item[\Key{DIPOLE}] 
%
Calculate the ground state--excited state dipole (length) transition properties including
the oscillator strength.
Integrals needed for operator DIPLEN.

\item[\Key{DIPVEL}] 
%
Calculate the ground state--excited state dipole-velocity  transition properties including
the oscillator strength in the velocity form.
Integrals needed for operator DIPVEL.
% Does not make sense:
% (The dipole length form is recommended
%for standard calculations).

\item[\Key{ECD}] 
%
Calculate the ground state--excited state rotatory strength governing electronic circular dichroism
(ECD) using both the length and velocity forms (see below).
Integrals needed for operator DIPLEN, DIPVEL, and ANGMOM.

\item[\Key{ECDLEN}] 
%
Calculate the ground state--excited state rotatory strength governing electronic circular dichroism
(ECD) using the length form. Note that the result will be origin-dependent.
Integrals needed for operator DIPLEN and ANGMOM.

\item[\Key{ECDVEL}] 
%
Calculate the ground state--excited state rotatory strength governing electronic circular dichroism
(ECD) using the origin invariant velocity form.
Integrals needed for operator DIPVEL and ANGMOM.

\item[\Key{NO2N+1}] 
%
Use an alternative, and normally less efficient, formulation for calculation
the transition matrix elements (involving solution of response equations for 
all operators instead of solving for the so-called $M$ vectors which is the default).

\item[\Key{OECD}] 
%
Calculate the ground state--excited state rotatory strength tensor governing oriented electronic circular dichroism
(OECD) using both the length and velocity forms (see below).
Integrals needed for operator DIPLEN, DIPVEL, ANGMOM, SECMOM, and ROTSTR.

\item[\Key{OECDLE}] 
%
Calculate the ground state--excited state rotatory strength tensor governing oriented electronic circular dichroism
(OECD) using the length form. Note that the result will be origin-dependent.
Integrals needed for operator DIPLEN, ANGMOM, and SECMOM.

\item[\Key{OECDVE}] 
%
Calculate the ground state--excited state rotatory strength tensor governing oriented electronic circular dichroism
(OECD) using the origin invariant velocity form.
Integrals needed for operator DIPVEL, ANGMOM, and ROTSTR.

\item[\Key{OPERAT}] \verb| |\newline
\verb|READ (LUCMD,'(2A8)') LABELA, LABELB|\newline
%
Read pairs of operator labels for which the residue of the linear response function is desired.
Can be used to calculate the transition property for a given operator
by specifying that operator twice. The operator can be any of the one-electron
operators for which integrals are available in the \Sec{*INTEGRALS} input part.

\item[\Key{SELEXC}]\verb| |\newline 
\verb|READ(LUCMD,*) IXSYM,IXST|\newline
%
Select for which excited states the calculation of transition properties
are carried. The default is all states according to the \Sec{CCEXCI} input section
(the program takes into account symmetry). For calculating selected states only,
provide a list of symmetry and state numbers (order after increasing energy in 
each symmetry class). This list is read until next input label is found.

\end{description}
