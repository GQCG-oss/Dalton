~%%%%%%%%%%%%%%%%%%%%%%%%%%%%%%%%%%%%%%%%%%%%%%%%%%%%%%%%%%%%%%%%%%%
\section{Ground state first-order properties: \Sec{CCFOP}}
\label{sec:ccfop}
%%%%%%%%%%%%%%%%%%%%%%%%%%%%%%%%%%%%%%%%%%%%%%%%%%%%%%%%%%%%%%%%%%%
\index{expectation values}
\index{one-electron properties}
\index{first-order properties}

In this Section, the keywords for the calculation of ground-state first-order
(one-electron) properties are described. The calculation
is evoked with the \Sec{CCFOP} flag followed by the appropriate
keywords as described in the list below. Note that \Sec{CCFOP}
assumes that the proper integrals are written on the
AOPROPER file, and one therefore has to set the correct property
integral keyword(s) in the \Sec{*INTEGRALS} input Section. For properties
that have both an electronic and a nuclear contribution, these will
be printed separately with a print level of 10 or above.
\index{AOPROPER}

The calculation of first-order properties is implemented for the 
coupled cluster models CCS (which gives SCF first-order properties),
CC2, MP2, CCD,  CCSD, CCSD(T), and CC3.  
By default, the chosen properties include orbital relaxation
contributions\index{orbital relaxation},
{\it i.e.\/} they are calculated from the relaxed CC (or MP) densities. 
To disable orbital relaxation for the CC2, CCSD and CC3 models, 
see \Key{NONREL} below. Relaxation is always included for MP2. 
Note also that the present implementation does not allow for 
CC2 relaxed first-order
properties in the frozen core approximation 
(\Key{FREEZE} and \Key{FROIMP}, see Sec.~\ref{sec:ccgeneral}).

For details on the implementation, see 
Refs.~\cite{Halkier:CCFOP,Halkier:CC2RLXFOP,Hald:CCPTFOP}.

\begin{center}
\fbox{
\parbox[h][\height][l]{12cm}{
\small
\noindent
{\bf Reference literature:}
\begin{list}{}{}
\item A.~Halkier, H.~Koch, O.~Christiansen, P.~J{\o}rgensen, and T.~Helgaker \newblock {\em J.~Chem.~Phys.}, {\bf 107},\hspace{0.25em}849, (1997).
\item  K.~Hald, A.~Halkier, P.~J{\o}rgensen, S. Coriani, C. H{\"a}ttig, T. Helgaker
\newblock {\em J.~Chem        .~Phys.}, {\bf 118},\hspace{0.25em}2985, (2003)
\item  K.~Hald, P.~J{\o}rgensen \newblock {\em Phys.~Chem.~Chem.~Phys.}, {\bf 4},\hspace{0.25em}5221, (2002)
`
\end{list}
}}
\end{center}

\begin{description}
\item[\Key{2ELDAR}] 
        Calculate relativistic two-electron Darwin term.
        \index{Darwin term!two-electron}
%
\item[\Key{ALLONE}] 
        Calculate all first-order properties described in this Section 
        (all corresponding property integrals are needed).
%
\item[\Key{DIPMOM}] 
        Calculate the permanent molecular electric dipole moment
        (\verb+DIPLEN+ integrals).
        \index{electric dipole}
        \index{dipole moment}
%
\item[\Key{NONREL}] 
        Compute the properties using the unrelaxed CC densities instead
        of the default relaxed densities.
%
\item[\Key{NQCC}] 
        Calculate the electric field gradients at the nuclei
        (\verb+EFGCAR+ integrals).
        \index{electric field!gradient}
%
\item[\Key{OPERAT}] \verb| |\newline
\verb|READ (LUCMD,'(1X,A8)') LABPROP|\newline
        Calculate the electronic contribution to the property defined
        by the operator label \verb+LABPROP+ (corresponding 
        \verb+LABPROP+ integrals needed).
%
\item[\Key{QUADRU}] 
        Calculate the permanent traceless molecular electric
        quadrupole moment (\verb+THETA+ integrals). Note that the
        origin is the origin of the coordinate system specified
        in the \molinp\ file.
        \index{electric quadrupole}
        \index{quadrupole moment}
%
\item[\Key{RELCOR}] 
        Calculate scalar-relativistic one-electron
        corrections to the ground-state
        energy (\verb+DARWIN+ and \verb+MASSVELO+ integrals).
        \index{relativistic corrections!one-electron}
        \index{Darwin term!one-electron}
        \index{mass-velocity term}
%
\item[\Key{SECMOM}] 
        Calculate the electronic second moment of charge
        (\verb+SECMOM+ integrals).
        \index{second moment of charge}
%
\item[\Key{TSTDEN}] 
        Calculate the CC energy using the two-electron CC density.
        Programmers keyword used for debugging purposes---Do not use.
%
\end{description}
