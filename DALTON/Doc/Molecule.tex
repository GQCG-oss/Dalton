\chapter{\mol\ input style}\label{ch:molinp}
\index{geometry!Cartesian coordinate input}
\index{Cartesian coordinate input}
\index{geometry!Z-matrix input}
\index{Z-matrix input}

The {\mol}-style input is originally based on the input for the \mol\
integral program by Alml\"{o}f~\cite{moleculeref}. However, there are
few remnants of the original input structure. For users of earlier
releases of the \dalton\ program, note should be taken of the fact that
the input structure for the \mol\ file has undergone major changes,
however, backward compatibility has in most cases been
retained\footnote{The two exceptions to the backward are that when
using the {\tt ATOMBASIS} keyword, the basis set name has to be
preceded by ``Basis=''. When specifying the Cartesian
coordinates of the atoms, it is no longer required that the
coordinates start at position 5. However, blanks are no longer
allowed in the names of atoms, and the atom names are still
restricted to 4 characters.}.
The program supports both Cartesian\index{Cartesian coordinate input}
and Z-matrix input\index{Z-matrix input} of the
molecular coordinates. However, the Z-matrix input provided is only a
convenient way of describing the molecular geometry, as the Z-matrix is
converted to Cartesian coordinates, which are then used in the
subsequent calculations. Note that the Z-matrix input can only be
used together with the basis set library (\quotekw{BASIS} in first line).

The program includes an extensive basis set
library\index{basis set!library}, which are described below. 
There are additional possibilities for choosing the number of primitive
and contracted orbitals to be used with the Atomic Natural Orbital
(ANO) basis sets\index{ANO basis set}\index{basis set!ANO}
and the ``Not Quite van Duijneveldt'' (NQvD) basis
sets\index{NQvD basis set}\index{basis set!NQvD}. The large number of
different basis sets provided is in part
related to the variety of molecular properties with very different
basis set requirements that can be calculated with \dalton.
Most of the basis sets have been downloaded from the EMSL basis set
library service\index{EMSL basis set library service}\index{basis set!EMSL library}\footnote{http://www.emsl.pnl.gov:2080/forms/basisform.html~\cite{emslref}} Only a small number of the basis sets were obtained from different sources:

\begin{list}{--}{}
\item The ano-1, ano-2, ano-3, ano-4 and Sadlej-pVTZ basis sets which where
downloaded from the MOLCAS home-page\index{NQvD basis set}\index{basis set!NQvD}\index{ANO basis set}
\index{basis set!ANO}\index{sadlej basis set}\index{basis set!sadlej}
(http://www.teokem.lu.se/molcas/).
\item The NQvD (The Not Quite van Duijneveldt) basis sets were constructed
by Knut F\ae gri~\cite{nqvdref}.
\item The Turbomole-X {X = SV,SVP,DZ,DZP,TZ,TZP,TZV,TZVP,TZVPP,TZVPPP} basis sets have
been downloaded from Turbomole\index{Turbomole basis set}\index{basis set!Turbomole}(http://www.turbomole.com)
\item The pc-$n$ and apc-$n$ polarization-consistent basis sets were provided to us by Frank Jensen.
\end{list}

We note that each file containing a given basis set in the BASIS
directory contains the proper reference to be used when doing a 
calculation with a given basis set. For convenience we also list 
these references with the basis sets in Section~\ref{sec:basislist}.

The description of the \mol\ input is divided into five parts.
Section~\ref{sec:molgeneral} describes the general section of the
molecule input,
section~\ref{sec:molcart} describes the Cartesian coordinate
input\index{Cartesian coordinate input},
section~\ref{sec:molzmat} describes the Z-matrix
input\index{Z-matrix input}, and finally
Section~\ref{sec:molbasis} describes the basis set
library\index{basis set!library}.
Section~\ref{sec:basislist} lists the basis sets (including references and supported elements) 
in the basis set library\index{basis set!library}.

\section{General \mol\ input}\label{sec:molgeneral}

In the general input section of the \mol\ input file, we will consider
such information as molecular symmetry\index{symmetry}, number of
symmetry distinct atoms\index{symmetry-distinct atom}, generators of a
given molecular point group\index{symmetry!generator}\index{symmetry!group}, and so on.
This information usually constitutes the four/five first lines of the
input.

The input is best described by an example.
The following is the first lines of an input for
tetrahedrane\index{tetrahedrane}, treated in
$C_{2v}$~symmetry, with a 4-31G** basis.  The line numbers are for
convenience in the subsequent input description and should {\em
not} appear in the actual input.  Note also that in order to fit
the example across the page some liberties have been taken with
column spacings.
\begin{verbatim}
 1:INTGRL
 2:        Tetrahedrane, Td_symmetric geometry
 3:                 4-31G** basis
 4:Atomtypes=2 Generators=2 X Y Integrals=1.00D-15
\end{verbatim}

We now define the input line-by-line.  The {\tt FORMAT} is given
in parenthesis.
\begin{description}
\item[1] The word \verb|INTGRL|\index{INTGRL} {\tt (A6)}.
\item[2-3] Two arbitrary title lines {\tt (A72)}.
\item[4] General instructions about the molecule.

This line is keyword-driven. The general structure of the input is
\verb|Keyword=|. The input is case sensitive, but \dalton\ will recognize
the keywords whether specified with only three characters (minimum) or
the full name (or any intermediate option). The order of the keywords
is arbitrary. The following keywords are recognized for this line:
\begin{description}
\item[Angstrom] Indicates that the atomic Cartesian coordinates are
  given in \AA ngstr\"{o}m, and not in bohr (atomic units) which is
  the default.
\item[Atomtypes] \verb|(Integer)|. {\em This keyword is
  required\/}. Number of atom types (number of atoms specified 
in separate blocks). For a Z-matrix input this will be the total
number of atoms in the molecule, the Z-matrix module will then
extract the number of atom types.
\item[Cartesian] Indicates that a Cartesian Gaussian basis set will be
  used in the calculations.\index{Cartesian basis function}\index{basis function!Cartesian}
\item[Charge] \verb|(Integer)|. The charge of the molecule\index{charge
  of molecule}. Will be used  by the program to determine the Hartree--Fock
occupation\index{HF occupation}\index{Hartree--Fock occupation}.
\item[Generators] \verb|(Integer|+\verb|Character)|. Number of symmetry
  generators\index{symmetry!generator}. If this keyword is not
  specified (and \verb|Nosymmetry| not invoked)
  the automatic symmetry detection routines of the program will be
invoked. Symmetry can be turned off (needed for instance if starting a
walk at a highly symmetric structure which one knows will break
symmetry) using the keyword \verb|Nosymmetry|. \dalton\ is restricted
  to the Abelian subgroups of D$_{2h}$, and thus there can be 1 to 3
  generating elements.

The number of generators\index{symmetry!generator} is followed the
equally many blocks of characters specifying which Cartesian axis
change sign during each of the generators. {\tt X}
is reflection\index{reflection}
in the \mbox{$yz$-plane}, {\tt XY} is rotation\index{rotation} about
the \mbox{$z$-axis},
and {\tt XYZ} denotes inversion\index{inversion}. Due to the handling
of symmetry in
the program, it is recommended to use mirror planes as symmetry
generating elements if possible.
\item[Integrals] \verb|(Real)|. Indicates the threshold for which
  integrals smaller than this will be considered to be zero. If not
  specified, a threshold of 1.0D-15 will be used. A threshold
of 1.0D-15 will give integrals correct to approximately 1.0D-13.
\item[Nosymmetry] Indicates that the calculation is to be run without
  the use of point-group symmetry. Automatic symmetry detection will
  also be disabled.\index{symmetry!automatic detection}
\item[Own] Indicates that a user-supplied scheme for generating
  transformed angular momentum basis functions will be used.
\item[Spherical] Default. Indicates that a spherical Gaussian basis
  set will be used in the calculations.\index{spherical basis function}\index{basis function!spherical}
\end{description}
\end{description}

Note that if one wants to use a basis set
library\index{basis set!library}, there are two
options. One option is to use a common basis set for the entire
molecule in which the first line should be replaced by two lines,
which for a calculation using the 4-31G** basis would look like:
\begin{verbatim}
 1:BASIS
 2:4-31G**
\end{verbatim}
This option will not be active with customizable basis sets like the
ANO or NQvD sets.


Alternatively you may specify different basis sets for different
atoms, in which case the first line should read
\begin{verbatim}
 1:ATOMBASIS
\end{verbatim}

The fourth line (fifth in a calculation using the basis set library with "BASIS" in line 1)
looks a bit devastating. However, for ordinary
Hartree--Fock\index{HF}\index{MP2}\index{SCF}\index{Hartree--Fock}\index{M{\o}ller-Plesset!second-order}
or MP2 calculations, only the number of different atom types and the charge
need to be given (if the molecule is charged), as symmetry and
Hartree--Fock occupation\index{HF occupation}\index{Hartree--Fock occupation}
will be taken care of by the program. Thus
this line could in the above example be reduced to
\begin{verbatim}
 4:Atomtypes=2
\end{verbatim}
or even more concisely (though not more readable) as
\begin{verbatim}
 4:Ato=2
\end{verbatim}


Let us finally give some remarks about the symmetry
detection\index{symmetry!automatic detection}
routines. These routines will detect any symmetry of a molecule
by explicit testing for the occurrence of rotation axes, mirror planes
and center of inversion. The occurrence of a symmetry element is
tested in the program against a threshold which may be adjusted by the
keyword \Key{SYMTHR} in the \Sec{MOLBAS} input section. By default,
the program will require
geometries that are correct to the sixth decimal place in order to
detect all symmetry elements.

The program will translate and rotate the molecule into a suitable
reference geometry before testing for the occurrence of symmetry
operations. The program will not, due to the handling of symmetry
in the program, transform  the molecule back to original input
coordinates. Furthermore, if there are symmetry equivalent nuclei,
these will be removed from the input, and a new, standardized
molecule input file will be generated and used in subsequent
iterations of for instance a geometry optimization. This
standardized input file (including basis set) is printed to the file
\verb|DALTON.BAS|, which is among the files copied back after the end
of a calculation.

\dalton\ can only take advantage of point groups that are subgroups
of D$_{2h}$. If symmetry higher than that is detected, the program
will use the highest common subgroup of  the symmetry group
detected and D$_{2h}$.

We recommend that the automatic symmetry detection feature is not
used when doing MCSCF\index{MCSCF} calculations, as symmetry
generators\index{symmetry!generator} and their order in the input
determines the order of the irreducible representations
needed when specifying active spaces. Thus, for MCSCF calculations
we recommend that the symmetry is  explicitly specified through
the appropriate symmetry generators, as well as the explicit
Hartree--Fock occupation numbers.

\section{Cartesian geometry input}\label{sec:molcart}
\index{Cartesian coordinate input}

Assuming that we have given the general input as indicated above, we
now want to specify the spatial arrangements of the atoms in a
Cartesian coordinate system. We will also for sake of illustration
assume that we have given explicitly the generators of the point group
to be used in the calculation (in this case C$_{2v}$, with the yz- and
xz-planes as mirror planes).

In tetrahedrane we will have two different kinds of atoms, carbon
and hydrogen, as indicated by the number 2 on the fourth line of the input.
We will also assume that we enter the basis set ourselves,
in order to present the input format for the basis set.

For tetrahedrane, the input would then look like
\begin{verbatim}
 1:INTGRL
 2:        Tetrahedrane, Td_symmetric geometry
 3:                 4-31G** basis
 4:Atomtypes=2 Generators=2 X Y Integrals=1.00D-15
 5:Charge=6.0 Atoms=2 Blocks=3 1 1 1
 6:C1    1.379495419          .0                 0.975450565
 7:C2     .0                 1.379495419         -.975450565
 8:    8    3
 9:486.9669   .01772582
10:73.37109   .1234779
11:16.41346   .4338754
12:4.344984   .5615042
13:8.673525            -.1213837
14:2.096619            -.2273385
15:.6046513             1.185174
16:.1835578                      1.00000
17:    4    2
18:8.673525   .06354538
19:2.096619   .2982678
20:.6046513   .7621032
21:.1835578             1.000000
22:    1    1
23:0.8        1.0
24:Charge=1.0 Atoms=2 Blocks=2 1 1
25:H1    3.020386510         .0                  2.1357357837
26:H2     .0                 3.020386510         -2.1357357837
27:    4    2
28:18.73113   .03349460
29:2.825394   .2347270
30:.6401218   .8137573
31:.1612778             1.000000
32:    1    1
33:0.75       1.0
\end{verbatim}

Lines 1-4 are already described. The different new types of lines are:
\begin{description}
\item[5] This line is keyword-driven. The general structure of the input is
\verb|Keyword=|. The input is case sensitive, but it will recognize
the keywords whether specified with only three characters (minimum) or
the full name (or any intermediate option). The order of the keywords
is arbitrary. The following keywords are
recognized for this line:
\begin{description}
\item[Atoms] \verb|(Integer)|. Number of {\em
  symmetry-distinct}\index{symmetry-distinct atom} atoms of 
this type (or, if the symmetry detection routines are being used, all
atoms of this kind).
\item[Basis] \verb|(Character)|. If \verb|ATOMBASIS| has been specified,
  the keyword is required, and have to be followed by the name of the
  basis set that is to be used for this group of atoms, {\it e.g.\/}
  \verb|Basis=6-31G**|. By specifying
  \verb|Basis=pointcharge|\index{point charge}, the
  atoms in this block will be treated as point charges, that is,
  having only a charge but no basis functions attached to
  them.

  Different effective core potentials (ECP) could be used when \verb|ATOMBASIS|
  is specified. For instance, Stuttgart ECPs with corresponding
  Stuttgart double zeta basis sets\index{effective core
  potentials}\index{ECP} can be used by specifying
  \verb|Basis=ecp-sdd-DZ| \verb|ECP=ecp-sdd-DZ| (see \verb|test/rsp_ecp| for example).
\item[Blocks] \verb|(Integers)|. Maximum angular quantum number + 1 used in the
basis set for this atom type ($s=1$, $p=2$, etc.).
Ignored if library basis sets are being used (\verb|BASIS| or
\verb|ATOMBASIS| in first line). This number is followed by one
integer for each angular momentum used in the basis, indicating 
the number of groups (blocks) of generally contracted
functions of angular quantum number~{\tt I-1}.
Ignored if the basis set library is used. \newline
It is noteworthy that
\dalton\ collects all basis functions into one such shell, and
evaluates all integrals arising from that 
shell\index{shell of basis functions} simultaneously, and
the memory requirements grow rapidly with the number of basis
functions in a shell (note for instance that four $g$ functions
actually are 36 basis functions,
as there are 9 components of each $g$ function).
Memory requirements\index{memory} can therefore be reduced by splitting
basis functions of the quantum number into different blocks. However,
this will decrease the performance of the
integral\index{performance of integral program} calculation.
\item[Charge] \verb|(Real)|. {\em This keyword is required\/}.
  Charge of this atom\index{charge of atom} or point charge. 
\item[Pol] \verb|(Integer+real)|. This keyword adds single, primitive
  basis function of a given quantum number (quantum number + 1 given
  in the input) and a given exponent. An arbitrary number of
  polarization functions can be given. For instance, we can add a $p$
  function with exponent $0.05$ and a $d$ function with exponents
  $0.6$ we can write \verb|Pol 2 0.05 3 0.6|.
\item[Set] \verb|(Integer)|. Indicates whether the basis set specified
  is the ordinary orbital basis or the auxiliary basis set needed for
  instance in certain r12 calculations, see
  Sec.~\ref{sec:r12aux}. The keyword is only active when the keyword
  \Key{R12AUX} has been specified in the \Sec{MOLBAS} input section.
\end{description}
\item[6] \verb|NAME X Y Z Isotope=18|
\begin{description}
\item[NAME] Atom name.  A different name should be used for
each atom of the same type, although this is not required. Note that
only the first four characters of the atom name will be used by the
program.
\item[X] $x$-coordinate (in atomic units, unless \AA ngstr\"{o}m
has been requested on line 4 of the input).
\item[Y] $y$-coordinate.
\item[Z] $z$-coordinate.
\item[Isotope=] \index{Isotope=} Specify the atomic mass of the nucleus (closest
  integer number). By default the mass of the most abundant isotope
  of the element will be used. When automatic symmetry detection is
  used, the program will distinguish between different nuclei if they
  have different atomic mass number. A calculation of HDO would thus
  be run in $C_s$ symmetry. 
\end{description}
The Cartesian coordinates may
be given in free format.
\item[7] This is the other symmetry-distinct center of this type.
\item[8] \verb|FRMT, NPRIM, NCONT, NOINT| {\tt (A1,I4,2I5)}.
\begin{description}
\item[FRMT] A single character describing the input format of the
basis set in this block. The default format is {\tt (8F10.4)} which
will be used if {\tt FRMT} is left blank. In this format
the first column is the orbital exponent and the seven last columns
are contraction coefficients. If no numbers are given, a zero is
assumed. If more than 7 contracted functions occur in a given block,
the contraction coefficients may be continued on the next line, but
the first column (where the orbital exponents are given) must then be
left blank.

An {\tt F} or {\tt f} in the first position will indicate that the
input is in free format. This will of course require that all
contraction coefficients need to be typed in, as all numbers need
to be present on each line. However, note that this options is
particularly handy together with completely decontracted basis
sets, as described below. Note that the program reads the free
format input from an internal file that is 80 characters long, and
no line can therefore exceed 80 characters.

One may also give the format {\tt H} or {\tt h}. This corresponds to
high precision format {\tt (4F20.8)}, where the first column again is
reserved for the orbital exponents, and the three next columns are
designated to the contraction coefficients. If no number is given, a zero
is assumed. If there are more than three contracted orbitals in a
given block, the contraction coefficients may be continued on the next
line, though keeping the column of the orbital exponents blank.

\item[NPRIM] Number of primitive\index{primitive basis function}
Gaussians in this block.
\item[NCONT] Number of contracted\index{contracted basis function}
Gaussians in this block. If a zero
is given, an uncontracted basis set will be assumed, and only orbital
exponents need to be given.
\end{description}
\item[9] \verb|EXP, (CONT(I), I=1,NCONT)|
\begin{description}
\item[EXP] Exponent of this primitive.
\item[CONT(I)] Coefficient of this primitive in contracted
function~{\tt I}.
\end{description}
We note that the format of the orbital exponents\index{orbital exponent} 
and the contraction\index{contraction coefficient}
coefficients are determined from the value of {\tt FRMT} defined on
line~8.
\item[10-16] These lines complete the specification of this
contraction block: the $s$~basis here.
\item[17-21] New contraction block (see lines~8 and~9 above).
\item[22-23] New contraction block.
\item[24-33] Specifies a new atom type: coordinates and basis set.
\end{description}

\section{Z-matrix input}\label{sec:molzmat}
\index{geometry!Z-matrix input}
\index{Z-matrix input}

The Z-matrix input provided with \dalton\ is quite rudimentary, and
common options like parameter representations of bond length and
angles as well as dummy atoms are not provided. Furthermore,
the Z-matrix input is not used in the program, but instead immediately
converted to Cartesian coordinates which are then used in the
subsequent calculation.
Another restriction is that if Z-matrix input is used, one cannot
punch ones own basis set, but must instead resort to one of the basis sets
provided with the basis set library (i.e. {\tt BASIS} in first line).
Finally, you cannot explicitly specify higher symmetry in input line 4
when using Z-matrix input,
you can only request no symmetry ($C_1$) with \verb|Nosymmetry|
or allow \dalton\ to detect symmetry automatically.

The input format is free, with
the restriction that the name of each atom must be given a space
of 4 characters, and none of the other input variables needed must be
placed in these positions.

The program will use Z-matrix input\index{Z-matrix input} if there is
the word \quotekw{ZMAT} 
in the first four position of line 6 in the molecule input.
The following {\tt NONTYP} lines contain the Z-matrix specification
for the {\tt NONTYP} atoms.

A typical Z-matrix input could be:
\begin{verbatim}
BASIS
6-31G**
   Test Z-matrix input of ammonia
   6-31G** basis set
Atomtypes=4
ZMAT
N   1 7.0
H1  2 1 1.0116 1.0
H2  3 1 1.0116 2 106.7 1.0
H3  4 1 1.0116 2 106.7 3 106.7 1 1.0
\end{verbatim}
The five first lines should be familiar by now, and will be discussed
no further here. 
The special 6'th line tells that this is Z-matrix input.
The Z-matrix input starts on line 7, and on this first Z-matrix
line only the atom name, a running number and the charge of the
atom\index{charge of atom}
is given. The running number is only for
ease of reference to a given atom, and is actually not used within the
program, where any reference to an atom, is the number of the
atom consecutively in the input list.

The second Z-matrix line consists of the atom name, a running number, the
number of the atom to which this atom is bonded with a given bond
length in {\AA}ngstr{\"o}ms, and then finally the charge of this atom.

The third Z-matrix line is identical to the second, except that an extra atom
number, to which the two first atoms on this line is bonded to with a
given bond angle in degrees.

On the fourth Z-matrix line yet another atom has been added, and the position
of this atom relative to the three previous ones on this line is
dependent upon on an extra number inserted just before the nuclear
charge of this atom. If the next to last number is a 0, the
position of this atom is given by the dihedral angle {\tt
(A1,A2,A3,A4)} in degrees, where {\tt Ai} denotes atom {\tt i}. If, on the other
hand, this next to last number is $\pm 1$, the position of the fourth
atom is given with respect to two angles, namely {\tt (A1,A2,A3)} and
{\tt (A2,A3,A4)}. The sign is to be $+ 1$ if the triple product
$\overrightarrow{\left(A_{2}A_{1}\right)}\cdot\left[\overrightarrow{\left(A_{2}A_{3}\right)}\times\overrightarrow{\left(A_{2}A_{4}\right)}\right]$
is positive.

\section{Using basis set libraries}\label{sec:molbasis}
\index{basis set!library}
\index{BASIS}
\index{ATOMBASIS}

The use of predefined basis sets is indicated
by the word {\tt BASIS} or {\tt ATOMBASIS} on the first line of the
molecular input.
(If you want Z-matrix input you must use {\tt BASIS}.)

The specified basis set(s) are searched for in the following directories:
\begin{itemize}
\item all user specified basis set directories (with {\tt dalton -b dir1 -b dir2 ...}
\item the job directory
\item the basis set library supplied with \dalton.
\end{itemize}

If {\tt BASIS} is used, a common basis set is used for all atoms in
the molecule, and the name of this basis set is given on the second line.
If we want to use one basis set for all the
atoms in a molecule, the molecule input file can be significantly
simplified, as we may delete all the input information regarding the
basis set. Thus, the input in the previous section for tetrahedrane
with the 6-31G** basis will, if the basis set library is used,  be reduced
to:

\begin{verbatim}
 1:BASIS
 2:6-31G**
 3:        Tetrahedrane, Td_symmetric geometry
 4:                 4-31G** basis
 5:Atomtypes=2 Generators=2 X Y Integrals=1.00D-15
 6:Charge=6.0 Atoms=2
 7:C1    1.379495419          .0                 0.975450565
 8:C2     .0                 1.379495419         -.975450565
 9:Charge=1.0 Atoms=2
10:H1    3.020386510         .0                  2.1357357837
11:H2     .0                 3.020386510         -2.1357357837
\end{verbatim}

The use of the basis set library is indicated by the presence of the
\verb|BASIS| word in the beginning of \mol-file instead of
\verb|INTGRL|.

An alternative approach would be to use different basis sets for
different atoms, {\it e.g.\/} the concept of locally
dense\index{locally dense basis set} basis sets
introduced in NMR calculations by Chesnut {\it et
al.\/}~\cite{dbcberkdmdaejcc14}. This is for instance also required
when using the ANO\index{ANO basis set}\index{basis set!ANO} or
NQvD\index{NQvD basis set}\index{basis set!NQvD}
basis sets. Another option is to use
standards basis sets from the basis set 
library\index{basis set!library} and add your own sets
of diffuse, tight or polarizing\index{polarization function} basis
functions. Returning to
tetrahedrane, we could for instance use the 6-31G* basis set for
carbon and the 4-31G** basis set for hydrogen. This could be achieved
as

\begin{verbatim}
 1:ATOMBASIS
 2:        Tetrahedrane, Td_symmetric geometry
 3:    Mixed basis (6-31G* on C and 4-31G** on H)
 4:Atomtypes=2 Generators=2 X Y Integrals=1.00D-15
 5:Charge=6.0 Atoms=2 Basis=6-31G*
 6:C1    1.379495419          .0                 0.975450565
 7:C2     .0                 1.379495419         -.975450565
 8:Charge=1.0 Atoms=2 Basis=4-31G Pol 2 0.75D0
 9:H1    3.020386510         .0                  2.1357357837
10:H2     .0                 3.020386510         -2.1357357837
\end{verbatim}


Thus, when using {\tt ATOMBASIS}\index{ATOMBASIS} the name of the
basis set for a given
set of identical atoms is given on the same line as the nuclear
charge, indicated by the keyword ``Basis=''.

The string {\tt Pol} denotes that the rest of the line specifies
diffuse, tight or polarizing\index{polarization function}
functions, all which will be added as segmented basis functions.
For each basis function, its ``angular momentum'' ($l+1$) and its
exponent must be given. Thus, in the above input we indicate that
we add a $p$ function with exponent 0.75 to the hydrogen basis
set. The order of these functions are arbitrary (that is, a $p$
function can be given before an $s$ function and so on).

Augmenting\index{basis!augmented} the correlation-consistent\index{basis!correlation-consistent}\index{correlation-consistent basis set} 
sets of Dunning~\cite{dewthdjcp100} is a straight-forward process in \dalton. 
The aug-cc-pVXZ and aug-cc-pCVXZ basis sets are extended in an even-tempered manner 
(in the manner of Dunning~\cite{dewthdjcp100})
by including a 'd-', 't-' or 'q-' prefix to give doubly, triply or quadruply 
augmented basis sets, respectively. For example, specifying t-aug-cc-pVDZ will
produce a triply augmented cc-pVDZ basis.
%hjaaj: aug-ecp does not exist in basis/ directory, thus commented here. /Aug-2005 hjaaj.
%The aug-ecp basis sets may be augmented 
%in the same manner by using the same prefixes, eg d-aug-ecp. 
Note that these basis sets are not listed 
explicitly in the basis library directory\index{basis set!library}, 
but are automatically generated within \dalton\ from the respective 
aug-cc-pVXZ
% and aug-ecp basis sets. 
basis set.

The ANO basis sets require that you give the number of contracted
functions you would like to use for each of the primitive sets defined
in the basis sets. Thus, assuming we would like to simulate the
6-31G** basis set input using an ANO basis set but with the
polarization functions of the 6-31G** set, this could be achieved
through an input like

\begin{verbatim}
 1:ATOMBASIS
 2:        Tetrahedrane, Td_symmetric geometry
 3:    Mixed basis (6-31G* on C and 4-31G** on H)
 4:Atomtypes=2 Generators=2 X Y Integrals=1.00D-15
 5:Charge=6.0 Atoms=2 Basis=ano-1 3 2 0 0 Pol 3 0.8
 6:C1    1.379495419          .0                 0.975450565
 7:C2     .0                 1.379495419         -.975450565
 8:Charge=1.0 Atoms=2 Basis=ano-1 2 0 0 Pol 2 0.75D0
 9:H1    3.020386510         .0                  2.1357357837
10:H2     .0                 3.020386510         -2.1357357837
\end{verbatim}

This input will give a [3s2p0d0f] ANO basis set\index{ANO basis set}\index{basis set!ANO}
on carbon, with a
polarizing $d$ function with exponent 0.8, and a [2s0p0d] ANO basis
set on hydrogen with a polarizing $p$ function with exponent 0.75 as
above.

Note that the number of contracted functions in the ANO\index{ANO basis set}\index{basis set!ANO}
set has to be
given for all primitive blocks, even though you do not want any
contracted functions of a given quantum number. Here also, {\tt Pol}
separates the number of contracted functions from polarization
functions.

The NQvD basis set~\cite{nqvdref} was constructed in order to provide,
in electronic form, a basis set compilation very similar to original
set of van Duijneveldt~\cite{fbvdibmrap}\index{NQvD basis set}\index{basis set!NQvD}.
The sets are in general as good, or slightly better, than the original
van Duijneveldt basis, with only minor changes in the orbital
exponents.

In the NQvD basis set, you need not only to pick the number of
contracted functions, but also your primitive set. The contracted
basis set will be constructed contracting the (NPRIM-NCONT + 1)
tightest functions with contraction coefficients based on the
eigenvectors from the atomic optimization, keeping the outermost
orbitals uncontracted.

NOTE: As is customary, the orbital exponents of all hydrogen basis
functions are automatically multiplied by a factor of 1.44.

Thus, an input for tetrahedrane employing the NQvD basis set might
look like

\begin{verbatim}
 1:ATOMBASIS
 2:        Tetrahedrane, Td_symmetric geometry
 3:    Mixed basis (6-31G* on C and 4-31G** on H)
 4:Atomtypes=2 Generators=2 X Y Integrals=1.00D-15
 5:Charge=6.0 Atoms=2 Basis=NQvD 8 4 3 2 Pol 3 0.8D0
 6:C1    1.379495419          .0                 0.975450565
 7:C2     .0                 1.379495419         -.975450565
 8:Charge=1.0 Atoms=2 Basis=NQvD 4 2 Pol 2 0.75D0
 9:H1    3.020386510         .0                  2.1357357837
10:H2     .0                 3.020386510         -2.1357357837
\end{verbatim}

This input will use an (8s4p/4s) primitive basis set on carbon and
hydrogen respectively, contracting it to a [3s2p/2s] set. The
polarization functions should not require further explanation at this
stage.

The only limitations to the use of polarization functions when {\tt
ATOMBASIS} is used, is that the length of the line must note exceed 80
characters. If that happens, we recommend collecting a standard basis set
from the file \verb|DALTON.BAS|, and then adding
functions to this set.

\section{Auxiliary basis sets}
\label{sec:r12aux}

It is possible to specify more than one basis set. For example, by typing
\begin{verbatim}
 1:BASIS
 2:4-31G** 6-311++G(3df,3pd)
\end{verbatim}
the basis set 6-311++G(3df,3pd) will be used as an auxiliary basis.
When using an auxiliary basis, each atom line must contain a basis-set
identifier, which at present may take the values \verb|Set=1| (orbital basis) or 
\verb|Set=2| (auxiliary basis). Basis sets with \verb|Set=1| must be read first,
basis sets with \verb|Set=2| thereafter.
The above also applies to the \verb|ATOMBASIS| and \verb|INTGRL| input modes.
Examples are provided by the following input files:
\begin{verbatim}
 1:BASIS
 2:cc-pVDZ cc-pCVQZ
 3:Direct MP2-R12/cc-pVDZ calculation on H2O
 4:Auxiliary basis: cc-pCVQZ
 5:Atomtypes=1 Generators=1 X
 6:Charge=8.0 Atoms=1 Set=1
 7:O       .000000000000000    .000000000000000  -0.124309000000000       *
 8:Charge=1.0 Atoms=1 Set=1
 9:H      1.427450200000000    .000000000000000   0.986437000000000       *
10:Charge=8.0 Atoms=1 Set=2
11:O       .000000000000000    .000000000000000  -0.124309000000000       *
12:Charge=1.0 Atoms=1 Set=2
13:H      1.427450200000000    .000000000000000   0.986437000000000       *
\end{verbatim}
\begin{verbatim}
 1:ATOMBASIS
 2:Direct MP2-R12/cc-pVDZ calculation on H2O
 3:Auxiliary basis: cc-pCVQZ
 4:Atomtypes=4 Generators=1 X
 5:Charge=8.0 Atoms=1 Set=1 Basis=cc-pVDZ
 6:O       .000000000000000    .000000000000000  -0.124309000000000       *
 7:Charge=1.0 Atoms=1 Set=1 Basis=cc-pVDZ
 8:H      1.427450200000000    .000000000000000   0.986437000000000       *
 9:Charge=8.0 Atoms=1 Set=2 Basis=cc-pCVQZ
10:O       .000000000000000    .000000000000000  -0.124309000000000       *
11:Charge=1.0 Atoms=1 Set=2 Basis=cc-pCVQZ
12:H      1.427450200000000    .000000000000000   0.986437000000000       *
\end{verbatim}
Note that the keyword \Key{R12AUX} must be specified in the 
\Sec{MOLBAS} input section to be able to read the above inputs.

\section{The basis sets supplied with \dalton }
\label{sec:basislist}
\index{basis set!library}

As was mentioned above, all the basis sets supplied with this release
of the \dalton\ program --- with a few exceptions --- have been obtained 
from the EMSL basis set library~\cite{emslref}. 
Supplied basis sets include the STO-$n$G and Pople style basis sets,
Dunning's correlation-consistent basis 
sets\index{basis!correlation-consistent}\index{correlation-consistent basis set}, 
Ahlrichs Turbomole basis sets and Huzinaga basis sets. 
In a very few cases we have corrected the
files as obtained from EMSL, however we take no responsibility
for any errors inherent in these files.

The ANO and Sadlej-pVTZ polarization basis sets have been obtained from the
MOLCAS homepage without any further processing, and should therefore
be free of errors. The NQvD basis provided to us by Knut F\ae gri 
along with the Turbomole basis sets have been slightly reformatted for more 
convenient processing of the file, hopefully without having introduced any errors. 
The pc-$n$ and apc-$n$ basis sets have been provided to us by Frank Jensen.

We have included several Turbomole basis sets,
although we have retained the Ahlrichs basis sets
from the EMSL basis set library. We recommend the Turbomole
basis sets rather than the Ahlrichs basis sets from EMSL, which
contain some errors. The Ahlrichs-VDZ basis
is similar to the Turbomole-SV basis, while the Ahlrichs-VTZ is a
combination of the Turbomole-TZ (H--Ar) and Turbomole-DZ (K--Kr).

Below we give a comprehensive list of all basis sets included in 
the basis set library\index{basis set!library}, together with 
a list of the elements supported, and the complete reference to be 
cited when employing a given basis set in a calculation. 

\setlongtables
\begin{longtable}{lll}
\multicolumn{3}{}{\bf Basis Sets included in \dalton\ distribution.} \\ 
\hline \hline
\bf{Basis set name} & \bf{Elements} & \bf{References}\\
\hline\hline
\endfirsthead
\multicolumn{3}{}{\bf{Basis Sets included in \dalton\ distribution.}} \\
\hline \hline
\bf{Basis set name} & \bf{Elements} & \bf{References}\\
\hline \hline
\endhead
\endfoot \endlastfoot
\bf{\emph{STO-$n$G}} & & \\
STO-2G & H--Ca, Sr & \cite{wjhrfsjapjcp51,wjhrdrfsjapjcp52} \\
STO-3G & H--Cd & \cite{wjhrfsjapjcp51,wjhrdrfsjapjcp52,
   wjpbalwjhrfsic19,wjpwjhjcc4} \\
STO-6G & H--Ar & \cite{wjhrfsjapjcp51,wjhrdrfsjapjcp52} \\
\hline
\bf{\emph{Pople-style basis sets}} & & \\
3-21G & H--Cs & \cite{jsbjapwjhjacs102,msgjsbjapwjpwjhjacs104,
   kddwjhjcc7,kddwjhjcc8-1,kddwjhjcc8-2,edgdfjpc99} \\
3-21G* & H--Cl & 3-21G, with polariz. functions from
  \cite{wjpmmfwjhdjdjapjsbjacs104}.\\
  & & \emph{Note}: Polariz. functions only on Na--Cl \\
3-21++G & H, Li--Ar & 3-21G, with diffuse functions from
  \cite{tcjcgwsprsjcc4}.\\
3-21++G* & H--Cl & 3-21++G and 3-21G*\\
4-31G & H--Ar & \cite{rdwjhjapjcp54,msgjsbjapwjpwjhjacs104} 
   He,Ne from Gaussian 90 \\
6-31G & H--Ar, Zn & \cite{wjhrdjapjcp56,jddjapjcp62,
   mmfwjpwjhjsbmsgdjdjapjcp77,vrjapmrtlwjcp109} 
   He,Ne from Gaussian 90 \\
6-31G* & H--Ar & 6-31G, with polariz. functions from
  \cite{pchjaptca28,mmfwjpwjhjsbmsgdjdjapjcp77}\\
6-31G** & H--Ar  & 6-31G*, with polariz. functions from
  \cite{pchjaptca28}\\
6-31+G & H--Ar & 6-31G, with diffuse functions from
  \cite{tcjcgwsprsjcc4}.\\
6-31++G & H--Ar & 6-31G, with diffuse functions from
  \cite{tcjcgwsprsjcc4}.\\
6-31+G*   & H--Ar & 6-31+G and 6-31G*\\
6-31++G*  & H--Ar & 6-31++G and 6-31G*\\
6-31++G** & H--Ar & 6-31++G and 6-31G**\\
6-31G(3df,3pd) & H--Ar & 6-31G, with polariz. functions from
  \cite{mjfjapjsbjcp80}.\\
6-311G & H--Ar, Br, I & \cite{rkjsbrsjapjcp72,admgscjcp72,
   lacmpmjpbnedrcblrjcp103,mngapmpmlrjcp103}\\
6-311G* & H--Ar, Br, I & 6-311G, with polariz. functions from
  \cite{rkjsbrsjapjcp72,lacmpmjpbnedrcblrjcp103}.\\
6-311G** & H--Ar, Br, I & 6-311G, with polariz. functions from
  \cite{rkjsbrsjapjcp72,lacmpmjpbnedrcblrjcp103}.\\
6-311+G* & H--Ne & 6-311G* with diffuse functions from 
  \cite{tcjcgwsprsjcc4}.\\
6-311++G** & H--Ne & 6-311G** with diffuse functions from
  \cite{tcjcgwsprsjcc4}.\\
6-311G(2df,2pd) & H--Ne & 6-311G, with polariz. functions from
  \cite{mjfjapjsbjcp80}.\\
6-311++G(2d,2p) & H--Ne & 6-311++G, with polariz. functions from
  \cite{mjfjapjsbjcp80}.\\
6-311++G(3df,3pd) & H--Ar & 6-311++G, with polariz. functions from
  \cite{mjfjapjsbjcp80}.\\
\hline
%
Huckel & H--Cd & \\
MINI(Huzinaga) & H--Ca & \cite{huzinagabasis} \\
MINI(Scaled) & H--Ca & \cite{huzinagabasis,htshjcc1} \\
\hline
%
\bf{\emph{Dunning-Hay basis sets}} & & \\
SV(Dunning-Hay) & H, Li--Ne & \cite{thdpjhhfs1977} \\
SVP(Dunning-Hay) & H, Li--Ne & SV, with polariz. functions from
  \cite{thdpjhhfs1977,emhfsjcp83}.\\
SVP+Diffuse(Dunning-Hay) & H, Li--Ne & SVP with diffuse functions from
  \cite{thdpjhhfs1977,emhfsjcp83}.\\
SV+Rydberg(Dunning-Hay) & H, Li--Ne & SV, with polariz. functions from
  \cite{thdpjhhfs1977-2}.\\
\small{SV+DoubleRydberg(Dunning-Hay)} & H, Li--Ne & SV, with polariz. functions from
  \cite{thdpjhhfs1977-2}.\\
DZ(Dunning) & H, B--Ne, Al--Cl & \cite{thdjcp53,thdpjhhfs1977} \\
DZP(Dunning) & H, B--Ne, Al--Cl & DZ, with polariz. functions from 
  \cite{thdpjhhfs1977,emhfsjcp83}.\\
DZP+Diffuse(Dunning) & H, B--Ne & DZP with diffuse functions from
  \cite{thdpjhhfs1977,emhfsjcp83}.\\
DZ+Rydberg(Dunning) & H, B--Ne, Al--Cl & DZ, with polariz. functions from
  \cite{thdpjhhfs1977-2}.\\
DZP+Rydberg(Dunning) & H, B--Ne, Al--Cl & DZP, with polariz. functions from 
  \cite{thdpjhhfs1977-2}.\\
TZ(Dunning) & H, Li--Ne & \cite{thdjcp55}\\
\hline
\multicolumn{2}{l}{\bf{\emph{Dunning's correlation-consistent basis 
sets\index{basis set!correlation-consistent}\index{correlation-consistent basis set}}}} \\
\multicolumn{3}{l}{\emph{Note: H,He (valence only) are included in all core-valence basis sets 
  for convenience.}} \\
cc-pVXZ (X = D,T,Q,5,6) & H--Ne, Al--Ar, & \cite{thdjcp90,dewthdjcp100,dewthdjcp98,
  jkkapjpca106,akwdewkapthdjcp110} \\
  & Ca, Ga--Kr & \emph{Note}: 6Z only includes H,C--O\\
cc-pCVXZ (X = D,T,Q,5) & H, He, B--Ne, & cc-pVXZ, with core functions from
  \cite{thdjcp90,jkkapjpca106,dewthdjcp103}. \\
  & Na--Ar & \emph{Note}: 5Z only includes H, He, B--Ne. \\
cc-pwCVXZ (X = D,T,Q,5) & H, He, C--Ne,  & cc-pCVXZ with core functions from 
  \cite{thdjcp90,kapthdjcp117} \\
  & Al--Ar & \\
aug-cc-pVXZ (X = D,T,Q,5) & H, He, B--Ne, & cc-pVXZ, with aug. functions from 
  \cite{thdjcp90,rakthdrjhjcp96,dewthdjcp98,dewthdjcp100}.\\
  & Al--Ar, Ga--Kr & \\
aug-cc-pV6Z & H, He, B--Ne, & \cite{akwtvmthdjsm388,tmakwtdhmp96} with aug. functions from 
  \cite{akwtvmthdjsm388,tmakwtdhmp96,tmthdijqc76}.\\
  & Al--Ar & \\
aug-cc-pCVXZ (X = D,T,Q,5) & H, He, B--F, & aug-cc-pVXZ, cc-pCVXZ and \cite{dewthdjcp103,kapthdjcp117}.\\
  & Ne, Al-Ar & \emph{Note}: Ne only available for TZ and QZ \\
$n$-aug-cc-pVXZ ($n$ = d,t,q) & as aug-cc-pVXZ & aug-cc-pVXZ. See Sec.~\ref{sec:molbasis} \\
$n$-aug-cc-pCVXZ ($n$ = d,t,q) & as aug-cc-pCVXZ & aug-cc-pCVXZ. See Sec.~\ref{sec:molbasis} \\
\hline
\newpage
cc-pVXZdenfit (X=T,Q,5) & H, B--F, Al--Cl & Turbomole program \\
cc-pVXZ-DK (X = D,T,Q,5) & H,He,B--Ne, & \cite{thdjcp90,dewthdjcp100,dewthdjcp98,akwdewkapthdjcp110} 
  cc-pVXZ re-contracted for \\
  & Al--Ar, Ga--Kr & Douglas-Kroll calculations.\\
\hline
\multicolumn{2}{l}{\bf{\emph{Frank Jensen's polarization-consistent basis 
sets\index{basis set!polarization-consistent}\index{polarization-consistent basis set}}}} \\
pc-$n$ ($n$ = 0,1,2,3,4)  & H, C--F & \cite{fjjcp115,fjjcp116} \\
                      & Si--Cl & \cite{fjthjcp121} \\
apc-$n$ ($n$ = 0,1,2,3,4) & H, C--F & pc-$n$, with aug. functions from \cite{fjjcp117} \\
                      & Si--Cl & \cite{fjthjcp121} \\
\hline
\bf{\emph{Ahlrich's Turbomole basis sets}} & & \\
\multicolumn{3}{l}{\emph{Note}: See text above -- Turbomole basis sets are preferred over EMSL sets} \\
Turbomole-SV & H--Kr & \cite{ashhrajcp97} Turbomole SV basis\\
Turbomole-XZ (X=D,T) & H--Kr & \cite{ashhrajcp97} Turbomole DZ,TZ basis\\
Turbomole-TZV & H--Kr & \cite{aschrajcp100} Turbomole TZV basis \\
Turbomole-XP & H--Kr & \cite{ashhrajcp97,aschrajcp100} Turbomole polariz. basis \\
 ~~~~(X=SV,DZ,TZ,TZV) & & \\
Turbomole-TZVPP & H--Kr & \cite{aschrajcp100} Turbomole TZVPP basis \\
Turbomole-TZVPPP & H--He,B--Ne,Al--Ar & \cite{aschrajcp100} Turbomole TZVPPP basis \\
Ahlrichs-VXZ (X=D,T) & H--Kr & \cite{ashhrajcp97} From EMSL \\
Ahlrichs-pVDZ & H--Kr & \cite{ashhrajcp97} From EMSL: polariz. functions unpublished.\\
\hline
%
\bf{\emph{Huzinaga basis sets}} & & \\
Huz-II & H, C--F, P, S & \cite{wkijc19,mswkjcp76,huzinagaintern} All the Huz basis sets are of\\
Huz-IIsu2 & H, C--F, P, S & \cite{wkijc19,mswkjcp76,huzinagaintern} approximate valence TZ quality\\
Huz-III & H, C--F, P, S & \cite{wkijc19,mswkjcp76,huzinagaintern} \\
Huz-IIIsu3 & H, C--F, P, S & \cite{wkijc19,mswkjcp76,huzinagaintern}\\
Huz-IV & H, C--F, P, S & \cite{wkijc19,mswkjcp76,huzinagaintern}\\
Huz-IVsu4 & H, C--F, P, S & \cite{wkijc19,mswkjcp76,huzinagaintern}\\
\hline
GAMESS-VTZ & H, Be--Ne, Na--Ar & \cite{thdjcp55,admgscjcp72,ajhwjcp52} From GAMESS program \\
GAMESS-PVTZ & H, Be--Ne & \cite{thdjcp55,admgscjcp72,ajhwjcp52} From GAMESS program \\
%
McLean-Chandler-VTZ & Na--Ar & \cite{admgscjcp72} \\
Wachtersa+f & Sc--Cu & \cite{ajhwjcp52,wachterintern1969} with $f$ functions from \cite{cwbsrllabjcp91} \\
Sadlej-pVTZ & H--I & \cite{ajstca79,ajsmujmst234,ajstca81,ajstca81-2} \\
  &  & \emph{Note}: No rare-gas, B, Al, Ga or In basis sets\\
%
Sadlej-pVTZ-J & H, C--O, S & \cite{pfpgaaspasjcp115} Sadlej-pVTZ optimized for NMR calcs.\\
aug-cc-pVTZ-J & H, C--F, S & \cite{pfpgaaspasjcp115} aug-cc-pVTZ optimized for NMR calcs. \\
\hline
\multicolumn{2}{l}{\bf{\emph{ANO basis sets\index{basis!ANO} -- see Sec.~\ref{sec:molbasis}.}}} & \\
NQvD & & \cite{nqvdref} \\
Almlof-Taylor-ANO & H--Ne & \cite{japrtjcp86} Almlof and Taylor ANO \\
NASA-Ames-ANO & H, B--Ne, & \cite{japrtjcp86,cwbsrlaktca77} \\
  & Al, P, Ti, Fe, Ni & \\
ano-1 & H--Ne & \cite{powpambortca77} Roos Augmented ANO basis sets\\
ano-2 & Na--Ar & \cite{powbjpbortca79} \\
ano-3 & Sc--Zn & \cite{rpamminpowbortca92} \\
ano-4 & H--Kr & \cite{kpbdpowbortca90} Roos ANO basis sets\\
\hline
raf-r & O, Y--Pd, Hf--Tl, & Wahlgren/Faegri Relavistic Basis Set\\
 & Po, Th, U & \\
\hline
\multicolumn{3}{l}{\bf{\emph{Effective core potential (ECP) basis sets}}} \\
\multicolumn{3}{l}{\emph{Note}: ecp-sdd-DZ is Stuttgart ECP valence basis sets included in previous Dalton releases,}\\
\multicolumn{3}{l}{see http://www.theochem.uni-stuttgart.de/pseudopotentials/index.en.html.}\\
\multicolumn{3}{l}{Others are from EMSL with the EMSL name given in the third column,}\\
\multicolumn{3}{l}{please check EMSL for complete reference.}\\
aug\_cc\_pvdz\_pp & Cu-Kr, Y-Xe, & aug-cc-pVDZ-PP\\
                  & Hf-Rn &\\
aug\_cc\_pvtz\_pp & Cu-Kr, Y-Xe, & aug-cc-pVTZ-PP\\
                  & Hf-Rn &\\
aug\_cc\_pvqz\_pp & Cu-Kr, Y-Xe, & aug-cc-pVQZ-PP\\
                  & Hf-Rn &\\
aug\_cc\_pv5z\_pp & Cu-Kr, Y-Xe, & aug-cc-pV5Z-PP\\
                  & Hf-Rn &\\
cc\_pvdz\_pp & Cu-Kr, Y-Xe, & cc-pVDZ-PP\\
             & Hf-Rn &\\
cc\_pvtz\_pp & Cu-Kr, Y-Xe, & cc-pVTZ-PP\\
             & Hf-Rn &\\
cc\_pvqz\_pp & Cu-Kr, Y-Xe, & cc-pVQZ-PP\\
             & Hf-Rn &\\
cc\_pv5z\_pp & Cu-Kr, Y-Xe, & cc-pV5Z-PP\\
             & Hf-Rn &\\
cc\_pwcvdz\_pp & Cu, Zn, Y-Cd, & cc-pwCVDZ-PP\\
               & I, Hf-Hg &\\
cc\_pwcvtz\_pp & Cu, Zn, Y-Cd, & cc-pwCVTZ-PP\\
               & I, Hf-Hg &\\
cc\_pwcvqz\_pp & Cu, Zn, Y-Cd, & cc-pwCVQZ-PP\\
               & I, Hf-Hg &\\
cc\_pwcv5z\_pp & Cu, Zn, Y-Cd, & cc-pwCV5Z-PP\\
               & Hf-Hg &\\
crenbl\_ecp & H, Li-Uus & CRENBL ECP\\
crenbs\_ecp & Sc-Zn, Y-Cd, La, & CRENBS ECP\\
            & Hf-Rn, Rf-Uus &\\
def2\_qzvp & H-La, Hf-Rn & Def2-QZVP\\
def2\_qzvpp & H-La, Hf-Rn & Def2-QZVPP\\
def2\_sv\_p & H-La, Hf-Rn & Def2-SV(P)\\
def2\_svp & H-La, Hf-Rn & Def2-SVP\\
def2\_tzvp & H-La, Hf-Rn & Def2-TZVP\\
def2\_tzvpp & H-La, Hf-Rn & Def2-TZVPP\\
dzq & Y-Ag & DZQ\\
ecp-sdd-DZ & Li, B-F, Na-Cl, & Stuttgart ECP valence basis sets\\
           & K-Sc, Cr-Kr, Sr, &\\
           & Zr-Ba, Hf-Bi &\\
hay\_wadt\_mb\_n1\_ecp & K-Cu, Rb-Ag, & Hay-Wadt MB (n+1) ECP\\
                       & Cs-Au &\\
hay\_wadt\_vdz\_n1\_ecp & K-Cu, Rb-Ag, & Hay-Wadt VDZ (n+1) ECP\\
                        & Cs-Au &\\
lanl08 & Na-La, Hf-Bi & LANL08\\
lanl08\_f & Sc-Cu, Y-Ag, & LANL08(f)\\
          & La, Hf-Au &\\
lanl08\_p & Sc-Zn & LANL08+\\
lanl08d & Si-Cl, Ge-Br, & LANL08d\\
        & Sn-I, Pb-Bi &\\
lanl2dz\_ecp & H, Li-La, Hf-Au, & LANL2DZ ECP\\
             & Pb-Bi, U-Pu &\\
lanl2dzdp\_ecp & C-F, Si-Cl, & LANL2DZdp ECP\\
               & Ge-Br, Sn-I, &\\
               & Pb-Bi &\\
lanl2tz & Sc-Zn, Y-Cd, & LANL2TZ\\
        & La, Hf-Hg &\\
lanl2tz\_f & Sc-Cu, Y-Ag, & LANL2TZ(f)\\
           & La, Hf-Au &\\
lanl2tz\_p & Sc-Zn & LANL2TZ+\\
modified\_lanl2dz & Sc-Cu, Y-Ag, & modified LANL2DZ\\
                  & La, Hf-Au &\\
sbkjc\_polarized\_p\_2d\_lfk & H-Ca, Ge-Sr, & SBKJC Polarized (p,2d) - LFK\\
                             & Sn-Ba, Pb-Rn &\\
sbkjc\_vdz\_ecp & H-Ce, Hf-Rn & SBKJC VDZ ECP\\
sdb\_aug\_cc\_pvqz & Ga-Br, In-I & SDB-aug-cc-pVQZ\\
sdb\_aug\_cc\_pvtz & Ga-Br, In-I  & SDB-aug-cc-pVTZ\\
sdb\_cc\_pvqz & Ga-Kr, In-Xe & SDB-cc-pVQZ\\
sdb\_cc\_pvtz & Ga-Kr, In-Xe & SDB-cc-pVTZ\\
stuttgart\_rlc\_ecp & Li-Ca, Zn-Sr, & Stuttgart RLC ECP\\
                    & In-Ba, Hg-Rn, &\\
                    & Ac-Lr &\\
stuttgart\_rsc\_1997\_ecp & K-Zn, Rb-Cd, & Stuttgart RSC 1997 ECP\\
                          & Cs-Ba, Ce-Yb, &\\
                          & Hf-Hg, Ac-Lr, &\\
                          & Db &\\
stuttgart\_rsc\_ano\_ecp & La-Lu & Stuttgart RSC ANO/ECP\\
stuttgart\_rsc\_segmented\_ecp & La-Lu & Stuttgart RSC Segmented/ECP\\
\hline
\end{longtable}

In the following, we give the comprehensive list of all ECPs included in the current release,
together with a list of the elements supported. The ecp-sdd-DZ is Stuttgart
ECPs included in previous Dalton releases~(see http://www.theochem.uni-stuttgart.de/pseudopotentials/index.en.html),
while others are from EMSL, please check EMSL for the complete reference to be cited.

\setlongtables
\begin{longtable}{lll}
\multicolumn{3}{}{\bf ECPs included in \dalton\ distribution.} \\ 
\hline \hline
\bf{ECP name} & \bf{Elements} & \bf{EMSL name}\\
\hline\hline
\endfirsthead
\multicolumn{3}{}{\bf{ECPs included in \dalton\ distribution.}} \\
\hline \hline
\bf{ECP name} & \bf{Elements} & \bf{EMSL name}\\
\hline \hline
\endhead
\endfoot \endlastfoot
aug\_cc\_pvdz\_pp & Cu-Kr, Y-Xe, Hf-Rn & aug-cc-pVDZ-PP\\
aug\_cc\_pvtz\_pp & Cu-Kr, Y-Xe, Hf-Rn & aug-cc-pVTZ-PP\\
aug\_cc\_pvqz\_pp & Cu-Kr, Y-Xe, Hf-Rn & aug-cc-pVQZ-PP\\
aug\_cc\_pv5z\_pp & Cu-Kr, Y-Xe, Hf-Rn & aug-cc-pV5Z-PP\\
cc\_pvdz\_pp & Cu-Kr, Y-Xe, Hf-Rn & cc-pVDZ-PP\\
cc\_pvtz\_pp & Cu-Kr, Y-Xe, Hf-Rn & cc-pVTZ-PP\\
cc\_pvqz\_pp & Cu-Kr, Y-Xe, Hf-Rn & cc-pVQZ-PP\\
cc\_pv5z\_pp & Cu-Kr, Y-Xe, Hf-Rn & cc-pV5Z-PP\\
cc\_pwcvdz\_pp & Cu, Zn, Y-Cd, I, Hf-Hg & cc-pwCVDZ-PP\\
cc\_pwcvtz\_pp & Cu, Zn, Y-Cd, I, Hf-Hg & cc-pwCVTZ-PP\\
cc\_pwcvqz\_pp & Cu, Zn, Y-Cd, I, Hf-Hg & cc-pwCVQZ-PP\\
cc\_pwcv5z\_pp & Cu, Zn, Y-Cd, Hf-Hg & cc-pwCV5Z-PP\\
crenbl\_ecp & Li-Uus & CRENBL ECP\\
crenbs\_ecp & Sc-Zn, Y-Cd, La, Hf-Rn, Rf-Uus & CRENBS ECP\\
def2\_qzvp & Rb-La, Hf-Rn & Def2-QZVP\\
def2\_qzvpp & Rb-La, Hf-Rn & Def2-QZVPP\\
def2\_sv\_p & Rb-La, Hf-Rn & Def2-SV(P)\\
def2\_svp & Rb-La, Hf-Rn & Def2-SVP\\
def2\_tzvp & Rb-La, Hf-Rn & Def2-TZVP\\
def2\_tzvpp & Rb-La, Hf-Rn & Def2-TZVPP\\
dzq & Y-Ag & DZQ\\
ecp-sdd-DZ & Li-Mg, Si-Ce, Nd, Sm-Tb, & Stuttgart ECPs\\
           & Ho, Yb-Bi, Rn &\\
hay\_wadt\_mb\_n1\_ecp & K-Cu, Rb-Ag, Cs-La, Hf-Au & Hay-Wadt MB (n+1) ECP\\
hay\_wadt\_vdz\_n1\_ecp & K-Cu, Rb-Ag, Cs-La, Ta-Au & Hay-Wadt VDZ (n+1) ECP\\
lanl08 & Na-La, Hf-Bi & LANL08\\
lanl08\_f & Sc-Cu, Y-Ag, La, Hf-Au & LANL08(f)\\
lanl08\_p & Sc-Zn & LANL08+\\
lanl08d & Si-Cl, Ge-Br, Sn-I, Pb-Bi & LANL08d\\
lanl2dz\_ecp & Na-La, Hf-Au, Pb-Bi, U-Pu & LANL2DZ ECP\\
lanl2dzdp\_ecp & Si-Cl, Ge-Br, Sn-I, Pb-Bi & LANL2DZdp ECP\\
lanl2tz & Sc-Zn, Y-Cd, La, Hf-Au & LANL2TZ\\
lanl2tz\_f & Sc-Cu, Y-Ag, La, Hf-Au & LANL2TZ(f)\\
lanl2tz\_p & Sc-Zn & LANL2TZ+\\
modified\_lanl2dz & Sc-Cu, Y-Ag, La, Hf-Au & modified LANL2DZ\\
sbkjc\_polarized\_p\_2d\_lfk & Li-Ca, Ge-Sr, Sn-Ba, Pb-Rn & SBKJC Polarized (p,2d) - LFK\\
sbkjc\_vdz\_ecp & Li-Ce, Hf-Rn & SBKJC VDZ ECP\\
sdb\_aug\_cc\_pvqz & Ga-Br, In-I & SDB-aug-cc-pVQZ\\
sdb\_aug\_cc\_pvtz & Ga-Br, In-I & SDB-aug-cc-pVTZ\\
sdb\_cc\_pvqz & Ga-Kr, In-Xe & SDB-cc-pVQZ\\
sdb\_cc\_pvtz & Ga-Kr, In-Xe & SDB-cc-pVTZ\\
stuttgart\_rlc\_ecp & Li-Ca, Zn-Sr, In-Ba, Hg-Rn, Ac-Lr & Stuttgart RLC ECP\\
stuttgart\_rsc\_1997\_ecp & K-Zn, Rb-Cd, Cs-Ba, Cs-Yb, & Stuttgart RSC 1997 ECP\\
                          & Hf-Hg, Ac-Lr, Db &\\
stuttgart\_rsc\_ano\_ecp & La-Lu & Stuttgart RSC ANO/ECP\\
stuttgart\_rsc\_segmented\_ecp & La-Lu & Stuttgart RSC Segmented/ECP\\
\hline
\end{longtable}
