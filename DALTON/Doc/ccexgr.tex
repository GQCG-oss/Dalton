
%%%%%%%%%%%%%%%%%%%%%%%%%%%%%%%%%%%%%%%%%%%%%%%%%%%%%%%%%%%%%%%%%%%
\section{Excited-state first-order properties: \Sec{CCEXGR}}\label{sec:ccexgr}
%%%%%%%%%%%%%%%%%%%%%%%%%%%%%%%%%%%%%%%%%%%%%%%%%%%%%%%%%%%%%%%%%%%
\index{expectation values!excited state!Coupled Cluster}
\index{one-electron properties!excited state!Coupled Cluster}
\index{properties!one-electron}
\index{first-order properties!excited state!Coupled Cluster}
\index{properties!first-order}

In the \Sec{CCEXGR} section the input that is specific for coupled cluster 
response calculation of excited-state first-order properties is read in.
This section includes presently for example calculation of excited-state 
dipole moments and second moments of the electronic charge distribution.
In many cases the information generated in this way is helpful 
for making qualitative assignments of the electronic states, for example
in conjunction with the oscillator strengths and the 
orbital analysis of the response eigenvectors presented in the output.

The excited-state properties are available for the models CCS, CC2 and CCSD,
but only for singlet states.
%Publications that report results obtained this module should cite Ref.\ \cite{Christiansen:CCLR}.
The theoretical background for the implementation is detailed in Ref.\ \cite{Christiansen:CCLR,Christiansen:QEL}.
This section has to be used in connection with \Sec{CCEXCI} for the calculation of excited states.

The properties calculated are in the approach now generally known as coupled cluster
response---for these frequency independent properties this coincides with the so-called
orbital-unrelaxed energy derivatives (and thus the orbital-unrelaxed finite-field result)
for the excited-state total energies as obtained by the sum of the CC ground state energy
and the CC response excitation energy.

{\bf Note of caution:}
Default in this section is therefore orbital {\it unrelaxed}, while for the ground state 
first order properties \Sec{CCFOP} default is {\it relaxed}. 
To find results in a consistent approximation turn orbital relaxation off
for the ground state (for CCS, CC2, CCSD) calculation.

\begin{center}
\fbox{
\parbox[h][\height][l]{12cm}{
\small
\noindent
{\bf Reference literature:}
\begin{list}{}{}
\item O.~Christiansen, A.~Halkier, H.~Koch, P.~J{\o}rgensen, and T.~Helgaker \newblock {\em J.~Chem.~Phys.}, {\bf 108},\hspace{0.25em}2801, (1998).
\end{list}
}}
\end{center}

\begin{description}
\item[\Key{ALLONE}]
        Calculate all of the above-mentioned excited state properties (all the
        above-mentioned property integrals are needed).
%
\item[\Key{DIPOLE}]
        Calculate the excited state permanent molecular electric dipole moment
        (\verb+DIPLEN+ integrals).
        \index{electric dipole}
        \index{dipole moment}
%
\item[\Key{NQCC}]
        Calculate the excited state electric field gradients at the nuclei
        (\verb+EFGCAR+ integrals).
        \index{electric field!gradient}
%
\item[\Key{OPERAT}] \verb| |\newline
\verb|READ (LUCMD,'(1X,A8)') LABPROP|\newline
        Calculate the excited state electronic contribution to the property defined
        by the operator label \verb+LABPROP+ (corresponding
        \verb+LABPROP+ integrals needed).

\item[\Key{QUADRU}]
        Calculate the excited state permanent traceless molecular electric
        quadrupole moment (\verb+THETA+ integrals). Note that the
        origin is the origin of the coordinate system specified
        in the MOLECULE.INP file.
        \index{electric quadrupole}
        \index{quadrupole moment}
%
\item[\Key{RELCOR}]
        Calculate excited state scalar-relativistic one-electron
        corrections to the 
        energy (\verb+DARWIN+ and \verb+MASSVELO+ integrals).
        \index{relativistic corrections!one-electron}
        \index{Darwin term!one-electron}
        \index{mass-velocity term}
%
\item[\Key{SECMOM}]
        Calculate the excited state electronic second moment of charge
        (\verb+SECMOM+ integrals).
        \index{second moment of charge}
%

\item[\Key{SELEXC}]  \verb| |\newline
\verb|READ(LUCMD,*) IXSYM,IXST|

Select which excited states the calculation of excited state properties 
are carried out for. The default is all states according to the CCEXCI input section.
When calculating selected states only,
provide a list of symmetry and state numbers (order after increasing energy in
each symmetry class).
This list is read until next input label is found.


%\item[\Key{SELXST}] 
%Effect unknown at this moment... - in fact it is obsolete and without any effect.

\end{description}

