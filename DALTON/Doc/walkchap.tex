\chapter{Potential energy surfaces}\label{ch:geometrywalks}
\index{geometry walk}\index{geometry optimization}\index{convergence!geometry}

This section describes one of the most important features of any
quantum chemical software package; locating equilibrium
geometries\index{equilibrium structure} and
transition structures\index{transition state} of molecules. In
{\dalton} this is done using either second-order trust-region\index{second-order optimization}\index{geometry optimization!second-order}\index{trust region}
optimizations~\cite{tuhjahjajpjjcp84} (energies, gradients and
Hessians are calculated) or a variety of first-order\index{first-order optimization}\index{geometry optimization!first-order}
methods~\cite{Fletcher} (only energies and gradients are
calculated). These methods
have been implemented for Hartree--Fock, Density Functional Theory,
MCSCF  and various Coupled-Cluster wave functions. For other
non-variational wave functions---such as CI---the program can 
only do first-order geometry optimizations using a
numerically\index{numerical gradient}
calculated gradient. This will be invoked automatically by the program
in case of a non-variational wave function. Some comments connected to
such geometry optimizations are collected in Sec.~\ref{sec:nonvargeom}.

For historical reasons, {\dalton} actually contains two different
modules for exploring potential energy surfaces\index{PES}\index{potential energy surface}:
\Sec{WALK} which is
a pure second-order module and \Sec{OPTIMIZE} which contains both
first- and second-order methods. While there is a lot of overlap
between the two modules, certain calculations can only be done using
one or the other of the two. The main strength of \Sec{WALK} is its
robustness, while \Sec{OPTIMIZE} focuses more on speed and
efficiency. For regular optimizations of minima and transition states,
the recommended module is \Sec{OPTIMIZE}.

One of the strengths of the {\dalton} program is the
stable algorithms for locating first-order transition states\index{transition state}\index{geometry optimization!transition state}.
As described below, this can done by one of three algorithms in
\Sec{WALK}: by trust-region second-order image surface
minimization~\cite{thcpl182}\index{image surface},
by gradient extremal walks~\cite{pjhjajthtca73}\index{gradient extremal},
or by following a specific mode~\cite{hjajpjthjcp85}\index{mode following} (all
requiring the calculation of the Hessian at every point). These
options will be discussed in more detail below. The \Sec{OPTIMIZE}
module also contains the stable second-order trust-region image
surface minimization (using analytical or approximate Hessians), as
well as a partitioned rational function method~\cite{abnajsrsjpc89}.

Another feature of the {\dalton} program system is the options for
calculating essential information about the potential energy
surface so that subsequent molecular dynamics analysis can be
made. There are two options for doing this in the program: One can
either follow an Intrinsic Reaction Coordinate
(IRC)~\cite{pjhjajthtca73}\index{IRC}\index{intrinsic reaction coordinate}, 
or solve Newton's equations for the nuclei under the
potential put up by the
electrons~\cite{theuhjajcpl173}\index{dynamics}. Whereas the first
option 
can be used for modeling the immediate surroundings of a
molecular reaction path, the second will give a precise
description of {\em one} molecular trajectory. Chemical reactions
may thus be monitored in time from an initial set of starting
conditions. The program will automatically generate a complete
description of the energy distribution into internal energies and
relative translational energies. By calculating a large number of
trajectories, a more complete description of the chemical reaction
may be obtained~\cite{rsgmtjdwrjbjcp106}.

Please note that for geometry optimizations using the Self-Consistent
Reaction Field model, certain restrictions apply, as discussed in
Sec.~\ref{sec:solventgeoopt}.

\section{Locating stationary points}

\subsection{Equilibrium geometries}
\label{sec:minimization}

\begin{center}
\fbox{
\parbox[h][\height][l]{12cm}{
\small
\noindent
{\bf Reference literature:}
\begin{list}{}{}
\item 2nd-order methods: T.U.Helgaker, J.Alml\"{o}f, H.J.Aa.Jensen,
and P.J{\o}rgensen.\newblock {\em J.Chem.Phys.}, {\bf
84},\hspace{0.25em}6266, (1986).
\item 1st-order methods: V.Bakken, and T.Helgaker, \newblock {\em J.Chem.Phys.}, {\bf 117},\hspace{0.15em} 9160 (2002).
\item Excited-state optimization: A.Cesar, H.\AA gren, T.Helgaker, P.J\o rgensen, and H.J.Aa.Jensen.\newblock {\em J.Chem.Phys.}, {\bf
95},\hspace{0.25em}5906, (1991).
\item Solvent Hessian: P.-O.\AA strand, K.V.Mikkelsen, K.Ruud and
T.Helgaker. \newblock {\em J.Phys.Chem.}, {\bf
100},\hspace{0.25em}19771, (1996).
\end{list}
}}
\end{center}

A short input file for the location of a molecular geometry
corresponding to minimum molecular energy\index{geometry optimization!equilibrium geometry}
(using a second-order optimization\index{second-order optimization}\index{geometry optimization!second-order}
method) and performing a vibrational
analysis\index{vibrational analysis} at the stationary point, as well
as determining the nuclear
shielding\index{nuclear shielding} constants at the optimized
geometry, will look like
(assuming a Hartree--Fock wave function):

\begin{verbatim}
**DALTON INPUT
.OPTIMIZE
*OPTIMIZE
.2NDORD
**WAVE FUNCTIONS
.HF
**PROPERTIES
.VIBANA
.SHIELD
**END OF DALTON INPUT
\end{verbatim}

The keyword \Key{OPTIMIZE} by default signals a search for a minimum,
that is, a stationary point on the molecular surface with Hessian
index 0. With the \Sec{OPTIMIZE}
module the second-order method has to be explicitly requested
(\Key{2NDORD})~\cite{tuhjahjajpjjcp84}, as first-order methods are the default. At the final,
optimized geometry the input in the \Sec{*PROPE} section requests a
vibrational analysis and the evaluation of nuclear shieldings.

In second-order methods the Hessian\index{Hessian} is calculated at every
geometry. The analytical Hessian naturally gives a better description
of the potential energy surface than the approximate Hessians of
first-order methods, and fewer steps will usually be needed to reach
the minimum~\cite{vbthjcp117}. However, the price one has to pay for this, is that each
iteration uses significantly more CPU time. Another advantage of
second-order methods, is that the program will automatically break the symmetry\index{symmetry!breaking}\index{geometry optimization!symmetry breaking}
of the molecule, if needed, in order to reach a minimum
(unless the user specifies that symmetry should not be broken). If
one wants to minimize a water molecule starting from a linear
geometry and using molecular symmetry, a first-order method will
``happily'' yield a linear geometry with optimized bond lengths, whereas
the second-order method will correctly decide that symmetry should be
reduced to C$_{2v}$ and minimize the energy of the molecule.
The bottom-line is, however, that while second-order methods
definitely are the more robust, they will generally be outperformed by
the much more time-efficient first-order methods.

If you are convinced that the symmetry of your starting geometry
is the correct, you may take advantage of the fact there will only
be forces in the totally symmetric modes of the molecules,
and thus only evaluate the Hessian in the totally symmetric
Cartesian symmetry coordinates. This can be achieved with the
input:

\begin{verbatim}
**DALTON INPUT
.WALK
.TOTSYM
**WAVE FUNCTIONS
.HF
**PROPERTIES
.VIBANA
.SHIELD
**END OF DALTON INPUT
\end{verbatim}

In case a vibrational analysis has been requested at the optimized
geometry, the program will recalculate the complete molecular Hessian
at the optimized geometry in order to be able to evaluate all
frequencies. Note that only the \Sec{WALK} module, indicated by the
keyword \Key{WALK}, supports the \Key{TOTSYM} keyword.
We also note that any requests for static
polarizabilities\index{polarizability}, Cioslowski population
analysis\index{population analysis}, dipole 
gradients\index{dipole gradient} or vibrational circular dichroism
(VCD)\index{vibrational circular dichroism}\index{VCD} will be ignored
as all the necessary property vectors will not be available. However,
at the final optimized geometry, the full Hessian will be evaluated,
as may then also these properties.

For direct calculations\index{direct calculation}---where the
two-electron integrals are not stored on disc---the cost of a
second-order geometry optimization algorithm may become
prohibitively expensive, despite the rather few iterations normally
needed to converge the geometry. In these cases, it may be
advantageous to use first-order geometry
optimization\index{first-order optimization}\index{geometry optimization!first-order} routines as illustrated in the following input:

\begin{verbatim}
**DALTON INPUT
.OPTIMIZE
**WAVE FUNCTIONS
.HF
**PROPERTIES
.VIBANA
.SHIELD
**END OF DALTON INPUT
\end{verbatim}

This will use the default method of \Key{OPTIMIZE}: a first-order
optimization~\cite{Fletcher} in redundant internal
coordinates~\cite{gfxfzpwtppjacs114,ppgfjcp96,vbthjcp117}\index{redundant internal coordinates}\index{coordinate system!redundant internal coordinates}
using the BFGS\index{BFGS update}\index{Hessian update!BFGS}
Hessian update. This means that
only the energy and the gradient are calculated in each iteration, and
an approximate Hessian is obtained using the gradients and the
displacements. The initial Hessian is diagonal in the redundant
internal coordinates~\cite{rlabgkpamcpl241}.
To obtain the properties at the optimized 
geometry, the Hessian has to be calculated. By looking at the
vibrational frequencies one can then verify that a true minimum has
been reached (as all frequencies should be real, corresponding to a
positive definite Hessian).

Several first-order methods have been implemented in \dalton\, the
recommended method being the Broyden-Fletcher-Goldfarb-Shanno (BFGS)
Hessian updating\index{BFGS update}\index{Hessian update!BFGS}
scheme. Though other updates may perform better in
certain cases, the BFGS update generally seems to be the most
reliable and thus is the default method for minimization~\cite{vbthjcp117}.

Without the calculation of the Hessian at the optimized geometry, one
can never be sure that the geometry is indeed a minimum, and this is the
main problem with first-order methods.

Both first- and second-order methods are based on
trust-region\index{trust region}\index{geometry optimization!trust region}
optimization. The trust-region is the region of the potential energy
surface where our quadratic model (based on an analytical or
approximate Hessian) gives a good representation of the true
surface. This region is given the shape of a
hypersphere. In the trust-region based optimization algorithm, the
radius of the trust-region is
automatically updated during the optimization by a feedback mechanism.
Occasionally, however, the potential surface may show locally large
deviations from quadratic form. This will result in a large
disagreement between the predicted energy change and the energy
calculated at the new geometry. If this deviation is larger than a
given threshold, the step will be reduced through a decrease in the
trust radius and a simple line search will be performed. A quadratic
function is fitted 
to the rejected energy and the previous energy and gradient. The
minimum of this function is then used as the new step, and this
process may be repeated.

Another consideration concerning optimizations, is the choice of\index{coordinate system}\index{geometry optimization!coordinate system}
coordinate system. Second-order methods do not seem to be very
sensitive to this, and the \Sec{WALK} module thus only uses
Cartesian coordinates. However, the coordinate system may be crucial
to the performance of first-order methods, especially when starting
from approximate Hessians. \Sec{OPTIMIZE} provides the normal Cartesian
coordinates\index{Cartesian coordinates}\index{coordinate system!Cartesian coordinates}
and redundant internal coordinates,\index{redundant internal coordinates}\index{coordinate system!redundant internal coordinates}
the latter being highly recommended when using
first-order methods. Redundant internal coordinates consists of all
bond lengths, angles and dihedral angles in a molecule, and the
redundancies are removed through projection. The default for
second-order methods is Cartesian coordinates, while redundant
internal coordinates are the default for first-order methods, but it is
possible to override this with the keywords \Key{REDINT} and
\Key{CARTES} respectively.

The \Sec{OPTIMIZE} module decides when a minimum is reached through a
set of convergence criteria\index{convergence!geometry, criteria}. These will be
set automatically depending on the accuracy of the wave function
using three different criteria, namely the change in energy
between two iterations, the norm of the molecular
gradient\index{norm of gradient} and the norm of the
step\index{norm of step}. The convergence threshold for the change
in energy is the maximum of $1.0\cdot 10^{-6}$ Hartree and two times the
convergence threshold for the wave function, whereas for the norm of
the gradient and step it is the maximum of $1.0\cdot 10^{-4}$ a.u.\ and
two times the 
convergence threshold for the wave function. Two out of these
three criteria have to decrease below their threshold before the program
declares convergence. These criteria ensure a good geometry
suitable for vibrational analysis, but may be unnecessary tight if
one only wants the energy.

The convergence criteria can be controlled through the input file. One
should keep in mind that while second-order methods are rather
insensitive to changes in these thresholds due to their quadratic
convergence, first-order methods may run into severe trouble if the
criteria are too close to the accuracy of the wave function and the
gradient. The quickest way to get somewhat looser convergence
criteria, is to use the keyword \Key{BAKER} enforcing the convergence
criteria of Baker\cite{Baker} (see the Reference Manual for
details). This will normally give energies converged to within
$1.0\cdot 10^{-6}$ Hartree.

In case of non-variational\index{non-variational wave functions}
wave functions, where a numerical 
gradient\index{numerical gradient} is used, the threshold for
convergence is reset to 
$1.0\cdot 10^{-4}$ a.u., due to possible numerical
instabilities\index{numerical gradient} occurring in the 
finite difference\index{finite difference} methods employed to
determine the molecular gradient\index{molecular gradient}.

We also note that for the optimization of excited
states\index{excited state!geometry optimization}
where the state of interest is not the lowest of its
symmetry, most of the parameters controlling the trust-region
optimization method will be reset to much more conservative
values, as one in such calculations will lose the possibility of
back-stepping during the wave function optimization, as well as
to ensure that the start wave function at the new geometry
is within the local region of the optimized wave function.

Calculations involving large basis sets may be very time-consuming,
and it becomes very important to start the optimization from a decent
geometry. One way to solve this is through
preoptimization\index{preoptimization}\index{geometry optimization!preoptimization},
that is, one or more smaller basis sets are first used to optimize the
molecule. \Sec{OPTIMIZE} supports the use of up to ten different
basis sets for preoptimization. This may also be used as a very
effective way to get the optimized energy of a molecule for a series
of basis sets. One may also want to calculate the energy of a molecule
using a large basis set at the optimized geometry of a smaller basis
set, and this is also supported in {\dalton}, but limited to only one
single-point basis set. Both these features are illustrated in the
input file below:

\begin{verbatim}
**DALTON INPUT
.OPTIMIZE
*OPTIMIZE
.PREOPT
2
STO-3G
4-31G
.SP BAS
d-aug-cc-pVTZ
**WAVE FUNCTIONS
.HF
**END OF DALTON INPUT
\end{verbatim}

As one can see, two small basis sets have been chosen 
for\index{preoptimization}\index{geometry optimization!preoptimization}
preoptimization. Note that they will be
used in the order they appear in the input file, so one should
sort the sets and put the smaller one at the top for maximum
efficiency. The ``main'' basis set, that is, the one that will be
used for the final optimized geometry, is specified in the file
\molinp\ as usual. After the last optimization, the energy of
the molecule will be calculated at the final geometry with the
d-aug-cc-pVTZ basis. Please note that these features will work only
when using basis sets from the basis set library.

It is possible to control the optimization procedure more closely
by giving the program instructions on how to do the different parts of
the calculation. Below we have listed all the modules that may
affect the geometry optimization, as well as a short description of
which part of the calculation that module controls. The reader is
referred to the section describing each module for a closer
description of the different options. These sections may, with the
exception of the \Sec{OPTIMIZE} and \Sec{WALK} section, be specified
in any of the modules \Sec{*START}, \Sec{*EACH STEP}
and \Sec{*PROPERTIES}. The \Sec{OPTIMIZE} and \Sec{WALK} are
to be placed in the \Sec{*DALTON} module, and will apply to the
entire calculation.

\begin{description}
\item[\Sec{GEOANA}] Describes what kind  of information about the
molecular geometry is to be printed.
\item[\Sec{GETSGY}] Controls the set up of the right-hand sides
(gradient terms).
\item[\Sec{NUCREP}] Controls the calculation of the nuclear contributions.
\item[\Sec{ONEINT}] Controls the calculation of one-electron contributions.
\item[\Sec{OPTIMIZE}] Controls the first- and second-order
optimization methods (both minima and transition states).
\item[\Sec{RELAX}] Controls the adding of solution and right-hand side vectors
into relaxation contributions.
\item[\Sec{REORT}] Controls the calculation of reorthonormalization terms.
\item[\Sec{RESPON}] Controls the solution of the geometric response equations.
\item[\Sec{TROINV}] Controls the use of translation and rotational invariance.
\item[\Sec{TWOEXP}] Controls the calculation of two-electron
expectation values.
\item[\Sec{VIBANA}] Set up the vibrational and rotational analysis of the
molecule.
\item[\Sec{WALK}] Controls the walk (see also ``Locating transition
states'' and ``Doing Dynamical walks'').
\end{description}

\subsection{Transition states using the image method}\index{image surface}
\label{sec:image}

\begin{center}
\fbox{
\parbox[h][\height][l]{12cm}{
\small
\noindent
{\bf Reference literature:}
\begin{list}{}{}
\item T.Helgaker. \newblock {\em Chem.Phys.Lett.}, {\bf
182},\hspace{0.25em}503, (1991).
\end{list}
}}
\end{center}

\index{transition state}\index{geometry optimization!transition state}
Transition states are found as saddle points
on the potential energy surface\index{image surface}.
The simplest way of locating a first-order transition state, which are the
chemically most interesting ones, is to use the trust-region
image\index{image surface}\index{trust region} minimization method
described in Ref.~\cite{thcpl182}. Such geometry optimizations may
be considered as a special case of walks on second-order surfaces, and
can be done using either the \Sec{WALK} module or the \Sec{OPTIMIZE}
module. Just like for minimization the former uses a pure second-order
method (analytical Hessians calculated at every geometry), while the
latter gives you a choice of first- and second-order methods or
combinations of the two.

A second-order optimization of a transition state can be requested by
either adding \Key{SADDLE} and \Key{2NDORD} in the \Sec{OPTIMIZE}
section:
\begin{verbatim}
**DALTON INPUT
.OPTIMIZE
*OPTIMIZE
.SADDLE
.2NDORD
**WAVE FUNCTIONS
.HF
**END OF DALTON INPUT
\end{verbatim}
or by adding the keyword \Key{IMAGE} in the \Sec{WALK} section:
\begin{verbatim}
**DALTON INPUT
.WALK
*WALK
.IMAGE
**WAVE FUNCTIONS
.HF
**END OF DALTON INPUT
\end{verbatim}

Any property may of course be specified at all stages of the
optimization in the same fashion as for geometry minimizations.

The principle behind the trust-region image minimization is simple. A
first-order transition state is characterized by having one negative
Hessian\index{Hessian} eigenvalue. By reversing the sign of this
eigenvalue, we have
taken the mirror image of our potential surface along the associated
mode, thus turning our problem into an ordinary
minimization problem. A global one-to-one correspondence between the
image surface and our potential energy surface is only valid for a
second-order surface, but in general the lack of a global one-to-one
correspondence seldom gives any problems.

The advantage of the trust-region image optimization as compared to
for instance following gradient extremals\index{gradient extremal}
lies mainly in the
fact that we may take advantage of well-known techniques for
minimization. In addition, the method does not need to be started at
a stationary point of the potential surface which is necessary when
following a gradient extremal (in fact, when using \Sec{OPTIMIZE} the
starting geometry should {\em not} be a minimum). We
have so far not experienced that the trust-region image optimization
fails to locate a first order transition state, even though this is by
no means globally guaranteed from the approach. Note that the
first-order saddle-points normally obtained using the image method
starting at a minimum, often corresponds to conformational transition
states, and thus not necessarily to the chemically most interesting
transition states.

There are two approaches for locating several first-order saddle
points with the trust-region image optimization. One may take
advantage of the fact the image method is not dependent upon starting
at a stationary point, and thus start the image minimization from
several different geometries, and thus hopefully ending up at different
first-order saddle points\index{saddle-point}, as the lowest eigenmode
at different
regions of the potential energy surface may lead to different
transition states.

The other approach is to request
that not the lowest mode, but some other eigenmode is to be inverted.
This can be achieved by explicitly giving the mode which is to be
inverted through the keyword \Key{MODE}. This keyword is the same in
both the \Sec{WALK} and the \Sec{OPTIMIZE} module. However, one
should keep in mind that there will be crossings where a given mode
will switch from the chosen mode to a lower mode. However, what will happen
in these crossing points cannot be predicted in advance. Thus, for such
investigations, the gradient extremal approach may prove equally well
suited. Let us give an example of an input for a trust-region
image optimization where the third mode is inverted:

\begin{verbatim}
**DALTON INPUT
.WALK
*WALK
.IMAGE
.MODE
 3
**WAVE FUNCTIONS
.HF
**END OF DALTON INPUT
\end{verbatim}

\subsection{Transition states using first-order
methods}\label{sec:saddle1stord}

While second-order methods are robust, we have already pointed out
in section~\ref{sec:minimization} that Hessians might be expensive to
compute. The \Sec{OPTIMIZE} module therefore provides first-order
methods for locating transition states, where approximate rather than
exact Hessians are used. The step control method is, however, the same
trust-region image optimization.

When locating transition states it is important to have a good
description of the mode that should be maximized ({\it i.e.}~have a negative
eigenvalue). It is therefore not recommended to start off with a
simple model Hessian, but rather to calculate the initial Hessian
analytically. Alternatively it can be calculated using a smaller basis
set/cheaper wave function and then be read in. The minimal input for a
transition state optimization using \Sec{OPTIMIZE} is:
\begin{verbatim}
**DALTON INPUT
.OPTIMIZE
*OPTIMIZE
.SADDLE
**WAVE FUNCTIONS
.HF
**END OF DALTON INPUT
\end{verbatim}
This will calculate the Hessian at the initial geometry, then update
it using Bofill's
update~\cite{jmbjcc15}\index{Bofill's update}\index{Hessian update!Bofill's update}
(the BFGS update is not suitable since
it tends towards a positive definite Hessian). As for minimizations,
redundant internal coordinates are used by default for first-order
methods.

It is highly recommended that a Hessian calculation/vibrational
analysis be performed once a stationary point has been found, to
verify that it's actually a first-order transition state. There should
be one and only one negative eigenvalue/imaginary frequency.

If no analytical second derivatives (Hessians) are available, it is
still possible to attempt a saddle point optimization by starting from
a model Hessian indicated by the keyword \Key{INIMOD}:
\begin{verbatim}
**DALTON INPUT
.OPTIMIZE
*OPTIMIZE
.INIMOD
.SADDLE
**WAVE FUNCTIONS
.HF
**END OF DALTON INPUT
\end{verbatim}
Provided the starting geometry is reasonably near the transition
state, such optimization will usually converge correctly. If not, it
is usually a good idea to start from different geometries and also to
try to follow different Hessian modes, as described in
section~\ref{sec:image} (through the \Key{MODE} keyword).



\subsection{Transition states following a gradient
extremal}\label{sec:gradext}

\begin{center}
\fbox{
\parbox[h][\height][l]{12cm}{
\small
\noindent
{\bf Reference literature:}
\begin{list}{}{}
\item P.J{\o}rgensen, H.J.Aa.Jensen, and T.Helgaker. \newblock {\em
Theor.Chem.Acta}, {\bf 73},\hspace{0.25em}55, (1988).
\end{list}
}}
\end{center}

\index{gradient extremal}
A gradient extremal is defined as a locus of points in the contour
space where the gradient is extremal~\cite{pjhjajthtca73}. These
gradient extremals connect
stationary points on a molecular potential energy surface and are
locally characterized by requiring that the molecular gradient is an
eigenvector of the mass-weighted molecular Hessian at each point on
the line. From a stationary point there will be gradient extremals
leaving in all normal coordinate directions, and stationary points
on a molecular potential surface may thus be characterized by
following these gradient extremals. The implementation of this approach
in \dalton\ is described in Ref.~\cite{pjhjajthtca73}.

Before discussing more closely which keywords are of importance in
such a calculation and how they are to be used, we give an example of
a typical gradient extremal input in a search for a first-order
transition state along the second-lowest mode of a mono-deuterated
ethane\index{ethane} molecule:

\begin{verbatim}
**DALTON INPUT
.WALK
*WALK
.GRDEXT
.INDEX
 1
.MODE
 2
**WAVE FUNCTIONS
.HF
**END OF DALTON INPUT
\end{verbatim}


The request for a gradient extremal\index{gradient extremal}
calculation is controlled by the {\Key{GRDEXT}}, and in this example
we have chosen to follow the second-lowest mode, as specified by the
\Key{MODE} keyword.

As the calculation of the gradient extremal uses mass-weighted
coordinates\index{mass-weighted coordinates}, it is recommended to
specify the isotopic constitution\index{isotopic constitution} of
the molecule. If none is specified, the most abundant
isotope of each atom is used by default. The isotopic constitution of
a molecule is given in the \molinp\ file as described in Chapter~\ref{ch:molinp}.

A requirement in a gradient extremal calculation is that the
calculation is started on a gradient extremal. In practice this is
most conveniently ensured by starting at a stationary point, a minimum
or a transition state. The index\index{Hessian!index} of the critical point
--- that is,
the number of negative Hessian eigenvalues\index{Hessian!eigenvalue}
--- sought, need to
be specified (by the keyword \Key{INDEX}), as the calculation would
otherwise continue until a critical point with index zero
(corresponding to a minimum), is found.

\subsection{Level-shifted mode-following}\label{sec:modfol}

\begin{center}
\fbox{
\parbox[h][\height][l]{12cm}{
\small
\noindent
{\bf Reference literature:}
\begin{list}{}{}
\item C.J.Cerjan, and W.H.Miller. \newblock {\em
J.Chem.Phys.}, {\bf 75},\hspace{0.25em}2800, (1981).
\item H.J.Aa.Jensen, P.J{\o}rgensen, and T.Helgaker. \newblock {\em
J.Chem.Phys.}, {\bf 85},\hspace{0.25em}3917, (1986).
\end{list}
}}
\end{center}

The input needed for doing a level-shifted mode
following\index{mode following} is very similar to the input for
following a gradient extremal\index{gradient extremal}, and the
keyword that is needed in order to invoke this kind of calculation is
\Key{MODFOL}. As for gradient extremals, we need to specify which
mode we follow. However, a mode following does not use mass-weighted
molecular coordinates\index{mass-weighted coordinates} as default, and
isotopic composition of the
molecule is therefore not needed. Note, however, that mass-weighted
coordinates can be requested through the keyword \Key{MASSES} as
described in the input section for the \Sec{WALK} module. A typical
input following the third mode will thus look like:

\begin{verbatim}
**DALTON INPUT
.WALK
*WALK
.MODFOL
.INDEX
 1
.MODE
 3
**WAVE FUNCTIONS
.HF
**END OF DALTON INPUT
\end{verbatim}

The level-shifted mode-following uses an algorithm similar to the
one used in the ordinary geometry optimization of a molecule, but
whereas one in minimizations chooses a step so that the level
shift parameter is less than the lowest eigenvalue of the
molecular Hessian\index{Hessian}, this level shift parameter is
chosen to be in-between the eigenvalues $\lambda_{t-1}$ and
$\lambda_{t}$ if we are following mode number $t$. This approach
was pioneered by Cerjan and
Miller~\cite{cjcwhmjcp75}, and is also described in
Ref.~\cite{hjajpjthjcp85}. As for the gradient extremal approach,
higher-order transition states\index{transition state} can be
requested through the use of the keyword \Key{INDEX}.

Note that it may often be necessary to start the mode-following
calculation by stepping out of the stationary point along the
mode of interest using the keyword \Key{EIGEN} in the
\Key{WALK} module. We refer to the reference manual for a further
description of this option.

The index\index{Hessian!index} of the critical point---that is,
the number of negative Hessian
eigenvalues\index{Hessian!eigenvalue}---sought, need to 
be specified (by the keyword \Key{INDEX}), as the calculation would
otherwise continue until a critical point with index zero
(corresponding to a minimum), is found.

\section{Trajectories and Dynamics}

\subsection{Intrinsic reaction coordinates}\label{sec:irc}

\begin{center}
\fbox{
\parbox[h][\height][l]{12cm}{
\small
\noindent
{\bf Reference literature:}
\begin{list}{}{}
\item K.Fukui. \newblock {\em Acc.Chem.Res.}, {\bf 14},\hspace{0.25em}363, (1981).
\end{list}
}}
\end{center}

A tool that has proved valuable in the study of molecular
dynamics\index{dynamics}
is the use of steepest descent\index{steepest descent}-based algorithms
for following a molecular reaction from a transition
state\index{transition state} towards a minimum. One of the
most successful ways of doing this is the Intrinsic Reaction Coordinate (IRC)
approach~\cite{kfacr14}\index{IRC}\index{intrinsic reaction
coordinate}. The IRC is calculated by taking small steps along the
negative gradient in a mass-weighted 
coordinate\index{mass-weighted coordinates} system. In {\dalton}, the
size of the step is adjusted with the use of the trust-region based
algorithm. In order to get a sufficiently accurate potential energy surface,
rather small steps must be taken, and the default trust radius is thus reset
to 0.020 when an IRC calculation is being done.

In many respects the input to an IRC calculation is very similar to
the input for a trust-region image\index{image surface} optimization,
and a typical input would look like:

\begin{verbatim}
**DALTON INPUT
.WALK
.MAX IT
 150
*WALK
.IRC
 1
**WAVE FUNCTIONS
.HF
**END OF DALTON INPUT
\end{verbatim}

Most of this input should now be self-explanatory. The request for an
Intrinsic Reaction Coordinate calculation is done by using the keyword
\Key{IRC}. On the next line there is a positive or negative integer
indicating in which direction the reaction should proceed. It is,
however, not possible to determine in advance which reaction path a
given sign is connected to, and the calculation should therefore
always be checked after a few iterations in order to ensure that the
reaction proceeds in the correct direction. If not, the calculation
should be stopped and started from the transition state again with a
different sign for the integer specified after the \Key{IRC} keyword.

As the IRC is defined with respect to mass-weighted
coordinates\index{mass-weighted coordinates}, care
has to be taken in order to specify the correct isotopic substitution
of the molecule. The specification of the isotopic constitution of a
molecule is given in the \molinp\ file, as described in
Chapter~\ref{ch:molinp}. 

Due to the small steps that must be used in a calculation of an IRC,
such a calculation may require a large number of
iterations\index{geometry iteration}, and it
may thus be necessary to increase the maximum number of iterations
that can be taken.  This can be done by
the keyword \Key{MAX IT} in the \Sec{*DALTON} input
section. Default value for this parameter is 20 iterations.

All the information about the Intrinsic Reaction Coordinate will be
collected in a file called \texttt{DALTON.IRC}\index{DALTON.IRC}. If a
calculation stops
because it has reached the maximum number of iterations, it may be
restarted from that point, and the new information about the IRC will
be added to the old \texttt{DALTON.IRC} file. This also implies that if a
calculation is restarted from the beginning (because it went in the
wrong direction) the \texttt{DALTON.IRC} file {\em must\/} be removed
first. Thus it may often be useful to take a backup of the
\texttt{DALTON.IRC} during the calculation of the IRC.

Finally, some comments on the interest of a calculation of IRCs.
Whereas it will give results that mimic the behavior of what is
considered to be a good description of the reaction
pathway\index{reaction pathway} of a
molecular reaction, it does not include any dynamical aspects of the
reaction. There exists several models for approximating local regions of a
molecular potential energy surface from the results of an IRC
calculation, and from the
potential energy surface valley, the dynamics\index{dynamics} of a
chemical reaction may be
mimicked. However, \dalton\ gives another, more direct, opportunity
for studying molecular dynamics through dynamic walks as described in
more detail in the next section.

\subsection{Doing a dynamical walk}\label{sec:dynamic}

\begin{center}
\fbox{
\parbox[h][\height][l]{12cm}{
\small
\noindent
{\bf Reference literature:}
\begin{list}{}{}
\item T.Helgaker, E.Uggerud, H.J.Aa.Jensen. \newblock {\em
Chem.Phys.Lett.}, {\bf 173},\hspace{0.25em}145, (1990).
\end{list}
}}
\end{center}

The theory behind the ``direct dynamics'' as implemented in \dalton\ is
described in Ref.~\cite{theuhjajcpl173}\index{dynamics}. The
main idea behind this
approach is that Newton's equations of motion for the nuclei are
integrated in the presence of the quantum mechanical potential set up
by the electrons. Thus one may follow a molecular reaction from a
given starting point (usually a transition state) as it would behave
if the nuclei could be treated exactly as classical particles. One
should also keep in mind that the Hamiltonian used is constructed within
the framework of the Born-Oppenheimer approximation, which may turn
out not to be a good approximation at given points during the
reaction. Furthermore, the calculation describes the way molecules
with a predefined orientation and momentum will react. Thus
the trajectory obtained is only one of a large number of
possible trajectories depending on the initial state of the molecule.

The necessary input in order to do a dynamical walk of for instance
protonated formaldehyde\index{formaldehyde} would look like:

\begin{verbatim}
**DALTON INPUT
.WALK
.MAX IT
 200
*WALK
.DYNAMI
.FRAGME
 5
 1 1 1 2 2
.MOMENT
 1
1 -.00001
.MODE
 1
**WAVE FUNCTIONS
.HF
**END OF DALTON INPUT
\end{verbatim}

The walk is specified to
be a dynamic walk  through the keyword \Key{DYNAMI}. The
starting trust radius will in dynamical calculations be changed to a
new default value of~0.005.

The keyword \Key{FRAGME} dictates which atoms belong to which
molecular fragment\index{molecular fragments}. In this particular
case, we assume that two
protons leave the protonated formaldehyde\index{formaldehyde} as a
hydrogen\index{hydrogen} molecule, and
that the leaving hydrogen atoms are the last atoms of the \molinp\ input
file. This partitioning is mainly needed in order to get
proper values for the relative translational energy between the two
fragments, as well as for deciding how much of the energy has been
distributed into internal degrees of freedom.

The default isotopic substitution is that the most abundant
isotopes are to be used in the calculation. Isotopic
substitution is important as the masses of the nuclei enters when
Newton's equations of motion are integrated. The specification of the
isotopic constitution of the molecule is given in the
\molinp\ file, as described in Chapter~\ref{ch:molinp}.

In this calculation we start the calculation at a transition
state\index{transition state},
and in order to get the reaction started we need to give the molecule
a slight push. This is achieved by the keyword \Key{MOMENT}. In the next
line the user then specifies the number of modes in which there is
 an initial momentum, followed by lines containing pairs of numbers,
of which the first specifies the mode, and the second the momentum in
this mode. There must be as many pairs of modes and momenta as
specified in the line after the \Key{MOMENT} keyword. It is
impossible to predict in advance which way the reaction will proceed,
and the calculation should be checked after a few iterations, in order
to ensure that it proceeds in the right direction. If not, the
calculation should be started from the transition state again with a
different sign on the initial momentum. The
\texttt{DALTON.TRJ}\index{DALTON.TRJ} must
also be removed as discussed below.

It is in principle possible to start a calculation from any point on a
molecular potential energy surface, and in cases where these starting points
do not correspond to a stationary point, \Key{MOMENT} may be
skipped, as there exist a downward slope (in other words, an attractive
force) driving the molecule(s) in a specific direction. One may of
course also start the molecule with a given initial momentum in
different energy modes.

During the dynamical calculation, care has also to be taken in order
to ensure that the steps taken are not too long. If this occurs, the
initial trust radius and/or the trust radius increment should be
reduced by the keyword \Key{TRUST}. In the \dalton\ output one will
find ``Accumulated kinetic energy since start'', and this property
will be calculated in two ways: From conservation of the total energy,
and from integrated momenta. If the difference between these numbers
is larger than approximately 1\% , the calculation should be stopped and the
starting trust radius be decreased and the calculation restarted from
the starting point again after removal of the \texttt{DALTON.TRJ}-file.

The calculation of dynamical walks may take from about 70 to 1200
iterations (as a general rule) and one must therefore adjust the
maximum number of iterations allowed. This is done by  the
\Key{MAX IT} keyword. In the present example the maximum number of
iterations have been reset to 200. If the calculation
cannot be closely monitored, it is recommended not to set the maximum
number of iterations\index{geometry iteration} too high, and rather
restart the calculation if
this turns out to be necessary. This can be accomplished by specifying
the iteration at which the calculation will restarted by the keyword
\Key{ITERAT} in the \Sec{*DALTON} input module.

The calculation should be stopped (at least for ordinary hydrogen
elimination reactions) when the ``Relative velocity'' starts to
decrease, as this indicates that the molecules are so far apart that
basis set superposition errors\index{BSSE}\index{basis set!superposition error}
become apparent. We will below return to how
one then calculates translational energy release of the reaction.

During the whole calculation, a file
\texttt{DALTON.TRJ}\index{DALTON.TRJ} is updated. This
file contains information from the entire dynamic walk. If a
walk is restarted from a given point, the new information will be
appended to the old \texttt{DALTON.TRJ}-file. Note that this also implies
that if you need to restart the calculation from the beginning
(because the reaction went the wrong way or because of a too large trust
radius), the
\texttt{DALTON.TRJ}-file {\em must} be removed. Thus, it may often be
advisable to take a backup of this file in certain parts of the
calculation. As this file contains all the information about the
dynamical walk, this file can be used to generate a video-sequence of
the molecular reaction along this specific trajectory with the correct
time-scaling~\cite{krtheujms393}.

\subsection{Calculating relative translational energy release}

It is often of interest to calculate the relative translational energy
release\index{relative translation energy} in a given reaction, as this can be compared to experimental
values determined from {\it e.g.\/} mass
spectrometry~\cite{theuhjajcpl173}\index{mass spectrometry}. Although
this quantity
is printed in the output from \dalton\ in the entire dynamical walk, the
relative translational energy release should be calculated, due to
basis set superposition errors\index{BSSE}\index{basis set!superposition error}
and vibrational\index{vibrational excitation} and rotational
excitation\index{rotational excitation} in the departing molecular
fragments, in the way described here.

The geometry of the last iterations for which relative translational
energy release is known, is used as a starting point for minimizing
the two molecular fragments as described in
Section~\ref{sec:minimization}. As a check of this minimization one
might also minimize the two molecular fragments separately, and check
this total energy against the energy obtained when minimizing the
molecular supersystem. The energies should be almost identical, but
small differences due to basis set superposition errors may be
noticeable.

The barrier height can then be calculated by subtracting the energy of
the molecule at the transition state and the energy for the separated
molecular fragments, or an experimentally determined barrier height
may be used. The relative translational energy release may then be
obtained  by dividing the translational energy release
from the last \dalton\ iteration by the barrier height. This number will
not be identical to the number printed in the \dalton -output, because
of the different vibrational and rotational state of
the molecule in the final iteration point as compared to the minimized
structure.

\section{Geometry optimization using non-variational wave
functions}\label{sec:nonvargeom}

\dalton\ does not have any support for the calculation of molecular
gradients and Hessians for the non-variational wave functions 
CI and NEVPT2\index{CI}\index{Configuration Interaction}\index{NEVPT2}. However, in order to
exploit the facilities of the first-order
geometry optimization routines in {\dalton}, a numerical
gradient\index{numerical gradient} based
on energies will be calculated if a geometry optimization is invoked
for a non-variational wave function. As a simple example, to optimize
the MP2 geometry of a molecule using numerical gradients\footnote{Note
that MP2, CCSD, and CCSD(T) analytical gradients are available through the {\cc}
module}, the only input needed is

\begin{verbatim}
**DALTON INPUT
.OPTIMIZE
**WAVE FUNCTIONS
.HF
.MP2
**END OF DALTON INPUT
\end{verbatim}

The size of the displacements used during the evaluation of the
numerical gradient can be controlled through the keyword
\Key{DISPLA} in the \Sec{OPTIMI} input module. Default value is
$1.0\cdot 10^{-3}$ a.u. By default, the threshold for convergence
of the geometry will be changed because of estimated inaccuracies
in the numerical gradients. However, if the threshold for
convergence is altered manually, this user supplied threshold for
convergence will be used also in geometry optimizations using
numerical gradients\index{numerical gradient}. Note that due to
the possibility of larger numerical errors in the gradient, too
tight convergence criteria for an optimized geometry may make it
difficult for the program to obtain a converged geometry.
