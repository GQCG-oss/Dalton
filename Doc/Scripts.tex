\chapter{Scripts}\label{ch:scripts}

\section{General}

          On UNIX systems geometry optimizations with \siraba\ can
conveniently be controlled by an elaborate Bourne shell script,
found in the directory \verb|abashell| and conventionally named
\verb|abacus.sh|.  This script is best called from another, much
shorter, shell script, as discussed briefly in
Sec.~\ref{sec:quickscr}.  In this chapter we shall discuss how
parameters to these scripts are used to create job and file names,
and to control restarting calculations, etc.
\begin{quote}
SPECIAL NOTE:  On CRAY \mbox{X--MP} and \mbox{Y--MP} UNICOS
systems a problem may be encountered when trying to execute the
script \verb|abacus.sh|.  The error message will be similar to
\begin{verbatim}
-sh: stack space exceeded
\end{verbatim}
This comes about because the Bourne shell implementation in UNICOS
cannot dynamically expand its memory (the \mbox{CRAY--2} version
cannot either, but its default size is much larger).  There are
two possible solutions.  The easier is to specifically request
that the script execute under Korn shell (\verb|/bin/ksh|) --- by
setting the shell to be used under NQS, say --- although the Korn
shell is only available in UNICOS~6.0 and later.  The other
solution is to increase the stack space in the Bourne shell, which
requires a system patch.  More details are given inside the script
itself.
\end{quote}

\section{Executing the {\tt abacus.sh} script}

A typical optimization is initiated by running a start-up script
similar to the following, which is taken from the
\verb|testin/CH2| directory.  (The line numbers are added for
convenience.)
\begin{verbatim}
1:root=/users/helgaker
2:ABA=$root/abashell/abacus.sh
3:scr=/scratch/taylor
4:exedir=$root/exe \ 
5:rootscr=$scr \ 
6:herexe=$exe/hermit.x \ 
7:sirexe=$exe/sirius.x \ 
8:abaexe=$exe/abacus.x \ 
9:wf=MCSCF $ABA CH2 A
\end{verbatim}
Line~1 defines the first part of the directory path in which the
programs are to be found. Line~2 is the
full pathname of the \verb|abacus.sh| script.  Line~3 defines the
pathname for the directory \verb|$scr|, in which the various files
(integrals, wave function information, etc.) will be generated.

Lines~4 to~9 are effectively a single line, since each line ends
with a continuation character.  (Recall that no blanks may follow
the continuation backslash.)  Lines~4 and~5 specify particular
directories required by the script, while lines~6 through~8
specify the explicit pathnames for the programs.  Line~9 includes
the actual script invocation as~\verb|$ABA|.  Note that this
approach to executing the scripts utilitizes the Bourne shell
convention that keyword=value pairs precede the script name, while
positional parameters follow it.  We now itemize the parameters
that can be passed to the script.

\subsection{{\tt abacus.sh} keyword parameters}
\label{sec:keyscr}

          The following keyword parameters can be defined for the
script.
\begin{description}
\item[\verb|exedir|] Directory to search for executable programs.
If the latter are named explicitly the value of this keyword is
irrelevant.
\item[\verb|inpdir|] Directory to search for the molecular input. If
not specified the directory from which the calculation is submitted is
chosen as input directory.
\item[\verb|outdir|] Directory to put output from the calculation. If
not specified the directory from which the job was submitted will be
chosen as output directory.
\item[\verb|rootscr|] Initial pathname for scratch directory.
This will normally be the same as~\verb|$scr|.
\item[\verb|herexe|] Full pathname of \her\ executable.
\item[\verb|sirexe|] Full pathname of \sir\ executable.
\item[\verb|abaexe|] Full pathname of \aba\ executable.
\item[\verb|notest|] If set to \verb|true|, the shell script will not
check whether \sir\ accepted the  \aba\ prediction of the wave
function at the next step was within acceptable accuracy.  Notice that
in order to  make this option perform adequately, the \verb|.ABACUS|
keyword un the \verb|*CI VEC| input module added in the {\tt
sirwlk.inp} file must not be added.
\item[\verb|sirf21|] If set to \verb|true|, \sir\ will use its own
interface to provide a guess at the wave function at the next
step, rather than the \aba\ prediction.  
\item[\verb|wf    |] Type of wave function to be computed. Three
different options exist:
\begin{description}
\item[\verb|wf=RHF|] To be used in conventional Hartree-Fock
calculations.
\item[\verb|wf=DIR|] To be used for direct and/or parallel
Hartree-Fock calculations.
\item[\verb|wf=MCSCF|] To be used for MCSCF as well as MP2
calculations (but note that MP2 wave functions only can be used to
compute energies).
\end{description}
\item[\verb|walk  |] Specifies whether this a continuation of a
previous optimization (\verb|walk=old|) or a new optimization
(\verb|walk=new|, the default).  See also the positional
parameters in Sec.~\ref{sec:pospscr}.
\item[\verb|restart|] Specifies a restart point within a walk
step.  Possible values are
\begin{description}
\item[\verb|restart=sir|] Restart at the \sir\ step.
\item[\verb|restart=aba|] Restart at the \aba\ step.
\item[\verb|restart=pro|] Restart at the final \aba\ calculation
to obtain properties from a converged optimization.
\item[\verb|restart=fin|] The same as \verb|restart=pro|.
\end{description}
All restart points can be specified in upper-case or lower-case
letters (but not mixed case).  Note that restarts of this type are
only possible when ``scratch'' files --- those containing
integrals, etc. --- are preserved after a job completes or aborts.
See also the discussion in Sec.~\ref{sec:oldwscr}.
\item[\verb|clean |] Various data files created in each walk step
will be removed when no longer required if \verb|clean=true| is
specified.
\item[\verb|sironly|] If \verb|sironly=true| is set, the walk will
stop after the first \sir\ calculation.
\item[\verb|max|\_\verb|iter=abc|] Specify the maximum number of iterations
allowed in an optimization, dynamical walk or numerical
differentiation. Default value is $30$. This is also a way of
preventing \aba\ from calculating more than one geometry by setting
\verb|max_iter=1|.
\end{description}

\subsection{{\tt abacus.sh} positional parameters}
\label{sec:pospscr}

Positional parameters are denoted \verb|$1, $2...| here.
\begin{description}
\item[{\tt \$1}] Major identifier: used to construct working
directory pathname.  Conventionally set to the molecular formula;
there is no default for this parameter.  See also
Sec.~\ref{sec:labscr}.
\item[{\tt \$2}] Minor identifier: used to construct working
directory pathname and input and output file names.
Conventionally a single letter; default is~\verb|A|.  See also
Sec.~\ref{sec:labscr}.
\item[{\tt \$3}] Walk step to restart at if \verb|walk=old| was
specified (see Sec.~\ref{sec:keyscr}).  Note that restart will be
from the results of the step preceding this one.  Default is~0 (no
old walk).
\end{description}

\section{Job labels and file names}\label{sec:labscr}

The first positional parameter \verb|$1| to the script
\verb|abacus.sh| is the major identifier, which is conventionally
set to the formula of the molecule.  The second parameter
\verb|$2| is the minor identifier, which is used to distinguish
between different calculations on the same molecule.  It is
conventionally set to a single letter.  While a string of any
length can be used in principle, this distinguishing identifier is
used to construct many file names: if it is a single letter, none
of the resulting file names will exceed 14~characters in length.
{\em On some UNIX systems no filename can exceed this length, and so
use of a longer identifier may cause problems.}

The first use of the two identifiers is to construct a pathname
for the working directory, which will be the expansion of
\begin{verbatim}
$scr/$1/$2
\end{verbatim}
using the current values of these shell variables.  All files
such as integrals, density matrices, etc., generated
during the run will be written in this directory.  However, some
temporary files are opened as \verb|SCRATCH| by FORTRAN, and are
written to \verb|$TMPDIR| in some system versions and to a
subdirectory of \verb|/tmp| in others. (Does this still apply to any
computers of today?)

The second use of the identifiers is to define file names.  Only
the minor identifier is used for this purpose.  Input files which
do not depend on a particular walk step number have names of the
form \verb|$2_*.inp|.  Specifically, the directive input to \her\
is in \verb|$2_her.inp|, the input to the first \sir\ calculation
in the walk is in \verb|$2_sirstr.inp|, and the input for
subsequent \sir\ calculations in the walk is \verb|$2_sirwlk.inp|.
The reason for having different file names is that it in the first
step, no guess at the MCSCF wave function may be available.
However, after the first step, the MCSCF orbitals and CI
coefficients from one step will provide an excellent guess for the
next, and this will require different \sir\ input.  The \aba\
input for the first walk step is in \verb|$2_abastr.inp|, while
that for subsequent steps is in \verb|$2_abawlk.inp|.  Usually, at
the conclusion of the walk \aba\ is executed again to compute
molecular properties like nuclear shieldings and/or optical
properties. The input to this step is \verb|$2_abafin.inp|. If only
properties depending on the molecular Hessian is 
to be printed, it is recommended to use the keyword \verb|.VIBCNV| in
the \verb|*WALK  | input module instead of a seperate {\em abafin.inp}
file. 
The geometry input to \her, which changes from
one walk step to another, has the step number (two digits)
appended, becoming \verb|$2nn_mol.inp|.  In the \verb|testin/CH2|
example considered above, the files would be \verb|A_*.inp|, with
\verb|A00_mol.inp| containing the geometry information for the
first walk step.

The naming of output files is more complicated.  Some parts of a
step, such as the \sir\ or \aba\ calculations, could be repeated
if the geometry change predicted by the previous step is
determined to be inappropriate --- the script is then said to
``backstep''.  Since multiple backsteps can be taken, the sequence
number of a particular backstep is appended to the two-digit step
number, making a three-digit number.  For minimizations, at least,
the current geometry optimization methods in \aba\ almost never
require backsteps, and most of these three-digit numbers will end
in zero.  Typically, in step \verb|nn| the \her\ output is named
\verb|$2nn_her.out|, the \sir\ output \verb|$2nn0_sir.out| and the
\aba\ output \verb|$2nn0_aba.out|.  Returning to the CH$_2$
example, the files from the first walk step would be
\verb|A00_her.out|, \verb|A000_sir.out|, and \verb|A000_aba.out|.

After the optimization is complete, the script will re-execute
\aba\ to calculate any desired properties.  The output from this
step is named \verb|$2nnX_aba.out|, where \verb|nn|~is the final
step number of the optimization.

Corresponding to files with a \verb|.out| suffix are files with a
\verb|.err| suffix, which contain the error output from the
corresponding calculation.  In general, any serious problem will
cause the optimization to terminate, and so information in these
files can be ignored if the walk continues.  Error information for
the walk is also written to \verb|$2_abacus.err|, which should be
consulted if the calculation aborts.  The file \verb|$2.log|
contains output messages from the script itself, including some
timing information.

Several other files are written back to the job directory from the
working directory at the end of a given iteration.  The file
\verb|$2nn_sir.f21| is the \sir\ interface file, with the
converged MCSCF orbitals and CI vector on it.  The file
\verb|$2nn_sir.geo| contains information such as transformed
integrals to be passed from \sir\ to \aba.  These files are not
required after iteration \verb|nn+1| is complete and can then be
removed (unless, of course, the user wishes to stop the walk and
restart it from some earlier point).  Other data files written
back are mostly small files: these can also be deleted as
described above.

\section{Restarting calculations}\label{sec:oldwscr}

The \verb|abacus.sh| script includes several possibilities for
restarting optimizations.  The simplest is to set the keyword
parameter \verb|walk=old| and set positional parameter~\verb|$3|
to one more than the last {\em complete\/} walk step.  Also
simple, but more laborious, is to copy the last
\verb|$2nn_mol.inp| file to \verb|$200_mol.inp| and then start the
optimization again.  However, this loses any information \aba\ has
accumulated about updating the trust radius, etc.  Copying over
the last \sir\ \verb|.f21| interface file to the current step will
provide a better wave function starting guess.

It is also possible to restart at some point within a walk step,
as described in Sec.~\ref{sec:keyscr},
but some complications arise from the operating system
environment here, since the working directory contents must be
preserved for such restarts to work.  If the job is being run on a
machine in which the working directory
\begin{verbatim}
$scr/$1/$2
\end{verbatim}
is on a device where files survive the end of a job, like
conventional UNIX systems, there is no problem.  However, some
machines, such as many CRAY systems, are configured so that most
of the disk space is ``scratch'' in nature, and all such space is
deleted at termination of a job.  In this situation restarting
within a walk step is not an option.  

Finally, it is necessary to
be cautious when restarting within a step if the previous walk
aborted for some reason.  The previous walk may have left behind
certain files that the restarted walk will try to open afresh with
a FORTRAN status of \verb|NEW|, causing an abort.  The user should
remove all files with names ending in \verb|.DA| or \verb|.DA1|
from the working directory before attempting a restart.  This
cleaning up has to be done only
when the previous job was aborted by the system:
any termination of the walk by the script itself, whether because
the optimization is converged or because of some error, involves
cleaning up these temporary files.
