\chapter{Getting the property you want}\label{ch:rspchap}


In the preceding chapters we have shown how to calculate a number of
properties that are associated with specific spectroscopic applications.
%such as NMR-parameters (Chapter~\ref{ch:magnetic}). 
These properties are in part
calculated in the \resp\ program, but given that a large number of
standard calculations usually are carried out in a similar fashion, some
applications have a simplified input (under {\tt **PROPERTIES}), 
and an appealing output that meets common demands 
(e.g. customary unit conversions). In this chapter we
describe how to set up the input for calculating a general property that
can be defined in terms of electronic response functions. A very
helpful document detailing various linear and nonlinear properties
that can be calculated with the response program, and the conversion
of the results obtain to other units have been written by
Jaszu\'{n}ski and Rizzo~\cite{}, and this document is available on the
dalton homepage
(\verb|http://www.kjemi.uio.no/software/dalton/dalton.html| or \verb|http://www.sdsc.edu/dalton/dalton.html|).

\section{General considerations}
\label{sec:rspgen}

\begin{center}
\fbox{
\parbox[h][\height][l]{12cm}{
\small
\noindent
{\bf Reference literature:}
\begin{list}{}{}
\item Response theory:
Jeppe Olsen and Poul J{\o}rgensen, \newblock {\em J. Chem. Phys.} {\bf 82}, \hspace{0.25em}3235, (1985)
\end{list}
}}
\end{center}

A response function is a measure of how a property of a system changes in
the presence of one or more perturbations. With our notation (see e.g.
Ref.~\cite{jopjjcp82}),  $\langle\!\langle A;B\rangle\!\rangle_{\omega_b}$,
$\langle\!\langle A;B,C\rangle\!\rangle_{\omega_b,\omega_c}$, and 
$\langle\!\langle A;B,C,D\rangle\!\rangle_{\omega_b,\omega_c,\omega_d}$
denote linear, quadratic and cubic response\index{linear response}\index{quadratic response}\index{cubic response}\index{response!linear}\index{response!quadratic}\index{response!cubic}\index{response function}
functions, respectively, which
provide the first, second, and third-order corrections to the
expectation-value of $A$, due to the perturbations $B$, $C$, and $D$, each of
which is associated with a frequency $\omega_b$, $\omega_c$, and
$\omega_d$. Often the perturbations are considered to be 
external monochromatic fields, or static (e.g. relativistic) perturbations,
in which case the frequency is zero.   In general, the perturbations $B$,
$C$, and $D$ represent Fourier components of an arbitrary time-dependent
perturbation.

\section{Input description}
\label{sec:rspex}
 
In this section we describe a few minimal input examples for calculating some
molecular properties that can be expressed in terms of linear, quadratic, and
cubic response functions.
Note that only one of these three types of response functions can be
requested in the same calculation.

For more information on keywords, see the Reference Manual, chapter~\ref{ch:response}.
 
\subsection{Linear response}
\label{subsec:linrsp}
%\subsection{Polarizability, $\alpha_{ij}(0;0)$ $i,j \in \{x,y,z\}$}

\begin{center}
\fbox{
\parbox[h][\height][l]{12cm}{
\small
\noindent
{\bf Reference literature:}
\begin{list}{}{}
\item Singlet linear response:
Poul J{\o}rgensen, Hans J{\o}rgen Aagaard Jensen, and Jeppe Olsen, \newblock {\em J. Chem. Phys.} {\bf 89}, \hspace{0.25em}3654, (1988)
\item Triplet linear response:
Jeppe Olsen, Danny L. Yeager, and Poul J{\o}rgensen, \newblock {\em J. Chem. Phys.} {\bf 91}, \hspace{0.25em}381, (1989)
\item SOPPA linear response:
Martin J. Packer, Erik K. Dalskov, Thomas Enevoldsen, Hans J{\o}rgen Aagaard Jensen and Jens Oddershede, 
\newblock {\em J. Chem. Phys.}, {\bf 105}, \hspace{0.25em}5886, (1996)
\end{list}
}}
\end{center}

A well-known example of a linear response\index{linear response}\index{response!linear}\index{polarizability}
function is the polarizability.
A typical input for SCF static and dynamic polarizability tensors
$\alpha_{ij}(-\omega;\omega)\equiv-\langle\!\langle
x_i;x_j\rangle\!\rangle_\omega$ for a few selected frequencies (in
atomic units) will be:
\begin{verbatim}
**DALTON INPUT
.RUN RESPONSE
**WAVE FUNCTIONS
.HF
**RESPONSE
*LINEAR
.DIPLEN
.FREQUENCIES
 3
 0.0 0.5 1.0
*END OF INPUT
\end{verbatim}
The {\tt .DIPLEN} keyword has the effect of defining the $A$ and $B$
operators as all components of the electric dipole operator.

A Second Order Polarization Propagator Approximation
(SOPPA)\index{SOPPA}\cite{esnpjjodjcp73,jopjdycpr2,mjpekdtehjajjojcp}
calculation of linear response functions
can be invoked if the additonal keyword \Key{SOPPA} is specified in the 
\Sec{*RESPONSE} input module and an MP2 calculation is requested by the 
keyword \Key{MP2} in the \Sec{*WAVE FUNCTIONS} input module.  A typical input 
for SOPPA dynamic polarizability tensors  will be:
\begin{verbatim}
**DALTON INPUT
.RUN RESPONSE
**WAVE FUNCTIONS
.HF
.MP2
**RESPONSE
.SOPPA
.NOITRA
*LINEAR
.DIPLEN
.FREQUENCIES
 3
 0.0 0.5 1.0
*END OF INPUT
\end{verbatim}
The {\tt .NOITRA} keyword has the effect that the transformation of the two 
electron integrals necessary for a MP2 and SOPPA calculation is only perfomed
once in the \Sec{*WAVE FUNCTIONS} module.

The linear response function contains a wealth of
information about the spectrum of a given Hamiltonian. 
It has poles\index{pole of response function} at the excitation
energies\index{electronic excitation}, 
relative to the reference state (not necessarily the ground state) and the
corresponding residues\index{residue} are transition
moments\index{transition moment} between the reference and
excited states. To calculate the excitation energies\index{electronic
excitation} and dipole transition 
moments\index{transition moment} for the three lowest excited states
in the fourth symmetry, a small 
modification of the input above will suffice;
\begin{verbatim}
**RESPONSE
*LINEAR
.SINGLE RESIDUE
.DIPLEN
.ROOTS
 0 0 0 3
*END OF INPUT
\end{verbatim}

\subsection{Quadratic response}
\label{subsec:quadrsp}
%\subsection{First hyperpolarizability, 
%$\beta_{zzz}(0;0,0)

\begin{center}
\fbox{
\parbox[h][\height][l]{12cm}{
\small
\noindent
{\bf Reference literature:}
\begin{list}{}{}
\item Singlet quadratic response: 
Hinne Hettema, Hans J{\o}rgen Aa. Jensen, Poul J{\o}rgensen, and Jeppe Olsen, \newblock {\em J. Chem. Phys.} {\bf 97}, \hspace{0.25em}1174, (1992)
\item Triplet quadratic response: 
Olav Vahtras, Hans {\AA}gren, Poul J{\o}rgensen, Hans J{\o}rgen Aa. Jensen, Trygve Helgaker, and Jeppe Olsen, \newblock {\em J. Chem. Phys.} {\bf 97}, \hspace{0.25em}9178, (1992)
\item Integral direct quadratic response: 
Hans {\AA}gren, Olav Vahtras, Henrik Koch, Poul J{\o}rgensen, and Trygve Helgaker, \newblock {\em J. Chem. Phys.} {\bf 98}, \hspace{0.25em}6417, (1993)
\end{list}
}}
\end{center}

An example of a quadratic response\index{quadratic response} function
is the first 
hyperpolarizability\index{first hyperpolarizability}. If we are
interested in
$\beta_{zzz}\equiv-\langle\!\langle z;z,z\rangle\!\rangle_{0,0}$ 
only we may use the following input:
\begin{verbatim}
**DALTON INPUT
.RUN RESPONSE
**WAVE FUNCTIONS
.HF
**RESPONSE
*QUADRATIC
.DIPLNZ
*END OF INPUT
\end{verbatim}
When no frequencies are given in the input, the static value is assumed by
default. If we wish to calculate dynamic hyperpolarizabilities we supply
frequencies\index{frequency}, but in this case we have two frequencies
$\omega_b, \omega_c$ which are given by the keywords \texttt{.BFREQ} and
{\tt .CFREQ} (see the Reference Manual, chapter~\ref{ch:response}).
%LRHYP module in rspvec.F:
The non-zero linear response functions from the operators can be
generated with no additional computational costs, and all
$\langle\!\langle A;B\rangle\!\rangle_{\omega_b}$ results
will also be printed (in this example $\alpha_{zz}$).

The residue of a quadratic response function gives two-photon
transition amplitudes\index{two-photon!amplitude}. For such a
calculation we supply the same extra 
keywords as in the linear case (Sec.~\ref{subsec:linrsp}):
\begin{verbatim}
**RESPONSE
*QUADRATIC
.DIPLNZ
.SINGLE RESIDUE
.ROOTS
 2 0 0 0
*END OF INPUT
\end{verbatim}
which in this case means the two-photon transition
amplitude\index{two-photon!amplitude} between the
reference state and the first two excited states in the first symmetry.  In
general the residue of a quadratic response function corresponds to the
induced transition moment of an operator $A$ due to a perturbation $B$.
The $C$ operator is arbitrary and is not specified.  A typical example is
the dipole matrix element between a singlet and triplet state that is
induced by spin-orbit coupling
(phosphorescence)\index{phosphoresence}. For this special case we 
have the keyword, {\tt .PHOSPHORESCENCE} under {\tt *QUADRATIC}, which sets
$A$ to electric dipole operators and $B$ to spin-orbit operators.

The residue of a quadratic response function can be used to identify
the two-photon transition amplitudes. The input below refers to the
calculation of the two-photon absorption from the ground state to the
first 3 excited states in point symmetry group one. In the program
output the two-photon transition matrix element is given as well as
the two-photon transition probability relevant for an isotropic gas or
liquid. The evaluation of the transition probabilities can be done based
on the transition matrix elements although they, in principle, are
connected with the imaginary part of the second
hyperpolarizability. The absorption cross sections are evaluated
assuming a monochromatic light source that is either linearly or
circularly polarized.
\begin{verbatim}
**RESPONSE
*QUADRATIC
.TWO-PHOTON
.ROOTS
 3 0 0 0
*END OF
\end{verbatim}

Another special case of a residue of the quadratic response function 
is the ${\cal{B}}(0\to f)$ term of magnetic circular dichroism (MCD).
\begin{verbatim}
**RESPONSE
*QUADRATIC
.SINGLE RESIDUE
.ROOTS
 2 2 0 0
.MCDBTERM
*END OF INPUT
\end{verbatim}
For each dipole-allowed excited state among those specified in 
{\tt .ROOTS}, the {\tt .MCDBTERM} keyword automatically 
calculates all symmetry allowed products of the single residue of the 
quadratic response function for $A$ corresponding to the electric dipole
operator and $B$ to the angular momentum operator with the single residue 
of the linear response function for $C$ equal to the electric dipole operator.
In other words the mixed electric dipole---magnetic dipole two-photon
transition moment\index{two-photon!transition moment}\index{transition moment!two-photon}
for final state $f$ times the dipole one-photon moment for the same state $f$.
Note that in the current implementation (for SCF and MCSCF) degeneracies between
excited states may lead to numerical divergencies. 
The final ${\cal{B}}(0\to f)$ must be obtained from a combination 
of the individual components, see the original paper~\cite{Coriani:MCDRSP},
or the document by Jaszu\'{n}ski and Rizzo on the Dalton
homepage. 

It is possible to construct double
residues\index{residue}\index{double residue} of the quadratic
response function, the interpretation of which is transition
moments\index{transition moment}\index{excited state}
between two 
excited states. Specifying \Key{DOUBLE} in the example above thus gives
the matrix elements of the $z$-component of the dipole moment between
all excited states specified in {\tt .ROOTS}. Note that the diagonal contributions
gives  not the expectation value in the excited state, but rather the
difference relative to the reference state expectation value.

\subsection{Cubic response}
\label{subsec:cubrsp}
%\subsection{Second hyperpolarizability, $\gamma_{ijkl}(0;0,0,0)$
%$i,j,k,l \in \{x,y,z\}$} \cite{pndjhapdkrthhkcpl253}

\begin{center}
\fbox{
\parbox[h][\height][l]{12cm}{
\small
\noindent
{\bf Reference literature:}
\begin{list}{}{}
\item SCF cubic response:
Patrick Norman, Dan Jonsson, Olav Vahtras, and Hans {\AA}gren, \newblock {\em Chem. Phys. Lett.} {\bf 242}, \hspace{0.25em}7, (1995)
\item MCSCF cubic response:
Dan Jonsson, Patrick Norman, and Hans {\AA}gren, \newblock {\em J. Chem. Phys.} {\bf 105}, \hspace{0.25em}6401, (1996)
\end{list}
}}
\end{center}

All components of the static second hyperpolarizability\index{cubic response}\index{response!cubic}\index{second hyperpolarizability}
defined as $\gamma_{ijkl}(-0;0,0,0)\equiv$
$-\langle\!\langle x_i;x_j,x_k,x_l\rangle\!\rangle_{000}$,
 may be obtained by the following input

\begin{verbatim}
**DALTON INPUT
.RUN RESPONSE
**WAVE FUNCTIONS
.HF
**RESPONSE
*CUBIC
.DIPLEN
*END OF INPUT
\end{verbatim}


%\subsection{Polarizability of the first excited state of symmetry 1, \\
%$\alpha^e_{zz}(-0.01;0.01) - \alpha^0_{zz}(-0.01;0.01)$}
%\cite{djpnylhajcp105}

As  mentioned above (Sec.~\ref{subsec:quadrsp}), the "diagonal" 
double residue\index{residue}\index{double residue} of the quadratic
response function is the change in the expectation value relative to the
reference state. The analogue for cubic
response functions is the change in
polarizability\index{polarizability}\index{excited state} relative to the
reference state polarizability, which is demonstrated by the following
input.
\begin{verbatim}
**DALTON INPUT
.RUN RESPONSE
**WAVE FUNCTIONS
.HF
**RESPONSE
*CUBIC
.DOUBLE RESIDUE
.DIPLNZ
.FREQUENCIES
 1
 0.01
.ROOTS
  1 0 0 0
*END OF INPUT
\end{verbatim}



