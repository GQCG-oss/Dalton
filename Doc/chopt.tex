~%%%%%%%%%%%%%%%%%%%%%%%%%%%%%%%%%%%%%%%%%%%%%%%%%%%%%%%%%%%%%%%%%%%
\section{Cholesky based CCSD(T): \Sec{CHO(T)}}
\label{sec:chopt}
%%%%%%%%%%%%%%%%%%%%%%%%%%%%%%%%%%%%%%%%%%%%%%%%%%%%%%%%%%%%%%%%%%%
\index{CCSD(T)}
\index{Cholesky decomposition-based methods}

In this Section, we describe the keywords controlling the calculation of the 
CCSD(T) energy correction using Cholesky decomposed energy denominators. The
calculation can also be invoked by using \Key{CHO(T)} in the \Sec{CC INPUT} 
section. In this case, default values will be used. 


The calculation is driven in a batched loop over virtual orbitals followed by
the computation of the contributions from the occupied terms in the order 
H, H1, F1, C1 and, finally, C2. Note that in the case of restarted calculations,
the user must provide the value of the already computed contributions by means 
of the keywords \Key{OLD4V}, \Key{OLD5V}, \Key{OLD4O}, and \Key{OLD5O}.
The required numerical values can be found in the CHOPT\_RST file from the
previous calculation.
For further details on the implementation, see 
Refs.~\cite{jcp_chopt,ijqc_chopt}.

\begin{center}
\fbox{
\parbox[h][\height][l]{12cm}{
\small
\noindent
{\bf Reference literature:}
\begin{list}{}{}
\item H.~Koch and {A.~M.~J.~S{\'a}nchez~de~Mer{\'a}s} \newblock {\em J.~Chem.~Phys.}, {\bf 113},\hspace{0.25em}508, (2000).
\end{list}
}}
\end{center}

\begin{description}
\item[\Key{MXCHVE}]\verb| |\newline
\verb|READ (LUCMD,*) MXCHVE|\newline
        Maximum number of vector to include in the expansion of the
        orbital energy denominators. Defaults to 10, but normally 
        6 vectors are enough to get micro-hartree precision.
%
\item[\Key{OLD4O}] \verb| |\newline
\verb|READ (LUCMD,*) OLD4O|\newline
        Read in the contribution to the 4th-order correction from
        occupied orbitals as specified in the CHOPT\_RST file from a
        previous calculation. (Default: 0.0D0).
%
\item[\Key{OLD5O}] \verb| |\newline
\verb|READ (LUCMD,*) OLD5O|\newline
        Read in the contribution to the 5th-order correction from
        occupied orbitals as specified in the CHOPT\_RST file from a
        previous calculation. (Default: 0.0D0).
%
\item[\Key{OLD4V}] \verb| |\newline
\verb|READ (LUCMD,*) OLD4V|\newline
        Read in the contribution to the 4th-order correction from
        virtual orbitals as specified in the CHOPT\_RST file from a
        previous calculation. (Default: 0.0D0).
%
\item[\Key{OLD5V}] \verb| |\newline
\verb|READ (LUCMD,*) OLD5V|\newline
        Read in the contribution to the 5th-order correction from
        virtual orbitals as specified in the CHOPT\_RST file from a
        previous calculation. (Default: 0.0D0).
%

%
\item[\Key{RSTC1}] 
        Restart the calculation from the C1 term. The virtual contribution
        as well as those from some occupied terms will
        not be computed (see above).
%
\item[\Key{RSTC2}] 
        Restart the calculation from the C2 term. The virtual contribution
        as well as those from some occupied terms will
        not be computed (see above).
%
\item[\Key{RSTF1}] 
        Restart the calculation from the F1 term. The virtual contribution
        as well as those from some occupied terms will
        not be computed (see above).
%
\item[\Key{RSTH}] 
        Restart the calculation from the H term. The virtual contribution
        will not be computed (see above).
%
\item[\Key{RSTH1}] 
        Restart the calculation from the H1 term. The virtual contribution
        as well as that from the occupied H term will
        not be computed (see above).
%
\item[\Key{RSTVIR}]\verb| |\newline
\verb|READ (LUCMD,*) IFVISY,IFVIOR|\newline 
        Restart the virtual loop at B-orbital IFVIOR of symmetry IFVISY.
        The contribution due to previous virtual orbitals will 
        not be computed.

%
\item[\Key{SKIVI1}] 
        Use existing CHO\_VI1 file instead of building it up 
        from CC3\_VI.
%
\item[\Key{SKIVI2}] 
        Use existing CHO\_VI2 file instead of building it up 
        from CC3\_VI12.
%
        
%
\item[\Key{THRCHO}] \verb| |\newline
\verb|READ (LUCMD,*)') THRCHO|\newline
        Threshold for skipping remaining Cholesky vectors in 
        each term. (Default: 0.0D0).
%
\end{description}
