\chapter{New features in Dalton 1.2}
%\pagestyle{myheadings}
%\markboth{ }
%{ }
%\markright{ }
%\addcontentsline{toc}{chapter}{New features in Dalton 1.2}

Dalton 1.2 includes several large modules that have been added in order
to increase the possibilities for user applications of the code. In
addition, several improvements in the general performance of the code
has been made relative to Dalton~1.1. The most important new features
and improvements are presented here.

\begin{enumerate}
\item {\bf Coupled Cluster:} The most important and largest improvement is the addition of a
a complete integral-direct coupled cluster module, capable of
calculating all common coupled cluster models, most of them in an
integral-direct manner. Frequency-dependent response functions have
also been implemented up to the cubic response functions for the CCS,
CC2, and CCSD models. Several models for calculating molecular
properties of excited states are also available. Geometry optimization
using analytical gradients for electronic ground states using
first-order optimization methods is also available.

\item {\bf Non-equilibrium solvation} Non-equilibrium
solvation linear, quadratic, and cubic response functions have been
implemented, and this
makes for instance solvent shifts on excitation energies directly
accessible in a one-step calculation. Solvent effects on non-linear
optical processes can also be calculated with this approach.

\item {\bf SOPPA} The Second-Order Polarization Propagator Approach (SOPPA) has
been introduced in the \aba\ module, allowing for convenient and easy
calculations of second-order properties such as polarizabilities,
indirect spin--spin coupling constants, nuclear shieldings and
magnetizabilities, the latter two only without London orbitals.

\item {\bf CSF} Configurations State Functions can now be used also in
property calculations in both the \resp\ and \aba\ modules, allowing
for better control of the spin-state of the reference wave function
also in property calculations.

\item {\bf AMFI} An Atomic Mean-Field Approximation of the spin-orbit
integrals have been implemented. Even though this mean-field
spin-orbit approximation only involves one-electron integrals, it has
proven to be a very good approximation to the full one- and
two-electron Breit-Pauli spin--orbit operator.

\item {\bf ECP} Effective Core Potentials can now be used for calculations
of energies and response functions in which the basis set does not
depend on the applied perturbations. That is, molecular gradients and
Hessians, as well as London orbital-based magnetic properties, can not
be calculated with ECPs.

\item {\bf Douglas-Kroll} The Douglas-Kroll scalar relativistic Hamiltonian
has been implemented to second-order. It can be used to account for
scalar relativistic effects in molecular energies and response
functions, and although molecular gradients and Hessians can be
calculated, they are not strictly correct, as they use the
non-relativistic derivatives of the one-electron Hamiltonian.

\item {\bf Linear coupling model} The linear coupling model used for
estimating Franck-Condon factors.

\item  {\bf Vibrational averaging} A new approach for calculating
zero-point~\cite{poakrprtjcp112,krpoaprtjcp112} or
temperature-dependent~\cite{krjljv} vibrational averaged molecular 
properties. For general properties, this have to be in a two-step
procedure in which the first step determines the vibrational averaged
geometry, and the second calculates the average of the molecular
property over a harmonic oscillator approximation for the different
vibrational modes. 

\item {\bf Magnetic Circular Dichroism} The ${\cal{B}}(0\rightarrow f)$ term contributing to
magnetic circular dichroism (MCD) has been
implemented~\cite{Coriani:MCDRSP}. 

\item {\bf Two-photon absorption} Although possible to calculate also with
Dalton~1.1, the input required for calculating two-photon absorption
has been significantly improved.

\item{\bf Solvent geometry optimizations} Geometry optimizations using
the spherical cavity model can now be done using symmetry. However,
only the second-order geometry optimization routines in the
\verb|*WALK| module will be able to do geometry optimizations with the
solvent model.

\item {\bf File splitting} For file-system where the maximum file length is
2~Gb, Dalton will automatically split large files  so that
calculations can be performed on these systems even though some files
may be longer than 2~Gb. Currently the implementation limits the
maximum file size is limited to 22~Gb.

\item {\bf Restart features in \resp } Assuming that the files
\verb|RESULTS.RSP| and \verb|RSPVEC| are available, the \resp\ program
can now be restarted at any point of a linear, quadratic or cubic
response calculation. At the most a few micro-iterations of a solution
of a linear set of equations will be lost. One can now also reuse the
converged response vectors of a quadratic response calculation in a
cubic response vector, making it significantly more computationally
efficient to determine both linear, quadratic and cubic response
functions using SCF or MCSCF wave functions.
\end{enumerate}
