\chapter{Installation}\label{ch:install}

\section{Hardware/software
supported}\label{sec:hardsoft}\index{hardware/software support}

{\dalton} can be run on a variety of systems running the UNIX
operating system. The current release of the program supports
Cray-UNICOS\index{Cray},
Cray-T3D/E\index{Cray!T3D}\index{Cray!T3E}\index{Cray},
HP-UX,
IBM-AIX\index{IBM-AIX},
IBM-SP2,
Linux\index{Linux} using g77, pgf77 or ifc,
SGI-IRIX\index{SGI}\index{SGI-IRIX},
Sun\index{Sun},
MacOS X (darwin)\index{MacOSX} using xlf/xlc or g77/gcc,
and DEC-Alpha\index{DEC-Alpha}.
We furthermore note that response calculations involving the
spin--orbit\index{spin-orbit} operator only will work on some of these
computers (more details are given in the file
\verb|dalton/test/KNOWN_PROBLEMS|).

The program is written in FORTRAN~77\index{FORTRAN~77} and C~\index{C}, with
machine dependencies isolated using C preprocessor directives\index{C
preprocessor}.  All floating-point computations are
performed in 64-bit precision, but if 32-bit integers are
available the code will take advantage of this to reduce storage
requirements in some sections.

The program should be easily portable to other UNIX
platforms\index{porting}: indeed, the IBM-AIX version
may very well work without change on typical 32-bit UNIX workstations.
Users who port the codes to other platforms are encouraged to
communicate any required changes in the original source with the
appropriate C preprocessor directives to the authors.

For installation under MacOS X\index{MacOSX} with the G5 architecture, note that the 
current release of \dalton\ requires OSX 10.3, XTools version 1.5 
and gcc version 3.3, all available from the Apple website. 
We recommend the use of the IBM xlf (v8.1) and xlc (v6.0) 
compilers, as development and testing is with these compilers. 

\section{Source files}\label{sec:source}

{\dalton} is distributed as a \verb|tar| file obtainable from
the {\dalton} homepage\\
(\verb|http://www.daltonprogram.org|) if the
license agreement for the program has been completed and returned to
the authors.  If you have accessed this documentation off the
\verb|tar|\index{tar-file} file you will
already know how to extract the required directory structure, but
for completeness, assuming the \verb|tar| file is called
\verb|dalton.tar.gz|\index{tar-file}, the commands
\begin{verbatim}
gunzip dalton.tar.gz
tar xf dalton.tar
\end{verbatim}
will produce the following subdirectory structure in the current
directory:
\begin{verbatim}
dalton/abacus    dalton/dft        dalton/include   dalton/test
dalton/amfi      dalton/Doc        dalton/pdpack    dalton/tools
dalton/basis     dalton/eri        dalton/rsp
dalton/cc        dalton/gp         dalton/sirius
\end{verbatim}
Most of the subdirectories contain source code for the different sections
constituting the program (\verb|abacus|, \verb|amfi|, \verb|cc|, \verb|dft|,
\verb|eri|, \verb|gp|, \verb|rsp| and \verb|sirius|). Furthermore,
there's a directory containing
various public domain routines (\verb|pdpack|), a directory with the
necessary include files containing common blocks and
machine dependent routines (\verb|include|), a directory containing
all the basis sets supplied with this distribution (\verb|basis|), a
fairly large set of test jobs including reference output files
(\verb|test|), a directory containing some useful pre- and
post-processing programs supplied to us from various users
(\verb|tools|), and finally this documentation (\verb|Doc|). 

In addition to the directories, the main dalton directory will
contain several files including a shell script (\verb|configure|)
which will build a suitable \verb|Makefile.config| for use when
installing the program. The \verb|configure| script will also create a
\verb|Makefile| and a run script \verb|bin/dalton| from the skeletal files
(\verb|Makefile.in| and \verb|dalton.gnr|) that are present in the
directory.


\section{Installing the program using the
Makefile}\label{sec:Makefile}
\index{Makefile}
\index{installation!Makefile}
\index{installation!Makefile.config}

The program is easily installed through the use of the supplied
\verb|configure| script\index{configure script}. Based on the
architecture supplied by the user, the
script will try to build a suitable
\verb|Makefile.config|\index{Makefile.config} on the
basis of what kind of mathematical libraries are found, and user
input. Thus, to execute the script, type
\begin{verbatim}
> ./configure -hosttype
\end{verbatim}
where hosttype may be one of

\bigskip

\begin{tabular}{lll}
aix \hspace{3cm} & darwin \hspace{3cm} & linux-alpha\\
cray             & dec-alpha             & nec\\
cray-t90         & hal                   & sgi\\
cray-t3d         & hp                    & sun\\
cray-t3e         & linux                 &\\\end{tabular}
\index{installation!supported platforms}

\bigskip

If no hosttype is specified, the script will try to guess what sort of
machine it is running on. This usually works fine for most common
platforms.

Although this script in most cases is capable of making a correct
\verb|Makefile.config|, we always recommend users to check the created
\verb|Makefile.config| against local system set-up. This applies in
particular in case of parallel installations, where the user manually
{\em has} to specify certain variables like the path for the
MPI-libraries\index{MPI}.

During the execution of the \verb|configure| script, you will be
asked a few questions, most of which require a quite obvious
answer. Let us only comment upon four of the questions asked:

\begin{enumerate}
\item Scratch memory\index{scratch memory}\index{installation!memory}
size: Dalton uses approximately 6.5 Mwords (54 Mbytes)
in static memory allocations. The program defines a large scratch-memory
array, from which it allocates space for temporary arrays
during the execution of the program. This value is given in Words, and
should be chosen according to available memory on your computer.
\verb|WRKMEM| may be changed at execution time by supplying a
different value for \verb|WRKMEM| through the shell script running
{\dalton}.

\item Default basis set library
\index{basis set!library}
\index{installation!basis set library}
location: This defines the directory where the program will look for the
basis sets supplied with the distribution, and this need to be
changed according to the local directory structure. We recommend
that the basis sets in this directory are {\em not} changed, but
that changes to the basis set rather is done in a separate
directory, and then supply this basis set directory to the program
at execution time using the \verb|dalton -b basdir| option.

{\sc NOTE TO SYSTEM ADMINISTRATORS:} Supplied with {\dalton} is an
extensive basis set library. This basis set directory must be made
readable to all users. {\sc END NOTE}

\item Default scratch
space\index{scratch space}\index{installation!scratch space}:
Determines the default head scratch
directory where temporary files will be placed. This value will be put
in the \verb|dalton| run script. However, note that jobs will be run in
a subdirectory of this head scratch-directory, according to the name
of the job files. If \verb|/work| or \verb|/scratch| is defined in the
local directory structure, the script will normally suggest
\verb|/work/$USER| or  \verb|/scratch/$USER| as default head scratch space.

\item Default install 
directory\index{install directory}\index{installation!install directory}:
Denotes the directory where the {\dalton} executable and the
\verb|dalton| run script will be moved to.
Default: a subdirectory {\tt ./bin} to the main \dalton\ directory.
\end{enumerate}

Compiler options will be supplied in \verb|Makefile.config|, in the manner
we use ourselves. {\em These options often do not include aggressive
optimization\index{optimization (f77)} (for instance on the Cray), as
it is our experience that
the code is optimized incorrectly if the optimization is too
aggressive.} The proper options to the C preprocessor for an ordinary
installation of the program are also supplied with
\verb|Makefile.config|\index{Makefile.config}.

The \verb|configure| script attempts to detect mathematical libraries
available on the system and use them whenever possible. {\dalton} can
use third-party BLAS and LAPACK libraries or libraries providing
equivalent functionality like ESSL on AIX, COMPLIB.sgimath on SGI,
DXML and CXML on Alpha architectures, ATLAS, ACML, MKL on Intel
architectures, VECLIB on HP/UX.
You will be asked which one you want of the found libraries.
You can also point the \verb|configure|
script to the location of the preferred libraries by setting
\verb|LIBDIR| environment variable before running the script. For
example, if one has high-performance BLAS routines in \verb|mylibs|
subdirectory of the user's home directory, \verb|configure| script can
be instructed to use them as follows:
\begin{verbatim}
env LIBDIR=\$HOME/mylibs ./configure
\end{verbatim}


NOTE: If problems with I/O\index{I/O-problems} are experienced on
any computer,
manifesting themselves as error messages saying that a read statement
has passed the EOF mark, the *.F files in the gp directories should be
touched, and the code rebuilt using the additional C preprocessor
directive \verb|VAR_MFDS|.

When  \verb|Makefile.config|\index{Makefile.config} has been properly
created and checked
to agree with local system set-up, all that is needed
to build an executable\index{building an executable} version of the
code is to type
(in the same directory as the \verb|Makefile.config| file):
\begin{verbatim}
> make
\end{verbatim}


For MacOS X\index{MacOSX}, the program may be installed either with the IBM
xlf compiler (version 8.1 at the time of release) for best  performance 
or with the g77 compiler (version 3.4.2 at the time of release). Be sure
to use the latest Xcode from \verb|http://developer.apple.com/|. The
g77 compiler may be obtained from \verb|http://hpc.sourceforge.net/|.  With
the g77 compiler, the program may be installed as configured by default
(-O2 optimization level). However, for slightly better performance,
perform the following two-step process: First, make a copy of the original
\verb|Makefile.config| file. Next, in \verb|Makefile.config|, 
change -O2 to -O3 and type
\begin{verbatim}
> make -k
\end{verbatim}
Finally, revert to the original \verb|Makefile.config| file and then type
\begin{verbatim} 
> make 
\end{verbatim} 
This process results in a code where -O3 has been used wherever possible.

%Note to users who want to install a parallel program\index{parallel
%program} employing PVM3\index{PVM}
%as message passing interface: Whereas a normal installation will
%create a single executable, the PVM installation will
%create both a master\index{master program} and a slave
%program\index{slave program}.
%The slave program, called \verb|node.x|\index{node}, must be placed in
%the proper directory, as defined by the PVM3 installation on the
%platform you are using. NOTE: Default location of the slave program
%will be the same directory as the master program, that is, the default
%installation directory.
%
%MPI\index{MPI} parallel executables will be called \verb|dalpar.x|,
%PVM\index{PVM} parallel executables are called \verb|dalpvm.x| and
%\verb|node.x|, whereas a single-processor executable will be called
%\verb|dalton.x|.

\section{Running the {\dalton} test suite}\label{sec:testsuite}
\index{test suite}\index{installation!test suite}

To check that {\dalton} has been successfully installed, a fairly
elaborate automatic test suite is provided in the distribution. A test
script, all the test jobs and reference output files can be found in
the \verb|dalton/test| directory. It is highly recommended that all
these tests be run once the program has been compiled. Depending on
your hardware, this usually takes 2---12 hours.

The tests can be run one by one or in groups, by using the test script
\verb|TEST|. Try \verb|TEST -h| to see the different options this
script takes. Also, have a look at \verb|CONTENTS|
for short descriptions of the various tests. To run the
complete test suite, simply go to the \verb|dalton/test| directory and
type:
\begin{verbatim}
> ./TEST all
\end{verbatim}
You can follow the progress of the tests directly, but all messages
are also printed to a log (\verb|TESTLOG| by default). After all the
tests have completed you should hopefully be presented with the
message ``ALL TESTS ENDED PROPERLY!''. If not, you will be given a
list of the tests that failed to run correctly. Please consult the file
\verb|KNOWN_PROBLEMS| too see if these tests have documented problems
on your particular platform.

Any tests that fail will leave behind the \verb|.mol| and \verb|.dal|
input-files (these are described in more detail in
Chapter~\ref{ch:starting}), and the output-file from the test
calculation which will have the extension \verb|.log| ({\dalton}
output-files usually have the extension \verb|.out|). For all
successful tests these files, as well as some other auxiliary files,
will be deleted as soon as the output has been checked, unless
\verb|TEST| is being run with the option \verb|-keep|.

If most of the tests fail, it is quite likely that there's something
\index{troubleshooting}\index{installation!troubleshooting}
wrong with the installation. Look carefully through
\verb|Makefile.config|, and consider turning down or even off
optimization.

If there's only a few tests that fail, and {\dalton} seems to exit
normally in each case, there may just be some issues with numerical
accuracy. Different machines give slightly different results, and
while we've tried to allow for some slack in the tests, it may be
that your machine yields numbers just outside the intervals we've
specified as acceptable. A closer comparison of the results with
numbers in the test script and/or the reference output files should
reveal whether this is actually the case. If numerical (in)accuracy is
the culprit, feel free to send your output-file(s) to
\verb|dalton-admin@kjemi.uio.no| so that we can adjust the numerical
intervals accordingly.
