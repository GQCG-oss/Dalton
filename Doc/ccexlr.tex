
%%%%%%%%%%%%%%%%%%%%%%%%%%%%%%%%%%%%%%%%%%%%%%%%%%%%%%%%%%%%%%%%%%%
\section{Excited state linear response functions and
         two-photon transition moments between two excited states:
         \Sec{CCEXLR}}
\label{sec:ccexlr}
%%%%%%%%%%%%%%%%%%%%%%%%%%%%%%%%%%%%%%%%%%%%%%%%%%%%%%%%%%%%%%%%%%%
\index{linear response, excited states}
\index{excited states linear response}
\index{polarizabilities, frequency-dependent}
\index{two-photon transition moment between excited states}
\index{second residues of cubic response functions}
\index{second-order properties, excited states}

In the \Sec{CCEXLR} section input that is specific for 
second residues of coupled cluster cubic response functions 
is read in.
Results obtained using this functionality should cite
\cite{Haettig:EXCITED,Haettig:EXLR}. 
This section includes:
\begin{itemize}
\item frequency-depend second-order properties of excited states
      $$ \alpha^{(i)}_{AB}(\omega) = 
         -\langle\langle A; B\rangle\rangle^{(i)}_\omega $$
      where $A$ and $B$ can be any of the one-electron operators
      for which integrals  are available in the \Sec{*INTEGRALS}
      input part.
\item two-photon transition moments between excited states.
\end{itemize}
Second residues of coupled cluster cubic response functions are
implemented for the models CCS, CC2, and CCSD.
Publications that report results obtained with second residues
of CC cubic response functions should cite Ref.\ \cite{Haettig:EXLR}.

\begin{description}
\item[\Key{OPERAT}] \verb| |\newline
\verb|READ (LUCMD,'(2A)') LABELA, LABELB|\newline
\verb|DO WHILE (LABELA(1:1).NE.'.' .AND. LABELA(1:1).NE.'*')|\newline
\verb|  READ (LUCMD,'(2A)') LABELA, LABELB|\newline
\verb|END DO|

Read pairs of operator labels.
For each of these operator pairs the second residues of the cubic response
function will be evaluated at all frequencies.
Operator pairs which do not correspond to symmetry allowed
combination will be ignored during the calculation.
 
\item[\Key{DIPOLE}] 
Evaluate all symmetry allowed elements of the dipole--dipole tensor
of the second residues of cubic response function 
(max. 6 components for second-order properties, 
 max. 9 for two-photon transition moments).
 
\item[\Key{SELSTA}] \verb| |\newline
\verb|READ (LUCMD,'(A80)') LABHELP|\newline
\verb|DO WHILE(LABHELP(1:1).NE.'.' .AND. LABHELP(1:1).NE.'*')|\newline
\verb|  READ(LUCMD,*) ISYMS(1), IDXS(1), ISYMS(2), IDXS(2)|\newline
\verb|END DO|

Read symmetry and index of the initial state and the final state.
If initial and final state coincide one obtains excited state
second-order properties, if they are different one obtains the
two-photon transition moments between the two excited states.
 
\item[\Key{PRINT }] \verb| |\newline
\verb|READ (LUCMD,*) IPRINT|

Set print parameter for the \Sec{CCEXLR} section.
 
\item[\Key{ALLSTA}] 
calculate polarizabilities for all excited states.
 
\item[\Key{HALFFR}] 
Use half the excitation energy as frequency argument for two-photon
transition moments.
Note, that the \Key{HALFFR} keyword is incompatible with a 
user-specified list of frequencies. \\
For excited state second-order properties the \Key{HALFFR} keyword is
equivalent to the \Key{STATIC} keyword.
 
\item[\Key{USELEF}] 
Use left excited state response vectors instead of the right excited
state response vectors (default is to use the right excited state
response vectors).
 
\item[\Key{FREQ  }]  \verb| |\newline
%  or \Key{FREQUE}
\verb|READ (LUCMD,*) MFREQ|\newline
\verb|READ (LUCMD,*) (BEXLRFR(IDX),IDX=NEXLRFR+1,NEXLRFR+MFREQ)|

Frequency input for $\alpha^{(i)}_{AB}(\omega)$.
 
\item[\Key{STATIC}] 
Add $\omega = 0$ to the frequency list.
 
\end{description}
