\chapter{Electronic transitions}\label{ch:electronic}

In this chapter we discuss the different properties that are
associated with electronic transitions in molecules, and restricted to
those which are
obtainable from \aba . Currently this includes the determination of
the electronic excitation energies, their corresponding transition
moments, and  rotatory strengths. The latter property is related  to
the optical phenomenon Electronic Circular Dichroism (ECD). Note that
more general excitation energies and transition moments can be
obtained using the \resp\ program.

\section{Excitation energies and transition moments}\label{sec:excitation}

Electronic excitations are of general interest as these often
correspond to observed colors of molecules. In addition the
electronic absorption spectrum in the visible and UV part of
the electromagnetic spectrum may be a characteristic fingerprint of a
given molecule. The electronic
absorption spectrum also gives a very good picture of the electronic
state of the molecule, both occupied as well as unoccupied orbitals. 

Note that \aba\ always will use
determinantal basis when doing MCSCF calculations, and a true spin
state may thus not necessarily be obtained. In order to obtain a true
spin state a true state Configuration State Functions should be used
for the CI expansion, and this can be obtained in the \resp\
program. 

In this connection it is of interest to determine the transition
moment of a given excitation. This can easily be done by the following
input which calculates the three lowest excitation energies in irrep 1, the 2
lowest in irrep 2 and the lowest excitation in irrep 3 and their
corresponding transition moments. An input for \aba\ will look like:

\begin{verbatim}
**GENERAL
.RUNALL
*END OF GENERAL
**SIRIUS
*GENERAL
.HARTREE-FOCK
*END OF GENERAL
**ABACUS
.EXCITA
*EXCITA
.DIPSTR
.NEXCITA
    3    2    1    0
*END OF ABACUS
\end{verbatim}
This input will calculate the dipole strength \Key{DIPSTR} of the
6 lowest electronic excitations distributed in a total of 4 symmetries
(as in C$_{2v}$). The dipole strength will with this input only be
calculated in the length form. It is often convenient to calculate it
in the velocity form as well. This can be achieved by adding the
keyword \Key{ROTVEL} in the \Sec{EXCITA} input module in addition
to the \Key{DIPSTR}. Good agreement between the results
obtained in the length and velocity form is an indication that the
basis set used is adequate for describing the electronic transition.
An input for the calculation of transition moments both in the length
and velocity form will therefore look as:

\begin{verbatim}
**GENERAL
.RUNALL
*END OF GENERAL
**SIRIUS
*GENERAL
.HARTREE-FOCK
*END OF SIRIUS
**ABACUS
.EXCITA
*EXCITA
.DIPSTR
.ROTVEL
.NEXCITA
    3    2    1    0
*END OF
\end{verbatim}

\section{Electronic Circular Dichroism (ECD)}\label{sec:ecd}

The calculation of Electronic Circular Dichroism (ECD) is invoked by
the keyword \Key{ECD} in the \aba\ input module. However,
as in calculation of transition moments it is also necessary to
specify the number of excitation energies in 
each symmetry. As ECD is only observed for chiral molecules, such
calculations will in general not have any symmetry, and a complete
input for a molecule without symmetry will thus look like:

\begin{verbatim}
**GENERAL
.RUNALL
*END OF GENERAL
**SIRIUS
*GENERAL
.HARTREE-FOCK
*END OF SIRIUS
**ABACUS
.ECD
*EXCITA
.NEXCIT
    3
*END OF ABACUS
\end{verbatim}

In this run we will calculate the rotatory strength corresponding to
the three lowest electronic excitations (the \Key{NEXCIT} keyword)
using London atomic orbitals. By default the rotatory strengths will
be calculated using London atomic orbitals in order to obtain a
correct dependence upon the gauge origin. The method is thoroughly
described in Ref.~\cite{klbaehkrthjopjtca90}. 
If rotatory strengths obtained without London atomic orbitals is also
wanted, this is easily accomplished by adding the keyword
\Key{ROTVEL} in the \Sec{EXCITA} input module.

There has so far only been presented one study of Electronic Circular
Dichrosim using London atomic orbitals \cite{klbaehkrthjopjtca90}, and the
results of this investigations indicate that the aug-cc-pVDZ, which is
supplied with the \aba\ basis set library, is adequate for such
calculations. 

The two properties may of course be combined a single run, with an
input that would then look like (where we also request the rotatory
strength to be calculated without the use of London orbitals):

\begin{verbatim}
**GENERAL
.RUNALL
*END OF GENERAL
**SIRIUS
*GENERAL
.HARTREE-FOCK
*END OF SIRIUS
**ABACUS
.ECD
.EXCITA
*EXCITA
.DIPSTR
.ROTVEL
.NEXCITA
    6
*END OF
\end{verbatim}

For a more detailed control of the individual parts of the total
calculation of properties related to electronic excitation energies,
we refer to modules affecting the different parts of such
calculations:

\begin{list}{}{\itemsep 0.10cm \parsep 0.0cm}
\item[\bf *EXCITA] Controls the calcution of electronic excitation
energies and the evaluation of all terms contributing to for instance
dipole strength or electronic circular dichroism.

\item[\bf *GETSGY] Controls the setup of the necessary right-hand
sides.
\typeout{>>>>>>Hmmmm, paavirker *LINRES ECD-beregninger?. Uansett, det
ser ut som om det er en overfloeding test paa ECD i GETMRH, for det er
ikke mulig aa komme dit med ECD alene.}
\end{list}
