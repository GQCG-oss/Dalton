\chapter{Molecular response functions, RESPONSE}
\label{ch:response}

\section{General}

\resp\ is the part of the code for calculating molecular
response properties.

\section{\Sec{*RESPONSE} directives}\label{sec:rspinp}

The following directives may be included in the input to \resp.
They are organized according to the program section names in which they
appear.

In addition, SOPPA\index{SOPPA}\index{polarization propagator}
(Second-Order Polarization Propagator 
approximation) for the calculation of linear response
\index{linear response}\index{response!linear}
properties are requested in this input section. The current
implementation of the SOPPA method is described 
in Ref.~\cite{mjpekdtehjajjojcp}. Note that a SOPPA calculation
requires the keyword \Key{SOPPA}.
%M. J. Packer, E. K. Dalskov, T. Enevoldsen, H. J. Aa. Jensen, and
%J. Oddershede, J. Chem. Phys., submitted. 


\subsection{End of input: \Sec{END OF}}

The last directive of the \Sec{*RESPONSE} input may be \Sec{END OF}

\subsection{General: \Sec{*RESPONSE}}

General-purpose directives are given in the \Sec{*RESPONSE} section.

\begin{description}

\item{\Key{NOAVDI}}
Do not use Fock type decoupling of the two-electron density matrix.
Add $F^ID$ instead of $(F^I+F^A)D$ to $E^{[2]}$ approximate
orbital diagonal. 

\item{\Key{INPTES}}
Input test. For debugging purposes only. The program stops after the
input section.

\item{\Key{HIRPA}}
Invoke the higher RPA approximation for the calculation of linear
response properties. This approximation is identical to that of McKoy
and coworkers~\cite{jrtsvmjcp58,tsjrvmjcp58}. The requirements to the
preceding wave function 
calculation is the same as for the \Key{SOPPA} keyword.

\item{\Key{MAXPHP}}\\
\verb|READ *, MAXPHP|\\
Change the maximum dimension of $H_0$ subspace.   Default is 100.
PHP is a subblock of the CI matrix which is calculated explicitly
in order to obtain improved CI trial vectors compared to the
straight Davidson algorithm\cite{erdjcp17}.  The configurations
corresponding to 
the lowest diagonal elements are selected, unless \Key{PHPRESIDUAL} is
specified. MAXPHP is the maximum dimension of PHP, the 
actual dimension will be less if MAXPHP will split degenerate configurations.
 
\item{\Key{MAXRM}}\\
\verb|READ *, MAXRM |\\
Change the maximum dimension of the reduced space. Default is 200.
When solving a linear system of equations or an eigenvalue equation,
the reduced space is increased by 1 in each iteration. For single root
calculations this should exceed the number of iterations required.

\item{\Key{NODOIT}}
Turns off direct one-index transformation \cite{ovhahjajjcc15}. 
In this way all one-index
transformed integrals are stored on disk.

\item{\Key{NOITRA}}
No integral transformation. Normally the two-electron integrals are 
transformed in the beginning of a response calculation. In some cases
this is not desirable, e.g., if the response part is only used for
calculating average values of operator, or if the transformed two-electron
integrals already exist from a previous response calculation. 

\item{\Key{S0MIX}}
Sum rule is calculated in mixed representation, that is, calculate
$N_e=\langle0\mid [r,p] \mid0\rangle$ provided that dipole length and
velocity integrals are available on the property integral file 
(calculated with \Sec{*HERMIT} options \Key{DIPLEN} and \Key{DIPVEL}).
The calculated quantity gives a measure of the quality of the basis
set\index{basis set!quality}.

\item{\Key{OPTORB}}
Orbital trial vectors are calculated with optimal orbital
trial\index{optimal orbital trial vector} vector
algorithm \cite{tuhjahjajpjjcp84}.

\item{\Key{ORBSFT}}\\
\verb|READ *, ORBSFT|\\
Change the amount for shifting the orbital
diagonal\index{orbital diagonal Hessian} of the MCSCF Hessian.
May be used if there is a large number of negative eigenvalues.
Default is $10^{-4}$. 

\item{\Key{ORBSPC}}
Calculation with only orbital operators. 

\item{\Key{PHPRESIDUAL}}
Select configurations for PHP matrix based on largest residual
rather than lowest diagonal elements.

\item{\Key{PROPAV}} \\
\verb|READ '(A)', LABEL|\\
Property average\index{property average}. The average value of an
operator is calculated. 
The line following this option must contain the
label of the operator given in the integral property file.
(See section \ref{ch:hermit}.)

\item{\Key{QRREST}}
Restart quadratic response\index{quadratic response}\index{response!quadratic}
calculation.

\item{\Key{RSPCI}}
Configuration interaction\index{Configuration Interaction}\index{CI}
response calculation. 

\item{\Key{SOPPA}}
Required, no defaults. \\
The SOPPA\index{SOPPA}\index{polarization propagator} flag requires that
the preceding {\sir} calculation has generated the MP2 correlation
coefficients and written them to disk (set \Key{RUN RESPONSE} in \Sec{*DALTON}
input as well as \Key{HF} and \Key{MP2} in \Sec{*WAVE FUNCTIONS}). See
example input in Chapter~\ref{ch:starting}.

\item{\Key{SOPW4}}
Calculate explicitly the W4 term described by Odderhede {\it et
al.\/}~\cite{jopjdlycpr2}. This term is already included in the normal
SOPPA\index{SOPPA}\index{polarization propagator} result, and used mostly for comparing to older
calculations. Note that this keyword requires that \Key{SOPPA} is set.

\item{\Key{TRPFLG}}
Triplet flag. This option is set whenever triplet
(spin-dependent)\index{triplet response}
operators must be used in a response calculation
\cite{jodlypjjcp91,ovhapjhjajthjojcp97}
\end{description}

\subsection{Linear reponse calculation: \Sec{LINEAR}}

A\index{linear response}\index{response!linear} linear response
\cite{jodlypjjcp91,pjhjajjojcp89} calculation is performed for a given
choice of operators, $\langle\!\langle A; B
\rangle\!\rangle_{\omega}$.

\begin{description}

\item{\Key{SINGLE}}
The single residue\index{single residue} of the linear
response\index{linear response}\index{response!linear} function is
computed. Residues of a linear response function correspond to
transition moments\index{transition moment}.

\item{\Key{DIPMAG}}
Sets $A$ and $B$ to angular momentum operators\index{angular momentum}.

\item{\Key{DIPLEN}}
Sets $A$ and $B$ to dipole operators\index{dipole length}.

\item{\Key{DIPVEL}}
Sets $A$ and $B$ to velocity operators\index{dipole velocity}.

\item{\Key{SPIN-O}}
Sets $A$ and $B$ to spin-orbit operators\index{spin-orbit}.

\item{\Key{MAXIT}}\\
\verb|READ (LUCMDS,*) MAXITP|\\
Maximum number of iterations for solving the linear response eigenvalue
equation. Default is 20.

\item{\Key{MAXITO}}\\
\verb|READ (LUCMDS,*) MAXITO|\\
Maximum number of iterations in the optimal orbital
algorithm\index{optimal orbital trial vector}
\cite{tuhjahjajpjjcp84}. 
Default is 5.

\item{\Key{ROOTS}}\\
\verb|READ '*',(ROOTS(I) I=1,NSYM)|\\
Number of roots.  The line following this option contains the number
of excited states\index{excited state} per symmetry. Excitation
energies\index{excitation energy} are calculated for each state and if
any operators are given, 
symmetry-allowed transition moments\index{transition moment} are
calculated between the 
reference state and the excited states.

\item{\Key{PRINT}}\\
\verb|READ *,IPRINT|\\
Sets print level.

\item{\Key{PROPRT}}\\
\verb|READ '(A)', LABEL|\\
Sets $A$ and $B$ to a given operator with label; LABEL.
(See section \ref{ch:hermit}.)

\item{\Key{QUADMO}}
Sets $A$ and $B$ to the quadrupole\index{quadrupole operator} operators.

\item{\Key{RESTPP}}
Restart\index{restart!excitation energy} of response
calculation. This can only be used if the root which is 
specified is the same which was used \textit{last} in the previous
response calculation.

\item{\Key{THCPP}}\\
\verb|READ *,|\\
Threshold for solving the linear response eigenvalue equation. 
Default is $10^{-3}$.

\item{\Key{FREQUE}}\\
\verb|READ *, NFREQ|\\
\verb|READ *, FREQ(1:NFREQ)|\\
Response equations are evaluated at given
frequencies\index{frequency}. Two lines following 
this option must contain 1) The number of frequencies, 2) Frequencies;
 
\item{\Key{RESTLR}}
Restart\index{restart!linear response} of response calculation. This
can only be used if the  
operator specified is the same which was used \textit{last} in the previous
response calculation.

%\item{\Key{OLSEN}}
%CI trial vectors are obtained with Olsen algorithm (for the single
%residue calculation only).

\end{description}

\subsection{Quadratic response calculation: \Sec{QUADRA}}

Calculation of third-order properties as quadratic response
functions\index{quadratic response}\index{response!quadratic}.
$A,B$, and $C$-named options refer to the operators in the quadratic
response function 
$\langle\!\langle A;B,C \rangle\!\rangle_{\omega_b,\omega_c}$
\cite{ovhapjhjajthjojcp97,hhhjajpjjojcp97,haovhkpjthjcp98}

\begin{description}

\item{\Key{SINGLE}}
Computes the single residue\index{single residue} of the quadratic
response function\index{quadratic response}\index{response!quadratic}.
For the case of dipole operators this corresponds to two-photon
transition
moments\index{two-photon!transition moment}\index{transition moment}\index{transition moment!two-photon}.

\item{\Key{DOUBLE}}
Computes the double residue\index{double residue} of the quadratic
response function\index{quadratic response}\index{response!quadratic}.
Double residues of the quadratic repsonse function correspond to transition
moments between excited states\index{transition moment!excited states}. 

\item{\Key{DIPLEN}}
Sets $A$, $B$, and $C$ to dipole operators\index{dipole length}.

\item{\Key{DIPLNX}}
Sets $A$, $B$, and $C$ to the $x$ dipole operator\index{dipole length}.

\item{\Key{DIPLNY}}
Sets $A$, $B$, and $C$ to the $y$ dipole operator\index{dipole length}.

\item{\Key{DIPLNZ}}
Sets $A$, $B$, and $C$ to the $z$ dipole operator\index{dipole length}.

\item{\Key{KERR}}
Only response functions connected to the Kerr effect\index{Kerr effect}, 
$\beta(-\omega; \omega,0)$, are computed.

\item{\Key{FREQUE}}\\
\verb|READ *, NFREQ|\\
\verb|READ *, FREQ(1:NFREQ)|\\
Response equations are evaluated at given
frequencies\index{frequency}. Two lines 
following this option must contain 1) The number of frequencies, 2)
Frequencies.  The double residue does not contain frequencies, for the
single residue and the Kerr effect only the $B$-frequency is set,
and in other cases both $B$ and $C$-frequencies are set.

\item{\Key{PHOSPHORESCENCE}}
Specifies a phosphoresence\index{phosphoresence} calculation, i.e.,
the spin-orbit\index{spin-orbit} 
induced singlet-triplet transition\index{singlet-triplet transition}. This keyword sets up the  
calculation so that no further response input is required; the
$A$ operator is set to the dipole operators\index{dipole length} and
the $B$ operator  
is set to the spin-orbit\index{spin-orbit}
operators. \cite{ovhapjhjajthjojcp97,haovbmaqc27} 

\item{\Key{MCDBTERM}}
Specifies the calculation of all individual components to the 
${\cal{B}}(0\to f)$ term of 
MCD\index{magnetic circular dichroism}\index{B-term}\index{MCD}.
This keyword sets up the calculation so that no further response input is required. 
The $A$ operator is set equal to the $\alpha$ component of dipole 
operator\index{dipole length} and
the $B$ operator to the $\beta$ component of the angular momentum\index{angular momentum}
operator. The resulting "mixed" two-photon transition moment to state $f$ 
is then multiplied for the dipole-allowed one-photon transition moment 
from state $f$ (for the $\gamma$ component, with $\alpha \neq \beta \neq \gamma$).
\cite{Coriani:MCDRSP} 

\item[\Key{APROP}, \Key{BPROP}, \Key{CPROP}]
Specify the operators $A$, $B$, and $C$. The line following this
option should be the label of the operator as it appears in the file
AOPROPER.

\item[\Key{ASPIN}, \Key{BSPIN}, \Key{CSPIN}]
\index{quadratic response}\index{response!quadratic}
Spin information for quadratic response calculations.
The line following these options contains the spin
rank\index{spin rank} of the operators 
$A$, $B$, and $C$, respectively, 0 for singlet operators and 1 for triplet
operators. If \Key{SINGLE} is specified, \Key{CSPIN} denotes the
spin of the excited state. If \Key{DOUBLE} is specified,
both \Key{BSPIN} and \Key{CSPIN} denote excited state spins.
In a triplet response calculations two of these operators are of rank one,
and the remaining operator of rank zero.


\item[\Key{BFREQ}, \Key{CFREQ}]
The frequencies $\omega_b$ and $\omega_c$. Input as in
\Sec{LINEAR}\Key{FREQUE}.

\item{\Key{MAXIT}}
Maximum number of iterations in this section.

\item{\Key{MAXITO}}
Maximum number of iterations in the optimal
orbital\index{optimal orbital trial vector} algorithm 
\cite{tuhjahjajpjjcp84}. 
Default is 5.

\item{\Key{PRINT}}\\
\verb|READ *,IPRINT|\\
Print level.

\item{\Key{SHG}}
Only response functions connected with second harmonic
generation\index{second harmonic generation} 
are computed, $\beta(-2\omega,\omega,\omega)$ .

\item{\Key{THCLR}}
Threshold for solving the linear response equations.
Default is $10^{-3}$.

\item{\Key{MAXITP}}
Maximum number of iterations in solving the linear
response\index{linear response}\index{response!linear} eigenvalue 
equations.

\item{\Key{THCPP}}\\
\verb|READ *, THCPP|\\
Threshold for solving the linear response
\index{linear response}\index{response!linear}
eigenvalue equation. Default is $10^{-3}$.

\end{description}

\subsection{Cubic response calculation: \Sec{CUBIC}}
Calculation of fourth-order properties as cubic response functions\index{cubic response}\index{response!cubic}
\cite{pndjovhacpl242,djpnhajcp105,pndjhapdkrthhkcpl253}.
$A,B$,$C$, and $D$-named options refer to the operators in the cubic
response function 
$\langle\!\langle A;B,C,D \rangle\!\rangle_{\omega_b,\omega_c,\omega_d}$

\begin{description}

\item[\Key{APROP}, \Key{BPROP}, \Key{CPROP}, \Key{DPROP}]
Specify the operators $A$, $B$, $C$, and $D$. The line following this
option should be the label of the operator as it appears in the file
AOPROPER.

\item[\Key{BFREQ}, \Key{CFREQ}, \Key{DFREQ}]
The frequencies\index{frequency} $\omega_b$, $\omega_c$ and $\omega_d$. Input as in
\Sec{LINEAR}\Key{FREQUE}.

\item{\Key{THG}}
Only response functions connected to the third harmonic
generation\index{third harmonic generation} are
computed, $\gamma(-3\omega;\omega,\omega,\omega)$ \cite{djpnylhajcp105}.

\item{\Key{DC-SHG}}
Only response functions connected to the static electric field-induced
second harmonic generation\index{electric field!induced SHG} are computed,
$\gamma(-2\omega;\omega,\omega,0)$.

\item{\Key{DC-KER}}
Only response functions connected to the static electric field induced
Kerr effect\index{electric field!induced Kerr} are computed,
$\gamma(-\omega;\omega,0,0)$.

\item{\Key{IDRI}}
Only response functions connected to the intensity dependent 
refractive\index{refractive index!intensity dependent} index are computed,
$\gamma(-\omega;\omega,-\omega,\omega)$.

\item{\Key{FREQUE}}\\
\verb|READ *, NFREQ|\\
\verb|READ *, FREQ(1:NFREQ)|\\
Sets the frequencies\index{frequency} whenever a optical process is specified.
Can also be used for the residue calculation and it does then set 
both $\omega_b$ and $\omega_c$ for the single residue and only
$\omega_b$ for the double residue.

\item{\Key{OLDVEC}}
Read the response vectors specified in the file CRINFO from
file.

\item{\Key{SINGLE}}
Computes the single residue\index{single residue} of the cubic
response function\index{cubic response}\index{response!cubic}.
In the case of dipole operators this corresponds to
three-photon absorption\index{three-photon!absorption}.

\item{\Key{DOUBLE}}
Computes the double\index{double residue} residue of the cubic
response function\index{cubic response}\index{response!cubic}.
In the case of dipole operators this corresponds to excited
state polarizabilities and two-photon transition
moments\index{two-photon!transition moment!excited states}\index{excited state!polarizability} 
between excited states \cite{djpnylhajcp105}.

\item{\Key{DIPLEN}}
Sets $A$, $B$, $C$, and $D$ to dipole operators\index{dipole length}.

\item{\Key{DIPLNX}}
Sets $A$, $B$, $C$, and $D$ to the $x$ dipole operator\index{dipole length}.

\item{\Key{DIPLNY}}
Sets $A$, $B$, $C$, and $D$ to the $y$ dipole operator\index{dipole length}.

\item{\Key{DIPLNZ}}
Sets $A$, $B$, $C$ and $D$ to the $z$ dipole operator\index{dipole length}.

\item{\Key{MAXIT}}
Maximum nuber of iterations for solving linear equations, default value is 20.

\item{\Key{MAXITO}}
Maximum number of ORPCTL-microiterations, default value is 10.

\item{\Key{PRINT}}
Print flag for output, default value is 2. Timer information is printed
out if printflag greater than 5. Response vectors printed out if
printflag greater than 10.

\item{\Key{THCLR}}
Threshold for convergence of response vectors, default value is $10^{-4}$.

\item{\Key{THRNRM}}
Threshold for norm of property vector $X^{[1]}$ in order to solve the linear
equation \\
$\left( E^{[2]} - S^{[2]} \right)N^{X} = X^{[1]}$, default
value is $10^{-9}$. 

\item{\Key{MAXITP}}
Maximum number of iteration for solving eigenvalue equation, default
value is 20.

\item{\Key{ROOTS}}
Number of roots to converge. \\
\verb|READ (LUCMDS,*) (NTMCNV(J),J=1,NSYM)|\\

\item{\Key{THCPP}}
Threshold for convergence of eigenvector, default value is $10^{-6}$.

\end{description}

\subsection{Spherical multipole moments: \Sec{C6}}\index{multipole moment}

\begin{description}

\item{\Key{C6SPH}, \Key{C8SPH}, \Key{C10SPH}}
Specification of one of \Key{C6SPH}, \Key{C8SPH}, \Key{C10SPH}
calculates and writes to an interface file (LURSP7) the spherical multipole
moments in the specified/default grid points needed for C6, C8, and C10
coefficients, respectively $(L=1, L=1,2,$ or $L=1,2,3$; all for $M_L =
-L,...,0,...,L)$.

\item{\Key{C6ATM}, \Key{C8ATM}, \Key{C10ATM}}
\Key{C6ATM}, \Key{C8ATM}, \Key{C10ATM} do the same as \Key{C6SPH} etc. for
atoms. Only $M_L=0$ is
calculated and written to file (all $M_L$ vaules give same multipole moment 
for 
atoms).

\item{\Key{C6LMO}, \Key{C8LMO}, \Key{C10LMO}}
\Key{C6LMO}, \Key{C8LMO}, \Key{C10LMO} is \Key{C6SPH} etc. for linear
molecules\index{linear molecule}. Only
multipole moments\index{multipole moment} with zero or positive $M_L$
is calculated and written to 
file.

\end{description}

\subsection{Hyperfine Coupling Elements: \Sec{ESR}}

Calculation of ESR properties\index{ESR}\index{hyperfine coupling}: hyperfine coupling tensors using 
the Restricted-Unrestricted Approach\index{restricted-unrestricted method}.

\begin{description}

\item{\Key{MAXIT}}      \\
\verb|READ (LUCMDS,'(I5)'),MAXESR |\\
   The line following gives the maximum number of iterations.  (Default = 20)

\item{\Key{PRINT}}     \\
\verb|READ (LUCMDS,'(I5)'),IPRESR |\\
   The line following gives the print level for ESR routines.

\item{\Key{SNGPRP}}    \\
\verb|READ (LUCMDS,'(A)'), LABEL|\\
   Singlet Operator. The line following is the label in the AOPROPER file.

\item{\Key{THCESR}}     \\
\verb|READ (LUCMDS,*),THCESR|\\
   The line following is the threshold for convergence (Default = 1.0D-5)

\item{\Key{TRPPRP}}    \\
\verb|READ (LUCMDS,'(A)'), LABEL |\\
   Triplet Operator. The line following is the label in the AOPROPER file.


\end{description}




