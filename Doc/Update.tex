\chapter{UPDATE}\label{ch:update}\index{UPDATE|textbf}

\section{General}

\siraba\ is maintained using a utility named UPDATE\index{UPDATE} --- this is a
FORTRAN code that mimics some features of the CDC and CRAY system\index{Cray}
software of the same name.  In particular, it provides an
``include'' function by allowing the contents of {\em common
decks\/} to be inserted into the source code on demand.  In
addition, it allows {\em conditional generation\/} of source code:
depending on whether certain flags are set or not code will be
generated or ignored.  The version distributed was originally
written by N.~H.~F.~Beebe, and has been extensively modified by
H.~J.~Aa.~Jensen.  We will describe the various features only briefly
here.

\section{Operation}

UPDATE\index{UPDATE}\index{UPDATE|options} is normally invoked by a
script, conventionally named 
\verb|upd|.  Source file names presented to the script have a
\verb|.u| suffix, and each is processed in turn to give a
\verb|.f| file, that is, true FORTRAN source.  Apart from the list
of input file names, the \verb|upd| script takes several options.
\begin{description}
\item[\verb|-c|~{\em file\/}] Specifies {\em file\/} is a common
deck file.  This replaces any existing common deck file
specifications.
\item[\verb|-C|~{\em file\/}] Add {\em file\/} to the existing
list of common deck files.
\item[\verb|-d|~{\em name\/}] Specifies {\em name\/} is a defined
name (see Sec.~\ref{sec:upddef}) for this update.  This replaces
any existing defined names.
\item[\verb|-D|~{\em name\/}] Add {\em name\/} to the existing
list of defined names.
\item[\verb|-o|~{\em file\/}] Write informative output to {\em
file\/}.  Default is \verb|upd.lis|
\item[\verb|-e|~{\em file\/}] Write diagnostic output to {\em
file\/}.  Default is \verb|upd.err|
\end{description}

Typically, the common deck list is empty and the define list
contains only the current target machine name at start-up.  The
initial values are set inside the script.  A typical invocation
might be
\begin{verbatim}
upd -c gpf9.cdk -D NOBLAS *.u
\end{verbatim}
to update all the files in the current directory, using
\verb|gpf9.cdk| as the common deck file and adding the name
\verb|NOBLAS| to the define list.

\section{Source files and UPDATE directives}\label{sec:upddef}

Source files can contain anything, in principle, but usually
contain FORTRAN code, together with UPDATE directives.  The latter
are described here --- they must begin in column~1.\index{UPDATE|directives}
\begin{description}
\item[\verb|*DECK|~{\em deck\/}] Introduces a deck named {\em
deck\/}; conventionally each module constitues a separate
deck, called by the name of the routine.  A deck ends at the next
\verb|*DECK| or \verb|*COMDECK| directive, or at end-of-file.
\item[\verb|*COMDECK|~{\em comdeck\/}] Introduces a common deck
named {\em comdeck\/}.  A common deck ends at the next
\verb|*DECK| or \verb|*COMDECK| directive, or at end-of-file.
\item[\verb|*CALL |~{\em comdeck\/}] Insert the contents of common
deck {\em comdeck\/} at this point in the source.  It is a fatal
error if {\em comdeck\/} cannot be found in any of the common deck
files specified for this update.
\item[\verb|*DEFINE|~{\em name\/}] Add {\em name\/} to the list of
defined names for this update.
\item[\verb|*IF DEF,|{\em name1, name2...\/}] If {\em name1\/} AND
{\em name2\/}, etc., are defined, retain the next block of text
in the source.
\item[\verb|*ENDIF|] Terminates a block of text begun with an
\verb|*IF| statement.  \verb*|*END IF| is also acceptable.
\item[\verb|*IF -DEF,|{\em name1, name2...\/}] If {\em name1\/} OR
{\em name2\/}, etc., are defined, delete the next block of text
in the source.
\item[\verb|*ELSE|] Provides an alternative for the current
\verb|*IF|: if the condition causes the text following the
\verb|*IF| to be included, the text between \verb|*ELSE| and
\verb|*ENDIF| is deleted, and vice versa.
\end{description}

It is possible to nest \verb|*IF| constructions with this version
of UPDATE.  However, this should be avoided where possible,
especially when \verb|*ELSE| is being used as well, since it
becomes almost impossible to decide what will happen under a given
set of circumstances.

In this version of UPDATE, the convention has been adopted that
common decks in a \verb|.u| file are ignored without messages.
This provides a mechanism for including general comments and text
in a file.  Thus all common decks needed for a particular update
must reside in their own file or files.

With the above descriptions, it should be simple to see how the
different versions of the source code are generated from the one
master copy.  In fact, UPDATE itself can be maintained with UPDATE
--- all that is required is a FORTRAN version for some machine to
which the user has access to ``bootstrap'' others.
