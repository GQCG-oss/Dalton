\chapter{Introduction}\label{ch:intro}
\pagenumbering{arabic}

          The \siraba\ program system is designed to allow
convenient, automated determination of a large number of molecular
properties based on an SCF, MP2, Coupled Cluster, or MCSCF reference wave function.
 The program consists
of six separate components, developed more or less independently.
\her\ is the integral generator, generating ordinary
atomic\index{atomic integrals} and molecular integrals appearing
in the time-independent, non-relativistic Schr\"{o}dinger
equation, as well as an extensive number of integrals related to
different molecular properties. \eri\ is a vectorized and
distribution-oriented integral generator that may be invoked in
certain calculations, in particular in integral-direct coupled-cluster
calculations. \sir\ is the (MC)SCF wave function 
optimization part, and is described comprehensively in the
METECC-94 book~\cite{hjajhajomotecc}. \aba\ evaluates the
second-order molecular properties of interest for SCF and MCSCF wave
functions, in particular second-order static molecular properties in
which the basis set depends on the applied pertturbation. \resp\ is a general-purpose
program for evaluating response functions, up to cubic response
functions for (MC)SCF wave functions and linear response for the
Second-Order Polarization Propagator Approach (SOPPA). New for this
release is a Coupled Cluster program that can calculate
coupled-cluster energies and response properties for a number of
coupled-cluster approximations. 

Throughout this manual there will be references to articles
describing the implementation of a specific molecular property or
input option. This will hopefully suffice to give the reader a proper
theoretical understanding of the current implementation. Only one
reference otherwise not mentioned in the text, the treatment of
symmetry\index{symmetry}, is given here~\cite{prttca69}.

\siraba\ is in many respects an ``expert's'' program. This is most
noticeable in the range and selection of molecular properties that may
be calculated and the flexibility and stability of the wave function
implementations available. As described in the individual sections, the range of
molecular properties, some of which are highly non-standard, is fairly
large. On the other hand, several common properties can perhaps more
easily be calculated using other, commercially available, quantum
chemistry program packages.

We have tried when writing this manual to emphasize the modularity of
the input, as well as indicating  the advantages that may be obtained from
the flexibility of the input. Yet, whether the authors of
this manual have succeeded or not, is up to the reader to decide, and
any comments and suggestions for improvements, both on the manual as
well as the program, will be much welcome.

\section{General description of the manual}

The manual is divided into three main sections:

\begin{description}
\item The Installation Guide (Chapter 2---3) describes the
installation and configuration of the \siraba\ program. Although of
interest primarily to those installing the program, the section
describes the program's use of memory and the default settings for
various parameters in the program and how they may be changed. Thus,
we strongly recommend that all users read through this section at least
once.

\item The User's Guide (Chapter 4---14) will be the most important part
of this manual for most users. These chapters provide a  short
introduction on how to do a calculation with the \siraba\ program,
considering both the supplied shell-script as well as some examples of
input files for a complete run of the program. A short description of
the different output files is also given.

These chapters also describe the input file needed to
calculate different molecular properties. Some
suggestions for how to do a calculation most efficiently, as well as
recommendations for basis sets to be used when calculating non-standard
molecular properties are also included. Although highly biased, we
still hope that they may be of some
use to the user. These are the most useful sections for
ordinary calculations done with the \siraba\ program.

{\em In these chapters there are boxes indicating what we call {\em
reference literature}\index{reference literature}. This reference
literature not only serve as
the best introduction to the implementational aspects of a given
calculational procedure, but is also our recommended reference in published
work employing specific parts of the code.}

\item The Reference Manual (Chapter 15---20) will for most calculations be of
little use to the ordinary user apart from Chapter 17, which describes
the most important file in any quantum chemistry calculation: the
description of the molecule being studied. However, all possible
keywords that may be given to direct the calculation using \siraba\ is
documented in this Reference Manual.

\item Appendix A gives a description of various tools provided to us
from different users for pre- or post-processing of \siraba\
input/output.
\end{description}

Before starting we thus recommend reading\index{recommended reading}
the following chapters: 4, 5, and 17, and one or more of chapters
6---14, depending on the
calculations you would like to do with the program. Chapters
2 and 3 are also highly recommended.

Questions\index{questions} and bug reports\index{bug fixes} should be directed to
\verb|dalton-admin@kjemi.uio.no|. Help on input and/or error messages
produced during execution of the program should be directed to the
\siraba\ mailing list\index{mailing list}, \verb|dalton-users@kjemi.uio.no|.

\section{Acknowledgement}

We are grateful to Prof. Bj\"{o}rn O. Roos and Dr. Per-Olof
Widmark, University of Lund, for kindly allowing us to include in
this distribution of the \siraba\ program routines for integral
presorting\index{integral presorting} (BOR), two-electron integral
transformation routines\index{integral transformation} (BOR) and
routines for syncronous and asyncronous I/O (POW). The
distribution of these routines in the \siraba\ distribution is
done with the consent of the authors of these routines.

We are grateful to professor Knut F\ae gri (University of Oslo) who
took the time to reoptimize the basis set of van Duijneveldt in order
to provide in electronic form a basis set of similar performance to
these widely used basis set, included here as the Not Quite van
Duijneveldt (NQvD) basis sets.

We would also like to express our gratitude to Sonia Coriani (Trieste
University), Gilbert Hangartner (University of Freiburg), and Antonio
Rizzo (Istituto di Chimica Quantistica ed Energetica Molecolare del
CNR, Pisa), for allowing us to distribute their utility programs for
pre- and post-processing of \siraba\ input/output-files, as described
in the Appendix.

A long list of users, too long to mention here, is thanked for an
enormous amount of valuable feedback on the performance of the
code. Without their assistance, the code would probably not looked the
way it does now.
