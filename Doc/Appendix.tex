\section*{Appendix: DALTON Tool box}

This appendix describes the pre- and postprocessing programs
supplied to us by various users and authors. The programs are designed
to create files directly usable by the \dalton\ program, or do
various final analyses of one or more \dalton\ output files. We
strongly encourage users to supply us with any such programs they have
written in connection with their own use of the \dalton\ program,
which we will be glad to distribute along with the \dalton\
program.

This appendix gives a short description of the programs supplied with
the current distribution of the program, with proper references and a
short description of the use of the program.

Executable versions of the programs in this directory will
automatically be produced during installation of the \dalton\ program,
but they will by default not be installed in the installation
directory of the \dalton\ program, but rather placed in the
\verb|tools| directory.

\subsection*{A1: FChk2HES}

\noindent
{\large\bf Author: \normalsize\large Gilbert Hangartner,
Universit\"{a}t Freiburg, Switzerland}

\smallskip

\noindent 
{\bf Purpose:} Reads the Formatted Checkpoint file of Gaussian  
     and creates a \verb|DALTON.HES| file with Hessian and Atomic coordinates
     as well as a dummy \verb|MOLECULE.INP| file.

\smallskip
\noindent
{\bf Commandline:} FChk2HES [-n] [filename]

-n does not read and write geometry; filename is optional, default is
"\verb|Test.FChk|" .

\smallskip
\noindent
{\bf Comments:}     The Gaussian checkpointfile "Test.FChk" is created
     with the keyword "FormCheck=ForceCart", and where the keyword "NoSymm"
     has  been specified

If symmetry is used: Hessian\index{Hessian} will be in standard(Gaussian) orientation,
        but geometry will be in input orientation. So do not use the
        \verb|MOLECULE.INP| file from this utility, 
        neither the coordinates in the end of \verb|DALTON.HES|.
        Use either the standard-orientation from Gaussian output file, or
        use the converted checkpointfile (created with the command
        "formchk checkpointfilename convertedfilename") as an input for
        this program. Anyway, this will not lead to the geometry you 
        specified in the input, and the Hessian will be incompatible with the
        calculation you want to do in \dalton. (If this was already carried
        out at the input orientation ...)

The only way to get the geometry orientation you want is to turn
        off symmetry. 
        Then, both the Hessian and the geometry in the Test.FChk file are
        in the input orientation, and everything is fine.

The program has been made for the purpose of being used in VCD or
VROA\index{VCD}\index{ROA}\index{vibrational circular
dichroism}\index{Raman optical activity} 
calculations where it may be of interest to compare predicted spectra
obtained from SCF London orbital invariants/AATs\index{atomic axial
tensor}\index{AAT} with a MP2 or DFT force field.

\subsection*{A2: labread}

\noindent
{\large\bf Author: \normalsize\large Hans J\o rgen Aa.Jensen,
Odense University, Denmark}

\smallskip

\noindent 
{\bf Purpose:} Reads unformatted AO-integral files and prints the
INTEGRAL labels found on the file

\smallskip
\noindent
{\bf Commandline:} labread $<$ infile $>$ outfile

\smallskip
\noindent
{\bf Comments:}  A convenient tool in connection with \dalton\ errors
connected to missing labels on a given file in order to check that the
given integrals do indeed, or do not, exist on a given file (usually
\verb|AOPROPER|). 

\subsection*{A3: ODCPROG}

\noindent
{\large\bf Author: \normalsize\large Antonio Rizzo (Istituto di
Chimica Quantistica ed Energetica Molecolare, Pisa, Italy), and
Sonia Coriani (University of Trieste, Italy)}

\smallskip

\noindent 
{\bf Purpose:} Analyze the magnetizability and nuclear shielding
polarizabilities from a set of finite-field magnetizability and
nuclear shielding calculations.

\smallskip
\noindent
{\bf Commandline:} odcprog

Requires the existence of a readmg.dat file, and $n$ \verb|DALTON.CM| files,
where $n$ is the number of finite field output files.

\smallskip
\noindent
{\bf readmg.dat:} 

\begin{verbatim}
TITLE (1 line)
N ZPRT IPG ISSYM NIST NLAST NSTEP
FILE1
FILE2
... 
FILEN 
\end{verbatim}
where
\begin{itemize}
\item[N] Number of output DALTON.CM files for the FF calculations
\item[ZPRT]  Sets the print level (T for maximum print level, F otherwise)
\item[IPG] Point group (1=Td, 2=Civ, 3=D2h, 4=C2v, 5=C3v, 6=Dih)
\item[ISSYM] Site symmetry of the atom for which the shielding
               polarizabilities are required
\item[NIST] First atom of which shielding polarizabilities must be computed
               (of those in  the \verb|DALTON.CM| files)
\item[NLAST] Last atom of which shielding polarizabilities must be computed
               (of those in  the \verb|DALTON.CM| files)
\item[NSTEP] Step to go from nist to nlast in the do-loop for
               shielding polarizabilities
\item[FILEn] $n$th \verb|DALTON.CM| file (current name and location)
\end{itemize}

\smallskip
\noindent
{\bf Comments:} 
    Only 6 point groups are presently implemented, of which
    C3v only for shielding-polarizabilities and 
    Dih only for hyperpolarizabilities 

    When \verb|issym.ne.ipg| and both
    hypermagnetizabilities\index{magnetizability polarizability}\index{shielding polarizability}
    and shielding 
    polarizabilities are required, the number of field set-ups should 
    be equal to the one for shieldings

    Different field set-ups are needed according to
    molecular  and/or nuclear site symmetries
    See for reference Raynes and Ratcliffe~\cite{wtrrrmp37}.
    Linear molecules along the Z axis, planar molecules
    on XZ plane. In general, follow standard point
    symmetry arrangements

\begin{itemize}
\item[T$_d$]   Eg. CH$_4$ - No symmetry - 3 calculations

\begin{verbatim}
Need     0  0  0
Need     0  0  z
Need     x  0  z
\end{verbatim}

\item[C$_{\infty v}$] Eg. CO - No symmetry - 5 calculations

\begin{verbatim}
Need     0  0  0
Need     x  0  0
Need     0  0  z
Need     0  0 -z
Need     x  0  z
\end{verbatim}

\item[D$_{2h}$] Eg. C$_2$H$_4$ -  No symmetry - 7 calculations

\begin{verbatim}
Need     0  0  0
Need     0  0  z
Need     x  0  0
Need     0  y  0
Need     x  y  0
Need     x  0  z
Need     0  y  z
\end{verbatim}

\item[C$_{2v}$] Eg. H$_2$O - No symmetry - 8 calculations

\begin{verbatim}
Need     0  0  0
Need     0  0  z
Need     0  0 -z
Need     x  0  0
Need     0  y  0
Need     x  y  0
Need     x  0  z
Need     0  y  z
\end{verbatim}

\item[C$_{3v}$] Eq. Shielding H1 in CH$_4$

\begin{verbatim}
Need     0  0  0
Need     0  0  z
Need     0  0 -z
Need     x  0  0
Need    -x  0  0
Need     x  0  z
\end{verbatim}

\item[D$_{\infty h}$] Eg. N$_2$ -  4 calculations

\begin{verbatim}
Need     0  0  0
Need     0  0  z
Need     x  0  0
Need     x  0  z
\end{verbatim}
\end{itemize}

%\subsection*{A4: arturo}
%\noindent
%{\large\bf Author: \normalsize\large Asger Halkier (Aarhus University, Denmark)
%and Sonia Coriani (University of Trieste, Italy)}
%
%\smallskip
%
%\noindent
%{\bf Purpose:} Performs central difference analysis of (finite
%field) energy calculations with \cc.
%
%\smallskip
%\noindent
%{\bf Commandline:} arturo
%
%Requires the existence of 
%
%\smallskip
%\noindent
