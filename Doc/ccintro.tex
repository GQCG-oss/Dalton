\chapter{Coupled Cluster calculations, CC}\label{ch:CC}
\index{coupled cluster calculations}
\index{coupled cluster program}
\index{CCS}
\index{CC2}
\index{CCD}
\index{CCSD}
\index{CCSD(T)}
\index{CC3}

The coupled cluster program \cc\ is designed for large-scale
correlated calculations of energies and properties using the
hierachy of coupled cluster models: CCS, CC2, CCSD and CC3.
The program offers almost the same options as the other
parts of the Dalton program for SCF and MCSCF wave functions.
It thus contains a wave function optmization section and 
a response function section
where linear, quadratic and cubic response functions are calculated.
For CC3 however, it is only possible to calculate the wave function and 
singlet excitation energies. 
\index{response functions}
At the moment molecular gradients are available 
only up to the CCSD level \index{gradient}.  
London orbitals have so far not been implemented 
for the calculation of magnetic properties.
\index{magnetic properties}\index{London orbitals} 
In the manual more details can be found about the specific implementations.

An additional feature of the program is that all levels of correlation
treatment have been implemented using integral-direct techniques making
it possible to run calculations using large basis sets.  
\index{integral-direct techniques}

In this chapter the general structure of the input for the
coupled cluster program is described.
The complete input for the coupled cluster program appears as
sections in the input for the \sir\ program, with the general
input in \Sec{CC INPUT}. In order to get into the CC program
one has to specify the \Key{CC} keyword in the general input
section of \sir\ \Sec{*WAVE FUNCTIONS}. A minimal
input file f.x.\ for a CCSD(T) energy calculation would be:
\begin{verbatim}
**GENERAL
.RUNSIR
**WAVE FUNCTIONS
.CC
*CC INPUT
.CC(T)
*END OF INPUT
\end{verbatim}

