\chapter{Quick guide}\label{ch:quick}

\section{Introduction}\label{sec:quickintro}

          This chapter is designed to provide a quick introduction for
the impatient user.  All of the topics discussed here are dealt with
in considerable more detail in the other chapters.  Here we shall
consider the CH$_2$ test job from the \verb|testin/CH2| subdirectory as a
simple illustrative example.

\section{Job script and input files}\label{sec:quickscr}

The following files will be found in the subdirectory
\verb|testin/CH2|: \verb|A.sh|, \verb|A_her.inp|, \newline
\verb|A00_mol.inp|,
\verb|A_sirstr.inp|, \verb|A_sirwlk.inp|, \verb|A_abastr.inp|,
\verb|A_abawlk.inp|, and \newline \verb|A_abafin.inp|.  There is also a file
named \verb|A.nqs|, which illustrates one way of running the job under
NQS. We now discuss these files in turn.

\subsection{{\tt A.sh}}

\begin{verbatim}
ABAWRK=12000000
HERWRK=5000000
export ABAWRK
export HERWRK
root=/users/helgaker
ABA=$root/exeabashell/abacus.sh
scr=/scratch/taylor
inpdir=$root/testin/CH2 \
outdir=$root/testin/CH2 \
rootscr=$scr \ 
herexe=$exe/hermit.x \ 
sirexe=$exe/sirius.x \ 
abaexe=$exe/abacus.x \ 
wf=MCSCF $ABA CH2 A
\end{verbatim}

This script specifies the location of various executable files and the
``master script'' \verb|abacus.sh|, the directory path to generate run
files in~(\verb|$scr|), as well as the directory for input and output
files. These latter two options, inpdir and outdir, are optional, and
will by default be set to the current directory. The ``molecule name''
for this run 
is~\verb|CH2|, and the letter identifying this particular run
is~\verb|A|.  The run files will therefore be generated in the directory
\begin{verbatim}
/scratch/taylor/CH2/A
\end{verbatim}
for this calculation.  The output files will be copied back to the
\verb|outdir| directory automatically.

The user should modify these pathnames to reflect the local
installation and disk structure.  The file \verb|A.sh| can then be
submitted for execution.  (The individual input files \verb|A_*.inp|
are discussed further in the next sections, as are output files.)  If
the run is successful, the output files can be compared to those
supplied in the subdirectory \verb|testout/CH2|.  While there will be
many small differences due to differences in hardware, precision, etc,
the energies and properties should be the same.

After the test run, the naming convention for files should be clear.
The first step in a walk is assigned the number~\verb|00|, and this is
incremented by successive steps.  Some files receive just this number
--- for example, the output from the \her\ integral calculation for
the first step of this walk would be named \verb|A00_her.out|.  Some
files appear with an extra digit or letter, reflecting the possibility
that in some walks a step that is later deemed unsuccessful, according
to some optimization criterion, will be taken; the script will then
try to ``backstep'' and the backstep number will be appended to the
two-digit number.  The first \sir\ output from the walk is thus named
\verb|A000_sir.out|.  The final \aba\ calculation after convergence
receives an~\verb|X| in this position, so the output might be called
\verb|A02X_aba.out|. 

\subsection{{\tt A\_her.inp}}

\begin{verbatim}
*HERMIT
*END OF
\end{verbatim}

This file contains control information for the \her\ calculation.  It
rarely requires any changes between different calculations, and on
some machines a \verb|here| document contained within the master
script can replace it.

\subsection{{\tt A00\_mol.inp}}

(Note that the column spacing in this file has been adjusted to
fit the printed page and does not reflect the true format of some
lines.) 
\begin{verbatim}
BASIS
4-31G**
         Methylene test job --- triplet geometry
      4-31G** basis
    2    2  X  Y  Z   1.00D-15
    9999.0       3.0
        6.    1
C     0.0000000000000000  0.0000000000000000  0.               *
        1.    1
H     1.871093            0.0000000000000000  0.82525          *
\end{verbatim}

This file contains the necessary information about molecular
coordinates and basis set for the first step, numbered~\verb|00|.
The equivalent file for subsequent steps is generated
automatically by \aba\ and the master script.

\subsection{{\tt A\_sirstr.inp}}

\begin{verbatim}
**SIRIUS
*GENERAL INPUT
.ABACUS
.TITLE
Methylene test
.HARTREE-FOCK
.MP2
.MCSCF
.NSYM
 4
*RHF CALCULATION
.HF OCCUPATION
 3 1 0 0
*MP2 CALCULATION
.MP2 FROZEN
 1 0 0 0
*WAVE FUNCTION
.SYMMETRY
 3
.SPIN MULT
 3
.INACTIVE
 1 0 0 0
.ELECTRONS
 6
.CAS SPACE
 3 2 1 0
*OPTIMI
.DETERM
.TRACI
.MAX MICRO ITERATIONS
 30
.MAX MACRO ITERATIONS
 30
.MAX CI ITERATIONS AT START
 10
**END OF SIRIUS
\end{verbatim}

This file describes the input, for the first walk step, for a
CASSCF calculation on the 
methylene ${}^3B_1$ ground state.  The active space comprises the
six valence orbitals.  The program first performs a
Hartree-Fock calculation on the ${}^1A_1$~state, and then
generates MP2 natural orbitals for this state to provide starting
orbitals for the MCSCF calculation.  Note that it is currently a
requirement that determinants be used as the many-particle basis
in \sir\ (the \verb|.DETERM| option in the \verb|*OPTIMIZ|
section).

\subsection{{\tt A\_sirwlk.inp}}

\begin{verbatim}
**SIRIUS
*GENERAL INPUT
.ABACUS
.TITLE
Methylene test
.MCSCF
.NSYM
 4
*WAVE FUNCTION
.SYMMETRY
 3
.SPIN MULT
 3
.INACTIVE
 1 0 0 0
.ELECTRONS
 6
.CAS SPACE
 3 2 1 0
*ORBITAL INPUT
.MOSTART
 NEWORB
*CI VEC
.ABACUS
*OPTIMI
.DETERM
.TRACI
.MAX MICRO ITERATIONS
 30
.MAX MACRO ITERATIONS
 30
**END OF SIRIUS
\end{verbatim}

This file also describes the input for a CASSCF calculation on the 
methylene ${}^3B_1$~ground state, but here for all walk steps
other than the first.  Hence starting orbitals are available from the
previous walk step. 

\subsection{{\tt A\_abastr.inp}}

\begin{verbatim}
*ABACUS
.WALK
.TOTSYM
*GEOANA
.ANGLES
    1
    2    1    3
*WALK
.START
*END OF INPUT
\end{verbatim}

This file contains the \aba\ input for the first walk step.  For
minimizations, the only input that would normally be changed here
is the specification of geometry parameters for printing in the
\verb|*GEOANA| section. Additional molecular properties may also be
wanted to be requested, in particular if the start geometry
corresponds to the experimental geometry.

\subsection{{\tt A\_abawlk.inp}}

\begin{verbatim}
*ABACUS
.WALK
.TOTSYM
*GEOANA
.ANGLES
    1
    2    1    3
*WALK
*END OF INPUT
\end{verbatim}

This file contains the \aba\ input for walk steps subsequent to
the first.  The only difference with the file \verb|A_abastr.inp|
is in the absence of the \verb|.START| keyword from the
\verb|*WALK| section, since now information about previous steps
in the walk is available for use in the optimization. Furthermore, no
property calculations are usually wanted during a geometry optimization.


\subsection{{\tt A\_abarej.inp}}

\begin{verbatim}
*ABACUS
.WALK
.REJECT
.FROM
*WALK
.REJECT
*READIN
.UNIT
    9
*GEOANA
.ANGLES
    1
    2    1    3
*END OF INPUT
\end{verbatim}

This input file will only be used if a step is being rejected, that
is, the potential energy surface show deviations from a quadratic
surface within the given boundaries. The main purpose of the input is
to reuse as much of previously calculated information as possible, by
reading this from file and starting in the \verb|walk|-part of the
calculation with a reduced trust radius.

\subsection{{\tt A\_abafin.inp}}

\begin{verbatim}
*ABACUS
.PRINT
    2
.VIBANA
.DIPGRA
.POLARI
*READIN
.UNIT
    9
*GEOANA
.ANGLES
    1
    2    1    3
*END OF INPUT
\end{verbatim}

This file controls the final \aba\ calculation, in which specified
properties such as vibrational frequencies, intensities, dipole
gradiant and frequency independent polarizabilities will be computed.
It might often be that a given set of properties is wanted both in an
experimental as well as in an optimized geometry. The same keywords
are then added in the {\tt A\_abastr.inp} and {\tt A\_abafin.inp}
files, whereas no such properties will then be calculated during the
run.

\section{Output files}

The contents of the output files will not be discussed in any
detail here, since information in the formatted output files
should be self-evident.  Some general points follow.

\subsection{{\tt A*her.out}, {\tt A*sir.out}, {\tt A*aba.out}}

These files contain the output from \her, \sir, and \aba,
respectively.  They allow the progress of a walk to be monitored
step by step.  The volume of information in these files can be
controlled by setting print levels, as described in later
chapters.

\subsection{{\tt A\_abacus.err}, {\tt A*.err}}

These files contain diagnostic information and error messages (if
any).  Note that some error messages from the master script are
routine and can be ignored if the run otherwise appears to have
been satisfactory.

\subsection{{\tt A.log}}

This file is a log of messages generated by the master script.
Since the codes have developed considerably since the master
script was first written, the log contains a number of warning
messages that can safely be ignored, particularly about testing in
\sir, if the script has continued to execute.

\subsection{{\tt A*.f21}, {\tt A*.geo}}

These files contain wave function and restart information
essential for the next step in a walk.  They can be very large for
large MCSCF calculations and/or large basis sets.

\subsection{Miscellaneous files}

Small files with names ending \verb|.mol|, \verb|.rst| and several
other suffixes are generated to be passed to other walk steps.
These can be deleted once the job has finished.
