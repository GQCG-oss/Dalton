%%%%%%%%%%%%%%%%%%%%%%%%%%%%%%%%%%%%%%%%%%%%%%%%%%%%%%%%%%%%%%%%%%%
\chapter{Examples of generalized active space CI calculations}
\label{ch:lucitaexamples}
%%%%%%%%%%%%%%%%%%%%%%%%%%%%%%%%%%%%%%%%%%%%%%%%%%%%%%%%%%%%%%%%%%%

In this Chapter we discuss two examples of input files (\verb|DALTON.INP|) 
for generalized active space CI calculations based on the \lucita\ module 
of the Dalton program. Other examples may be found in the test suite. 

\section{Energy calculation with a GAS-type active space decomposition I}
%
\lucita\  allows for the calculation of the ground and excited state energies
of a given system based on the definition of active spaces using the concept of generalized active spaces. 
A list of compulsory and optional keywords can be found in Section~\ref{sec:lucita-inp}. 
The following input describes the calculation of SCF and GASCI ground state energies of HBr as well as of the 
first excited singlet state of A1 symmetry:

%
\begin{verbatim}
**DALTON INPUT
.RUN WAVE FUNCTIONS
**WAVE FUNCTIONS
.HF
.GASCI
*SCF INPUT
.DOUBLY OCCUPIED
 9 4 4 1
*LUCITA
.TITLE
 HBr molecule ground state + first excited state in A1
.INIWFC
 HF_SCF                       ! we start from a closed-shell HF reference wave function
.CITYPE
 GASCI                        ! GASCI calculation
.SYMMET
 1                            ! symmetry irrep A1
.MULTIP
 1                            ! singlet states
.NACTEL
 8                            ! number of active electrons
.INACTIVE
 7 3 3 1                      ! inactive (doubly occupied) orbitals
.GASSHE
 3                            ! number of GA spaces
  2  1  1  0                  ! orbital distribution in GA space I
  4  2  2  1                  ! orbital distribution in GA space II
  3  1  1  0                  ! orbital distribution in GA space III
.GASSPC
 1                            ! always 1
  5  8                        ! minimum # of accumulated e- in GAS I   <--> max. # of e-
  6  8                        ! minimum # of accumulated e- in GAS II  <--> max. # of e-
  8  8                        ! always: ``number of active e- number of active e-''
.NROOTS
 2                            ! we want to converge on two eigenstates
.MAXITR
 16                           ! stop the calculation after 16 Davidson CI iterations
.MXCIVE
 6                            ! subspace dimension
.ANALYZ                       ! print leading coefficients of the final CI wave function
.DENSI
 1                            ! print natural orbital occupation numbers
**END OF DALTON INPUT
\end{verbatim}

The basis set input (\verb|MOLECULE.INP|) used in this example reads as:

\begin{verbatim}
BASIS
cc-pVDZ
HBr with small basis set

    2    2  X  Y   a
        1.    1
H     0.0000000000            0.0000000000           1.414431
       35.    1
Br    0.0000000000            0.0000000000           0.000000
FINISH
\end{verbatim}

\section{Energy calculation with a GAS-type active space decomposition II}

We now repeat the GASCI calculation from the previous section 
with a slightly modified but otherwise equivalent input. Note that here we split the former GAS II space into three 
spaces with the ``$\sigma$''-, ``$\pi$''-\ and ``$\delta$''-like orbitals put into different spaces. We also omit 
the symmetry specification as the default is the totally symmetric irrep (here A1).

%
\begin{verbatim}
**DALTON INPUT
.RUN WAVE FUNCTIONS
**WAVE FUNCTIONS
.HF
.GASCI
*SCF INPUT
.DOUBLY OCCUPIED
 9 4 4 1
*LUCITA
.TITLE
 HBr molecule ground state + first excited state in A1
.INIWFC
 HF_SCF                       ! we start from a closed-shell HF reference wave function
.CITYPE
 GASCI                        ! GASCI calculation
.MULTIP
 1                            ! singlet states
.NACTEL
 8                            ! number of active electrons
.INACTIVE
 7 3 3 1                      ! inactive (doubly occupied) orbitals
.GASSHE
 5                            ! number of GA spaces
  2  1  1  0                  ! orbital distribution in GA space I
  4  0  0  0                  ! orbital distribution in GA space II
  0  2  2  0                  ! orbital distribution in GA space III
  0  0  0  1                  ! orbital distribution in GA space IV
  3  1  1  0                  ! orbital distribution in GA space V
.GASSPC
 1                            ! always 1
  5  8                        ! minimum # of accumulated e- in GAS I    <--> max. # of e-
  6  8                        ! minimum # of accumulated e- in GAS II   <--> max. # of e-
  6  8                        ! minimum # of accumulated e- in GAS III  <--> max. # of e-
  6  8                        ! minimum # of accumulated e- in GAS IV   <--> max. # of e-
  8  8                        ! always: ``number of active e- number of active e-''
.NROOTS
 2                            ! we want to converge on two eigenstates
.ANALYZ                       ! print leading coefficients of the final CI wave function
.DENSI
 1                            ! print natural orbital occupation numbers
**END OF DALTON INPUT
\end{verbatim}
