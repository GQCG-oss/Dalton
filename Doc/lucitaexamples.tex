%%%%%%%%%%%%%%%%%%%%%%%%%%%%%%%%%%%%%%%%%%%%%%%%%%%%%%%%%%%%%%%%%%%
\chapter{Examples of generalized active space CI calculations}
\label{ch:lucitaexamples}
%%%%%%%%%%%%%%%%%%%%%%%%%%%%%%%%%%%%%%%%%%%%%%%%%%%%%%%%%%%%%%%%%%%

In this Chapter we discuss three examples of input files (\verb|DALTON.INP|) 
for generalized active space (GAS) and restricted active space (RAS) 
CI calculations based on the \lucita\ module of the Dalton program. 
Other examples may be found in the test suite. 

\section{Energy calculation with a GAS-type active space decomposition I}\label{cc:lucitaex:gas1}
%
\lucita\  allows for the calculation of the ground and excited state energies
of a given system based on the definition of active spaces using the concept of generalized active spaces. 
A list of compulsory and optional keywords can be found in Section~\ref{sec:lucita-inp}. 
The following input describes the calculation of SCF and GASCI ground state energies of HBr as well as of the 
first excited singlet state of A1 symmetry:

%
\begin{verbatim}
**DALTON INPUT
.RUN WAVE FUNCTIONS
**WAVE FUNCTIONS
.HF
.GASCI
*SCF INPUT
.DOUBLY OCCUPIED
 9 4 4 1
*LUCITA
.TITLE
 HBr molecule ground state + first excited state in A1
.INIWFC
 HF_SCF                       ! we start from a closed-shell HF reference wave function
.CITYPE
 GASCI                        ! GASCI calculation
.SYMMET
 1                            ! symmetry irrep A1
.MULTIP
 1                            ! singlet states
.INACTIVE
 7 3 3 1                      ! inactive (doubly occupied) orbitals
.GAS SHELLS
 3                            ! number of GA spaces
 5  8 / 2  1  1  0            ! min max # of accumulated e- / orbitals in GA space I
 6  8 / 4  2  2  1            ! min max # of accumulated e- / orbitals in GA space II
 8  8 / 3  1  1  0            ! min max # of accumulated e- / orbitals in GA space III
.NROOTS
 2                            ! we want to converge on two eigenstates
.MAXITR
 16                           ! stop the calculation after 16 Davidson CI iterations
.MXCIVE
 6                            ! subspace dimension
.ANALYZ                       ! print leading coefficients of the final CI wave function
.DENSI
 1                            ! print natural orbital occupation numbers
**END OF DALTON INPUT
\end{verbatim}

The basis set input (\verb|MOLECULE.INP|) used in this example reads as:

\begin{verbatim}
BASIS
cc-pVDZ
HBr with small basis set

    2    2  X  Y   a
        1.    1
H     0.0000000000            0.0000000000           1.414431
       35.    1
Br    0.0000000000            0.0000000000           0.000000
FINISH
\end{verbatim}

\section{Energy calculation with a GAS-type active space decomposition II}\label{cc:lucitaex:gas2}

We now repeat the GASCI calculation from the previous section 
with a modified GA space input (yielding nevertheless the {\bf{same}}\ CI configuration space) 
and asking for the first two triplet states in B1 symmetry. 
Note in particular the splitting of the former GAS II space into three 
spaces with the ``$\sigma$''-, ``$\pi$''-\ and ``$\delta$''-like orbitals separated into different spaces. 
Remember that the min/max number of electrons in the final GAS space have to be always identical and 
specify the number of active (``correlated'') electrons. The kewyword \Key{NACTEL} is thus merely optional for 
GASCI calculations whereas it is mandatory for RASCI calculations (see next section \ref{cc:lucitaex:ras}). 

%
\begin{verbatim}
**DALTON INPUT
.RUN WAVE FUNCTIONS
**WAVE FUNCTIONS
.HF
.GASCI
*SCF INPUT
.DOUBLY OCCUPIED
 9 4 4 1
*LUCITA
.TITLE
 HBr molecule ground state + first excited state in A1
.INIWFC
 HF_SCF                       ! we start from a closed-shell HF reference wave function
.CITYPE
 GASCI                        ! GASCI calculation
.SYMMET
 3                            ! symmetry irrep B1
.MULTIP
 3                            ! triplet states
.INACTIVE
 7 3 3 1                      ! inactive (doubly occupied) orbitals
.GAS SHELLS
 5                            ! number of GA spaces
 5  8 / 2  1  1  0            ! min max # of accumulated e- / orbitals in GA space I
 5  8 / 4  0  0  0            ! min max # of accumulated e- / orbitals in GA space II
 5  8 / 0  2  2  0            ! min max # of accumulated e- / orbitals in GA space III
 6  8 / 0  0  0  1            ! min max # of accumulated e- / orbitals in GA space IV
 8  8 / 3  1  1  0            ! min max # of accumulated e- / orbitals in GA space  V
.NROOTS
 2                            ! we want to converge on two eigenstates
.ANALYZ                       ! print leading coefficients of the final CI wave function
.DENSI
 1                            ! print natural orbital occupation numbers
**END OF DALTON INPUT
\end{verbatim}

\section{Energy calculation with a RAS-type active space decomposition}\label{cc:lucitaex:ras}
%
In order to show the simple relation between RASCI and GASCI calculations with \lucita\
we now consider the test case from Section \ref{cc:lucitaex:gas1}\ 
based on the definition of RA spaces. Note, that internally the RASCI input is converted to a GASCI 
input which will be printed in the output.

%
\begin{verbatim}
**DALTON INPUT
.RUN WAVE FUNCTIONS
**WAVE FUNCTIONS
.HF
.GASCI
*SCF INPUT
.DOUBLY OCCUPIED
 9 4 4 1
*LUCITA
.TITLE
 HBr molecule ground state + first excited state in A1
.INIWFC
 HF_SCF                       ! we start from a closed-shell HF reference wave function
.CITYPE
 RASCI                        ! RASCI calculation
.SYMMET
 1                            ! symmetry irrep A1
.MULTIP
 1                            ! singlet states
.NACTEL
 8                            ! number of active electrons
.INACTIVE
 7 3 3 1                      ! inactive (doubly occupied) orbitals
.RAS1
  2  1  1  0                  ! orbital distribution in RA space I
 3                            ! maximum number of holes
.RAS2
  4  2  2  1                  ! orbital distribution in RA space II
.RAS3
  3  1  1  0                  ! orbital distribution in RA space III
 2                            ! maximum number of e-
.NROOTS
 2                            ! we want to converge on two eigenstates
.MAXITR
 16                           ! stop the calculation after 16 Davidson CI iterations
.MXCIVE
 6                            ! subspace dimension
.ANALYZ                       ! print leading coefficients of the final CI wave function
.DENSI
 1                            ! print natural orbital occupation numbers
**END OF DALTON INPUT
\end{verbatim}

