\chapter{Miscellaneous properties}\label{ch:miscprop}

This chapter shortly describes how to calculate certain properties not
otherwise described in the previous chapters. Most of these properties
are self-explanatory, as well as requiring only very modest amount of
input. In addition, it should be possible to extract the information
contained in this chapter from the chapter describing the general
abacus input~\ref{sec:abacus}. Because of this, they are only given a
very short description.

\section{Population analysis}\label{sec:popana}

A number of population analysises can be performed in \sir , including
Mulliken population analysis and gross and net molecular orbital
population analysis. We refer to the \sir\ input manual for more
information of how to perform these poulation analysis.

However, a population analysis based upon the dipole gradient was
presented by Cieslowski some years ago \cite{jcjacs111}, and it was
shown that this population analysis gives results that are more in the
spirit of chemical intuition for several molecular systems. This
population analysis can be invoked by addint the keyword
\verb|.POPANA| in the \verb|*ABACUS| input module.

\section{Static polarizabilites}\label{sec:polari}

Whereas we in the Sec.~\ref{sec:vroa} described how frequency
dependent polarizabilities could be calculated, we here only note that
frequency independent polarizabilities can be calculated by using the
keyword \verb|.POLARI| in the general \verb|*ABACUS| input module.

\section{Dipole moment and dipole gradient}\label{sec:dipmom}

Dipole moment and dipole moment gradients (also known as Atomic Axial
Tensors (APTs)) can be calculated by the
keyword \verb|.DIPGRA| in the general input section \verb|*ABACUS|.

\section{Vibrational analysis}\label{sec:vibrasjon}

