
%%%%%%%%%%%%%%%%%%%%%%%%%%%%%%%%%%%%%%%%%%%%%%%%%%%%%%%%%%%%%%%%%%%
\section{Transition moments between two excited states: \Sec{CCQR2R}}
\label{sec:ccqr2r}
\index{Coupled Cluster!quadratic response}
\index{quadratic response}
\index{response!quadratic}
\index{oscillator strength!excited states!Coupled Cluster}
\index{transition moment!excited states!Coupled Cluster}
\index{transition strength!excited states!Coupled Cluster}
%%%%%%%%%%%%%%%%%%%%%%%%%%%%%%%%%%%%%%%%%%%%%%%%%%%%%%%%%%%%%%%%%%%

In the \Sec{CCQR2R} section the input that is
specific for coupled cluster response calculation of excited state--excited state
electronic transition properties is read in.
This section includes presently for example calculation of excited state
oscillator strength.
The transition properties are available for the models CCS, CC2 and CCSD,
but only for singlet states.
%Publications that report results obtained with this CC module should cite Ref.\ \cite{Christiansen:CCLR}.
The theoretical background for the implementation is 
detailed in Ref.\ \cite{Christiansen:CCLR,Christiansen:QEL}.
This section has to be used in connection with \Sec{CCEXCI} for the 
calculation of excited states.

\begin{center}
\fbox{
\parbox[h][\height][l]{12cm}{
\small
\noindent
{\bf Reference literature:}
\begin{list}{}{}
\item O.~Christiansen, A.~Halkier, H.~Koch, P.~J{\o}rgensen, and T.~Helgaker \newblock {\em J.~Chem.~Phys.}, {\bf 108},\hspace{0.25em}2801, (1998).
\end{list}
}}
\end{center}

\begin{description}

\item[\Key{DIPOLE}]
%
Calculate the ground state--excited state dipole (length) transition properties including
the oscillator strength.
%
\item[\Key{DIPVEL}]
%
Calculate the excited state--excited state dipole-velocity  transition properties including
the oscillator strength in the dipole-velocity form. (The dipole length form is recommended
for standard calculations).
%
\item[\Key{NO2N+1}]
%
Use an alternative, and normally less efficient, formulation for calculating
the transition matrix elements (involving solution of response equations for
all operators instead of solving for the so-called $N$ vectors which is the default).
%
\item[\Key{OPERAT}]\verb| |  \newline
\verb|READ (LUCMD,'(2A8)') LABELA, LABELB|\newline
%
Read pairs of operator labels for which the residue of the linear response
function is desired.
Can be used to calculate the transition property for a given operator
by specifying that operator twice. The operator can be any of the one-electron
operators for which integrals are available in the \Sec{*INTEGRALS} input part.
%
\item[\Key{SELEXC}] \verb| | \newline
\verb|READ(LUCMD,*) IXSYM1,IXST1,IXSYM2,IXST2|\newline 
%
Select which excited states the calculation of transition properties
are carried out for. The default is all states according to the CCEXCI input section
(the program takes into account symmetry). For calculating selected states only,
provide a list of symmetry and state numbers (order after increasing energy in
each symmetry class) for state one and for state two.
This list is read until next input label is found.

\end{description}

