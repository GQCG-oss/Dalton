\chapter{Molecular properties, ABACUS}\label{ch:abacus}

\section{Directives for evaluation of molecular properties}\label{sec:abainp}

          The following directives may be included in the input to
\aba.  They are organized according to the program section (module) names
in which they can appear.  

\subsection{General: \Sec{*PROPERTIES}}\label{subsec:abacus}

This module controls the main features of the property calculation, that is,
which properties is to be calculated. In addition it includes
directives affecting the performance of several of the program
sections. It should be noted, however, that the specification of what
kind of walk (minimization\index{geometry optimization}, location of
transition states\index{transition state}, dynamical  
walks\index{dynamics}) is given in the \Sec{WALK} or \Sec{MINIMI}
submodules in the 
general input  module. See also Chapter~\ref{ch:geometrywalks}. 

\begin{description}

\item[\Key{ABALNR}] Invokes the calculation of frequency dependent
linear response\index{frequency}\index{linear response} calculations
of operators. At present this only 
includes the frequency dependent polarizability\index{polarizability} as well as directives
controlling the calculation of Raman related properties\index{Raman
intensity}\index{Raaman optical activity}\index{ROA}.

\item[\Key{CAVORG}]\verb| |\newline
\verb|READ (LUCMD,*) (CAVORG(ICOOR), ICOOR = 1, 3)|

Reads the origin to be used for the cavity\index{cavity
origin}\index{reaction field} during a solvent
calculation. By this default this is chosen to be the center of
mass\index{center of mass}. Should by used with care, as it has to
correspond to the center 
used in the evaluation of the undifferentiated solvent integrals in
the \her\ section.

\item[\Key{DIPGRA}] Invokes the calculation of dipole moment
gradients\index{dipole gradient}\index{atomic polar tensor}\index{APT}
(commonly known also as Atomic Polar Tensors 
(APTs)) as described in Ref.~\cite{tuhhjajpjjcp84}. If combined with a
request for \Key{VIBANA} this will generate IR intensities\index{IR
intensity}.  

\item[\Key{DIPORG}]\verb| |\newline
\verb|READ (LUCMD, *) (DIPORG(ICOOR), ICOOR = 1, 3)|

Reads in a user defined dipole origin\index{dipole origin}. This may
affect properties in 
which changes in dipole origin\index{dipole origin} is canceled by
similar changes in the 
nuclear part. It should also be used with care, as the same dipole
origin must be used during the integral evaluation sections , in
particular if one is doing numerical 
differentiation with respect to electric field perturbations. For such
finite-field calculations\index{finite field} we refer to Chapter
\ref{ch:finite}, which deals with finite field calculations. It's
use is mainly for debugging.  

\item[\Key{ECD}] Invokes the calculation of Electronic Circular
Dichroism (ECD)\index{electronic circular dichroism}\index{ECD} as
described in Ref.~\cite{klbaehkrthjopjtca90}. This 
necessitates the specification of the number electronic
excitations\index{electronic excitation} in 
each symmetry, given in the \Sec{EXCITA} module. The reader is
refered to the section where the calculation of ECD is described in
more detail (Sec.~\ref{sec:ecd}).

\item{\Key{EXCITA}} Invokes the calculation of electronic
excitation\index{electronic excitation}\index{excitation energy}
energies as residues of linear response functions\index{linear
response}\index{single residue} as described by
Olsen and J\o rgensen \cite{jopjjcp82}. It also calculates closely
related properties like transition moments\index{transition
moment}\index{rotatory strength} and 
rotatory strengths.

\item[\Key{EXPFCK}] Invokes the simultaneous calculation of
two-electron expectation values and derivative Fock-matrices. This is
default in direct and parallel runs in order to save memory. In
ordinary calculations the total CPU time will increase as a result of
invoking this option.

\item[\Key{GAUGEO}]\verb| |\newline
\verb|READ (LUCMD, *) (GAGORG(ICOOR), ICOOR = 1, 3)|

Reads in a user defined gauge origin\index{gauge origin} and overwrite
both the \Key{NOCMC} option, as well as the default value of
center-of-mass coordinates. Not that an unsymmetric position of the
gauge origin will lead to wrong results in calculations employing
symmetry. 

\item[\Key{INPTES}] Checks the input only and then stops.

\item[\Key{ISOTOP}]\verb| |\newline
\verb|READ (LUCMD,*) NIS|\newline
\verb|READ (LUCMD, *) (ISOTOP(IS), IS = 1, NIS)|

Read one more line indicating the number of nuclei for which the
isotope will be explicitly given. On the next line read the
isotope chosen for each nuclei\index{isotopic constitution}.

By default the isotope chosen for each nucleus is the most abundant
one. Although one may choose the number of nuclei for which an
isotopic substitution is to be given explicitly, one should notice
that this ordering follows the ordering of the atoms in the input.
Thus, if only the last atom is to be an isotope different from the
most common, all atoms has to be specified (and this applies to {\em all}
nuclei, not only the symmetry independent one, se also Sec.~\ref{sec:vibfreq}).

The ordering of the isotopes for each nuclei is in the order of
natural abundance. Thus deuterium will be isotope 2 of hydrogen, while
tritium will be isotope 3.

The choice of isotopic substituted molecules will affect the gauge
origin\index{gauge origin}, as well as dynamical walks\index{dynamics}
and walks using mass-weighted normal\index{mass-weighted coordinate} 
modes, which depend on the choice of 
isotopic substitution. We notice that for the vibrational
analysis\index{vibrational analysis},
the isotopic substitution may be introduced rather late, there
is a similar \Key{ISOTOP}| keyword in the \Sec{VIBANA} input
module. This can also be used to study the vibrational and rotational
structure of several isotopic substituted species.

\item[\Key{MAGNET}] Invokes the calculation of the molecular
magnetizability\index{magnetizability} (commonly known as magnetic
susceptibility) as 
described in Ref.~\cite{krthklbpjhjajjcp99} and the rotational {\em g}
tensor (see keyword \Key{MOLGFA})\index{rotational g tensor}.  By
default this is done 
using London orbitals\index{London orbitals} in order to 
ensure fast basis set convergence as shown in
Ref.~\cite{krthklbpjhjajjcp99}. The use of London
orbitals can be disabled by the keyword \Key{NOLOND}.

Furthermore, the natural connection\index{natural connection}
(Ref.~\cite{joklbkrthpjtca90,krthjopjklbcpl235}) is default in order to ensure
numerically stable results. The natural
connection can be turned off and instead use the symmetric
connection\index{symmetric connection}
by the keyword \Key{NODIFC}. 

The gauge origin\index{gauge origin} are chosen to be the center of
mass\index{center of mass} of the molecule. 
This origin can be changed by the two keywords \Key{GAUGEO} and
\Key{NOCMC}. This will of course not affect the total magnetizability,
only the size of the dia- and paramagnetic terms.

\item[\Key{MOLGFA}] Invokes the calculation of the rotational
{\em g} tensor\index{rotational g tensor} as described in
Ref.~\cite{jgkrthjcp105} and the molecular 
magnetizability\index{magnetizability} (see keyword \Key{MAGNET}). By
default this is done 
using London orbitals\index{London orbitals}  and the 
natural connection\index{natural connection}. The use of London
orbitals can be turned off by the keyword \Key{NOLOND}.

By definition the gauge origin\index{gauge origin} of the molecular
g-factor is to be the 
center of mass\index{center of mass} of the molecule, and although the
gauge origin can be 
changed through the keywords \verb|.NOCMC | and \verb|.GAUGEO|, this
is not recommended, and may give erroneous results.

\item[\Key{MOLGRA}] Invokes the calculation of the analytical
molecular gradient as \index{gradient}described in Ref.~\cite{tuhjahjajpjjcp84}.

\item[\Key{MOLHES}] Invokes the calculation of the analytical
molecular Hessian\index{Hessian} and gradient\index{gradient} as
described in Ref.~\cite{tuhjahjajpjjcp84}. 

\item[\Key{NACME}] Invokes the calculation of non-adiabatic
coupling\index{non-adiabatic coupling element} matrix elements as
described in 
Ref.~\cite{klbpjhjajjothjcp97}. Presently, complete non-adiabatic
coupling matrix elements cannot be obtained from this keyword alone,
but has to be combined with subsequent calculations in the
\resp\ program. Currently inactive option.

\item[\Key{NMR}] Invokes the calculation of both parameters
entering the NMR spin-Hamiltonian, that is nuclear
shieldings\index{nuclear shielding} and
indirect nuclear spin-spin coupling\index{spin-spin coupling}
constants. The reader is refered to the 
description of the to keywords \Key{SHIELD} and \Key{SPIN-S}.

\item[\Key{NOCMC}] This keyword sets the gauge origin\index{gauge
origin} to be equal to the origin of the Cartesian Coordinate system,
that is (0,0,0). This keyword is automatically invoked in case of VCD
calculations.

\item[\Key{NODARW}] Turns off the calculation of the Darwin
correction\index{Darwin correction}. By default the two relativistic
corrections to the 
energy, the mass-velocity\index{mass-velocity correction} and Darwin
corrections, are calculated 
perturbatively.

\item[\Key{NODIFC}] Disables the use of the natural
connection\index{natural connection}, and 
uses instead the symmetric connection\index{symmetric connection}. The
natural connection and 
its differences as compared to the symmetric connection is described
in Ref.~\cite{joklbkrthpjtca90,krthjopjklbcpl235}. 

As the symmetric connection may give numerical inaccurate results,
it's use is not recommended for other than comparisons with other
programs.

\item[\Key{NOHESS}] Turns off the calculation of the analytical
molecular Hessian\index{Hessian}. This option overrides any request for the
calculation of molecular Hessians.

\item[\Key{NOLOND}] Turns off the use of London atomic
orbitals\index{London orbitals} in
the calculation of molecular magnetic properties. The gauge origin is
by default then chosen to be the center of mass. This can be altered
by the keywords \Key{NOCMC} and \Key{GAUGEO}.

\item[\Key{NOMASV}] Turns off the calculation of the
mass-velocity\index{mass-velocity correction}
correction. By default the two relativistic corrections to the
energy, the mass-velocity and Darwin corrections\index{Darwin
correction}, are calculated 
perturbatively.

\item[\Key{NQCC}] Calculates the nuclear quadrupole moment
coupling constant\index{nuclear quadrupole coupling}\index{NQCC}. 

\item[\Key{PHASEO}]\verb| |\newline
\verb|READ (LUCMD, *) (ORIGIN(ICOOR), ICOOR = 1, 3)|

Changes the origin of the phase-factors entering the London atomic
orbitals. This will change the value of all of the contributions to
the different magnetic field dependent properties when using London
atomic orbitals. To be used for debugging purposes
only.

\item[\Key{POLARI}] Invokes the calculation of frequency-independent
polarizabilities\index{polarizability}. See the keyword \Key{ALFA} in
the \Sec{ABALNR} 
input module for the calculation of frequency-dependent
polarizabilities.

\item[\Key{POPANA}] Invokes a population analysis\index{population analysis}\index{dipole gradient} based on the 
dipole gradient as first introduced by Cioslowski \cite{jcjacs111}.
This flag also invokes the \Key{DIPGRA} flag.

\item[\Key{PRINT}]\verb| |
\newline
\verb|READ (LUCMD, *) IPRDEF|

Set default print level for the calculation.  Read one
more line containing print level. Default print level is the
value of \verb|IPRDEF| from the general input module.

\item[\Key{QUADRU}] Calculates the molecular quadrupole
moment\index{quadrupole moment}. This
includes both the electronic and nuclear contribution to the
quadrupole moment. These will printed separately only if a printed
level of 2 or higher has been chosen. Note that quadrupole moment is
defined according to Buckingham \cite{adbacp12}, and it is not
transformed to the principal moments of inertia coordinate system.

\item[\Key{RAMAN}] Calculates Raman intensities\index{Raman
intensity}, as described in 
Ref.~\cite{thkrklbpjjofd99}. This property needs a lot of settings
in order to perform correctly, and the reader is therefore refered to
Section~\ref{sec:vroa}, where the
calculation of this property is described in more detail.

\item[\Key{REPS}] \verb| |\newline
\verb|READ (LUMCD, *) NREPS|\newline
\verb|READ (LUMCD, *) (IDOSYM(I),I = 1, NREPS)|

Consider perturbations of selected symmetries only.  Read one more
line specifying how many symmetries, then one line listing the
desired symmetries. This option is currently only implemented
for geometric perturbations. 

\item[\Key{RESTAR}] Restart in the property evaluation section. This
keyword is currently disabled. 

\item[\Key{SELECT}]\verb| |
\newline
\verb|READ (LUCMD,*) NPERT|\newline
\verb|READ (LUCMD, *) (IPOINT(I),I=1,NPERT)|

Select which nuclear geometric perturbations are to be considered.
Read one more line specifying how many perturbations, then on a
new line the sequence of perturbations to be considered. By default,
all perturbations are to be considered, but by invoking this keyword,
only those perturbations specified in the sequence will be considered. 

The perturbation ordering follows the ordering of the symmetrized
nuclear coordinates. This ordering can be obtained by setting the
print level in the \verb|*READIN| module to 20 or higher.

\item[\Key{SHIELD}] Invokes the calculation of nuclear
shielding\index{nuclear shielding}
constants. By default this is done using London orbitals\index{London
orbitals} in order to 
ensure fast basis set convergence as shown in
Ref.~\cite{kwjfhppjacs112,krthrkpjklbhjajjcp100}. The use of London
orbitals can be disabled by the keyword \verb|.NOLOND|.

Furthermore, the natural connection\index{natural connection}
(Ref.~\cite{joklbkrthpjtca90,krthjopjklbcpl235}) is default in order to ensure
numerically stable results as well as physically interpretable
results for the paramagnetic and diamagnetic terms. The natural
connection can be turned off and instead use the symmetric
connection\index{symmetric connection}
by the keyword \verb|.NODIFC|. 

The gauge origin\index{gauge origin} are chosen to be the center of
mass\index{center of mass} of the molecule.
This origin can be changed by the two keywords \verb|.GAUGEO| and
\verb|.NOCMC |. This choice of gauge origin will of course not affect
the final shieldings, only the size of the dia- and paramagnetic
contributions.

\item[\Key{SPIN-R}] Invokes the calculation of
spin-rotation\index{spin-rotation constant}
constants as described in Ref.~\cite{jgkrthjcp105}. By default this is
done using London orbitals\index{London orbitals}  and the 
natural connection\index{natural connection}. The use of London
orbitals can be turned off by the keyword \verb|.NOLOND|.

By definition the gauge origin\index{gauge origin} of the
spin-rotation constant is to be the 
center of mass\index{center of mass} of the molecule, and although the
gauge origin can be 
changed through the keywords \verb|.NOCMC | and \verb|.GAUGEO|, this
is not recommended, and may give erroneous results.

In the current implementation, symmetry dependent nuclei cannot be
used during the calculation of spin-rotation constants.

\item[\Key{SPIN-S}] Invokes the calculation of indirect nuclear
spin-spin coupling\index{spin-spin coupling} constants. By default all
spin-spin couplings 
between nuclei with naturally occuring isotopes with abundance more
than 1\% and non-zero spin will be calculated, as well as all the different
contributions (Fermi contact, dia- and paramagnetic spin-orbit and
spin-dipole)\index{Fermi
contact}\index{spin-dipole}\index{paramagnetic spin-orbit}\index{diamagnetic spin-orbit}. The implementation is described in 
Ref.~\cite{ovhapjhjajsbpthjcp96}. 

As this is a very time consuming property, as well as requiring MCSCF
wave functions in order to get reliable results, it is recommended to
consult the chapter describing the calculation of NMR-parameters
(Ch.~\ref{ch:magnetic}). The main control of which
contributions and which nuclei to calculate spin-spin couplings
between is done in the \verb|*SPIN-S| module.

\item[\Key{VCD}] Invokes the calculation of Vibrational Circular
Dichroism (VCD)\index{VCD}\index{vibrational circular dichroism}
according to the scheme described in 
Ref.~\cite{klbpjthkrhjajjcp98}.  By default this is done using London
orbitals\index{London orbitals} in order to 
ensure fast basis set convergence as shown in
Ref.~\cite{klbpjthkrhjajjcp100}. The use of London
orbitals can be disabled by the keyword \verb|.NOLOND|.

Furthermore, the natural connection\index{natural connection}
(Ref.~\cite{joklbkrthpjtca90,krthjopjklbcpl235}) is default in order to ensure
numerically stable results. The natural
connection can be turned off and instead use the symmetric
connection\index{symmetric connection}
by the keyword \verb|.NODIFC|. 

The gauge origin\index{gauge origin} are chosen to be the center of
mass\index{center of mass} of the molecule. 
This origin can be changed by the two keywords \verb|.GAUGEO| and
\verb|.NOCMC |. This will of course not affect the final VCD results,
only the size of the contributing mechanisms.

In the current implementation, the keyword \Key{NOCMC} will be set
true in calculations of Vibrational Circular Dichroism, that is, the
coordinate system origin will be used as gauge origin.

\item[\Key{VIBANA}] Invokes a vibrational analysis\index{vibrational analysis} in the current
geometry. This will generate the vibrational frequencies in the
current point. If combined with \verb|.DIPGRA| the IR intensities
will be calculated as well\index{IR intensity}. 

\item[\Key{VROA}] Invokes the calculation of Vibrational Raman
Optical Activity\index{Raman optical activity}\index{ROA}, as
described in Ref.~\cite{thkrklbpjjofd99}. This 
property needs a lot of settings in order to perform correctly, and
the reader is therefore refered to Section~\ref{sec:vroa}, where the
calculation of this property is described in more detail.

\item[\Key{WRTINT}] Forces the magnetic first-derivate two-electron
integrals to be written to disc. This is default in MCSCF
calculations, but not for SCF runs. This file can be very large, and
it is not recommended to use this option for ordinary SCF runs.

\end{description}

\subsection{Calculation of Atomic Axial Tensors (AATs):
\Sec{AAT}}\label{sec:aat}  

Directives for controlling the calculation of Atomic Axial
Tensors\index{atomic axial tensor}\index{AAT},
needed when calculating Vibrational Circular Dichroism
(VCD)\index{vibrational circular dichroism}\index{VCD}.
\begin{description}

\item{\verb|.INTPRI|}\verb| |\newline
\verb|READ (LUCMD,*)| 

Set the print level in the calculation of the necessary differentiated
integrals when calculating Atomic Axial Tensors\index{atomic axial tensor}\index{AAT}. Read one more line
containing print level. Default value is value of \verb|IPRDEF|
from the general input module. The print level of the rest of the
calculation of Atomic Axial Tensors are controled by the keyword
\verb|.PRINT |.

\item{\verb|.NODBDR|} Skip contributions originating from first
half-differentiated overlap\index{overlap, half-differentiated}
integrals with respect to both nuclear 
distortions as well as magnetic field. This will give wrong results
for VCD\index{VCD}\index{vibrational circular dichroism}. Mainly for debugging purposes.

\item{\verb|.NODDY |} Checks the calculation of the electronic part of
the Atomic Axial Tensors\index{atomic axial tensor}\index{AAT} by calculating these both in the ordinary
fashion as well as by a noddy routine. The program will not
perform a comparison, and will not abort if differences is found.
Mainly for debugging purposes.

\item{\verb|.NOELC |} Skip the calculation of the pure electronic
contribution to the Atomic Axial Tensors\index{atomic axial tensor}\index{AAT}. This will give wrong results
for VCD\index{VCD}\index{vibrational circular dichroism}. Mainly for debugging purposes.

\item{\verb|.NONUC |} Skip the calculation of the pure nuclear
contribution to the Atomic Axial Tensors\index{atomic axial tensor}\index{AAT}. This will give wrong results
for VCD. Mainly for debugging purposes.

\item{\verb|.NOSEC |} Skip the calculation of second order orbital
contributions to the Atomic Axial Tensors\index{atomic axial tensor}\index{AAT}. This will give wrong
results for VCD\index{VCD}\index{vibrational circular dichroism}. Mainly for debugging purposes.

\item{\verb|.PRINT |}\verb| |\newline
\verb|READ (LUCMD,*) IPRINT|

Set print level in the calculation of Atomic Axial Tensors\index{atomic axial tensor}\index{AAT} (this does
not include the print level in the integral calculation, which are
controled by the keyword \verb|.INTPRI|). Read one
more line containing print level. Default value is the value of
\verb|IPRDEF| from the general input module.

\item{\verb|.SKIP  |} Skips the calculation of Atomic Axial Tensors\index{atomic axial tensor}\index{AAT}.
This will give wrong results for VCD\index{VCD}\index{vibrational circular dichroism}, but may be of interest for
debugging purposes.

\item{\verb|.STOP  |} Stops the entire calculation after finishing the 
calculation of the Atomic Axial Tensors\index{atomic axial tensor}\index{AAT}. Mainly for debugging purposes.
\end{description}

\subsection{Linear response calculation: {\tt
*ABALNR}}\label{sec:abalnr}

Directives to control the calculation of frequency dependent linear
response\index{frequency}\index{linear response} functions. At present
these directives only affect the 
calculation of frequency dependent linear response functions appearing
in connection with Vibrational Raman Optical Activity
(ROA)\index{Raman optical activity}\index{ROA}.

\begin{description}
\item{\verb|.ALFA  |} Calculates the frequency dependent
polarizability.\index{polarizability} 

\item{\verb|.FREQUE|}\verb| |\newline
\verb|READ (LUCMD,*) NFRVAL|\newline
\verb|READ (LUCMD,*) (FRVAL(I), I = 1, NFRVAL)|

Set the number of frequencies as well as the frequency\index{frequency} at which the
frequency dependent linear response equations are to be evaluated.
Read one more line containing the number of frequencies to be
calculated, and another line reading these frequencies. The
frequencies are to be entered in atomic units. By default only the
static case is evaluated.

\item{\verb|.IPRINT|}\verb| |\newline
\verb|READ (LUCMD,*) IPRLNR|

Set the print level in the calculation of frequency dependent linear
response properties. Read one more line containing the print level.
The default value is the value of \verb|IPRDEF| from the general input
module. 

\item{\verb|.MAXITE|}\verb| |\newline
\verb|READ (LUCMD,*) MAXITE|

Set the maximum number of micro iterations in the iterative solution of
the frequency dependent linear response functions. Read one more line
containing maximum number of micro iterations. Default value is
40.

\item{\verb|.MAXPHP|}\verb| |\newline
\verb|READ (LUCMD,*) MXPHP|

Set the maximum dimension for the sub-block of the configuration
Hessian that will be explicitly inverted. Read one more line
containing maximum dimension. Default value is~0.

\item{\verb|.MAXRED|}\verb| |\newline
\verb|READ (LUCMD,*) MXRM|

Set the maximum dimension of the reduced space to which new basis
vectors are added as described in Ref.~\cite{tuhjahjajpjjcp84}. Read
one more line containing maximum dimension. Default value is~400.

\item{\verb|.OPTORB|} Use optimal orbital trial vectors\index{optimal
orbital trial vector} in the 
iterative solution of the frequency dependent linear
response\index{linear response}
equations. These are generate as described in
Ref.~\cite{tuhjahjajpjjcp84} by solving the orbital response equation
exact, keeping the configuration part fixed.

\item{\verb|.SKIP  |} Skip the calculation of the frequency dependent
response functions. This will give wrong results for ROA. Mainly for
debugging purposes.

\item{\verb|.STOP  |} Stops the program after finishing the
calculation of the frequency dependent linear response equations. Mainly
for debugging purposes. 

\item{\verb|.THRLNR|}\verb| |\newline
\verb|READ (LUCMD,*) THCLNR|

Set the convergence threshold for the solution
of the frequency dependent response equations. Read one more line
containing the convergence threshold~(D12.6). The default value is
$5.0\cdot10^{-5}$.
\end{description}

\subsection{Dipole moment and dipole gradient contributions: {\tt
*DIPCTL}}\label{sec:dipctl} 

Directives controlling the calculation of contributions to the
dipole gradient\index{dipole gradient} appear in the \verb|*DIPCTL| section.

\begin{description}
\item{\verb|.AOMAT |}  Print in AO as well as SO basis dipole
matrices if printing of the latter has been requested. Obsolete keyword.
Do not use.

\item[\Key{NODC}] Neglect contributions to traces from
inactive one-electron density matrix. This will give wrong results for
the dipole gradient\index{dipole gradient}. Mainly for debugging purposes.

\item[\Key{NODV}] Neglect contributions to traces from
active one-electron density matrix. This will give wrong results for the
dipole gradient\index{dipole gradient}. Mainly for debugging purposes.

\item[\Key{PRINT}]\verb| |\newline
\verb|READ (LUCMD,*) IPRINT|

Set print level in the calculation of dipole gradient\index{dipole gradient}.  Read one more
line containing print level. The default 
is the variable \verb|IPRDEF| from the general input module.

\item[\Key{SKIP}] Skip the calculation of dipole gradient\index{dipole gradient}.

%\item[\Key{TEST}] Test dipole moments and dipole
%reorthonormalization(?) through a dummy routine. This test routine is
%currently only implemented for the symmetric connection, and must thus
%only be used together with the \verb|.NODIFC| keyword.

\item[\Key{STOP}] Stop the program after finishing the calculation
of the dipole gradient\index{dipole gradient}. Mainly for debugging purposes.
\end{description}

\subsection{End of input: \Sec{END OF}}

The last directive in the input should be \verb|*END OF|.

\subsection{Calculation of excitation energies: {\tt
*EXCITA}}\label{sec:excita}

Directives to control the calculations of electronic
transition\index{electronic excitation}
properties and excitation energies\index{excitation energy} appear in
the \verb|*EXCITA| input 
module. For SCF\index{SCF}\index{HF}\index{Hartree-Fock} wave
functions the properties are calculated using the 
random phase approximation (RPA) and for MCSCF\index{MCSCF} wave functions the
multiconfigurational (MC)-RPA is used. Implemented electronic transition
properties are at the moment:

\begin{enumerate}
\item Oscillator Strength\index{oscillator strength} which determines
visible and UV absorption. 
\item Rotatory Strength\index{rotatory strength} which determines
Electronic Circular Dichroism\index{electronic circular dichroism}\index{ECD}
(ECD).
\item Excitation Energies\index{electronic
excitation}\index{excitation energy}. These are always calculated when
invoking the \verb|.EXCITA| keyword in the general input module.
\end{enumerate}

\begin{description}
\item{\verb|.DIPSTR|} Calculates the dipole strength\index{dipole
strength}, that is, the 
dipole oscillator strength which determine the visible and UV
absorption, using the dipole length form.

\item{\verb|.FNAC  |} Calculate first-order non-adiabatic coupling
matrix\index{non-adiabatic coupling element} elements. This is not yet
fully implemented in the \aba\ 
program package and must be combined with subsequent
RESPONSE-calculations.

\item{\verb|.IPRINT|}\verb| |\newline
\verb|READ (LUCMD, *) IPRINT|

Set the print level in the calculation of the necessary differentiated
integrals when calculating the linear response functions. Read one
more line containing print level. Default value is the value of 
\verb|IPRDEF| from the general input module. The print level of the
rest of the calculation of electronic excitation energies are
controled by the keyword \verb|.PRINT |.

\item{\verb|.MAXITE|}\verb| |\newline
\verb|READ (LUCMD,*) MAXITE|

Set the maximum number of micro iterations in the iterative
solution of the linear response equations. Read
one more line containing maximum number of micro iterations.
Default value is 40.

\item{\verb|.MAXPHP|}\verb| |\newline
\verb|READ (LUCMD,*) MXPHP|

Set the maximum dimension for the sub-block of the configuration
Hessian that will be explicitly inverted. Read one more line
containing maximum dimension. Default value is~0.

\item{\verb|.MAXRED|}\verb| |\newline
\verb|READ (LUCMD,*) MXRM|

Set the maximum dimension of the reduced space to which new basis
vectors are added as described in Ref.~\cite{tuhjahjajpjjcp84}. Read
one more line containing maximum dimension. Default value is~400.

\item{\verb|.NEXCIT|}\verb| |\newline
\verb|READ (LUCMD, '(8I5)') (NEXCIT(I), I= 1,8)|

Set the number of excitation energies\index{excitation energy} to be
calculated in each 
symmetry. Read one more line containing the number of excitations in
each of the irreducible representations of the molecular point group.
The default is not to calculate any excitation energies in any of the
irreducible representations.

\item{\verb|.OPTORB|} Use optimal orbital trial vectors\index{optimal
orbital trial vector} in the
iterative solution of the frequency dependent linear
response\index{linear response}
equations. These are generate by solving the orbital response equation
exact, keeping the configuration part fixed as described in
Ref.~\cite{tuhjahjajpjjcp84}. 

\item{\verb|.PRINT |}\verb| |\newline
\verb|READ (LUCMD,*) IPREXE|

Set the print level in the calculation of electronic excitation
energies. Read one more line containing the print level.
The default value is the \verb|IPRDEF| from the general input module.

\item{\verb|.ROTVEL|} Calculate rotational strengths\index{electronic
circular dichroism}\index{ECD} in Electronic
Circular Dichroism (ECD) without using London orbitals.

\item{\verb|.SKIP  |} Skip the calculation of electronic excitation
energies. This will give wrong results for ECD and first-order NACMEs.
Mainly for debugging purposes.

\item{\verb|.STOP  |} Stops the program after finishing the 
calculation of the linear response functions. Mainly
for debugging purposes. 

\item{\verb|.THREXC|}\verb| |\newline
\verb|READ (LUCMD,*) THREXC|

Set the convergence threshold for the solution
of the linear response equations. Read one more line
containing the convergence threshold. The default value is
$1\cdot10^{-4}$.

\item[\Key{TRIPLE}]
Indicates that it is triplet excitations\index{excitation energy,
triplet} that is to be investigated. 
\end{description}

\subsection{One-electron expectation values: {\tt
*EXPECT}}\label{sec:expect}

Directive that control the calculation of one-electron expectation
values appear in the \verb|*EXPECT| input module. Notice, however,
that the directives controling the calculation of one-electron
expectation values to the geometric Hessian appear in the
\verb|*ONEINT| section.

\begin{description}
\item{\verb|.ALL CO|} Indicates that all components of the nuclear
shieldings\index{nuclear shielding} tensor is to be calculated at the
same time. This is the 
default for ordinary calculations. However, in direct and parallel
calculations on large molecules this may give to large memory
requirements, and instead the components of one symmetry independent
nucleus are calculated simultaneously instead. However, by invoking
this keyword, all components are calculated simultaneously even in
direct/parallel calculations.

\item{\verb|.DIASUS|} Invokes the calculation of the one-electron
contribution to the magnetizability\index{magnetizability} expectation
value. By default this 
is done using London atomic\index{London orbitals} orbitals. Default
value is \verb|TRUE| if 
magnetizability has been requested in the general input module,
otherwise \verb|FALSE|.

\item{\verb|.NODC  |} Do not calculate contributions from inactive
one-electron density matrix. This will give wrong results for the
one-electron expectation values. Mainly for debugging purposes.

\item{\verb|.NODV  |} Do not calculate contributions from active
one-electron density matrix. This will give wrong results for the
one-electron expectation values. Mainly for debugging purposes.

\item{\verb|.ELFGRA|} Invokes the calculation of the electronic
contribution to the nuclear quadrupole moment coupling
tensor\index{nuclear quadrupole coupling}\index{NQCC} (that
is, the electric field gradient\index{electric field
gradient}). Default value is \verb|TRUE| if 
nuclear quadrupole coupling constants have been requested in the
general input module, otherwise \verb|FALSE|.

\item{\verb|.NEFIEL|} Invokes the evaluation of the electric
field the individual nuclei\index{electric field at nucleus}. Default
value is \verb|TRUE| if 
spin-rotation\index{spin-rotation constant} constants have been
requested in the general input 
module, otherwise \verb|FALSE|. In the current implementation,
symmetry dependent nuclei cannot be used when calculating this property.

\item{\verb|.POINTS|}\verb| |\newline
\verb|READ (LUCMD,*)|

Set the number of integration points to be used in the Gaussian
quadrature\index{Gaussian quadrature}
when evaluating  the diamagnetic spin-orbit\index{diamagnetic
spin-orbit} integrals. Default 
value is 40.

\item{\verb|.PRINT |}\verb| |\newline
\verb|READ (LUCMD,*) MPRINT|

Set print level in the calculation of one-electron expectation values.
Read one more line containing print level. Default value is the
value of \verb|IPRDEF| from the general input module.

\item{\verb|.QUADRU|} Calculates the electronic contribution to the
molecular (traceless) quadrupole moment\index{quadrupole
moment}. Default value is \verb|TRUE| 
if molecular quadrupole moment have been requested in the general input
module, otherwise \verb|FALSE|.

\item{\verb|.SHIELD|} Invokes the calculation of the one-electron
contribution to the nuclear shielding\index{nuclear shielding}
expectation values. By default 
this is done using London atomic\index{London orbitals}
orbitals. Default value is 
\verb|TRUE| if nuclear shieldings have been requested in the general
input module, otherwise \verb|FALSE|.

\item{\verb|.SKIP  |} Skip the calculation of one-electron expectation
values. This may give wrong final results for some properties. Mainly
for debugging purposes.

\item{\verb|.SPIN-S|} Invokes the calculation of the diamagnetic
spin-orbital\index{diamagnetic spin-orbit} integral, which is the
diamagnetic contribution to 
indirect nuclear spin-spin coupling\index{spin-spin coupling}
constants. Default value is 
\verb|TRUE| if spin-spin couplings have been requested in the general
input module, otherwise false.

\item{\verb|.STOP  |} Stop the entire calculation after finishing the
calculation of one-electron expectation values. Mainly for debugging
purposes.

\end{description}

%\subsection{Floating orbitals: \Sec{FLOAT}}\label{sec:float}
%
%Directives that control the calculation when using floating orbitals
%as described by Helgaker and Alml{\"o}f \cite{thjajcp89} appear in
%the \verb|*FLOAT| input module.
%
%\begin{description}
%\item{\verb|.PRINT |}\verb| |\newline
%\verb|READ (LUCMD,'(I5)') IPRINT|
%
%Set the print level in the calculation of floating orbital
%contributions. Read one more line containing the print level~(I5).
%Default print level is the \verb|IPRDEF| variable from the general
%input module.
%
%\item{\verb|.RESPON|} Print the response contributions from
%the floating orbitals.
%
%\item{\verb|.SKIP  |} Skip the calculation of specific terms
%contributing when using floating orbitals. This may give wrong results
%when using floating orbitals. Mainly for debugging purposes.
%
%\item{\verb|.STOP  |} Stop the calculation after calculating the
%response contributions from the floating orbitals. Mainly for
%debugging purposes.
%\end{description}

\subsection{Geometry analysis: \Sec{GEOANA}}

Directives controlling the calculation and printing of bond
angles\index{bond distance}\index{bond angle}\index{dihedral angle}
and 
dihedral angles appear in the \verb|*GEOANA| section. By default all
bond distances will be calculated. The program will also define atoms
to be bonded to each other depending on their bond distance. For all atoms
defined to be bonded to each other, the bond distance and bond angles
will be printed.

\begin{description}
\item[\Key{ANGLES}]\verb| |\newline
\verb|READ (LUCMD,*) NANG|\newline
\verb|    DO 310 I = 1, NANG|\newline
\verb|       READ (LUCMD,*) (IANG(J,I), J = 1,3)|\newline
\verb|310 CONTINUE|

Calculate and print bond angles\index{bond angle}.  Read one
more line specifying the number of angles, and then read \verb|NANG|
lines containing triplets $A,B,C$ of atom labels, each
specifying a particular bond angle~$\angle ABC$. Notice that in the
current version of the program there is an upper limit of 20 bond
angles that will be printed. The rest will be ignored. We also note
that program always will print the angles between atoms defined to be
bonded to each other on the basis of the van der Waals\index{van der
Waals radius} radii of the atoms.

\item[\Key{DIHEDR}]\verb| |\newline
\verb|READ (LUCMD,*) NDIHED|\newline
\verb|    DO 410 I = 1, NDIHED|\newline
\verb|       READ (LUCMD,*) (IDIHED(J,I), J = 1,4)|\newline
\verb|410 CONTINUE|

Calculate and print dihedral\index{dihedral angle} (torsional)
angles.  Read one more line specifying the number of angles,
and then read \verb|NDIHED| lines containing quadruplets $A,B,C,D$ of atom
labels.  The angle computed is that between the planes~$ABC$
and $BCD$. Notice that in the current version of the program there is
an upper limit of 20 dihedral angles that will be printed. The rest
will be ignored.

\item[\Key{SKIP}] Skip the geometry analysis, with the exceptions
mentioned in the introduction to this section. This is the default
value, but it is overwritten by the keywords \verb|.ANGLES| and
\verb|.DIHEDR|. 
\end{description}

\subsection{Right-hand sides for response equations: \Sec{GETSGY}}

Directives affecting the construction of the right-hand
sides~(RHS)\index{property gradient}\index{right-hand side} --- that is,
gradient terms --- for the response 
calculation as well as some matrices needed for reorthonormalization
appear in the \verb|*GETSGY| section. 

\begin{description}
\item[\Key{ALLCOM}] Requests that all paramagnetic
spin-orbit\index{paramagnetic spin-orbit}
right-hand sides are to be calculated in one batch, and not for each
symmetry-independent center at a time which is the default. This will
slightly speed up the calculation, at the cost of significantly larger
memory requirements.

\item[\Key{AOMAT}] Print AO as well as SO basis derivative
overlap matrices if printing the latter has been requested. Obsolete
keyword. Do not use.

\item[\Key{DIAGTD}] Set derivative overlap matrices to diagonal
form for test purposes. Obsolete keyword. Do not use.

\item[\Key{FCKPRI}]\verb| |\newline
\verb|READ (LUCMD,*) IPRFCK|

Set print level for the calculation of derivative Fock matrices.  Read
one more line specifying print level. The default  is the value of
\verb|IPRDEF| in the general input module.

\item[\Key{FCKSKI}] Skip the derivative Fock matrix contributions
to the right-hand sides. This will give wrong results for all
properties depending on right hand sides. Mainly for debugging purposes.

\item[\Key{FSTTES}] Test one-index transformation of derivative
Fock matrices. 

\item[\Key{GDHAM}] Write out differentiated Hamiltonian and
differentiated Fock matrices to file for use in post-\siraba\ programs.

\item[\Key{GDYPRI}]\verb| |\newline
\verb|READ (LUCMD,*) IPRGDY|

Set print level for the calculation of the Y-matrix appearing in the
reorthonormalization terms, as for instance in
Ref.~\cite{tuhjahjajpjjcp84}. Default  is the value of \verb|IPRALL|
defined by the \verb|.PRINT | keyword. If 
this has not been specified, the default is the value of \verb|IPRDEF|
from the general input section.

\item[\Key{GDYSKI}] Skip the calculation of the lowest order
reorthonormalization contributions to the second-order molecular
properties. This will give wrong results for these properties. Mainly
for debugging purposes.

\item[\Key{INTPRI}]\verb| |\newline
\verb|READ (LUCMD, '(5I5)') IPRINT, IPRNTA, IPRNTB, IPRNTC, IPRNTD|

Set print level for the derivative integral calculation for a particular shell
quadruplet.  Read one more line containing print level and the four
shell indices~(5I5).  The print level is changed from the default
for this quadruplet only. Default value is the value of \verb|IPRDEF|
from the general input module. Note that the print level of all shell
quadruplets can be changed by the keyword \verb|.PRINT |.

\item[\Key{INTSKI}] Skip the calculation of derivative integrals.
This will give wrong results for the total molecular Hessian. Mainly
for debugging purposes.

\item[\Key{MAXSIM}]\verb| |\newline
\verb|READ (LUCMD,) MAXSIM|

Maximum number of RHS to generate
simultaneously.  Read one more line specifying this value.
The default is~3.  This value should not be confused with the same
keyword entered in the different sections for the solution of response
equations like  the \verb|*RESPON|, \verb|*LINRES|, \verb|*TRPDRV|,
\verb|*ABALNR|, and \verb|*EXCITA| sections. Obsolete keyword. Do not use.
\typeout{>>>>>>>>Har alle responsmodulseksjoner et MAXSIM keyword?
Foroevrig virker det paa meg som om keyword'et ikke virker.}

\item[\Key{NODC}] Do not calculate contributions from inactive
one-electron density matrix. This will give wrong results for the
total molecular property. Mainly for debugging purposes.

\item[\Key{NODDY}] Test the orbital part of the right-hand side.
The run will not be aborted. Mainly for debugging purposes.

\item[\Key{NODPTR}] The transformation of the two-electron density
matrix is back-transformed to atomic orbital basis using a
noddy-routine for comparison.

\item[\Key{NODV}] Do not calculate contributions from the active
one-electron density matrix. This will give wrong results for the
molecular property. Mainly for debugging purposes.

\item[\Key{NOFD}] Do not calculate the contribution from the
differentiated Fock-matrices to the total right-hand side. This will
give wrong results for the requested molecular property. Mainly for
debugging purposes.

\item[\Key{NOFS}] Do not calculate the contribution to the total
right-hand side from the one-index transformed Fock-matrices with the
differentiated connection matrix. This will give wrong results for
the requested molecular property. Mainly for debugging purposes.

\item[\Key{NOH1}] Do not calculate the contribution from the
one-electron terms to the total right-hand side. This will give wrong
results for the requested property. Mainly for debugging purposes.

\item[\Key{NOH2}] Do not calculate the contribution from the
two-electron terms to the total right-hand side. This will give wrong
results for the requested molecular property. Mainly for debugging
purposes.

\item[\Key{NOORTH}] Do not calculate the orbital orthonormalization
contribution (the one-index transformed contributions) to the total
right-hand side. This will give wrong results for the requested
molecular property. Mainly for debugging purposes.

\item[\Key{NOPV}] Do not calculate contributions from two-electron
density matrix. This will give wrong results for the requested
molecular property. Mainly for debugging purposes.

\item[\Key{NOSSF}] Do not calculate the contribution to the total
right-hand side from the double-one-index
transformation between the differentiated connection matrix and the
Fock-matrix. This option will only affect the calculation of molecular
Hessian, and will give a wrong result for this. Mainly for debugging
purposes. 

\item[\Key{PRINT}]\verb| |\newline
\verb|READ (LUCMD,) IPRALL|

Set print levels.  Read one more line containing the print level for
this part of the calculation.  This will be the default print
level in the calculation of differentiated two-electron integrals,
differentiated Fock-matrices, derivative
overlap matrices, two-electron density and derivative integral
transformation, as well as in the construction of the right-hand sides.
To set the print level in each of these parts individually, see the
keywords \verb|.FCKPRI|, \verb|.GDYPRI|, \verb|.INTPRI|,
\verb|.PTRPRI|, \verb|.SORPRI|, and \verb|.TRAPRI|.

\item[\Key{PTRPRI}]\verb| |\newline
\verb|READ (LUCMD,) IPRTRA|

Set print level for the  two-electron densities transformation. Read
one more line containing print level. 
Default value is the value of  \verb|IPRDEF| from the general input
module. Note also that this print level is also controled by the keyword
\verb|.PRINT |. 

\item[\Key{PTRSKI}] Skip transformation of active two-electron
density matrix. This will give wrong results for the total
second-order molecular property. Mainly for debugging purposes.

\item[\Key{RETURN}] Stop after the shell quadruplet specified
under \verb|.INTPRI| above. Mainly for debugging purposes.

\item[\Key{SDRPRI}]\verb| |\newline
\verb|READ (LUCMD,) IPRSDR|

Set the print level in the calculation of the differentiated connection
matrix. Read one more line containing the print level. Default
value is the value given by the keyword \verb|.PRINT |. If this
keyword has not been given, the default is the value of \verb|IPRDEF|
given in the general input module.

\item[\Key{SDRSKI}] Do not calculate the differentiated connection
matrices. This will give wrong results for properties calculated with
perturbation dependent basis sets. Mainly for debugging purposes.

\item[\Key{SDRTES}] The differentiated connection matrices will be
transformed and printed in atomic orbital basis. Mainly for debugging
purposes.

\item[\Key{SIRPR4}]\verb| |\newline
\verb|READ (LUCMD, *) IPRI4|

\sir\  ``output unit~4'' print level.  Read one more line specifying
print level. Default is~0.

\item[\Key{SIRPR6}]\verb| |\newline
\verb|READ (LUCMD, *) IPRI6|

\sir\ ``output unit~6'' print level.  Read one more line specifying
print level. Default is~0.

\item[\Key{SKIP}] Skip the calculation of right-hand sides. This
will give wrong values for the requested second-order properties.
Mainly for debugging purposes.

\item[\Key{SORPRI}]\verb| |\newline
\verb|READ (LUCMD,*) IPRSOR|

Set print level for the two-electron density matrix sorting. Read one
more line containing print level. Default value is the value of
\verb|IPRDEF| from the general input module. Note also that this print
level is also controled by the keyword \verb|.PRINT |.

\item[\Key{STOP}] Stop the the entire calculation after finishing
the construction of the right-hand side. Mainly for debugging purposes.

\item[\Key{TIME}] Provide detailed timing breakdown for the
two-electron integral calculation.

\item[\Key{TRAPRI}]\verb| |\newline
\verb|READ (LUCMD,*) IPRTRA|

Set print level for the derivative integrals transformation.  Read one more
line specifying print level. Default is the value of
\verb|IPRDEF| from the general input module. Notice that the default print
level is also affect by the keyword \verb|.PRINT |.

\item[\Key{TRASKI}] Skip transformation of derivative integrals.
Mainly for debugging purposes.

\item[\Key{TRATES}] Testing of derivative integral
transformation. The calculation will not be aborted. Mainly for
debugging purposes.

\item[\Key{TWOH1}] Obsolete keyword. Do not use.

\item[\Key{UNDIF}] Calculate undifferentiated integrals. Obsolete
keyword. Do not use.
\end{description}

\subsection{Linear response calculation: {\tt
*LINRES}}\label{sec:linres}

Directives to control the calculation of frequency independent linear
response functions\index{linear response}. At present these directives
only affect the 
calculation of frequency independent linear response functions appearing
in connection with singlet, magnetic perturbations. This
whole section will become obsolete and be replaced by \verb|*EXCITA|
and \verb|*ABALNR|, two sections which furthermore may become one.

\begin{description}
\item[\Key{EXCITA}] Obsolete keyword. Do not use. We refer instead
to the keyword \verb|.EXCITA| in the \verb|*EXCITA| input module.

\item[\Key{FREQUE}] Obsolete keyword. Do not use. We refer instead
to the keyword \verb|.FREQUE| in the \verb|*EXCITA| input module.

\item[\Key{MAXITE}]\verb| |\newline
\verb|READ (LUCMD,*) MAXITE|

Set the maximum number of micro iterations in the iterative solution of
the frequency independent linear response functions. Read one more line
containing maximum number of micro iterations. Default value is
40.

\item[\Key{MAXPHP}]\verb| |\newline
\verb|READ (LUCMD,*) MXPHP|

Set the maximum dimension of the sub-block of the configuration
Hessian that will be explicitly inverted. Read one more line
containing maximum dimension. Default value is~0.

\item[\Key{MAXRED}]\verb| |\newline
\verb|READ (LUCMD,*) MXRM|

Set the maximum dimension of the reduced space to which new basis
vectors are added as described in Ref.~\cite{tuhjahjajpjjcp84}. Read
one more line containing maximum dimension. Default value is~400.

\item[\Key{NEXVAL}] Obsolete keyword. Do not use. We refer instead to
the keyword \verb|.EXCITA| in the \verb|*EXCITA| input module.

\item[\Key{OPTORB}] Use optimal orbital trial vectors in
the\index{optimal orbital trial vector}
iterative solution of the frequency independent linear response
equations. These are generate by solving the orbital response equation
exact, keeping the configuration part fixed as described in
Ref.~\cite{tuhjahjajpjjcp84}. 

\item[\Key{PRINT}]\verb| |\newline
\verb|READ (LUCMD,*) IPRCLC|

Set the print level in the solution of the magnetic
frequency-independent linear response equations. Read one more line
containing print level. Default is the value of \verb|IPRDEF| in
the general input module.

\item[\Key{SKIP}] Skip the calculation of the frequency independent
response functions. This will give wrong results for shielding,
magnetizabilities, and spin-spin coupling constants\index{nuclear
shielding}\index{spin-spin coupling}\index{magnetizability}. Mainly for
debugging purposes.

\item[\Key{STOP}] Stops the program after finishing the
calculation of the frequency independent linear response equations. Mainly
for debugging purposes. 

\item[\Key{THRESH}]\verb| |\newline
\verb|READ (LUCMD,*) THRCLC|

Set the convergence threshold for the solution
of the frequency dependent response equations. Read one more line
containing the convergence threshold. The default value is
$1.0\cdot10^{-4}$ for calculations which cannot take advantage of Sellers
formular for quadratic errors in the response
property~\cite{hsijqc30}, and $2.0\cdot10^{-3}$ for those calculations
that can.
\end{description}

\subsection{Nuclear contributions: \Sec{NUCREP}}

Directives affecting the nuclear contribution to the molecular
gradient\index{gradient} and molecular Hessian\index{Hessian}
calculation appear in the 
\verb|*NUCREP| section.
\begin{description}
\item[\Key{PRINT}]\verb| |\newline
\verb|READ (LUCMD,*) IPRINT|

Set the print level in the calculation of the nuclear contributions.
Read one more line containing print level. Default value is the
value of \verb|IPRDEF| from the general input module.

\item[\Key{SKIP}] Skip the calculation of the nuclear
contribution. This will give wrong
results for the total molecular gradient and Hessian. Mainly for
debugging purposes. 

\item[\Key{STOP}] Stop the program after finishing the calculation
of the nuclear contributions. Mainly for debugging purposes.
\end{description}

\subsection{One-electron integrals: \Sec{ONEINT}}

Directives affecting the calculation of one-electron integrals in the
calculation of molecular gradients\index{gradient} and molecular
Hessians\index{Hessian} appear in the 
\verb|*ONEINT| section. 
\begin{description}
\item[\Key{NODC}] Do not calculate contributions from inactive
one-electron density matrix. This will give wrong results for the
total molecular gradient and
Hessian\index{gradient}\index{Hessian}. Mainly for debugging
purposes. 

\item[\Key{NODV}] Do not calculate contributions from active
one-electron density matrix. This will give wrong results for the
total molecular gradient and Hessian\index{gradient}\index{Hessian}. Mainly for debugging purposes.

\item[\Key{PRINT}]\verb| |\newline
\verb|READ (LUCMD,*) IPRINT|

 Set print level in the calculation of one-electron contributions to
the molecular gradient and Hessian\index{gradient}\index{Hessian}.  Read one more line containing
print level. Default value is the value of \verb|IPRDEF| from the
general input module.

\item[\Key{SKIP}] Skip the calculation of one-electron integral
contributions to the molecular gradient and Hessian\index{gradient}\index{Hessian}. This will give
wrong total results for these properties. Mainly for debugging
purposes.

\item[\Key{STOP}] Stop the entire calculation after the
one-electron integral contributions to the molecular gradients and
Hessians has been evaluated\index{gradient}\index{Hessian}. Mainly for debugging purposes.
\end{description}

\subsection{Relaxation contribution to Hessian: \Sec{RELAX}}

Directives controlling the calculation of the relaxation
contributions (i.e., those from the response terms) to the different
second-order molecular properties, appear in the \verb|*RELAX| section.
\begin{description}
\item[\Key{NOSELL}] Do not use Sellers' method \cite{hsijqc30} that
ensures that the error in the relaxation Hessian is quadratic in the
error of the response equation solution, rather than linear. Mainly
for debugging purposes.

\item[\Key{PRINT}]\verb| |\newline
\verb|READ (LUCMD,*) IPRINT|

Set the print level in the calculation of the relaxation
contributions.  Read one more line containing print level.
Default value is the value of \verb|IPRDEF| from the general input
module.

\item[\Key{SKIP}] Skip the calculation of the relaxation
contributions.  This does not skip the solution of the response
equations. This will give wrong results for a large number of
second-order molecular properties. Mainly for debugging purposes.

\item[\Key{STOP}] Stop the entire calculation after the
calculation of the relaxation contributions to the requested
properties. Mainly for debugging purposes.

\item[\Key{SYMTES}] Calculate both the $ij$ and $ji$ elements of
the relaxation Hessian to verify its Hermiticity or anti-Hermiticity
(depending on the property being calculated). Mainly for debugging
purposes.  
\end{description}

\subsection{Reorthonormalization contributions: \Sec{REORT}}

Directives affecting the calculation of reorthonormalization
contributions to the geometric Hessian appear in the \verb|*REORT |
section. 
\begin{description}
\item[\Key{PRINT}]\verb| |\newline
\verb|READ (LUCMD,*) IPRINT|

Set print level in the calculation of the lowest-order
reorthonormalization contributions to the molecular Hessian.  Read one
more line containing print level. Default value is the value of
\verb|IPRDEF| from the general input module.

\item[\Key{SKIP}] Skip the calculation of the reorthonormalization
contributions to the molecular Hessian. This will give wrong results
for this property. Mainly for debugging purposes.

\item[\Key{STOP}] Stop the entire calculation after finishing the
calculation of the reorthonormalization contributions to the molecular
Hessian. Mainly for debugging purposes.
\end{description}

\subsection{Response calculation: \Sec{RESPON}}
\label{sec:abares}

Directives affecting the response (coupled-perturbed MCSCF)
calculation  of geometric perturbations appear in the \verb|*RESPON|
section. 
\begin{description}
\item[\Key{D1DIAG}] Neglect diagonal elements of the orbital
Hessian when generating trial vectors. Mainly for debugging
purposes. 

\item[\Key{DONEXT}] Force the use of optimal orbital
trial\index{optimal orbital trial vector} vectors in 
the solution of the geometric response equations as described in
Ref.~\cite{tuhjahjajpjjcp84}. This is done by solving the orbital part
exact while keeping the configuration part fixed.

\item[\Key{MAXITE}]\verb| |\newline
\verb|READ (LUCMD,*) MAXNR|

Maximum number of iterations to be used when solving the geometric
response equations.  Read one more line specifying value.
Default value is~50.

\item[\Key{MAXRED}]\verb| |\newline
\verb|READ (LUCMD,*) MAXRED|

Set the maximum dimension of the reduced space to which new basis
vectors are added as described in Ref.~\cite{tuhjahjajpjjcp84}. Read
one more line containing maximum dimension. Default value is~400.

\item[\Key{MAXSIM}]\verb| |\newline
\verb|READ (LUCMD,*) MAXSIM|

Maximum number of geometric perturbations to solve simultaneously in a
given symmetry.  Read one more line specifying value.  Default
is~15.

\item[\Key{MCHESS}] Explicitly calculate electronic Hessian and
test its symmetry. Does not abort the calculation. Mainly for
debugging purposes.

\item[\Key{NEWRD}] Forces the solution vectors to be written to a
new file. This will also imply that \verb|.NOTRIA| will be set to
\verb|TRUE|, that is, that no previous solution vectors will be used
as trial vectors.

\item[\Key{NOAVER}] Use an approximation to the orbital Hessian
diagonal when generating trial vectors.

\item[\Key{NONEXT}] Do not use optimal orbital trial\index{optimal
orbital trial vector} vectors. 

\item[\Key{NOTRIA}] Do not use old solutions as trial vectors, even
though they may exist.

\item[\Key{NRREST}] Restart geometric response calculation using old
solution vectors.

\item[\Key{PRINT}]\verb| |\newline
\verb|READ (LUCMD,*) IPRINT|

Set the print level during the solution of the geometric response
equations.  Read one more line containing print level. Default
value is the value of \verb|IPRDEF| in the general input module.

\item[\Key{RDVECS}]\verb| |\newline
\verb|READ (LUCMD, *) NRDT|\newline
\verb|READ (LUCMD, *) (NRDCO(I), I = 1, NRDT)|

Solve for specific geometric perturbations only.  Read
one more line specifying number to solve for and then another
line specifying their sequence numbers. This may give wrong results
for some components of the molecular Hessian. Mainly for debugging
purposes. 

\item[\Key{SKIP}] Skip the solution of the geometric response
equations. This will give wrong results for the geometric Hessian.
Mainly for debugging purposes.

\item[\Key{THRESH}]\verb| |\newline
\verb|READ (LUCMD,*) THRNR|

Threshold for convergence of the geometric response
equations.  Read one more line specifying value.  Default
value is~10$^{-3}$.

\item[\Key{STOP}] Stop the entire calculation after solving all
the geometric  response equations. Mainly for debugging purposes.
\end{description}

\subsection{Indirect nuclear spin-spin couplings: {\tt
*SPIN-S}}\label{sec:spin-s}

This input module controls the calculation of which indirect nuclear
spin-spin coupling constants and what contributions to the total
spin-spin coupling constants that are to be calculated.

\begin{description}
\item[\Key{ABUNDA}]\verb| |\newline
\verb|READ (LUCMD,*) ABUND|

Set the natural abundance\index{abundance} threshold for discarding coupling between
certain nuclei. By default all isotopes in the molecule with a natural
abundance above this limit will be included in the list of nuclei for which
spin-spin coupling constants will be calculated. Read one more line
containing the abundance threshold. The default value is~1.0,
which includes both proton and $^{13}$C nuclides.

\item[\Key{NODSO}] Do not calculate diamagnetic
spin-orbit\index{diamagnetic spin-orbit}
contributions to the total indirect spin-spin coupling\index{spin-spin
coupling} constants. This 
will give wrong results for the total spin-spin couplings.

\item[\Key{NOFC}] Do not calculate the Fermi contact\index{Fermi
contact} contribution 
to the total indirect spin-spin coupling\index{spin-spin coupling}
constants. This will give 
wrong results for the total spin-spin couplings.

\item[\Key{NOPSO}] Do not calculate the paramagnetic
spin-orbit\index{paramagnetic spin-orbit}
contribution to the indirect spin-spin coupling\index{spin-spin
coupling} constants. This will 
give wrong results for the total spin-spin couplings.

\item[\Key{NOSD}] Do not calculate the spin-dipole\index{spin-dipole}
contribution to 
the total indirect spin-spin coupling\index{spin-spin coupling}
constants. This will give wrong 
results for the total spin-spin couplings.

\item[\Key{PRINT}]\verb| |\newline
\verb|READ (LUCMD,*) ISPPRI|

Set the print level in the output of the final results from the
spin-spin coupling constants. In order to get all individual tensor
components (in a.u.) a print level of at least~5 is needed. Read one
more line containing the print level. Default value is the value
of \verb|IPRDEF| from the general input module.

\item[\Key{SD+FC}] Do not split the
spin-dipole\index{spin-dipole}\index{Fermi contact} and Fermi
contact contributions in the calculations.

\item[\Key{SDxFC ONLY}]\verb| |\newline

Will only calculate the spin dipole--Fermi\index{spin-dipole}\index{Fermi contact} contact cross term, and the
Fermi contanct--Fermi contact contribution for the triplet
responses. The first of these to terms only contribute to the
anistotropy, and one may in this way obtain the most important triplet
contributions to the isotropic and anisotropic\index{spin-spin
coupling}\index{spin-spin anisotropy} spin-spin couplings by
only solving one instead of ten response equations for each nucleus.

\item[\Key{SELECT}]\verb| |\newline
\verb|READ (LUCMD,*) NPERT|\newline
\verb|READ (LUCMD, *) (IPOINT(I), I = 1, NPERT)|

Select which symmetry-independent nuclei for which there is to be calculated
indirect nuclear spin-spin couplings. This option will override any
selection based on natural abundance (the \verb|.ABUNDA| keyword), and
at least one isotope of the nuclei requested will be evaluated (even
though the most abundant isotope with a non-zero spin has a lower
natural abundance 
than the abundance threshold). Read one more line containing the
number of nuclei selected, and another line their number (sorted after
the input order). By default, all nuclei with an isotope with non-zero spin
and with a natural abundance larger than the threshold will be included in
the list of nuclei for which indirect spin-spin couplings will be
calculated.
\end{description}

\subsection{Translational and rotational invariance: 
\Sec{TROINV}}\label{sec:abatro}

Directives affecting the use of translational and rotational
invariance\index{translational invariance}\index{rotational invariance}
appear in the \verb|*TROINV| section.
\begin{description}
\item[\Key{COMPAR}] Use translational and rotational\index{translational invariance}\index{rotational invariance} symmetry
 and check the molecular Hessian against the Hessian obtained
without use of translational and rotational invariance. This is
default in a calculation of vibrational circular dichroism
(VCD)\index{vibrational circular dichroism}\index{VCD}.

\item[\Key{GD}] Obsolete keyword. Do not use.

\item[\Key{NOROT}] Obsolete keyword. Do not use.

\item[\Key{PRINT}]\verb| |\newline
\verb|READ (LUCMD,*) IPRINT|

Set print level for the setting up and use of translational and
rotational invariance.  Read one more line containing print
level. Default value is the value of \verb|IPRDEF| from the
general input module.

\item[\Key{RD}] Obsolete keyword. Do not use.

\item[\Key{SKIP}] Skip the setting up and use of translational
and rotational invariance\index{translational invariance}\index{rotational invariance}.

\item[\Key{STOP}] Stop the entire calculation after the setup of
translational and rotational invariance\index{translational invariance}\index{rotational invariance}. Mainly for debugging purposes.

\item[\Key{THRESH}]\verb| |\newline
\verb|READ (LUCMD,*) THRESH|

Threshold defining linear dependence among
supposedly independent coordinates.  Read one more line specifying
value.  Default is~0.1.
\end{description}

\subsection{Response equations for triplet operators: {\tt
*TRPRSP}}\label{sec:trprsp}

Directives controlling the set-up of right-hand sides for triplet
perturbing operators (for instance the Fermi contact and spin-dipole
operators entering the nuclear spin-spin coupling
constants)\index{spin-dipole}\index{Fermi contact}\index{spin-spin
coupling}, as well 
as when solving the triplet response equations appear in the
\verb|*TRPRSP| input module. 

\begin{description}
\item[\Key{INTPRI}]\verb| |\newline
\verb|READ (LUCMD ,*) INTPRI|

Set the print level in the calculation of the atomic integrals
contributing to the different triplet operator right-hand sides. Read
one more line containing the print level. Default is the value
of \verb|IPRDEF| from the general input module.

\item[\Key{MAXITE}]\verb| |\newline
\verb|READ (LUCMD,*) MAXTRP|

Set the maximum number of micro iterations in the iterative solution of
the triplet response equations. Read one more line containing the
maximum number of iterations. Default is~40.

\item[\Key{MAXPHP}]\verb| |\newline
\verb|READ (LUCMD,*) MXPHP|

Set the maximum dimension for the sub-block of the configuration
Hessian that will be explicitly inverted. Read one more line
containing maximum dimension. Default value is~0.

\item[\Key{MAXRED}]\verb| |\newline
\verb|READ (LUCMD,*) MXRM|

Set the maximum dimension of the reduced space to which new basis
vectors are added as described in Ref.~\cite{tuhjahjajpjjcp84}. Read
one more line containing maximum dimension. Default value is~400.

\item[\Key{NORHS}] Skip the construction of the right-hand sides
for triplet perturbations. As this by necessity implies that all
right-hand sides and solution vectors are zero, this option is
equivalent to \verb|.SKIP  |. This will furthermore give wrong results
for the total spin-spin\index{spin-spin coupling} couplings. Mainly for debugging purposes.

\item[\Key{NORSP}] Skip the solution of the triplet response
equations. This will give wrong results for the total spin-spin
couplings\index{spin-spin coupling}. Mainly for debugging purposes.

\item[\Key{OPTORB}] Optimal orbital trial\index{optimal orbital trial
vector} vectors used in the 
solution of the triplet response equations. These are generate by
solving the orbital response equation 
exact, keeping the configuration part fixed as described in
Ref.~\cite{tuhjahjajpjjcp84}. 

\item[\Key{PRINT}]\verb| |\newline
\verb|READ (LUCMD,*) IPRTRP|

Set the print level during the setting up of triplet operator
right-hand sides and in the solution of the response equations for
the triplet perturbation operators. Read one more line containing the
print level. Default is the value of \verb|IPRDEF| from the
general input module.

\item[\Key{SKIP}] Skip the construction of triplet right-hand
sides as well as the solution of the response equations for
the triplet perturbation operators. This will give wrong results for
the indirect nuclear spin-spin\index{spin-spin coupling}
couplings. Mainly for debugging 
purposes.

\item[\Key{STOP}] Stop the entire calculation after generating the
triplet right-hand sides, and solution of the triplet response
equations. Mainly for debugging purposes.

\item[\Key{THRESH}]\verb| |\newline
\verb|READ (LUCMD,*) THRTRP|

Set the threshold for convergence in the solution of the triplet
response equations. Read one more line containing the
threshold. Default is~$1\cdot10^{-4}$.
\end{description}

\subsection{Two-electron contributions: \Sec{TWOEXP}}

Directives affecting the calculation of two-electron derivative
integral contributions to the molecular gradient\index{gradient} and
Hessian\index{Hessian} appear in 
the \verb|*TWOEXP| section.  

\begin{description}
\item[\Key{DIRTST}] Test the direct calculation of Fock matrices and
integral distributions. Mainly for debugging purposes.

\item[\Key{FIRST}] Compute first derivative integrals but not
second derivatives. This is default if only molecular gradients and
not the molecular Hessian has been requested.

\item[\Key{INTPRI}]\verb| |\newline
\verb|READ (LUCMD, '(5I5)') IPRINT, IPRNTA, IPRNTB, IPRNTC, IPRNTD|

Set print level for the derivative integral calculation for a particular shell
quadruplet.  Read one more line containing print level and the four
shell indices~(5I5).  The print level is changed from the default
for this quadruplet only. Default value is the value of \verb|IPRDEF|
from the general input module. Note that the print level of all shell
quadruplets can be changed by the keyword \verb|.PRINT |.

\item[\Key{INTSKI}] Skip the calculation of derivative integrals.
This will give wrong results for the total molecular gradients and
Hessians. Mainly for debugging purposes.

\item[\Key{NOCONT}] Do not contract derivative integrals
(program back-transforms density matrices to the primitive Gaussian
basis instead).

\item[\Key{NODC}] Do not calculate contributions from inactive
one-electron density matrix. This will give wrong results for the
total molecular gradient and Hessian. Mainly for debugging purposes.

\item[\Key{NODV}] Do not calculate contributions from the active
one-electron density matrix. This will give wrong results for the
total molecular gradient and Hessian. Mainly for debugging purposes.

\item[\Key{NOPV}] Do not calculate contributions from two-electron
density matrix. This will give wrong results for the total molecular
gradient and Hessian. Mainly for debugging purposes.

\item[\Key{PRINT}]\verb| |\newline
\verb|READ (LUCMD,*) IPRALL|

Set print levels.  Read one more line containing the print level for
this part of the calculation.  The will be the default print
level in the two-electron density matrix transformation, the 
symmetry orbital two-electron density matrix sorting, as well as the
print level in the integral derivative evaluation. To set the print
level in each of these parts individually, see the keywords
\verb|.INTPRI|, \verb|.PTRPRI|, \verb|.SORPRI|.

\item[\Key{PTRNOD}] The transformation of the two-electron density
matrix is back-transformed to atomic orbital basis using a
noddy-routine for comparison.

\item[\Key{PTRPRI}]\verb| |\newline
\verb|READ (LUCMD,*) IPRPRT|

Set print level for the the two-electron density matrix transformation.
Read one more line containing print level. Default value is the
value of  \verb|IPRDEF| from the general input module. Note also that
this print level is controled by the keyword \verb|.PRINT |.

\item[\Key{PTRSKI}] Skip transformation of active two-electron
density matrix. This will give wrong results for the total molecular
Hessian. Mainly for debugging purposes.

\item[\Key{RETURN}] Stop after the shell quadruplet specified
under \verb|.INTPRI| above. Mainly for debugging purposes.

\item[\Key{SORPRI}]\verb| |\newline
\verb|READ (LUCMD,*) IPRSOR|

Set print level for the two-electron density matrix sorting. Read one
more line containing print level. Default value is the value of
\verb|IPRDEF| from the general input module. Note also that this print
level is controled by the keyword \verb|.PRINT |.

\item[\Key{SORSKI}] Skip sorting of symmetry orbital two-electron
density matrix. This will give wrong results for the total molecular
Hessian. Mainly for debugging purposes.

\item[\Key{SECOND}] Compute both first and second derivative
integrals. This is default when calculating molecular Hessians.

\item[\Key{SKIP}] Skip all two-electron derivative integral
and two-electron density matrix processing.

\item[\Key{STOP}] Stop the the entire calculation after finishing
the calculation of the two-electron derivative integrals. Mainly for
debugging purposes.

\item[\Key{TIME}] Provide detailed timing breakdown for the
two-electron integral calculation.
\end{description}

\subsection{Vibrational analysis: \Sec{VIBANA}}
\label{sec:abavib}

Directives controlling the calculation of harmonic
vibrational\index{vibrational analysis}
frequencies appear in the \verb|*VIBANA| section, as well as properties
depending on a normal coordinate analysis or vibrational frequencies.
Such properties include in the present version of the program:
Vibrational Circular Dichroism (VCD), Raman intensities, and Raman
Optical Activity (ROA)\index{vibrational circular
dichroism}\index{VCD}\index{Raman intensity}\index{IR
intensity}\index{Raman optical activity}\index{ROA}.

\begin{description}
%\item[\Key{INTERN}] Use internal coordinates for the vibrational
%analysis.  This has no effect on vibrational analyses performed at
%stationary points, but is asserted to provide more reliable
%results at non-stationary points.  Read more lines (see below) to
%specify the internal coordinates, variables are \verb|TYPE|,
%\verb|A|, \verb|B|, \verb|C|, \verb|D|, \verb|COEF|,
%\verb|SCAL|~(1X,A4,4I5,2F10.5). Currently this option is not bug-free,
%and it's use is not to be recommended.
%\begin{description}
%\item[\bf TYPE] This has the value \verb*|    | if this line continues
%a linear combination of primitive internal coordinates.  Otherwise
%the possible values are
%\begin{description}
%\item[\bf STRE] Bond stretch
%\item[\bf INVR] Reciprocal of bond stretch
%\item[\bf BEND] Angle bend
%\item[\bf OUT ] Angle between bond and plane
%\item[\bf TORS] Torsional angle
%\item[\bf LIN1] First of a collinear bending pair
%\item[\bf LIN2] Second of a collinear bend
%\end{description}
%\item[\bf A-D ] Atom numbers specifying the internal coordinates.
%Note that for an angle bend the angle is $\angle ACB$, including
%collinear bends!  Also, the out-of-plane mode is the angle between
%$AC$ and $BCD$??  Finally , in the collinear bend case the
%collinear angle is $\angle ACB$, and $D$ is used to specify a
%plane such that the first bend is in the plane~$ABD$.
%\item[\bf COEF] Coefficient of this primitive internal coordinate in
%the current linear combination.  Default is~1.
%\item[\bf SCAL] Overall scale factor for this linear combination
%(specify on first line only).  Default is~1.
%\end{description}

\item[\Key{HESFIL}] Read the molecular Hessian\index{Hessian} from the file
\verb|DALTON.HES|. This file may have been made in an earlier
calculation using the keyword \Key{HESPUN}, or constructed from a
calculation with the Gaussian94 program and converted to \siraba\
format using the \verb|FChk2HES.f| program. Useful in VCD and VROA
analyses\index{VCD}\index{ROA}\index{vibrational circular
dichroism}\index{Raman optical activity}.

\item[\Key{HESPUN}] Write the molecular Hessian\index{Hessian} to the file
\verb|DALTON.HES| for use as starting Hessian in
first-order\index{first-order optimization} geometry
optimizations (see keyword \Key{HESFIL} in the \Sec{MINIMIZE} input
module), or for later use in a vibrational analysis (see keyword
\Key{HESFIL} in this input module)\index{vibrational analysis}.

\item[\Key{NOCID}] Do not calculate the Circular Intensity
Differentials\index{circular intensity differential}\index{CID} (CIDs)
as defined by Barron~\cite{barronbook}, but 
instead print the chirality\index{chirality number} numbers defined by
Hug~\cite{whc48} in 
calculation of vibrational Raman Optical Activity
(VROA)\index{ROA}\index{Raman optical activity}.

\item[\Key{ISOTOP}]\verb| |\newline
\begin{verbatim}
READ (LUCMD,*) NISOTP, NATM
DO 305 ICOUNT = 1, NISOTP
   READ (LUCMD,*) (ISOTP(ICOUNT,N), N = 1, NATM)
END DO
\end{verbatim}

Read in the number of different isotopically\index{isotopic
constitution} substituted species 
\verb|NISOTP| fore which we will do a vibrational analysis. The
isotopic species containing only the most abundant isotopes is always
calculated.

\verb|NATM| is the total number of atoms in the molecules (see
discussion in Section~\ref{sec:vibfreq}. For each isotopic species,
the isotope for each atom in the molecule is read in. A 1 denotes the
most abundant isotope, a 2 the second-most abundant isotope and so on.

\item[\Key{PRINT}]\verb| |\newline
\verb|READ (LUCMD,*) PRINT|

Set the print level in the vibrational\index{vibrational analysis}
analysis of the molecule.  Read 
one more line containing print level. Default value is the value
of \verb|IPRDEF| from the general input module.

\item[\Key{SKIP}] Skip the analysis of the vibrational frequencies
and modes of the molecule.
\end{description}

