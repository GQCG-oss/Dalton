\chapter{SOPPA, SOPPA(CC2), SOPPA(CCSD) and RPA(D)}\label{ch:soppa}

The Dalton program system can also be used to perform Second-Order
Polarization Propagator Approximation (SOPPA) \index{SOPPA}
\index{polarization propagator}
\cite{esnpjjodjcp73,mjpekdtehjajjojcp,spascpl260,tejospastcan100,spas037},
Second-Order Polarization Propagator Approximation with CC2 Amplitudes
[SOPPA(CC2)] \index{SOPPA(CC2)} \cite{spas097} or Second-Order
Polarization Propagator Approximation with Coupled Cluster Singles and
Doubles Amplitudes [SOPPA(CCSD)] \index{SOPPA(CCSD)} \cite{soppaccsd}
calculations of optical properties like singlet or triplet excitation
energies\index{electronic excitation} and oscillator
strengths\index{transition moment} as well as the following list of
electric and magnetic properties
\begin{center}
\begin{list}{}{}
\item polarizability\index{polarizability}
\item magnetizability\index{magnetizability}
\item rotational {\em g} tensor\index{rotational g tensor}
\item nuclear magnetic shielding constant\index{nuclear shielding}
\item nuclear spin--rotation constant\index{spin-rotation constant}
\item indirect nuclear spin--spin coupling constant\index{spin-spin coupling}
\end{list}
\end{center}
as well as all the linear response functions described in chapter
\ref{ch:rspchap}. Furthermore it can be used to calculate singlet
excitation energies\index{electronic excitation} and oscillator
strengths\index{transition moment} at the SOPPA,\cite{spas037}
SOPPA(CCSD) \cite{spas089} and RPA(D) \cite{spas025}\index{RPA(D)}
using an atomic integral direct implementation of the SOPPA methods.


\section{General considerations}\label{sec:soppageneral}

The Second-Order Polarization Propagator Approximation is a
generalization of the SCF linear response function \cite{esnpjjodjcp73,
jopjdycpr2, mjpekdtehjajjojcp, spas037}. In SOPPA, the SCF reference
wave function in the linear response function or polarization
propagator is replaced by a M{\o}ller-Plesset wave function and all
matrix elements in the response function are then evaluated through
second order in the fluctuation potential. This implies that electronic
excitation energies and oscillator strengths as well as linear response
functions are correct through second order. Although it is a
second-order method like MP2, the SOPPA equations differ significantly
from the expressions for second derivatives of an MP2 energy.

In the RPA(D) model \cite{spas025}\index{RPA(D)} the excitation
energies and transition moments of the random phase approximation /
time-dependent Hartree-Fock or SCF linear response theory are corrected
with a non-iterative second order doubles correction derived from the
SOPPA model using pseudo-perturbation theory
\cite{Christiansen:PERTURBATIVE_TRIPLES}. The RPA(D) is thus similar to
the CIS(D) model \cite{Head-Gordon:94},
\index{CIS(D)} but is based on the RPA model instead of on a simple CIS
model. The performance of both methods has been compared e.g. in Ref.
\cite{spas089}.

In the Second Order Polarization Propagator Approximation with Coupled
Cluster Singles and Doubles Amplitudes [SOPPA(CCSD)] \cite{soppaccsd,
ekdspasjpca102, tejospastcan100, ctocd, spas089} \index{SOPPA(CCSD)} or
the Second Order Polarization Propagator Approximation with CC2
Amplitudes [SOPPA(CC2)] \cite{spas097} methods,\index{SOPPA(CC2)} the
M{\o}ller-Plesset correlation coefficients are replaced by the
corresponding CCSD or CC2 singles and doubles amplitudes. Apart from
the use of the CCSD or CC2 amplitudes, the equations are essentially
the same as for SOPPA. Note that the SOPPA(CCSD) or SOPPA(CC2)
polarization propagators are not coupled cluster linear response
functions and neither SOPPA, SOPPA(CC2) nor SOPPA(CCSD) are the same as
the CC2 model, and although they all three are of at least second order
in all terms they differ in the terms included. The SOPPA(CC2) and
SOPPA(CCSD) models are thus not implemented in the CC module but are
implemented in the RESPONSE and ABACUS modules of the Dalton program,
using the same code as SOPPA. Starting with \dalton\ a second
implementation of the SOPPA equations is available which makes use of
an atomic integral direct algorithm thereby avoiding the storage of the
two-electron repulsion integrals in the molecular orbital basis
\cite{spas025, spas037, spas089}. This implementation can currently be
used to calculate singlet electronic excitation energies and
corresponding transition moments and oscillator strengths using the
SOPPA, SOPPA(CCSD) and RPA(D) models.\index{RPA(D)}


\section{Input description molecular orbital based SOPPA}\label{sec:soppainput}

\begin{center}
\fbox{
\parbox[h][\height][l]{12cm}{
\small
\noindent
{\bf Reference literature:}
\begin{list}{}{}
\item General reference : E.~S. Nielsen, P.~J{\o}rgensen, and J.~Oddershede.
\newblock {\em J.~Chem.~Phys.}, {\bf 73},\hspace{0.25em}6238, (1980).
\item General reference : J.~Oddershede, P.~J{\o}rgensen and D.~Yeager,
\newblock {\em Comput. Phys. Rep.}, {\bf 2},\hspace{0.25em}33, (1984).
\item General reference SOPPA(CCSD): S.~P.~A.~Sauer,
\newblock {\em J.~Phys.\ B: At.~Mol.~Phys.}, {\bf 30},\hspace{0.25em}3773, (1997).
\item General reference SOPPA(CC2): H.~Kj{\ae}r, S.~P.~A.~Sauer, and J.~Kongsted.
\newblock {\em J. Chem. Phys.}, {\bf 133},\hspace{0.25em} 144106, (2010).
\item Excitation energy : M.~J.~Packer, E.~K.~Dalskov, T.~Enevoldsen,
H.~J.~Aa.~Jensen and J.~Oddershede,
\newblock {\em J. Chem. Phys.}, {\bf 105}, \hspace{0.25em}5886, (1996).
\item SOPPA(CCSD) excitation energy :
H.~H.~Falden, K.~R.~Falster-Hansen, K.~L.~Bak, S.~Rettrup and
S.~P.~A.~Sauer,
\newblock {\em J. Phys. Chem. A}, {\bf 113}, \hspace{0.25em} 11995,
(2009).
\item Rotational {\em g} tensor : S.~P.~A.~Sauer,
\newblock {\em Chem. Phys. Lett.}  {\bf 260},\hspace{0.25em}271,
(1996).
\item Polarizability : E.~K.~Dalskov and S.~P.~A.~Sauer.
\newblock {\em J.~Phys.~Chem.~{\bf A}}, {\bf 102},\hspace{0.25em}5269,
(1998).
\item  Spin-Spin Coupling Constants : T.~Enevoldsen, J.~Oddershede,
and S.~P.~A.~Sauer.
\newblock {\em Theor.~Chem.~Acc.}, {\bf 100},\hspace{0.25em}275, (1998)
\item CTOCD-DZ nuclear shieldings: A.~Ligabue, S.~P.~A.~Sauer, P.~Lazzeretti.
\newblock {\em J.Chem.Phys.}, {\bf 118},\hspace{0.25em}6830, (2003).

\end{list}
}}
\end{center}

A prerequisite for any SOPPA calculation is that the calculation of
the MP2 energy and wavefunction is invoked by the keyword \Key{MP2} in
the \Sec{*WAVE FUNCTIONS} input module. Furthermore in the
\Sec{*PROPERTIES} or \Sec{*RESPONSE} input modules it has to be specified
by the keyword \Key{SOPPA} that a SOPPA calculation of the properties
should be carried out.

A typical input file for a SOPPA calculation of the indirect nuclear
spin-spin coupling constants of a molecule will be:

\begin{verbatim}
**DALTON INPUT
.RUN PROPERTIES
**WAVE FUNCTIONS
.HF
.MP2
**PROPERTIES
.SOPPA
.SPIN-S
**END OF DALTON INPUT
\end{verbatim}
whereas as typical input file for the calculation of triplet
excitation energies\index{electronic excitation} with the
\Sec{*RESPONSE} module will be:
\begin{verbatim}
**DALTON INPUT
.RUN RESPONSE
**WAVE FUNCTIONS
.HF
.MP2
**RESPONSE
.TRPFLG
.SOPPA
*LINEAR
.SINGLE RESIDUE
.ROOTS
 4
**END OF DALTON INPUT
\end{verbatim}


A prerequisite for any SOPPA(CC2)\index{SOPPA(CC2)} calculation is that
the calculation of the CC2 amplitudes for the SOPPA program is invoked
by the keyword \Key{CC} in the \Sec{*WAVE FUNCTIONS} input module
together with the \Key{SOPPA2} option in the \Sec{CC INPUT} section.
Furthermore, in the \Sec{*PROPERTIES} or \Sec{*RESPONSE} input modules
it has to specified by the keyword \Key{SOPPA(CCSD)} that a SOPPA(CC2)
calculation of the properties should be carried out.

A typical input file for a SOPPA(CC2) calculation of the indirect
nuclear spin-spin coupling constants of a molecule will be:

\begin{verbatim}
**DALTON INPUT
.RUN PROPERTIES
**WAVE FUNCTIONS
.HF
.CC
*CC INPUT
.SOPPA2
**PROPERTIES
.SOPPA(CCSD)
.SPIN-S
**END OF DALTON INPUT
\end{verbatim}
whereas as typical input file for the calculation of triplet
excitation energies\index{electronic excitation} with the
\Sec{*RESPONSE} module will be:
\begin{verbatim}
**DALTON INPUT
.RUN RESPONSE
**WAVE FUNCTIONS
.HF
.CC
*CC INPUT
.SOPPA2
**RESPONSE
.TRPFLG
.SOPPA(CCSD)
*LINEAR
.SINGLE RESIDUE
.ROOTS
 4
**END OF DALTON INPUT
\end{verbatim}




A prerequisite for any SOPPA(CCSD) calculation\index{SOPPA(CCSD)} is that the calculation of
the CCSD amplitudes for the SOPPA program is invoked by the keyword \Key{CC}
in the \Sec{*WAVE FUNCTIONS} input module together with the \Key{SOPPA(CCSD)}
option in the \Sec{CC INPUT} section. Furthermore, in the \Sec{*PROPERTIES} or
\Sec{*RESPONSE} input modules it has to specified by the keyword \Key{SOPPA(CCSD)}
that a SOPPA(CCSD) calculation of the properties should be carried out.

A typical input file for a SOPPA(CCSD) calculation of the indirect nuclear
spin-spin coupling constants of a molecule will be:

\begin{verbatim}
**DALTON INPUT
.RUN PROPERTIES
**WAVE FUNCTIONS
.HF
.CC
*CC INPUT
.SOPPA(CCSD)
**PROPERTIES
.SOPPA(CCSD)
.SPIN-S
**END OF DALTON INPUT
\end{verbatim}
whereas as typical input file for the calculation of triplet
excitation energies\index{electronic excitation} with the
\Sec{*RESPONSE} module will be:
\begin{verbatim}
**DALTON INPUT
.RUN RESPONSE
**WAVE FUNCTIONS
.HF
.CC
*CC INPUT
.SOPPA(CCSD)
**RESPONSE
.TRPFLG
.SOPPA(CCSD)
*LINEAR
.SINGLE RESIDUE
.ROOTS
 4
**END OF DALTON INPUT
\end{verbatim}



\section{Input description atomic orbital based SOPPA module}\label{sec:AOsoppa}

\begin{center}
\fbox{
\parbox[h][\height][l]{12cm}{
\small \noindent {\bf Reference literature:}
\begin{list}{}{}
\item General reference :
K.~L.~Bak, H.~Koch, J.~Oddershede, O.~Christiansen and S.~P.~A.~Sauer,
\newblock {\em J. Chem. Phys.}, {\bf 112}, \hspace{0.25em} 4173,
(2000).
\item RPA(D) excitation energy :
O.~Christiansen, K.~L.~Bak, H.~Koch and S.~P.~A.~Sauer,  Chem. Phys.
\newblock {\em Chem. Phys.}, {\bf 284}, \hspace{0.25em} 47, (1998).
\item SOPPA(CCSD) excitation energy :
H.~H.~Falden, K.~R.~Falster-Hansen, K.~L.~Bak, S.~Rettrup and
S.~P.~A.~Sauer,
\newblock {\em J. Phys. Chem. A}, {\bf 113}, \hspace{0.25em} 11995,
(2009).
\end{list}
}}
\end{center}

A prerequisite for any atomic orbital based RPA(D), SOPPA or
SOPPA(CCSD) calculation of electronic singlet excitation energies and
corresponding oscillator strengths is the calculation of the
M{\o}ller-Plesset first order doubles and second order singles
correlation coefficients or the CCSD singles and doubles amplitudes.
This is invoked by the keyword \Key{CC} in the \Sec{*WAVE FUNCTIONS}
input module together with the keywords \Key{MP2} and \Key{AO-SOPPA} or
\Key{CCSD} and \Key{AO-SOPPA} in \Sec{CC INPUT} section of the
\Sec{*WAVE FUNCTIONS} input module. Furthermore in the
\Sec{*PROPERTIES} input module it has to be specified by the keywords
\Key{SOPPA} and \Key{EXCITA} that a SOPPA calculation of optical
properties should be carried out. Finally in the \Sec{SOPPA} section of
the \Sec{*PROPERTIES} input module the atomic integral direct
calculation has to be invoked by the one of the three keywords
\Key{AOSOP}, \Key{DCRPA}  or \Key{AOSOC} for either a SOPPA, RPA(D) or
SOPPA(CCSD) calculation. Alternatively an atomic integral direct SOPPA
or SOPPA(CCSD) calculation can be invoked by adding the \Key{DIRECT}
keyword in the \Sec{SOPPA} section of the \Sec{*PROPERTIES} input
module in addition to the \Key{SOPPA} and \Key{EXCITA} or
\Key{SOPPA(CCSD)} and \Key{EXCITA} keywords in the \Sec{*PROPERTIES}
input module. Further keywords, which control the details of an atomic
orbital direct SOPPA, RPA(D) or SOPPA(CCSD) calculation, are described
in chapter \ref{sec:soppa}.

A typical input file for an atomic integral direct calculation of 5
electronic singlet excitation energies to states which transform like
the totally symmetric irreducible representation and their
corresponding oscillator and rotatory strengths at the RPA(D), SOPPA and SOPPA(CCSD)
level of a molecule will be:\index{SOPPA}\index{RPA(D)}\index{SOPPA(CCSD)}

\begin{verbatim}
**DALTON INPUT
.RUN PROPERTIES
**WAVE FUNCTIONS
.HF
.CC
*CC INPUT
.CCSD
.AO-SOPPA
**PROPERTIES
.SOPPA
.EXCITA
*SOPPA
.DCRPA
.AOSOP
.AOSOC
*EXCITA
.DIPSTR
.ROTVEL
.NEXCITA
    5    0    0    0    0    0    0    0
**END OF DALTON INPUT
\end{verbatim}
The corresponding inputs for calculations using only the RPA(D) model will be
\index{RPA(D)}
\begin{verbatim}
**DALTON INPUT
.RUN PROPERTIES
**WAVE FUNCTIONS
.HF
.CC
*CC INPUT
.MP2
.AO-SOPPA
**PROPERTIES
.SOPPA
.EXCITA
*SOPPA
.DCRPA
*EXCITA
.DIPSTR
.ROTVEL
.NEXCITA
    5    0    0    0    0    0    0    0
**END OF DALTON INPUT
\end{verbatim}
and using only the SOPPA model:\index{SOPPA}
\begin{verbatim}
**DALTON INPUT
.RUN PROPERTIES
**WAVE FUNCTIONS
.HF
.CC
*CC INPUT
.MP2
.AO-SOPPA
**PROPERTIES
.SOPPA
.EXCITA
*SOPPA
.DIRECT
*EXCITA
.DIPSTR
.ROTVEL
.NEXCITA
    5    0    0    0    0    0    0    0
**END OF DALTON INPUT
\end{verbatim}
and finally using only the SOPPA(CCSD) model:\index{SOPPA(CCSD)}
\begin{verbatim}
**DALTON INPUT
.RUN PROPERTIES
**WAVE FUNCTIONS
.HF
.CC
*CC INPUT
.CCSD
.AO-SOPPA
**PROPERTIES
.SOPPA(CCSD)
.EXCITA
*SOPPA
.DIRECT
*EXCITA
.DIPSTR
.ROTVEL
.NEXCITA
    5    0    0    0    0    0    0    0
**END OF DALTON INPUT
\end{verbatim}

However, one should note, that is computationally advantageous to
combine calculations at different SOPPA levels in one input file, as
the program automatically uses the converged solutions of the lower
level as start guess for the higher levels.
