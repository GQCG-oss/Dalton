\chapter{Calculation of NMR parameters}\index{NMR}\label{ch:nmrchap}

This chapter describes the calculation of properties important in
nuclear magnetic resonance (NMR) and related spectroscopic techniques.
This includes the the two 
contributions to the ordinary spin-Hamiltonian used in NMR, nuclear
shieldings and indirect nuclear spin-spin couplings constants. In
addition we describe the calculation of molecular magnetizability, a
property of importance when doing NMR experiments with the reference
in another tube than the sample.\typeout{Hva heter forsoek der
referanse og proeve er i hvert sitt NMR-roer?} We also shortly
describe two properties very closely related to the magnetizability
and nuclear shieldings respectively, the molecular g-factor and the
spin-rotation constants.

Gauge origin independent nuclear shieldings and magnetizabilities are
obtained through the use of London atomic orbitals, and the theory is
presented in several references
\cite{kwjfhppjacs112,krthrkpjklbhjajjcp100,krthklbpjhjajjcp99,krthklbpjjo}.
These properties are very easily to calculate (from a users point of
view), and thus is given only a very brief description below.

The indirect spin-spin couplings are calculated by using the quadratic
response function, as described in Ref.~\cite{ovhapjhjajsbpthjcp96}.
These are in principle equally simple to calculate with \aba\ as
nuclear shieldings and magnetizabilities. However, there are 13
contributions to the spin-spin coupling constant from {\em each}
nucleus. Furthermore, the spin-spin coupling constants put severe
requirements on the quality of the basis set as well as a proper
treatment of correlation, making the evaluation of spin-spin coupling
constants a very time consuming task. Some notes about how this time
can be reduced is given below.

\section{Magnetizabilities}\label{sec:magnetizability}

{\it \aba\ input files:} abastr.inp

\bigskip

The calculation of molecular magnetizabilities is invoked by the
keyword \verb|.MAGNET| in the general input module of abastr.inp. Thus
a complete input file for the calculation of molecular
magnetizabilities will look as:

\smallskip

\begin{tabular}{l}
\verb|*ABACUS|\\
\verb|.MAGNET|\\
\verb|*END OF|
\end{tabular}

\smallskip

This will invoke the calculation of molecular magnetizabilities using
London Atomic Orbitals to ensure fast basis set convergence and gauge
orgin independent results. The natural connection \cite{joklbkrthpj}
is used in order to get numerically accurate results as well as give
the diamagnetic and paramagnetic contribution to the magnetizability a
proper physical interpretation. By default the center of mass is
chosen as gauge origin.

A basis set especially well suited for the calculation of
magnetizabilites, the cc+p-basis, has been developed
\cite{pdkrthklbpj}, and it is obtainable from the basis set library.
The excellent performance of this basis set with London orbitals is
well documented
\cite{krthklbpjhjajjcp99,krthpjklbcpl223,krhsthklbpjjacs}.

Notice that a general print level of 2 or higher is needed in order to
get the individual contributions (relaxation, one- and
two-electron expectation values and so on) to the total magnetizability.

If more close control of the different parts of the calculation of the
magnetizability is wanted we refer the reader to the section
describing the options available. The modules that controls the
calculation of molecular magnetizabilities are:

\begin{list}{}{\itemsep 0.10cm \parsep 0.0cm}
\item[\verb|*EXPECT|] Controls the calculation of one-electron
expectation values contributing to the diamagnetic magnetizability.
\item[\verb|*GETSGY|] Controls the set up of the right-hand sides
(gradient terms) as well as the calculation of two-electron
expectation values and reorthonormalization terms.
\item[\verb|*LINRES|] Controls the solution of the magnetic response
equations
\item[\verb|*RELAX |] Controls the adding of solution and right-hand
side vectors into relaxation contributions
\end{list}

\section{Nuclear shielding constants}\label{sec:shieldings}

{\it \aba\ input files:} abastr.inp

\bigskip

The calculation of nuclear shieldings are invoked by the
keyword \verb|.SHIELD| in the general input module of abastr.inp. Thus
a complete input file for the calculation of nuclear shieldings will
look as: 

\smallskip

\begin{tabular}{l}
\verb|*ABACUS|\\
\verb|.SHIELD|\\
\verb|*END OF|
\end{tabular}

\smallskip

This will invoke the calculation of nuclear shieldings using
London Atomic Orbitals to ensure fast basis set convergence and gauge
orgin independent results. The natural connection \cite{joklbkrthpj}
is used in order to get 
numerically accurate results as well as give the diamagnetic and
paramagnetic contribution to the nuclear shieldings a proper physical
interpretation. By default the center of mass is chosen as gauge
origin.

A basis set well suited for the calculation of nuclear shieldings (and
indirect nuclear spin-spin coupling constants) has been recommended
\cite{pdkrthklbpj} and is available from the basis set library as H
III. Several investigations of nuclear shieldings using this basis
sets have been done
\cite{mjthkrklbpjcpl220,krthrkpjklbhjajjcp100,krthklbpjcpl226}.

Notice that a general print level of 2 or higher is needed in order to
get the individual contributions (relaxation, one- and
two-electron expectation values and so on) to the total nuclear shieldings.
 
If more close control of the different parts of the calculation of the
nuclear shieldings is wanted we refer the reader to the section
describing the options available. For the calculation of nuclear
shieldings, these are the same as listed above for magnetizability
calculations.

\section{Molecular g-factor}\label{sec:gfac}

Not implemented yet.

\section{Spin-rotation constants}\label{sec:spinrotasjon}

Not implemented yet.

\section{Indirect nuclear spin-spin coupling
constants}\label{sec:spinspin}

{\it \aba\ input files:} abastr.inp

\bigskip

As mentioned in the introduction of this chapter, the calculation of indirect nuclear
spin-spin coupling constants is a time consuming task due to the
large number of contribution to the total spin-spin coupling
constants. Still, if all spin-spin couplings in a molecule are wanted,
with some restrictions mentioned below, the input will look as
follows:

\smallskip

\begin{tabular}{l}
\verb|*ABACUS|\\
\verb|.SPIN-S|\\
\verb|*END OF|
\end{tabular}

\smallskip

By default this will calculate the indirect nuclear spin-spin coupling
constants between isotopes with a non-zero magnetic moments and a
natural abundance of more than 1\% . This limit will automatically
include proton and $^{13}$C spins-spin coupling constants. By default
all contributions to the coupling constants will be calculated. 

Often one is interested in only certain kinds of nuclei. For example,
 one may want to calculate only the proton spin-spin couplings of the molecule.
This can be accomplished in two ways: either by changing the abundancy
threshold so that it only includes that single isotope (most useful
for proton couplings), or by selecting the particular nuclei of interest. The
first of these options is useful when several isotopes of a single
nuclei is wanted, some of which have low abundacy.

All the keywords necessary to control such adjustments to the
calculation is given in the \verb|*SPIN-S| input module. Thus an input
in which we both have reduced the the abundancy threshold as well as
selected three atoms will look as:

\smallskip

\begin{tabular}{l}
\verb|*ABACUS|\\
\verb|.SPIN-S|\\
\verb|*SPIN-S|\\
\verb|.ABUNDA|\\
\verb| 0.10|\\
\verb|.SELECT|\\
\verb|    3  |\\
\verb|    2    4    5|\\
\verb|*END OF|
\end{tabular}

\smallskip

We refer to the section describing the \verb|*SPIN-S| input module for
the complete description of the syntax for these keywords, as well as
the numbering of the atoms which are selected.

We also notice that it is often of interest to calculate only specific
contributions (usually the Fermi-contact contribution) at a high level
of approximation. There is some indication that Hartree-Fock may be
able to predict the relative importance of the different
contributions, thus helping in the decision of which contributions
should be calculated at a correlated level \cite{krthklbpjcpl226}.
The calculation of only certain contributions can be accomplished in
the input by turning off the different 
contributions by the keywords \verb|.NODSO |, \verb|.NOPSO |,
\verb|.NOSD  |, and \verb|.NOFC  |. We refer to the description of the
\verb|*SPIN-S| input module for a somewhat more thorough discussion.

If a closer a control of the individual parts of the calculation of
indirect nuclear spin-spin coupling constants is wanted, this can be
done through the use of keywords in the following input modules:

\begin{list}{}{\itemsep 0.10cm \parsep 0.0cm}
\item[\verb|*EXPECT|] Controls the calculation of one-electron
expectation value contribution to the diamagnetic spin-spin coupling
constants. 
\item[\verb|*GETSGY|] Controls the set up of the right-hand sides
(gradient terms) 
expectation values and reorthonormalization terms.
\item[\verb|*LINRES|] Controls the solution of the singlet magnetic
response equations
\item[\verb|*TRPRSP|] Controls the solution of the triplet magnetic
response equations (for Fermi-contact and spin-dipole contributions)
\item[\verb|*RELAX |] Controls the adding of solution and right hand
side vectors into relaxation contributions
\end{list}
