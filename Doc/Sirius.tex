\chapter{\label{chap:inpref} Molecular wave functions, {\sir}}

\section{\label{sec:ref-notes} General notes for the {\sir} input reference
manual}

{\sir} is the part of the code that computes the wave function.

The following sections contain a list of all generally relevant keywords to
{\sir}, only currently inactive keywords and some special debug
options are omitted.

\begin{enumerate}
\item {The input for the wave function section must begin with

\begin{inputex} \begin{verbatim}
**WAVE FUNCTIONS
\end{verbatim} \end{inputex}
   with no leading blanks.  The preceding lines in the input file may
   contain arbitrary information.
}
\item{ Input is directed by keywords written in upper case.
   Only the first 7 characters including the prompt are significant.
   The keywords are divided in a number of main input groups. Each main
   input group is initiated by a {\starkey}, for example

\begin{inputex} \begin{verbatim}
*ORBITAL INPUT
\end{verbatim} \end{inputex}
   marks the beginning of the input group for orbital input.
}
\item { The keywords belonging to one of the main input groups begin with
   the prompt {\dotkey}.
}
\item { Keywords that are necessary to specify are marked by "Required".
   For other keywords the default values can be used in ordinary runs.
}
\item {Any keyword line beginning with a \quotekw{!} or
   \quotekw{\#} will be treated as a
   comment line.  An illegal keyword will cause a dump of all keywords
   for the current input section.
}
\item{A dump of keywords can be obtained in any input section by
specifying the keyword \quotekw{\Key{OPTIONS}}.  For example, the input

\begin{inputex} \begin{verbatim}
**WAVE FUNCTIONS
.OPTIONS
**END OF INPUT
\end{verbatim} \end{inputex}
   will cause a dump of the labels for the main input groups in {\sir},
   while

\begin{inputex} \begin{verbatim}
**WAVE FUNCTIONS
*ORBITAL INPUT
.OPTIONS
**END OF INPUT
\end{verbatim} \end{inputex}
   will cause a dump of the labels for the \quotekw{*ORBITAL INPUT} input group
   in {\sir}.
}
\item{ The {\sir} input is finished with a line beginning with two stars,
   e.g.

\begin{inputex} \begin{verbatim}
**END OF INPUT
\end{verbatim} \end{inputex}
}
\end{enumerate}


\pagebreak[3]
\section{\label{sec:ref-newinp}
   Main input groups in the **WAVE FUNCTIONS input module}

\noindent
The main input groups (those with the {\starkey} prompt) are listed here and
the full descriptions are given in the designated sections.

\noindent
The first input group is always required in order to specify the type of
calculation, and follows immediately after the \Sec{*WAVE FUNCTIONS}
keyword.

%Section~\ref{ref-geninp} \Sec{GENERAL INPUT}

\noindent The remaining input groups may be specified in any
order. In this chapter they are grouped alphabetically, although
the short presentation below gather them according to purpose.

%\ifsolvent
The following two input groups are used to modify the
molecular environment by adding field-dependent
terms in the Hamiltonian and by invoking
the self-consistent reaction field model for solvent
effects, respectively:

Section~\ref{ref-haminp} \Sec{HAMILTONIAN}

Section~\ref{ref-solinp} \Sec{SOLVENT}
%\else
%The following input group describes additional field-dependent
%terms in the Hamiltonian :
%
%Section~\ref{ref-haminp} \Sec{HAMILTONIAN}
%\fi

\noindent
The next input group specifies the configurations included
in the MCSCF and CI wave functions:

Section~\ref{ref-wavinp} \Sec{CONFIGURATION INPUT}

\noindent
The two next groups are used to specify initial orbitals and initial
guess for the CI vector:

Section~\ref{ref-orbinp} \Sec{ORBITAL INPUT}

Section~\ref{ref-civinp} \Sec{CI VECTOR}

\noindent
The two following input groups control the second-order MCSCF
optimization:

Section~\ref{ref-optinp} \Sec{OPTIMIZATION}

Section~\ref{ref-stpinp} \Sec{STEP CONTROL}

\noindent
The next three groups have special input only relevant for the
respective calculation types:

Section~\ref{ref-cicinp} \Sec{CI INPUT}

Section~\ref{ref-rhfinp} \Sec{HF INPUT}

Section~\ref{ref-mp2inp} \Sec{MP2 INPUT}

Section~\ref{ref-nevpt2inp} \Sec{NEVPT2 INPUT}

\noindent
The next section is used to select some types of analysis of the final
MCSCF or CI wave function:

Section~\ref{ref-popinp} \Sec{POPULATION ANALYSIS}

\noindent
The next section is used to change the default integral transformation
and specify any final integral transformation after convergence (a
program following {\sir} may need a higher transformation level):

Section~\ref{ref-trainp} \Sec{TRANSFORMATION}

\noindent
The next two input groups control the amount of printed output and
collect options not fitting in any of the other groups:

Section~\ref{ref-priinp} \Sec{PRINT LEVELS}

Section~\ref{ref-auxinp} \Sec{AUXILLIARY INPUT}

\noindent
Finally we note that there is an input module controling the
calculation of Coupled Cluster wave functions. This is treated in a
separate chapter:

Chapter~\ref{ch:CC} \Sec{CC INPUT}

\bigskip
\noindent
The wave function input is finished when a line is encountered beginning
with two stars, for example

\begin{inputex} \begin{verbatim}
   **END OF INPUT
\end{verbatim} \end{inputex}
or

\begin{inputex} \begin{verbatim}
   **MOLORB
   ... formatted molecular orbitals coefficients
   **END OF INPUT
\end{verbatim} \end{inputex}

\noindent
The \Sec{*MOLORB} keyword or the \Sec{*NATORB} keyword
must be somewhere on the input file and be
followed by molecular orbital coefficients if the option for formatted
input of molecular orbitals has been specified.  Apart from this
requirement, arbitrary information can be written to the following lines
of the input file.



\pagebreak[3]
\subsection{\label{ref-geninp}\Sec{*WAVE FUNCTIONS}}

{\bf Purpose:}

Specification of which calculation is to be performed and
optionally of the number of symmetries for consistency check of
input.

\begin{description}
\item[\Key{BASIS SET}]
   Default: specified by integral program \\
   \verb"READ (LUINP,*) (NBAS(I), I = 1,NSYM)" \\
   Number of basis functions per symmetry.

\item[\Key{CC}]
  \index{Coupled Cluster}\index{CC}
  Coupled Cluster calculation. After a Hartree--Fock calculation
  the \cc\ program is called to do a Coupled Cluster (response) calculation.
  For further input options for the \cc\ program see
  Section~\ref{sec:ccgeneral}. 

\item[\Key{CC ONLY}] Skip the calculation of a Hartree--Fock wave
  function, and start directly in the Coupled Cluster part. Convenient
  for restarts in the Coupled Cluster module.

\item[\Key{CI}]
  \index{CI}\index{Configuration Interaction}
  Configuration interaction calculation.

\item[\Key{FLAGS}]
  \verb"READ (LUINP,NMLSIR)" \\
  Read namelist \verb"$NMLSIR ... $END" \\
  Only for debugging. Set internal control flags directly.
  Usage is not documented.

\item[\Key{HF}]
  Restricted closed-shell or one open-shell 
  Hartree--Fock\index{HF}\index{SCF}\index{Hartree--Fock}\index{open shell!HF} calculation.

\item[\Key{INTERFACE}]
  Write the "SIRIFC" interface file\index{interface file} for post-processing programs.

\item[\Key{MCSCF}]
  Multiconfigurational self-consistent field (MCSCF) calculation.\index{MCSCF}

\item[\Key{MP2}]
 \index{M{\o}ller-Plesset!second-order}\index{MP2}
  M{\o}ller-Plesset second-order perturbation theory calculation.

\item[\Key{NEVPT2}]
 \index{NEVPT2}
  Multireference second-order perturbation theory calculation.

\item[\Key{NSYM}]
%  Required, no defaults. \\
  Default: specified by integral program \\
  \verb"READ (LUINP,*) NSYM" \\
  Number of spatial Abelian symmetries (1, 2, 4, or 8), corresponding
  to $D_{2h}$ or one of its subgroups.


\item[\Key{PRINT}]
  \verb"READ (LUINP,*) IPRSIR" \\
  General {\sir} print level and default for all other print parameters.

\item[\Key{RESTART}]
  restart {\sir}\index{restart!wave function},
  the {\sir} restart file (\verb|SIRIUS.RST|) must be available

\item[\Key{SCF}]
    Alias for the \quotekw{\Key{HF}} keyword.

\item[\Key{STOP}]
  \kw{READ (LUINP,'(A20)') REWORD} \\
  Terminate {\sir} according to the instruction given on the following line.
  Three stop points are defined:
\begin{enumerate}

\item \hspace{2em} \quotekw{ AFTER INPUT}

\item \hspace{2em} \quotekw{ AFTER MO-ORTHONORMALIZATION}

\item \hspace{2em} \quotekw{ AFTER GRADIENT}
\end{enumerate}

\item[\Key{TITLE}] 
  \verb"READ (LUINP,'(A)') TITLE(NTIT)" 
  Any number of title lines (until next line beginning with
  \quotekw{.} or \quotekw{*} prompt).
  Up to 6 title lines will be saved and used in the output, additional
  lines will be discarded.

\item[\Key{WESTA}]
  Write the "SIRIFC" interface file for the WESTA post-processing program.

%\item[\Key{BASIS SET}]
%   Default: specified by integral program \\
%   \verb"READ (LUINP,*) (NBAS(I), I = 1,NSYM)" \\
%   Number of basis functions per symmetry.
\end{description}

\noindent{\bf Comments:}

%\ifabacus ABACUS: 
%If the full molecular Hessian is calculated in
%ABACUS and the number of symmetries (\verb|NSYM|) is greater than
%one, then the MCSCF wave function will be automatically calculated
%in determinants\index{determinants} and, if singlet,
%\quotekw{.PLUS COMBINATIONS}).  This is so because the CSFs can
%only have one spatial symmetry, and it is generally necessary to
%solve linear response equations of several symmetries to get the
%full molecular Hessian. 
%\fi

BASIS SET is provided such that the number of basis functions in each
symmetry may be specified if {\sir} is modified to interface to an
integral program which doesn't write this information to the integral
file.


\pagebreak[3]
\subsection{\label{ref-auxinp}\Sec{AUXILLIARY INPUT}}

{\bf Purpose:}

Input which does not naturally fit into any of the other
categories.

\begin{description}
\item[\Key{NOSUPMAT}]
  Do not use P-SUPERMATRIX integrals, but calculate Fock matrices
  from AO integrals (slower, but requires less disk space). The
  default is to use the supermatrix file if it exists.

%\item[\Key{ONESUP}]
%  Use same unit for P-SUPERMATRIX\index{supermatrix} and ONE-ELECTRON
%  integrals \index{one-electron integral}
%  (LUSUPM=LUONEL, default is different units).
\end{description}

\pagebreak[3]
\subsection{\label{ref-cicinp}\Sec{CI INPUT}}

{\bf Purpose:}

Options for a CI calculation.

\begin{description}
\item[\Key{CIDENSITY}]
  Calculate CI one-electron density matrix and natural
  orbital\index{natural orbital}
  occupations after convergence.

\item[\Key{CINO}]
  Generate CI\index{CI}\index{Configuration Interaction} natural
  orbitals\index{natural orbital} for CI root
  number \kw{ISTACI},
  clear the \verb|SIRIUS.RST| file and write the orbitals with label \quotekw{NEWORB  }.
  The \quotekw{\Key{STATE}} option must be specified.

\item[\Key{CIROOTS}]
  Default: NROOCI = 1\\
  \kw{READ (LUINP,*) NROOCI} \\
  Converge the lowest \kw{NROOCI} CI roots\index{root!CI} to threshold.

\item[\Key{DISKH2}]
  Active two-electron MO integrals on disk (see comments below).

\item[\Key{MAX ITERATIONS}]
  \kw{READ (LUINP,*) MXCIMA} \\
  Max iterations in iterative diagonalization of CI matrix (default = 20).

\item[\Key{STATE}]
  Default: not specified\\
  \kw{READ (LUINP,*) ISTACI} \\
  Alternative to \quotekw{\Key{CIROOTS}}.  Converge root number \kw{ISTACI}
  to threshold, converge all lower roots only to THQMIN
  (from the \quotekw{\Sec{OPTIMIZATION}} input group, see
  p.~\pageref{ref-optinp}).

\item[\Key{THRESH}]
  Default = 1.D-05\\
  \kw{READ (LUINP,*) THRCI} \\
  Threshold for CI energy gradient.  The CI energy will be converged to
  approximately the square of this number.

\item[\Key{THRPWF}]
  Default = 0.05D0 for electronic ground states, and 0.10D0 for
  excited states\\
  \kw{READ (LUINP,*) THRPWF} \\
  Only CI coefficients greater than threshold are printed
  (PWF: print wave function).

\item[\Key{WEIGHTED RESIDUALS}]
  Use energy weighted residuals\index{residual} (see comments below).

\item[\Key{ZEROELEMENTS}]
  Zero small elements in CI trial vectors (see comments below).
\end{description}


\noindent{\bf Comments:}

DISKH2: By default the CI module will attempt to place the two-electron
integrals with four active indices in memory for more efficient
calculation of CI sigma vectors, if memory is insufficient for this
the
integrals will automatically be placed on disk.  The DISKH2 keyword
forces the integrals always to be on disk.

WEIGHTED RESIDUALS:  Normally the CI states will be converged to a
residual less than the specified threshold, and this will give
approximately the squared number of decimal places in the energy.
Depending on the value of the energy, the eigenvectors will be converged
to different accuracy. If the eigenvectors are wanted with, e.g., at
least 6 decimal places for property calculations, specify a threshold of
1.0D-6 and weighted residuals.

ZEROELEMENTS: an experimental option to perhaps save time (if the CI
module can use sparseness) by zeroing all elements less than 1.0D-3
times the largest element in the CI trial vector before
orthonormalization against previous trial vectors.
See also \quotekw{.SYM CHECK} under \quotekw{*OPTIMIZATION}
(p.~\pageref{ref-optinp}).


\pagebreak[3]
\subsection{\label{ref-civinp}\Sec{CI VECTOR}}

{\bf Purpose:}

To obtain initial guess for CI vector(s).

\begin{description}
\item[\Key{PLUS COMBINATIONS}]
  Use with \quotekw{\Key{STARTHDIAGONAL}} to choose plus combination
  of degenerate   diagonal elements ({\bf STRONGLY RECOMMENDED} for
  calculation of singlet states  with \quotekw{\Key{DETERMINANTS}}).

\item[\Key{SELECT}]
  \kw{READ (LUINP,*) ICONF} \\
  Select CSF (or determinant if \quotekw{\Key{DETERMINANTS}}) no.
  ICONF as start configuration.\index{configuration!start}

\item[\Key{STARTHDIAGONAL}]
  Select configurations corresponding to the lowest diagonal elements in
  the configuration part of the Hessian (this is the default option).

\item[\Key{STARTOLDCI}]
  Start from old CI-vector stored saved on the \verb|SIRIUS.RST| file.

%\ifabacus
%\item[.ABACUS]
%  Geometry walk, use CI vector and "GEOSAVE" information saved by
%  ABACUS at previous geometry.  The "GEOSAVE" information is used
%  to decide as early as possible in the wave function optimization
%  if the step should be rejected, thus saving CPU time if the step
%  is rejected.
%\fi

\end{description}

\pagebreak[3]
\subsection{\label{ref-wavinp}\Sec{CONFIGURATION INPUT}}

{\bf Purpose:}

To specify the configuration part in MCSCF and CI calculations.

\begin{description}
\item[\Key{CAS SPACE}]
  \kw{READ (LUINP,*) (NASH(I),I=1,NSYM)} \\
  CAS calculation: Active orbitals\index{active orbital} each symmetry.

\item[\Key{ELECTRONS}]
  Required.\\
  \kw{READ (LUINP,*) NACTEL} \\
  Number of active electrons\index{active electrons} (the number of
  electrons to be distributed
  in the active orbitals).  The total number of electrons is this number
  plus two times the total number of inactive orbitals.

\item[\Key{RAS1 ELECTRONS}]
   \kw{READ (LUINP,*) NEL1MN,NEL1MX} \\
   Minimum and maximum number of RAS1 electrons; this can alternatively
   be specified with \quotekw{\Key{RAS1 HOLES}}

\item[\Key{RAS1 HOLES}]
  \kw{READ (LUINP,*) NHL1MN,NHL1MX} \\
  Minimum and maximum number of holes\index{electron hole} in RAS1; alternative
  to \quotekw{\Key{RAS1 ELECTRONS}}

\item[\Key{RAS1 SPACE}]
   \kw{READ (LUINP,*) (NAS1(I),I=1,NSYM)} \\
   RAS calculation: RAS1 orbital space\index{RAS1 orbital space}

\item[\Key{RAS2 SPACE}]
   \kw{READ (LUINP,*) (NAS2(I),I=1,NSYM)} \\
   RAS calculation: RAS2 orbital space\index{RAS2 orbital space}

\item[\Key{RAS3 ELECTRONS}]
   \kw{READ (LUINP,*) NEL3MN, NEL3MX} \\
   Minimum and maximum number of RAS3 electrons

\item[\Key{RAS3 SPACE}]
   \kw{READ (LUINP,*) (NAS3(I),I=1,NSYM)} \\
   RAS calculation: RAS3 orbital space\index{RAS3 orbital space}

\item[\Key{SPIN MULTIPLICITY}]
  Required for MCSCF and CI wave
  functions\index{MCSCF}\index{CI}\index{Configuration Interaction}.\\
  \kw{READ (LUINP,*) ISPIN}\\
  For CSF basis: state spin multiplicity\index{spin multiplicity} = $2S + 1$,
  where $S$ is the spin quantum number. \\
  For determinant basis this option determines the minimum spin
  multiplicity.  The $M_S$ value is determined as (ISPIN-1)/2.

\item[\Key{SYMMETRY}]
  Required for MCSCF and CI wave functions.\\
  \kw{READ (LUINP,*) LSYM} \\
  Spatial symmetry\index{symmetry} of MCSCF wave function

\item[\Key{INACTIVE ORBITALS}]
  Required.\\
  \kw{READ (LUINP,*) (NISH(I),I=1,NSYM)} \\
  Number of inactive orbitals\index{inactive orbital} each symmetry.

\end{description}

\noindent{\bf Comments:}

\noindent SYMMETRY   Specifies total spatial symmetry of the wave
function in $D_{2h}$ symmetry and its subgroups: $C_{2v}$, $C_2h$,
$D_2$, $C_s$, $C_i$, $C_2$, $C_1$. The  symmetry number of wave
function follows MOLECULE output ordering of symmetries ($D_{2h}$
subgroup IR's).

\noindent
CAS and RAS\index{RASSCF}\index{CASSCF}\index{MCSCF} are exclusive and
both cannot be specified in the same
MCSCF or CI\index{MCSCF}\index{CI}\index{Configuration Interaction}
calculation. One of them {\em must} be specified.


\pagebreak[3]
\subsection{\label{ref-haminp}\Sec{HAMILTONIAN}}

{\bf Purpose:}

Add extra terms to the Hamiltonian (for finite field\index{finite field} calculations).

\begin{description}
\item[\Key{FIELD TERM}]
  Default = no fields. \\
  \verb"READ (LUINP,*) EFIELD(NFIELD)" \\
  \verb"READ (LUINP,'(1X,A8)') LFIELD(NFIELD)" \\
  Enter field strength (in au) and label,
  where label is a MOLE\-CULE-style property label produced
  by the property program, see Chapter~\ref{ch:hermit}.


\item[\Key{PRINT}]
  Default = 0.\\
  \verb"READ (LUINP,*) IPRH1" \\
  If greater than zero:
  print the one-electron Hamiltonian matrix, including
  specified field-dependent terms, in AO basis.
\end{description}

\noindent{\bf Comments:}

Up to \mxfelt simultaneous fields may be specified by repeating the
\quotekw{\Key{FIELD TERM}} keyword.
The field integrals are read from file \verb|AOPROPER| with the specified label.


\pagebreak[3]
\subsection{\label{ref-rhfinp}\Sec{HF INPUT}}

{\bf Purpose:}

This section deals with the closed shell and one open shell
Hartree--Fock cases\index{SCF}\index{HF}\index{Hartree--Fock}.  The
input here will usually only be used if
\quotekw{\Key{HF}} (or its alias \quotekw{\Key{SCF}}) or
\quotekw{\Key{MP2}}\index{MP2}\index{M{\o}ller-Plesset!second-order}
have been specified under \quotekw{\Sec{*WAVE
FUNCTIONS}}. Single configuration cases with more than one open
shell\index{open shell}
are handled by the general \Sec{CONFIGURATION INPUT} section.

\begin{description}
\item[\Key{AUTOCCUPATION}]
  Default for SCF\index{SCF}\index{HF}\index{Hartree--Fock} and
  MP2\index{MP2}\index{M{\o}ller-Plesset!second-order} calculations
  starting from H\"{u}ckel or H1DIAG
  starting orbitals\index{starting orbitals}\index{H\"{u}ckel}.

  Allow the Hartree--Fock
occupation\index{Hartree--Fock occupation}\index{HF occupation} to
  change based on changes in
  orbital ordering during DIIS\index{DIIS} optimization.

\item[\Key{C2DIIS}]
  Use Harell Sellers' C2-DIIS algorithm instead of Pulay's C1-DIIS algorithm
  (see comments).

\item[\Key{COREHOLE}]
  \kw{READ (LUINP,*) JCHSYM,JCHORB} \\
  JCHSYM = symmetry of core orbital\\
  JCHORB = the orbital in symmetry JCHSYM with a single core hole\\
  Single core hole\index{core hole} open shell RHF calculation, \quotekw{\Key{OPEN
  SHELL}} must not
  be specified.  The specified core orbital must be
  inactive\index{inactive orbital}.
  The HF occupation in symmetry \kw{JCHSYM} will be reduced with one
  and instead an
  open shell orbital will be added for the core hole orbital.
  If the specified core orbital is not the last occupied orbital in symmetry
  \kw{JCHSYM} it will switch places with that orbital and user-defined reordering
  is not possible.
  If explicit reordering is required you must also reorder
  the core orbital yourself and let \kw{JCHORB} point to the last occupied orbtial
  of symmetry \kw{JCHSYM}.  See comments below.

\item[\Key{CORERELAX}]
  (ignored if \quotekw{\Key{COREHOLE}} isn't also specified)\\
  Optimize core hole\index{core hole} state with relaxed
  core\index{relaxed core} orbital using Newton-Raphson algorithm.
  It is assumed that this calculation follows an optimization
  with frozen core orbital and the specific value of
  \quotekw{JCHORB} under \quotekw{\Key{COREHOLE}} is ignored (no
  reordering will take place).

%\item[\Key{DIRFOCK}]
%  Direct Fock matrix constructions (recalculate integrals when needed).
%  Default: AO integrals or P-supermatrix integrals read from disk.
%\fi

\item[\Key{ELECTRONS}]
  \kw{READ (LUINP,*) NRHFEL} \\
  Number of electrons in the molecule\index{electrons in molecule}. By
  default, this number will be
  determined on the basis of the nuclear charges and the total charge
  of the molecule\index{charge of molecule} as specified in the
  \verb|MOLECULE.INP| file. The
  keyword is incompatible with the keywords \quotekw{\Key{HF
  OCCUPATION}} and \quotekw{\Key{OPEN SHELL}}.

\item[\Key{FC MVO}]
  See Ref.~\cite{cwbjcp72}.

\item[\Key{FOCK ITERATIONS}]
  \kw{READ (LUINP,*) MAXFCK} \\
  Maximum number of closed-shell Roothaan\index{Roothaan iteration}
  Fock iterations (default = 0).

\item[\Key{FROZEN CORE ORBITALS}]
  \kw{READ (LUINP,*) (NFRRHF(I),I=1,NSYM)} \\
  Frozen orbitals per symmetry (if MP2 follows then at least these orbitals
  must be frozen in the MP2 calculation).
  NOTE: no Roothaan Fock iterations if frozen orbitals.

\item[\Key{H1VIRTUALS}]
  Virtual orbitals that diagonalize the one-electron Hamiltonian matrix
  (see comments below).

\item[\Key{HF OCCUPATION}]
    \kw{READ (LUINP,*) (NRHF(I),I=1,NSYM)} \\
  \index{HF}\index{SCF}\index{Hartree--Fock}\index{MP2}\index{M{\o}ller-Plesset!second-order}
  Hartree--Fock occupancy for RHF and MP2 calculations. This keyword
  is required is Hartree--Fock or MP2 is part of a multistep
  calculation which includes an MCSCF wave function. 

\item[\Key{MAX DIIS ITERATIONS}]
  \kw{READ (LUINP,*) MXDIIS} \\
  Maximum number of DIIS iterations\index{DIIS!iteration} (default = 30).

\item[\Key{MAX ERROR VECTORS}]
  \kw{READ (LUINP,*) MXEVC} \\
  Maximum number of DIIS error vectors\index{DIIS!error vector}
  (default = 10, if there is sufficient memory available to hold these
  vectors in memory).

\item[\Key{MAX MACRO ITERATIONS}]
  \kw{READ (LUINP,*) MXHFMA} \\
  Maximum number of QCHF macro\index{HF!quadratic convergent} iterations (default = 15).

\item[\Key{MAX MICRO ITERATIONS}]
  \kw{READ (LUINP,*) MXHFMI} \\
  Maximum number of QCHF\index{HF!quadratic convergent} micro iterations per macro iteration (default = 12).

\item[\Key{NODIIS}]
  Do not use DIIS algorithms\index{DIIS} (default: use DIIS algorithm).

\item[\Key{NONCANONICAL}]
  No transformation to canonical orbitals\index{canonical orbital}

\item[\Key{NOQCHF}]
  No quadratically convergent Hartree--Fock\index{HF!quadratic convergent} iterations

\item[\Key{OPEN SHELL}]
  Default = no open shell\\
  \kw{READ (LUINP,*) IOPRHF} \\
  Symmetry of the open shell in a one open shell\index{open shell!HF}\index{HF!open shell}
  calculation.

\item[\Key{PRINT}]
  \kw{READ (LUINP,*) IPRRHF} \\
  Resets general print level to \verb|IPRRHF| in Hartree--Fock calculation
  (if not specified, global print levels will be used).

\item[\Key{THRESH}]
  Default = 1.0D-06\\
  \kw{READ (LUINP,*) THRRHF} \\
  Hartree--Fock convergence threshold for energy gradient.  The convergence
  of the energy will be approximately the square of this number.

\end{description}


\noindent{\bf Comments:}

By default, the RHF part of a {\sir} calculation will consist of :
\begin{enumerate}
\item {MAXFCK Roothaan Fock iterations (early exit if convergence
    or oscillations). However, the default is that no Roothaan Fock
iterations are done unless explicitly requested through the keyword
\quotekw{.FOCK I}.
}
\item {MXDIIS DIIS iterations (exit if convergence, i.e. gradient norm
    less than THRRHF, and if convergence rate too slow or even diverging).
}
\item {Unless NOQCHF, quadratically convergent Hartree--Fock until
    gradient norm less than THRRHF.
}
\item{If \quotekw{.H1VIRT} then transformation of
    virtual orbitals to diagonalize
    the one-electron Hamiltonion, i.e. virtual orbitals will be defined
    for a bare nuclei potential.
    If the RHF calculation is going to be followed
    directly by an MCSCF calculation,
    \quotekw{.H1VIRT} will usually provide much
    better start orbitals than the canonical orbitals (canonical
    orbitals will usually put diffuse, non-correlating orbitals in the
    active space).  Note that if the basis set contains compact,
    core-correlating orbitals, \quotekw{.H1VIRT} will put those in the
    active space. 
    WARNING: if both \quotekw{.MP2} and \quotekw{.H1VIRT} are specified,
    then the MP2 orbitals will be destroyed and replaced with \kw{H1VIRT}
    orbitals.
}
\end{enumerate}

In general \quotekw{.HF OCCUPATION} should be specified for CI or MCSCF\index{HF occupation}\index{Hartree--Fock occupation}\index{CI}\index{MCSCF}\index{Configuration Interaction}
wave function calculations.  If not specified, the HF occupation
will be the number of inactive orbitals in the MCSCF calculation (or
CI calculation if no MCSCF). For SCF or MP2\index{HF}\index{SCF}\index{Hartree--Fock}\index{MP2}\index{M{\o}ller-Plesset!second-order}
calculations the Hartree--Fock occupation will be determined on the
basis of the nuclear charges and molecular charge of the molecule as
specified in the \verb|MOLECULE.INP| file.

By default, starting orbitals and initial Hartree--Fock occupation will
be determined on the basis of a H\"{u}ckel\index{H\"{u}ckel}
calculation (for molecules where all nuclear charges are
less than or equal to 36). If problems is experienced due to the
Huckel starting guess, it can be avoided by requiring another set of
starting orbitals (e.g. \verb|H1DIAG|).

%The default convergence threshold is quite sharp (compare with the
%default for MCSCF), this is done in order to have good orbitals
%for MP2 calculations.  For Hartree--Fock
%calculations with many basis functions
%which are not to be followed by MP2 or used for finite difference
%property calculations, some CPU time may be save by lowering the
%threshold to the minimum acceptable accuracy.

It is our experience that
it is usually most efficient not to perform any Roothaan Fock iterations
before DIIS is activated, therefore, MAXFCK = 0 as default.
The algorithm described in
Harrell Sellers, Int. J. Quant. Chem. {\bf 45}, 31-41 (1993) is
also implemented, and may be selected with \quotekw{\Key{C2DIIS}}.


H1VIRTUALS: This option can be used without a Hartree--Fock calculation
to obtain compact virtual orbitals, but \quotekw{\Key{HF OCCUPATION}} must be
specified anyway in order to identify the virtual orbitals to be transformed.

COREHOLE: Enable SCF
single core-hole\index{core hole} calculations. To perform
an SCF core hole calculation just add the \quotekw{\Key{COREHOLE}}
keyword to the input for the closed-shell RHF ground state
calculation, specifying from which orbital to remove an electron,
and provide the program with the ground state orbitals using the
appropriate \quotekw{\Key{MOSTART}} option (normally \kw{NEWORB}).
Note that this is different from the MCSCF version of
\quotekw{\Key{COREHOLE}} under \quotekw{\Sec{OPTIMIZATION}}
(p.~\pageref{ref-optinp}); in the MCSCF case the user must
explicitly move the core hole orbital from the inactive class to
RAS1 by modifying the \quotekw{\Sec{CONFIGURATION INPUT}}
(p.~\pageref{ref-wavinp}) specifications between the initial
calculation with filled core orbitals and the core hole
calculation. The core hole\index{core hole} orbital will be
frozen\index{frozen core hole} in the following optimization.
After this calculation has converged, the CORERELAX option may be
added and the core orbital will be relaxed\index{relaxed core hole}.  
When CORERELAX is specified it is assumed that the
calculation was preceded by a frozen core calculation, and that
the orbital has already been moved to the open shell orbital. Only
the main peak can be obtained in SCF calculations, for shake-up
energies MCSCF must be used.

\pagebreak[3]
\subsection{\label{ref-mp2inp}\Sec{MP2 INPUT}}

{\bf Purpose:}

\index{MP2}\index{M{\o}ller-Plesset!second-order}
To direct MP2 calculation. Note that MP2 energies as well as
properties also are available through the Coupled Cluster module, see
Chapter~\ref{ch:CC}.

\begin{description}
\item[\Key{MP2 FROZEN}]
  Default = no frozen orbitals\\
  \kw{READ (LUINP,*) (NFRMP2(I),I=1,NSYM)} \\
  Orbitals frozen in MP2 calculation

\item[\Key{PRINT}]
  \kw{READ (LUINP,*) IPRMP2} \\
  Print level for MP2 calculation
\end{description}


\noindent{\bf Comments:}


The MP2 module expects canonical Hartree--Fock
orbitals\index{canonical orbital}, the module will
check the orbitals and exit if the Fock matrix has off-diagonal non-zero
elements.

The MP2 calculation will produce the MP2 energy and the natural orbitals
for the density matrix through second order.  The primary purpose of
this option is to generate good starting orbitals for CI or MCSCF wave
functions, but
may of course also be used to obtain the MP2 energy, perhaps with frozen
core orbitals. {\em For valence MCSCF calculations it is recommended that the
\quotekw{\Key{MP2 FROZEN}} option is used in order to obtain the appropriate
correlating orbitals\index{correlating orbitals}\index{MCSCF} as start
for an MCSCF calculation.\/}  As the commonly
used basis sets do not contain correlating orbitals for the core
orbitals and as the core correlation energy therefore becomes arbitrary,
the \quotekw{\Key{MP2 FROZEN}} option can also be of benefit in MP2 energy
calculations.

\pagebreak[3]
\subsection{\label{ref-nevpt2inp}\Sec{NEVPT2 INPUT}}

{\bf Purpose:}

\index{NEVPT2}\index{multireference PT!second-order}
Calculation of the second order correction to the energy for a
CAS--SCF or CAS--CI zero order wavefunction.
The user is referred to Chapter~\ref{ch:nevpt2} at
page~\pageref{ch:nevpt2}  for a brief
introduction to the $n$--electron valence state second order
perturbation theory (NEVPT2).

\begin{description}
\item[\Key{THRESH}]
 Default = 0.0D0\\
  \kw{READ (LUINP,*) THRNEVPT} \\
  Threshold to discard small coefficients in the CAS wavefunction


\item[\Key{FROZEN}]
  Default = no frozen orbitals\\
  \kw{READ (LUINP,*) (NFRNEVPT2(I),I=1,NSYM)} \\
  Orbitals frozen in NEVPT2 calculation

\item[\Key{STATE}]
 No default provided\\
\kw{READ (LUINP,*) ISTNEVCI} \\
Root number in a CASCI calculation. This keyword is unnecessary
(ignored) in the CASSCF case.
\end{description}


\noindent{\bf Comments:}


%The present version of the NEVPT2 module requires the
%\quotekw{\Key{DETERMINANTS}} option  to be set.

The use of canonical orbitals for the core and virtual orbitals is
strongly recommended since this choice guarantees compliance of the
results with a totally invariant form of NEVPT2 (see page~\pageref{ch:nevpt2})

At present the NEVPT2 module can deal with active spaces of dimension
not higher than 14.


\pagebreak[3]
\subsection{\label{ref-optinp}\Sec{OPTIMIZATION}}

{\bf Purpose:}

To change defaults for optimization of an MCSCF\index{MCSCF} wave function.
Some of the options also affect a QC-HF optimization.

\begin{description}
\item[\Key{ABSORPTION}]
  \kw{READ (LUINP,'(A8)') RWORD} \\
  RWORD = ` LEVEL 1', ` LEVEL 2', or ` LEVEL 3'\\
  Orbital absorption\index{orbital absorption} in MCSCF optimization
  at level 1, 2, or 3, as specified
  (normally level 3, see comments below).  This keyword may be repeated to
  specify more than one absorption level, the program will then begin with
  the lowest level requested and, when that level is converged,
  disable the lower level and shift to the next level.
% 940816-hjaaj: The following may not be true for RAS ????
% Absorption at several levels are only useful in
% first macro iteration, therefore the lower levels are disabled after
% convergence.

\item[\Key{ACTROT}]
  include specified active-active rotations
\begin{verbatim}
  READ (LUINP,*) NWOPT
  DO I = 1,NWOPT
    READ (LUINP,*) JWOP(1,I),JWOP(2,I)
  END DO
\end{verbatim}
  JWOP(1:2,I) denotes normal molecular orbital numbers (not the active
  orbital numbers).

\item[\Key{ALWAYS ABSORPTION}]
  Absorption\index{orbital absorption} in all MCSCF macro iterations
  (default is to disable absorption in
  local region or after \quotekw{\Key{MAXABS}} macro iterations, whichever comes first).
  Absorption is always disabled after Newton-Raphson algorithm has been used,
  and thus also when doing \quotekw{\Key{CORERELAX}},
  because absorption may cause variational collapse if the desired state is excited.

\item[\Key{CI PHP MATRIX}]
  Default : MAXPHP = 1 (Davidson's algorithm)\\
  \kw{READ (LUINP,*) MAXPHP} \\
  PHP is a subblock of the CI matrix which is calculated explicitly
  in order to obtain improved CI trial vectors compared to the
  straight Davidson\index{Davidson algorithm} algorithm.  The
  configurations corresponding to
  the lowest diagonal elements are selected, unless
  \quotekw{\Key{PHPRESIDUAL}} is specified.
  \kw{MAXPHP} is the maximum dimension of PHP, the actual dimension
  will be less if \kw{MAXPHP} will split degenerate configurations.

\item[\Key{COREHOLE}]
  \kw{READ (LUINP,*) JCHSYM,JCHORB} \\
  JCHSYM = symmetry of core orbital\\
  JCHORB = the orbital in symmetry JCHSYM with a single core hole\\
  Single core hole\index{core hole} MCSCF calculation. The calculation must be of RAS type
  with only the single core-hole orbital in RAS1, the state specified with
  \quotekw{\Key{STATE}} is optimized with the core-hole orbital
  frozen\index{frozen core hole}.
  The specified core hole orbital must be either inactive or
  the one RAS1 orbital, if it is inactive then it will switch places with
  the RAS1 orbital and it will not be possible to also
  specify \quotekw{\Key{REORDER}}. If explicit reordering is required you must reorder
  the core orbital yourself and let \kw{JCHORB} point to the one RAS1 orbital.
  Orbital absorption is activated at level 2. See comments below for more information.

\item[\Key{CORERELAX}]
  (ignored if \quotekw{\Key{COREHOLE}} isn't also specified)\\
  Optimize state with relaxed core orbital\index{relaxed core hole} (using Newton-Raphson algorithm,
  it is not necessary to explicitly specify \quotekw{\Key{NR ALWAYS}}).
  It is assumed that this calculation follows an optimization
  with frozen core orbital and that the orbital has already been
  moved to the RAS1 space (i.e., the specific value of
  \quotekw{JCHORB} under \quotekw{\Key{COREHOLE}} is ignored). Any
  orbital absorption   will be ignored.

\item[\Key{DETERMINANTS}]
  Use determinant\index{determinants} basis instead of CSF basis (see comments).

\item[\Key{EXACTDIAGONAL}]
  Default for RAS calculations.\\
  Use the exact orbital Hessian\index{orbital Hessian} diagonal.

\item[\Key{FOCKDIAGONAL}]
  Default for CAS calculations.\\
  Use an approximate orbital Hessian diagonal which only uses Fock
  contributions.

\item[\Key{FOCKONLY}]
  Activate TRACI option (default : program decides).\\
  Modified TRACI option where all orbitals, also active orbitals, are
  transformed to Fock type orbitals in each iteration.

\item[\Key{MAX CI}]
  \kw{READ (LUINP,*) MAXCIT} \\
  maximum number of CI iterations before MCSCF (default = 3).

\item[\Key{MAX MACRO ITERATIONS}]
  \kw{READ (LUINP,*) MAXMAC} \\
  maximum number of macro iterations in MCSCF optimization (default = 15).

\item[\Key{MAX MICRO ITERATIONS}]
  \kw{READ (LUINP,*) MAXJT} \\
  maximum number of micro iterations per macro iteration in MCSCF optimization
  (default = 24).

\item[\Key{MAXABS}]
  \kw{READ (LUINP,*) MAXABS} \\
  maximum number of macro iterations with 
  absorption\index{orbital absorption} (default = 3).

\item[\Key{MAXAPM}]
  \kw{READ (LUINP,*) MAXAPM} \\
  maximum number orbital absorptions\index{orbital absorption} within
  a macro iteration
  (APM : Absorptions Per Macro iteration; default = 5)

\item[\Key{NATONLY}]
  Activate TRACI option (default : program decides).\\
  Modified TRACI option where the inactive and secondary orbitals are not
  touched (these two types of orbitals are already natural orbitals).

\item[\Key{NEO ALWAYS}]
  Always norm-extended optimization (never switch to New\-ton-Raph\-son).
  Note: \quotekw{\Key{NR ALWAYS}} and \quotekw{\Key{CORERELAX}}
  takes precedence over \quotekw{\Key{NEO ALWAYS}}.

\item[\Key{NO ABSORPTION}]
  Never orbital absorption\index{orbital absorption} (default settings removed)

\item[\Key{NO ACTIVE-ACTIVE ROTATIONS}]
  No active-active rotations in RAS optimization.

\item[\Key{NOTRACI}]
  Disable TRACI option (default : program decides).

\item[\Key{NR ALWAYS}]
  Always Newton-Raphson optimization (never NEO optimization).
  Note: \quotekw{\Key{NR ALWAYS}} takes precedence over
  \quotekw{\Key{NEO ALWAYS}}.

\item[\Key{OLSEN}]
  Use Jeppe Olsen's generalization of the Davidson
  algorithm\index{Davidson algorithm}.

\item[\Key{OPTIMAL ORBITAL TRIAL VECTORS}]
  Generate "optimal" orbital trial
  vectors~\cite{hjajpjhajcp87}.\index{optimal orbital trial vector} 

\item[\Key{ORB\_TRIAL VECTORS}]
  Use also orbital trial vectors as start vectors for auxiliary roots
  in each macro iteration (CI trial vectors are always generated).

\item[\Key{PHPRESIDUAL}]
  Select configurations for PHP matrix based on largest residual
  rather than lowest diagonal elements.

\item[\Key{SIMULTANEOUS ROOTS}]
  Default : NROOTS = ISTATE, LROOTS = NROOTS\\
  \kw{READ (LUINP,*) NROOTS, LROOTS} \\
  NROOTS = Number of simultaneous roots in NEO\\
  LROOTS = Number of simultaneous roots in NEO at start

\item[\Key{STATE}]
  Default = 1\\
  \kw{READ (LUINP,*) ISTATE} \\
  Index of MCSCF Hessian\index{MCSCF Hessian} at convergence (1 for
  lowest state, 2 for first
  excited state, etc. within the spatial symmetry\index{symmetry} and
  spin symmetry\index{spin symmetry}
  specified under \Sec{CONFIGURATION INPUT}).

\item[\Key{SYM CHECK}]
  Default: ICHECK = 2 when NROOTS $>$ 1, else ICHECK = -1.\\
  \kw{READ (LUINP,*) ICHECK} \\
  Check symmetry of the LROOTS start CI-vectors and remove those which
  have wrong symmetry (e.g. vectors of delta symmetry in a sigma
  symmetry calculation).
\begin{verbatim}
  ICHECK < 0  : No symmetry check.
  ICHECK = 1  : Remove those vectors which do not have the same
                symmetry as the ISTATE vector, reassign ISTATE.
  ICHECK = 2  : Remove those vectors which do not have the same
                symmetry as the lowest state vector before selecting
                the ISTATE vector.
  other values: check symmetry, do not remove any CI vectors.
\end{verbatim}
  The \quotekw{\Key{SIMULTANEOUS ROOTS}} input will automatically be
  updated if CI vectors are removed.

\item[\Key{THRCGR}]
  \kw{READ (LUINP,*) THRCGR} \\
  Threshold for print of CI gradient. Default is 0.1D0.

\item[\Key{THRESH}]
  Default = 1.0D-05\\
  \kw{READ (LUINP,*) THRMC} \\
  Convergence threshold for energy gradient in MCSCF optimization.
  The convergence of the energy will be approximately the square of this
  number.

\item[\Key{TRACI}]
  Activate TRACI option (default : program decides).\\
  Active orbitals are transformed to natural orbitals and the CI-vectors
  are counter-rotated such that the CI states do not change.  The
  inactive and secondary orbitals are transformed to Fock type orbitals
  (corresponding to canonical orbitals for closed shell Hartree--Fock).
  For RAS wave functions the active orbitals are only transformed
  within their own class (RAS1, RAS2, or RAS3) as the wave function is
  not invariant to orbital rotations between the classes.  For RAS, the
  orbitals are thus not true natural orbitals, the density matrix is
  only block diagonalized.  Use \quotekw{\Key{IPRCNO}} (see
  p.~\pageref{ref-priinp})   to control output from this
  transformation.

\end{description}


\noindent{\bf Comments:}

COREHOLE: Single core-hole\index{core hole} calculations are
performed as RAS calculations where the opened core orbital is in
the RAS1 space.  The RAS1 space must therefore contain one and
only one orbital when the COREHOLE option is used, and the
occupation must be restricted to be exactly one electron. The
orbital identified as the core orbital must be either inactive or
the one RAS1 orbital, if it is inactive it will switch places with
the one RAS1 orbital. The core orbital (now in RAS1) will be
frozen in the following optimization. After this calculation has
converged, the CORERELAX option may be added and the core orbital
will be relaxed\index{relaxed core hole}.  When CORERELAX is
specified it is assumed that the calculation was preceded by a
frozen core\index{frozen core hole} calculation, and that the
orbital has already been moved to the RAS1 space. Default
corresponds to the main peak, shake-up energies may be obtained by
specifying \quotekw{\Key{STATE}} larger than one. Absorption is
very beneficial in core hole calculations because of the large
orbital relaxation following the opening of the core hole.

ABSORPTION: Absorption\index{orbital absorption} level 1 includes occupied - occupied rotations
only (including active-active rotations); level 2 adds inactive -
secondary rotations and only active - secondary rotations are excluded
at this level; and finally level 3 includes all non-redundant rotation
for the frozen CI vector.  Levels 1 and 2 require the same integral
transformation (because the inactive - secondary rotations are
performed using the P-supermatrix integrals) and level 1 is therefore
usually not used. Level 3 is the normal and full level, but it can be
advantageous to activate level 2 together with level 3 if big
inactive-active or occupied-occupied rotations are expected.

ORB\_TRIAL: Orbital trial\index{orbital trial vector}\ vectors as
start vectors can be used for
excited states and other calculations with more than one simultaneous
roots.  The orbital start trial vectors are based on the eigenvectors of
the NEO matrix in the previous macro iterations.  However, they are
probably not cost-effective for multiconfiguration calculations where
optimal orbital trial\index{optimal orbital trial vector} vectors are
used and they are therefore not used
by default.

SYM CHECK: The symmetry check is performed on the matrix element
$\langle VEC1 \mid oper \mid VEC2\rangle$, where "oper" is
the CI-diagonal.
It is recommended and the default to use \quotekw{\Key{SYM CHECK}}
for excited states, including
CI vectors of undesired symmetries is a waste of CPU time.

DETERMINANTS: The kernels of the CI sigma routines and density matrix
routines are always performed in determinant\index{determinants}
basis.  However, this
keyword specifies that the external representation is Slater
determinants as well.  The default is that the external representation
is in CSF\index{CSF}\index{configuration state function} basis as
described in chapter 8 of MOTECC-90.  The external
CSF\index{CSF}\index{configuration state function} basis is
generally to be preferred to be sure that the converged
state(s) have pure and correct spin symmetry\index{spin symmetry}, and
to save disk space.
It is recommended to specify \quotekw{\Key{PLUS COMBINATIONS}} under
\quotekw{\Sec{CI VECTOR}} for
calculations on singlet states\index{singlet state} with
determinants\index{determinants},
in particular for
excited singlet\index{excited state} states which often have lower lying triplet states.


\pagebreak[3]
\subsection{\label{ref-orbinp}\Sec{ORBITAL INPUT}}

{\bf Purpose:}

To define an initial set of molecular orbitals\index{molecular orbital}
and to control the use of supersymmetry\index{supersymmetry}, frozen
orbitals\index{frozen orbitals}, deletion of orbitals\index{delete orbitals},
reordering and punching of orbitals.

\begin{description}
\item[\Key{5D7F9G}]
  Delete unwanted components in Cartesian d, f, and g orbitals.
  (s in d; p in f; s and d in g). By default, \her\ provides atomic
  integrals in spherical basis, and this option should therefore not
  be needed.

\item[\Key{AO DELETE}]
  \kw{READ (LUINP,*) THROVL } \\
  Delete MO's based on canonical orthonormalization using eigenvalues
  and eigenvectors of the AO overlap matrix.\index{linear dependence} \\
  THROVL: limit for basis
  set numerical linear dependence (eigenvectors with eigenvalue less
  than THROVL are excluded). Default is 1.0$\cdot$10$^{-6}$.

%\item[\Key{AVERAGE}]
%  Default = no average (\kw{KAVER = 0})\\
%  \kw{READ (LUINP,*) KAVER,(KSYM(I),I=1,4)} \\
%  Old option for averaging degenerate symmetries in $D_{2h}$, it is
%  generally recommended to use the new SUPSYM feature.
%  If you want to use \quotekw{\Key{AVERAGE}} the SUPSYM
%  feature must be disabled with \quotekw{\Key{NOSUPSYM}}.
%\begin{verbatim}
%  KAVER=1: double degeneracy, average KSYM(1) and KSYM(2)
%  KAVER=2: a) KSYM(1)=KSYM(3): triple degeneracy, average
%              KSYM(1), KSYM(2), KSYM(4)
%           b) else: two double degeneracies,
%              average KSYM(1) and KSYM(2), and
%              average KSYM(3) and KSYM(4).
%  KAVER .ne. 0 : MO coefficients are copied from the KSYM(1) and
%           KSYM(3) to the other degenerate symmetries to avoid
%           different orbitals through numerical roundoff errors.
%\end{verbatim}

\item[\Key{CMOMAX}]
  \kw{READ (LUINP,*) CMAXMO} \\
  Abort calculation if the absolute value of any initial MO coefficient is
  greater than CMAXMO (default : CMAXMO = 10**4).  Large MO coefficients
  can cause significant loss of accuracy in the two-electron integral
  transformation.

\item[\Key{DELETE}]
  \kw{READ (LUINP,*) (NDEL(I),I = 1,NSYM) } \\
  Delete orbitals\index{delete orbitals}. The delete starts from the
  last orbital of the symmetry and counts downward.

\item[\Key{FREEZE}]
  Default: no frozen orbitals
\begin{verbatim}
  READ (LUINP,*) (NNOR(ISYM), ISYM = 1,NSYM)
  DO ISYM = 1,NSYM
    IF (NNOR(ISYM) .GT. 0) THEN
      READ (LUINP,*) (INOROT(I), I = 1,NNOR(ISYM))
      ...
    END IF
  END DO
\end{verbatim}
  where \kw{INOROT} = orbital numbers of the orbitals to be
          frozen\index{frozen orbitals}
          (not rotated) in symmetry \quotekw{ISYM}
          after any reordering (counting from 1 in each symmetry).\\
  Must be specified after all options reducing the number of orbitals.

\item[\Key{FROZEN CORE ORBITALS}]
  \kw{READ (LUINP,*) (NFRO(I),I=1,NSYM)} \\
  Frozen orbitals : Inactive orbitals to be frozen.

\item[\Key{GRAM-SCHMIDT ORTHONORMALIZATION}]
  Default.\\
  Gram--Schmidt orthonormalization\index{orthonormalization!Gram--Schmidt} of input orbitals.

\item[\Key{MOSTART}]
   Molecular orbital input\index{molecular orbital}\\
   \kw{READ (LUINP,'(1X,A6)') RWORD} \\
   where RWORD is one of the following:
   \begin{description}
   \item[{\tt FORM12\ }] Formatted input (6F12.8)  supplied after
        \Sec{*MOLORB} or \Sec{*NATORB} keyword
   \item[{\tt FORM18\ }] Formatted input (4F18.14) supplied after
        \Sec{*MOLORB} or \Sec{*NATORB} keyword
   \item[{\tt H1DIAG\ }] Start orbitals that diagonalize
        one-electron Hamiltonian matrix (default
        for molecules containing elements with a nuclear larger than 36).
   \item[{\tt HUCKEL\ }] Start orbitals generated by projecting the
        H{\"u}ckel eigenvectors in a good generally contracted ANO basis set
        onto the present basis set
        (default for molecules in which all atoms have a nuclear charge less than
        or equal to 36; Note: HUCKEL is not implemented yet if any element has a
        charge larger than 36).
        The start density will thus be close to one generated from atomic densities,
        but with molecular valence interaction in the H{\"u}ckel model.
        This works a lot better than using a minimal basis set for H{\"u}ckel.
   \item[{\tt NEWORB\ }] Input from {\sir} restart file
                            (\verb|SIRIUS.RST| file) with label \quotekw{NEWORB  }
   \item[{\tt OLDORB\ }] Input from {\sir} restart file
                            (\verb|SIRIUS.RST| file) with label \quotekw{OLDORB  }
   \item[{\tt SIRIFC\ }] Input from {\sir} interface file ("SIRIFC")
   \end{description}

\item[\Key{NOSUPSYM}]
  Deactivate automatic identification of "super
  symmetry"\index{supersymmetry} (see
  comments). This is automatically enforced in case of \aba\ or
  \resp\ calculations.

\item[\Key{PUNCHINPUTORBITALS}]
  Punch input orbitals with label \Sec{*MOLORB}, Format (4F18.14)

\item[\Key{PUNCHOUTPUTORBITALS}]
  Punch final orbitals with label \Sec{*MOLORB}, Format (4F18.14)

\item[\Key{REORDER}]
Default: no reordering.
\begin{verbatim}
  READ (LUINP,*) (NREOR(I), I = 1,NSYM)
  DO I = 1,NSYM
     IF (NREOR(I) .GT. 0) THEN
        READ (LUINP,*) (IMONEW(J,I), IMOOLD(J,I), J = 1,NREOR(I))
     END IF
  END DO
  NREOR(I) = number of orbitals to be reordered in symmetry I
  IMONEW(J,I), IMOOLD(J,I) are orbital numbers in symmetry I.

For example if orbitals 1 and 5 in symmetry 1 should change place, specify
.REORDER
 2 0 0 0
 1 5 5 1
\end{verbatim}
  Reordering of molecular orbitals (see comments).

\item[\Key{SUPSYM}]
  Default.\\
  Enforce automatic identification of "super
  symmetry"\index{supersymmetry} (see comments).

\item[\Key{SYMMETRIC ORTHONORMALIZATION}]
  Default: Gram-Schmidt orthonormalization\\
  Symmetric orthonormalization of input
  orbitals\index{orthonormalization!symmetric}.

\item[\Key{THRSSY}]
  \kw{READ (LUINP,*) THRSSY} \\
  Threshold for identification of "super
  symmetry"\index{supersymmetry} and degeneracies among
  "super symmetries" from matrix elements of the kinetic energy matrix
  (default: 5.0D-8).

\end{description}


\noindent{\bf Comments:}

{\sir} automatically identifies "super
symmetry"\index{supersymmetry}, {\it i.e.\/} of irreps of the true point
group of the molecule\index{symmetry!group} which is a
"supergroup" of the Abelian group used in the calculation.
Degenerate orbitals will be averaged and the "super
symmetry"\index{supersymmetry} will be enforced in the orbitals.
The use of "super symmetry" may be deactivated with the
\quotekw{\Key{NOSUPSYM}} keyword, for example in finite field
calculations where the field lowers the symmetry.
%\ifabacus
For
\aba\ and \resp\ calculations, the "super symmetry" is deactivated
unless explicitly enforced with \Key{SUPSYM}.
%\fi
The initial
orbitals must be symmetry orbitals, and the super symmetry
analysis is performed on the kinetic energy matrix in this basis.
The \quotekw{.THRSSY} option is used to define when the kinetic
energy matrix element between two orbitals is considered to be
zero and when two diagonal matrix elements are degenerate. In the
first case the orbitals can belong to different irreps of the
supergroup and in the second case the two orbitals are considered
to be degenerate. The analysis will fail if there are accidental
degeneracies in diagonal elements.  This can happen if the nuclear
geometry deviates slightly from a higher symmetry point group, for
example because too few digits has been used in the input of the
nuclear geometry. If the program stops because the super symmetry
analysis fails with a degeneracy error, you might consider to use
more digits in the nuclear coordinates, to change \kw{THRSSY}, or
to disable super symmetry with \quotekw{.NOSUPSYM}.  The value of
\kw{THRSSY} should be sufficiently small to avoid accidental
degeneracies and sufficiently large to ignore small errors in
geometry and numerical round-off errors.


\Key{REORDER}\index{orbital reordering} can for instance be used for
linear molecules to interchange
undesired delta orbitals among the active orbitals in symmetry 1 with
sigma orbitals.  Another example is movement of the core orbital to the
RAS1 space for core hole calculation.  In general, use of this option
necessitates a pre-calculation with STOP AFTER MO-ORTHONORMALIZATION and
identification of the various orbitals by inspection of the output.


\pagebreak[3]
\subsection{\label{ref-popinp}\Sec{POPULATION ANALYSIS}}

{\bf Purpose:}

To direct population analysis\index{population analysis} of the wave function.
Requires a set of natural orbitals\index{natural orbital} and their occupation.

\begin{description}
\item[\Key{ALL}]
  Do all options.

\item[\Key{DIPMOM}]
  Calculate dipole moments. Note that this requires that the dipole
  length integrals are available on the file \verb|AOPROPER|.\index{dipole moment}

\item[\Key{GROSSALL}]
  Do all gross population analysis. Note that this requires that the dipole
  length integrals are available on the file \verb|AOPROPER|\index{population analysis}

\item[\Key{GROSSMO}]
  Do gross MO population analysis.\index{population analysis}

\item[\Key{MULLIKEN}]
  Do Mulliken population analysis\index{population analysis}\index{population analysis!Mulliken}\index{Mulliken population analysis}

\item[\Key{NETALL}]
  Do all net population analysis.\index{population analysis}

\item[\Key{NETMO}]
  Do net MO population analysis.\index{population analysis}

\item[\Key{PRINT}]
  Default = 1\\
  \kw{READ (LUINP,*) IPRMUL} \\
  Print level for population analysis.

\item[\Key{QUADRP}]
  Calculate quadrupole moments. Note that this requires that the dipole
  length integrals are available on the file \verb|AOPROPER|\index{quadrupole moment}

\item[\Key{VIRIAL}]\index{virial analysis}
  Do virial analysis.
\end{description}

\pagebreak[3]
\subsection{\label{ref-priinp}\Sec{PRINT LEVELS}}

{\bf Purpose:}

To control the printing of output.

\begin{description}
\item[\Key{CANONI}]
  Generate canonical/natural orbitals if the wave function has
  converged\index{canonical orbital}\index{natural orbital}.

\item[\Key{IPRAVE}]
  \kw{READ (LUINP,*) IPRAVE} \\
  Sets print level for routines used in "super symmetry" averaging
  (default = 0).

\item[\Key{IPRCIX}]
  \kw{READ (LUINP,*) IPRCIX} \\
  Sets print level for setup of determinant/CSF index information (default = 0).

\item[\Key{IPRCNO}]
  \kw{READ (LUINP,*) IPRCNO} \\
  Sets print level for \quotekw{.TRACI} option (default = 1,
  to print the natural orbital occupations in each iteration set
  IPRCNO = 1, higher values will give more print).

\item[\Key{IPRDIA}]
  \kw{READ (LUINP,*) IPRDIA} \\
  Sets print level for calculation of CI diagonal (default = 0)

\item[\Key{IPRDNS}]
  \kw{READ (LUINP,*) IPRDNS} \\
  Sets print level for calculation of CI density matrices (default = 0)

%\item[\Key{IPRERR}]
%  \kw{READ (LUINP,*) IPRERR} \\
%  Sets print level for statistics in error file, LUERR (default = 1)

\item[\Key{IPRFCK}]
  \kw{READ (LUINP,*) IPRFCK} \\
  Sets print level in the supersymmetry section (default=0).

\item[\Key{IPRKAP}]
  \kw{READ (LUINP,*) IPRKAP} \\
  Sets print level in routines for calculation of optimal orbital trial
  vectors (default = 0)

\item[\Key{IPRSIG}]
  \kw{READ (LUINP,*) IPRSIG} \\
  Sets print level for calculation of CI sigma vectors (default = 0)

\item[\Key{IPRSOL}]
  \kw{READ (LUINP,*) IPRSOL} \\
  Sets print level in the solvent contribution parts of the
  calculation (default = 5).

\item[\Key{NOSUMMARY}]
  No final summary of calculation.

\item[\Key{PRINTFLAGS}]
 Default: flags set by general levels in \quotekw{\Key{PRINTLEVELS}}
\begin{verbatim}
  READ (LUINP,*) NUM6, NUM4
  IF (NUM6 .GT. 0) READ (LUINP,*) (NP6PTH(I), I=1,NUM6)
  IF (NUM4 .GT. 0) READ (LUINP,*) (NP4PTH(I), I=1,NUM4)
\end{verbatim}
  Individual print flag settings (debug option).

\item[\Key{PRINTLEVELS}]
  Default: IPRI6 = 0 and IPRI4 = 5 \\
  \kw{READ (LUINP,*) IPRI6,IPRI4 } \\
  Print levels on units LUW6 and LUW4, respectively.
%
%\item[\Key{PRINTUNITS}]
%  \kw{READ (LUINP,*) LUW6,LUW4 } \\
%  Unit numbers for general output and summary output, respectively
%  (default: LUW4 = 6 and LUW6 = 6).
%
\item[\Key{THRPWF}]
  \kw{READ (LUINP,*) THRPWF} \\
  Threshold for printout of wave function CI coefficients (default = 0.05).
 \end{description}



%\ifsolvent
\pagebreak[3]
\subsection{\label{ref-solinp}\Sec{SOLVENT}}

{\bf Purpose:}

Model solvent effects with the self-consistent
reaction\index{reaction field} field model.
Any specification of dielectric constant(s)\index{dielectric constant}
will activate this model.

\begin{description}
\item[\Key{CAVITY}]
  Required, no defaults.\\
  \kw{READ (LUINP,*) RSOLAV}\\
  Enter radius of spherical cavity\index{cavity!radius} in atomic units (bohr).

\item[\Key{DIELECTRIC CONSTANT}]
  \kw{READ (LUINP,*) EPSOL}\\
  Enter relevant dielectric constant\index{dielectric constant} of solvent.

\item[\Key{INERSINITIAL}]
  \kw{READ (LUINP,*) EPSOL}\\
  Enter static dielectric constant\index{dielectric constant} of solvent for calculation
  of the initial state defining inertial polarization\index{inertial polarization}.

\item[\Key{INERSFINAL}]
  \kw{READ (LUINP,*) EPSTAT,EPSOL}\\
  Enter static and optical dielectric\index{dielectric constant} constants of solvent
  for calculation of the final state with inertial polarization
  from a previous calculation with \quotekw{\Key{INERSINITIAL}}\index{final polarization}. 
  This keyword should also be used to specify the
  static and optical dielectric constants for non-equilibrium
  solvation linear, quadratic or cubic response functions, see also Sec.~\ref{sec:solvnoneqrsp}.

\item[\Key{MAX L}]
  Required, no defaults.\\
  \kw{READ (LUINP,*) LSOLMX}\\
  Enter maximum L quantum number in multipole expansion of charge
  distribution in cavity.

\item[\Key{PRINT}]
  \kw{READ (LUINP,*) IPRSOL} \\
  Print level in solvent module routines (default = 0).
\end{description}

\noindent{\bf Comments:}

One and only one of \quotekw{\Key{DIELECTRIC CONSTANT}},
\quotekw{\Key{INERSINITIAL}}, and \quotekw{\Key{INERSFINAL}} must be
specified.
%\fi % end of \ifsolvent



\pagebreak[3]
\subsection{\label{ref-stpinp}\Sec{STEP CONTROL}}

{\bf Purpose:}

User control of the NEO restricted step optimization.

\begin{description}
\item[\Key{DAMPING FACTOR}]
  Default = 1.0D0\\
  \kw{READ (LUINP,*) BETA} \\
  Initial value of damping (BETA).\index{damping}

\item[\Key{DECREMENT FACTOR}]
  Default = 0.67D0\\
  \kw{READ (LUINP,*) STPRED} \\
  Decrement factor on trust radius\index{trust radius}

\item[\Key{GOOD RATIO}]
  Default = 0.8D0 \\
  \kw{READ (LUINP,*) RATGOD} \\
  Threshold ratio for good second order agreement: the trust radius can
  be increased if ratio is better than RATGOD.

\item[\Key{INCREMENT FACTOR}]
  Default = 1.2D0\\
  \kw{READ (LUINP,*) STPINC} \\
  Increment factor on trust radius.\index{trust radius}

\item[\Key{MAX DAMPING}]
  Default = 1.0D6\\
  \kw{READ (LUINP,*) BETMAX} \\
  Maximum damping value.\index{damping}

\item[\Key{MAX STEP LENGTH}]
  Default = 0.7\\
  \kw{READ (LUINP,*) STPMAX} \\
  Maximum acceptable step length, trust radius will never be larger than
  STPMAX even if the ratio is good as defined by GOOD RATIO.

\item[\Key{MIN DAMPING}]
  Default = 0.2\\
  \kw{READ (LUINP,*) BETMIN} \\
  Minimum damping value

\item[\Key{MIN RATIO}]
  Default = 0.4 for ground state, 0.6 for excited states\\
  \kw{READ (LUINP,*) RATMIN} \\
  Threshold ratio for bad second order agreement: the trust radius is
  to be decreased if ratio is worse than RATMIN.

\item[\Key{NO EXTRA TERMINATION TESTS}]
  Skip extra termination tests and converge micro iterations to
  threshold.   Normally the micro iterations are terminated if the
  reduced NEO matrix has more negative eigenvalues than corresponding
  to the desired state, because then we are in a "superglobal" region
  and we just want to step as quickly as possible to the region where
  the Hessian (and NEO matrix) has the correct structure.  Further
  convergence is usually wasted.

\item[\Key{REJECT THRESHOLD}]
  Default = 0.25 for ground state, 0.4 for excited states\\
  \kw{READ (LUINP,*) RATREJ} \\
  Threshold ratio for unacceptable second order agreement: the step
  must be rejected if ratio is worse than RATREJ.

\item[\Key{THQKVA}]
  Default: 8.0 for MCSCF; 0.8 for QCHF\\
  \kw{READ (LUINP,*) THQKVA} \\
  Convergence factor for micro iterations in local (quadratic) region:
  THQKVA*(norm of gradient)**2

\item[\Key{THQLIN}]
  Default: 0.2\\
  \kw{READ (LUINP,*) THQLIN} \\
  Convergence factor for micro iterations in global (linear) region: \\
  THQLIN*(norm of gradient)

\item[\Key{THQMIN}]
  Default: 0.1\\
  \kw{READ (LUINP,*) THQMIN} \\
  Convergence threshold for auxilliary roots in NEO MCSCF optimization.

\item[\Key{TIGHT STEP CONTROL}]
  Tight step control also for ground state calculations
  (tight step control is always enforced for excited states)

\item[\Key{TOLERANCE}]
  Default = 1.1D0\\
  \kw{READ (LUINP,*) RTTOL} \\
  Acceptable tolerance in deviation of actual step from trust radius
  (the default value of 1.1 corresponds to a maximum of 10\% deviation).

\item[\Key{TRUST RADIUS}]
  Default = STPMAX=0.7D0 or, if restart, trust radius determined by previous
            iteration.\index{trust radius}\\
  \kw{READ (LUINP,*) RTRUST} \\
  Initial trust radius.

\end{description}


\pagebreak[3]
\subsection{\label{ref-trainp}\Sec{TRANSFORMATION}}

{\bf Purpose:}

Transformation\index{integral transformation} of two-electron
integrals\index{two-electron integral} to molecular orbital
basis\index{molecular orbital}.

\begin{description}
\item[\Key{FINAL LEVEL}]
  \kw{READ (LUINP,*) ITRFIN} \\
  Final integral transformation\index{integral transformation} level (only active if the keyword
  \quotekw{\Key{INTERFACE}} has been specified, or this is an \aba\ or
  \resp\ calculation.

\item[\Key{LEVEL}]
  \kw{READ (LUINP,*) ITRLVL} \\
  Integral transformation level (see comments).

\item[\Key{OLD TRANSFORMATION}]
  Use existing transformed integrals

\item[\Key{PRESORT}]
Indicates that the transformed two-electron integrals are already
present generated through the keyword \Key{SORT I} in the
\Sec{*INTEGRALS} input module. This keyword is true if
\quotekw{PRESORT} has been set in the general input module.

\item[\Key{PRINT}]
  \kw{READ (LUINP,*) IPRTRA} \\
  Print level in integral transformation module

\item[\Key{RESIDENT MEMORY}]
  \kw{READ (LUINP,*) MWORK} \\
  On virtual memory computers, the transformation will run more
  efficiently if it can be kept within the possible resident memory
  size: the real memory size.  {\sir} will attempt to only use MWORK
  double precision words in the transformation.
\end{description}


\noindent{\bf Comments:}

There are several types of integral transformations which may be
specified by the two transformation level keywords.
\begin{itemize}
   \item[0:] CI calculations, MCSCF gradient (default if CI, but
             no MCSCF specified).
             One index all orbitals, three indices only active
             orbitals.

   \item[1:] Obsolete, do not use.

   \item[2:] Default for MCSCF optimization. All integrals needed for {\sir}
             second-order MCSCF optimization, including the integrals
             needed to explicitly construct the diagonal of the orbital
             hessian. Two indices occupied orbitals, two indices all
             orbitals, with some reduction for inactive indices.
             Both (cd/ab) and (ab/cd) are stored.

   \item[3:] Same integrals as 2, including also the (ii/aa) and
             (ia/ia) integrals for exact inactive-secondary diagonal elements
             of the orbitals Hessian.

   \item[4:] All integrals with minimum two occupied indices.

   \item[5:] 3 general indices, one occupied index.  Required for MP2
             natural orbital analysis (the MP2 module automatically
             performs an integral transformation of this level).

  \item[10:] Full transformation.
\end{itemize}


\pagebreak[3]
\section{\label{sec:ref-molorbinp} \Sec{*MOLORB} input module}

If formatted input of the molecular orbitals has been specified in
the \Sec{ORBITAL INPUT} section, then {\sir} will attempt to find
the two-star label "\verb|**MOLORB|" in the input file and read
the orbital coefficients from the lines following this label.
