
%%%%%%%%%%%%%%%%%%%%%%%%%%%%%%%%%%%%%%%%%%%%%%%%%%%%%%%%%%%%%%%%%%%
\section{Linear response functions: \Sec{CCLR}}\label{sec:cclr}
%%%%%%%%%%%%%%%%%%%%%%%%%%%%%%%%%%%%%%%%%%%%%%%%%%%%%%%%%%%%%%%%%%%
\index{linear response}
\index{polarizabilities, frequency-dependent}
\index{polarizabilities, coupled cluster}
\index{polarizabilities, static}
\index{dipole polarizability}
\index{dispersion coefficients}
\index{Cauchy moments}

In the \Sec{CCLR  } section the input that is
specific for coupled cluster linear response properties is read in. 
This section includes presently 
\begin{itemize}
\item frequency-dependent linear response properties 
      $\alpha_{AB}(\omega)  = - \langle\langle A; B \rangle\rangle_\omega$
      where $A$ and $B$ can be any of the one-electron
      operators for which integrals are available in the 
      \Sec{*INTEGRALS} input part.
\item dispersion coefficients $D_{AB}(n)$ for $\alpha_{AB}(\omega)$
      which for $n \ge 0$ are defined by the expansion
      $$ \alpha_{AB}(\omega) = \sum_{n=0}^{\infty} \omega^n \, D_{AB}(n) $$
      In addition to the dispersion coefficients for $n \ge 0$
      there are also coefficients available for $ n = -1, \ldots, -4$,
      which are related to the Cauchy moments
       by $ D_{AB}(n) = S_{AB}(-n-2)$.
      \\
      Note, that for real response functions only even moments
      $D_{AB}(2n) = S_{AB}(-2n-2)$ with $n \ge -2$ are available,
      while for imaginary response functions only odd moments
      $D_{AB}(2n+1) = S_{AB}(-2n-3)$ with $n \ge -2$ are available.
\end{itemize}
Coupled cluster linear response functions and dispersion coefficients
are implemented for the models CCS, CC2 and CCSD. 
%Publications that report results obtained with CC linear response
%calculations should cite Ref.\ \cite{Christiansen:CCLR}. 
The theoretical background for the implementation is detailed in Ref.\ \cite{Christiansen:CCLR,Christiansen:QEL,Haettig:CAUCHY}.
%For dispersion coefficients also a citation of Ref.\ \cite{Haettig:CAUCHY} 
%should be included.
The properties calculated are in the approach now generally known as coupled cluster 
linear response---in the frequency-independent limit this coincides with the so-called 
orbital-unrelaxed energy derivatives (and thus the orbital-unrelaxed finite field result).

\begin{center}
\fbox{
\parbox[h][\height][l]{12cm}{
\small
\noindent
{\bf Reference literature:}
\begin{list}{}{}
\item Linear response:  O.~Christiansen, A.~Halkier, H.~Koch, P.~J{\o}rgensen, and T.~Helgaker \newblock {\em J.~Chem.~Phys.}, {\bf 108},\hspace{0.25em}2801, (1998).
\item Dispersion coefficients: C.~H\"{a}ttig, O.~Christiansen, and P.~J{\o}rgensen \newblock {\em J.~Chem.~Phys.}, {\bf 107},\hspace{0.25em}10592, (1997).
\end{list}
}}
\end{center}

\begin{description}
\item[\Key{ASYMSD}] 
Use an asymmetric formulation of the linear response function which
does not require the solution of response equations for the operators $A$, 
but solves two sets of response equations for the operators $B$.
%
% \item[\Key{RELAXE}] 
%
% \item[\Key{UNRELA}] 
%
\item[\Key{AVERAG}] \verb| |\newline
   \verb|READ (LUCMD,'(A)') AVERAGE|\newline
   \verb|READ (LUCMD,'(A)') SYMMETRY|

Evaluate special tensor averages of linear response functions.
Presently implemented are the isotropic average of the dipole polarizability
$\bar{\alpha}$ and the dipole polarizability anisotropy $\alpha_{ani}$.
Specify \verb+ALPHA_ISO+ for \verb+AVERAGE+ to obtain $\bar{\alpha}$ and
\verb+ALPHA_ANI+ to obtain $\alpha_{ani}$ and $\bar{\alpha}$.
The \verb+SYMMETRY+ input defines the selection rules that can be
exploited to reduce the number of tensor elements that have to be
evaluated. Available options are
\verb+ATOM+, \verb+SPHTOP+ (spherical top), \verb+LINEAR+,
\verb+XYDEGN+ ($x$- and $y$-axis equivalent, i.e.\ a $C_z^n$
symmetry axis with $n \ge 3$),  and \verb+GENER+ (use point
group symmetry from geometry input).
\index{dipole polarizability}
 
\item[\Key{DIPOLE}] 
Evaluate all symmetry allowed elements of the dipole polarizability
(max. 6 components).
\index{dipole polarizability}

\item[\Key{DISPCF}] \verb| |\newline
   \verb|READ (LUCMD,*) NLRDSPE|

   Calculate the dispersion coefficients 
   $D_{AB}(n)$ up to $n = $ \verb+NLRDSPE+.
%
\item[\Key{FREQUE}] \verb| |\newline
   \verb|READ (LUCMD,*) NBLRFR |\newline
   \verb|READ (LUCMD,*) (BLRFR(I),I=1,NBLRFR)|

Frequency input for $\langle\langle A;B \rangle\rangle_{\omega}$.
 
%
%\item[\Key{ALLDSP}] 
%
% \item[\Key{FTST  }] 
%
\item[\Key{OPERAT}] \verb| |\newline
   \verb|READ (LUCMD,'(2A)') LABELA, LABELB|\newline
   \verb|DO WHILE (LABELA(1:1).NE.'.' .AND. LABELA(1:1).NE.'*')|\newline
   \verb|  READ (LUCMD,'(2A)') LABELA, LABELB|\newline
   \verb|END DO|

Read pairs of operator labels. 
 
\item[\Key{PRINT }] \verb| |\newline
   \verb|READ (LUCMD,*) IPRSOP|

   Set print level for linear response output.
 
\end{description}
