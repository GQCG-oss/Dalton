
%%%%%%%%%%%%%%%%%%%%%%%%%%%%%%%%%%%%%%%%%%%%%%%%%%%%%%%%%%%%%%%%%%%
\section{Linear response functions: \Sec{CCLR}}\label{sec:cclr}
%%%%%%%%%%%%%%%%%%%%%%%%%%%%%%%%%%%%%%%%%%%%%%%%%%%%%%%%%%%%%%%%%%%
\index{linear response}
\index{response!linear}
\index{Coupled Cluster!linear response}
\index{polarizability!frequency-dependent}
\index{polarizability!Coupled Cluster}
\index{polarizability!static}
\index{dipole polarizability}
\index{dispersion coefficients}
\index{Cauchy moments}
\index{optical rotation!Coupled Cluster}
\index{dipole gradient!Coupled Cluster}
\index{magnetizabilty!Coupled Cluster}
\index{nuclear shielding!Coupled Cluster}

In the \Sec{CCLR} section the input that is
specific for coupled cluster linear response properties is read in. 
This section includes presently 
\begin{itemize}
\item frequency-dependent linear response properties 
      $\alpha_{AB}(\omega)  = - \langle\langle A; B \rangle\rangle_\omega$
      where $A$ and $B$ can be any of the one-electron
      operators for which integrals are available in the 
      \Sec{*INTEGRALS} input part.
\item dispersion coefficients $D_{AB}(n)$ for $\alpha_{AB}(\omega)$
      which for $n \ge 0$ are defined by the expansion
      $$ \alpha_{AB}(\omega) = \sum_{n=0}^{\infty} \omega^n \, D_{AB}(n) $$
      In addition to the dispersion coefficients for $n \ge 0$
      there are also coefficients available for $ n = -1, \ldots, -4$,
      which are related to the Cauchy moments
       by $ D_{AB}(n) = S_{AB}(-n-2)$.
      \\
      Note, that for real response functions only even moments
      $D_{AB}(2n) = S_{AB}(-2n-2)$ with $n \ge -2$ are available,
      while for imaginary response functions only odd moments
      $D_{AB}(2n+1) = S_{AB}(-2n-3)$ with $n \ge -2$ are available.
\end{itemize}

Coupled cluster linear response functions and dispersion coefficients
are implemented for the models CCS, CC2 and CCSD. 
%Publications that report results obtained with CC linear response
%calculations should cite Ref.\ \cite{Christiansen:CCLR}. 
The theoretical background for the implementation is detailed in Ref.\ \cite{Christiansen:CCLR,Christiansen:QEL,Haettig:CAUCHY,Haettig:DIPGRA}.
%For dispersion coefficients also a citation of Ref.\ \cite{Haettig:CAUCHY} 
%should be included.
The properties calculated are in the approach now generally known as coupled cluster 
linear response---in the frequency-independent limit this coincides with the so-called 
orbital-unrelaxed energy derivatives (and thus the orbital-unrelaxed finite field result).

\begin{center}
\fbox{
\parbox[h][\height][l]{12cm}{
\small
\noindent
{\bf Reference literature:}
\begin{list}{}{}
\item Linear response:  O.~Christiansen, A.~Halkier, H.~Koch, P.~J{\o}rgensen, and T.~Helgaker \newblock {\em J.~Chem.~Phys.}, {\bf 108},\hspace{0.25em}2801, (1998).
\item Dispersion coefficients: C.~H\"{a}ttig, O.~Christiansen, and P.~J{\o}rgensen \newblock {\em J.~Chem.~Phys.}, {\bf 107},\hspace{0.25em}10592, (1997).
\end{list}
}}
\end{center}

\begin{description}
\item[\Key{ASYMSD}] 
Use an asymmetric formulation of the linear response function which
does not require the solution of response equations for the operators $A$, 
but solves two sets of response equations for the operators $B$.
%
% \item[\Key{RELAXE}] 
%
% \item[\Key{UNRELA}] 
%
\item[\Key{AVERAG}] \verb| |\newline
   \verb|READ (LUCMD,'(A)') AVERAGE|\newline
   \verb|READ (LUCMD,'(A)') SYMMETRY|

Evaluate special tensor averages of linear response functions.
Presently implemented are the isotropic average of the dipole polarizability
$\bar{\alpha}$ and the dipole polarizability anisotropy $\alpha_{ani}$.
Specify \verb+ALPHA_ISO+ for \verb+AVERAGE+ to obtain $\bar{\alpha}$ and
\verb+ALPHA_ANI+ to obtain $\alpha_{ani}$ and $\bar{\alpha}$.
The \verb+SYMMETRY+ input defines the selection rules that can be
exploited to reduce the number of tensor elements that have to be
evaluated. Available options are
\verb+ATOM+, \verb+SPHTOP+ (spherical top), \verb+LINEAR+,
\verb+XYDEGN+ ($x$- and $y$-axis equivalent, {\it i.e.\/} a $C_z^n$
symmetry axis with $n \ge 3$),  and \verb+GENER+ (use point
group symmetry from geometry input).
Integrals needed for operator DIPLEN.
\index{dipole polarizability}

\item[\Key{CTOSHI}]
Determine the CTOCD-DZ\index{CTOCD-DZ} magnetic shielding tensors. Integrals 
needed for computation are DIPVEL, ANGMOM, RPSO, and PSO. Specific atoms can 
be selected by using the option \Key{SELECT} in the \Sec{*INTEGRAL} input 
module.
\index{nuclear shielding!Coupled Cluster}

\item[\Key{CTOSUS}]
Determine the CTOCD-DZ\index{CTOCD-DZ} magnetizabilty. Integrals needed for 
computation are DIPVEL, RANGMO, and ANGMOM.
\index{magnetizabilty!Coupled Cluster}

\item[\Key{DIPGRA}]
Evaluate dipole gradients, i.e. compute the Atomic Polar Tensor, and
perform Cioslowski population analysis. Note that this keyword should
only be used for static calculations.
Integrals needed for operator DIPLEN, DIPGRA, DEROVL, and DERHAM.
\index{dipole gradient!Coupled Cluster}
 
\item[\Key{DIPOLE}] 
Evaluate all symmetry allowed elements of the dipole polarizability
(max. 6 components).
Integrals needed for operator DIPLEN.
\index{dipole polarizability}

\item[\Key{DISPCF}] \verb| |\newline
   \verb|READ (LUCMD,*) NLRDSPE|

   Calculate the dispersion coefficients 
   $D_{AB}(n)$ up to $n = $ \verb+NLRDSPE+.

\item[\Key{FREQUE}] \verb| |\newline
   \verb|READ (LUCMD,*) NBLRFR |\newline
   \verb|READ (LUCMD,*) (BLRFR(I),I=1,NBLRFR)|

Frequency input for $\langle\langle A;B \rangle\rangle_{\omega}$ in
Hartree (may be combined with wave length input).

%
%\item[\Key{ALLDSP}] 
%
% \item[\Key{FTST}] 
%
\item[\Key{OPERAT}] \verb| |\newline
   \verb|READ (LUCMD,'(2A)') LABELA, LABELB|\newline
   \verb|DO WHILE (LABELA(1:1).NE.'.' .AND. LABELA(1:1).NE.'*')|\newline
   \verb|  READ (LUCMD,'(2A)') LABELA, LABELB|\newline
   \verb|END DO|

Read pairs of operator labels. 

\item[\Key{OR}] 
Evaluate the specific and molar optical rotation using the length gauge
and modified velocity gauge\index{Modified velocity gauge}
expressions.
Integrals needed for operators DIPLEN, DIPVEL, and ANGMOM.
\index{optical rotation!Coupled Cluster}

\item[\Key{OR LEN}] 
Evaluate the specific and molar optical rotation using the length gauge
expression. Note that the result is origin-dependent and the origin employed
is the gauge origin used for computing the angular momentum integrals.
Integrals needed for operators DIPLEN and ANGMOM.
\index{optical rotation!Coupled Cluster}

\item[\Key{OR MVE}] 
Evaluate the specific and molar optical rotation using the modified velocity
gauge\index{Modified velocity gauge}
expression to ensure origin invariance as described in
Ref.\ \cite{Pedersen:ORMVE}.
Integrals needed for operators DIPVEL and ANGMOM.
\index{optical rotation!Coupled Cluster}

\item[\Key{ORGANL}] 
Evaluate the origin-dependence of the length gauge optical rotation
(see Ref.\ \cite{Pedersen:ORMVE}).
Integrals needed for operators DIPLEN and DIPVEL.
\index{optical rotation!Coupled Cluster}

\item[\Key{ORIGIN}] \verb| |\newline
  \verb|READ (LUCMD,*) NORGIN |\newline
  \verb|DO J = 1,NORGIN       |\newline
  \verb|   READ (LUCMD,*) (ORIGIN(I,J),I=1,3)|\newline
  \verb|END DO|

Additionally evaluate the length gauge optical rotation at the
specified origins (relative to the origin of the angular momentum
operator).
Integrals needed for operators DIPLEN, DIPVEL, and ANGMOM.
\index{optical rotation!Coupled Cluster}
 
\item[\Key{PRINT}] \verb| |\newline
   \verb|READ (LUCMD,*) IPRSOP|

Set print level for linear response output.

\item[\Key{STATIC}]
Evaluate the linear response functions at zero frequency.
May be combined with frequency/wave length input.

\item[\Key{WAVELE}] \verb| |\newline
   \verb|READ (LUCMD,*) NBLRWL |\newline
   \verb|READ (LUCMD,*) (BLRWL(I),I=1,NBLRWL)|

Frequency input for $\langle\langle A;B \rangle\rangle_{\omega}$
supplied as wave length in nm (may be combined with frequency input).
 
\end{description}
