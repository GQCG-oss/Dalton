\chapter{Calculation of magnetic properties}\label{ch:magnetic}

This chapter describes the calculation of properties depending on
magnetic fields, both as created by an external magnetic field as well
as the magnetic field created by a nuclear magnetic moment.
This includes the two
contributions to the ordinary spin-Hamiltonian used in NMR, nuclear
shieldings\index{nuclear shielding} and indirect nuclear spin--spin
couplings\index{spin-spin coupling} constants. We
also describe the calculation of the magnetic analogue of the
polarizability, the molecular
magnetizability\index{magnetizability}. This property
is of importance in NMR experiments where the reference substance is placed
in another tube than the sample. We also shortly
describe two properties very closely related to the magnetizability
and nuclear shieldings respectively, the rotational {\em g} factor and the
nuclear spin--rotation constants\index{rotational g
tensor}\index{spin-rotation constant}.

Three properties that in principle depend on the %nuclear
\typeout{Sonia: they are not all "nuclear"} magnetic moments are
not treated here, namely the properties associated with optical
activity or, more precisely, with circular dichroism. These
properties are Vibrational Circular Dichroism
(VCD)\index{VCD}\index{vibrational circular dichroism}, Raman
Optical activity (ROA)\index{ROA}\index{Raman optical activity},
Electronic Circular Dichroism (ECD)\index{ECD}\index{electronic
circular dichroism} will be treated in Chapter~\ref{ch:optchap}.
Another (magneto-)optical property, the ${\cal{B}}$ term of
Magnetic Circular Dichroism (MCD) is not described here but in
Chapter~\ref{ch:rspchap}.

Gauge-origin\index{gauge origin} independent nuclear shieldings,
magnetizabilities
and rotational {\em g} tensors are
obtained through the use of London atomic orbitals, and the theory is
presented in several references
\cite{kwjfhppjacs112,krthrkpjklbhjajjcp100,krthklbpjhjajjcp99,krthklbpjjocp195}.
These properties are easy to calculate (from the user's point of
view), and is thus given only a very brief description here.

The indirect spin--spin couplings are calculated by using the triplet linear
response function\index{triplet response}, as described in
Ref.~\cite{ovhapjhjajsbpthjcp96}.
These are in principle equally simple to calculate with \siraba\ as
nuclear shieldings and magnetizabilities. However, there are 13
contributions to the spin--spin coupling constant from {\em each}
nucleus. Furthermore, the spin--spin coupling constants put severe
requirements on the quality of the basis set as well as a proper
treatment of correlation, making the evaluation of spin--spin coupling
constants a time consuming task. Some notes about how this time
can be reduced is given below.

\section{Magnetizabilities}\label{sec:magnetizability}

\begin{center}
\fbox{
\parbox[h][\height][l]{12cm}{
\small
\noindent
{\bf Reference literature:}
\begin{list}{}{}
\item SCF magnetizabilities: K.Ruud, T.Helgaker, K.L.Bak, P.J\o rgensen and H.J.Aa.Jensen. \newblock {\em J.Chem.Phys.}, {\bf 99},\hspace{0.25em}3847, (1993).
\item MCSCF magnetizabilities: K.Ruud, T.Helgaker, K.L.Bak, P.J\o
rgensen, and J.Olsen. \newblock {\em Chem.Phys.}, {\bf
195},\hspace{0.25em}157, (1995).
\item Solvent effects: K.V.Mikkelsen,
P.J{\o}rgensen, K.Ruud, and T.Helgaker. \newblock {\em J.Chem.Phys.}, {\bf
107},\hspace{0.25em}1170, (1997).
\end{list}
}}
\end{center}

The calculation of molecular magnetizabilities\index{magnetizability}
is invoked by the
keyword \Key{MAGNET} in the \Sec{*PROPERTIES} input module. Thus
a complete input file for the calculation of molecular
magnetizabilities will look like:

\begin{verbatim}
**DALTON INPUT
.RUN PROPERTIES
**WAVE FUNCTIONS
.HF
**PROPERTIES
.MAGNET
*END OF INPUT
\end{verbatim}

This will invoke the calculation of molecular magnetizabilities using
London Atomic Orbitals\index{London orbitals} to ensure fast basis set
convergence and
gauge-origin\index{gauge origin} independent results. The natural
connection\index{natural connection}
\cite{joklbkrthpjtca90} is used in order to get numerically accurate
results. By default the center of mass\index{center of mass} is chosen
as gauge origin.

The augmented cc-pVXZ basis sets of Dunning and
coworkers~\cite{thdjcp90,rakthdrjhjcp96,dewthdjcp98,dewthdjcp100} have
been shown to give  to give excellent results for
magnetizabilities~\cite{pdkrthklbpj,krthklbpjhjajjcp99,krthpjklbcpl223,krhsthklbpjjacs116},
and these basis sets are obtainable from the basis set library.

Notice that a general print level of 2 or higher is needed in order to
get the individual contributions (relaxation, one- and
two-electron expectation values and so on) to the total magnetizability.

If more close control of the different parts of the calculation of the
magnetizability is wanted we refer the reader to the section
describing the options available. The modules that controls the
calculation of molecular magnetizabilities are:

\begin{list}{}{\itemsep 0.10cm \parsep 0.0cm}
\item[\Sec{EXPECT}] Controls the calculation of one-electron
expectation values contributing to the diamagnetic magnetizability.
\item[\Sec{GETSGY}] Controls the set up of the right-hand sides
(gradient terms) as well as the calculation of two-electron
expectation values and reorthonormalization terms.
\item[\Sec{LINRES}] Controls the solution of the magnetic response
equations
\item[\Sec{RELAX}] Controls the adding of solution and right-hand
side vectors into relaxation contributions
\end{list}

\section{Nuclear shielding constants}\label{sec:shieldings}

\begin{center}
\fbox{
\parbox[h][\height][l]{12cm}{
\small
\noindent
{\bf Reference literature:}
\begin{list}{}{}
\item K.Wolinski, J.F.Hinton, and P.Pulay. \newblock {\em
J.Am.Chem.Soc.}, {\bf 112},\hspace{0.25em}8251, (1990)
\item K.Ruud, T.Helgaker, R.Kobayashi, P.J\o rgensen, K.L.Bak, and H.J.Aa.Jensen. \newblock {\em J.Chem.Phys.}, {\bf 100},\hspace{0.25em}8178, (1994).
\item Solvent effects: K.V.Mikkelsen,
P.J{\o}rgensen, K.Ruud, and T.Helgaker. \newblock {\em J.Chem.Phys.}, {\bf
107},\hspace{0.25em}1170, (1997).
\end{list}
}}
\end{center}


The calculation of nuclear shieldings\index{nuclear shielding} are
invoked by the
keyword \Key{SHIELD} in the \Sec{*PROPERTIES} input module. Thus
a complete input file for the calculation of nuclear shieldings will
be:

\begin{verbatim}
**DALTON INPUT
.RUN PROPERTIES
**WAVE FUNCTIONS
.HF
**PROPERTIES
.SHIELD
*END OF INPUT
\end{verbatim}

This will invoke the calculation of nuclear shieldings using
London Atomic Orbitals\index{London orbitals} to ensure fast basis set
convergence and gauge
origin\index{gauge origin} independent results. The natural connection
\cite{joklbkrthpjtca90}\index{natural connection}
is used in order to get
numerically accurate results. By default the center of mass is chosen as gauge
origin\index{center of mass}.

A basis set well suited for the calculation of nuclear shieldings (and
indirect nuclear spin--spin coupling constants) is the TZ basis set of
Ahlrichs and coworkers~\cite{ashhrajcp97,aschrajcp100} with two
polarization functions~\cite{pdkrthklbpj}. This basis set is available
from the basis set library as TZ2P.

Notice that a general print level of 2 or higher is needed in order to
get the individual contributions (relaxation, one- and
two-electron expectation values and so on) to the total nuclear shieldings.

If more close control of the different parts of the calculation of the
nuclear shieldings is wanted we refer the reader to the section
describing the options available. For the calculation of nuclear
shieldings, these are the same as listed above for magnetizability
calculations.

\section{Rotational {\em g} tensor}\label{sec:gfac}

\begin{center}
\fbox{
\parbox[h][\height][l]{12cm}{
\small
\noindent
{\bf Reference literature:}
\begin{list}{}{}
\item J.Gauss, K.Ruud, and T.Helgaker. \newblock {\em J.Chem.Phys.},
{\bf 105},\hspace{0.25em}2804, (1996).
\end{list}
}}
\end{center}

The calculation of the rotational  {\em g} tensor\index{rotational g
tensor} is invoked through the
keyword \Key{MOLGFA} in the \Sec{*PROPERTIES} input module. Thus a complete
input file for the calculation of the molecular g tensor is:

\begin{verbatim}
**DALTON INPUT
.RUN PROPERTIES
**WAVE FUNCTIONS
.HF
**PROPERTIES
.MOLGFA
*END OF INPUT
\end{verbatim}

The molecular g tensor consists of two terms: a nuclear term and a
term which may be interpreted as one definition of a paramagnetic part
of the magnetizability tensor as
described in \cite{jgkrthjcp105}.
By default the center of mass\index{center of mass} is chosen as
rotational origin
origin, as this corresponds to the point about which the molecule
rotates. The use of rotational London atomic\index{London orbitals}
orbitals can be turned off through
the keyword \verb|.NOLOND|.

The basis set requirements for the rotational {\em g} tensors are more or
less equivalent with the ones for the molecular magnetizability,
that is, the augmented cc-pVDZ of Dunning and
Woon~\cite{thdjcp90,dewthdjcp98}, available from the basis set library
as \verb|aug-cc-pVDZ|.

If more close control of the different parts of the calculation of the
rotational {\em g}  tensor is wanted we refer the reader to the section
describing the options available. For the calculation of the molecular
g tensor, these are the same as listed above for magnetizability
calculations.

\section{Nuclear spin--rotation constants}\label{sec:spinrotasjon}

\begin{center}
\fbox{
\parbox[h][\height][l]{12cm}{
\small
\noindent
{\bf Reference literature:}
\begin{list}{}{}
\item R.Ditchfield. \newblock {\em J.Chem.Phys.}, {\bf
56},\hspace{0.25em}5688 (1972)
\item J.Gauss, K.Ruud, and T.Helgaker. \newblock {\em J.Chem.Phys.},
{\bf 105},\hspace{0.25em}2804, (1996).
\end{list}
}}
\end{center}


In \siraba\ the nuclear spin--rotation\index{spin-rotation constant}
constants are calculated using
rotational orbitals, giving an improved
basis set convergence~\cite{jgkrthjcp105}. We use the
expression for the spin--rotation constant where the paramagnetic term
is evaluated around the center of mass, and thus will only solve three
response equations at the most.

An input requesting the calculation of spin--rotation constants will
look like

\begin{verbatim}
**DALTON INPUT
.RUN PROPERTIES
**WAVE FUNCTIONS
.HF
**PROPERTIES
.SPIN-R
.ISOTOP
    3
    2   1    1
*END OF INPUT
\end{verbatim}

As the nuclear spin--rotation\index{spin-rotation constant} constants
depend on the isotopic substitution of
the molecule, both through the nuclear magnetic moments and through the
center of mass, the isotopic constitution need to be specified if this
is different  from the most abundant isotopic constitution. Note that
some of the most common isotopes do not have a magnetic moment.

We note that in the current release of \siraba , nuclear spin--rotation
constants can not be calculated employing symmetry-dependent
nuclei. Thus for a molecule like N$_2$, the symmetry plane
perpendicular to the molecular bond will have to be removed during the
calculation.

\section{Indirect nuclear spin--spin coupling
constants}\label{sec:spinspin}

\begin{center}
\fbox{
\parbox[h][\height][l]{12cm}{
\small
\noindent
{\bf Reference literature:}
\begin{list}{}{}
\item O.Vahtras, H.\AA gren, P.J\o rgensen, H.J.Aa.Jensen, S.B.Padkj\ae
r, and T.Helgaker. \newblock {\em J.Chem.Phys.}, {\bf
96},\hspace{0.25em}6120, (1992).
\item Solvent effects: P.-O.\AA strand, K.V.Mikkelsen, P.J{\o}rgensen,
K.Ruud and T.Helgaker. To be published.
\end{list}
}}
\end{center}

As mentioned in the introduction of this chapter, the calculation of
indirect nuclear
spin--spin coupling constants is a time consuming task due to the
large number of contribution to the total spin---spin coupling
constants. Still, if all spin---spin couplings\index{spin-spin
coupling} in a molecule are wanted,
with some restrictions mentioned below, the input will look as
follows:

\begin{verbatim}
**DALTON INPUT
.RUN PROPERTIES
**WAVE FUNCTION
.HF
**PROPERTIES
.SPIN-SPIN
*END OF INPUT
\end{verbatim}

This input will calculate the indirect nuclear spin--spin
coupling\index{spin-spin coupling}
constants between isotopes with non-zero magnetic
moments\index{magnetic moment} and a
natural abundance\index{abundance} of more than 1\% . This limit will
automatically
include proton and $^{13}$C spin--spin coupling constants. By default
all contributions to the coupling constants will be calculated.

Often one is interested in only certain kinds of nuclei. For example,
 one may want to calculate only the proton spin--spin couplings of a molecule.
This can be accomplished in two ways: either by changing the abundance
threshold so that only this single isotope is included (most useful
for proton couplings), or by selecting the particular nuclei of interest.

All the keywords necessary to control such adjustments to the
calculation is given in the section describing the input for the
\Sec{SPIN-S} submodule. Thus an input
in which we  have reduced the abundance threshold as well as
selected three atoms will look as:

\begin{verbatim}
**DALTON INPUT
.RUN PROPERTIES
**WAVE FUNCTIONS
.HF
**PROPERTIES
.SPIN-S
*SPIN-S
.ABUNDA
 0.10
.SELECT
    3
    2    4    5
*END OF INPUT
\end{verbatim}

We refer to the section describing the \Sec{SPIN-S} input module for
the complete description of the syntax for these keywords, as well as
the numbering of the atoms which are selected.

We also notice that it is often of interest to calculate only specific
contributions (usually the Fermi-contact\index{Fermi contact} contribution) at a high level
of approximation. Sometimes the results obtained with a Hartree-Fock
wave function may help in predicting the relative importance of the different
contributions, thus helping in the decision of which contributions
should be calculated at a correlated level \cite{krthklbpjcpl226}.
The calculation of only certain contributions can be accomplished in
the input by turning off the different
contributions by the keywords \Key{NODSO}, \Key{NOPSO},
\Key{NOSD}, and \Key{NOFC}. We refer to the description of the
\Sec{SPIN-S} input module for a somewhat more thorough discussion.

If a closer a control of the individual parts of the calculation of
indirect nuclear spin--spin coupling constants is wanted, this can be
done through the use of keywords in the following input modules:

\begin{list}{}{\itemsep 0.10cm \parsep 0.0cm}
\item[\Sec{EXPECT}] Controls the calculation of one-electron
expectation value contribution to the diamagnetic spin--spin coupling
constants.
\item[\Sec{GETSGY}] Controls the set up of the right-hand sides
(gradient terms)
expectation values and reorthonormalization terms.
\item[\Sec{LINRES}] Controls the solution of the singlet magnetic
response equations.
\item[\Sec{TRPRSP}] Controls the solution of the triplet magnetic
response equations (for Fermi-contact and spin--dipole contributions).
\item[\Sec{RELAX}] Controls the adding of solution and right hand
side vectors into relaxation contributions.
\item[\Sec{SPIN-S}] Controls the choice of nuclei for which the
spin--spin coupling constants will be calculated, as well as which
contributions to the total spin--spin coupling constants are to be
calculated.
\end{list}

\section{Hyperfine Coupling Tensors}

\begin{center}
\fbox{
\parbox[h][\height][l]{12cm}{
\small
\noindent
{\bf Reference literature:}
\begin{list}{}{}
\item B.Fernandez, P.J{\o}rgensen, J.Byberg, J.Olsen, T.Helgaker, and
H.J.Aa.Jensen. \newblock {\em J.Chem.Phys.}, {\bf
97},\hspace{0.25em}3412, (1992).
\item Solvent effects: B.Fernandez, O.Christiansen, O.Bludsky,
P.J{\o}rgensen, K.V. Mikkelsen. \newblock {\em J.Chem.Phys.}, {\bf
104},\hspace{0.25em}629, (1996).
\end{list}
}}
\end{center}

    The calculation of hyperfine coupling\index{hyperfine
coupling} tensors (in vacuum or in
solution) is invoked by the keyword \Sec{ESR} in the \Sec{*RESPONSE}
input module. Thus a complete input file for the calculation of
hyperfine coupling tensors will be:

\begin{verbatim}
**DALTON INPUT
.RUN RESPONSE
**INTEGRALS
.FC
.SD
**WAVE FUNCTIONS
.HARTREE-FOCK
**RESPON
.TRPFLG
*ESR
.ESRCAL
.MAXIT
   30
.TRPPRP
FC Cl 01
.TRPPRP
.
.
.

.TRPPRP
SD  01 x
.TRPPRP
SD  01 y
.TRPPRP
SD  01 z
.TRPPRP
.
.
.

*END OF
\end{verbatim}

    This will invoke the calculation of hyperfine
coupling\index{hyperfine coupling} tensors
using the Restricted-Unrestricted\index{restricted-unrestricted method}
methodology~\cite{bfpjjbjothhjajjcp97}. In this approach, the
unperturbed molecular system is described with a spin-restricted MCSCF
wave function, and when the perturbation - Fermi Contact\index{Fermi
contact} or Spin Dipole\index{spin-dipole}
operators - is turned on, the wave function spin relaxes and all
first-order molecular properties are evaluated as the sum of the
conventional average value term and a relaxation term that includes
the response of the wave function to the perturbations.

    The selection of a flexible atomic orbital basis set is decisive
in these calculations. Dunning's cc-pVTZ or Widmark's basis sets with some
functions uncontracted, and one or two sets of diffuse functions and
several tight s-functions added have been shown to provide accurate
hyperfine coupling tensors~\cite{bfpjcpl232}.

    If more close control of the different parts of the calculation
of hyperfine coupling tensors is wanted, we refer the reader to the
sections describing the options available.
