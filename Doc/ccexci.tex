
%%%%%%%%%%%%%%%%%%%%%%%%%%%%%%%%%%%%%%%%%%%%%%%%%%%%%%%%%%%%%%%%%%%
\section{Calculation of excitation energies: \Sec{CCEXCI}}\label{sec:ccexci}
%%%%%%%%%%%%%%%%%%%%%%%%%%%%%%%%%%%%%%%%%%%%%%%%%%%%%%%%%%%%%%%%%%%
\index{linear response}
\index{excitation energies, coupled cluster}
\index{excited states, coupled cluster}

In the \Sec{CCEXCI} section the input that is
specific for coupled cluster linear response calculation of
electronic excitation energies is given. 
Coupled cluster linear response excitation energies 
are implemented for the iterative CC models CCS, CC2, CCSD, and CC3 for 
both singlet and triplet excited states (for CC3 only singlet).
For singlet excited states the non-iterative models CC(2)(=CIS(D)) and CCSDR(3)
are also avaiable.
Publications that report results obtained with CC excitation energies
should cite Ref.\ \cite{Christiansen:JCP105} for singlet excitation energies
and \cite{Hald:JCP113} for triplet excitation energies.
For understanding the theoretical background for some
aspects of the CC3 calculations consult also Ref.\ \cite{Christiansen:JCP103}.

\begin{description}
\item[\Key{NCCEXC}] 

\verb|READ (LUCMD,*) (NCCEXCI(ISYM,1),ISYM=1,MSYM)|

Give the number of states desired.
For singlet states only, only one line is required with
the number of excitation energies for each symmetry class (max. 8). \\ 
If also triplet states are desired an additional line is given in same format, 
else nothing.
%
\item[\Key{THREXC}] 

The threshold for the solution of the excitation energies and
corresponding response eigenvectors. 
The threshold is the norm of the residual for the eigenvalue equation.
(Default: 1.0D-04). 

%
%\item[\Key{FDJAC }] 
%
%\item[\Key{FDEXCI}] 
%
%\item[\Key{JACEXP}] 
%
%\item[\Key{JACTST}] 
%
%\item[\Key{LHTR  }] 
%
\item[\Key{NOSCOM}] 
%
For CC3 calculations only: indicates that no self-consistent solution
should be seeked in the partitioned CC3 algorithm.

%\item[\Key{STSD  }] 
%
\item[\Key{TOLSC }] 
%
For CC3 calculations only: 
Set tolerance for the self-consistent solution for obtaining self-consistent
solution to the partitioned CC3 algorithm.
Tolerance refers to the eigenvalue itself.
(Default:1.0D-04).


\item[\Key{OMEINP}] 
%
A way to provide an input omega for the partitioned CC3 algorithm or restrict
the self-consistent solution to specific states.
If OMEINP is not specified the program uses the best choice available to it at that
moment based on previous levels of approximations (CCSD or even better CCSDR(3)) 
and calculates all states as given by NCCEXC.
IOMPINP is 1 for the lowest excited state of a given symmetry, 2 for the second lowest etc. \\
By giving an 0.0 input excitation energy (as EOMINP) the program takes the best previous
approximation found in this run - otherwise the user can specify a qualified guess 
(perhaps from a previous calculation which is now restarted).\newline
\verb|READ (LUCMD,*) (NOMINP(ISYM,1),ISYM=1,MSYM)|\newline
\verb|DO ISYM = 1, MSYM|\newline
\verb|  DO IOM = 1, NOMINP(ISYM,1)|\newline
\verb|     READ (LUCMD,*) IOMINP(IOM,ISYM,1),EOMINP(IOM,ISYM,1)|\newline
\verb|  ENDDO|\newline
\verb|ENDDO|

%\item[\Key{STVEC }] 
%
%\item[\Key{STOLD }] 
%
\item[\Key{CCSPIC}] 

\verb|READ(LUCMD,*) OMPCCS| \newline
Keyword for picking a state with a given CCS excitation energy.
Optimize a number of states in CCS with different symmetries and this option
will pick the one in closest correspondence with the given input excitation
energy, and skip the states at higher energies and other symmetries in 
the following calculation.
Useful for example in numerical Hessian calculations on excited states.

\item[\Key{CC2PIC}] 
Functions as CCSPIC but the picking is based on CC2 excitation energies.
\item[\Key{CCSDPI}] 
Functions as CCSPIC but the picking is based on CCSD excitation energies.
%
\item[\Key{MARGIN}] 
Specifies the maximum allowed deviation of the actual excitation energy
from the input excitation energy when using CCSPIC, CC2PIC and CCSDPI. 

\end{description}
