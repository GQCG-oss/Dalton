\chapter{Molecular vibrations and rotations}\label{ch:vibrot}

In this chapter we discuss properties related  to the
vibrational and rotational motions of a molecule. This includes
vibrational frequencies and the associated infrared
(IR)\index{IR intensity} and Raman intensities\index{Raman intensity}.

To the best of our knowledge, {\dalton} is the only quantum chemical
software package that can calculate Raman intensities at the frequency
of the incident laser light. Thus, {\dalton} gives an unique
opportunity to study the effect of the dispersion on the Raman
intensity.

\section{Vibrational frequencies}\label{sec:vibfreq}

The calculation of vibrational frequencies\index{vibrational
frequency} and rotational constants\index{rotation constant} are
controlled by the keyword \Key{VIBANA}. Thus, in order to
calculate the vibrational frequencies and the rotational constants
of a molecule all that is is needed is the input:

\begin{verbatim}
**DALTON INPUT
.RUN PROPERTIES
**WAVE FUNCTIONS
.HF
**PROPERTIES
.VIBANA
**END OF DALTON INPUT
\end{verbatim}

This keyword will, in addition to calculating the molecular frequencies
and rotational constants, also calculate the zero-point vibrational
energy corrections\index{zero-point vibrational energy} and vibrational
and rotational partition functions\index{partition functions}
at selected temperatures.

\siraba\ evaluates the molecular Hessian\index{Hessian} in Cartesian
coordinates\index{Cartesian coordinates}, and
the vibrational frequencies of any isotopically substituted  species
may therefore easily be obtained on the basis of the full
Hessian. Thus, if we would like to calculate the vibrational
frequencies of isotopically substituted molecules\index{isotopic
constitution}, this may be obtained through an input like:

\begin{verbatim}
**DALTON INPUT
.RUN PROPERTIES
**WAVE FUNCTIONS
.HF
**PROPERITES
.VIBANA
*VIBANA
.ISOTOP
   2   5
 1 2 1 1 1
 2 1 1 1 1
**END OF DALTON INPUT
\end{verbatim}

The keyword \Key{ISOTOP} in the \Sec{VIBANA} input module
indicates that more than only the isotopic species containing the most
abundant isotopes are to be calculated, which will always be
calculated. The numbers on the second line denotes the number of
isotopically substituted species that is requested and the number of
atoms in the system. The following lines then
lists the isotopic constitution of each of these species. 1
corresponds to the most abundant isotope, 2 corresponds to the second
most abundant isotope and so on. The isotopic substitution have to be
given for all atoms in the molecule (not only the symmetry
independent), and the above input could for instance correspond to a
methane\index{methane} molecule, with the isotopic species $CH_3D$
and $^{13}CH_4$.

As the isotopic substitution of all atoms in the molecule has to be
specified, let us mention the way symmetry-dependent atoms will be
generated. The atoms will be grouped in symmetry-dependent atom
blocks. The specified symmetry-independent atom will be the first of
this block, and the symmetry-dependent atoms will be generated
according to the order of the symmetry elements. Thus, assuming
D$_{2h}$ symmetry with symmetry generating elements \verb|X  Y  Z|,
the atoms generated will come in the order \verb|X|, \verb|Y|,
\verb|XY|, \verb|Z|, \verb|XZ|, \verb|YZ|, and \verb|XYZ|.

\section{Infrared (IR) intensities}\label{sec:irint}

\begin{center}
\fbox{
\parbox[h][\height][l]{12cm}{
\small
\noindent
{\bf Reference literature:}
\begin{list}{}{}
\item R.D.Amos. \newblock {\em Chem.Phys.Lett.}, {\bf
108},\hspace{0.25em}185, (1984).
\item T.U.Helgaker, H.J.Aa.Jensen, and P.J{\o}rgensen. \newblock {\em
J.Chem.Phys.}, {\bf 84},\hspace{0.25em}6280, (1986).
\end{list}
}}
\end{center}

\index{IR intensity} The evaluation of infrared intensities
requires the calculation of the
dipole gradients\index{dipole gradient}\index{APT}\index{atomic polar
tensor} (also known as Atomic Polar Tensors (APTs)). Thus, by
combining the calculation of vibrational frequencies with the
calculation of dipole gradients, IR intensities will be obtained. Such
an input may look like:

\begin{verbatim}
**DALTON INPUT
.RUN PROPERTIES
**WAVE FUNCTIONS
.HF
**PROPERTIES
.VIBANA
.DIPGRA
*VIBANA
.ISOTOP
   2   5
 1 2 1 1 1
 2 1 1 1 1
**END OF DALTON INPUT
\end{verbatim}

\noindent The keyword \Key{DIPGRA} invokes the calculation of the dipole
gradients.

\section{Dipole gradient based population analysis}

\begin{center}
\fbox{
\parbox[h][\height][l]{12cm}{
\small
\noindent
{\bf Reference literature:}
\begin{list}{}{}
\item J.Cioslowski. \newblock {\em J.Am.Chem.Soc.}, {\bf
111},\hspace{0.25em}8333, (1989).
\item T.U.Helgaker, H.J.Aa.Jensen, and P.J{\o}rgensen. \newblock {\em
J.Chem.Phys.}, {\bf 84},\hspace{0.25em}6280, (1986).
\end{list}
}}
\end{center}

\index{population analysis} As dipole gradients\index{dipole
gradient}\index{APT}\index{atomic polar tensor} are readily available in the
\siraba\ program, the
population analysis basis on the Atomic Polar Tensor as suggested by
Cioslowski~\cite{jcjacs111,poakrkvmthjpca102} can be obtained from an input like

\begin{verbatim}
**DALTON INPUT
.RUN PROPERTIES
**WAVE FUNCTIONS
.HF
**PROPERTIES
.POPANA
*END OF DALTON
\end{verbatim}

This population analysis is of course significantly more expensive
than the ordinary Mulliken population analysis\index{population analysis!Mulliken}\index{Mulliken population analysis} obtainable directly
from the molecular wave functions through an input like

\begin{verbatim}
**DALTON INPUT
.RUN WAVE FUNCTIONS
**WAVE FUNCTIONS
.HF
*POPULATION ANALYSIS
.MULLIKEN
*END OF DALTON
\end{verbatim}


\section{Raman intensities}\label{sec:ramanint}

\begin{center}
\fbox{
\parbox[h][\height][l]{12cm}{
\small
\noindent
{\bf Reference literature:}
\begin{list}{}{}
\item T.Helgaker, K.Ruud, K.L.Bak, P.J{\o}rgensen, and J.Olsen. \newblock {\em
Faraday Discuss.}, {\bf 99},\hspace{0.25em}165, (1994).
\end{list}
}}
\end{center}

\index{Raman intensity} Calculating Raman intensities is by no means
a trivial task, and
because of the computational cost of such calculations (as there does
not exist any program which avoids the numerical differentiation of the
polarizability with respect to nuclear distortions), there are
few theoretical investigations of basis set requirements and
correlation effects on calculated Raman intensities.

To the best of our knowledge, \siraba\ is the only quantum chemical program package capable of
calculating the Raman intensities at the frequency of the incident
laser beam. \siraba\ also has the same possibilities as other
common quantum chemical software packages to calculate static
Raman intensities (that is, where the frequency dependence of the
polarizability has been neglected). The Raman intensities calculated
are the ones obtained within the Placzek
approximation~\cite{placzek}\index{Placzek approximation},
and the implementation is described in Ref.~\cite{thkrklbpjjofd99}.

The Raman intensity is the differentiated frequency-dependent
polarizability\index{polarizability} with respect to nuclear displacements.
As it is a third derivative depending on the nuclear positions through
the basis set, numerical differentiation of
the polarizability with respect to nuclear coordinates is
necessary.

The input looks very similar to the input needed for the calculation
of Raman optical activity\index{ROA}\index{Raman optical activity}  described
in Section~\ref{sec:vroa}

\begin{verbatim}
**DALTON INPUT
.WALK
.MAX IT
 31
*WALK
.NUMERI
**WAVE FUNCTIONS
.HF
*SCF INPUT
.THRESH
1.0D-8
**START
.RAMAN
.ABALNR
*ABALNR
.THRESH
1.0D-7
.FREQUE
     2
0.0 0.09321471
**PROPERTIES
.RAMAN
.ABALNR
*ABALNR
.THRESH
1.0D-7
.FREQUE
     2
0.0 0.09321471
**FINAL
.RAMAN
.ABALNR
.VIBANA
*RESPONSE
.THRESH
1.0D-6
*ABALNR
.THRESH
1.0D-7
.FREQUE
     2
0.0 0.09321471
*VIBANA
.PRINT
 1
.ISOTOP
   1   5
 1 1 1 2 3
**END OF DALTON INPUT
\end{verbatim}

The keyword \Key{ABALNR} in the general input module indicates that
a frequency dependent linear response\index{linear response}\index{response!linear}
calculation is to be done, in
this case the calculation of the frequency-dependent
polarizability\index{polarizability} as
specified by the \Key{ALFA} keyword in the \Sec{ABALNR} input
module. The keyword \Key{RAMAN} indicates that we are only
interested in the Raman intensities\index{Raman intensity} and
depolarization ratios\index{depolarization ratio}. Note
that these parameters are also obtainable by using the keyword
\Key{VROA}. In this calculation we calculate the Raman intensities for two
frequencies, the static case and a frequency of the incident light
corresponding to a laser of wavelength 488.8 nm. Note, however, that
Raman intensities corresponding to zero frequency cannot be calculated
with the \Key{POLARI} keyword, even though static
polarizabilities may be obtained using this keyword.

Due to the numerical differentiation\index{numerical differentiation}
that is done, the threshold for
the iterative solution of the response equations are by default
10$^{-7}$, in order to get Raman intensities that are numerically
stable to one decimal digit.

In the \Sec{WALK} input module we have specified
that the walk is a numerical differentiation. This will automatically
turn off the calculation of the geometric Hessian\index{Hessian},
putting limitations
on what kind of properties that may be calculated at the same time as
Raman intensities. Because the Hessian is not calculated,
there will not be any prediction of the energy at the new
point.

It should also be noted that  in a numerical
differentiation\index{numerical differentiation}, the
program will
step plus and minus one displacement unit along each Cartesian coordinate
of all nuclei, as well as calculating the property at the reference
geometry. Thus, for a molecule with $N$ atoms the properties will need
to be calculated in a total of 2*3*$N$ + 1 points, which for a 
molecule with five atoms will amount to 31 points. The default maximum number of
steps in \siraba\ is 20. Thus one often need to
change the maximum number of allowed iterations\index{geometry
iteration} by adding the keyword
\Key{MAX IT} in the \Sec{*DALTON} input module.

The default step length in the numerical
differentiation\index{numerical differentiation} is $1.0\cdot 10^{-4}$
a.u., and this step length may be adjusted by the keyword
\Key{DISPLA} in the \Sec{WALK} input module. The steps are taken
in the Cartesian directions and
not along normal modes. This enables us to study the Raman intensities
of a large number of isotopically substituted molecules at once. This
is done in the \Sec{*FINAL} input section, where we
have requested one isotopically substituted species in addition to the
isotopic species containing the most abundant isotope of each element.

It should be evident that the calculation of Raman intensities is
a task for connoisseurs. However, the fact that \siraba\ can
calculate Raman intensities at the frequency of the incident
light puts \siraba\ in a special position among {\em ab initio}
programs, as these  at most  can calculate Raman intensities
based on static polarizabilities.

Concerning basis sets requirement for Raman intensities,  there is
yet much work to be done. In the only study presented of Raman
Optical Activity using London atomic orbitals, it is argued in
favor of the aug-cc-pVDZ basis set supplied with the basis set
library. However, in order to get Hartree--Fock limit quality of
the vibrational frequencies as well, a basis set of at least
aug-cc-pVTZ seems to be necessary. Whether these observations will
also apply to the Raman intensities is not known.
