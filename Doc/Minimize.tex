\subsection{Equilibrium geometries\label{subsec:eqgeo}}

\cite{MEST} \cite{Fletcher} \cite{Peng}
A minimal input file for the location of a molecular geometry minimum
(using a first-order optimization method) and performing a vibrational
analysis at the stationary point, as well as determining the nuclear
shielding constants at the optimized geometry, will look like
(assuming a Hartree-Fock wave function):

\begin{verbatim}
**DALTON INPUT
.MINIMIZE
**WAVE FUNCTIONS
.HARTREE-FOCK
**FINAL
.VIBANA
.SHIELD
*END OF INPUT
\end{verbatim}

The keyword \Key{MINIMIZE} signalizes a search for a minimum, that is a stationary
point on the molecular surface with Hessian index 0. At the final,
optimized geometry, the input in the \Sec{*FINAL} section requests
a vibrational analysis and the evaluation of nuclear shieldings.

The above input yields a first-order
optimization~\cite{MEST}~\cite{Fletcher} in redundant internal
coordinates~\cite{Peng} using the BFGS Hessian update. This means that
only the energy and the gradient is calculated in each iteration, and
an approximate Hessian is obtained using the gradients and the
displacements. The initial Hessian is diagonal in the redundant
internal coordinates. To obtain the properties at the optimized
geometry, the Hessian has to be calculated. By looking at the
vibrational frequencies one can then verify that a true minimum has
been reached (as all frequencies are real, corresponding to a positive
definite Hessian).

Several first-order methods have been implemented in DALTON, the
recommended method being the Broyden-Fletcher-Goldfarb-Shanno (BFGS)
Hessian updating scheme. Though other updates may perform better in
certain cases, the BFGS update generally seems to be the most
reliable and thus is the default method.

Without the calculation of the Hessian at the optimized geometry, one
can never be sure that the geometry is indeed a minimum, and this is the
main problem with first-order methods. The alternative is to
use a second-order method:

\begin{verbatim}
**DALTON INPUT
.MINIMIZE
*MINIMIZE
.2NDORD
**WAVE FUNCTIONS
.HARTREE-FOCK
**FINAL
.VIBANA
.SHIELD
*END OF INPUT
\end{verbatim}

In second-order methods the Hessian is calculated at every
geometry. The analytical Hessian naturally gives a better description
of the potential energy surface than the approximate Hessians of
first-order methods, and fewer steps will usually be needed to reach
the minimum. However, the price one has to pay for this, is that each
iteration uses significantly more CPU time. Another advantage of
second-order methods, is that the program will automatically break the
symmetry of the molecule, if that is necessary to reach a minimum
(unless the user specifies that symmetry not should be broken). If
one wants to minimize a water molecule starting from a linear
geometry and using molecular symmetry, a first-order method will
happily yield a linear geometry with optimized bond lengths, whereas
the second-order method will correctly decide that symmetry should be
reduced to C$_{2v}$ and minimize the molecule.
The bottom-line, however, is that while second-order methods
definitely are the more robust, they will generally be outperformed by
the much more time-efficient first-order methods.

Both first- and second-order methods are based on trust-region
optimization. The trust-region is the region of the potential energy
surface where our quadratic model (based on an analytical or
approximate Hessian) gives a good representation of the true
surface. To make things easier, this region is given the shape of a
hypersphere. In the trust-region based optimization algorithm, the
radius of the trust-region is
automatically updated during the optimization by a feedback mechanism.
Occasionally, however, the potential surface may show locally large
deviations from quadratic form. This will result in a large
disagreement between the predicted energy change and the energy
calculated at the new geometry. If this deviation is larger than a
given threshold the step will be reduced through a decrease in the
trust radius and a simple line search. A quadratic function is fitted
to the rejected energy and the previous energy and gradient. The
minimum of this function is then used as the new step, and this
process may be repeated.

Another consideration concerning optimizations, is the choice of
coordinate system. Second-order methods doesn't seem to be very
sensitive to this, but it may be crucial to the performance of
first-order methods. {\siraba} provides the normal Cartesian
coordinates and redundant internal coordinates, the latter being
highly recommended when using first-order methods. Redundant internal
coordinates consists of all bond lengths, angles and dihedral angles
in a molecule, and the redundancies are removed through
projection. The default for second-order methods is Cartesian
coordinates, while redundant internal coordinates defaults for
first-order methods.

The program decides when a minimum is reached through a set of
convergence criteria. These will be set automatically depending on the
accuracy of the wavefunction. In a standard Hartree-Fock calculation, the
thresholds for the energy difference from the last iteration, the norm
of the gradient and the norm of the step will all be 1.0D-6. Two out
of these three have to decrease below its threshold, before the program
declares convergence. These criteria ensure a good geometry suitable
for vibrational analysis, but may be unnecessary tight if one only
wants the energy. The convergence criteria can be controlled through
the input file. One should keep in mind that while second-order
methods are rather insensitive to changes in theses thresholds due to
their quadratic convergence, first-order methods may run into severe
trouble if the criteria are too close to the accuracy of the wavefunction
and the gradient.

Calculations involving large basis sets may be very time-consuming,
and it becomes very important to start the optimization from a decent
geometry. One way to solve this is through preoptimization, that is
one or more smaller basis sets are first used to optimize the
molecule. {\siraba} supports the use of up to ten different basis sets
for preoptimization. This may also be used as a very effective way to
get the optimized energy of a molecule for a series of basis
sets. One may also want to calculate the energy of a molecule using a
large basis set at the optimized geometry of a smaller basis set, and
this is also supported in {\siraba}, but limited to only one single-point
basis set. Both thess features are illustrated in the input file below:

\begin{verbatim}
**DALTON INPUT
.MINIMIZE
*MINIMIZE
.PREOPT
2
STO-3G
4-31G
.SP BAS
cc-daug-pVTZ
**WAVE FUNCTIONS
.HARTREE-FOCK
*END OF INPUT
\end{verbatim}

As one can see, two small basis sets have been chosen for
preoptimization. Note that they will be used in the order they appear
in the input file, so one should sort the sets and put the smaller one
at the top for maximum efficiency. The ``main'' basis set, that is the
one that will be used for the final optimized geometry, is specified
in the file MOLECULE.INP as usual. After the last optimization, the
energy of the molecule will be calculated at the final geometry with the
cc-daug-pVTZ basis. Please note that these features will work only
when using basis sets from the basis set library.

It is possible to control the optimization procedure more closely
by giving the program instructions on how to do the different parts of
the calculation. Below we have listed all the modules that may
affect the geometry optimization  as well as a short description of
which part of the calculation that module controls. The reader is
referred to the section describing each module for a closer
description of the different options. These sections may, with the
exception of the \Sec{MINIMIZE} and \Sec{WALK} section, be specified
in any of the modules \Sec{*START}, \Sec{*PROPERTIES},
\Sec{*FINAL}. The \Sec{MINIMIZE} and \Sec{WALK} is
to be placed in the \Sec{*DALTON} module, and will apply to the
entire calculation.

\begin{description}
\item[\Sec{GEOANA}] Describes what kind  of information about the
molecular geometry is to be printed 
\item[\Sec{GETSGY}] Controls the set up of the right-hand sides
(gradient terms) 
\item[\Sec{MINIMIZE}] Controls the first- and second-order
minimization methods
(gradient terms) 
\item[\Sec{NUCREP}] Controls the calculation of the nuclear contributions
\item[\Sec{ONEINT}] Controls the calculation of one-electron contributions
\item[\Sec{RELAX}] Controls the adding of solution and right-hand side vectors
into relaxation contributions
\item[\Sec{REORT}] Controls the calculation of reorthonormalization terms
\item[\Sec{RESPON}] Controls the solution of the geometric response equations
\item[\Sec{TROINV}] Controls the use of translation and rotational invariance
\item[\Sec{TWOEXP}] Controls the calculation of two-electron
expectation values 
\item[\Sec{VIBANA}] Set up the vibrational and rotational analysis of the
molecule
\item[\Sec{WALK}] Controls the walk (see also ``Locating transition
states'' and ``Doing Dynamical walks'')
\end{description}

\subsection{General: \Sec{MINIMIZE}}\label{subsec:minimize}

This submodule is an alternative to the \Sec{WALK} submodule. It
contains both first and second order methods for energy 
minimization (geometry optimization). Most of the Hessian updating
schemes were taken from \cite{MEST} and \cite{Fletcher}. The
implementation of redundant internal coordinates follows the work done
by Peng et. al. \cite{Peng}. In addition to this, several
keywords for VRML visualization are included \cite{VRML}.

\begin{description}

\item[\Key{1STORD}]\verb| |

Use default first order method. This means that the BFGS update will
be used, and the optimization carried out in redundant internal
coordinates. Same effect as the combination of the two keywords
\Key{BFGS} and \Key{REDINT}.

\item[\Key{2NDORD}]\verb| |

Use default second order method. The level-shifted Newton method and
Cartesian coordinates are used. Identical to specifying the two
keywords \Key{NEWTON} and \Key{CARTES}. This is the default, and it
will be used if no other optimization method has been selected.

\item[\Key{BAKER}]\verb| |

Activates the convergence criteria of Baker \cite{Baker}. The minimum is then said
to be found, when the largest element of the gradient vector (in
Cartesian or redundant internal coordinates) falls below 3.0D-4 and
either the energy change from the last iteration is less than 1.0D-6
or the largest element of the predicted step vector is less 3.0D-4.

\item[\Key{BFGS}]\verb| |

Specifies the use of a first order method with the
Broyden-Fletcher-Goldfarb-Shanno (BFGS) update formula for
optimization. This is the preferred first order method.

\item[\Key{CARTES}]\verb| |

Indicates that Cartesian coordinates should be used in the
optimization. This is the default for second order methods.

\item[\Key{CONDIT}]\verb| |
\newline
\verb|READ(LUCMD,*) ICONDI|

Set the number of convergence criteria that should be fulfilled before
convergence occurs. There are three different convergence thresholds,
one for the energy, one for the gradient norm and one for the step
norm. The possible values for this variable is therefore between 1 and
3. Default is 2. The three convergence thresholds can be adjusted with
the keywords \Key{ENERGY}, \Key{GRADIE} and \Key{STEP T}.

\item[\Key{DFP}]\verb| |

Specifies that a first order method with the
Davidon-Fletcher-Powell (DFP) update formula should be used
for optimization.

\item[\Key{ENERGY}]\verb| |
\newline
\verb|READ(LUCMD,*) THRERG|

Set the convergence threshold for the energy. This is one of the three
convergence thresholds (the keywords \Key{GRADIE} and \Key{STEP T}
control the other two). Default value is 1.0D-6.

\item[\Key{GRADIE}]\verb| |
\newline
\verb|READ(LUCMD,*) THRGRD|

Set the convergence threshold for the gradient norm. This is one of
the three convergence thresholds (the keywords \Key{ENERGY} and
\Key{STEP T} control the other two). Default value is 1.0D-6.

\item[\Key{GRDINI}]\verb| |

Specifies that the Hessian should be reinitialized every time the norm
of the gradient is larger than norm of the gradient two iterations
earlier. This keyword should only be used when it's difficult to
obtain a good approximation to the Hessian during optimization. Only
applies to first order methods.

\item[\Key{HESFIL}]\verb| |

Specifies that the initial Hessian should be read from the file
\verb|DALTON.HES|. This applies to first order methods, and the
Hessian in the file must have the correct dimensions. This option
overrides other options for the initial Hessian.

Each time a Hessian is calculated or updated, it's written to this
file. If an optimization is interrupted, it can be restarted with the
last geometry and the Hessian in \verb|DALTON.HES|, minimizing the
loss of information. Another useful possibility, is to transfer
the Hessian from a calculation on the same molecule with another
(smaller) basis. Finally, one can go in an edit the file directly to
set up a specific forcefield.

\item[\Key{INIRED}]\verb| |

Specifies that the initial Hessian should be diagonal in redundant
internal coordinates, and it is then transformed to Cartesian
coordinates. This only applies to first order optimizations in
Cartesian coordinates.

\item[\Key{INITEV}]\verb| |
\newline
\verb|READ(LUCMD,*) EVLINI|

The default initial Hessian for first order minimizations is the
identity matrix when Cartesian coordinates are used, and a diagonal
matrix when redundant internal coordinates are used. If \Key{INITEV}
is used all the diagonal elements (and therefore the eigenvalues) are
set equal to the value EVLINI. This option only has effect when first
order methods are used and \Key{INITHE} and \Key{HESFIL} are
non-present.

\item[\Key{INITHE}]\verb| |

Specifies that the initial Hessian should be
calculated (analytical Hessian), thus yielding a first step that is
identical to second order methods. This provides an excellent starting
point for first order methods, but should only be used when the
Hessian can be calculated within a reasonable amount of time. Only has
effect for first order methods, and it overrides the keywords
\Key{INITEV} and \Key{INIRED}. It has no effect when \Key{HESFIL} has
been specified.

\item[\Key{INTERA}]\verb| |

Specifies a interactive run. The energy, gradient
norm, step length and Hessian index is then written to the unit LUERR
in each iteration, allowing easy monitoring of the optimization.

\item[\Key{MAX IT}]\verb| |
\newline
\verb|READ(LUCMD,*) ITRMAX|

Read the maximum number of geometry iterations. Default value is 25.

\item[\Key{MAX RE}]\verb| |
\newline
\verb|READ(LUCMD,*) MAXREJ|

Read maximum number of rejected steps in each iterations, default is
5.

\item[\Key{NEWTON}]\verb| |

Specifies that a second order level-shifted Newton method should
be used for optimization. This is the default method.

\item[\Key{NOBREA}]\verb| |

Disables breaking of symmetry. The geometry will
be optimized within the given symmetry, even if a non-zero Hessian
index is found. The default is to let the symmetry be broken until
a minimum is found with a Hessian index of zero. This option only has
effect when second order methods are in use.

\item[\Key{NOREIN}]\verb| |

When first order methods are used, the Hessian is reinitialized every
time the Hessian index becomes non-zero (due to negative
eigenvalues). This guarantees that the Hessian describes a minimum,
but valuable information gathered in the Hessian may be
lost.

\Key{NOREIN} disables this reinitialization, relying on the
optimization method to restore the Hessian to its correct structure
(by locating the area near the minimum). This option is particularly
useful in conjunction with the keyword \Key{INITHE}, as it is pretty
pointless to calculate the Hessian at the initial geometry, then reset
it to the identity matrix because some negative eigenvalue showed up.

\item[\Key{NOTRUS}]\verb| |

Turns off trust radius, so that a full Newton step
is taken in each iteration. This should be used with caution, as
global convergence is no longer guaranteed. If long steps are desired,
it is safer to adjust the initial trust radius and the limits for the
actual/predicted energy ratio. 

\item[\Key{OPTION}]\verb| |

Causes all the allowed keywords in the \Sec{MINIMIZE}
module to be listed.

\item[\Key{PREOPT}]\verb| |
\newline
\verb|READ(LUCMD,*) NUMPRE, (PREBTX(I), I = 1, NUMPRE)|

First we read the number of basis sets that should be used for
preoptimization, then we read those basis set names as strings. These
sets will be used for optimization in the order they appear in the
input. One should therefore place the smaller basis at the top. After
the preoptimization, optimization is performed with the basis
specified in the molecule input file.

\item[\Key{PRINT}]\verb| |
\newline
\verb|READ(LUCMD,*) IPRINT|

Set print level for this module.  Read one more line containing print
level. Default value is 0.

\item[\Key{PSB}]\verb| |

Specifies that a first order method with the
Powell-symmetric-Broyden (PSB) update formula should be used
for optimization.

\item[\Key{QUADSD}]\verb| |

Specifies that a second order quadratic steepest
descent method should be used for optimization. This has not been
implemented yet.

\item[\Key{RANKON}]\verb| |

Specifies that a first order method with the rank
one update formula should be used for optimization.

\item[\Key{REDINT}]\verb| |

Specifies that redundant internal coordinates should be used in the
optimization. This is the default for first order methods.

\item[\Key{REJINI}]\verb| |

Specifies that the Hessian should be reinitialized after every
rejected step, as a rejected step indicates that the Hessian models the
true potential surface poorly. Only applies to first order methods.

\item[\Key{SCHLEG}]\verb| |

Specifies that a first order method with Schlegel's updating scheme
\cite{Schlegel} should be used. This makes use of all previous
displacements and gradients, not just the last, to update the
Hessian.

\item[\Key{SP BAS}]\verb| |
\newline
\verb|READ(LUCMD,*) SPBSTX|

Read a string containing the name of a basis set. When the geometry
has converged, a single point energy will be calculated using this
basis.

\item[\Key{STEEPD}]\verb| |

Specifies that the first order steepest descent method should be
used. No update is done on the Hessian, so the optimization will be
guided by the gradient alone. The ``pure'' steepest descent method is
obtained when the Hessian is set equal to the identity matrix. Each
step will then be the negative of the gradient vector, and the
convergence towards the minimum will be extremely slow. However, this
option can be combined with other initial Hessians in Cartesian or
redundant internal coordinates, giving a method where the main feature
is the lack of Hessian updates.

\item[\Key{STEP T}]\verb| |
\newline
\verb|READ(LUCMD,*) THRSTP|

Set the convergence threshold for the step norm. This is one of the
three convergence thresholds (the keywords \Key{ENERGY} and
\Key{GRADIE} control the other two). Default value is 1.0D-6.

\item[\Key{SYMTHR}]\verb| |
\newline
\verb|READ(LUCMD,*) THRSYM|

Determines the gradient threshold for breaking of the symmetry. That
is, if the index of the Hessian is non-zero when the gradient norm
drops below this value, the symmetry is broken to avoid unnecessary
iterations within the wrong symmetry. This option only applies to
second order methods and when the keyword \Key{NOBREA} is
non-present. The default value of this threshold is 1.0D-3.

\item[\Key{TR FAC}]\verb| |
\newline
\verb|READ(LUCMD,*) TRSTIN, TRSTDE|

Read two factors that will be used for increasing and decreasing the
trust radius respectively. Default values are 1.2D0 and 0.7D0.

\item[\Key{TR LIM}]\verb| |
\newline
\verb|READ(LUCMD,*) RTENBD, RTENGD, RTRJMN, RTRJMX|

Read four limits for the ratio between the actual and predicted
energies. This ratio indicates how good the step is, that
is how accurately the the quadratic model describes the true energy
surface. If the ratio is below RTRJMN or above RTRJMX, the step is
rejected. With a ratio between RTRJMN and RTENBD, the step is
considered bad an the trust radius decreased to less than the step
length. Ratios between RTENBD and RTENGD are considered satisfactory,
the trust radius is set equal to the norm of the step. Finally ratios
above RTENGD (but below RTRJMX) indicate a good step, and the trust
radius is given a value larger than the step length. The amount the
trust radius is increased or decreased can be adjusted with \Key{TR
FAC}. The default values of RTENBD, RTENGD, RTRJMN and RTRJMX are
0.4D0, 0.8D0, 0.1D0 and 3.0D0 respectively.

\item[\Key{TRUSTR}]\verb| |
\newline
\verb|READ(LUCMD,*) TRSTRA|

Set initial trust radius for calculation. This will also be the
maximum step length for the first iteration. The trust radius is
updated after each iteration depending on the ratio between predicted
and actual energy change. The default trust radius is 0.5D0.

\item[\Key{VISUAL}]\verb| |

Specifies that the molecule should be visualized,
writing a VRML file of the molecular geometry. {\em{No}} optimization
will be performed when this keyword is given. See also related
keywords \Key{VR-BON} and \Key{VR-EIG}.

\item[\Key{VRML}]\verb| |

Specifies that the molecule should be
visualized. VRML files describing both the initial and final geometry
will be written (as \verb|initial.wrl| and \verb|final.wrl|). The file
\verb|final.wrl| is  updated in each iteration, so that it always
reflects the latest geometry. See also related keywords \Key{VR-BON}
and \Key{VR-EIG}.

\item[\Key{VR-BON}]\verb| |

Only has effect together with \Key{VRML} or \Key{VISUAL}. Specifies
that the VRML files should include bonds between nearby atoms. The
bonds are drawn as grey cylinders, making it easier to see the
structure of the molecule. If \Key{VR-BON} is omitted, only the
spheres representing the different atoms will be drawn.

\item[\Key{VR-EIG}]\verb| |

Only has effect together with \Key{VRML} or \Key{VISUAL}. Specifies
that the eigenvectors of the molecule (that is the eigenvectors of the
Hessian, which differs from the normal modes as they are not
mass-scaled) should be visualized. These are written to the files
\verb|eigv_###.wrl|.

\end{description}

\begin{thebibliography}{}
 \bibitem[1]{MEST}{Modern Electronic Structure Theory}
 \bibitem[2]{Fletcher}{R. Fletcher, Practical Methods of Optimization
 vol. 1 - Unconstrained Optimization, Wiley, New York, 1981}
 \bibitem[3]{Peng}{C. Peng, P. Y. Ayala, H. B. Schlegel, M. J. Frisch,
 Using Redundant Internal Coordinates to Optimize Equilibrium
 Geometries and Transition States, manuscript dated May 1 1995}
 \bibitem[4]{VRML}{A. J. Robinson, VRML, Chemistry and the Web - A New
 Reality, Chemical Design Automation News, Vol. 10, No. 6, 50-55 (1995)}
 \bibitem[5]{Baker}{J. Baker, Techniques for Geometry Optimization: A
 Comparison of Cartesian and Natural Internal Coordinates,
 J. Comp. Chem., Vol. 14, No. 9, 1085-1100 (1993)}
 \bibitem[6]{Schlegel}{H. B. Schlegel, Optimization of Equilibrium
 Geometries and Transition Structures, J. Comp. Chem., Vol. 3, No. 2,
 214-218 (1982)}
\end{thebibliography}
