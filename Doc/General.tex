\chapter{General input module}\label{chap:general}

In this chapter the general structure of the input file for \siraba\
is described, as well as the possible input cards that can be entered in
the \Sec{*DALTON} input module. This input section should always begin an
input file for the \siraba\ program.

\section{General input to DALTON : \Sec{*DALTON INPUT}}\label{sec:general}

This input module describes the overall type of calculation that is to
be done, and also which of the four programs that \siraba\ consists of,
that is to be executed. It also contains two submodules describing the
performance of parallel calculations and the geometry optimization
routines. We note that this input module has to start all input files
for \siraba .

\begin{description}

\item[\Key{DIRECT}] The calculation is to be done in a direct
manner\index{direct calculation}, that
is, the two-electron integrals\index{two-electron integral} are to be
constructed ``on the fly'' 
and not written to disc as is the default. This keyword will only work
for SCF\index{SCF}\index{HF}\index{Hartree-Fock} wave functions, and
two-electron integrals (and differentiated 
two-electron integrals) will not be written to disc in any part of the
calculation. 

\item[\Key{INPTES}] Test the input of the \Sec{*GENERAL} input
module. The program will abort after the completion of the input test,
and no calculation will be executed.

\item[\Key{INTEGRALS}] Invoke the \her\ program for generating molecular one-
and two-electron integrals. See
Chapter~\ref{ch:hermit}\index{one-electron
integral}\index{two-electron integral}.

\item[\Key{ITERAT}]\verb| |\newline
\verb|READ (LUCMD, '(I5)') ITERNR|

Tells the program at which iteration to start the geometry
optimization\index{iteration number}. Note that this will {\em not}
affect which molecule 
input file that is going to be read, as this have to be handled by the
shell script\index{shell script}. It only determines what number the
output of the predicted molecular geometry will be.

\item[\Key{MAX IT}]\verb| |\newline
\verb|READ (LUCMD, '(I5)') ITERMX|

Change the maximum number of geometry iterations\index{iteration
number}\index{geometry iteration} that can be done. Default
is~20. This number has to be increased in Intrinsic Reaction Coordinate
(IRC)\index{IRC}\index{intrinsic reaction coordinate} or dynamical
trajectory studies\index{dynamics}, as these 
usually require a much larger number of iterations.

\item[\Key{OPTIMI}] Do a geometry walk\index{geometry walk}. If no
input is given in the 
\Sec{WALK} input submodule, an optimization of the molecular
geometry\index{geometry optimization} to a stationary point with no
negative Hessian eigenvalues\index{Hessian} (a 
local minimum) will be done. However, this may changed be appropriate
keywords in the submodule \Sec{WALK}, and we refer to examples in
the chapter on potential energy surfaces
(Chapter~\ref{ch:geometrywalks}), and  subsection~\ref{sec:abawalk}
describing the input cards for the \Sec{WALK} submodule for a more
detailed description of possible options.

\item[\Key{PARALL}] Denotes that the calculation of two-electron
integrals\index{two-electron integral}
are to be done in parallel\index{parallel calculation}. This also
implies that the calculation is 
done without writing two-electron integrals to disc. This keyword only
applies to SCF\index{SCF}\index{HF}\index{Hartree-Fock} wave
functions, but all two-electron integral 
evaluations in an SCF calculation will be done parallel. More details
about the parallelization strategy in \siraba\ can be found
elsewhere~\cite{pndjhapdkrthhkcpl253}.

The keyword requires that the program has been installed and compiled
with the appropriate preprocessor directives for an MPI
installation\index{MPI}, or the construction of a slave program for
the PVM installation\index{PVM}.  

If MPI\index{MPI} is used as message passing\index{message passing}
interface, no further keywords are 
needed, as the number of nodes will be set equal to the number of 
nodes asked for when submitting the job. However, if PVM\index{PVM} is used as
message passing interface, the number of nodes needs to be given in the
\Sec{PARALL} submodule input, and this number have to be equal to
the number of nodes asked for when submitting the job. Note also that
in order to evaluate the parallelization efficiency\index{parallel
efficiency}, a print level of 
at least 2 is needed in the \Sec{PARALL} submodule.

\item[\Key{PRESORT}]\index{integral sort} Requests that the
two-electron integrals should be 
sorted and that the integral transformation routines of Bj\o rn
Roos should be used during execution of the program. This will only
work on 32-bit architectures, but is needed in order to be able to run
with more than 255 basis functions.

\item[\Key{PRINT}]\verb| |\newline
\verb|READ (LUCMD, *) IPRUSR|
\index{print level}
Reads in the print level that is to be used the rest of the subsequent
calculations. Default is a print level of 0.

\item[\Key{PROPERTIES}] Invoke the \aba\ program for the evaluation of static
and dynamic properties. See Chapter~\ref{ch:abacus}.

\item[\Key{RESPONSE}]

Invoke the \resp\ program for the evaluation of static and dynamic
properties. See Chapter~\ref{ch:response}.

\item[\Key{RUN ALL}]

Invoke all the programs \her , \sir , \resp , and \aba\ for a single point
     calculation. 

\item[\Key{RUN PROPERTIES}]

Invoke the programs \her , \sir , and \aba\ for a single point
calculation.

\item[\Key{RUN RESPONSE}] 

Invoke all the programs \her , \sir , and \resp\ for a single point
calculation.

\item[\Key{TOTSYM}] Consider only totally symmetric
perturbations\index{symmetry}. 
This option only affects geometric perturbations and static
electric-field perturbations requested through the keyword \Key{POLARI}. 

\item[\Key{WAVE FUNCTIONS}]

Invoke the \sir\ program for the evaluation of SCF, MP2, and MCSCF
wave
functions\index{SCF}\index{HF}\index{Hartree-Fock}\index{MP2}\index{M\o
ller-Plesset}\index{MCSCF}. See Chapter~\ref{chap:inpref}.  
\end{description}

\subsection{End of General input: \Sec{END OF}}

The last input card for the general input section (\Sec{*DALTON INPUT})
may be \Sec{END OF}.

\subsection{General: \Sec{MINIMIZE}}\label{subsec:minimize}

This submodule is driver for geometry optimizations, although we note
that the \Sec{WALK} module contains another second-order gometry
optimization driver-\index{geometry optimization}. The \Sec{MINIMIZE} module
contains both first and second-order methods for
energy\index{first-order optimization}\index{second-order
optimization}\index{geometry optimization}
minimization (geometry optimization). Most of the Hessian
updating\index{Hessian update}
schemes were taken from ref.\cite{thkrprt95} and \cite{Fletcher}. The
implementation of redundant internal coordinates\index{redundant
internal coordinates} follows the work done
by Peng {\it et al.\/}~\cite{Peng}. In addition to this, several
keywords for VRML visualization are included~\cite{VRML}.

\begin{description}

\item[\Key{1STORD}]\verb| |

Use default first-order method\index{first-order optimization}. This
means that the BFGS update will\index{BFGS update}
be used, and the optimization carried out in redundant internal
coordinates\index{redundant internal coordinates}. Same effect as the
combination of the two keywords 
\Key{BFGS} and \Key{REDINT}. 

\item[\Key{2NDORD}]\verb| |

Use default second-order method\index{second-order optimization}. The
level-shifted Newton method and 
Cartesian coordinates\index{cartesian coordinate} are used. Identical
to specifying the two 
keywords \Key{NEWTON} and \Key{CARTES}.

\item[\Key{BAKER}]\verb| |

Activates the convergence criteria of Baker \cite{Baker}. The minimum is then said
to be found, when the largest element of the gradient vector (in
Cartesian or redundant internal coordinates) falls below 3.0D-4 and
either the energy change from the last iteration is less than 1.0D-6
or the largest element of the predicted step vector is less 3.0D-4.

\item[\Key{BFGS}]\verb| |

Specifies the use of a first-order method\index{first-order
optimization} with the 
Broyden-Fletcher-Goldfarb-Shanno (BFGS) update\index{BFGS update} formula for
optimization. This is the preferred first-order method.

\item[\Key{CARTES}]\verb| |

Indicates that Cartesian coordinates should be used in the
optimization. This is the default for second-order methods.

\item[\Key{CONDIT}]\verb| |
\newline
\verb|READ(LUCMD,*) ICONDI|

Set the number of convergence criteria\index{convergence criteria}
that should be fulfilled before 
convergence occurs. There are three different convergence thresholds,
one for the energy\index{energy change}, one for the gradient
norm\index{norm of gradient} and one for the step
norm\index{norm of step}. The possible values for this variable is
therefore between 1 and 
3. Default is 2. The three convergence thresholds can be adjusted with
the keywords \Key{ENERGY}, \Key{GRADIE} and \Key{STEP T}.

\item[\Key{DFP}]\verb| |

Specifies that a first-order\index{first-order optimization} method with the
Davidon-Fletcher-Powell (DFP) update\index{DFP update} formula should be used
for optimization.

\item[\Key{DISPLA}]\verb| |
\newline
\verb|READ(LUCMD,*) DISPLA|

Read one more line containing the norm of the displacement vector to
be used during numerical evaluation of the molecular gradient, as is
needed when doing geometry optimizations with CI or MP2 wave
functions. Default is $1.0\cdot 10^{-3}$ a.u.

\item[\Key{ENERGY}]\verb| |
\newline
\verb|READ(LUCMD,*) THRERG|

Set the convergence threshold for the energy. This is one of the three
convergence thresholds\index{convergence criteria} (the keywords
\Key{GRADIE} and \Key{STEP T}
control the other two). Default value is the maximum of 1.0D-6 and two
times the threshold for the wave function gradient.

\item[\Key{GRADIE}]\verb| |
\newline
\verb|READ(LUCMD,*) THRGRD|

Set the convergence threshold for the gradient norm. This is one of
the three convergence thresholds (the keywords \Key{ENERGY} and
\Key{STEP T} control the other two). Default value is the maximum of
1.0D-6 and two times the threshold for the wave function gradient.

\item[\Key{GRDINI}]\verb| |

Specifies that the Hessian\index{hessian reinitialization} should be
reinitialized every time the norm 
of the gradient is larger than norm of the gradient two iterations
earlier. This keyword should only be used when it's difficult to
obtain a good approximation to the Hessian during optimization. Only
applies to first-order methods\index{first-order optimization}.

\item[\Key{HESFIL}]\verb| |

Specifies that the initial Hessian\index{Hessian, initial} should be
read from the file 
\verb|DALTON.HES|. This applies to first-order methods, and the
Hessian in the file must have the correct dimensions. This option
overrides other options for the initial Hessian.

Each time a Hessian is calculated or updated\index{Hessian update},
it's written to this 
file. If an optimization is interrupted, it can be restarted with the
last geometry and the Hessian in \verb|DALTON.HES|, minimizing the
loss of information. Another useful possibility, is to transfer
the Hessian from a calculation on the same molecule with another
(smaller) basis. Finally, one can go in and edit the file directly to
set up a specific forcefield.

\item[\Key{INIRED}]\verb| |

Specifies that the initial Hessian\index{initial Hessian} should be
diagonal in redundant 
internal coordinates, and it is then transformed to Cartesian
coordinates. This only applies to first-order
optimizationsindex{first-order optimization} in
Cartesian coordinates\index{cartesian coordinate}.

\item[\Key{INITEV}]\verb| |
\newline
\verb|READ(LUCMD,*) EVLINI|

The default initial Hessian\index{Hessian index} for first-order
minimizations\index{first-order optimization} is the 
identity matrix when Cartesian coordinates are used, and a diagonal
matrix when redundant internal coordinates are used. If \Key{INITEV}
is used all the diagonal elements (and therefore the eigenvalues) are
set equal to the value EVLINI. This option only has effect when
first-order methods are used and \Key{INITHE} and \Key{HESFIL} are 
non-present.

\item[\Key{INITHE}]\verb| |

Specifies that the initial Hessian\index{initial Hessian} should be
calculated (analytical Hessian), thus yielding a first step that is
identical to second-order methods. This provides an excellent starting
point for first-order methods, but should only be used when the
Hessian can be calculated within a reasonable amount of time. It has only
effect for first-order methods and overrides the keywords
\Key{INITEV} and \Key{INIRED}. It has no effect when \Key{HESFIL} has
been specified.

\item[\Key{INTERA}]\verb| |

Specifies a interactive run\index{interactive geometry
optimization}. The energy, gradient 
norm\index{norm of gradient}, step length\index{norm of step} and
Hessian index\index{Hessian index} is then written to the unit LUERR(=0) 
in each iteration, allowing easy monitoring of the optimization.

\item[\Key{MAX IT}]\verb| |
\newline
\verb|READ(LUCMD,*) ITRMAX|

Read the maximum number of geometry iterations\index{geometry
iteration}. Default value is 25. 

\item[\Key{MAX RE}]\verb| |
\newline
\verb|READ(LUCMD,*) MAXREJ|

Read maximum number of rejected steps\index{rejected geometry step} in
each iterations, default is 
5.

\item[\Key{NEWTON}]\verb| |

Specifies that a second-order level-shifted Newton method should
be used for optimization. This is the default method.

\item[\Key{NOBREA}]\verb| |

Disables breaking of symmetry\index{symmetry breaking}. The geometry will
be optimized within the given symmetry, even if a non-zero Hessian
index is found. The default is to let the symmetry be broken until
a minimum is found with a Hessian index\index{Hessian index} of zero. This option only has
effect when second-order methods are used.

\item[\Key{NOREIN}]\verb| |

When first-order methods are used, the Hessian\index{Hessian
reinitialization} is reinitialized every 
time the Hessian index\index{Hessian index} becomes non-zero (due to negative
eigenvalues). This guarantees that the Hessian describes a minimum,
but valuable information gathered in the Hessian may be
lost.

\Key{NOREIN} disables this reinitialization, relying on the
optimization method to restore the Hessian to its correct structure
(by locating the area near the minimum). This option is particularly
useful in conjunction with the keyword \Key{INITHE}, as it is
not meaningful to calculate the Hessian at the initial geometry, then reset
it to the identity matrix because some negative eigenvalue showed up.

\item[\Key{NOTRUS}]\verb| |

Turns off the trust radius\index{trust radius}, so that a full Newton step
is taken in each iteration. This should be used with caution, as
global convergence is no longer guaranteed. If long steps are desired,
it is safer to adjust the initial trust radius and the limits for the
actual/predicted energy ratio. 

\item[\Key{PREOPT}]\verb| |\newline
\begin{verbatim}
               READ(LUCMD,*) NUMPRE
               DO 144 I = 1, NUMPRE 
                  READ(LUCMD,*) PREBTX(I)
 144           CONTINUE
\end{verbatim}

First we read the number of basis sets\index{basis set} that should be used for
preoptimization\index{preoptimization}, then we read those basis set
names as strings. These 
sets will be used for optimization in the order they appear in the
input. One should therefore place the smaller basis at the top. After
the preoptimization, optimization is performed with the basis
specified in the molecule input file.

\item[\Key{PRINT}]\verb| |
\newline
\verb|READ(LUCMD,*) IPRINT|

Set print level for this module.  Read one more line containing print
level. Default value is 0.

\item[\Key{PSB}]\verb| |

Specifies that a first-order method\index{first-order optimization} with the
Powell-symmetric-Broyden (PSB)\index{PSB update} update formula should be used
for optimization.

\item[\Key{RANKON}]\verb| |

Specifies that a first-order method\index{first-order optimization}
with the rank 
one update\index{rank one update} formula should be used for optimization.

\item[\Key{REDINT}]\verb| |

Specifies that redundant internal coordinates\index{redundant internal
coordinates} should be used in the 
optimization. This is the default for first-order
methods\index{first-order optimization}. 

\item[\Key{REJINI}]\verb| |

Specifies that the Hessian should be reinitialized\index{Hessian
reinitialization} after every 
rejected step\index{rejected geometry step}, as a rejected step
indicates that the Hessian models the 
true potential surface poorly. Only applies to first-order
methods\index{first-order optimization}.

\item[\Key{SCHLEG}]\verb| |

Specifies that a first-order method\index{first-order optimization}
with Schlegel's updating scheme\index{Schlegel update}
\cite{Schlegel} should be used. This makes use of all previous
displacements and gradients, not just the last, to update the
Hessian.

\item[\Key{SP BAS}]\verb| |
\newline
\verb|READ(LUCMD,*) SPBSTX|

Read a string containing the name of a basis set. When the geometry
has converged, a single-point energy will be calculated using this
basis set\index{basis set}.

\item[\Key{STEEPD}]\verb| |

Specifies that the first-order steepest descent\index{first-order
optimization}\index{steepest descent} method should be
used. No update is done on the Hessian, so the optimization will be
guided by the gradient alone. The ``pure'' steepest descent method is
obtained when the Hessian is set equal to the identity matrix. Each
step will then be the negative of the gradient vector, and the
convergence towards the minimum will be extremely slow. However, this
option can be combined with other initial Hessians in Cartesian or
redundant internal coordinates\index{cartesian
coordinate}\index{redundant internal coordinates}, giving a method
where the main feature 
is the lack of Hessian updates.

\item[\Key{STEP T}]\verb| |
\newline
\verb|READ(LUCMD,*) THRSTP|

Set the convergence\index{convergence criteria} threshold for the step
norm. This is one of the 
three convergence thresholds (the keywords \Key{ENERGY} and
\Key{GRADIE} control the other two). Default value is 1.0D-6.

\item[\Key{SYMTHR}]\verb| |
\newline
\verb|READ(LUCMD,*) THRSYM|

Determines the gradient threshold for breaking of the
symmetry\index{symmetry breaking}. That
is, if the index of the Hessian\index{Hessian index} is non-zero when the gradient norm
drops below this value, the symmetry is broken to avoid unnecessary
iterations within the wrong symmetry. This option only applies to
second-order\index{second-order optimization} methods and when the
keyword \Key{NOBREA} is 
not present. The default value of this threshold is 1.0D-3.

\item[\Key{TR FAC}]\verb| |
\newline
\verb|READ(LUCMD,*) TRSTIN, TRSTDE|

Read two factors that will be used for increasing and decreasing the
trust radius\index{trust radius} respectively. Default values are
1.2D0 and 0.7D0. 

\item[\Key{TR LIM}]\verb| |
\newline
\verb|READ(LUCMD,*) RTENBD, RTENGD, RTRJMN, RTRJMX|

Read four limits for the ratio between the actual and predicted
energies. This ratio indicates how good the step is, that
is how accurately the the quadratic model describes the true energy
surface. If the ratio is below RTRJMN or above RTRJMX, the step is
rejected. With a ratio between RTRJMN and RTENBD, the step is
considered bad an the trust radius\index{trust radius} decreased to
less than the step 
length. Ratios between RTENBD and RTENGD are considered satisfactory,
the trust radius is set equal to the norm of the step. Finally ratios
above RTENGD (but below RTRJMX) indicate a good step, and the trust
radius is given a value larger than the step length. The amount the
trust radius is increased or decreased can be adjusted with \Key{TR
FAC}. The default values of RTENBD, RTENGD, RTRJMN and RTRJMX are
0.4D0, 0.8D0, 0.1D0 and 3.0D0 respectively.

\item[\Key{TRUSTR}]\verb| |
\newline
\verb|READ(LUCMD,*) TRSTRA|

Set initial trust radius\index{trust radius} for calculation. This
will also be the 
maximum step length for the first iteration. The trust radius is
updated after each iteration depending on the ratio between predicted
and actual energy change. The default trust radius is 0.5D0.

\item[\Key{VISUAL}]\verb| |

Specifies that the molecule should be
visualized\index{visualization}\index{VRML}, 
writing a VRML file of the molecular geometry. {\em{No}} optimization
will be performed when this keyword is given. See also related
keywords \Key{VR-BON} and \Key{VR-EIG}.

\item[\Key{VRML}]\verb| |

Specifies that the molecule should be
visualized\index{visualization}\index{VRML}. VRML files describing
both the initial and final geometry 
will be written (as \verb|initial.wrl| and \verb|final.wrl|). The file
\verb|final.wrl| is  updated in each iteration, so that it always
reflects the latest geometry. See also related keywords \Key{VR-BON}
and \Key{VR-EIG}.

\item[\Key{VR-BON}]\verb| |

Only has effect together with \Key{VRML} or \Key{VISUAL}. Specifies
that the VRML files should include bonds\index{bonded atoms} between nearby
atoms\index{visualization}\index{VRML}. The 
bonds are drawn as grey cylinders, making it easier to see the
structure of the molecule. If \Key{VR-BON} is omitted, only the
spheres representing the different atoms will be drawn.

\item[\Key{VR-EIG}]\verb| |

Only has effect together with \Key{VRML} or
\Key{VISUAL}\index{visualization}\index{VRML}�index{eigenvector}.
Specifies that the eigenvectors of the molecule (that is the eigenvectors of the
Hessian, which differs from the normal modes as they are not
mass-scaled) should be visualized. These are written to the files
\verb|eigv_###.wrl|.

\end{description}

\subsection{Parallel calculations : \Sec{PARALL}}

This submodule controls the performance of the parallel
version\index{parallel calculation}
of \siraba . The implementation has been described
elsewhere~\cite{pndjhapdkrthhkcpl253}. When PVM\index{PVM} is used as
message passing interface\index{message passing}, the 
keyword \Key{NODES} is required, otherwise all keywords are
optional.

\begin{description}

\item[\Key{DEBUG}]\verb| |
Transfers the print level from the master\index{master}\index{slave}
to the slaves, otherwise the print level on the slaves will always be
zero. Only for debugging purposes. 

\item[\Key{DEGREE}]\verb| |\newline
\verb|READ (LUCMD,*) NDEGDI|

Determines the percent of available tasks\index{parallel tasks} that
is to be distributed in 
a given distribution of tasks, where a distribution of tasks is defined
as the process of giving batches to {\em all} slaves. The default is
5\% , which ensures that each slave will receive 20 tasks during one
integral evaluation, which will give a reasonable control with the
idle time of each slave.

\item[\Key{ENCODE}]\verb| |\newline
\verb|READ (LUCMD, '(A7)') WORD|

This keyword only applies if PVM\index{PVM} has been chosen as message passing
interface. Three different options exists:

\begin{list}{--}{}
\item\verb|PVMDEFA|
\item\verb|PVMRAW|
\item\verb|PVMINPL|
\end{list}

The default is \verb|PVMDEFA|. We refer to PVM\index{PVM} manuals for a
closer description of the different ways of encoding the transfer of
data between nodes.

\item[\Key{NODES}]\verb| |\newline
\verb|READ (LUCMD,*) NODES|

When MPI\index{MPI} is used as message passing interface, the
default value is the number of nodes that has been assigned to the
job, and these nodes will be partitioned into one master\index{master} and
\verb|NODES-1| slaves\index{slave}. When PVM is used for message
passing\index{message passing}, the default is 0,  and it is 
therefore required to specify the number of nodes to be used in the
calculation when PVM is used. \verb|NODES| will in such calculations
correspond to the number of slaves the user would like to use in the
calculation. Note that this number may often have to be adjusted in
accordance with limits in the queuing system of various computers.

\item[\Key{NTASK}]\verb| |\newline
\verb|READ (LUCMD,*) NTASK|

The number of tasks to send to each node when distributing the
calculation of two-electron integrals\index{two-electron
integral}. The default is 1. A task is 
defined as a shell of atomic integrals\index{basis functions, shell
of}, a shell being an input 
block. One may therefore increase the number of shells given to each
node in order to reduce the amount of communication. However, the
program uses dynamical allocation of work to each 
node, and thus this option should be used with some care, as too large
tasks may cause the dynamical load balancing to fail\index{load
balancing}, giving an 
overall decrease in efficiency\index{parallel efficiency}. The
parallelization is also very 
coarse grained, so that the amount of
communication\index{communication} seldom represents 
any significant problem. Will be come obsolete, and replaced by \Key{DEGREE}.

\item[\Key{PRINT}]\verb| |\newline
\verb|READ (LUCMD, '(A7)') KPRINT|

Read in the print level for the parallel calculation. A print level of
at least 2 is needed in order to be able to evaluate the
parallelization efficiency\index{parallel efficiency}. A complete
timing for all nodes will be 
given if the print level is 4 or higher.
\end{description}

\subsection{Geometry optimization: \Sec{WALK}}
\label{sec:abawalk}

Directives controlling one of the two second-order
geometry\index{second-order optimization}
optimizations  as well as the
execution of dynamical walks and numerical
differentiation\index{numerical differentiation} in
calculations of Raman intensities and optical activity\index{Raman
intensity}\index{Raman optical activity}\index{ROA},
appear in the \Sec{WALK} section.

\begin{description}
\item[\Key{DISPLA}]\verb| |\newline
\verb|READ (LUCMD, *) DISPLC|

Displacement taken in a numerical differentiation\index{numerical
differentiation}. This applies both 
for a numerical molecular Hessian\index{Hessian}, as well as in
calculation of Raman intensities and optical activity\index{Raman
intensity}\index{Raman optical activity}\index{ROA}. Read one more line specifying
value~(*).  Default is~10$^{-4}$ a.u. However, note that this variable
do not determine the displacements used when evaluating numerical 
gradient for use in first-order geometry optimizations with MP2 or CI
wave functions\index{MP2}\index{CI}\index{M\o
ller-Plesset}\index{configuration interaction}, which is controled by
the \Key{DISPLA} keyword in the 
\Sec{MINIMI} module.

\item[\Key{DYNAMI}] Perform a ``dynamic walk''\index{dynamics}: integrate the
classical equations of motion\index{equation of motion} for the nuclei
analytically on a locally 
quadratic surface. The method is discussed in
Ref.~\cite{theuhjajcpl173} as well as in Section~\ref{sec:dynamic}. 

\item[\Key{EIGEN}] Take a step to the boundary of the trust
region\index{trust radius}
along the eigenvector mode\index{eigenvector} specified by \Key{MODE}.

\item[\Key{FRAGME}]\verb| |\newline
\verb|READ (LUCMD, *) NIP|\newline
\verb|READ (LUCMD, *) (IPART(IP), IP = 1, NIP)|

Identify which fragments\index{molecular fragments} atoms belong to in
a dynamic walk.  Read one more line specifying the number of
atoms (the total number of atoms in the molecule), then one more
line identifying which fragment an atom belongs to. The atoms in the
molecule are given a number, different for each fragment. See also the
discussion in Sec.~\ref{sec:dynamic}.

\item[\Key{GRDEXT}] Perform a gradient extremal-based\index{gradient
extremal} optimization. The algorithm used in this kind of optimization is
thoroughly described in Ref.\cite{pjhjajthtca73}. This is default walk
type if the  index of the critical point searched is higher than
1. See also the discussion in Sec.~\ref{sec:gradext}.

\item[\Key{HARMON}]\verb| |\newline
\verb|READ (LUCMD, *) ANHFAC|

Threshold for harmonic dominance.  Read one
more line specifying value. Default is~100. This is  another
way of changing the criterion for changes of the trust
radius\index{trust radius}. See also the keyword \Key{TRUST}.

\item[\Key{IMAGE}] Locate a transition state\index{transition state} using a
trust-region-based image surface\index{image surface} minimization.
Note that only a 
point with a Hessian index of~1 can currently be located with this
method, not higher-order stationary points. See also the discussion in
Sec.~\ref{sec:image}. 

\item[\Key{INDEX}]\verb| |\newline
\verb|READ (LUCMD,*) IWKIND|

Desired Hessian index\index{Hessian index} (strictly, of the
totally symmetric block of the Hessian) at the optimized geometry.
Read one more line specifying value.  Default is~0 (minimum).
Note that a stationary point with the wrong Hessian index will not
be accepted as an optimized geometry.

\item[\Key{IRC}]\verb| |\newline
\verb|READ (LUCMD, *) IRCSGN|

Set the geometry walk to be  an
Intrinsic Reaction Coordinate (IRC)\index{intrinsic reaction
coordinate}\index{IRC} as described in
Ref.~\cite{kfacr14,pmjcp88}. Read one  
more line containing the sign (-1 or 1) of the reaction coordinate. It
cannot be decided in advance which reaction pathway a specific sign is
associated with. See also the discussion in Sec.~\ref{sec:irc}.

\item[\Key{ISOTOP}]\verb| |\newline
\verb|READ (LUCMD, *) NIS|\newline
\verb|READ (LUCMD, *) (ISOTPS(IS), IS = 1, NIS)|

Specify the isotopic constitution\index{isotopic constitution} of the
molecule under investigation. 
This is most interesting in dynamic walks\index{dynamics}, as well as
when using mass-scaled atomic coordinates\index{mass-weighted
coordinate}. Note that the internal structure of 
\aba\ uses Cartesian coordinates, and for vibrational analysis alone
the keyword \Key{ISOTOP} in the \Sec{VIBANA} input section is to be
preferred. Note also that for defining center of mass
coordinates, there is a similar \verb|ISOTOP| keyword in the general
input module, and this will become the only keyword for specifying
isotopic substitutions in later versions of the program. The
\Key{ISOTOP} keyword in this input module (\Sec{WALK}) is to become
obsolete. 

\item[\Key{KEEPSY}] Ensure that the symmetry of the molecule is not
broken. The threshold for determining a mode as breaking symmetry is
controled by the keyword \Key{ZERGRD}.

\item[\Key{MASSES}] Mass-scale the atomic coordinates\index{mass-weighted
coordinate}.  This is the
default for dynamic walks\index{dynamics}, gradient
extremal\index{gradient extremal} walks and in calculations
of Intrinsic Reaction Coordinates (IRCs)\index{intrinsic reaction
coordinate}\index{IRC}. 

\item[\Key{MAXNUC}]\verb| |\newline
\verb|READ (LUCMD, *) XMXNUC|

Maximum displacement\index{displacement of atom} allowed for any one
atom as a result of the 
geometry update.  Read one more line specifying value.  Default
is~0.5. 

\item[\Key{MAXTRU}]\verb| |\newline
\verb|READ (LUCMD, *) TRUMX1|

Set the maximum arc length in an Intrinsic Reaction Coordinate (IRC)
walk\index{intrinsic reaction coordinate}\index{IRC}. Read one more
line containing the maximum arc length. 
Default is~0.10. Note that this arc length is also affected by the
\Key{TRUST} keyword, and if both are specified, the arc length will
be set to the minimum value of these to.

\item[\Key{MODE}]\verb| |\newline
\verb|READ (LUCMD,*) IMODE|

Mode to follow in level-shifted Newton optimizations for transition
states\index{transition state}.  Read one more line specifying
mode. Default is to follow 
the lowest mode (mode~1). 

\item[\Key{MODFOL}] Perform a mode-following (level-shifted
Newton) optimization. This is the default for minimizations and
localization of transition states. See also discussion in
Section~\ref{sec:modfol}. 

\item[\Key{MOMENT}]\verb| |\newline
\verb|      READ (LUCMD, *) NSTMOM|\newline
\verb|      DO 265 IP = 1, NSTMOM|\newline
\verb|         READ (LUCMD, *) ISTMOM(IP), STRMOM(IP)|\newline
\verb|  265 CONTINUE|

Initial momentum for a dynamic walk\index{dynamics}\index{momentum}.
Read one more line specifying 
the number of modes to which there is added an initial momentum. Then
read one line for each of these modes, containing first the number of
the mode, and then the momentum. The default is to have no momentum.
See also the section describing how to perform a dynamic
walk, Sec.~\ref{sec:dynamic}.

\item[\Key{NATCON}] Use the natural connection\index{natural connection} when orthogonalizing
the predicted molecular orbitals at the new geometry. By default the
symmetric connection is used.

\item[\Key{NEWTON}] Use a strict Newton-Raphson\index{Newton-Raphson
step} step to update 
the geometry. This means that no trust region will be used.

\item[\Key{NOGRAD}]\verb| |\newline
\verb|READ (LUCMD, *) NZEROG|\newline
\verb|READ (LUCMD, *) (IZEROG(I), I = 1, NZEROG)|

 Set some gradient elements to zero.  Read
one more line specifying how many elements to zero, then one
or more lines listing their sequence numbers.

\item[\Key{NOORTH}] The predicted molecular orbitals at the new
geometry are {\em not} orthogonalized. Default is that the orbitals
are orthogonalized with the symmetric connection. Orthogonalization
can also be done with the natural
connection~\cite{joklbkrthpjtca90}. See the keyword \Key{NATCON}. 

\item[\Key{NOPRED}] No prediction of the energy of the wave function
at updated geometry. 

\item[\Key{NUMERI}] Do a numerical differentiation\index{numerical
differentiation}, for instance when 
calculating Raman intensities or Raman optical activity, see
Sections~\ref{sec:ramanint} and~\ref{sec:vroa}.

\item[\Key{PRINT}]\verb| |\newline
\verb|READ (LUCMD,*) IPRWLK|

Set the print level in the prediction of new geometry steps.  Read one 
more line containing print level. Default value is the value of
\verb|IPRDEF| in the general input module.

\item[\Key{RATLIM}]\verb| |\newline
\verb|READ (LUCMD, *) RTMIN, RTRGOD, REJMIN, REJMAX|

Limits on ratios between predicted and
observed energy change.  Read one more line specifying four
values~(*).  These are respectively the bad prediction ratio, good
prediction ratio, low rejection ratio and high rejection ratio.
Defaults are~0.4, 0.8, 0.1, and 1.9.

\item[\Key{REJECT}] Signals that the previous geometry step was
rejected\index{rejected geometry step}, and the trust
region\index{trust radius} is
reduced. This keyword is 
used in case of restarts to tell the program that when the program
stopped, the last geometry was in fact rejected.

\item[\Key{REPS}]\verb| |\newline
\verb|READ (LUCMD, *) NREPS|\newline
\verb|READ (LUCMD, *) (IDOREP(I), I = 1, NREPS)|

Consider perturbations of selected
symmetries only.  Read one more line specifying how many
symmetries, then one line listing the desired symmetries. Note that
only those symmetries previously defined to be true with the keyword
\Key{REPS} from the \aba\ input modules will be calculated. This
keyword thus represents a subset of the \Key{REPS} of the general
input module.

\item[\Key{SCALE}]\verb| |\newline
\verb|      READ (LUCMD, *) NUMNUC|\newline
\verb|      DO 7000 INUC = 1, NUMNUC|\newline
\verb|         READ (LUCMD, *) IATOM,(SCALCO(J,IATOM), J = 1, 3)|\newline
\verb| 7000 CONTINUE|

Scale the atomic coordinates.  Read one more
line specifying how many atoms to scale, then one line for
each of these atoms~(*) specifying the atom number and scale
factors for all three Cartesian coordinates. Default is no scaling of
the atomic coordinates. 

\item[\Key{RESTART}] Tells the program that this is a restarted
geometry optimization\index{restart, geometry optimization} and that
information may therefore be available 
on the \verb|DALTON.WLK| file.

\item[\Key{STRICT}] Strict mode following. Obsolete keyword. Do not use.

\item[\Key{TOLERA}]\verb| |\newline
\verb|READ (LUCMD, *) TOLST|

Threshold for convergence of the geometry optimization (on gradient
norm).  Read one more line specifying the threshold~(*).  Default
is~10$^{-5}$. 

\item[\Key{TRUST}]\verb| |\newline
\verb|READ (LUCMD, *) TRUSTR, TRUSTI, TRUSTD|

Trust region information\index{trust radius}.  Read one more
line specifying three values~(*): initial trust radius, factor by
which radius can be incremented, and factor by which it can be
decremented.  Defaults are~0.5, 1.2 and 0.7, respectively; initial
trust radius default is~0.3 if desired Hessian index is
greater than zero. In dynamic walks\index{dynamics} the trust radius
is by default put 
to 0.005, and in walks along an Intrinsic Reaction Coordinate
(IRC)\index{intrinsic reaction coordinate}\index{IRC} the
default trust radius is 0.020. For dynamical walks the default
increment and decrement factor is changed to~2.0 and~0.8
respectively. 

\item[\Key{ZERGRD}]\verb| |\newline
\verb|READ (LUCMD, *) ZERGRD|

Threshold below which gradient elements are
treated as zero.  Read one more line specifying value~(*). Default
is~10$^{-5}$. This keyword is mainly used for judging which modes are
symmetry breaking when using the keyword \Key{KEEPSY} as well as
when deciding what step to take when starting a walk from a transition
state.
\end{description}

