%%%%%%%%%%%%%%%%%%%%%%%%%%%%%%%%%%%%%%%%%%%%%%%%%%%%%%%%%%%%%%%%%%%
\chapter{Coupled Cluster calculations: CC}
\label{chap:ccgeneral}
%%%%%%%%%%%%%%%%%%%%%%%%%%%%%%%%%%%%%%%%%%%%%%%%%%%%%%%%%%%%%%%%%%%

In this chapter the general structure of the input for the 
coupled cluster program is described.

The complete input for the coupled cluster program appears as
sections in the input for the Sirius program, with the general
input in \Sec{*CC INPUT}. 

\section{General input for CC: \Sec{*CC INPUT}}\label{sec:ccgeneral}

In this section keywords for the coupled
cluster program are defined. In particular the coupled cluster 
model(s) and other parameters common to all submodules are specified.

\begin{description}
\item[\Key{CCS   }] 
        Run calculation for Coupled Cluster Singles (CCS) model. 
%
%\item[\Key{CC1A  }]   
%        Run calculation for CC1A (?) model. 
%
%\item[\Key{CC1B  }]    
%        Run calculation for CC1B (?) model. 
%
\item[\Key{CIS   }]    
        Run calculation for CI Singles (CIS) method. 
%
\item[\Key{MP2   }]    
        Run calculation for second-order M{\o}ller-Plesset 
        perturbation theory (MP2) method. 
%
\item[\Key{CC2   }]    
        Run calculation for CC2 model. 
%
%\item[\Key{CC(2) }]  
%        Run calculation for CC(2) model. 
%
\item[\Key{CCD   }]    
        Run calculation for Coupled Cluster Doubles (CCD) model. 
%
\item[\Key{CCSD  }]   
        Run calculation for Coupled Cluster Singles and Doubles model.
%
%\item[\Key{CCR(A)}] 
%        Run calculation for CCR(A) (?) model.
%
%\item[\Key{CCR(B)}]  
%        Run calculation for CCR(A) (?) model.
%
%\item[\Key{CC(3) }]  
        Run calculation for CC(3) (?) model.

\item[\Key{CCR(3)}] 
        Run calculation for CCR(3) (?) model.
%
%\item[\Key{CCR(T)}] 
%        Run calculation for CCR(T) (?) model.
%
\item[\Key{CC(T) }]  
        Run calculation for CCSD(T) (Coupled Cluster Singles and 
        Doubles with perturbational treatment of triples) model.
%
\item[\Key{CC3   }]     
        Run calculation for CC3 model.
%
\item[\Key{PRINT }]  \verb| |\newline
\verb|READ (LUCMD,'(I5)') IPRINT|

       Set print parameter for coupled cluster program
       (default is to take the value \verb+IPRUSR+, set in the general
       input section of dalton).
%
\item[\Key{HERDIR}] 
       Run coupled cluster program AO integral direct using HERMIT
       (Default is to run the program only AO integral direct
       if the \verb+DIRECT+ keyword was set in the general
       input section of dalton. If \verb+HERDIR+ is not specified the ERI
       program is used as integral generator.) 
%
\item[\Key{RESTAR}] 
       Try to restart the calculation from the cluster amplitudes,
       Lagrange multipliers, response amplitudes etc.\ stored on
       disk.
%
\item[\Key{NSYM  }] \verb| |\newline
       \verb|READ (LUCMD,'(I5)') MSYM2|

       Set number of irreducibel representations. 
       Due to difficulties to parse the number of irrep's from
       the integral program to the coupled cluster input section,
       this information must be specified in the coupled cluster
       input section, if one of the options \verb+.FROIMP+ or
       \verb+.NCCEXCI+ (see \verb+*CCEXCI+ subsection) are to be used.
       Note that \verb+.NSYM+ has to be set before these keywords.
 
\item[\Key{THRENR}] \verb| |\newline
       \verb|READ (LUCMD,*) THRENR|

       Set threshold for convergence of the ground state energy.
 
\item[\Key{THRLEQ}] \verb| |\newline
       \verb|READ (LUCMD,*) THRENR|

       Set threshold for convergence of the response equations.
 
\item[\Key{FROIMP}] \verb| |\newline
      \verb|READ (LUCMD,*) (NRHFFR(I),I=1,MSYM)|\newline
      \verb|READ (LUCMD,*) (NVIRFR(I),I=1,MSYM)|

      Specify for each irreducible representation how
      many orbitals should be frozen (deleted) for the coupled
      cluster calculation. In calculation the first \verb+NRHFFR(I)+
      orbitals will be kept frozen in symmetry class \verb+I+ and
      the last \verb+NVIRFR(I)+ orbitals will be deleted form the 
      orbital list.
 
\item[\Key{FROEXP}]  \verb| |\newline
    \verb|READ (LUCMD,*) (NRHFFR(I),I=1,MSYM)|\newline
    \verb|DO ISYM = 1, MSYM|\newline
    \verb|  IF (NRHFFR(ISYM.NE.0) THEN|\newline
    \verb|    READ (LUCMD,*) (KFRRHF(J,ISYM),J=1,NRHFFR(ISYM))|\newline
    \verb|  END IF|\newline
    \verb|END DO|\newline
    \verb|READ (LUCMD,*) (NVIRFR(I),I=1,MSYM)|\newline
    \verb|DO ISYM = 1, MSYM|\newline
    \verb|  IF (NVIRFR(ISYM.NE.0) THEN|\newline
    \verb|    READ (LUCMD,*) (KFRVIR(J,ISYM),J=1,NVIRFR(ISYM))|\newline
    \verb|  END IF|\newline
    \verb|END DO|

    Specify explicitly for each irreducible representation the
    orbitals that should be frozen (deleted) for the coupled cluster
    calculation.
 
\item[\Key{NSIMLE}] \verb| |\newline
  \verb|READ (LUCMD, *) NSIMLE|

  Set the maximum number of response equations that should be 
  solved simultaneously. Default is 0 which means that all
  compatible equations (same equation type and symmetry class) 
  are solved simultaneously.
 
\item[\Key{MAXRED}] \verb| |\newline 
  \verb|READ (LUCMD,'(I5)') MAXRED|

  Maximum dimension of the reduced space for the 
  solution of linear equations (default is \verb+MAXRED = 200+).
 
\item[\Key{MAXITE}] \verb| |\newline
  \verb|READ (LUCMD,'(I5)') MAXITE|

  Maximum number of iterations for wave function optimization.
  (default is \verb+MAXITE = 40+).
 
\item[\Key{MXDIIS}] \verb| |\newline
  \verb|READ (LUCMD,'(I5)') MXDIIS|

  Maximum number of iterations for DIIS algorithm
  before wave function information is discarded and a new DIIS 
  sequence is started. 
  (default is \verb+MXDIIS = 8+).
 
\item[\Key{MXLRV}] \verb| |\newline
  \verb|READ (LUCMD, *) MXLRV|

  Maximum number of trial vectors in the solution of 
  linear equations. If the number of trial vectors reaches this
  value, all trial vectors and there transformations with the
  Jacobian are skipped and the iterative procedure for the solution
  linear (i.e. the response) equations is restarted from the current 
  optimal solution. 
 
\item[\Key{FIELD }] \verb| |\newline
    \verb|READ (LUCMD,*) EFIELD|\newline
    \verb|READ (LUCMD,'(1X,A8)') LFIELD|

    Include external fields (operator labels \verb+EFIELD+)
    of strength \verb+EFIELD+.
 
\item[\Key{CCSTST}] 
   Test option which allows to run CCS finite field as pseudo CC2
   calculation. \verb+CCSTST+ disables all terms which depend on the
   double excitation amplitudes or multipliers.
%
\item[\Key{DEBUG }]  
   Test option: print additional debug output.
%
\item[\Key{NOCCIT}]
   No iterations in the wave function optimization is carried out.
%
%\item[\Key{NEWCAU}] % unfinished  option !
%   Solve Cauchy equations with different Cauchy order simultaneous. 
%   (The algorithm behind this is still in the test phase...)
%
\item[\Key{IMSKIP}] 
   Skip the calculation of some response intermediates
   in a restarted run.
    (which intermediates exactly?
    is there any check if it is really a restarted calculation?
    this questions applies also for all the other skip options
    down here!) 
%
\item[\Key{M1SKIP}] 
   Skip the calculation of the special zeroth-order Lagrange 
   multipliers for ground-excited state transition moments,
   the so-called $M$-vectors, in a restarted run.
%
\item[\Key{FRSKIP}] 
   Skip the calculation of the F-matrix transformed ?-vectors
   in a restarted run.
%
\item[\Key{R1SKIP}] 
   Skip the calculation of the first-order amplitude responses
   in a restarted run.
%
\item[\Key{L1SKIP}] 
   Skip the calculation of the first-order responses of the 
   ground-state Lagrange multipliers in a restarted run.
%
\item[\Key{RESKIP}] 
   Skip the calculation of the first-order responses of the 
   ground-state Lagrange multipliers in a restarted run.
%
\item[\Key{LESKIP}]  
   Skip the calculation of left eigenvectors
   in a restarted run.
%
\item[\Key{F1SKIP}] 
   Skip the calculation of F-matrix transformed first-order
   cluster amplitude responses in a restarted run.
%
\item[\Key{E0SKIP}] 
   Skip the calculation of the $E0$-vectors (?)
   in a restarted run.
%
\item[\Key{L0SKIP}]  
   Skip the calculation of the zeroth-order ground-state Lagrange
   multipliers in a restarted run.
%
\item[\Key{LISKIP}] 
   Skip the calculation of ???
   in a restarted run.
%
\item[\Key{B0SKIP}] 
   Skip the calculation of the $B0$-vectors (?)
   in a restarted run.
%
\item[\Key{O2SKIP}] 
   Skip the calculation of right hand side vectors for the 
   second-order cluster amplitude equations
   in a restarted run.
%
\item[\Key{R2SKIP}] 
   Skip the calculation of the 
   second-order cluster amplitude equations
   in a restarted run.
%
\item[\Key{X2SKIP}] 
   Skip the calculation of the $\eta^{(2)}$ intermediates (needed
   to build the right hand side vectors for the second-order 
   ground state Lagrangian multiplier response equations) 
   in a restarted run.
%
\item[\Key{F2SKIP}] 
   Skip the calculation of F-matrix transformed second-order
   cluster amplitude responses in a restarted run.
%
\item[\Key{L2SKIP}]  
   Skip the calculation of the second-order responses of the 
   ground-state Lagrange multipliers in a restarted run.
\end{description}

%All the special skip options for the restart of response calculations
%would better be placed in a seperate input subsection.
%(And the corresponding logical variables on a seperate 
%common block.)
