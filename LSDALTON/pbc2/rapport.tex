\documentclass[12pt,a4paper,english]{article}
\usepackage{ucs}
\usepackage[utf-8]{inputenc}
\usepackage{fontenc}
\usepackage{pstricks,graphicx,psfig}
\usepackage{babel,a4wide,times}
\usepackage{epsfig}
\usepackage{epic}
\usepackage{eepic}


\setlength{\parindent}{0in} %Ingen indent

\begin{document}
\newgray{gray1}{.50}
\newgray{gray2}{.70}
\newgray{gray3}{.90}
\newgray{gray4}{.95}

\title{ Short review of my linking to the integral code of Simen Reine}
  \author{Johannes Rekkedal} 
\date{March 2011}
\maketitle

The module \emph{lattice$\_$vectors} in lsutil/TYPE-DEF.f90 takes care of the positions to each atom in the different unit cells. 

The integrals ar computed in the file pbc2/pbc$\_$int.f90. 

The subroutine pbc$\_$overlap$\_$int computes the overlaps

\begin{verbatim}
pbc_overlap_int(lupri,luerr,setting,molecule,cutoff,nbast,overlap,sizeov,&
                ll,latt_cell,refcell,numvecs)
\end{verbatim}
The inputs of pbc$\_$overlap$\_$int 
\begin{verbatim}
INTEGER :: lupri,luerr,nbast,sizeov, numvecs
\end{verbatim}
nbast is the number of basis functions and numvecs is the amount of lattice vectors, sizeov is the dimension to the overlap matrix overlap.
\begin{verbatim}
TYPE(LSSETTING) :: SETTING
TYPE(moleculeinfo) :: latt_cell, refcell
TYPE(lvec_list_t),intent(INOUT) :: ll
\end{verbatim}
The pointer \emph{setting} points to \emph{latt$\_$cell} which contains information of the atomic positions in the corresponding unit cell. 
The pointer \emph{referencecell} contains information of the positons of each atom in the reference cell.
Further the pointer \emph{ll} has information about the unit cells and is used to find the atomic positions in each unit cell.
The variable
\begin{verbatim}
REAL(realk) :: cutoff 
\end{verbatim}
is for the moment not used in any subroutine, I will try to use it as a screening parameter. For the moment I just include integrals up to nearest neighbour interactions.

\begin{verbatim}
SUBROUTINE pbc_kinetic_int(lupri,luerr,setting,molecule,cutoff,nbast,& 
                           fock_mtx,sizef,ll,latt_cell,refcell,numvecs)
\end{verbatim}

The inputs in the routine for kinetic contributions are identical to the input elements in the subroutine for overlaps, this holds also for routines for nuclear attraction,
\begin{verbatim}
SUBROUTINE pbc_nucattrc_int(lupri,luerr,setting,molecule,cutoff,nbast,&
                            fock_mtx,sizef,ll,latt_cell,refcell,numvecs)
\end{verbatim}
and electron-electron repulsion,
\begin{verbatim}
SUBROUTINE pbc_electron_rep(lupri,luerr,setting,molecule,cutoff,nbast,&
                            fock_mtx,sizef,ll,latt_cell,refcell,numvecs)
\end{verbatim}.

The subroutine \emph{set$\_$pbc$\_$molecules}, calls to all the above mentioned routines, it will be given a better name in the future.

\begin{verbatim}
SUBROUTINE set_pbc_molecules(INPUT,SETTING,lupri,luerr,nbast,ll)
\end{verbatim}
The only nonmentioned variable which is given as input in the subroutine is the pointer \emph{INPUT}.
\begin{verbatim}
TYPE(DALTONINPUT) :: INPUT
\end{verbatim}



\end{document}
