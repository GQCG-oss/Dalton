\chapter{Interfacing to {\lsdalton}}\label{interfacing}

In this chapter we give an introduction to the normalization, basis set and ordering of atomic orbitals used in the {\lsdalton} program. 

This is provided in case you want to interface to the {\lsdalton} program or want to read one of the files written by the {\lsdalton} program, for instance 
the local orbitals.
Certain functionality can also be incorporated into other programs by linking to the
{\lsdalton}-library bundle, as outlined in section~\ref{sec:LSlib}.

\section{The Grand Canonical Basis}

Unlike all other Quantum Chemistry Programs the {\lsdalton} program, uses the so called grand canonical basis {\insertref} as the internal default basis. The results therefore cannot directly be compared to other programs unless the 
\begin{verbatim}
.NOGCBASIS
\end{verbatim}
keyword is specified. This keyword deactivates the use of the grand canonical basis and uses the input basis instead. 

\section{Basisset and Ordering}

In {\lsdalton} the default basis set are real-valued spherical harmonic Gaussian type
orbitals (GTOs)
\begin{equation}
G_{ilm}(\textbf{r}; a_{i};\textbf{A}) = S_{lm}(x_{A}; y_{A}; z_{A}) \exp(-a_{i}r^{2}_{A})
\end{equation}
where $S_{lm}(x_{A}; y_{A}; z_{A})$ is a real solid harmonic introduced in section 6.4.2 of Ref. [THE BOOK {\insertref}], (see Table 6.3).
Concerning the ordering of basis functions, it is one atomtype after the other as given
in the input, and for each atomtype, it is one atom after another as given in input. For
each atom increasing with angular momentum $l$, within one angular momentum $l$, the ordering 
is first the components (e.g. $p_{x},p_{y},p_{z}$), and then the contracted functions.



So for an atom consiting of 3 s functions 2 p functions and 1 d function, the ordering is
$\{1s; 2s; 3s; 1p_{x};1p_{y};1p_{z}; 2p_{x};2p_{y};2p_{z}; 1d_{xy}; 1d_{yz}; 1d_{z2-r2} ; 1d_{xz}; 1d_{x2-y2}\}$
In {\lsdalton} the ordering of the basis functions are $m = \{-l,\cdots,0,\cdots,l\}$ for a given
angular momentum $l$. Although for computational efficiency the p orbitals are treated as a
special case with $p = \{ p_{x},p_{y},p_{z} \}$ which corresponds to $m = \{1,-1,0 \}$.

\section{Normalization}

For an unnormalized primitive spherical harmonic GTO, the overlap is given by
\begin{equation}
\left\langle \chi^{\eqtext{GTO}}_{i lm} | \chi^{\eqtext{GTO}}_{i lm} \right\rangle = (N_i^{\eqtext{p}})^2 \left(\frac{\pi}{2a_i}\right)^{3/2}\frac{1}{(4a_i)^l}
\end{equation}
(where the last term corresponds to $1/(2p)^l$ according to for example Eq. 9.3.8 of Ref. [THE BOOK {\insertref}]). This gives 
the primitive normalization factor $N^{\eqtext{p}}$
\begin{equation}
  N^{\eqtext{p}}_i = \frac{(4a_i)^{l/2+3/4}}{\pi^{3/4}}
\end{equation}
Naturally the default in {\lsdalton} is not primitive GTOs but contracted GTOs written as linear combinations of primitive spherical-harmonic GTOs of different exponents
\begin{equation}
G_{\mu lm}(\textbf{r}; a;\textbf{A}) = \sum_{i} G_{ilm}(\textbf{r}; a_{i};\textbf{A})d^{\eqtext{norm}}_{i\mu}
\end{equation}
in {\lsdalton} we determine $d^{\eqtext{norm}}_{i\mu}$ as
\begin{equation}
d^{\eqtext{norm}}_{i\mu} = N^{\eqtext{p}}_{i}N^{\eqtext{c}}_{i\mu}d^{\eqtext{basis}}_{i\mu}
\end{equation}
As explained, $N^{\eqtext{p}}_{i}$ is the normalization of the primitive GTOs, $d^{\eqtext{basis}}_{i\mu}$ are the 
unmodified contraction coefficients read from the basis set file. $N^{\eqtext{c}}_{i}$ is the normalization of the contracted functions and is determined using the overlap between two normalized primitive GTOs of the same angular momentum, but different exponents
\begin{eqnarray}
\left\langle \chi^{\eqtext{GTO}}_{a_{i}lm} | \chi^{\eqtext{GTO}}_{a_{j}lm} \right\rangle = \biggl( \frac{\sqrt{ 4 a_{i} a_{j}}}{a_{i}+a_{j}} \biggr)^{\frac{3}{2}+l}
\end{eqnarray}
so that the contraction coefficients for the contracted function $N^{\eqtext{c}}_{i}$ are found by first finding the overlap
\begin{eqnarray}
\mathcal{S}_{\mu} = \sum_{ij} d_{i\mu}^{\eqtext{basis}} d_{j\mu}^{\eqtext{basis}} \left\langle \chi^{\eqtext{GTO}}_{ilm} | \chi^{\eqtext{GTO}}_{jlm} \right\rangle = \sum_{ij} d_{i\mu}^{\eqtext{basis}} d_{j\mu}^{\eqtext{basis}} \biggl( \frac{\sqrt{ 4 a_{i} a_{j}}}{a_{i}+a_{j}} \biggr)^{\frac{3}{2}+l}
\end{eqnarray}
where $d_{i\mu}^{\eqtext{basis}}$ are the unmodified coefficients and then construct the coefficients
\begin{eqnarray}
d_{i\mu}^{\eqtext{norm}} &=& N^{\eqtext{p}}_{i}N^{\eqtext{c}}_{i\mu}d^{\eqtext{basis}}_{i\mu} = N^{\eqtext{p}}_{i}\frac{d^{\eqtext{basis}}_{i\mu}}{\sqrt{ \mathcal{S}_{\mu} }} = \frac{d_{i\mu}^{\eqtext{basis}}}{\sqrt{\mathcal{S}_{\mu}}} \biggl( 4 a_{i}\biggr)^{\frac{l}{2} + \frac{3}{4}} \biggl( \frac{1}{\pi} \biggr)^{\frac{3}{4}}\end{eqnarray}

\section{Matrix File Format}

During an {\lsdalton} calculation a number of files may be generated. This list of files include 

\begin{verbatim}
dens.restart - The density matrix
fock.restart - The fock matrix
overlapmatrix - The overlap matrix
cmo_orbitals.u - The canonical Molecular orbitals 
lcm_orbitals.u - The local Molecular orbitals 
\end{verbatim}

the files are all written using the Fortran 90 code

\begin{lstlisting}
OPEN(UNIT=IUNIT,FILE='dens.restart',STATUS='UNKNOWN',FORM='UNFORMATTED',IOSTAT=IOS)
WRITE(iunit) A%Nrow, A%Ncol
WRITE(iunit)(A%elms(I),I=1,A%nrow*A%ncol)
CLOSE(UNIT=IUNIT,,STATUS='KEEP')
\end{lstlisting}

Where A is a derived type containing the number of rows (A$\%$nrow), the number of columns (A$\%$ncol), and the elements stored in a vector array (A$\%$elms). 

The dens.restart is different as it contains an additional logical 
\begin{lstlisting}
OPEN(UNIT=IUNIT,FILE='dens.restart',STATUS='UNKNOWN',FORM='UNFORMATTED',IOSTAT=IOS)
WRITE(iunit) A%Nrow, A%Ncol
WRITE(iunit)(A%elms(I),I=1,A%nrow*A%ncol)
WRITE(iunit) GCBASIS
CLOSE(UNIT=IUNIT,,STATUS='KEEP')
\end{lstlisting}
{\sc important:} When interfacing to other programs, the logical GCBASIS should always be false, 
i.e. by specifying the keyword
\begin{verbatim}
.NOGCBASIS
\end{verbatim}
in the LSDALTON.INP.
If the logical GCBASIS is true 
the dens.restart is given in terms of the grand canonical basis rather than the input basis.


\section{The {\lsdalton} library bundle}
\label{sec:LSlib}

\dots Simen will write this section soon!!!!! \dots




