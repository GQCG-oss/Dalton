\chapter{Recommendations }\label{recommendations}

In this chapter we give an few recommendations to keywords that is advantages to use 

\section{Small Molecular systems}

the {\lsdalton} program have been design to treat large molecular systems and as such the performance for small molecules is not expected to be competative.   

\section{Large Molecular systems}

the {\lsdalton} program have been designed to treat large molecular systems and the default 
settings, will mostly likely work well. However, large molecular systems present a number of 
challengens and we suggest to use the following Keywords.


\subsection{Starting Guess}
The Trilevel \cite{trilevel1, trilevel2} starting guess usually provide a good starting guess
for the HF/KS calculation, especially for large molecular systems. 
\begin{verbatim}
.START
TRILEVEL
\end{verbatim}

\section{Large MPI Jobs}


\section{Response Calculations}


\section{Accuracy}

In order to debug other programs against the {\lsdalton} program, it may be advantages to remove screening and other approximation that reduce the accuarcy of the calculation, at the expense of speed.

\begin{verbatim}
.THRESH
1.d-20
\end{verbatim}
sets the main integral screening threshold which sets all other thresholds in the integral code. 
\begin{verbatim}
.NO SCREEN
\end{verbatim}
removes all screening. 
\begin{verbatim}
.NO OMP 
\end{verbatim}
Deactivats the use of OpenMP which makes the code non deterministic. Alternative the 
\begin{verbatim}
export OMP_NUM_THREADS=1
\end{verbatim}
sets the number of OpenMP threads to 1.

\section{Comparison}

In order to compare the {\lsdalton} program
we suggest to deactivate the so called grand canonical basis. 
\begin{verbatim}
.NOGCBASIS
\end{verbatim}
This keyword deactivates the use of the grand canonical basis and uses the input basis instead. 
\begin{verbatim}
.NOFAMILY
\end{verbatim}
Deactivates the exploitation of family basis sets where s and p functions share the same exponents. 






