\chapter{Recommendations }\label{recommendations}

In this chapter we give an few recommendations to keywords that is advantages to use 

\section{Small Molecular systems}

the {\lsdalton} program have been design to treat large molecular systems and as such the performance for small molecules is not expected to be competative.
\begin{verbatim}
**WAVE FUNCTIONS
.HF
*DENSOPT
.RH
.DIIS
.CONVTHR
1.0D-6
*END OF INPUT
\end{verbatim}
The .RH (.i.e. Roothaan-Hall) and .DIIS keywords requests a standard diagonalization combined with the DIIS scheme for convergence acceleration, which is usually the fast option for small uncomplicated molecules. 
With the keyword .CONVTHR (i.e. convergence threshold) it is requested that iterations are terminated when the Frobenius norm of the SCF gradient is smaller than 10$^{-6}$. 

\section{Large Molecular systems}

the {\lsdalton} program have been designed to treat large molecular systems and the default 
settings, will mostly likely work well. However, large molecular systems present a number of 
challengens and we suggest to use the following Keywords.

\begin{verbatim}
**INTEGRAL
.DENSFIT
**WAVE FUNCTIONS
.DFT
B3LYP
*DENSOPT
.ARH
.START
TRILEVEL
.CONVDYN
STANDARD
*END OF INPUT
\end{verbatim}
Under **INTEGRAL, we have requested the use of density-fitting to speed up the calculation of 
the Coulomb contribution (the use of J-engine is default). The exact exchange contribution will not 
be calculated using density fitting, as no such method have been implemented in the {\lsdalton} program.

Under **WAVE FUNCTIONS, we now request a DFT
calculation using the B3LYP exchange-correlation functional.

Since this is a calculation on a large molecule, we request the trilevel starting 
guess\cite{trilevel1, trilevel2} (atomic, valence, and then full optimization) under **DENSOPT. 
The Trilevel starting guess usually provide a good starting guess 
for the HF/KS calculation, especially for large molecular systems. 
 
We also use the dynamical convergence threshold, appropiate also for small molecules. 

We employ the The Augmented Rothaan-Hall (ARH) method for density optimization. 
Note that ARH is default, so this keyword is actually unnecessary.
ARH is more robust than the standard diagonalization/DIIS scheme.

For Large molecules it may be beneficial to use the new davidson algorithm 

\begin{verbatim}
.ARH DAVID
\end{verbatim}

keyword instead of the default ARH

\begin{verbatim}
.ARH
\end{verbatim}
which uses the CROP solver or use the 
\begin{verbatim}
.ARH(LS) DAVID
\end{verbatim}
where the ARH method have been Augmented by a linesearch. 

\section{Large MPI Jobs}

For large HF/KS calculation all matrix operations are parallized through the use of the SCALAPACK library, which is activated through the .SCALAPACK keyword 

\begin{verbatim}
**GENERAL
.SCALAPACK
**WAVE FUNCTIONS
.DFT
B3LYP
*DENSOPT
.ARH
.START
TRILEVEL
.CONVDYN
STANDARD
*END OF INPUT
\end{verbatim}

\section{Response Calculations}

For response calculations we recommend to use a tight convergence threshld for the initial SCF optimization. 

\begin{verbatim}
.CONVDYN
STANDARD
\end{verbatim}

\section{Accuracy}

In order to debug other programs against the {\lsdalton} program, it may be advantages to remove screening and other approximation that reduce the accuarcy of the calculation, at the expense of speed.

\begin{verbatim}
.THRESH
1.d-20
\end{verbatim}
sets the main integral screening threshold which sets all other thresholds in the integral code. 
\begin{verbatim}
.NO SCREEN
\end{verbatim}
removes all screening. 
\begin{verbatim}
.NO OMP 
\end{verbatim}
Deactivats the use of OpenMP, which makes the code non deterministic. Alternative the 
\begin{verbatim}
export OMP_NUM_THREADS=1
\end{verbatim}
sets the number of OpenMP threads to 1.

\section{Comparison}

In order to compare the {\lsdalton} program
we suggest to deactivate the so called grand canonical basis. 
\begin{verbatim}
.NOGCBASIS
\end{verbatim}
This keyword deactivates the use of the grand canonical basis and uses the input basis instead. 
\begin{verbatim}
.NOFAMILY
\end{verbatim}
Deactivates the exploitation of family basis sets where s and p functions share the same exponents. 






