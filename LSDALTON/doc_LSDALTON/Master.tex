\documentclass[11pt]{report}  
\usepackage{a4}
\usepackage{makeidx}
\usepackage{supertabular}
\usepackage{color}
\usepackage{amsmath}

\usepackage[unicode=true,pdfusetitle,
 bookmarks=true,bookmarksnumbered=false,bookmarksopen=false,
 breaklinks=false,pdfborder={0 0 0},backref=false,colorlinks=true]
 {hyperref}
\usepackage[dvipsnames,svgnames,x11names,hyperref]{xcolor}
\hypersetup{
 urlcolor=webbrown,linkcolor=blue,citecolor=red}


% Specifies the document style.
\usepackage{bibentry}
\usepackage[numbers]{natbib}
\usepackage{listings}

\newcommand{\ignore}[1]{}
\newcommand{\nobibentry}[1]{{\let\nocite\ignore\bibentry{#1}}}
% apsrev entries in the text need definitions of these commands
\newcommand{\bibfnamefont}[1]{#1}
\newcommand{\bibnamefont}[1]{#1}


\typeout{***               }
\typeout{***   Modification history: }
\typeout{***   version 1.0 by S. H{\o}st, June 2010 }
\typeout{***               }

\setcounter{secnumdepth}{3}

\pagestyle{headings}
\oddsidemargin=0in
\evensidemargin=0in
\textwidth=6.0in
%\headheight=0pt
%\headsep=0pt
\topmargin=0in
\textheight=8.0in
\parindent=35pt
\makeindex
%
%    Options for Sirius
%
\newcommand{\beq}{\begin{equation}}
\newcommand{\eeq}{\end{equation}}
\newcommand{\beqar}{\begin{eqnarray}}
\newcommand{\eeqar}{\end{eqnarray}}
\newcommand{\bcen}{\begin{center}}
\newcommand{\ecen}{\end{center}}
\newcommand{\lrar}{\mbox{$\leftrightarrow$}}
\newcommand{\kwlabel}[1]{
{\tt \obeyspaces \frenchspacing #1 \hfill}
\index{#1@{\normalfont\ttfamily #1}}}

\lstset{
language=[90]Fortran,           % choose the language of the code
basicstyle=\footnotesize,       % the size of the fonts that are used for the code
numbers=left,                   % where to put the line-numbers
numberstyle=\footnotesize,      % the size of the fonts that are used for the line-numbers
stepnumber=1,                   % the step between two line-numbers. If it is 1 each line will be numbered
numbersep=5pt,                  % how far the line-numbers are from the code
backgroundcolor=\color{white},  % choose the background color. You must add \usepackage{color}
showspaces=false,               % show spaces adding particular underscores
showstringspaces=false,         % underline spaces within strings
showtabs=false,                 % show tabs within strings adding particular underscores
frame=single,                   % adds a frame around the code
tabsize=2,                      % sets default tabsize to 2 spaces
captionpos=b,                   % sets the caption-position to bottom
breaklines=true,                % sets automatic line breaking
breakatwhitespace=false,        % sets if automatic breaks should only happen at whitespace
escapeinside={\%*}{*}           % if you want to add a comment within your code
}

%\newenvironment{keywords}{
%  \noindent{\bf Keywords:}
%  \list{}{
%    \settowidth{\leftmargin}{\tt MM}
%    \setlength{\rightmargin}{1.0cm}
%    \setlength{\labelwidth}{\linewidth}
%      \addtolength{\labelwidth}{-\rightmargin}
%    \setlength{\labelsep}{0cm}
%    \setlength{\itemindent}{-\leftmargin}
%      \addtolength{\itemindent}{\labelwidth}
%      \addtolength{\itemindent}{\labelsep}
%    \let\makelabel\kwlabel
%  }
%}
%}{\endlist}
%\newcommand{\bkw}{\begin{keywords}}
%\newcommand{\ekw}{\end{keywords}}
%hjorig \newcommand{\kw}[1]{\verb+#1+}
%hjorig \newcommand{\quotekw}[1]{``\kw{#1}\nolinebreak''}
\newcommand{\kw}[1]{#1}
\newcommand{\quotekw}[1]{"#1"}
%agren \newcommand{\kw}[1]{\tt #1}
%agren \newcommand{\quotekw}[1]{"{\tt #1}"}
\newcommand{\starkey}{\quotekw{*}}
\newcommand{\dotkey}{\quotekw{.}}
\newcommand{\f}[1]{{"{f#1}" }}
\newenvironment{inputex}%
{ \hspace{1cm} \begin{minipage}{15cm} }%
{ \end{minipage} }

%\newif\ifmetecc
%\meteccfalse
%\newif\ifddfock
%\newif\ifsolvent
%\newif\ifabacus
%\newif\ifrespons
%\ifmetecc
%   \ddfockfalse \solventtrue \abacusfalse \responsfalse
%\else
%   \ddfocktrue \solventtrue \abacustrue \responstrue
%\fi

% Sirius limits for Sirius manual:
\def\sirver{n07 }
\def\mxfelt{20 }
% end of SIRIUS limits file
%
%           
%\renewcommand{\thetable}{\Roman{table}}
\newcommand{\insertref}{{\sc Insert reference here!!! }}
\newcommand{\lsdalton}{{\sc lsdalton}}
\newcommand{\dalton}{{\sc dalton}}
\newcommand{\resp}{{\sc response}\index{\sc response}}
\newcommand{\response}{{\sc response}\index{\sc response}}
\newcommand{\mol}{{\sc molecule}\index{\sc molecule}}
\newcommand{\molecule}{{\sc molecule}\index{\sc molecule}}
\newcommand{\cc}{{\sc cc}\index{\sc cc}}
\newcommand{\OM}{\mbox{1-matrix}}
\newcommand{\TM}{\mbox{2-matrix}}
\newcommand{\grp}[1]{\mbox{$\cal #1$}}
\newcommand{\dcr}[1]{\mbox{\Fatnormalfont #1}}
\newcommand{\sdcr}[1]{\mbox{\Fatscriptfont #1}}
\newcommand{\bigtit}[1]{\centering{\Titfont #1}}
\newcommand{\Forall}{\mbox{$\;\forall\;$}}
\newcommand{\pgrp}[1]{\mbox{${\cal{S}}(#1)$}}
\newcommand{\oper}[1]{\mbox{$\hat #1$}}
\newcommand{\fixindent}[1]{\mbox{\hspace{#1 in}}}
\newcommand{\tone}{${\cal{T}}_1$}
\newcommand{\qone}{${\cal{Q}}_1$}
\newcommand{\NP}{\mbox{$N$-particle}}
\newcommand{\NE}{\mbox{$N$-electron}}
\newcommand{\OP}{\mbox{one-particle}}
\newcommand{\COP}{\mbox{One-particle}}
\newcommand{\OEL}{\mbox{one-electron}}
\newcommand{\eqtext}[1]{\text{#1}}
\newcommand{\usp}[2]{(#1$s$~#2$p$)}
\newcommand{\uspd}[3]{(#1$s$~#2$p$~#3$d$)}
\newcommand{\uspdf}[4]{(#1$s$~#2$p$~#3$d$~#4$f$)}
\newcommand{\uspdfg}[5]{(#1$s$~#2$p$~#3$d$~#4$f$~#5$g$)}
\newcommand{\uspdfgh}[6]{(#1$s$~#2$p$~#3$d$~#4$f$~#5$g$~#6$h$)}
\newcommand{\csp}[2]{[#1$s$~#2$p$]}
\newcommand{\cspd}[3]{[#1$s$~#2$p$~#3$d$]}
\newcommand{\cspdf}[4]{[#1$s$~#2$p$~#3$d$~#4$f$]}
\newcommand{\cspdfg}[5]{[#1$s$~#2$p$~#3$d$~#4$f$~#5$g$]}
\newcommand{\cspdfgh}[6]{[#1$s$~#2$p$~#3$d$~#4$f$~#5$g$~#6$h$]}
\newcommand{\cspdsp}[5]{[#1$s$~#2$p$~#3$d$/#4$s$~#5$p$]}
\newcommand{\cspdfspd}[7]{[#1$s$~#2$p$~#3$d$~#4$f$/#5$s$~#6$p$~#7$d$]}
%
% New definitions, KR
%
\newcommand{\Key}[1]{{\normalfont
\ttfamily{.#1}}\index{#1@{\normalfont \ttfamily{.#1}}}}
\newcommand{\Keyinfo}[1]{{\normalfont
\ttfamily{#1}}\index{#1@{\normalfont \ttfamily{#1}}}}
\newcommand{\Keym}[1]{{\normalfont
\ttfamily{.#1}}\index{#1@{\normalfont \ttfamily{.#1}}|textbf}}
\newcommand{\Sec}[1]{{\normalfont
\ttfamily{*#1}}\index{#1@{\normalfont \ttfamily{*#1}}}}
\newcommand{\Secm}[1]{{\section{*#1}}\index{#1@{\normalfont
\ttfamily{*#1}}|textbf}}
\newcommand{\Subsecm}[1]{{\subsection{*#1}}\index{#1@{\normalfont
\ttfamily{*#1}}|textbf}}
\renewcommand{\descriptionlabel}[1]{\hspace{\labelsep}\ttfamily{#1}}
\begin{document}
%
\bibliographystyle{unsrt}
%
\baselineskip=15pt
%
\title{DALTON2013 -- LSDALTON Program Manual}

%
\author{S.~Coriani,
P.~Ettenhuber, 
T.~Helgaker,\\ 
S.~H{\o}st,  
I.-M.~H{\o}yvik,
B.~Jans{\'i}k,\\
P.~J{\o}rgensen, 
J. Kauczor,
T.~Kj{\ae}rgaard,\\
K. Kristensen,
P.~Merlot,
J.~Olsen, \\ 
S.~Reine,
P.~Sa{\l}ek,
A. Thorvaldsen, \\
L. Th{\o}gersen, 
V. Rybkin,
V. Bakken,\\
M. Watson,
 A. Krapp,
and M.~Ziolkowski}
%
\date{\mbox{\ }}
%
\maketitle
%\begin{titlepage}
%\mbox{ }\newline
%\end{titlepage}

\pagenumbering{roman}
\tableofcontents

\pagenumbering{roman}
\tableofcontents

\chapter*{Preface}
%\pagestyle{myheadings}
%\markboth{ }
%{ }
%\markright{ }
\addcontentsline{toc}{chapter}{Preface}

This is the documentation for the \dalton\ quantum chemistry program
--- Release 1.2.1, November 2001 --- for computing SCF,
MCSCF, MP2, and Coupled Cluster wave functions and for calculating 
molecular properties and potential energy surfaces.

          We emphasize again here the conditions under which the
program is distributed.  It is furnished for your own use,
and you may not redistribute it further, either in whole or in
part.  Any one interested in obtaining \dalton\ should check out the
\dalton\ homepage at
\verb|http://www.kjemi.uio.no/software/dalton/dalton.html| or the
American mirror site at
\verb|http://www.sdsc.edu/dalton/dalton.html|.

          Any use of the program that results in published
material should cite the following:
\begin{quote}
``\dalton , a molecular electronic  structure program, Release
1.2 (2001)'', written by 
T.~Helgaker, H.~J.~Aa.~Jensen, P.~J{\o}rgensen, J.~Olsen,
K.~Ruud, H.~{\AA}gren,
A.~A.~Auer,
K.~L.~Bak,
V.~Bakken,
O.~Christiansen,
S.~Coriani,
P.~Dahle,
E.~K.~Dalskov,
T.~Enevoldsen,
B.~Fernandez,
C.~H{\"a}ttig,
K.~Hald,
A.~Halkier,
H.~Heiberg,
H.~Hettema,
D.~Jonsson,
S.~Kirpekar,
R.~Kobayashi,
H.~Koch,
K.~V.~Mikkelsen,
P.~Norman,
M.~J.~Packer,
T.~B.~Pedersen,
T.~A.~Ruden,
A.~Sanchez,
T.~Saue,
S.~P.~A.~Sauer,
B.~Schimmelpfenning,
K.~O.~Sylvester-Hvid,
P.~R.~Taylor,
and O.~Vahtras
\end{quote}

          The program represents experimental code that is
under constant development.  No guarantees of any kind are
provided, and the authors accept no responsibility for the
performance of the code or for the correctness of the results.

\pagenumbering{arabic}
\part{\lsdalton\ Installation Guide}
\chapter{Installation}\label{ch:install}

\section{Hardware/software
supported}\label{sec:hardsoft}\index{hardware/software support}

{\dalton} can be run on a variety of systems running the UNIX
operating system. The current release of the program supports
%Cray-UNICOS\index{Cray},
%Cray-T3D/E\index{Cray!T3D}\index{Cray!T3E}\index{Cray},
%HP-UX,
IBM-AIX\index{IBM-AIX},
Linux\index{Linux} using gfortran, pgf90/95 or ifort, and 
MacOS X (darwin)\index{MacOSX} using xlf/xlc or gfortran/gcc.
%We furthermore note that response calculations involving the
%spin--orbit\index{spin-orbit} operator only will work on some of these
%computers (more details are given in the file
%\verb|dalton/test/KNOWN_PROBLEMS|).

The program is written in FORTRAN~77\index{FORTRAN~77}, 
FORTRAN~90\index{FORTRAN~90} and C~\index{C}, with
machine dependencies isolated using C preprocessor directives\index{C preprocessor}.  
All floating-point computations are performed in 64-bit precision, 
but if 32-bit integers are available the code will take advantage of this to reduce storage
requirements in some sections.

The program should be easily portable to other UNIX
platforms\index{porting}.
Users who port the codes to other platforms are encouraged to
communicate any required changes in the original source with the
appropriate C preprocessor directives to the authors.

\section{Source files}\label{sec:source}

{\dalton} is distributed as a \verb|tar| file obtainable from
the {\dalton} homepage\\
(\verb|http://www.daltonprogram.org|) if the
license agreement for the program has been completed and returned to
the authors.  If you have accessed this documentation off the
\verb|tar|\index{tar-file} file you will
already know how to extract the required directory structure, but
for completeness, assuming the \verb|tar| file is called
\verb|dalton.tar.gz|\index{tar-file}, the commands
\begin{verbatim}
gunzip dalton.tar.gz
tar xf dalton.tar
\end{verbatim}
will produce the following subdirectory structure in the current
directory:
\begin{verbatim}
dalton/abacus    dalton/densfit    dalton/eri       dalton/pdpack     dalton/test_cc
dalton/amfi      dalton/dft        dalton/gp        dalton/rsp        dalton/tools
dalton/basis     dalton/dftxcfun   dalton/include   dalton/sirius
dalton/cc        dalton/Doc        dalton/lucita    dalton/soppa
dalton/choles    dalton/Doc_cc     dalton/modules   dalton/test
\end{verbatim}
Most of the subdirectories contain source code for the different sections
constituting the program (\verb|abacus|, \verb|amfi|, \verb|cc|, \verb|choles|, \verb|densfit|,
\verb|dft|, \verb|eri|, \verb|gp|, \verb|lucita|,\verb|rsp| and \verb|sirius|, and \verb|soppa|). Furthermore,
there's a directory containing
various public domain routines (\verb|pdpack|), a directory with the
necessary include files containing common blocks and
machine dependent routines (\verb|include|), a directory 
which will (after compilation) contain necessary module files (\verb|modules|), 
a directory containing all the basis sets supplied with this distribution (\verb|basis|), a
fairly large set of test jobs including reference output files
(\verb|test|, \verb|test_cc|), a directory containing some useful pre- and
post-processing programs supplied to us from various users
(\verb|tools|), and finally this documentation (\verb|Doc|). 

In addition to the directories, the main dalton directory will
contain several files including a shell script (\verb|configure|)
which will build a suitable \verb|Makefile.config| for use when
installing the program. The \verb|configure| script will also create a
\verb|Makefile| and a run script \verb|bin/dalton| from the skeletal files
(\verb|Makefile.in| and \verb|dalton.gnr|) that are present in the
directory.


\section{Installing the program using the
Makefile}\label{sec:Makefile}
\index{Makefile}
\index{installation!Makefile}
\index{installation!Makefile.config}

The program is easily installed through the use of the supplied
\verb|configure| script\index{configure script}. Based on the
automatically guessed architecture, the
script will try to build a suitable
\verb|Makefile.config|\index{Makefile.config} on the
basis of what kind of mathematical libraries are found, and user
input. Thus, to execute the script, type
\begin{verbatim}
> ./configure
\end{verbatim}

\bigskip

If the automatic hosttype detection (which usually works fine for most common
platforms) is not correct, the user may provide the correct \verb|<architecture>|\ type 
to the script by typing
\begin{verbatim}
> ./configure -<architecture>
\end{verbatim}

\bigskip

Although this script in most cases is capable of making a correct
\verb|Makefile.config|, we always recommend users to check the created
\verb|Makefile.config| against local system set-up. 

During the execution of the \verb|configure| script, you will be
asked a few questions, most of which require a quite obvious
answer. Let us only comment upon five of the questions asked:

\begin{enumerate}
\item 64-bit integers\index{64-bit integers}\index{installation!64-bit}:
When installing \dalton\ on a 64-bit architecture (for example Linux x86\_64) 
the user will be asked if 64-bit integers should be used throughout the program. 
\dalton\ is designed such that it can be excuted with either 32-bit or 64-bit 
integers but in some cases it may be advantageous and/or compulsory 
to use 64-bit integers, e.g., memory allocation for more than 16Gbytes  
of scratch-memory (see also next paragraph) and execution of some modules 
with more than 255 basis functions.

\item Scratch memory\index{scratch memory}\index{installation!memory}
size: Dalton uses approximately 6.5 Mwords (54 Mbytes)
in static memory allocations. The program defines a large scratch-memory
array, from which it allocates space for temporary arrays
during the execution of the program. This value is given in Words, and
should be chosen according to available memory on your computer.
\verb|WRKMEM| may be changed at execution time by supplying a
different value for \verb|WRKMEM| through the shell script running
{\dalton}.

\item Default basis set library
\index{basis set!library}
\index{installation!basis set library}
location: This defines the directory where the program will look for the
basis sets supplied with the distribution, and this need to be
changed according to the local directory structure. We recommend
that the basis sets in this directory are {\em not} changed, but
that changes to the basis set rather is done in a separate
directory, and then supply this basis set directory to the program
at execution time using the \verb|dalton -b basdir| option.

{\sc NOTE TO SYSTEM ADMINISTRATORS:} Supplied with {\dalton} is an
extensive basis set library. This basis set directory must be made
readable to all users. {\sc END NOTE}

\item Default scratch
space\index{scratch space}\index{installation!scratch space}:
Determines the default head scratch
directory where temporary files will be placed. This value will be put
in the \verb|dalton| run script. However, note that jobs will be run in
a subdirectory of this head scratch-directory, according to the name
of the job files. If \verb|/work| or \verb|/scratch| is defined in the
local directory structure, the script will normally suggest
\verb|/work/$USER| or  \verb|/scratch/$USER| as default head scratch space.

\item Default install 
directory\index{install directory}\index{installation!install directory}:
Denotes the directory where the {\dalton} executable and the
\verb|dalton| run script will be moved to.
Default: a subdirectory {\tt ./bin} to the main \dalton\ directory.
\end{enumerate}

Compiler options will be supplied in \verb|Makefile.config|, in the manner
we use ourselves. {\em These options often do not include aggressive
optimization\index{optimization (f90)}, as
it is our experience that
the code is optimized incorrectly if the optimization is too
aggressive.} The proper options to the C preprocessor for an ordinary
installation of the program are also supplied with
\verb|Makefile.config|\index{Makefile.config}.

The \verb|configure| script attempts to detect mathematical libraries
available on the system and use them whenever possible. {\dalton} can
use third-party BLAS and LAPACK libraries or libraries providing
equivalent functionality like ESSL on AIX, ATLAS, ACML, MKL on Intel
architectures.
You will be asked which one you want of the found libraries.
You can also point the \verb|configure|
script to the location of the preferred libraries by setting
\verb|LIBDIR| environment variable before running the script. For
example, if one has high-performance BLAS routines in \verb|mylibs|
subdirectory of the user's home directory, \verb|configure| script can
be instructed to use them as follows:
\begin{verbatim}
env LIBDIR=\$HOME/mylibs ./configure
\end{verbatim}


NOTE: If problems with I/O\index{I/O-problems} are experienced on
any computer,
manifesting themselves as error messages saying that a read statement
has passed the EOF mark, the *.F files in the gp directories should be
touched, and the code rebuilt using the additional C preprocessor
directive \verb|VAR_MFDS|.

When  \verb|Makefile.config|\index{Makefile.config} has been properly
created and checked
to agree with local system set-up, all that is needed
to build an executable\index{building an executable} version of the
code is to type
(in the same directory as the \verb|Makefile.config| file):
\begin{verbatim}
> make
\end{verbatim}

NOTE: If you encounter a (similar) error message as described in the following during the 
compilation of a parallel \dalton\ version, please follow one of the advices given below. 
\begin{verbatim}
> make
> lot's of output ...
> Error:
> ... This module file was not generated by any release of this compiler.   [MPI]
>   use mpi
> ------^
\end{verbatim}
The error message arises due to a mismatch between the F90 compiler which 
has been used to compile the \verb|mpi.mod|\index{mpi.mod} module of your MPI-library 
and the F90 compiler which has been chosen to compile \dalton. 
Two possible solutions are:
\begin{enumerate}
\item add the C preprocessor directive \verb|VAR_USE_MPIF|\ to \verb|CPPFLAGS|\ in \verb|Makefile.config|\index{Makefile.config} 
and recompile \dalton.
\item ask your local system administrator to provide you with a MPI-library which has been 
set up and compiled with the identical F90 compiler version which you have chosen for \dalton.
\end{enumerate}
Although most likely being more cumbersome, we recommend solution 2. The simple solution 1 bypasses the 
F90 type checking and argument list checking functionalities of F90 which is preferable in order to 
easily identify programming bugs.

For MacOS X\index{MacOSX}, the program may be installed either with the IBM
xlf compiler for best performance or with the gfortran/ifort compiler. Be sure
to use the latest Xcode from \verb|http://developer.apple.com/|.

\section{Running the {\dalton} test suite}\label{sec:testsuite}
\index{test suite}\index{installation!test suite}

To check that {\dalton} has been successfully installed, a fairly
elaborate automatic test suite is provided in the distribution. A test
script, all the test jobs and reference output files can be found in
the \verb|dalton/test| directory. It is highly recommended that all
these tests be run once the program has been compiled. Depending on
your hardware, this usually takes 2---12 hours.

The tests can be run one by one or in groups, by using the test script
\verb|TEST|. Try \verb|TEST -h| to see the different options this
script takes. Also, have a look at \verb|CONTENTS|
for short descriptions of the various tests. To run the
complete test suite, simply go to the \verb|dalton/test| directory and
type:
\begin{verbatim}
> ./TEST all
\end{verbatim}
You can follow the progress of the tests directly, but all messages
are also printed to a log (\verb|TESTLOG| by default). After all the
tests have completed you should hopefully be presented with the
message ``ALL TESTS ENDED PROPERLY!''. If not, you will be given a
list of the tests that failed to run correctly. Please consult the file
\verb|KNOWN_PROBLEMS| too see if these tests have documented problems
on your particular platform.

Any tests that fail will leave behind the \verb|.mol| and \verb|.dal|
input-files (these are described in more detail in
Chapter~\ref{ch:starting}), and the output-file from the test
calculation which will have the extension \verb|.log| ({\dalton}
output-files usually have the extension \verb|.out|). For all
successful tests all files, except the output-file (\verb|.log|),
will be deleted as soon as the output has been checked, unless
\verb|TEST| is being run with the option \verb|-keep|.

If most of the tests fail, it is quite likely that there's something
\index{troubleshooting}\index{installation!troubleshooting}
wrong with the installation. Look carefully through
\verb|Makefile.config|, and consider turning down or even off
optimization.

If there's only a few tests that fail, and {\dalton} seems to exit
normally in each case, there may just be some issues with numerical
accuracy. Different machines give slightly different results, and
while we've tried to allow for some slack in the tests, it may be
that your machine yields numbers just outside the intervals we've
specified as acceptable. A closer comparison of the results with
numbers in the test script and/or the reference output files should
reveal whether this is actually the case. If numerical (in)accuracy is
the culprit, feel free to send your output-file(s) to
\verb|dalton-admin@kjemi.uio.no| so that we can adjust the numerical
intervals accordingly.

\chapter{Maintenance}\label{ch:maintain}

\section{New versions, patches}

New versions will be announced on the Dalton homepage
(\verb|http://daltonprogram.org|)
and the \dalton\ mailing-list
(\verb|dalton-users@kjemi.uio.no|)\index{mailing list}.

In between releases, bug fixes will be distributed as
patches\index{patches} to an
existing version. New patches will be
freely available from the 
Dalton homepage, and will be announced on the dalton-users mailing
list. Patches will normally be distributed in order to correct
significant errors. In case of minor changes, explicit
instructions on modifying the source file(s) may be given.

In general, users should not make local
modifications in the source code, as this usually makes it
much harder to incorporate the changes into subsequent versions.
It is more convenient to use the C preprocessor code.  Indeed, by
judicious use of local \verb|#define|\index{define} variables (this
is described in any book on C that also describes the C
preprocessor) it is often possible to isolate local
modifications\index{modifications} completely, making it much easier
to apply them to later
releases.

\section{Reporting bugs and user support}

The \lsdalton\ program is distributed to the computational chemistry
community with no obligations on the side of the
authors. The authors thus take no responsibility\index{responsibility}
for the performance
of the code or for the correctness\index{correctness} of the
results. This distribution
policy gives the authors no responsibility for problems experienced by
the users when using the \lsdalton\ program.

A mailing-list\index{mailing list} ---
\verb|dalton-users@kjemi.uio.no| --- has been created for
the purpose of communication among the authors and the users about
new patches and releases, and communication between users of the
program. The authors hope that the
users will be able to help each others out when problems
arise. All authors of the \lsdalton\ program also receive mail from the
mailing-list, but we do not guarantee that any author will reply to
requests posted to the mailing-list. By default, all users signing a
license agreement form will be added to this mailing list. If more
people use the same licensed version, these users may send mail to
\verb|dalton-admin@kjemi.uio.no| asking to be added to this list as
well as indicating whose licensed version of the program they are using.

Bug reports\index{bugs} are to be reported to \verb|dalton-admin@kjemi.uio.no| and
will be dealt with by one of the authors, although no responsibility
on the promptness of this response is given. In general, serious bugs
that have been
reported and fixed will lead to a new patch\index{patches} of the program, distributed from the
Dalton homepage (\verb|http://daltonprogram.org|).

\part{\lsdalton\ User's Guide}
\chapter{Getting started with {\lsdalton}}\label{ch:starting}

In this chapter we give an introduction to the two input files needed
for doing a calculation with the {\lsdalton} program
(MOLECULE.INP and LSDALTON.INP), as well as the
shell script file that is supplied with the program for moving
relevant files to the scratch-directory and back to the home directory
after a calculation has completed. A couple of examples of
input files  are also provided. Finally, the different output files
generated by the program are discussed.

\section{The \mol\ input file}

Basically, the \mol\ file contains the following information:
\begin{itemize}
\item The Cartesian coodinates for all atoms
\item The basis set(s) to be used
\end{itemize}

Please note the following:
\begin{itemize}
\item Supplying the molecular geometry in the form of a Z-matrix 
is not supported
by \lsdalton\, since this format is ill suited for large scale 
calculations.
\item Molecular symmetry is not exploited by \lsdalton\, since
the target systems of the program are large biomolecules, which
hardly ever contains any symmetry. 
\end{itemize}

There are two different types of molecule file, using either BASIS or ATOMBASIS.
ATOMBASIS is relevant only if different basis sets on individual atoms are required.
In the following, we describe these two types of \mol\ files using simple examples.

\subsection{BASIS}
A BASIS \mol\ file may look like this:
\begin{verbatim}
1     BASIS
2     cc-pVDZ Aux=Ahlrichs-Coulomb-Fit
3     water R(OH) = 0.95Aa , <HOH = 109 deg.
4     Distance in Aangstroms
5     Atomtypes=2 Angstrom
6     Charge=8.0 Atoms=1
7     O      0.00000   0.00000   0.00000
8     Charge=1.0 Atoms=2
9     H      0.55168   0.77340   0.00000
10    H      0.55168  -0.77340   0.00000
\end{verbatim} 

We now describe the \mol\ file line by line:
\begin{itemize}
\item Line 1: The word "BASIS", indicating that the same basis set should be used for all 
atoms.
\item Line 2: The name of the basis set, in this case cc-pVDZ. Optional: The name
of the auxiliary basis set, in this case Ahlrichs-Coulomb-Fit
(only referenced if density-fitting is requested in LSDALTON.INP)
\item Lines 3-4: Comments (may be left blank)
\item Line 5: {\bf Atomtypes=} Number of different atoms - in the case of H$_2$O, there are two different
atoms type, H and O. {\bf Angstrom} If this keyword is omitted, the program will assume that coordinates
are given in atomic units (Bohr). Other options in this line are:
{\bf Charge=} Molecular charge - assumed to be zero if omitted. 
{\bf Nosymmetry} This have no effect as no pointgroup symmetry have been implemented. 
\item Lines 6 and 8: A line of this type has to come before the Cartesian coordinates of each atom
type, indicating a) The nuclear charge of the atom type, and b) The number of atoms of this type.
\item Lines 7, 9-10: Cartesian coordinates of the atoms (in Angstrom, since this was specified 
in line 5).
\end{itemize}

\subsection{ATOMBASIS}

An ATOMBASIS \mol\ file may look like this:
\begin{verbatim}
ATOMBASIS
water R(OH) = 0.95Aa , <HOH = 109 deg.
Distance in Aangstroms
Atomtypes=2 Angstrom
Charge=8.0 Atoms=1 Basis=cc-pVDZ Aux=Ahlrichs-Coulomb-Fit
O      0.00000   0.00000   0.00000
Charge=1.0 Atoms=2 Basis=cc-pVDZ Aux=Ahlrichs-Coulomb-Fit
H      0.55168   0.77340   0.00000
H      0.55168  -0.77340   0.00000
\end{verbatim}
As seen from the example, there is little difference between the two
formats. Line 2 in the BASIS file has been removed, and instead the basis
sets (and optionally, the auxiliary basis sets) are now given for each
atom. In this example they are the same, as it would probably not make
much sense to use different basis sets for the atoms in water. This can
make sense, though, for larger molecules. An example could be a biomocule containing
an active site with one or more transition metal atoms. In may be useful to
describe such an active site at a higher level of theory than the surroundings. 

\subsection{User defined basis sets}

The user may supply his or her own basis set in two ways. Both ways require that the MOLECULE.INP file is argumented with a basis as in the following two examples

\begin{verbatim}
1     BASIS
2     USERDEFINED
3     water R(OH) = 0.95Aa , <HOH = 109 deg.
4     Distance in Aangstroms
5     Atomtypes=2 Angstrom
6     Charge=8.0 Atoms=1
7     O      0.00000   0.00000   0.00000
8     Charge=1.0 Atoms=2
9     H      0.55168   0.77340   0.00000
10    H      0.55168  -0.77340   0.00000
11 
12    USERDEFINED BASIS
13    a 1
14    $ HYDROGEN     (4s,1p) -> [2s,1p]
15    $ HYDROGEN     segmented contracted
16    $ S-TYPE FUNCTIONS
17        4    2    0
18         18.7311370  0.03349460  0.00000000
19          2.8253937  0.23472695  0.00000000
20          0.6401217  0.81375733  0.00000000
21          0.1612778  0.00000000  1.00000000
22    $ P-TYPE FUNCTIONS
23        1    1    0
24          1.1000000  1.00000000
25    a 8
26    $ OXYGEN       (10s,4p) -> [3s,2p]                               
27    $ S-TYPE FUNCTIONS
28       10    3    0
29       5484.6717000  0.00183110  0.00000000  0.00000000
30        825.2349500  0.01395010  0.00000000  0.00000000
31        188.0469600  0.06844510  0.00000000  0.00000000
32         52.9645000  0.23271430  0.00000000  0.00000000
33         16.8975700  0.47019300  0.00000000  0.00000000
34          5.7996353  0.35852090  0.00000000  0.00000000
35         15.5396160  0.00000000 -0.11077750  0.00000000
36          3.5999336  0.00000000 -0.14802630  0.00000000
37          1.0137618  0.00000000  1.13076700  0.00000000
38          0.2700058  0.00000000  0.00000000  1.00000000
39    $ P-TYPE FUNCTIONS
40        4    2    0
41         15.5396160  0.07087430  0.00000000
42          3.5999336  0.33975280  0.00000000
43          1.0137618  0.72715860  0.00000000
44          0.2700058  0.00000000  1.00000000
45    $ D-TYPE FUNCTIONS
46        1    1    0
47          0.8000000  1.00000000
\end{verbatim} 
Line 13 to 24 contain the Basis set information for the Hydrogen atom. 
Line 13 denote the nuclear charge. Line 14 through 16 are comments. Line 17 states 
that there are 4 primitive functions and 2 contracted functions.
Line 18 to 21 contain information about the primitive functions. The first number is the exponent 
and the next 2 numbers are the coefficients for the 2 contracted functions.
Line 22-24 contain info about p orbitals on hydrogen, while Line 25-47 contain similar information 
about oxygen.

Alternative the userdefined basis may be used in connection with the ATOMBASIS
\begin{verbatim}
ATOMBASIS
water R(OH) = 0.95Aa , <HOH = 109 deg.
Distance in Aangstroms
Atomtypes=2 Angstrom
Charge=8.0 Atoms=1 Basis=cc-pVDZ
O      0.00000   0.00000   0.00000
Charge=1.0 Atoms=2 Basis=USERDEFINED
H      0.55168   0.77340   0.00000
H      0.55168  -0.77340   0.00000

USERDEFINED BASIS
a 1
$ HYDROGEN     (4s) -> [2s]                                      
$ HYDROGEN     (1p)                                              
$ S-TYPE FUNCTIONS
    4    2    0
     18.7311370  0.03349460  0.00000000
      2.8253937  0.23472695  0.00000000
      0.6401217  0.81375733  0.00000000
      0.1612778  0.00000000  1.00000000
$ P-TYPE FUNCTIONS
    1    1    0
      1.1000000  1.00000000
\end{verbatim}

Finally the basis set files from the Basis Set exchange (EMSL) 
https://bse.pnl.gov/bse/portal may be downloaded and put into 
the basis set directory and used directly. The name specified by 
the input should match the name of the file in the basis set directory
(see testcase {linsca/linsca\_emsl} for an example).

\section{The LSDALTON.INP file}\label{sec:daltoninp}
In this section, we show two typical examples of a LSDALTON.INP file, one
for small molecules and one for large molecules. Many other examples
of input files for requesting e.g. specific molecular properties may be found
at \verb|http://daltonprogram.org|, and
a complete reference manual of all keywords is found in Chapter \ref{ch:keywords}.

In general, the LSDALTON.INP is divided in four different sections under
the headlines:
\begin{itemize}
\item **INTEGRAL contain settings for the calculation of integrals (optional)
\item **WAVE FUNCTIONS contains info about the wave function (e.g. HF/DFT) and settings
for optimization of the wave function (mandatory)
\item **OPTIMIZE contains settings for geometry optimization (optional). If this keyword
      is present in the input a full geometry optimization will be performed, otherwise a single-point 
      calculation is carried out.
\item **RESPONS contains info about requested properties (optional)
\end{itemize}
Each of these sections may contain subsections, indicated by a single asterisk.
The LSDALTON.INP file should always end with \verb|*END OF INPUT|. 

A typical example of LSDALTON.INP for a {\bf small molecule} (say, less than 100 atoms) could look like:
\begin{verbatim}
**INTEGRAL
.NOJENGINE
.NOLINK
**WAVE FUNCTIONS
.HF
*DENSOPT
.RH
.DIIS
.CONVTHR
1.0D-6
**RESPONS
.NEXCIT
5
*END OF INPUT
\end{verbatim}
Here, we have requested a Hartree-Fock optimization under **WAVE FUNCTION. The subsection
*DENSOPT contains setting for how the density should be optimized. The .RH (.i.e. Roothaan-Hall)
and .DIIS keywords requests a standard diagonalization combined with the DIIS scheme for convergence
acceleration. With the keyword .CONVTHR (i.e. convergence threshold) 
it is requested that iterations are terminated when the 
Frobenius norm of the SCF gradient is smaller than 10$^{-6}$. Finally, we have under **RESPONS requested
the calculation of the five lowest excitation energies with .NEXCIT (i.e. number of excitation energies).
Note that under **INTEGRAL we have turned off the LinK and J-engine schemes for speeding up the calculation
of the exact exchange and Coulomb contributions, respectively, since these are probably not needed for a small
molecule.

A typical example of LSDALTON.INP for a {\bf large molecule} (say, more than 100 atoms) could look like: 
\begin{verbatim}
**INTEGRAL
.DENSFIT
**WAVE FUNCTIONS
.DFT
 BP86
*DENSOPT
.START
 TRILEVEL
.ARH
.CONVDYN
 STANDARD
*END OF INPUT
\end{verbatim}
Under **INTEGRAL, we have requested the use of density-fitting to speed up the calculation (the use of J-engine is default). 
Under **WAVE FUNCTIONS, we now request a DFT
calculation using the BP86 exchange-correlation functional.

Since this is a calculation on a large molecule, 
we request the trilevel starting guess (atomic, valence, and then full
optimization) and the Augmented Rothaan-Hall (ARH) method for density optimization. 
Note that ARH is default, so this keyword is actually unnecessary.
ARH is more robust than the standard diagonalization/DIIS scheme, and linear-scaling if sparse-matrix algebra is employed. You may use
sparse-matrix algebra by putting .CSR (i.e. Compressed-Sparse Row) under *DENSOPT (NB: requires linking to the MKL library). 


\chapter{Interfacing to {\lsdalton}}\label{interfacing}

In this chapter we give an introduction to the normalization, basis set and ordering of atomic orbitals used in the {\lsdalton} program. 

This is provided in case you want to interface to the {\lsdalton} program or want to read one of the files written by the {\lsdalton} program, for instance 
the local orbitals.
Certain functionality can also be incorporated into other programs by linking to the
{\lsdalton}-library bundle, as outlined in section~\ref{sec:LSlib}.

\section{The Grand Canonical Basis}

Unlike all other Quantum Chemistry Programs the {\lsdalton} program, uses the so called grand canonical basis {\insertref} as the internal default basis. The results therefore cannot directly be compared to other programs unless the 
\begin{verbatim}
.NOGCBASIS
\end{verbatim}
keyword is specified. This keyword deactivates the use of the grand canonical basis and uses the input basis instead. 

\section{Basisset and Ordering}

In {\lsdalton} the default basis set are real-valued spherical harmonic Gaussian type
orbitals (GTOs)
\begin{equation}
G_{ilm}(\textbf{r}; a_{i};\textbf{A}) = S_{lm}(x_{A}; y_{A}; z_{A}) \exp(-a_{i}r^{2}_{A})
\end{equation}
where $S_{lm}(x_{A}; y_{A}; z_{A})$ is a real solid harmonic introduced in section 6.4.2 of Ref. [THE BOOK {\insertref}], (see Table 6.3).
Concerning the ordering of basis functions, it is one atomtype after the other as given
in the input, and for each atomtype, it is one atom after another as given in input. For
each atom increasing with angular momentum $l$, within one angular momentum $l$, the ordering 
is first the components (e.g. $p_{x},p_{y},p_{z}$), and then the contracted functions.



So for an atom consiting of 3 s functions 2 p functions and 1 d function, the ordering is
$\{1s; 2s; 3s; 1p_{x};1p_{y};1p_{z}; 2p_{x};2p_{y};2p_{z}; 1d_{xy}; 1d_{yz}; 1d_{z2-r2} ; 1d_{xz}; 1d_{x2-y2}\}$
In {\lsdalton} the ordering of the basis functions are $m = \{-l,\cdots,0,\cdots,l\}$ for a given
angular momentum $l$. Although for computational efficiency the p orbitals are treated as a
special case with $p = \{ p_{x},p_{y},p_{z} \}$ which corresponds to $m = \{1,-1,0 \}$.

\section{Normalization}

For an unnormalized primitive spherical harmonic GTO, the overlap is given by
\begin{equation}
\left\langle \chi^{\eqtext{GTO}}_{i lm} | \chi^{\eqtext{GTO}}_{i lm} \right\rangle = (N_i^{\eqtext{p}})^2 \left(\frac{\pi}{2a_i}\right)^{3/2}\frac{1}{(4a_i)^l}
\end{equation}
(where the last term corresponds to $1/(2p)^l$ according to for example Eq. 9.3.8 of Ref. [THE BOOK {\insertref}]). This gives 
the primitive normalization factor $N^{\eqtext{p}}$
\begin{equation}
  N^{\eqtext{p}}_i = \frac{(4a_i)^{l/2+3/4}}{\pi^{3/4}}
\end{equation}
Naturally the default in {\lsdalton} is not primitive GTOs but contracted GTOs written as linear combinations of primitive spherical-harmonic GTOs of different exponents
\begin{equation}
G_{\mu lm}(\textbf{r}; a;\textbf{A}) = \sum_{i} G_{ilm}(\textbf{r}; a_{i};\textbf{A})d^{\eqtext{norm}}_{i\mu}
\end{equation}
in {\lsdalton} we determine $d^{\eqtext{norm}}_{i\mu}$ as
\begin{equation}
d^{\eqtext{norm}}_{i\mu} = N^{\eqtext{p}}_{i}N^{\eqtext{c}}_{i\mu}d^{\eqtext{basis}}_{i\mu}
\end{equation}
As explained, $N^{\eqtext{p}}_{i}$ is the normalization of the primitive GTOs, $d^{\eqtext{basis}}_{i\mu}$ are the 
unmodified contraction coefficients read from the basis set file. $N^{\eqtext{c}}_{i}$ is the normalization of the contracted functions and is determined using the overlap between two normalized primitive GTOs of the same angular momentum, but different exponents
\begin{eqnarray}
\left\langle \chi^{\eqtext{GTO}}_{a_{i}lm} | \chi^{\eqtext{GTO}}_{a_{j}lm} \right\rangle = \biggl( \frac{\sqrt{ 4 a_{i} a_{j}}}{a_{i}+a_{j}} \biggr)^{\frac{3}{2}+l}
\end{eqnarray}
so that the contraction coefficients for the contracted function $N^{\eqtext{c}}_{i}$ are found by first finding the overlap
\begin{eqnarray}
\mathcal{S}_{\mu} = \sum_{ij} d_{i\mu}^{\eqtext{basis}} d_{j\mu}^{\eqtext{basis}} \left\langle \chi^{\eqtext{GTO}}_{ilm} | \chi^{\eqtext{GTO}}_{jlm} \right\rangle = \sum_{ij} d_{i\mu}^{\eqtext{basis}} d_{j\mu}^{\eqtext{basis}} \biggl( \frac{\sqrt{ 4 a_{i} a_{j}}}{a_{i}+a_{j}} \biggr)^{\frac{3}{2}+l}
\end{eqnarray}
where $d_{i\mu}^{\eqtext{basis}}$ are the unmodified coefficients and then construct the coefficients
\begin{eqnarray}
d_{i\mu}^{\eqtext{norm}} &=& N^{\eqtext{p}}_{i}N^{\eqtext{c}}_{i\mu}d^{\eqtext{basis}}_{i\mu} = N^{\eqtext{p}}_{i}\frac{d^{\eqtext{basis}}_{i\mu}}{\sqrt{ \mathcal{S}_{\mu} }} = \frac{d_{i\mu}^{\eqtext{basis}}}{\sqrt{\mathcal{S}_{\mu}}} \biggl( 4 a_{i}\biggr)^{\frac{l}{2} + \frac{3}{4}} \biggl( \frac{1}{\pi} \biggr)^{\frac{3}{4}}\end{eqnarray}

\section{Matrix File Format}

During an {\lsdalton} calculation a number of files may be generated. This list of files include 

\begin{verbatim}
dens.restart - The density matrix
fock.restart - The fock matrix
overlapmatrix - The overlap matrix
cmo_orbitals.u - The canonical Molecular orbitals 
lcm_orbitals.u - The local Molecular orbitals 
\end{verbatim}

the files are all written using the Fortran 90 code

\begin{lstlisting}
OPEN(UNIT=IUNIT,FILE='dens.restart',STATUS='UNKNOWN',FORM='UNFORMATTED',IOSTAT=IOS)
WRITE(iunit) A%Nrow, A%Ncol
WRITE(iunit)(A%elms(I),I=1,A%nrow*A%ncol)
CLOSE(UNIT=IUNIT,,STATUS='KEEP')
\end{lstlisting}

Where A is a derived type containing the number of rows (A$\%$nrow), the number of columns (A$\%$ncol), and the elements stored in a vector array (A$\%$elms). 

The dens.restart is different as it contains an additional logical 
\begin{lstlisting}
OPEN(UNIT=IUNIT,FILE='dens.restart',STATUS='UNKNOWN',FORM='UNFORMATTED',IOSTAT=IOS)
WRITE(iunit) A%Nrow, A%Ncol
WRITE(iunit)(A%elms(I),I=1,A%nrow*A%ncol)
WRITE(iunit) GCBASIS
CLOSE(UNIT=IUNIT,,STATUS='KEEP')
\end{lstlisting}
{\sc important:} When interfacing to other programs, the logical GCBASIS should always be false, 
i.e. by specifying the keyword
\begin{verbatim}
.NOGCBASIS
\end{verbatim}
in the LSDALTON.INP.
If the logical GCBASIS is true 
the dens.restart is given in terms of the grand canonical basis rather than the input basis.


\section{The {\lsdalton} library bundle}
\label{sec:LSlib}

\dots Simen will write this section soon!!!!! \dots





\chapter{Recommendations }\label{recommendations}

In this chapter we give an few recommendations to keywords that is advantages to use 

\section{Small Molecular systems}

the {\lsdalton} program have been design to treat large molecular systems and as such the performance for small molecules is not expected to be competative.
\begin{verbatim}
**WAVE FUNCTIONS
.HF
*DENSOPT
.RH
.DIIS
.CONVTHR
1.0D-6
*END OF INPUT
\end{verbatim}
The .RH (.i.e. Roothaan-Hall) and .DIIS keywords requests a standard diagonalization combined with the DIIS scheme for convergence acceleration, which is usually the fast option for small uncomplicated molecules. 
With the keyword .CONVTHR (i.e. convergence threshold) it is requested that iterations are terminated when the Frobenius norm of the SCF gradient is smaller than 10$^{-6}$. 

\section{Large Molecular systems}

the {\lsdalton} program have been designed to treat large molecular systems and the default 
settings, will mostly likely work well. However, large molecular systems present a number of 
challengens and we suggest to use the following Keywords.

\begin{verbatim}
**INTEGRAL
.DENSFIT
**WAVE FUNCTIONS
.DFT
B3LYP
*DENSOPT
.ARH
.START
TRILEVEL
.CONVDYN
STANDARD
*END OF INPUT
\end{verbatim}
Under **INTEGRAL, we have requested the use of density-fitting to speed up the calculation of 
the Coulomb contribution (the use of J-engine is default). The exact exchange contribution will not 
be calculated using density fitting, as no such method have been implemented in the {\lsdalton} program.

Under **WAVE FUNCTIONS, we now request a DFT
calculation using the B3LYP exchange-correlation functional.

Since this is a calculation on a large molecule, we request the trilevel starting 
guess\cite{trilevel1, trilevel2} (atomic, valence, and then full optimization) under **DENSOPT. 
The Trilevel starting guess usually provide a good starting guess 
for the HF/KS calculation, especially for large molecular systems. 
 
We employ the The Augmented Rothaan-Hall (ARH) method for density optimization. 
Note that ARH is default, so this keyword is actually unnecessary.
ARH is more robust than the standard diagonalization/DIIS scheme.

We also use the dynamical convergence threshold. 

\section{Large MPI Jobs}

For large HF/KS calculation all matrix operations are parallized through the use of the SCALAPACK library, which is activated through the .SCALAPACK keyword 

\begin{verbatim}
**WAVE FUNCTIONS
.DFT
B3LYP
*DENSOPT
.SCALAPACK
.ARH
.START
TRILEVEL
.CONVDYN
STANDARD
*END OF INPUT
\end{verbatim}

\section{Response Calculations}

For response calculations we recommend to use a tight convergence threshld for the initial SCF optimization. 

\begin{verbatim}
.CONVDYN
STANDARD
\end{verbatim}

\section{Accuracy}

In order to debug other programs against the {\lsdalton} program, it may be advantages to remove screening and other approximation that reduce the accuarcy of the calculation, at the expense of speed.

\begin{verbatim}
.THRESH
1.d-20
\end{verbatim}
sets the main integral screening threshold which sets all other thresholds in the integral code. 
\begin{verbatim}
.NO SCREEN
\end{verbatim}
removes all screening. 
\begin{verbatim}
.NO OMP 
\end{verbatim}
Deactivats the use of OpenMP, which makes the code non deterministic. Alternative the 
\begin{verbatim}
export OMP_NUM_THREADS=1
\end{verbatim}
sets the number of OpenMP threads to 1.

\section{Comparison}

In order to compare the {\lsdalton} program
we suggest to deactivate the so called grand canonical basis. 
\begin{verbatim}
.NOGCBASIS
\end{verbatim}
This keyword deactivates the use of the grand canonical basis and uses the input basis instead. 
\begin{verbatim}
.NOFAMILY
\end{verbatim}
Deactivates the exploitation of family basis sets where s and p functions share the same exponents. 







%\chapter{\label{chap:wf-guide}Getting the wave function you want}
 
Currently the following wave function types are implemented in DALTON:

\begin{description}

\item[RHF] Closed-shell singlet and Open-shell doublet (only one open shell)
restricted Hartree-Fock (SCF, self-consistent
field)\index{HF}\index{SCF}\index{Hartree-Fock}. 

\item[MP2] M{\o}ller-Plesset second-order perturbation
theory\index{MP2}\index{M\o ller-Plesset} based on
closed-shell restricted Hartree-Fock.  Energy, second-order one-electron
density matrix and first-order wave function (for SOPPA) may be
calculated, as well as first-order geometry optimizations using
numerical gradients.

\item[CI] Configuration interaction\index{CI}\index{configuration
interaction}. Two types of configuration selection 
schemes are implemented: the complete active space model (CAS) and the
restricted active space model (RAS).

\item[MCSCF] Multiconfiguraion self-consistent field based on
the complete active space model (CASSCF) and
the restricted active space model
(RASSCF)\index{MCSCF}\index{CASSCF}\index{RASSCF}.

\end{description}

SIRIUS is the part of the code that computes all of these
wave functions. 
This part of the DALTON manual discusses aspects of converging to the
wave function you want.

\section{\label{sec:ig_necinp} Necessary input to SIRIUS}
 
In order for DALTON to perform a calculation for a given wave function a
minimum amount of  
information is required. This information is collected in the
\Sec{*WAVE FUNCTION} input module. Fig.~\ref{fig-nec.inp.} collects the most
essential information.
 
\begin{figure}
    \newlength{\mpwidth}
    \settowidth{\mpwidth}{\tt M}
    \addtolength{\mpwidth}{65\mpwidth}
\begin{tabular}{|c|}
\hline
\begin{minipage}{\mpwidth}
\begin{verbatim}
 
   Calculation type(s): RHF, MP2, CI, MCSCF.
 
   Number of symmetries (needed only for MCSCF wave functions)
   If MCSCF or CI calculation then
      Symmetry of wave function.
      Spin multiplicity of wave function.
   else if one open shell RHF
      Symmetry of open shell, doublet spin multiplicity
   else
      Totally symmetric, singlet spin multiplicity
   end if
 
   Number of basis orbitals.
   Number of molecular orbitals.
 
   Number of inactive orbitals in each symmetry.
   If MCSCF or CI calculation then
      Number of active electrons.
      If CAS calculation desired then
         Number of active orbitals in each symmetry.
      else if RAS calculation desired then
         Number of active orbitals in RAS1 space in each symmetry.
         Number of active orbitals in RAS2 space in each symmetry.
         Number of active orbitals in RAS3 space in each symmetry.
         Limits on number of electrons in RAS1 and RAS3.
      end if
   end if
 
   if not MP2 then convergence threshold.
 
   how to begin the calculation (initial guess).
 
\end{verbatim}
\end{minipage} \\ \hline
\end{tabular}
\vspace{0.5cm}
\caption{Necessary input information for determining the wave
function.}\label{fig-nec.inp.} 
\end{figure}
 
 
\section{\label{sec:si-ex} An input example for SIRIUS}

\noindent
The number of symmetries and the number of basis orbitals are read from
the one-electron integral file.  The number of symmetries may also be
specified in the \Sec{*WAVE FUNCTION} input module as a way of
obtaining a consistency check between the wave function input section
and available integral files. 
 
\noindent
The following sample input specifies a CASSCF\index{CASSCF}
calculation on water\index{water} 
with maximum use of defaults:
 
\vbox{
\begin{verbatim}
**WAVE FUNCTIONS
.MCSCF              -- specifies that this is an MCSCF calculation
*CONFIGURATION INPUT
.SPIN MULTIPLICITY  -- converge to singlet state
    1
.SYMMETRY           -- converge to state of symmetry 1 (i.e. A1)
    1
.INACTIVE ORBITALS  -- number of doubly occupied orbitals each symmetry
    1    0    0    0
.CAS SPACE          -- selects CAS and defines the active orbitals.
    4    2    2    0
.ELECTRONS          -- number of electrons distributed in the active
    8                  orbitals.
*ORBITAL INPUT
.MOSTART            -- initial orbitals are orbitals from a previous run
 NEWORB                (for example MP2 orbitals)
**END OF INPUTS
\end{verbatim}
} % end vbox
 
\noindent
Comparing to the list of necessary input information in
Fig.~\ref{fig-nec.inp.},
the following points may be noted:
\noindent
The number of basis functions are read from the one-electron integral
file. Because no orbitals are going to be deleted, the number of
molecular orbitals will be the same as the number of basis functions.
\noindent
No state has been specified as ground state calculation is the default.
\noindent
The default convergence threshold\index{convergence threshold} of
1.0D-5 is used. 
 
 
\clearpage
\section{\label{sec:ig_hints} Hints on the structure of the input for
the \\\Sec{*WAVE FUNCTION} input module}
 
 
The following list attempts to show the interdependence of the various
input modules through indentation  
(\kw{*R*}: required input) :
 
\begin{verbatim}
**WAVE FUNCTION
  .TITLE
  .NSYM                    (1)
  .HF
     *HF INPUT
       .HF OCCUPATION 
       ...
     *ORBITAL INPUT        (3)
     *STEP CONTROL         (if QCHF)
  .MP2
     *MP2 INPUT
     *ORBITAL INPUT        (3)
     *HF INPUT
       .HF OCCUPATION 
  .CI
     *CONFIGURATION INPUT  (2)
     *ORBITAL INPUT        (3)
     *CI VECTOR            (4)
     *CI INPUT
  .MCSCF
     *CONFIGURATION INPUT  (2)
     *ORBITAL INPUT        (3)
     *CI VECTOR            (4)
     *OPTIMIZATION
     *STEP CONTROL
  .STOP
  .RESTART
  .INTERFACE
  .PRINT
*HAMILTONIAN
*TRANSFORMATION
*POPULATION ANALYSIS
*PRINT LEVELS
*AUXILLIARY INPUT
\end{verbatim}
 
%\clearpage
{\bf Notes:}
 
\begin{verbatim}
 
(1) NSYM is normally provided by the integral module. May be provided
    for consistency checking.

(2) *CONFIGURATION INPUT
      .SPIN MULTIPLICITY (*R*)
      .SYMMETRY (*R*)
      .INACTIVE ORBITALS (*R*)

      .ELECTRONS (*R*)
      Either
        .CAS SPACE
      or some or all of
        .RAS1 SPACE
        .RAS2 SPACE
        .RAS3 SPACE
        .RAS1 HOLES or .RAS1 ELECTRONS
        .RAS3 ELECTRONS
      are required.
 
(3) *ORBITAL INPUT
      .MOSTART (*R*)
 
      .SYMMETRIC ORTHONORMALIZATION
      .GRAM-SCHMIDT ORTHONORMALIZATION
 
      .AVERAGE
      .FROZEN CORE ORBITALS
      .FREEZE ORBITALS
 
      .5D7F9G
      .AO DELETE
      .DELETE
 
      .REORDER MO'S
 
      .PUNCHINPUTORBITALS
      .PUNCHOUTPUTORBITALS
 
(4) *CI VECTOR
      One and only one of the following three
        .STARTHDIAGONAL
        .STARTOLDCI
        .SELECT CONFIGURATION
 
      .CNO START
\end{verbatim}
 
\pagebreak[3]
\section{\label{sec:ig_restart} How to restart a wave function calculation}
 
It is possible to restart\index{restart} after the last finished macro
iteration in 
(MC)SCF, provided the "f21"\index{fort.21} file was saved.  You will
then at most loose 
part of one macro iteration in case of system crashes, disks running
full, or other irregularities.
\noindent 
In general the only change needed to the wave function input is to add the
\Key{RESTART} keyword under \Sec{*WAVE FUNCTION}.
If the correct two-electron
integrals over molecular orbitals are available, you can skip the
integral transformation\index{integral transformation} in the
beginning of the macro iteration by 
means of the \Key{OLD TRANSFORMATION} keyword under
\Sec{TRANSFORMATION}.
When resubmitting the job, you must make sure that the correct "f21"
file is made available to the job. 
%(specify the old "f21" file using
%the format described in Section~\ref{VM-runs} for VM/CMS
%and in Section~\ref{AIX-runs} for AIX).
 
\noindent
%As done in Example~\ref{ex-6.1.4}, you can also use the restart feature if you
You can also use the restart feature if you
find it desirable to converge\index{convergence threshold} the wave
function sharper than specified 
in the original input. By default, SCF wave functions are converged to
1.0D-6, whereas MCSCF and CI wave functions are converged to 1.0D-5.
 
\pagebreak[2]
\section{\label{sec:ig_orbtransfer}
Transfer of molecular orbitals between different computers}
 
In order to be able to transfer molecular orbital coefficients between
different computer systems, 
an option is provided for
formatted punching and formatted reading of the molecular orbital
coefficients.  The options are
\begin{center}
   \Key{PUNCHOUTPUTORBITALS} or \Key{PUNCHINPUTORBITALS}
\end{center}
in input group \Sec{ORBITAL INPUT}.
The first option
is used to punch the orbitals at the end of the optimization, while the
second option is used to punch some molecular orbitals\index{molecular
orbital} already available 
on the "f21" file, for instance the converged orbitals from a previous
calculation.  The orbitals can then be transferred to another computer
and appended to the SIRIUS input there.
The orbitals may then be read by \siraba\ using the \Key{MOSTART}
followed by \Key{FORM18} option  (with this option \siraba\ will,
after having finished the \Sec{*WAVE FUNCTION} input module,
search the input file for either the \Sec{*MOLORB} keyword
or the \Sec{*NATORB} keyword
and expect the orbital coefficients to follow immediately afterwards).


\section{\label{examples} Wave function input examples}

Those who are used to the old {\sc sirius} input will not experience
any dramatic changes with respect to the DALTON
input, although a few minor differences exist in order
to create a more user-friendly environment. The input sections are the
same as before starting with a wavefunction specification, currently,
including options for SCF, MP2, and MCSCF reference states. Depending
on this choice of reference state the following input sections take
different form, from the simplest SCF input to the more complex MCSCF
input. Minimal SCF and SCF+MP2
\index{HF}\index{Hartree-Fock}\index{SCF}\index{MP2}\index{M\o
ller-Plesset} 
using the new feature of generating the HF
occupation on the basis of an initial H\"{u}ckel
calculation\index{H\"{u}ckel} and then 
possibly change the occupation during the first DIIS\index{DIIS}
iterations will be shown. There will also be an example showing how an
old f21 file is used for restart in the input. \index{fort.21}\index{restart}
In this case the Hartree-Fock
occupation will be then read from the f21-file and used as initial
Hartree-Fock occupation\index{HF occupation}\index{Hartree-Fock occupation}.
The HF occupation is needed for an
MCSCF calculation\index{MCSCF}, as this is anyway determined when
establishing a 
suitably chosen active space\index{active space}. An example of an
MCSCF-CAS\index{CASSCF} input without starting 
orbitals\index{starting orbitals} will be give, as well as an
MCSCF-RAS\index{RASSCF} with starting orbitals. We note that and MCSCF
wave function by 
default will be optimized using spin-adapted configurations
(CSFs)\index{CSF}\index{configuration state function},
unless the wave function optimization is followed by a calculation of
molecular properties or geometry optimization, when
determinants\index{determinants} will 
be used. The three following examples illustrate calculations
on excited states\index{excited state}, on a core
hole\index{core hole} state with frozen core orbital\index{frozen core} and 
on a core hole state with relaxed core\index{relaxed core} orbital.
The last example shows an MCSCF calculation of non-equilibrium 
solvation energy\index{non-equilibrium solvation}. 

\bigskip

The following input example gives a minimal
SCF\index{SCF}\index{HF}\index{Hartree-Fock} input with default 
starting  orbitals (that is, H\"{u}ckel guess), and automatic Hartree-Fock
occupation, first based on the H\"{u}ckel\index{H\"{u}ckel} guess, and
then updated during the DIIS\index{DIIS} iterations:

\begin{verbatim}
**DALTON INPUT
.RUN WAVE FUNCTIONS
**WAVE FUNCTIONS
.HF
*END OF DALTON
\end{verbatim}
\label{sirius_ex1a}

The following input give a minimal input for an
MP2\index{MP2}\index{M\o ller-Plesset}
calculation using all default settings for the Hartree-Fock
calculation (see previous example): 

\begin{verbatim}
**DALTON INPUT
.RUN WAVE FUNCTIONS
**WAVE FUNCTION
.HF
.MP2
*END OF DALTON
\end{verbatim}
\label{sirius_ex1b}

If we would like to calculate molecular properties in several
geometries, we may take advantage of the fact the molecular orbitals
at the previous geometry probably is quite close to the optimized MOs
at the new geometry, and thus restart\index{restart} from the MOs
contained in the f21\index{fort.21} file, as indicated in the
following example: 

\begin{verbatim}
**DALTON INPUT
.RUN WAVE FUNCTIONS
**WAVE FUNCTION
.HF
*ORBITAL INPUT
.MOSTART
 NEWORB
*END OF DALTON
\end{verbatim}
\label{sirius_ex2}

\begin{center}
\fbox{
\parbox[h][\height][l]{12cm}{
\small
\noindent
{\bf Reference literature:}
\begin{list}{}{}
\item Restricted-step second-order MCSCF: H.J.Aa.Jensen, and H.\AA
gren. \newblock {\em Chem.Phys.Lett.}, {\bf 110},\hspace{0.25em}1140,
(1984). 
\item Restricted-step second-order MCSCF: H.J.Aa.Jensen, and H.\AA
gren. \newblock {\em Chem.Phys.}, {\bf 104},\hspace{0.25em}229,
(1986). 
\item MP2 natural orbital occupation analysis: H.J.Aa.Jensen, P.J\o
rgensen, H.\AA gren, and J.Olsen. \newblock {\em J.Chem.Phys.}, {\bf
88},\hspace{0.25em}3824, (1987).
\end{list}
}}
\end{center}

The next input example gives the necessary input for an Complete
Active Space SCF (CASSCF)\index{CASSCF} calculation where we use MP2 to provide
starting orbitals\index{MP2}\index{M\o ller-Plesset}\index{starting
orbitals} for our MCSCF. The 
active space may for instance be 
chosen on the basis of an MP2 natural orbital occupation
analysis\index{active space} as
described in Ref.~\cite{hjajpjhajojcp88}. The input would like:

\begin{verbatim}
**DALTON INPUT
.RUN WAVE FUNCTIONS
**WAVE FUNCTIONS
.HF
.MP2
.MCSCF
*HF INPUT
.HF OCCUPATION
 2 0 0 0
*CONFIGURATION INPUT
.SYMMET
 1
.SPIN MULT
 1
.INACTIVE
 0 0 0 0
.ELECTRONS
 4
.CAS SPACE
 6 4 4 0
*ORBITAL INPUT
.NOSUPSYM
*END OF DALTON
\end{verbatim}
\label{sirius_ex3}

As for Hartree-Fock calculation, we may want to use available
molecular orbitals on the fort.21\index{fort.21} file from previous
calculations as starting orbitals\index{starting orbitals} for our
MCSCF\index{RASSCF} as indicated in this input example: 

\begin{verbatim}
**DALTON INPUT
.RUN WAVE FUNCTIONS
**WAVE FUNCTION
.MCSCF
*HF INPUT
.HF OCCUPATION
 2 0 0 0
*CONFIGURATION INPUT
.SYMMET
 1
.SPIN MULT
 1
.INACTIVE
 0 0 0 0
.ELECTRONS
 4
.RAS1 SPACE
 2 1 1 0
.RAS2 SPACE
 2 2 2 0
.RAS3 SPACE
 6 4 4 2
.RAS1 ELECTRONS
 0 2
.RAS3 ELECTRONS
 0 2
*OPTIMIZATION
.TRACI
.FOCKONLY
*ORBITAL INPUT
.MOSTART
 NEWORB
*END OF SIRIUS
\end{verbatim}
\label{sirius_ex4}

\begin{center}
\fbox{
\parbox[h][\height][l]{12cm}{
\small
\noindent
{\bf Reference literature:}
\begin{list}{}{}
\item Optimal orbital trial vectors: H.J.Aa.Jensen, P.J\o rgensen,
and H.\AA gren. \newblock {\em J.Chem.Phys.}, {\bf
87},\hspace{0.25em}451, (1987).
\item Excited state geometry optimizations: A.Cesar, H.\AA gren,
T.Helgaker, P.J\o rgensen, and H.J.Aa.Jensen. \newblock {\em
J.Chem.Phys.}, {\bf 95},\hspace{0.25em}5906, (1991).
\end{list}
}}
\end{center}

The next input describes the optimization of the first excited
state\index{excited state}
of the same symmetry as the ground state. To speed up convergence, we
employ optimal orbital\index{optimal orbital trial vector} trial
vectors as described in 
Ref.~\cite{hjajpjhajcp87}. Such an input would look like:

\begin{verbatim}
**DALTON INPUT
.RUN WAVE FUNCTIONS
**WAVE FUNCTION
.TITLE
4 2 2 0 CAS on first excited 1A_1 state, converging to 1.D-07
.MCSCF
.NSYM
  4
*CONFIGURATION INPUT
.SYMMETRY
  1                 | same symmetry as ground state
.SPIN MULTIPLICITY
  1
.INACTIVE
 1 0 0 0
.ELECTRONS
 8
.CAS SPACE
 4 2 2 0
*ORBITAL INPUT
.MOSTART
 NEWORB             
*CI VECTOR
.STARTHDIAG         | Compute start vector from Hessian CI-diagonal
*OPTIMIZATION
.THRESHOLD
  1.D-07
.SIMULTANEOUS ROOTS
  2 2
.STATE
  2                 | 2 since the first exited state has the same symmetry
!                     as the ground state
.OPTIMAL ORBITAL TRIAL VECTORS
*PRINT LEVELS
.PRINTUNITS
 6 6
.PRINTLEVELS
 5 5
**END OF DALTON
\end{verbatim}
\label{sirius_ex5}

\begin{center}
\fbox{
\parbox[h][\height][l]{12cm}{
\small
\noindent
{\bf Reference literature:}
\begin{list}{}{}
\item Core hole: H.\AA gren, and H.J.Aa.Jensen. \newblock {\em Chem.Phys.Lett.}, {\bf 137},\hspace{0.25em}431, (1987).
\item Core hole: H.\AA gren, and H.J.Aa.Jensen. \newblock {\em
Chem.Phys.}, {\bf 172},\hspace{0.25em}45, (1993).
\end{list}
}}
\end{center}

The next input describes the calculation of a core-hole\index{core
hole} state of the 
carbon 1s orbital in carbon monoxide\index{carbon monoxide}, the first
example employing a frozen core\index{frozen core}:

\begin{verbatim}
**DALTON INPUT
.RUN WAVE FUNCTIONS
**WAVE FUNCTION
.TITLE
C1s core hole state of CO, 4 2 2 0 valence CAS + C1s.
Frozen core orbital calculation.
.MCSCF
.NSYM
  4
*CONFIGURATION INPUT
.SYMMETRY
  1
.SPIN MULTIPLICITY
  2                | doublet spin symmetry because of opened core orbital
.INACTIVE
 2 0 0 0           | O1s and O2s orbitals are inactive while the
!                    opened core orbital, C1s, must be active
.ELECTRONS
 9                 | all valence electrons plus the core hole electron
!                    are active
.RAS1 SPACE
 1 0 0 0           | the opened core orbital (OBS always only this orbital)
.RAS1 ELECTRONS
 1 1               | We impose single occupancy for the opened core orbital
.RAS2 SPACE
 4 2 2 0           | The same as the CAS space in the ground state calculation
*OPTIMIZATION
.COREHOLE
 1 2               | symmetry of core orbital and the orbital in this symmetry
!                    with the core hole according to list of input orbitals.
!                    The same thing could be obatined by reordering
!                    the core orbital to the first active orbital (by .REORDER),
!                    and specifying .FREEZE and .NEO ALWAYS
.TRACI
*ORBITAL INPUT
.MOSTART
 NEWORB            | start from corresponding MCSCF ground state
*CI VECTOR
.STARTHDIAG
**END OF DALTON
\end{verbatim}
\label{sirius_ex6}


whereas we in a calculation where we would allow the core to
relax\index{relaxed core} only
would require the following changes compared to the previous input,
assuming that we start out from orbitals and CI vectors generated by the
previous calculation  ".STARTHDIAG" is therefore replaced by
".STARTOLDCI", and the core orbital has number 3 (instead of 1)
in the list of orbitals in the first symmetry. ".CORERELAX" is specifed
for relaxation of the core orbital using the NR algorithm.

\begin{verbatim}
*OPTIMIZATION
.COREHOLE
 1 3
.CORERELAX
*CI VEC
.STARTOLDCI
\end{verbatim}
\label{sirius_ex7}

\begin{center}
\fbox{
\parbox[h][\height][l]{12cm}{
\small
\noindent
{\bf Reference literature:}
\begin{list}{}{}
\item General reference: K.V.Mikkelsen, E.Dalgaard,
P.Svanstr{\o}m. \newblock {\em J.Phys.Chem}, {\bf
91},\hspace{0.25em}3081, (1987).
\item General reference: K.V.Mikkelsen, H.{\AA}gren, H.J.Aa.Jensen,
and T.Helgaker. \newblock {\em J.Chem.Phys.}, {\bf
89},\hspace{0.25em}3086, (1988).
\item Non-equilibrium solvation: K.V.Mikkelsen, A.Cesar, H.{\AA}gren,
H.J.Aa.Jensen.\newblock {\em J.Chem.Phys.}, {\bf
103},\hspace{0.25em}9010, (1995).
\end{list}
}}
\end{center}

This example describes calculations for non-equilibrium
solvation\index{non-equilibrium solvation}.
Ususally one starts with a calculation of a reference state
(most often the ground state) with equilibrium solvation, using
keyword "INESRF". The interface file is then used (without user 
interference) for a non-equilibrium excited state calculation; keyword
"INERSI". 

\begin{verbatim}
**DALTON INPUT
.RUN WAVE FUNCTIONS
**WAVE FUNCTION
.TITLE
 2-RAS(2p2p') : on F+ (1^D) in Glycol 
.MCSCF
.NSYM
 8
*CONFIGURATION INPUT
.SPIN MULTIPLICITY
 1
.SYMMETRY
 1
.INACTIVE ORBITALS
 1  0  0  0  0  0  0  0 
.ELECTRONS
 6
.RAS1 SPACE
 0  0  0  0  0  0  0  0
.RAS2 SPACE
 1  2  2  0  2  0  0  0
.RAS3 SPACE
 8  4  4  3  4  3  3  1
.RAS1 ELECTRONS 
 0  0 
.RAS3 ELECTRONS 
 0  2 
*OPTIMIZATION
.NEO ALWAYS
.OPTIMA
*ORBITAL INPUT
.MOSTART
 NEWORB
*CI VECTOR
.STARTOLDCI
*SOLVENT
.CAVITY
 2.5133D0
.INERSI   | initial state inertial polarization
 37.7D0  2.050D0  | static and optic dielectric constants for Glycol
.MAX L
 10
.PRINT
 6
*END OF DALTON
\end{verbatim}
\label{sirius_ex8}
 

%\chapter{Potential energy surfaces}\label{ch:geometrywalks}
\index{geometry walk}\index{geometry optimization}\index{convergence!geometry}

This section describes one of the most important features of any
quantum chemical software package; locating equilibrium
geometries\index{equilibrium structure} and
transition structures\index{transition state} of molecules. In
{\dalton} this is done using either second-order trust-region\index{second-order optimization}\index{geometry optimization!second-order}\index{trust region}
optimizations~\cite{tuhjahjajpjjcp84} (energies, molecular gradients and
molecular Hessians are calculated\index{molecular gradient}\index{molecular Hessian})
or a variety of first-order\index{first-order optimization}\index{geometry optimization!first-order}
methods~\cite{Fletcher} (only energies and molecular gradients are
calculated). These methods
have been implemented for Hartree--Fock, Density Functional Theory,
MCSCF  and various Coupled Cluster wave functions. For other
non-variational wave functions---such as CI---the program can 
only do first-order geometry optimizations using a
numerically\index{numerical gradient}\index{molecular gradient!numerical}
calculated molecular gradient. This will be invoked automatically by the program
in case of a non-variational wave function. Some comments connected to
such geometry optimizations are collected in Sec.~\ref{sec:nonvargeom}.

For historical reasons, {\dalton} actually contains two different
modules for exploring potential energy surfaces\index{PES}\index{potential energy surface}:
\Sec{WALK} which is
a pure second-order module and \Sec{OPTIMIZE} which contains both
first- and second-order methods. While there is a lot of overlap
between the two modules, certain calculations can only be done using
one or the other of the two. The main strength of \Sec{WALK} is its
robustness, while \Sec{OPTIMIZE} focuses more on speed and
efficiency. For regular optimizations of minima and transition states,
the recommended module is \Sec{OPTIMIZE}.

One of the strengths of the {\dalton} program is the
stable algorithms for locating first-order transition states\index{transition state}\index{geometry optimization!transition state}.
As described below, this can done by one of three algorithms in
\Sec{WALK}: by trust-region second-order image surface
minimization~\cite{thcpl182}\index{image surface},
by gradient extremal walks~\cite{pjhjajthtca73}\index{gradient extremal},
or by following a specific mode~\cite{hjajpjthjcp85}\index{mode following} (all
requiring the calculation of the Hessian at every point). These
options will be discussed in more detail below. The \Sec{OPTIMIZE}
module also contains the stable second-order trust-region image
surface minimization (using analytical or approximate Hessians), as
well as a partitioned rational function method~\cite{abnajsrsjpc89}.

Another feature of the {\dalton} program system is the options for
calculating essential information about the potential energy
surface so that subsequent molecular dynamics analysis can be
made. There are two options for doing this in the program: One can
either follow an Intrinsic Reaction Coordinate
(IRC)~\cite{pjhjajthtca73}\index{IRC}\index{intrinsic reaction coordinate}, 
or solve Newton's equations for the nuclei under the
potential put up by the
electrons~\cite{theuhjajcpl173}\index{dynamics}. Whereas the first
option 
can be used for modeling the immediate surroundings of a
molecular reaction path, the second will give a precise
description of {\em one} molecular trajectory. Chemical reactions
may thus be monitored in time from an initial set of starting
conditions. The program will automatically generate a complete
description of the energy distribution into internal energies and
relative translational energies. By calculating a large number of
trajectories, a more complete description of the chemical reaction
may be obtained~\cite{rsgmtjdwrjbjcp106}.

Please note that for geometry optimizations using the Self-Consistent
Reaction Field model, certain restrictions apply, as discussed in
Sec.~\ref{sec:solventgeoopt}.

\section{Locating stationary points}

\subsection{Equilibrium geometries}
\label{sec:minimization}

\begin{center}
\fbox{
\parbox[h][\height][l]{12cm}{
\small
\noindent
{\bf Reference literature:}
\begin{list}{}{}
\item 2nd-order methods: T.U.Helgaker, J.Alml\"{o}f, H.J.Aa.Jensen,
and P.J{\o}rgensen.\newblock {\em J.Chem.Phys.}, {\bf
84},\hspace{0.25em}6266, (1986).
\item 1st-order methods: V.Bakken, and T.Helgaker, \newblock {\em J.Chem.Phys.}, {\bf 117},\hspace{0.15em} 9160 (2002).
\item Excited-state optimization: A.Cesar, H.\AA gren, T.Helgaker, P.J\o rgensen, and H.J.Aa.Jensen.\newblock {\em J.Chem.Phys.}, {\bf
95},\hspace{0.25em}5906, (1991).
\item Solvent Hessian: P.-O.\AA strand, K.V.Mikkelsen, K.Ruud and
T.Helgaker. \newblock {\em J.Phys.Chem.}, {\bf
100},\hspace{0.25em}19771, (1996).
\end{list}
}}
\end{center}

A short input file for the location of a molecular geometry
corresponding to minimum molecular energy\index{geometry optimization!equilibrium geometry}
(using a second-order optimization\index{second-order optimization}\index{geometry optimization!second-order}
method) and performing a vibrational
analysis\index{vibrational analysis} at the stationary point, as well
as determining the nuclear
shielding\index{nuclear shielding} constants at the optimized
geometry, will look like
(assuming a Hartree--Fock wave function):

\begin{verbatim}
**DALTON INPUT
.OPTIMIZE
*OPTIMIZE
.2NDORD
**WAVE FUNCTIONS
.HF
**PROPERTIES
.VIBANA
.SHIELD
**END OF DALTON INPUT
\end{verbatim}

The keyword \Key{OPTIMIZE} by default signals a search for a minimum,
that is, a stationary point on the molecular surface with Hessian
index 0. With the \Sec{OPTIMIZE}
module the second-order method has to be explicitly requested
(\Key{2NDORD})~\cite{tuhjahjajpjjcp84}, as first-order methods are the default. At the final,
optimized geometry the input in the \Sec{*PROPE} section requests a
vibrational analysis and the evaluation of nuclear shieldings.

In second-order methods the Hessian\index{Hessian} is calculated at every
geometry. The analytical Hessian naturally gives a better description
of the potential energy surface than the approximate Hessians of
first-order methods, and fewer steps will usually be needed to reach
the minimum~\cite{vbthjcp117}. However, the price one has to pay for this, is that each
iteration uses significantly more CPU time. Another advantage of
second-order methods, is that the program will automatically break the symmetry\index{symmetry!breaking}\index{geometry optimization!symmetry breaking}
of the molecule, if needed, in order to reach a minimum
(unless the user specifies that symmetry should not be broken). If
one wants to minimize a water molecule starting from a linear
geometry and using molecular symmetry, a first-order method will
``happily'' yield a linear geometry with optimized bond lengths, whereas
the second-order method will correctly decide that symmetry should be
reduced to C$_{2v}$ and minimize the energy of the molecule.
The bottom-line is, however, that while second-order methods
definitely are the more robust, they will generally be outperformed by
the much more time-efficient first-order methods.

If you are convinced that the symmetry of your starting geometry
is the correct, you may take advantage of the fact there will only
be forces in the totally symmetric modes of the molecules,
and thus only evaluate the Hessian in the totally symmetric
Cartesian symmetry coordinates. This can be achieved with the
input:

\begin{verbatim}
**DALTON INPUT
.WALK
.TOTSYM
**WAVE FUNCTIONS
.HF
**PROPERTIES
.VIBANA
.SHIELD
**END OF DALTON INPUT
\end{verbatim}

In case a vibrational analysis has been requested at the optimized
geometry, the program will recalculate the complete molecular Hessian
at the optimized geometry in order to be able to evaluate all
frequencies. Note that only the \Sec{WALK} module, indicated by the
keyword \Key{WALK}, supports the \Key{TOTSYM} keyword.
We also note that any requests for static
polarizabilities\index{polarizability}, Cioslowski population
analysis\index{population analysis}, dipole 
gradients\index{dipole gradient} or vibrational circular dichroism
(VCD)\index{vibrational circular dichroism}\index{VCD} will be ignored
as all the necessary property vectors will not be available. However,
at the final optimized geometry, the full Hessian will be evaluated,
as may then also these properties.

For direct calculations\index{direct calculation}---where the
two-electron integrals are not stored on disc---the cost of a
second-order geometry optimization algorithm may become
prohibitively expensive, despite the rather few iterations normally
needed to converge the geometry. In these cases, it may be
advantageous to use first-order geometry
optimization\index{first-order optimization}\index{geometry optimization!first-order} routines as illustrated in the following input:

\begin{verbatim}
**DALTON INPUT
.OPTIMIZE
**WAVE FUNCTIONS
.HF
**PROPERTIES
.VIBANA
.SHIELD
**END OF DALTON INPUT
\end{verbatim}

This will use the default method of \Key{OPTIMIZE}: a first-order
optimization~\cite{Fletcher} in redundant internal
coordinates~\cite{gfxfzpwtppjacs114,ppgfjcp96,vbthjcp117}\index{redundant internal coordinates}\index{coordinate system!redundant internal coordinates}
using the BFGS\index{BFGS update}\index{Hessian update!BFGS}
Hessian update. This means that
only the energy and the molecular gradient are calculated in each iteration, and
an approximate molecular Hessian is obtained using the gradients and the
displacements. The initial molecular Hessian is diagonal in the redundant
internal coordinates~\cite{rlabgkpamcpl241}.
To obtain the properties at the optimized 
geometry, the Hessian has to be calculated. By looking at the
vibrational frequencies one can then verify that a true minimum has
been reached (as all frequencies should be real, corresponding to a
positive definite Hessian).

Several first-order methods have been implemented in \dalton\, the
recommended method being the Broyden-Fletcher-Goldfarb-Shanno (BFGS)
Hessian updating\index{BFGS update}\index{Hessian update!BFGS}
scheme. Though other updates may perform better in
certain cases, the BFGS update generally seems to be the most
reliable and thus is the default method for minimization~\cite{vbthjcp117}.

Without the calculation of the Hessian at the optimized geometry, one
can never be sure that the geometry is indeed a minimum, and this is the
main problem with first-order methods.

Both first- and second-order methods are based on
trust-region\index{trust region}\index{geometry optimization!trust region}
optimization. The trust-region is the region of the potential energy
surface where our quadratic model (based on an analytical or
approximate Hessian) gives a good representation of the true
surface. This region is given the shape of a
hypersphere. In the trust-region based optimization algorithm, the
radius of the trust-region is
automatically updated during the optimization by a feedback mechanism.
Occasionally, however, the potential surface may show locally large
deviations from quadratic form. This will result in a large
disagreement between the predicted energy change and the energy
calculated at the new geometry. If this deviation is larger than a
given threshold, the step will be reduced through a decrease in the
trust radius and a simple line search will be performed. A quadratic
function is fitted 
to the rejected energy and the previous energy and molecular gradient. The
minimum of this function is then used as the new step, and this
process may be repeated.

Another consideration concerning optimizations, is the choice of\index{coordinate system}\index{geometry optimization!coordinate system}
coordinate system. Second-order methods do not seem to be very
sensitive to this, and the \Sec{WALK} module thus only uses
Cartesian coordinates. However, the coordinate system may be crucial
to the performance of first-order methods, especially when starting
from approximate Hessians. \Sec{OPTIMIZE} provides the normal Cartesian
coordinates\index{Cartesian coordinates}\index{coordinate system!Cartesian coordinates}
and redundant internal coordinates,\index{redundant internal coordinates}\index{coordinate system!redundant internal coordinates}
the latter being highly recommended when using
first-order methods. Redundant internal coordinates consists of all
bond lengths, angles and dihedral angles in a molecule, and the
redundancies are removed through projection. The default for
second-order methods is Cartesian coordinates, while redundant
internal coordinates are the default for first-order methods, but it is
possible to override this with the keywords \Key{REDINT} and
\Key{CARTES} respectively.

The \Sec{OPTIMIZE} module decides when a minimum is reached through a
set of convergence criteria\index{convergence!geometry, criteria}. These will be
set automatically depending on the accuracy of the wave function
using three different criteria, namely the change in energy
between two iterations, the norm of the molecular
gradient\index{molecular gradient!norm of} and the norm of the
step\index{step!norm of}. The convergence threshold for the change
in energy is the maximum of $1.0\cdot 10^{-6}$ Hartree and two times the
convergence threshold for the wave function, whereas for the norm of
the molecular gradient and step it is the maximum of $1.0\cdot 10^{-4}$ a.u.\ and
two times the 
convergence threshold for the wave function. Two out of these
three criteria have to decrease below their threshold before the program
declares convergence. These criteria ensure a good geometry
suitable for vibrational analysis, but may be unnecessary tight if
one only wants the energy.

The convergence criteria can be controlled through the input file. One
should keep in mind that while second-order methods are rather
insensitive to changes in these thresholds due to their quadratic
convergence, first-order methods may run into severe trouble if the
criteria are too close to the accuracy of the wave function and the
molecular gradient. The quickest way to get somewhat looser convergence
criteria, is to use the keyword \Key{BAKER} enforcing the convergence
criteria of Baker\cite{Baker} (see the Reference Manual for
details). This will normally give energies converged to within
$1.0\cdot 10^{-6}$ Hartree.

In case of non-variational\index{non-variational wave functions}
wave functions, where a numerical 
gradient\index{molecular gradient!numerical} is used, the threshold for
convergence is reset to 
$1.0\cdot 10^{-4}$ a.u., due to possible numerical
instabilities occurring in the 
finite difference\index{finite difference} methods employed to
determine the molecular gradient\index{molecular gradient}.

We also note that for the optimization of excited
states\index{excited state!geometry optimization}
where the state of interest is not the lowest of its
symmetry, most of the parameters controlling the trust-region
optimization method will be reset to much more conservative
values, as one in such calculations will lose the possibility of
back-stepping during the wave function optimization, as well as
to ensure that the start wave function at the new geometry
is within the local region of the optimized wave function.

Calculations involving large basis sets may be very time-consuming,
and it becomes very important to start the optimization from a decent
geometry. One way to solve this is through
preoptimization\index{preoptimization}\index{geometry optimization!preoptimization},
that is, one or more smaller basis sets are first used to optimize the
molecule. \Sec{OPTIMIZE} supports the use of up to ten different
basis sets for preoptimization. This may also be used as a very
effective way to get the optimized energy of a molecule for a series
of basis sets. One may also want to calculate the energy of a molecule
using a large basis set at the optimized geometry of a smaller basis
set, and this is also supported in {\dalton}, but limited to only one
single-point basis set. Both these features are illustrated in the
input file below:

\begin{verbatim}
**DALTON INPUT
.OPTIMIZE
*OPTIMIZE
.PREOPT
2
STO-3G
4-31G
.SP BAS
d-aug-cc-pVTZ
**WAVE FUNCTIONS
.HF
**END OF DALTON INPUT
\end{verbatim}

As one can see, two small basis sets have been chosen 
for\index{preoptimization}\index{geometry optimization!preoptimization}
preoptimization. Note that they will be
used in the order they appear in the input file, so one should
sort the sets and put the smaller one at the top for maximum
efficiency. The ``main'' basis set, that is, the one that will be
used for the final optimized geometry, is specified in the file
\molinp\ as usual. After the last optimization, the energy of
the molecule will be calculated at the final geometry with the
d-aug-cc-pVTZ basis. Please note that these features will work only
when using basis sets from the basis set library.

It is possible to control the optimization procedure more closely
by giving the program instructions on how to do the different parts of
the calculation. Below we have listed all the modules that may
affect the geometry optimization, as well as a short description of
which part of the calculation that module controls. The reader is
referred to the section describing each module for a closer
description of the different options. These sections may, with the
exception of the \Sec{OPTIMIZE} and \Sec{WALK} section, be specified
in any of the modules \Sec{*START}, \Sec{*EACH STEP}
and \Sec{*PROPERTIES}. The \Sec{OPTIMIZE} and \Sec{WALK} are
to be placed in the \Sec{*DALTON} module, and will apply to the
entire calculation.

\begin{description}
\item[\Sec{GEOANA}] Describes what kind  of information about the
molecular geometry is to be printed.
\item[\Sec{GETSGY}] Controls the set up of the right-hand sides
(gradient terms).
\item[\Sec{NUCREP}] Controls the calculation of the nuclear contributions.
\item[\Sec{ONEINT}] Controls the calculation of one-electron contributions.
\item[\Sec{OPTIMIZE}] Controls the first- and second-order
optimization methods (both minima and transition states).
\item[\Sec{RELAX}] Controls the adding of solution and right-hand side vectors
into relaxation contributions.
\item[\Sec{REORT}] Controls the calculation of reorthonormalization terms.
\item[\Sec{RESPON}] Controls the solution of the geometric response equations.
\item[\Sec{TROINV}] Controls the use of translation and rotational invariance.
\item[\Sec{TWOEXP}] Controls the calculation of two-electron
expectation values.
\item[\Sec{VIBANA}] Set up the vibrational and rotational analysis of the
molecule.
\item[\Sec{WALK}] Controls the walk (see also ``Locating transition
states'' and ``Doing Dynamical walks'').
\end{description}

\subsection{Transition states using the image method}\index{image surface}
\label{sec:image}

\begin{center}
\fbox{
\parbox[h][\height][l]{12cm}{
\small
\noindent
{\bf Reference literature:}
\begin{list}{}{}
\item T.Helgaker. \newblock {\em Chem.Phys.Lett.}, {\bf
182},\hspace{0.25em}503, (1991).
\end{list}
}}
\end{center}

\index{transition state}\index{geometry optimization!transition state}
Transition states are found as saddle points
on the potential energy surface\index{image surface}.
The simplest way of locating a first-order transition state, which are the
chemically most interesting ones, is to use the trust-region
image\index{image surface}\index{trust region} minimization method
described in Ref.~\cite{thcpl182}. Such geometry optimizations may
be considered as a special case of walks on second-order surfaces, and
can be done using either the \Sec{WALK} module or the \Sec{OPTIMIZE}
module. Just like for minimization the former uses a pure second-order
method (analytical Hessians calculated at every geometry), while the
latter gives you a choice of first- and second-order methods or
combinations of the two.

A second-order optimization of a transition state can be requested by
either adding \Key{SADDLE} and \Key{2NDORD} in the \Sec{OPTIMIZE}
section:
\begin{verbatim}
**DALTON INPUT
.OPTIMIZE
*OPTIMIZE
.SADDLE
.2NDORD
**WAVE FUNCTIONS
.HF
**END OF DALTON INPUT
\end{verbatim}
or by adding the keyword \Key{IMAGE} in the \Sec{WALK} section:
\begin{verbatim}
**DALTON INPUT
.WALK
*WALK
.IMAGE
**WAVE FUNCTIONS
.HF
**END OF DALTON INPUT
\end{verbatim}

Any property may of course be specified at all stages of the
optimization in the same fashion as for geometry minimizations.

The principle behind the trust-region image minimization is simple. A
first-order transition state is characterized by having one negative
Hessian\index{Hessian} eigenvalue. By reversing the sign of this
eigenvalue, we have
taken the mirror image of our potential surface along the associated
mode, thus turning our problem into an ordinary
minimization problem. A global one-to-one correspondence between the
image surface and our potential energy surface is only valid for a
second-order surface, but in general the lack of a global one-to-one
correspondence seldom gives any problems.

The advantage of the trust-region image optimization as compared to
for instance following gradient extremals\index{gradient extremal}
lies mainly in the
fact that we may take advantage of well-known techniques for
minimization. In addition, the method does not need to be started at
a stationary point of the potential surface which is necessary when
following a gradient extremal (in fact, when using \Sec{OPTIMIZE} the
starting geometry should {\em not} be a minimum). We
have so far not experienced that the trust-region image optimization
fails to locate a first order transition state, even though this is by
no means globally guaranteed from the approach. Note that the
first-order saddle-points normally obtained using the image method
starting at a minimum, often corresponds to conformational transition
states, and thus not necessarily to the chemically most interesting
transition states.

There are two approaches for locating several first-order saddle
points with the trust-region image optimization. One may take
advantage of the fact the image method is not dependent upon starting
at a stationary point, and thus start the image minimization from
several different geometries, and thus hopefully ending up at different
first-order saddle points\index{saddle-point}, as the lowest eigenmode
at different
regions of the potential energy surface may lead to different
transition states.

The other approach is to request
that not the lowest mode, but some other eigenmode is to be inverted.
This can be achieved by explicitly giving the mode which is to be
inverted through the keyword \Key{MODE}. This keyword is the same in
both the \Sec{WALK} and the \Sec{OPTIMIZE} module. However, one
should keep in mind that there will be crossings where a given mode
will switch from the chosen mode to a lower mode. However, what will happen
in these crossing points cannot be predicted in advance. Thus, for such
investigations, the gradient extremal approach may prove equally well
suited. Let us give an example of an input for a trust-region
image optimization where the third mode is inverted:

\begin{verbatim}
**DALTON INPUT
.WALK
*WALK
.IMAGE
.MODE
 3
**WAVE FUNCTIONS
.HF
**END OF DALTON INPUT
\end{verbatim}

\subsection{Transition states using first-order
methods}\label{sec:saddle1stord}

While second-order methods are robust, we have already pointed out
in section~\ref{sec:minimization} that Hessians might be expensive to
compute. The \Sec{OPTIMIZE} module therefore provides first-order
methods for locating transition states, where approximate rather than
exact Hessians are used. The step control method is, however, the same
trust-region image optimization.

When locating transition states it is important to have a good
description of the mode that should be maximized ({\it i.e.}~have a negative
eigenvalue). It is therefore not recommended to start off with a
simple model Hessian, but rather to calculate the initial Hessian
analytically. Alternatively it can be calculated using a smaller basis
set/cheaper wave function and then be read in. The minimal input for a
transition state optimization using \Sec{OPTIMIZE} is:
\begin{verbatim}
**DALTON INPUT
.OPTIMIZE
*OPTIMIZE
.SADDLE
**WAVE FUNCTIONS
.HF
**END OF DALTON INPUT
\end{verbatim}
This will calculate the Hessian at the initial geometry, then update
it using Bofill's
update~\cite{jmbjcc15}\index{Bofill's update}\index{Hessian update!Bofill's update}
(the BFGS update is not suitable since
it tends towards a positive definite Hessian). As for minimizations,
redundant internal coordinates are used by default for first-order
methods.

It is highly recommended that a Hessian calculation/vibrational
analysis be performed once a stationary point has been found, to
verify that it's actually a first-order transition state. There should
be one and only one negative eigenvalue/imaginary frequency.

If no analytical second derivatives (Hessians) are available, it is
still possible to attempt a saddle point optimization by starting from
a model Hessian indicated by the keyword \Key{INIMOD}:
\begin{verbatim}
**DALTON INPUT
.OPTIMIZE
*OPTIMIZE
.INIMOD
.SADDLE
**WAVE FUNCTIONS
.HF
**END OF DALTON INPUT
\end{verbatim}
Provided the starting geometry is reasonably near the transition
state, such optimization will usually converge correctly. If not, it
is usually a good idea to start from different geometries and also to
try to follow different Hessian modes, as described in
section~\ref{sec:image} (through the \Key{MODE} keyword).



\subsection{Transition states following a gradient
extremal}\label{sec:gradext}

\begin{center}
\fbox{
\parbox[h][\height][l]{12cm}{
\small
\noindent
{\bf Reference literature:}
\begin{list}{}{}
\item P.J{\o}rgensen, H.J.Aa.Jensen, and T.Helgaker. \newblock {\em
Theor.Chem.Acta}, {\bf 73},\hspace{0.25em}55, (1988).
\end{list}
}}
\end{center}

\index{gradient extremal}
A gradient extremal is defined as a locus of points in the contour
space where the gradient is extremal~\cite{pjhjajthtca73}. These
gradient extremals connect
stationary points on a molecular potential energy surface and are
locally characterized by requiring that the molecular gradient is an
eigenvector of the mass-weighted molecular Hessian at each point on
the line. From a stationary point there will be gradient extremals
leaving in all normal coordinate directions, and stationary points
on a molecular potential surface may thus be characterized by
following these gradient extremals. The implementation of this approach
in \dalton\ is described in Ref.~\cite{pjhjajthtca73}.

Before discussing more closely which keywords are of importance in
such a calculation and how they are to be used, we give an example of
a typical gradient extremal input in a search for a first-order
transition state along the second-lowest mode of a mono-deuterated
ethane\index{ethane} molecule:

\begin{verbatim}
**DALTON INPUT
.WALK
*WALK
.GRDEXT
.INDEX
 1
.MODE
 2
**WAVE FUNCTIONS
.HF
**END OF DALTON INPUT
\end{verbatim}


The request for a gradient extremal\index{gradient extremal}
calculation is controlled by the {\Key{GRDEXT}}, and in this example
we have chosen to follow the second-lowest mode, as specified by the
\Key{MODE} keyword.

As the calculation of the gradient extremal uses mass-weighted
coordinates\index{mass-weighted coordinates}, it is recommended to
specify the isotopic constitution\index{isotopic constitution} of
the molecule. If none is specified, the most abundant
isotope of each atom is used by default. The isotopic constitution of
a molecule is given in the \molinp\ file as described in Chapter~\ref{ch:molinp}.

A requirement in a gradient extremal calculation is that the
calculation is started on a gradient extremal. In practice this is
most conveniently ensured by starting at a stationary point, a minimum
or a transition state. The index\index{Hessian!index} of the critical point
--- that is,
the number of negative Hessian eigenvalues\index{Hessian!eigenvalue}
--- sought, need to
be specified (by the keyword \Key{INDEX}), as the calculation would
otherwise continue until a critical point with index zero
(corresponding to a minimum), is found.

\subsection{Level-shifted mode-following}\label{sec:modfol}

\begin{center}
\fbox{
\parbox[h][\height][l]{12cm}{
\small
\noindent
{\bf Reference literature:}
\begin{list}{}{}
\item C.J.Cerjan, and W.H.Miller. \newblock {\em
J.Chem.Phys.}, {\bf 75},\hspace{0.25em}2800, (1981).
\item H.J.Aa.Jensen, P.J{\o}rgensen, and T.Helgaker. \newblock {\em
J.Chem.Phys.}, {\bf 85},\hspace{0.25em}3917, (1986).
\end{list}
}}
\end{center}

The input needed for doing a level-shifted mode
following\index{mode following} is very similar to the input for
following a gradient extremal\index{gradient extremal}, and the
keyword that is needed in order to invoke this kind of calculation is
\Key{MODFOL}. As for gradient extremals, we need to specify which
mode we follow. However, a mode following does not use mass-weighted
molecular coordinates\index{mass-weighted coordinates} as default, and
isotopic composition of the
molecule is therefore not needed. Note, however, that mass-weighted
coordinates can be requested through the keyword \Key{MASSES} as
described in the input section for the \Sec{WALK} module. A typical
input following the third mode will thus look like:

\begin{verbatim}
**DALTON INPUT
.WALK
*WALK
.MODFOL
.INDEX
 1
.MODE
 3
**WAVE FUNCTIONS
.HF
**END OF DALTON INPUT
\end{verbatim}

The level-shifted mode-following uses an algorithm similar to the
one used in the ordinary geometry optimization of a molecule, but
whereas one in minimizations chooses a step so that the level
shift parameter is less than the lowest eigenvalue of the
molecular Hessian\index{Hessian}, this level shift parameter is
chosen to be in-between the eigenvalues $\lambda_{t-1}$ and
$\lambda_{t}$ if we are following mode number $t$. This approach
was pioneered by Cerjan and
Miller~\cite{cjcwhmjcp75}, and is also described in
Ref.~\cite{hjajpjthjcp85}. As for the gradient extremal approach,
higher-order transition states\index{transition state} can be
requested through the use of the keyword \Key{INDEX}.

Note that it may often be necessary to start the mode-following
calculation by stepping out of the stationary point along the
mode of interest using the keyword \Key{EIGEN} in the
\Key{WALK} module. We refer to the reference manual for a further
description of this option.

The index\index{Hessian!index} of the critical point---that is,
the number of negative Hessian
eigenvalues\index{Hessian!eigenvalue}---sought, need to 
be specified (by the keyword \Key{INDEX}), as the calculation would
otherwise continue until a critical point with index zero
(corresponding to a minimum), is found.

\section{Trajectories and Dynamics}

\subsection{Intrinsic reaction coordinates}\label{sec:irc}

\begin{center}
\fbox{
\parbox[h][\height][l]{12cm}{
\small
\noindent
{\bf Reference literature:}
\begin{list}{}{}
\item K.Fukui. \newblock {\em Acc.Chem.Res.}, {\bf 14},\hspace{0.25em}363, (1981).
\end{list}
}}
\end{center}

A tool that has proved valuable in the study of molecular
dynamics\index{dynamics}
is the use of steepest descent\index{steepest descent}-based algorithms
for following a molecular reaction from a transition
state\index{transition state} towards a minimum. One of the
most successful ways of doing this is the Intrinsic Reaction Coordinate (IRC)
approach~\cite{kfacr14}\index{IRC}\index{intrinsic reaction
coordinate}. The IRC is calculated by taking small steps along the
negative molecular gradient in a mass-weighted 
coordinate\index{coordinates!mass-weighted} system. In {\dalton}, the
size of the step is adjusted with the use of the trust-region based
algorithm. In order to get a sufficiently accurate potential energy surface,
rather small steps must be taken, and the default trust radius is thus reset
to 0.020 when an IRC calculation is being done.

In many respects the input to an IRC calculation is very similar to
the input for a trust-region image\index{image surface} optimization,
and a typical input would look like:

\begin{verbatim}
**DALTON INPUT
.WALK
.MAX IT
 150
*WALK
.IRC
 1
**WAVE FUNCTIONS
.HF
**END OF DALTON INPUT
\end{verbatim}

Most of this input should now be self-explanatory. The request for an
Intrinsic Reaction Coordinate calculation is done by using the keyword
\Key{IRC}. On the next line there is a positive or negative integer
indicating in which direction the reaction should proceed. It is,
however, not possible to determine in advance which reaction path a
given sign is connected to, and the calculation should therefore
always be checked after a few iterations in order to ensure that the
reaction proceeds in the correct direction. If not, the calculation
should be stopped and started from the transition state again with a
different sign for the integer specified after the \Key{IRC} keyword.

As the IRC is defined with respect to mass-weighted
coordinates\index{mass-weighted coordinates}, care
has to be taken in order to specify the correct isotopic substitution
of the molecule. The specification of the isotopic constitution of a
molecule is given in the \molinp\ file, as described in
Chapter~\ref{ch:molinp}. 

Due to the small steps that must be used in a calculation of an IRC,
such a calculation may require a large number of
iterations\index{geometry iteration}, and it
may thus be necessary to increase the maximum number of iterations
that can be taken.  This can be done by
the keyword \Key{MAX IT} in the \Sec{*DALTON} input
section. Default value for this parameter is 20 iterations.

All the information about the Intrinsic Reaction Coordinate will be
collected in a file called \texttt{DALTON.IRC}\index{DALTON.IRC}. If a
calculation stops
because it has reached the maximum number of iterations, it may be
restarted from that point, and the new information about the IRC will
be added to the old \texttt{DALTON.IRC} file. This also implies that if a
calculation is restarted from the beginning (because it went in the
wrong direction) the \texttt{DALTON.IRC} file {\em must\/} be removed
first. Thus it may often be useful to take a backup of the
\texttt{DALTON.IRC} during the calculation of the IRC.

Finally, some comments on the interest of a calculation of IRCs.
Whereas it will give results that mimic the behavior of what is
considered to be a good description of the reaction
pathway\index{reaction pathway} of a
molecular reaction, it does not include any dynamical aspects of the
reaction. There exists several models for approximating local regions of a
molecular potential energy surface from the results of an IRC
calculation, and from the
potential energy surface valley, the dynamics\index{dynamics} of a
chemical reaction may be
mimicked. However, \dalton\ gives another, more direct, opportunity
for studying molecular dynamics through dynamic walks as described in
more detail in the next section.

\subsection{Doing a dynamical walk}\label{sec:dynamic}

\begin{center}
\fbox{
\parbox[h][\height][l]{12cm}{
\small
\noindent
{\bf Reference literature:}
\begin{list}{}{}
\item T.Helgaker, E.Uggerud, H.J.Aa.Jensen. \newblock {\em
Chem.Phys.Lett.}, {\bf 173},\hspace{0.25em}145, (1990).
\end{list}
}}
\end{center}

The theory behind the ``direct dynamics'' as implemented in \dalton\ is
described in Ref.~\cite{theuhjajcpl173}\index{dynamics}. The
main idea behind this
approach is that Newton's equations of motion for the nuclei are
integrated in the presence of the quantum mechanical potential set up
by the electrons. Thus one may follow a molecular reaction from a
given starting point (usually a transition state) as it would behave
if the nuclei could be treated exactly as classical particles. One
should also keep in mind that the Hamiltonian used is constructed within
the framework of the Born-Oppenheimer approximation, which may turn
out not to be a good approximation at given points during the
reaction. Furthermore, the calculation describes the way molecules
with a predefined orientation and momentum will react. Thus
the trajectory obtained is only one of a large number of
possible trajectories depending on the initial state of the molecule.

The necessary input in order to do a dynamical walk of for instance
protonated formaldehyde\index{formaldehyde} would look like:

\begin{verbatim}
**DALTON INPUT
.WALK
.MAX IT
 200
*WALK
.DYNAMI
.FRAGME
 5
 1 1 1 2 2
.MOMENT
 1
1 -.00001
.MODE
 1
**WAVE FUNCTIONS
.HF
**END OF DALTON INPUT
\end{verbatim}

The walk is specified to
be a dynamic walk  through the keyword \Key{DYNAMI}. The
starting trust radius will in dynamical calculations be changed to a
new default value of~0.005.

The keyword \Key{FRAGME} dictates which atoms belong to which
molecular fragment\index{molecular fragments}. In this particular
case, we assume that two
protons leave the protonated formaldehyde\index{formaldehyde} as a
hydrogen\index{hydrogen} molecule, and
that the leaving hydrogen atoms are the last atoms of the \molinp\ input
file. This partitioning is mainly needed in order to get
proper values for the relative translational energy between the two
fragments, as well as for deciding how much of the energy has been
distributed into internal degrees of freedom.

The default isotopic substitution is that the most abundant
isotopes are to be used in the calculation. Isotopic
substitution is important as the masses of the nuclei enters when
Newton's equations of motion are integrated. The specification of the
isotopic constitution of the molecule is given in the
\molinp\ file, as described in Chapter~\ref{ch:molinp}.

In this calculation we start the calculation at a transition
state\index{transition state},
and in order to get the reaction started we need to give the molecule
a slight push. This is achieved by the keyword \Key{MOMENT}. In the next
line the user then specifies the number of modes in which there is
 an initial momentum, followed by lines containing pairs of numbers,
of which the first specifies the mode, and the second the momentum in
this mode. There must be as many pairs of modes and momenta as
specified in the line after the \Key{MOMENT} keyword. It is
impossible to predict in advance which way the reaction will proceed,
and the calculation should be checked after a few iterations, in order
to ensure that it proceeds in the right direction. If not, the
calculation should be started from the transition state again with a
different sign on the initial momentum. The
\texttt{DALTON.TRJ}\index{DALTON.TRJ} must
also be removed as discussed below.

It is in principle possible to start a calculation from any point on a
molecular potential energy surface, and in cases where these starting points
do not correspond to a stationary point, \Key{MOMENT} may be
skipped, as there exist a downward slope (in other words, an attractive
force) driving the molecule(s) in a specific direction. One may of
course also start the molecule with a given initial momentum in
different energy modes.

During the dynamical calculation, care has also to be taken in order
to ensure that the steps taken are not too long. If this occurs, the
initial trust radius and/or the trust radius increment should be
reduced by the keyword \Key{TRUST}. In the \dalton\ output one will
find ``Accumulated kinetic energy since start'', and this property
will be calculated in two ways: From conservation of the total energy,
and from integrated momenta. If the difference between these numbers
is larger than approximately 1\% , the calculation should be stopped and the
starting trust radius be decreased and the calculation restarted from
the starting point again after removal of the \texttt{DALTON.TRJ}-file.

The calculation of dynamical walks may take from about 70 to 1200
iterations (as a general rule) and one must therefore adjust the
maximum number of iterations allowed. This is done by  the
\Key{MAX IT} keyword. In the present example the maximum number of
iterations have been reset to 200. If the calculation
cannot be closely monitored, it is recommended not to set the maximum
number of iterations\index{geometry iteration} too high, and rather
restart the calculation if
this turns out to be necessary. This can be accomplished by specifying
the iteration at which the calculation will restarted by the keyword
\Key{ITERAT} in the \Sec{*DALTON} input module.

The calculation should be stopped (at least for ordinary hydrogen
elimination reactions) when the ``Relative velocity'' starts to
decrease, as this indicates that the molecules are so far apart that
basis set superposition errors\index{BSSE}\index{basis set!superposition error}
become apparent. We will below return to how
one then calculates translational energy release of the reaction.

During the whole calculation, a file
\texttt{DALTON.TRJ}\index{DALTON.TRJ} is updated. This
file contains information from the entire dynamic walk. If a
walk is restarted from a given point, the new information will be
appended to the old \texttt{DALTON.TRJ}-file. Note that this also implies
that if you need to restart the calculation from the beginning
(because the reaction went the wrong way or because of a too large trust
radius), the
\texttt{DALTON.TRJ}-file {\em must} be removed. Thus, it may often be
advisable to take a backup of this file in certain parts of the
calculation. As this file contains all the information about the
dynamical walk, this file can be used to generate a video-sequence of
the molecular reaction along this specific trajectory with the correct
time-scaling~\cite{krtheujms393}.

\subsection{Calculating relative translational energy release}

It is often of interest to calculate the relative translational energy
release\index{relative translation energy} in a given reaction, as this can be compared to experimental
values determined from {\it e.g.\/} mass
spectrometry~\cite{theuhjajcpl173}\index{mass spectrometry}. Although
this quantity
is printed in the output from \dalton\ in the entire dynamical walk, the
relative translational energy release should be calculated, due to
basis set superposition errors\index{BSSE}\index{basis set!superposition error}
and vibrational\index{vibrational excitation} and rotational
excitation\index{rotational excitation} in the departing molecular
fragments, in the way described here.

The geometry of the last iterations for which relative translational
energy release is known, is used as a starting point for minimizing
the two molecular fragments as described in
Section~\ref{sec:minimization}. As a check of this minimization one
might also minimize the two molecular fragments separately, and check
this total energy against the energy obtained when minimizing the
molecular supersystem. The energies should be almost identical, but
small differences due to basis set superposition errors may be
noticeable.

The barrier height can then be calculated by subtracting the energy of
the molecule at the transition state and the energy for the separated
molecular fragments, or an experimentally determined barrier height
may be used. The relative translational energy release may then be
obtained  by dividing the translational energy release
from the last \dalton\ iteration by the barrier height. This number will
not be identical to the number printed in the \dalton -output, because
of the different vibrational and rotational state of
the molecule in the final iteration point as compared to the minimized
structure.

\section{Geometry optimization using non-variational wave
functions}\label{sec:nonvargeom}

\dalton\ does not have any support for the analytical calculation of molecular
gradients and Hessians for the non-variational wave functions 
CI and NEVPT2\index{CI}\index{Configuration Interaction}\index{NEVPT2}. However, in order to
exploit the facilities of the first-order
geometry optimization routines in {\dalton}, a numerical
gradient\index{numerical gradient} based
on energies will be calculated if a geometry optimization is invoked
for a non-variational wave function. As a simple example, to optimize
the MP2 geometry of a molecule using numerical gradients\footnote{Note
that MP2, CCSD, and CCSD(T) analytical gradients are available through the {\cc}
module}, the only input needed is

\begin{verbatim}
**DALTON INPUT
.OPTIMIZE
**WAVE FUNCTIONS
.HF
.MP2
**END OF DALTON INPUT
\end{verbatim}

The size of the displacements used during the evaluation of the
numerical gradient can be controlled through the keyword
\Key{DISPLA} in the \Sec{OPTIMI} input module. Default value is
$1.0\cdot 10^{-3}$ a.u. By default, the threshold for convergence
of the geometry will be changed because of estimated inaccuracies
in the numerical molecular gradients. However, if the threshold for
convergence is altered manually, this user supplied threshold for
convergence will be used also in geometry optimizations using
numerical molecular gradients\index{molecular gradient!numerical}. Note that due to
the possibility of larger numerical errors in the molecular gradient, too
tight convergence criteria for an optimized geometry may make it
difficult for the program to obtain a converged geometry.

%\chapter{Molecular vibrations and rotations}\label{ch:vibrot}

In this chapter we discuss properties related  to the
vibrational and rotational motions of a molecule. This includes
vibrational frequencies and the associated infrared
(IR)\index{IR intensity} and Raman intensities\index{Raman intensity}.

\section{Vibrational frequencies}\label{sec:vibfreq}

The calculation of vibrational frequencies\index{vibrational
frequency} and rotational constants\index{rotation constant} are
controlled by the keyword \Key{VIBANA}. Thus, in order to
calculate the vibrational frequencies and the rotational constants
of a molecule all that is is needed is the input:

\begin{verbatim}
**DALTON INPUT
.RUN PROPERTIES
**WAVE FUNCTIONS
.HF
**PROPERTIES
.VIBANA
**END OF DALTON INPUT
\end{verbatim}

This keyword will, in addition to calculating the molecular
frequencies, also calculate the zero-point vibrational 
energy corrections\index{zero-point vibrational energy} and vibrational
and rotational partition functions\index{partition functions}
at selected temperatures.

\siraba\ evaluates the molecular Hessian\index{Hessian} in Cartesian
coordinates\index{Cartesian coordinates}, and
the vibrational frequencies of any isotopically substituted  species
may therefore easily be obtained on the basis of the full
Hessian. Thus, if we would like to calculate the vibrational
frequencies of isotopically substituted molecules\index{isotopic
constitution}, this may be obtained through an input like:

\begin{verbatim}
**DALTON INPUT
.RUN PROPERTIES
**WAVE FUNCTIONS
.HF
**PROPERITES
.VIBANA
*VIBANA
.ISOTOP
   2   5
 1 2 1 1 1
 2 1 1 1 1
**END OF DALTON INPUT
\end{verbatim}

The keyword \Key{ISOTOP} in the \Sec{VIBANA} input module
indicates that more than only the isotopic species containing the most
abundant isotopes are to be calculated, which will always be
calculated. The numbers on the second line denotes the number of
isotopically substituted species that is requested and the number of
atoms in the system. The following lines then
lists the isotopic constitution of each of these species. 1
corresponds to the most abundant isotope, 2 corresponds to the second
most abundant isotope and so on. The isotopic substitution have to be
given for all atoms in the molecule (not only the symmetry
independent), and the above input could for instance correspond to a
methane\index{methane} molecule, with the isotopic species $CH_3D$
and $^{13}CH_4$.

As the isotopic substitution of all atoms in the molecule has to be
specified, let us mention the way symmetry-dependent atoms will be
generated. The atoms will be grouped in symmetry-dependent atom
blocks. The specified symmetry-independent atom will be the first of
this block, and the symmetry-dependent atoms will be generated
according to the order of the symmetry elements. Thus, assuming
D$_{2h}$ symmetry with symmetry generating elements \verb|X  Y  Z|,
the atoms generated will come in the order \verb|X|, \verb|Y|,
\verb|XY|, \verb|Z|, \verb|XZ|, \verb|YZ|, and \verb|XYZ|.

\section{Infrared (IR) intensities}\label{sec:irint}

\begin{center}
\fbox{
\parbox[h][\height][l]{12cm}{
\small
\noindent
{\bf Reference literature:}
\begin{list}{}{}
\item R.D.Amos. \newblock {\em Chem.Phys.Lett.}, {\bf
108},\hspace{0.25em}185, (1984).
\item T.U.Helgaker, H.J.Aa.Jensen, and P.J{\o}rgensen. \newblock {\em
J.Chem.Phys.}, {\bf 84},\hspace{0.25em}6280, (1986).
\end{list}
}}
\end{center}

\index{IR intensity} The evaluation of infrared intensities
requires the calculation of the
dipole gradients\index{dipole gradient}\index{APT}\index{atomic polar
tensor} (also known as Atomic Polar Tensors (APTs)). Thus, by
combining the calculation of vibrational frequencies with the
calculation of dipole gradients, IR intensities will be obtained. Such
an input may look like:

\begin{verbatim}
**DALTON INPUT
.RUN PROPERTIES
**WAVE FUNCTIONS
.HF
**PROPERTIES
.VIBANA
.DIPGRA
*VIBANA
.ISOTOP
   2   5
 1 2 1 1 1
 2 1 1 1 1
**END OF DALTON INPUT
\end{verbatim}

\noindent The keyword \Key{DIPGRA} invokes the calculation of the dipole
gradients.

\section{Dipole-gradient based population analysis}

\begin{center}
\fbox{
\parbox[h][\height][l]{12cm}{
\small
\noindent
{\bf Reference literature:}
\begin{list}{}{}
\item J.Cioslowski. \newblock {\em J.Am.Chem.Soc.}, {\bf
111},\hspace{0.25em}8333, (1989).
\item T.U.Helgaker, H.J.Aa.Jensen, and P.J{\o}rgensen. \newblock {\em
J.Chem.Phys.}, {\bf 84},\hspace{0.25em}6280, (1986).
\end{list}
}}
\end{center}

\index{population analysis} As dipole gradients\index{dipole
gradient}\index{APT}\index{atomic polar tensor} are readily available in the
\siraba\ program, the
population analysis basis on the Atomic Polar Tensor as suggested by
Cioslowski~\cite{jcjacs111,poakrkvmthjpca102} can be obtained from an input like

\begin{verbatim}
**DALTON INPUT
.RUN PROPERTIES
**WAVE FUNCTIONS
.HF
**PROPERTIES
.POPANA
*END OF DALTON
\end{verbatim}

This population analysis is of course significantly more expensive
than the ordinary Mulliken population analysis\index{population analysis!Mulliken}\index{Mulliken population analysis} obtainable directly
from the molecular wave functions through an input like

\begin{verbatim}
**DALTON INPUT
.RUN WAVE FUNCTIONS
**WAVE FUNCTIONS
.HF
*POPULATION ANALYSIS
.MULLIKEN
*END OF DALTON
\end{verbatim}


\section{Raman intensities}\label{sec:ramanint}

\begin{center}
\fbox{
\parbox[h][\height][l]{12cm}{
\small
\noindent
{\bf Reference literature:}
\begin{list}{}{}
\item T.Helgaker, K.Ruud, K.L.Bak, P.J{\o}rgensen, and J.Olsen. \newblock {\em
Faraday Discuss.}, {\bf 99},\hspace{0.25em}165, (1994).
\end{list}
}}
\end{center}

\index{Raman intensity} Calculating Raman intensities is by no means
a trivial task, and
because of the computational cost of such calculations, and there are
therefore few theoretical investigations of basis set requirements and
correlation effects on calculated Raman intensities. The Raman
intensities calculated are the ones obtained within the Placzek
approximation~\cite{placzek}\index{Placzek approximation},
and the implementation is described in Ref.~\cite{thkrklbpjjofd99}.

The Raman intensity is the differentiated frequency-dependent
polarizability\index{polarizability} with respect to nuclear displacements.
As it is a third derivative depending on the nuclear positions through
the basis set, numerical differentiation of
the polarizability with respect to nuclear coordinates is
necessary.

The input looks very similar to the input needed for the calculation
of Raman optical activity\index{ROA}\index{Raman optical activity}  described
in Section~\ref{sec:vroa}

\begin{verbatim}
**DALTON INPUT
.WALK
*WALK
.NUMERI
**WAVE FUNCTIONS
.HF
*SCF INPUT
.THRESH
1.0D-8
**START
.RAMAN
*ABALNR
.THRESH
1.0D-7
.FREQUE
     2
0.0 0.09321471
**PROPERTIES
.RAMAN
*ABALNR
.THRESH
1.0D-7
.FREQUE
     2
0.0 0.09321471
**FINAL
.RAMAN
.VIBANA
*RESPONSE
.THRESH
1.0D-6
*ABALNR
.THRESH
1.0D-7
.FREQUE
     2
0.0 0.09321471
*VIBANA
.PRINT
 1
.ISOTOP
   1   5
 1 1 1 2 3
**END OF DALTON INPUT
\end{verbatim}

The keyword \Key{ABALNR} in the general input module indicates that
a frequency dependent linear response\index{linear response}\index{response!linear}
calculation is to be done, in
this case the calculation of the frequency-dependent
polarizability\index{polarizability} as
specified by the \Key{ALFA} keyword in the \Sec{ABALNR} input
module. The keyword \Key{RAMAN} indicates that we are only
interested in the Raman intensities\index{Raman intensity} and
depolarization ratios\index{depolarization ratio}. Note
that these parameters are also obtainable by using the keyword
\Key{VROA}. In this calculation we calculate the Raman intensities for two
frequencies, the static case and a frequency of the incident light
corresponding to a laser of wavelength 488.8 nm. 

Due to the numerical differentiation\index{numerical differentiation}
that is done, the threshold for
the iterative solution of the response equations are by default
10$^{-7}$, in order to get Raman intensities that are numerically
stable to one decimal digit.

In the \Sec{WALK} input module we have specified
that the walk is a numerical differentiation. This will automatically
turn off the calculation of the geometric Hessian\index{Hessian},
putting limitations
on what kind of properties that may be calculated at the same time as
Raman intensities. Because the Hessian is not calculated,
there will not be any prediction of the energy at the new
point.

It should also be noted that  in a numerical
differentiation\index{numerical differentiation}, the
program will
step plus and minus one displacement unit along each Cartesian coordinate
of all nuclei, as well as calculating the property at the reference
geometry. Thus, for a molecule with $N$ atoms the properties will need
to be calculated in a total of 2*3*$N$ + 1 points, which for a 
molecule with five atoms will amount to 31 points. The default maximum number of
steps in \siraba\ is 20. However, in numerical differentiation
calculations, the number of iterations will always be reset (if there
are more than 20 steps that need to be taken) to 6$N$+1, as it is
assumed that the user always wants the calculation to complete
correctly. The maximum number of allowed iterations\index{geometry
iteration} can be manually set by adding the keyword
\Key{MAX IT} in the \Sec{*DALTON} input module.

The default step length in the numerical
differentiation\index{numerical differentiation} is $1.0\cdot 10^{-4}$
a.u., and this step length may be adjusted by the keyword
\Key{DISPLA} in the \Sec{WALK} input module. The steps are taken
in the Cartesian directions and
not along normal modes. This enables us to study the Raman intensities
of a large number of isotopically substituted molecules at once. This
is done in the \Sec{*FINAL} input section, where we
have requested one isotopically substituted species in addition to the
isotopic species containing the most abundant isotope of each element.

%\chapter{Electric properties}\label{ch:electric}

This chapter describes the calculation of the different electric
properties which have been implemented in the \dalton\ program.
These include the dipole moment\index{dipole moment}, the quadrupole
moment\index{quadrupole moment}, the nuclear quadrupole\index{nuclear quadrupole coupling}\index{electric field!gradient}\index{EFG}\index{NQCC}
interactions, and the static and frequency dependent
polarizability\index{polarizability}. Note that a number of different electric
properties may be
obtained by use of the {\resp} program if they can be expressed as a
linear, quadratic or cubic response function. For the non-linear
electric properties we refer to the chapter ``Getting the property you
want'' (Chapter~\ref{ch:rspchap}).

\section{Dipole moment}\label{sec:dipmom}

The dipole moment\index{dipole moment} of a  molecule is always
calculated if \Sec{*PROPERTIES} is
requested, and no special input is needed in order to evaluate this property.

\section{Quadrupole moment}\label{sec:quadmom}

The traceless molecular quadrupole moment\index{quadrupole moment}, as
defined by Buckingham
\cite{adbacp12}, is calculated by using the keyword \Key{QUADRU}, and
it can be requested from an input like:

\begin{verbatim}
**DALTON INPUT
.RUN PROPERTIES
**WAVE FUNCTIONS
.HF
**PROPERTIES
.QUADRU
**END OF DALTON INPUT
\end{verbatim}

Note that both the electronic and nuclear contributions are always
printed in the coordinate system chosen, that is, the tensors are not
transformed to the principal axis system nor to the principal inertia
system, as is often done in the literature.

The quadrupole moment is evaluated as an expectation value, and is
thus fast to evaluate. This is noteworthy, because experimentally
determined quadrupole moments obtained through microwave Zeeman experiments
(see e.g.  \cite{whmklwhfjcp48,jsdhszna46}) are derived
quantities and prone to errors,
whereas the calculation of rotational {\em g} factors and magnetizability
anisotropies\index{rotational g tensor}\index{magnetizability} (see Chapter~\ref{ch:magnetic})---obtainable from such
experiments---are difficult to calculate accurately~\cite{krthcpl264}. An input
requesting a large number of the properties obtainable from microwave
Zeeman experiments is (where we also include nuclear quadrupole
coupling constants)\index{nuclear quadrupole coupling}\index{electric field!gradient}\index{EFG}\index{NQCC}:

\begin{verbatim}
**DALTON INPUT
.RUN PROPERTIES
**WAVE FUNCTIONS
.HF
**PROPERTIES
.MAGNET
.MOLGFA
.QUADRU
.NQCC
**END OF DALTON INPUT
\end{verbatim}

Note that the program prints the final molecular rotational {\em g}
tensors\index{rotational g tensor} in the
principal inertia system, whereas this is not the case for the
magnetizabilities and molecular quadrupole moment.

\section{Nuclear quadrupole coupling constants}

This property is the interaction between the nuclear quadrupole moment\index{nuclear quadrupole coupling}\index{electric field!gradient}\index{EFG}\index{NQCC}
of a nucleus with spin greater or equal to 1, and the electric field gradient
generated by the movement of the electron cloud around the nucleus.
Quantum mechanically it is calculated as an expectation value of the electric
field gradient  at the nucleus, and it is obtained by the input:

\begin{verbatim}
**DALTON INPUT
.RUN PROPERTIES
**WAVE FUNCTIONS
.HF
**PROPERTIES
.NQCC
**END OF DALTON INPUT
\end{verbatim}

It is noteworthy that due to it's dependence on the
electronic environment close to the nucleus of interest, it puts strict
demands on the basis set, similar to that needed for spin-spin
coupling\index{spin-spin coupling}
constants (Sec.~\ref{sec:spinspin}). Electron correlation may also be of
importance, see for instance Ref.~\cite{mjssocpjthkrcpl243}.

\section{Static and frequency dependent
polarizabilities}\label{sec:polari}

Frequency-dependent polarizabilities\index{polarizability} is
calculated from a set of linear
response\index{linear response}\index{response!linear} functions as
described in Ref.~\cite{jopjjcp82}. In {\aba} the
calculation of frequency-dependent linear response functions is
requested through the keyword \Key{ALPHA} in the general input
module to the property section. An input file requesting the calculation of the frequency
dependent polarizability of a molecule may then be calculated using
the following input:

\begin{verbatim}
**DALTON INPUT
.RUN PROPERTIES
**END OF DALTON INPUT
**WAVE FUNCTIONS
.HF
**PROPERTIES
.ALPHA
*ABALNR
.FREQUE
    2
0.0 0.09321471
**END OF DALTON INPUT
\end{verbatim}

For a Second Order Polarization Propagator Approximation
(SOPPA)\index{SOPPA}\index{polarization propagator}
\cite{esnpjjodjcp73,jopjdycpr2,mjpekdtehjajjojcp,ekdspasjpca102}
calculation of frequency-dependent polarizabilities the additional
keyword \Key{SOPPA} has to be specified in the \Sec{*PROPERTIES} input
module and an MP2 calculation has to be requested by the keyword
\Key{MP2} in the \Sec{*WAVE FUNCTIONS} input module. Similarly, for a
SOPPA(CC2) \index{SOPPA(CC2)} \cite{spas097}
 or SOPPA(CCSD)\index{SOPPA(CCSD)} \cite{soppaccsd,ekdspasjpca102}
calculation of frequency dependent polarizabilities the additional
keyword \Key{SOPPA(CCSD)} has to be specified in the \Sec{*PROPERTIES}
input module and an CC2 or CCSD calculation has to be requested by the
keyword \Key{CC} in the \Sec{*WAVE FUNCTIONS} input module with the
option \Key{SOPPA2} or \Key{SOPPA(CCSD)} in the \Sec{CC INPUT} section.

The \Sec{ABALNR} input section controls the calculation of the
frequency-dependent linear response
function\index{linear response}\index{response!linear}.
We must here specify at which frequencies the polarizability is
to be calculated. This is done with the keyword \Key{FREQUE}, and in
this run the polarizability is to be evaluated at zero frequency
(corresponding to the static polarizability) and at a frequency (in
atomic units) corresponding to a incident laser beam of wavelength
488.8 nm.

There is also another way of calculating the static
 polarizability, and this is by using the
keyword \Key{POLARI} in the \Sec{*PROPERTIES} input modules. Thus, if we only
want to evaluate the static polarizability of a molecule, this may be
achieved by the following input:

\begin{verbatim}
**DALTON INPUT
.RUN PROPERTIES
**WAVE FUNCTIONS
.HF
**PROPERTIES
.POLARI
**END OF DALTON INPUT
\end{verbatim}

Furthermore, the general RESPONSE program will also calculate the
frequency-dependent polarizability\index{polarizability}\index{linear response}\index{response!linear}
as minus the linear response functions through the input


\begin{verbatim}
**DALTON INPUT
.RUN RESPONSE
**WAVE FUNCTIONS
.HF
**RESPONSE
*LINEAR
.DIPLEN
**END OF DALTON INPUT
\end{verbatim}
For further details about input for the response program, we refer
to Chapter~\ref{ch:rspchap}.

%\chapter{Calculation of magnetic properties}\label{ch:magnetic}

This chapter describes the calculation of properties depending on
magnetic fields, both as created by an external magnetic field as well
as the magnetic field created by a nuclear magnetic moment.
This includes the two
contributions to the ordinary spin-Hamiltonian used in NMR, nuclear
shieldings\index{nuclear shielding} and indirect nuclear spin--spin
couplings\index{spin-spin coupling} constants. We
also describe the calculation of the magnetic analogue of the
polarizability, the molecular
magnetizability\index{magnetizability}. This property
is of importance in NMR experiments where the reference substance is placed
in another tube than the sample. We also shortly
describe two properties very closely related to the magnetizability
and nuclear shieldings, respectively, the rotational {\em g} factor and the
nuclear spin--rotation constants\index{rotational g
tensor}\index{spin-rotation constant}.

Three properties that in principle depend on the %nuclear
magnetic moments are
not treated here, namely the properties associated with optical
activity or, more precisely, with circular dichroism. These
properties are Vibrational Circular Dichroism
(VCD)\index{VCD}\index{vibrational circular dichroism}, Raman
Optical activity (ROA)\index{ROA}\index{Raman optical activity} and
Electronic Circular Dichroism (ECD)\index{ECD}\index{electronic
circular dichroism} and these properties will be treated in
Chapter~\ref{ch:optchap}. 
Another (magneto-)optical property, the ${\cal{B}}$ term of
Magnetic Circular Dichroism (MCD) will be described in
Chapter~\ref{ch:rspchap}.

Gauge-origin\index{gauge origin} independent nuclear shieldings,
magnetizabilities and rotational {\em g} tensors are
obtained through the use of London atomic orbitals, and the theory is
presented in several references
\cite{kwjfhppjacs112,krthrkpjklbhjajjcp100,krthklbpjhjajjcp99,krthklbpjjocp195}.
Gauge-origin\index{gauge origin} independent nuclear shieldings and
magnetizabilities can also be obtained by using the 
the Continuous Transformation of the Origin of the Current Density method (CTOCD)
\index{Continuous Transformation of the Origin of the Current Density}
\index{CTOCD-DZ} approach~\cite{paololazz1,paololazz2,ctocd}. In the present
version of DALTON the CTOCD-DZ method is implemented and can be invoked by
the keyword \Key{CTOCD} in the \Sec{*PROPERTIES} input module.
More detailed information on CTOCD-DZ calculations can be found in section
\ref{sec:ctocdgeneral}. 


The indirect spin--spin couplings are calculated by using the triplet linear
response function\index{triplet response}, as described in
Ref.~\cite{ovhapjhjajsbpthjcp96}.
These are in principle equally simple to calculate with \siraba\ as
nuclear shieldings and magnetizabilities. However, there are 10
contributions to the spin--spin coupling constant from {\em each}
nucleus.\footnote{Dalton 2.0 now only calculates the symmetry-distinct
  contributions to the spin--dipole operator, that is---six instead of
  nine elements are calculated for this operator.} Furthermore, the
spin--spin coupling constants put severe 
requirements on the quality of the basis set as well as a proper
treatment of correlation, making the evaluation of spin--spin coupling
constants a time consuming task. Some notes about how this time
can be reduced is given below.

Second Order Polarization Propagator Approximation (SOPPA) \index{SOPPA} 
\index{polarization propagator} calculations 
\cite{esnpjjodjcp73,jopjdycpr2,mjpekdtehjajjojcp,spascpl260,tejospastcan100}
or SOPPA(CCSD) \index{SOPPA(CCSD)} calculations
\cite{soppaccsd,spascpl260,tejospastcan100}  of the indirect spin--spin
couplings, nuclear shieldings, magnetizabilities, rotational {\em g} tensors
and the nuclear spin--rotation constants can be invoked by the additional
keywords \Key{SOPPA} or \Key{SOPPA(CCSD)} in the \Sec{*PROPERTIES} input
module. This requires for SOPPA that the MP2 energy was calculated by
specifying the keyword \Key{MP2} 
in the \Sec{*WAVE FUNCTIONS} input module, whereas for a SOPPA(CCSD)
calculation the CCSD amplitudes have to be generated by specifying the
keyword \Key{CC} in the \Sec{*WAVE FUNCTIONS} input module and
\Key{SOPPA(CCSD)} in the \Sec{CC INPUT} section. The use of London orbitals is
automatically disabled in SOPPA calculations of the nuclear shieldings,
magnetizabilities and rotational {\em g} tensors. 

\section{Magnetizabilities}\label{sec:magnetizability}

\begin{center}
\fbox{
\parbox[h][\height][l]{12cm}{
\small
\noindent
{\bf Reference literature:}
\begin{list}{}{}
\item SCF magnetizabilities: K.Ruud, T.Helgaker, K.L.Bak, P.J\o rgensen and H.J.Aa.Jensen. \newblock {\em J.Chem.Phys.}, {\bf 99},\hspace{0.25em}3847, (1993).
\item MCSCF magnetizabilities: K.Ruud, T.Helgaker, K.L.Bak, P.J\o
rgensen, and J.Olsen. \newblock {\em Chem.Phys.}, {\bf
195},\hspace{0.25em}157, (1995).
\item Solvent effects: K.V.Mikkelsen,
P.J{\o}rgensen, K.Ruud, and T.Helgaker. \newblock {\em J.Chem.Phys.}, {\bf
107},\hspace{0.25em}1170, (1997).
\item CTOCD-DZ magnetizabilities: P.Lazzeretti, M.Malagoli and R.Zanasi.
 \newblock {\em Chem.Phys.Lett.}, {\bf 220},\hspace{0.25em}299, (1994)
\end{list}
}}
\end{center}

The calculation of molecular magnetizabilities\index{magnetizability}
is invoked by the
keyword \Key{MAGNET} in the \Sec{*PROPERTIES} input module. Thus
a complete input file for the calculation of molecular
magnetizabilities will look like:

\begin{verbatim}
**DALTON INPUT
.RUN PROPERTIES
**WAVE FUNCTIONS
.HF
**PROPERTIES
.MAGNET
**END OF DALTON INPUT
\end{verbatim}

This will invoke the calculation of molecular magnetizabilities using
London Atomic Orbitals\index{London orbitals} to ensure fast basis set
convergence and
gauge-origin\index{gauge origin} independent results. The natural
connection\index{natural connection}
\cite{joklbkrthpjtca90} is used in order to get numerically accurate
results. By default the center of mass\index{center of mass} is chosen
as gauge origin.

For a SOPPA\index{SOPPA} or SOPPA(CCSD)\index{SOPPA(CCSD)} calculation of
molecular magnetizabilities, the additional keywords \Key{SOPPA} or
\Key{SOPPA(CCSD)} have to be specified in the \Sec{*PROPERTIES} input
module. For SOPPA an MP2 calculation has to be requested by the keyword
\Key{MP2} in the \Sec{*WAVE FUNCTIONS} input module, whereas for SOPPA(CCSD) a
CCSD calculation has to be requested by the keyword \Key{CC} in the \Sec{*WAVE
  FUNCTIONS} input module with the \Sec{CC INPUT} option
\Key{SOPPA(CCSD)}. This will also automatically disable the use of London
orbitals. 

For a CTOCD-DZ\index{CTOCD-DZ} calculation of molecular magnetizabilities, the 
additional keyword \Key{CTOCD} has to be specified in the \Sec{*PROPERTIES} 
input module. This will automatically disable the use of London orbitals.
\Key{SOPPA} / \Key{SOPPA(CCSD)} and \Key{CTOCD} could be used together to get
SOPPA / SOPPA(CCSD) molecular magnetizabilities using the CTOCD-DZ
formalism. Information about suitable basis sets for CTOCD-DZ calculations can
be found in the section \ref{sec:ctocdgeneral}.  

The augmented cc-pVXZ basis sets of Dunning and
coworkers~\cite{thdjcp90,rakthdrjhjcp96,dewthdjcp98,dewthdjcp100} have
been shown to give  to give excellent results for
magnetizabilities~\cite{krthklbpjhjajjcp99,krthpjklbcpl223,krhsthklbpjjacs116},
and these basis sets are obtainable from the basis set library.

Notice that a general print level of 2 or higher is needed in order to
get the individual contributions (relaxation, one- and
two-electron expectation values and so on) to the total magnetizability.

If more close control of the different parts of the calculation of the
magnetizability is wanted, we refer the reader to the section
describing the options available. The modules that controls the
calculation of molecular magnetizabilities are:

\begin{list}{}{\itemsep 0.10cm \parsep 0.0cm}
\item[\Sec{EXPECT}] Controls the calculation of one-electron
expectation values contributing to the diamagnetic magnetizability.
\item[\Sec{GETSGY}] Controls the set up of the right-hand sides
(gradient terms) as well as the calculation of two-electron
expectation values and reorthonormalization terms.
\item[\Sec{LINRES}] Controls the solution of the magnetic response
equations
\item[\Sec{RELAX}] Controls the multiplication of solution and right-hand
side vectors into relaxation contributions
\end{list}

\section{Nuclear shielding constants}\label{sec:shieldings}

\begin{center}
\fbox{
\parbox[h][\height][l]{12cm}{
\small
\noindent
{\bf Reference literature:}
\begin{list}{}{}
\item K.Wolinski, J.F.Hinton, and P.Pulay. \newblock {\em
J.Am.Chem.Soc.}, {\bf 112},\hspace{0.25em}8251, (1990)
\item K.Ruud, T.Helgaker, R.Kobayashi, P.J\o rgensen, K.L.Bak, and H.J.Aa.Jensen. 
\newblock {\em J.Chem.Phys.}, {\bf 100},\hspace{0.25em}8178, (1994).
\item Solvent effects: K.V.Mikkelsen,
P.J{\o}rgensen, K.Ruud, and T.Helgaker. \newblock {\em J.Chem.Phys.}, {\bf
107},\hspace{0.25em}1170, (1997).
\item CTOCD-DZ nuclear shielding: A.Ligabue, S.P.A.Sauer, P.Lazzeretti.
\newblock {\em J.Chem.Phys.}, {\bf 118},\hspace{0.25em}6830, (2003).
\end{list}
}}
\end{center}


The calculation of nuclear shieldings\index{nuclear shielding} are
invoked by the
keyword \Key{SHIELD} in the \Sec{*PROPERTIES} input module. Thus
a complete input file for the calculation of nuclear shieldings will
be:

\begin{verbatim}
**DALTON INPUT
.RUN PROPERTIES
**WAVE FUNCTIONS
.HF
**PROPERTIES
.SHIELD
**END OF DALTON INPUT
\end{verbatim}

This will invoke the calculation of nuclear shieldings using
London Atomic Orbitals\index{London orbitals} to ensure fast basis set
convergence and gauge
origin\index{gauge origin} independent results. The natural connection
\cite{joklbkrthpjtca90}\index{natural connection}
is used in order to get
numerically accurate results. By default the center of mass is chosen as gauge
origin\index{center of mass}.

For a SOPPA\index{SOPPA} or SOPPA(CCSD)\index{SOPPA(CCSD)} calculation of
nuclear shieldings the additional keywords \Key{SOPPA} or
\Key{SOPPA(CCSD)} have to be specified in the \Sec{*PROPERTIES} input
module. For SOPPA an MP2 calculation has to be requested by the keyword
\Key{MP2} in the \Sec{*WAVE FUNCTIONS} input module, whereas for SOPPA(CCSD) a
CCSD calculation has to be requested by the keyword \Key{CC} in the \Sec{*WAVE
  FUNCTIONS} input module with the \Sec{CC INPUT} option
\Key{SOPPA(CCSD)}. This will also automatically disable the use of London
orbitals. 

For a CTOCD-DZ\index{CTOCD-DZ} calculation of nuclear shieldings the 
additional keyword \Key{CTOCD} has to be specified in the \Sec{*PROPERTIES} 
input module. This will automatically disable the use of London orbitals.
\Key{SOPPA} / \Key{SOPPA} and \Key{CTOCD} could be used together to get gauge
origin independent SOPPA / SOPPA(CCSD) nuclear shieldings using the CTOCD-DZ
formalism.  Information about suitable basis set for CTOCD-DZ calculations can
be found in the section \ref{sec:ctocdgeneral}. 

A basis set well suited for the calculation of nuclear shieldings (and
indirect nuclear spin--spin coupling constants) is the TZ basis set of
Ahlrichs and coworkers~\cite{ashhrajcp97,aschrajcp100} with two
polarization functions~\cite{thmjkrcr99}. This basis set is available
from the basis set library as TZ2P.

Notice that a general print level of 2 or higher is needed in order to
get the individual contributions (relaxation, one- and
two-electron expectation values and so on) to the total nuclear shieldings.

If more close control of the different parts of the calculation of the
nuclear shieldings is wanted we refer the reader to the section
describing the options available. For the calculation of nuclear
shieldings, these are the same as listed above for magnetizability
calculations.

\section{Rotational {\em g} tensor}\label{sec:gfac}

\begin{center}
\fbox{
\parbox[h][\height][l]{12cm}{
\small
\noindent
{\bf Reference literature:}
\begin{list}{}{}
\item J.Gauss, K.Ruud, and T.Helgaker. \newblock {\em J.Chem.Phys.},
{\bf 105},\hspace{0.25em}2804, (1996).
\end{list}
}}
\end{center}

The calculation of the rotational  {\em g} tensor\index{rotational g
tensor} is invoked through the
keyword \Key{MOLGFA} in the \Sec{*PROPERTIES} input module. A complete
input file for the calculation of the molecular g tensor is thus:

\begin{verbatim}
**DALTON INPUT
.RUN PROPERTIES
**WAVE FUNCTIONS
.HF
**PROPERTIES
.MOLGFA
**END OF DALTON INPUT
\end{verbatim}

The molecular g tensor consists of two terms: a nuclear term and a
term which may be interpreted as one definition of a paramagnetic part
of the magnetizability tensor as
described in Ref.~\cite{jgkrthjcp105}.
By default the center of mass\index{center of mass} is chosen as
rotational origin, as this corresponds to the point about which the
molecule rotates. The use of
rotational London atomic\index{London orbitals} 
orbitals can be turned off through
the keyword \verb|.NOLOND|.

For a SOPPA\index{SOPPA} or SOPPA(CCSD)\index{SOPPA(CCSD)} calculation of
rotational  {\em g} tensors the additional keywords \Key{SOPPA} or
\Key{SOPPA(CCSD)} have to be specified in the \Sec{*PROPERTIES} input
module. For SOPPA an MP2 calculation has to be requested by the keyword
\Key{MP2} in the \Sec{*WAVE FUNCTIONS} input module, whereas for SOPPA(CCSD) a
CCSD calculation has to be requested by the keyword \Key{CC} in the \Sec{*WAVE
  FUNCTIONS} input module with the \Sec{CC INPUT} option
\Key{SOPPA(CCSD)}. This will also disable automatically the use of London
orbitals. 
 
The basis set requirements for the rotational {\em g} tensors are more or
less equivalent with the ones for the molecular magnetizability,
that is, the augmented cc-pVDZ of Dunning and
Woon~\cite{thdjcp90,dewthdjcp98}, available from the basis set library
as \verb|aug-cc-pVDZ|.

If more close control of the different parts of the calculation of the
rotational {\em g}  tensor is wanted we refer the reader to the section
describing the options available. For the calculation of the molecular
g tensor, these are the same as listed above for magnetizability
calculations.

\section{Nuclear spin--rotation constants}\label{sec:spinrotasjon}

\begin{center}
\fbox{
\parbox[h][\height][l]{12cm}{
\small
\noindent
{\bf Reference literature:}
\begin{list}{}{}
\item R.Ditchfield. \newblock {\em J.Chem.Phys.}, {\bf
56},\hspace{0.25em}5688 (1972)
\item J.Gauss, K.Ruud, and T.Helgaker. \newblock {\em J.Chem.Phys.},
{\bf 105},\hspace{0.25em}2804, (1996).
\end{list}
}}
\end{center}


In \siraba\ the nuclear spin--rotation\index{spin-rotation constant}
constants are calculated using
rotational orbitals, giving an improved
basis set convergence~\cite{jgkrthjcp105}. We use the
expression for the spin--rotation constant where the paramagnetic term
is evaluated around the center of mass, and thus will only solve three
response equations at the most.

An input requesting the calculation of spin--rotation constants will
look like

\begin{verbatim}
**DALTON INPUT
.RUN PROPERTIES
**WAVE FUNCTIONS
.HF
**PROPERTIES
.SPIN-R
.ISOTOP
    3
    2   1    1
**END OF DALTON INPUT
\end{verbatim}

As the nuclear spin--rotation\index{spin-rotation constant} constants
depend on the isotopic substitution of
the molecule, both through the nuclear magnetic moments and through the
center of mass, the isotopic constitution need to be specified if this
is different  from the most abundant isotopic constitution. Note that
some of the most common isotopes do not have a magnetic moment.

For a SOPPA\index{SOPPA} or SOPPA(CCSD)\index{SOPPA(CCSD)} calculation of
nuclear spin--rotation constants the additional keywords \Key{SOPPA} or
\Key{SOPPA(CCSD)} have to be specified in the \Sec{*PROPERTIES} input
module. For SOPPA an MP2 calculation has to be requested by the keyword
\Key{MP2} in the \Sec{*WAVE FUNCTIONS} input module, whereas for SOPPA(CCSD) a
CCSD calculation has to be requested by the keyword \Key{CC} in the \Sec{*WAVE
  FUNCTIONS} input module with the \Sec{CC INPUT} option
\Key{SOPPA(CCSD)}. This will also automatically disable the use of London
orbitals.

We note that in the current release of \siraba , nuclear spin--rotation
constants can not be calculated employing symmetry-dependent
nuclei. Thus for a molecule like N$_2$, the symmetry plane
perpendicular to the molecular bond will have to be removed during the
calculation.

\section{Indirect nuclear spin--spin coupling
constants}\label{sec:spinspin}

\begin{center}
\fbox{
\parbox[h][\height][l]{12cm}{
\small
\noindent
{\bf Reference literature:}
\begin{list}{}{}
\item O.Vahtras, H.\AA gren, P.J\o rgensen, H.J.Aa.Jensen, S.B.Padkj\ae
r, and T.Helgaker. \newblock {\em J.Chem.Phys.}, {\bf
96},\hspace{0.25em}6120, (1992).
\item Solvent effects: P.-O.\AA strand, K.V.Mikkelsen, P.J{\o}rgensen,
K.Ruud and T.Helgaker.  \newblock {\em J.Chem.Phys.}, {\bf
108},\hspace{0.25em}2528, (1998)
\item SOPPA and SOPPA(CCSD): T.Enevoldsen, J.Oddershede, and S.P.A. Sauer.
\newblock {\em Theor.~Chem.~Acc.}, {\bf 100},\hspace{0.25em}275, (1998)
\end{list}
}}
\end{center}

As mentioned in the introduction of this chapter, the calculation of
indirect nuclear
spin--spin coupling constants is a time consuming task due to the
large number of contributions to the total spin---spin coupling
constant. Still, if all spin---spin couplings\index{spin-spin
coupling} in a molecule are wanted,
with some restrictions mentioned below, the input will look as
follows:

\begin{verbatim}
**DALTON INPUT
.RUN PROPERTIES
**WAVE FUNCTIONS
.HF
**PROPERTIES
.SPIN-SPIN
**END OF DALTON INPUT
\end{verbatim}

This input will calculate the indirect nuclear spin--spin
coupling\index{spin-spin coupling}
constants between isotopes with non-zero magnetic
moments\index{magnetic moment} and a
natural abundance\index{abundance} of more than 1\% . This limit will
automatically
include proton and $^{13}$C spin--spin coupling constants. By default,
all contributions to the coupling constants will be calculated.

Often one is interested in only certain kinds of nuclei. For example,
 one may want to calculate only the proton spin--spin couplings of a molecule.
This can be accomplished in two ways: either by changing the abundance
threshold so that only this single isotope is included (most useful
for proton couplings), or by selecting the particular nuclei of interest.

For a SOPPA\index{SOPPA} or SOPPA(CCSD)\index{SOPPA(CCSD)} calculation of
indirect nuclear spin--spin coupling constants the additional keywords
\Key{SOPPA} or \Key{SOPPA(CCSD)} have to be specified in the \Sec{*PROPERTIES}
input module. For SOPPA an MP2 calculation has to be requested by the keyword
\Key{MP2} in the \Sec{*WAVE FUNCTIONS} input module, whereas for SOPPA(CCSD) a
CCSD calculation has to be requested by the keyword \Key{CC} in the \Sec{*WAVE
  FUNCTIONS} input module with the \Sec{CC INPUT} option \Key{SOPPA(CCSD)}.

All the keywords necessary to control such adjustments to the
calculation is given in the section describing the input for the
\Sec{SPIN-S} submodule. An input
in which we  have reduced the abundance threshold as well as
selected three atoms will look as:

\begin{verbatim}
**DALTON INPUT
.RUN PROPERTIES
**WAVE FUNCTIONS
.HF
**PROPERTIES
.SPIN-S
*SPIN-S
.ABUNDA
 0.10
.SELECT
    3
    2    4    5
**END OF DALTON INPUT
\end{verbatim}

We refer to the section describing the \Sec{SPIN-S} input module for
the complete description of the syntax for these keywords, as well as
the numbering of the atoms which are selected.

We also notice that it is often of interest to calculate only specific
contributions (usually the Fermi-contact\index{Fermi contact} contribution) at a high level
of approximation. Sometimes the results obtained with a Hartree--Fock
wave function may help in predicting the relative importance of the different
contributions, thus helping in the decision of which contributions
should be calculated at a correlated level \cite{krthklbpjcpl226}.
The calculation of only certain contributions can be accomplished in
the input by turning off the different
contributions by the keywords \Key{NODSO}, \Key{NOPSO},
\Key{NOSD}, and \Key{NOFC}. We refer to the description of the
\Sec{SPIN-S} input module for a somewhat more thorough discussion.

If a closer a control of the individual parts of the calculation of
indirect nuclear spin--spin coupling constants is wanted, this can be
done through the use of keywords in the following input modules:

\begin{list}{}{\itemsep 0.10cm \parsep 0.0cm}
\item[\Sec{EXPECT}] Controls the calculation of one-electron
expectation value contribution to the diamagnetic spin--spin coupling
constants.
\item[\Sec{GETSGY}] Controls the set up of the right-hand sides
(gradient terms).
\item[\Sec{LINRES}] Controls the solution of the singlet magnetic
response equations.
\item[\Sec{TRPRSP}] Controls the solution of the triplet magnetic
response equations (for Fermi-contact and spin--dipole contributions).
\item[\Sec{RELAX}] Controls the multiplication of solution and right hand
side vectors into relaxation contributions.
\item[\Sec{SPIN-S}] Controls the choice of nuclei for which the
spin--spin coupling constants will be calculated, as well as which
contributions to the total spin--spin coupling constants are to be
calculated.
\end{list}

\section{Hyperfine Coupling Tensors}

\begin{center}
\fbox{
\parbox[h][\height][l]{12cm}{
\small
\noindent
{\bf Reference literature:}
\begin{list}{}{}
\item B.Fernandez, P.J{\o}rgensen, J.Byberg, J.Olsen, T.Helgaker, and
H.J.Aa.Jensen. \newblock {\em J.Chem.Phys.}, {\bf
97},\hspace{0.25em}3412, (1992).
\item Solvent effects: B.Fernandez, O.Christiansen, O.Bludsky,
P.J{\o}rgensen, K.V. Mikkelsen. \newblock {\em J.Chem.Phys.}, {\bf
104},\hspace{0.25em}629, (1996).
\item DFT: Z.Rinkevicius, L.Telyatnyk, O.Vahtras and H.{\AA}gren.
\newblock {\em J.Chem.Phys.}, {\bf 121},\hspace{0.25em} 7614 (2004)
\end{list}
}}
\end{center}

    The calculation of hyperfine coupling\index{hyperfine
coupling} tensors (in vacuum or in
solution) is invoked by the keyword \Sec{ESR} in the \Sec{*RESPONSE}
input module. Thus a complete input file for the calculation of
hyperfine coupling tensors will be:

\begin{verbatim}
**DALTON INPUT
.RUN RESPONSE
**INTEGRALS
.FC
.SD
**WAVE FUNCTIONS
.HF
**RESPON
.TRPFLG
*ESR
.ESRCAL
.ATOMS
 2
 1 2
.FCCALC
.SDCALC
.MAXIT
   30
**END OF DALTON INPUT
\end{verbatim}
%.TRPPRP
%FC Cl 01
%.TRPPRP
%.
%.
%.
%
%.TRPPRP
%SD  01 x
%.TRPPRP
%SD  01 y
%.TRPPRP
%SD  01 z
%.TRPPRP
%.
%.
%.

    This will invoke the calculation of hyperfine
coupling\index{hyperfine coupling} tensors
using the Restricted-Unrestricted\index{restricted-unrestricted method}
methodology~\cite{bfpjjbjothhjajjcp97}. In this approach, the
unperturbed molecular system is described with a spin-restricted MCSCF
wave function, and when the perturbation---Fermi Contact\index{Fermi
contact} or Spin Dipole\index{spin-dipole}
operators---is turned on, the wave function spin relaxes and all
first-order molecular properties are evaluated as the sum of the
conventional average value term and a relaxation term that includes
the response of the wave function to the perturbations.

    The selection of a flexible atomic orbital basis set is decisive
in these calculations. Dunning's cc-pVTZ or Widmark's basis sets with some
functions uncontracted, and one or two sets of diffuse functions and
several tight $s$ functions added have been shown to provide accurate
hyperfine coupling tensors~\cite{bfpjcpl232}.

    If more close control of the different parts of the calculation
of hyperfine coupling tensors is wanted, we refer the reader to the
sections describing the options available.

\section{Electronic g-tensors}

\begin{center}
\fbox{
\parbox[h][\height][l]{12cm}{
\small
\noindent
{\bf Reference literature:}
\begin{list}{}{}
\item ROHF and MCSCF: O.Vahtras, B.  Minaev and H.�gren
\newblock {\em Chem.Phys.Lett.}, {\bf
281},\hspace{0.25em}186, (1997).
\item DFT: Z.Rinkevicius, L.Telyatnyk, P.Sa{\l}ek, O.Vahtras and H.{\AA}gren.
\newblock {\em J.Chem.Phys.}, {\bf 119
},\hspace{0.25em}10489, (2003).
\end{list}
}}
\end{center}

The calculation of electronic g-tensors\index{g-tensor} is invoked 
with the keyword \Key{G-TENSOR} in the 
\Sec{ESR} section of the \Sec{*RESPONSE} input module

\begin{verbatim}
**DALTON INPUT
.RUN RESPONSE
**WAVE FUNCTIONS
.HARTREE-FOCK
**RESPON
*ESR
.G-TENSOR
**END OF DALTON INPUT
\end{verbatim}

which gives by default all contributions to the g-tensor to second order in
the fine-structure parameter. Keywords following the \Key{G-TENSOR} keyword
will be interpreted as g-tensor options, which are defined in section
\ref{sec:g-tensor}

\section{Zero field splitting}

\begin{center}
\fbox{
\parbox[h][\height][l]{12cm}{
\small
\noindent
{\bf Reference literature:}
\begin{list}{}{}
\item ROHF and MCSCF: O.Vahtras, O.Loboda, B.Minaev, H.{\AA}gren and K.Ruud.
\newblock {\em Chem.Phys.}, {\bf
279},\hspace{0.25em}133, (2002).
\end{list}
}}
\end{center}
The calculation of the zero-field splitting is invoked 
with the keyword \Key{ZFS} in the 
\Sec{ESR} section of the \Sec{*RESPONSE} input module
\begin{verbatim}
**DALTON INPUT
.RUN RESPONSE
**WAVE FUNCTIONS
.HARTREE-FOCK
**RESPON
*ESR
.ZFS
**END OF DALTON INPUT
\end{verbatim}
Note that only the first-order electron spin-spin contribution
is implemented.

\section{CTOCD-DZ calculations}\label{sec:ctocd}

The Dalton program system can be used to perform calculations of the magnetic 
properties using the Continuous Transformation of the Origin of the Current Density
approach (CTOCD) \index{CTOCD-DZ}. Setting the diamagnetic contribution 
to the current density zero, one obtains fully analytical solutions via equations 
in closed form for several magnetic properties. 


In the present version of \dalton , the following properties can be computed in Dalton by the CTOCD-DZ approach:

\begin{center}
\begin{itemize}{}{}
\item magnetizability\index{magnetizability}
\item nuclear magnetic shielding constant\index{nuclear shielding}
\item shielding polarizability\index{shielding polarizability}
\item hypermagnetizability\index{hypermagnetizability}
\end{itemize}
\end{center}
the last two properties have to be calculated as quadratic response functions as described 
in Chapter~\ref{ch:rspchap}.

\subsection{General considerations}\label{sec:ctocdgeneral}

The CTOCD-DZ approach is competitive with other methods when the dimension
of the basis set is not too small, both for magnetizabilities and
shieldings ~\cite{ctocd}. However, for small basis sets the results can be 
very unreliable. A good basis set for CTOCD-DZ nuclear magnetic shiedling 
calculations, both at the SCF and the correlated level, is the aug-ccp-CVTZ-CTOCD-uc
basis set (Ref.~\cite{ctocd}), derived from the aug-cc-pVTZ basis set.  
This basis set is included in the basis set library.

In SCF calculations, the convergence toward the HF limit is slower 
than when employing London Orbitals since in the CTOCD-DZ expressions for 
the magnetic properties (nuclear magnetic shieldings and magnetizabilities) 
the diamagnetic terms also depend on the first order perturbed density matrix.

On the other hand, the CTOCD scheme does not only fulfill the requirement of
translational invariance of the calculated magnetic properties but also
guarantees current-charge conservation which is not the case for methods using
London Orbitals. In the case of molecules with a vanishing electric dipole
moment, CTOCD magnetic susceptibilities are origin independent and the
continuity equation is automatically satisfied.  

At the present it is possible to obtain CTOCD-DZ magnetizabilities and nuclear
magnetic shieldings with SCF, MCSCF, MP2 (using SOPPA) and CCSD (using
SOPPA(CCSD)) wave functions via the ABACUS program. Is also possible to
compute these properties using the RESPONSE program and using various CC wave
functions.  Finally is possible to use the quadratic response functions to
compute hypermagnetizabilities and shieldings polarizabilities for SCF, MCSCF
and CC wave functions. 

Using **RESPONSE, one has to be sure that the calculations of the diamagnetic
and paramagnetic contributions are both carried out with the gauge origin set
at the same positions, since only the full property and not the diamagnetic or
paramagnetic contributions are gauge origin independent.

For calculations of nuclear magnetic shieldings (and shielding
polarizabilities) using symmetry, the gauge origin has to be placed at the
center of mass, otherwise DALTON will give wrong results! This is the
default choice of gauge origin. Nuclear magnetic
shielding calculations with the gauge origin set at the respective atoms can
only be carried without symmetry and by setting the gauge origin on that atom
in the **INTEGRAL section. 

\subsection{Input description}\label{sec:ctocdinput}

\begin{center}
\fbox{
\parbox[h][\height][l]{12cm}{
\small
\noindent
{\bf Reference literature:}
\begin{list}{}{}
\item P.~Lazzeretti, M.~Malagoli and R.~Zanasi.
\newblock {\em Chem.~Phys.~Lett.}, {\bf 220},\hspace{0.25em} 299 (1994)
\item P.~Lazzeretti
\newblock {\em Prog.~Nucl.~Mag.~Res.~Spec.}, {\bf 220},\hspace{0.25em} 1--88
(2000) 
\item  Correlated calculations:
A.~Ligabue, S.~P.~A. Sauer and P.~Lazzeretti
\newblock {\em J.Chem.Phys.}, {\bf 118},\hspace{0.25em}6830, (2003).
\end{list}
}}
\end{center}

The input file for a CTOCD-DZ calculation of the magnetizability and
the nuclear magnetic shieldings will be:

\begin{verbatim}
**DALTON INPUT
.RUN PROPERTIES
**WAVE FUNCTIONS
.HF
**PROPERTIES
.CTOCD
.MAGNET
.SHIELD
**END OF DALTON INPUT
\end{verbatim}
whereas for the same calculation at SOPPA level it will be:

\begin{verbatim}
**DALTON INPUT
.RUN PROPERTIES
**WAVE FUNCTIONS
.HF
.MP2
**PROPERTIES
.SOPPA
.CTOCD
.MAGNET
.SHIELD
**END OF DALTON INPUT
\end{verbatim}

A Coupled Cluster calculation of the diamagnetic term of the CTOCD-DZ
magnetizability requires the following input:

\begin{verbatim}
**DALTON INPUT
.RUN WAVE FUNCTIONS
**INTEGRALS
.RANGMO
.DIPVEL
**WAVE FUNCTIONS
.CC
*ORBITAL
.NOSUPSYM
*CC INPUT
.CCS
.CCSD
*CCLR
.OPERAT
YDIPVEL ZXRANG
ZDIPVEL YXRANG
ZDIPVEL ZXRANG
XDIPVEL YXRANG
XDIPVEL ZXRANG
YDIPVEL YXRANG
YDIPVEL ZYRANG
ZDIPVEL YYRANG
ZDIPVEL ZYRANG
XDIPVEL YYRANG
XDIPVEL ZYRANG
YDIPVEL YYRANG
YDIPVEL ZZRANG
ZDIPVEL YZRANG
ZDIPVEL ZZRANG
XDIPVEL YZRANG
XDIPVEL ZZRANG
YDIPVEL YZRANG
**END OF DALTON INPUT
\end{verbatim}

The value for each component of the CTOCD-DZ diamagnetic contribution to the
magnetizability can then be obtained as:
\begin{center}
$\chi_{\alpha\beta}^{\Delta} = \epsilon_{\beta\lambda\mu} 
<< \hat{p}_{\lambda} ; \hat{r_O}_{\mu}\hat{L}_{\alpha} >>$ 
\end{center}
where $\hat{r_O}_{\mu}\hat{L}_{\alpha}$ are the $\mu\alpha$RANG operators.
Input examples for shieldings and shielding polarizabilities cand be found in
the test directory. 

For \Sec{*RESPONSE} calculations of these properties the integrals needed in
the \Sec{*INTEGRALS}  
section are .DIPVEL and .RANGMO for magnetizabilty, and .DIPVEL and .RPSO for
nuclear magnetic shieldings.


%\chapter{Calculation of optical and Raman properties}\label{ch:optchap}

This chapter describes the calculation of different optical
properties which have been implemented in the \siraba\ program system.
This includes properties related to different kinds of circular
dichroism, more specifically vibrational circular dichroism
(VCD)\index{VCD}\index{vibrational circular dichroism} as 
described in Ref.~\cite{klbpjthkrhjajjcp98}, electronic circular
dichroism (ECD)\index{ECD}\index{electronic circular dichroism} as
described in Ref.~\cite{klbaehkrthjopjtca90}, Raman 
Optical Activity (ROA)\index{ROA}\index{Raman optical activity} as
described in Ref.~\cite{thkrklbpjjofd99}, and optical rotation~\cite{plpmp91,plpdkckrcpl319}.

By default all calculations of optical properties are done with
the use of London atomic orbitals\index{London orbitals} in order to
enhance the basis set 
convergence as well as to give the correct physical dependence on the
gauge origin\index{gauge origin}. 

\section{Vibrational Circular Dichroism calculations}

\begin{center}
\fbox{
\parbox[h][\height][l]{12cm}{
\small
\noindent
{\bf Reference literature:}
\begin{list}{}{}
\item K.L.Bak, P.J{\o}rgensen, T.Helgaker, K.Ruud, and H.J.Aa.Jensen. \newblock {\em J.Chem.Phys.}, {\bf 98},\hspace{0.25em}8873, (1993).
\item K.L.Bak, P.J{\o}rgensen, T.Helgaker, K.Ruud, and H.J.Aa.Jensen. \newblock {\em J.Chem.Phys.}, {\bf 100},\hspace{0.25em}6620, (1994).
\end{list}
}}
\end{center}

The calculation of vibrational circular
dichroism\index{VCD}\index{vibrational circular dichroism} is invoked
by the 
keyword \Key{VCD} in the \aba\ input module. Thus a complete
input file for the calculation of vibrational circular dichroism will
look like:

\begin{verbatim}
**DALTON INPUT
.RUN PROPERTIES
**WAVE FUNCTIONS
.HF
**PROPERTIES
.VCD
**END OF DALTON INPUT
\end{verbatim}

This will invoke the calculation of vibrational circular dichroism
using London atomic orbitals\index{London orbitals} to ensure fast
basis set convergence as 
well as gauge origin\index{gauge origin} independent results. By
default the natural 
connection\index{natural connection} is used in order to get
numerically accurate
results~\cite{joklbkrthpjtca90,krthjopjklbcpl235}.

We notice, however, that vibrational circular dichroism only arises in
vibrationally chiral molecules. An easy way of introducing
vibrational chirality into small molecular systems is by isotopic
substitution. This is in 
\siraba\ controled in the \Sec{VIBANA} submodule, and the reader is
refered to that section for an exemplification of how this is done. 

There has been a couple of investigation of basis set requirement for
the calculation of VCD given in
Ref.~\cite{klbpjthkrhjajjcp100,klbpjthkrfd99}, 
and the reader is refered to these references when choosing basis set
for the calculations of VCD. 

In the current implementation, the \Key{NOCMC} option is autmatically
turned on in VCD calculations, that is, the coordinate system origin
is always used as gauge origin\index{gauge origin}.

We note that if a different force field is wanted in the calculation
of the VCD paramaters, this can be obtained by reading in an
alternative Hessian\index{Hessian} matrix with the input

\begin{verbatim}
**DALTON INPUT
.RUN PROPERTIES
**WAVE FUNCTIONS
.HF
**PROPERTIES
.VCD
*VIBANA
.HESFIL
**END OF DALTON INPUT
\end{verbatim}

If more close control of the different parts of the calculation of
vibrational circular dichroism is wanted, we refer the reader to the
sections describing the options available. The input sections that control
the calculation of vibrational circular dichroism are:

\begin{description}
\item[\Sec{AAT}] Controls the final calculation of the different
contributions to the Atomic Axial Tensors.
\item[\Sec{GETSGY}] Controls the set up of both the magnetic and
geometric right hand sides (gradient terms).
\item[\Sec{LINRES}] Controls the solution of the magnetic response
equations.
\item[\Sec{RELAX}] Controls the multiplication of solution and right hand
side vectors into relaxation contributions.
\item[\Sec{NUCREP}] Controls the calculation of the nuclear
contribution to the geometric Hessian.
\item[\Sec{TROINV}] Controls the use of translation and rotational invariance.
\item[\Sec{ONEINT}] Controls the calculation of one-electron
contributions to the geometric Hessian.
\item[\Sec{TWOEXP}] Controls the calculation of two-electron
expectation values to the geometric Hessian.
\item[\Sec{REORT}] Controls the calculation of reorthonormalization
terms to the geometric Hessian.
\item[\Sec{RESPON}] Controls the solution of the geometric response equations.
\item[\Sec{GEOANA}] Describes what analysis of the molecular geometry
is to be printed.
\item[\Sec{VIBANA}] Sets up the vibrational and rotational analysis of the
molecule, for instance it's isotopic substitution.
\item[\Sec{DIPCTL}] Controls the calculation of the Atomic Polar
Tensors (dipole gradient).
\end{description}

\section{Electronic circular dichroism (ECD) and electronic absorption
calculations}\label{sec:ecd} 

\begin{center}
\fbox{
\parbox[h][\height][l]{12cm}{
\small
\noindent
{\bf Reference literature:}
\begin{list}{}{}
\item K.L.Bak, Aa.E.Hansen, K.Ruud, T.helgaker, J.Olsen, and
P.J{\o}rgensen. \newblock {\em Theor.Chim.Acta.}, {\bf 90},\hspace{0.25em}441, (1995).
\end{list}
}}
\end{center}

The calculation of Electronic Circular Dichroism
(ECD)\index{ECD}\index{electronic circular dichroism} is invoked by 
the keyword \Key{ECD} in the \aba\ input module. However,
it is also necessary to specify the number of electronic
excitations\index{electronic excitation} in 
each symmetry. As ECD only is observed for chiral molecules, such
calculations will in general not employ any symmetry, and a complete
input for a molecule without symmetry will thus look like:

\begin{verbatim}
**DALTON INPUT
.RUN PROPERTIES
**WAVE FUNCTIONS
.HF
**PROPERTIES
.ECD
*EXCITA
.NEXCIT
    3
**END OF DALTON INPUT
\end{verbatim}

In this run we will calculate the rotatory strength\index{rotatory
strength} corresponding to 
the three lowest electronic excitations\index{electronic excitation}
(the \Key{NEXCIT} keyword) 
using London atomic orbitals\index{London orbitals}. 
If rotatory strengths obtained without London atomic orbitals is also
wanted, this is easily accomplished by adding the keyword
\Key{ROTVEL} in the \Sec{EXCITA} input module.

There has so far only been presented one study of Electronic Circular
Dichroism using London atomic orbitals \cite{klbaehkrthjopjtca90}, and the
results of this investigation indicate that the aug-cc-pVDZ basis
set, which is supplied with the \siraba\ basis set library, is
reasonable for such calculations. 

Another property that may often be of interest is the
oscillatory strength\index{transition moment}. This property can be
calculated by an input 
similar to the one for ECD calculation, and for a molecule with
C$_{2v}$ symmetry an input would look like:

\begin{verbatim}
**DALTON INPUT
.RUN PROPERTIES
**WAVE FUNCTIONS
.HF
**PROPERTIES
.EXCITA
*EXCITA
.DIPSTR
.NEXCIT
    3    2    1    0
**END OF DALTON INPUT
\end{verbatim}

This input will calculate the dipole strength (\Key{DIPSTR}) of the
6 lowest electronic excitations distributed in a total of 4
irreducible representations (as in C$_{2v}$). The dipole strength will
be calculated both in 
length and velocity forms. It is expected that the same requirements
for basis set quality applies to this property as for ECD.

For a SOPPA\index{SOPPA} or SOPPA(CCSD)\index{SOPPA(CCSD)} calculation of
the oscillator strength the additional keywords \Key{SOPPA} or
\Key{SOPPA(CCSD)} have to be specified in the \Sec{*PROPERTIES} input
module. For SOPPA an MP2 calculation has to be requested by the keyword
\Key{MP2} in the \Sec{*WAVE FUNCTIONS} input module, whereas for SOPPA(CCSD) a
CCSD calculation has to be requested by the keyword \Key{CC} in the \Sec{*WAVE
  FUNCTIONS} input module with the \Key{CC INPUT} option
\Key{SOPPA(CCSD)}.

The two properties may of course be combined a single run, with an
input that would then look like (where we also request the rotatory
strength to be calculated without the use of London orbitals):

\begin{verbatim}
**DALTON INPUT
.RUN PROPERITES
**WAVE FUNCTIONS
.HF
**PROPERTIES
.ECD
.EXCITA
*EXCITA
.DIPSTR
.ROTVEL
.NEXCIT
    3
**END OF DALTON INPUT
\end{verbatim}

We also note that excitation energies also can be obtained using the
\resp\ program (see Chapter~\ref{ch:rspchap}).
For a more detailed control of the individual parts of the 
calculation of properties related to electronic excitation energies,
we refer to the input modules affecting the different parts of such
calculations:

\begin{description}
\item[\Sec{EXCITA}] Controls the calculation of electronic excitation
energies and the evaluation of all terms contributing to for instance
dipole strength or electronic circular dichroism.

\item[\Sec{GETSGY}] Controls the setup of the necessary right-hand
sides.
\end{description}

\section{Optical Rotation}\label{sec:optrot}

\begin{center}
\fbox{
\parbox[h][\height][l]{12cm}{
\small
\noindent
{\bf Reference literature:}
\begin{list}{}{}
\item T.Helgaker, K.Ruud, K.L.Bak, P.{\o}rgensen, and
J.Olsen. \newblock {\em Faraday Discuss.}, {\bf 99},\hspace{0.25em}165, (1994).
\item P.~L.~Polavarapu \newblock {\em Mol.~Phys.}, {\bf 91},\hspace{0.25em}551, (1997).
\end{list}
}}
\end{center}

The calculation of optical rotation is a special case of the
calculation of Vibrational Raman Optical Activity (see
Sec.~\ref{sec:vroa}), as the tensor determining the optical rotation,
the mixed electric-magnetic polarizability, also contributes to
vibrational Raman optical activity, although in the latter case it the
geometrical derivatives of the tensor which are the central.

Many of the comments made regarding basis set requirments for VROA
calculations will thus also be applicable for the calculation of
optical. However, we note that a very extensive basis set
investigation of the optical rotation have been
presented~\cite{jrcmjffjdpjsjpca104}.

A typical of an input for the calculation of optical rotation would
be:

\begin{verbatim}
**DALTON INPUT
.RUN PROPERTIES
**WAVE FUNCTIONS
.HF
**PROPERTIES
.OPTROT
.ABALNR
*ABALNR
.FREQUENCY
 2
 0.001 0.077318
.THRESH
 1.0D-4
**END OF DALTON INPUT
\end{verbatim}

Dalton will always calculate the optical rotation both with and
without London atomic orbitals. The optical rotation will only be
observed for molecules without symmetry, and by definition the optical
rotation will be zero in the static limit. One can approximate the
static limit by supplying the program with a very large wavelength
(small frequency), as in the example above, in order to be able to
compare with approximations that are only valid in the static
limit~\cite{rdacpl87,jrcmjffjdpjsjpca104}. 

\section{Vibrational Raman Optical Activity (VROA)}\label{sec:vroa}

\begin{center}
\fbox{
\parbox[h][\height][l]{12cm}{
\small
\noindent
{\bf Reference literature:}
\begin{list}{}{}
\item T.Helgaker, K.Ruud, K.L.Bak, P.{\o}rgensen, and
J.Olsen. \newblock {\em Faraday Discuss.}, {\bf 99},\hspace{0.25em}165, (1994).
\end{list}
}}
\end{center}

The calculation of vibrational Raman intensities and vibrational Raman optical
activity (VROA)\index{ROA}\index{Raman intensity}\index{Raman
optical activity} 
is one of the more computationally expensive properties that can be
evaluated with \siraba .

Due to the time spent in the numerical differentiation\index{numerical
differentiation}, we have chosen
to calculate ROA both with and without London atomic
orbitals\index{London orbitals} in the
same calculation, because the time used in the set-up of the right-hand
sides differentiated
with respect to the external magnetic field is negligible compared to
the time used in the solution of the time-dependent response
equations~\cite{thkrklbpjjofd99}. Because of this,  all relevant Raman
properties (intensities and depolarization ratios)\index{Raman
intensity}\index{depolarization ratio} is also calculated
at the same time as ROA.  


A very central part in the evaluation of Raman Optical Activity is the
evaluation the electric dipole-electric dipole, the electric
dipole-magnetic dipole, and the electric dipole-electric quadrupole
polarizabilities, and we refer to Section~\ref{sec:polari} for a more
detailed description of the input for such calculations.


When calculating Raman intensities\index{Raman intensity} and
ROA\index{ROA}\index{Raman optical activity} we need to do a numerical
differentiation\index{numerical differentiation} of the electric
dipole-electric dipole, the electric 
dipole-magnetic dipole, and the electric dipole-electric quadrupole
polarizabilities along the normal modes\index{normal mode} of the
molecule. The procedure 
is described in Ref.~\cite{thkrklbpjjofd99}. We thus need to do a
geometry walk of the type numerical differentiation. In each geometry
we need to evaluate the electric dipole-electric dipole, the electric
dipole-magnetic dipole, and the electric dipole-electric quadrupole
polarizabilities. This may be achieved by the following input:

\begin{verbatim}
**DALTON INPUT
.WALK
.MAX IT
 31
*WALK
.NUMERI
**WAVE FUNCTIONS
.HF
*HF INPUT
.THRESH
1.0D-8
**START
.VROA
.ABALNR
.ISOTOP
    5
 1 1 1 2 3
*ABALNR
.THRESH
1.0D-7
.FREQUE
     2
0.0 0.09321471
**PROPERTIES
.VROA
.ABALNR
.ISOTOP
    5
 1 1 1 2 3
*ABALNR
.THRESH
1.0D-7
.FREQUE
     2
0.0 0.09321471
**FINAL
.VROA
.ABALNR
.VIBANA
.ISOTOP
    5
 1 1 1 2 3
*ABALNR
.THRESH
1.0D-7
.FREQUE
     2
0.0 0.09321471
*RESPONSE
.THRESH
1.0D-6
*VIBANA
.PRINT
 2
.ISOTOP
 1 5
 1 1 1 2 3
**END OF DALTON INPUT
\end{verbatim}

This is the complete input for a calculation of
VROA\index{ROA}\index{Raman optical activity} on the CFHDT
molecule\index{fluoromethane}.  
Although the keyword \Key{VROA} is added in the different \aba\ input
modules, we still need to tell the program that frequencies are to be
read in the \Sec{ABALNR} section by also adding the keyword
\Key{ABALNR} in the general input module.

The only isotopic substitution of this molecule that shows
vibrational optical activity is the one containing one hydrogen, one
deuterium and one tritium nucleus. This is reflected in the keyword
\Key{ISOTOP}, as we want the center-of-mass\index{center of mass} to
be the gauge origin\index{gauge origin} for the VROA calculation not
employing London atomic orbitals. 
We note that a user specified gauge origin can be supplied with the
keyword \Key{GAUGEO} in the \aba\ input modules. The gauge origin
can also be chosen as the origin of the Cartesian Coordinate
system~(0,0,0) by using the keyword \Key{NOCMC}. Note that neither of
these options will affect the results obtained with London orbitals.

The input in the \Sec{ABALNR} input section should be
self-explanatory from the discussion of the frequency dependent
polarizability\index{polarizability} in Sec.~\ref{sec:polari}. Note
that because of the numerical differentiation\index{numerical
differentiation} the response equations need to be converged 
rather tightly (1.0$\cdot$10$^{-7}$). Remember also that this will
require you to converge your wave function\index{wave
function}\index{convergence!threshold} more
tightly than is the default. 

The numerical differentiation\index{numerical differentiation} is
invoked through the keyword 
\Key{NUMERI} in the \Sec{WALK} submodule. Note that this will
automatically 
turn off the calculation of the molecular Hessian\index{Hessian},
putting limitations 
on what properties may be calculated during a ROA calculation. Because
of this there will not be any prediction of the energy at the new
point.

It should also be noted that the program in a numerical differentiation will
step plus and minus one displacement along each Cartesian
coordinate
of all nuclei, as well as calculating the property in the reference
geometry. Thus, for a molecule with $N$ atoms the properties will be
calculated in a total of $2*3*N + 1$ points, which for a 5 atom 
molecule will amount to 31 points. The default maximum number of steps
of the program is 20. Thus one often also need to
change the maximum number of allowed iterations by for instance
adding the appropriate number of iterations in the general input
module using the keyword \Key{MAX IT} described in
Section~\ref{sec:general}.

The default step length in the numerical integration is 10$^{-4}$
a.u., and this step length may be adjusted by the keyword
\Key{DISPLA} in the \Sec{WALK} module. The steps are taken in
the Cartesian directions\index{Cartesian coordinates} and 
not along normal modes\index{normal mode}. This enables us to study a
large number of 
isotopically substituted\index{isotopic constitution} molecules at
once, as the London orbital \index{London orbitals}
results for ROA does not depend on the choice of gauge origin. This is
done in the \Sec{*FINAL} input module, but as only one isotopic
substituted species show optical activity, we have only requested a
vibrational analysis for this species.

We note that as in the case of Vibrational Circular Dichroism, a different
force field may be used in the estimation of the VROA intensity
parameters. Indeed, a number of force fields can be used to estimate
the VROA parameters obtained with a given basis set through the input:

\begin{verbatim}
**DALTON INPUT
.WALK
.MAX IT
 31
.ITERATION
 31
*WALK
.NUMERI
**FINAL
.VROA
.ABALNR
.VIBANA
.ISOTOP
    5
 1 1 1 2 3
*ABALNR
.THRESH
1.0D-7
.FREQUE
     2
0.0 0.09321471
*RESPONSE
.THRESH
1.0D-6
*VIBANA
.HESFIL
.PRINT
 2
.ISOTOP
 1 5
 1 1 1 2 3
**END OF DALTON INPUT
\end{verbatim}
by copying different \verb|DALTON.HES| files to the scratch
directory, which in turn is read through the keyword \Key{HESFIL}. By
choosing the start iteration to be 31 through the keyword
\Key{ITERAT}, we tell the program that the walk has finished (for
CHFDT with 31 points that need to be calculated). However, this
requires that all information is available in the \verb|DALTON.WLK|
file.

It is evident that the calculation of Vibrational Raman Optical
Activity\index{ROA}\index{Raman optical activity} is indeed a task for
connoisseurs. However, the use of London 
orbitals as well as frequency dependent properties, makes \siraba\ the
currently most accurate way of calculating vibrational Raman optical activity. 

Concerning basis sets requirement for Raman Optical Activity, 
there is yet much work to be done. In the only study
presented of Raman Optical Activity using London atomic orbitals it is
argued in favor of the aug-cc-pVDZ basis set supplied with the basis
set library. However, in order to get Hartree--Fock limit quality of
the vibrational frequencies as well, a basis set of at least
aug-cc-pVTZ seems to be necessary. However, this is far too large to
be used in a routine calculation of VROA of a naturally optical active molecule.

%\chapter{Getting the property you want}\label{ch:rspchap}


In the preceding chapters we have shown how to calculate a number of
properties that are associated with specific spectroscopic applications.
%such as NMR-parameters (Chapter~\ref{ch:magnetic}). 
For HF, DFT, SOPPA, and MCSCF these properties are in part
calculated in the \resp\ program, but given that a large number of
standard calculations usually are carried out in a similar fashion, some
applications have a simplified input (under {\tt **PROPERTIES}), 
and an appealing output that meets common demands 
({\it e.g.\/} customary unit conversions).
For CC calculations of properties, see Chapter~\ref{ch:CC}.
In this chapter we
describe how to set up the input for calculating a general property that
can be defined in terms of electronic response functions. 
%A very
%helpful document detailing various linear and nonlinear properties
%that can be calculated with the response program, and the conversion
%of the results obtain to other units have been written by
%Jaszu\'{n}ski and Rizzo~\cite{}, and this document is available on the
%dalton homepage
%(\verb|http://www.kjemi.uio.no/software/dalton/dalton.html| or \verb|http://www.sdsc.edu/dalton/dalton.html|).

\section{General considerations}
\label{sec:rspgen}

\begin{center}
\fbox{
\parbox[h][\height][l]{12cm}{
\small
\noindent
{\bf Reference literature:}
\begin{list}{}{}
\item Response theory:
Jeppe Olsen and Poul J{\o}rgensen, \newblock {\em J. Chem. Phys.} {\bf 82}, \hspace{0.25em}3235, (1985)
\end{list}
}}
\end{center}

A response function is a measure of how a property of a system changes in
the presence of one or more perturbations. With our notation (see {\it e.g.\/}
Ref.~\cite{jopjjcp82}),  $\langle\!\langle A;B\rangle\!\rangle_{\omega_b}$,
$\langle\!\langle A;B,C\rangle\!\rangle_{\omega_b,\omega_c}$, and 
$\langle\!\langle A;B,C,D\rangle\!\rangle_{\omega_b,\omega_c,\omega_d}$
denote linear, quadratic and cubic response\index{linear response}\index{quadratic response}\index{cubic response}\index{response!linear}\index{response!quadratic}\index{response!cubic}\index{response function}
functions, respectively, which
provide the first, second, and third-order corrections to the
expectation-value of $A$, due to the perturbations $B$, $C$, and $D$, each of
which is associated with a frequency $\omega_b$, $\omega_c$, and
$\omega_d$. Often the perturbations are considered to be 
external monochromatic fields, or static ({\it e.g.\/} relativistic) perturbations,
in which case the frequency is zero.   In general, the perturbations $B$,
$C$, and $D$ represent Fourier components of an arbitrary time-dependent
perturbation.

\section{Input description}
\label{sec:rspex}
 
In this section we describe a few minimal input examples for calculating some
molecular properties that can be expressed in terms of linear, quadratic, and
cubic response functions.
Note that only one of these three different orders of response functions can be
requested in the same calculation.

For more information on keywords, see the Reference Manual, chapter~\ref{ch:response}.
 
\subsection{Linear response}
\label{subsec:linrsp}
%\subsection{Polarizability, $\alpha_{ij}(0;0)$ $i,j \in \{x,y,z\}$}

\begin{center}
\fbox{
\parbox[h][\height][l]{12cm}{
\small
\noindent
{\bf Reference literature:}
\begin{list}{}{}
\item Singlet linear response:
Poul J{\o}rgensen, Hans J{\o}rgen Aagaard Jensen, and Jeppe Olsen, \newblock {\em J. Chem. Phys.} {\bf 89}, \hspace{0.25em}3654, (1988)
\item Triplet linear response:
Jeppe Olsen, Danny L. Yeager, and Poul J{\o}rgensen, \newblock {\em J. Chem. Phys.} {\bf 91}, \hspace{0.25em}381, (1989)
\item SOPPA linear response:
Martin J. Packer, Erik K. Dalskov, Thomas Enevoldsen, Hans J{\o}rgen Aagaard Jensen and Jens Oddershede, 
\newblock {\em J. Chem. Phys.}, {\bf 105}, \hspace{0.25em}5886, (1996)
\item SOPPA(CCSD) linear response:
Stephan P. A. Sauer,
\newblock {\em J. Phys. B: At. Opt. Mol. Phys.}, {\bf 30}, \hspace{0.25em}3773, (1997)
\item DFT open-shell linear response:
Zilvinas Rinkevicius, Ingvar Tunell, Pawe{\l} Sa{\l}ek, Olav Vahtras, and Hans {\AA}gren, 
\newblock {\em J. Chem. Phys.}, {\bf 119}, \hspace{0.25em}34, (2003)
\end{list}
}}
\end{center}

A well-known example of a linear response\index{linear response}\index{response!linear}\index{polarizability}
function is the polarizability.
A typical input for SCF static and dynamic polarizability tensors
$\alpha_{ij}(-\omega;\omega)\equiv-\langle\!\langle
x_i;x_j\rangle\!\rangle_\omega$ for a few selected frequencies (in
atomic units) will be:
\begin{verbatim}
**DALTON INPUT
.RUN RESPONSE
**WAVE FUNCTIONS
.HF
**RESPONSE
*LINEAR
.DIPLEN
.FREQUENCIES
 3
 0.0 0.5 1.0
**END OF DALTON INPUT
\end{verbatim}
The {\tt .DIPLEN} keyword has the effect of defining the $A$ and $B$
operators as all components of the electric dipole operator.

A Second Order Polarization Propagator Approximation
(SOPPA)\index{SOPPA}\cite{esnpjjodjcp73,jopjdycpr2,mjpekdtehjajjojcp}
calculation of linear response functions
can be invoked if the additional keyword \Key{SOPPA} is specified in the 
\Sec{*RESPONSE} input module and an MP2 calculation is requested by the 
keyword \Key{MP2} in the \Sec{*WAVE FUNCTIONS} input module.  A typical input 
for SOPPA dynamic polarizability tensors  will be:
\begin{verbatim}
**DALTON INPUT
.RUN RESPONSE
**WAVE FUNCTIONS
.HF
.MP2
**RESPONSE
.SOPPA
.NOITRA
*LINEAR
.DIPLEN
.FREQUENCIES
 3
 0.0 0.5 1.0
**END OF DALTON INPUT
\end{verbatim}
The {\tt .NOITRA} keyword has the effect that the transformation of the two 
electron integrals necessary for a MP2 and SOPPA calculation is only performed
once in the \Sec{*WAVE FUNCTIONS} module.

A Second Order Polarization Propagator Approximation with Coupled Cluster
Singles and Doubles Amplitudes - SOPPA(CCSD)\index{SOPPA(CCSD)}\cite{soppaccsd}
calculation of linear response functions can be invoked if the additional 
keyword \Key{SOPPA(CCSD)} is specified in the 
\Sec{*RESPONSE} input module and an CCSD calculation is requested by the 
keyword \Key{CC} in the \Sec{*WAVE FUNCTIONS} input module.  A typical input 
for SOPPA(CCSD) dynamic polarizability tensors  will be:
\begin{verbatim}
**DALTON INPUT
.RUN RESPONSE
**WAVE FUNCTIONS
.HF
.CC
*CC INPUT
.SOPPA(CCSD)
**RESPONSE
.SOPPA(CCSD)
*LINEAR
.DIPLEN
.FREQUENCIES
 3
 0.0 0.5 1.0
**END OF DALTON INPUT
\end{verbatim}

The linear response function contains a wealth of
information about the spectrum of a given Hamiltonian. 
It has poles\index{pole of response function} at the excitation
energies\index{electronic excitation}, 
relative to the reference state (not necessarily the ground state) and the
corresponding residues\index{residue} are transition
moments\index{transition moment} between the reference and
excited states. To calculate the excitation energies\index{electronic
excitation} and dipole transition 
moments\index{transition moment} for the three lowest excited states
in the fourth symmetry, a small 
modification of the input above will suffice;
\begin{verbatim}
**RESPONSE
*LINEAR
.SINGLE RESIDUE
.DIPLEN
.ROOTS
 0 0 0 3
\end{verbatim}

\subsection{Quadratic response}
\label{subsec:quadrsp}
%\subsection{First hyperpolarizability, 
%$\beta_{zzz}(0;0,0)

\begin{center}
\fbox{
\parbox[h][\height][l]{12cm}{
\small
\noindent
{\bf Reference literature:}
\begin{list}{}{}
\item Singlet quadratic response: 
Hinne Hettema, Hans J{\o}rgen Aa. Jensen, Poul J{\o}rgensen, and Jeppe Olsen, \newblock {\em J. Chem. Phys.} {\bf 97}, \hspace{0.25em}1174, (1992)
\item Triplet quadratic response: 
Olav Vahtras, Hans {\AA}gren, Poul J{\o}rgensen, Hans J{\o}rgen Aa. Jensen, Trygve Helgaker, and Jeppe Olsen, \newblock {\em J. Chem. Phys.} {\bf 97}, \hspace{0.25em}9178, (1992)
\item Integral direct quadratic response: 
Hans {\AA}gren, Olav Vahtras, Henrik Koch, Poul J{\o}rgensen, and Trygve Helgaker, \newblock {\em J. Chem. Phys.} {\bf 98}, \hspace{0.25em}6417, (1993)
\item DFT singlet quadratic response: 
Pawe{\l} Sa{\l}ek, Olav Vahtras, Trygve Helgaker, and Hans {\AA}gren, \newblock {\em J. Chem. Phys.} {\bf 117}, \hspace{0.25em}9630, (2002)
\item DFT triplet quadratic response:
Ingvar Tunell, Zilvinas Rinkevicius, Olav Vahtras, Pawe{\l} Sa{\l}ek, Trygve Helgaker, and Hans {\AA}gren,
\newblock {\em J. Chem. Phys.} {\bf 119}, \hspace{0.25em}11024, (2003)
\end{list}
}}
\end{center}

An example of a quadratic response\index{quadratic response} function
is the first 
hyperpolarizability\index{first hyperpolarizability}. If we are
interested in
$\beta_{zzz}\equiv-\langle\!\langle z;z,z\rangle\!\rangle_{0,0}$ 
only, we may use the following input:
\begin{verbatim}
**DALTON INPUT
.RUN RESPONSE
**WAVE FUNCTIONS
.HF
**RESPONSE
*QUADRATIC
.DIPLNZ
**END OF DALTON INPUT
\end{verbatim}
When no frequencies are given in the input, the static value is assumed by
default. If we wish to calculate dynamic hyperpolarizabilities we supply
frequencies\index{frequency}, but in this case we have two frequencies
$\omega_b, \omega_c$ which are given by the keywords \texttt{.BFREQ} and
{\tt .CFREQ} (see the Reference Manual, chapter~\ref{ch:response}).
%LRHYP module in rspvec.F:
The non-zero linear response functions from the operators can be
generated with no additional computational costs, and all
$\langle\!\langle A;B\rangle\!\rangle_{\omega_b}$ results
will also be printed (in this example $\alpha_{zz}$).

The residue of a quadratic response function gives two-photon
transition amplitudes\index{two-photon!amplitude}. For such a
calculation we supply the same extra 
keywords as in the linear case (Sec.~\ref{subsec:linrsp}):
\begin{verbatim}
**RESPONSE
*QUADRATIC
.DIPLNZ
.SINGLE RESIDUE
.ROOTS
 2 0 0 0
\end{verbatim}
which in this case means the two-photon transition
amplitude\index{two-photon!amplitude} between the
reference state and the first two excited states in the first symmetry.  In
general the residue of a quadratic response function corresponds to the
induced transition moment of an operator $A$ due to a perturbation $B$.
The $C$ operator is arbitrary and is not specified.  A typical example is
the dipole matrix element between a singlet and triplet state that is
induced by spin-orbit coupling
(phosphorescence)\index{phosphoresence}. For this special case we 
have the keyword, {\tt .PHOSPHORESCENCE} under {\tt *QUADRATIC}, which sets
$A$ to electric dipole operators and $B$ to spin-orbit operators.

The residue of a quadratic response function can be used to identify
the two-photon transition amplitudes. The input below refers to the
calculation of the two-photon absorption from the ground state to the
first 3 excited states in point group symmetry one. In the program
output the two-photon transition matrix element is given as well as
the two-photon transition probability relevant for an isotropic gas or
liquid. The evaluation of the transition probabilities can be done based
on the transition matrix elements although they, in principle, are
connected with the imaginary part of the second
hyperpolarizability. The absorption cross sections are evaluated
assuming a monochromatic light source that is either linearly or
circularly polarized.
\begin{verbatim}
**RESPONSE
*QUADRATIC
.TWO-PHOTON
.ROOTS
 3 0 0 0
\end{verbatim}

Another special case of a residue of the quadratic response function 
is the ${\cal{B}}(0\to f)$ term of magnetic circular dichroism (MCD).
\begin{verbatim}
**RESPONSE
*QUADRATIC
.SINGLE RESIDUE
.ROOTS
 2 2 0 0
.MCDBTERM
\end{verbatim}
For each dipole-allowed excited state among those specified in 
{\tt .ROOTS}, the {\tt .MCDBTERM} keyword automatically 
calculates all symmetry allowed products of the single residue of the 
quadratic response function for $A$ corresponding to the electric dipole
operator and $B$ to the angular momentum operator with the single residue 
of the linear response function for $C$ equal to the electric dipole operator.
In other words, the mixed electric dipole---magnetic dipole two-photon
transition moment\index{two-photon!transition moment}\index{transition moment!two-photon}
for final state $f$ times the dipole one-photon moment for the same state $f$.
Note that in the current implementation (for SCF and MCSCF), degeneracies between
excited states may lead to numerical divergencies. 
The final ${\cal{B}}(0\to f)$ must be obtained from a combination 
of the individual components, see the original paper~\cite{Coriani:MCDRSP}.
%, or the document by Jaszu\'{n}ski and Rizzo on the Dalton
%homepage. 

It is possible to construct double
residues\index{residue}\index{double residue} of the quadratic
response function, the interpretation of which is transition
moments\index{transition moment}\index{excited state}
between two 
excited states. Specifying \Key{DOUBLE} in the example above thus gives
the matrix elements of the $z$-component of the dipole moment between
all excited states specified in {\tt .ROOTS}. Note that the diagonal contributions
gives  not the expectation value in the excited state, but rather the
difference relative to the reference state expectation value.

\subsection{Cubic response}
\label{subsec:cubrsp}
%\subsection{Second hyperpolarizability, $\gamma_{ijkl}(0;0,0,0)$
%$i,j,k,l \in \{x,y,z\}$} \cite{pndjhapdkrthhkcpl253}

\begin{center}
\fbox{
\parbox[h][\height][l]{12cm}{
\small
\noindent
{\bf Reference literature:}
\begin{list}{}{}
\item SCF cubic response:
Patrick Norman, Dan Jonsson, Olav Vahtras, and Hans {\AA}gren, \newblock {\em Chem. Phys. Lett.} {\bf 242}, \hspace{0.25em}7, (1995)
\item MCSCF cubic response:
Dan Jonsson, Patrick Norman, and Hans {\AA}gren, \newblock {\em J. Chem. Phys.} {\bf 105}, \hspace{0.25em}6401, (1996)
\end{list}
}}
\end{center}

All components of the static second hyperpolarizability\index{cubic response}\index{response!cubic}\index{second hyperpolarizability}
defined as $\gamma_{ijkl}(-0;0,0,0)\equiv$
$-\langle\!\langle x_i;x_j,x_k,x_l\rangle\!\rangle_{000}$,
 may be obtained by the following input

\begin{verbatim}
**DALTON INPUT
.RUN RESPONSE
**WAVE FUNCTIONS
.HF
**RESPONSE
*CUBIC
.DIPLEN
**END OF DALTON INPUT
\end{verbatim}


%\subsection{Polarizability of the first excited state of symmetry 1, \\
%$\alpha^e_{zz}(-0.01;0.01) - \alpha^0_{zz}(-0.01;0.01)$}
%\cite{djpnylhajcp105}

As  mentioned above (Sec.~\ref{subsec:quadrsp}), the "diagonal" 
double residue\index{residue}\index{double residue} of the quadratic
response function is the change in the expectation value relative to the
reference state. The analogue for cubic
response functions is the change in
polarizability\index{polarizability}\index{excited state} relative to the
reference state polarizability, which is demonstrated by the following
input.
\begin{verbatim}
**DALTON INPUT
.RUN RESPONSE
**WAVE FUNCTIONS
.HF
**RESPONSE
*CUBIC
.DOUBLE RESIDUE
.DIPLNZ
.FREQUENCIES
 1
 0.01
.ROOTS
  1 0 0 0
**END OF DALTON INPUT
\end{verbatim}




%\chapter{Direct and parallel calculations}\label{ch:dirpar}

In this chapter we briefly discuss aspects connected to
direct\index{direct calculation} and parallel\index{parallel calculation}
methods as implemented in the \siraba\ program. 

\section{Direct methods}\label{sec:direct}

\begin{center}
\fbox{
\parbox[h][\height][l]{12cm}{
\small
\noindent
{\bf Reference literature:}
\begin{list}{}{}
\item H.\AA gren, O.Vahtras, H.Koch, P.J\o rgensen, and
T.Helgaker. \newblock {\em J.Chem.Phys.}, {\bf
98},\hspace{0.25em}6471, (1993). 
\item K.~Ruud, D.~Jonsson, P.~Norman, H.~{\AA}gren, T.~Saue,
H.~J.~Aa.~Jensen, P.~Dahle, and P.~Dahle. \newblock {\em
J.~Chem.~Phys.} {\bf 108}, 7973 (1998).
\end{list}
}}
\end{center}

The entire SCF\index{SCF}\index{HF}\index{Hartree--Fock} part
of the \siraba\ code is direct, including all derivative two-electron
integrals, and all the way up to the cubic response function. To
perform a direct\index{direct calculation} calculation, all that is
required is to add the 
keyword \Key{DIRECT} in the general input section, as indicated in the
following input example for the calculation of nuclear
shieldings\index{nuclear shielding} in a
direct fashion:

\begin{verbatim}
**DALTON INPUT
.RUN PROPERTIES
.DIRECT
**WAVE FUNCTIONS
.HF
**PROPERTIES
.SHIELD
*END OF INPUT
\end{verbatim}

By default the two-electron integrals will be
screened~\cite{krdjpnhatshjajpdthjcp108}, and a integral 
threshold of $10^{-10}$ is the default threshold for whether an
integral will be calculated or not. This can be changed with the
keywords \Key{IFTHRS} and \Key{ICEDIF}\index{integral screening}.

\section{Parallel methods}\label{sec:parallel}

\begin{center}
\fbox{
\parbox[h][\height][l]{12cm}{
\small
\noindent
{\bf Reference literature:}
\begin{list}{}{}
\item P.Norman, D.Jonsson, H.\AA gren, P.Dahle, K.Ruud, T.helgaker,
and H.Koch. \newblock {\em Chem.Phys.Lett.}, {\bf
253},\hspace{0.25em}1, (1996).
\end{list}
}}
\end{center}

As for direct methods, the entire Hartree--Fock part of the \siraba\
program has been parallelized, allowing the use of both PVM and
MPI\index{PVM}\index{MPI}\index{message passing} as
message passing interfaces. The use of the parallel code requires,
however, that the code has been installed as a parallel
code\index{parallel calculation}, which is
being determined during the building of the program as described in
Section~\ref{sec:Makefile}.

If MPI is used as message passing interface, all that is needed to do
of changes in the \verb|DALTON.INP| file is to add the keyword
\Key{PARALLEL} in the general input section, as demonstrated for a
calculation of vibrational frequencies\index{vibrational frequency}:

\begin{verbatim}
**DALTON INPUT
.RUN PROPERTIES
.PARALLEL
**WAVE FUNCTIONS
.HF
**PROPERTIES
.VIBANA
*END OF INPUT
\end{verbatim}

The number of nodes\index{node} to be used in the calculation is
requested to the 
\verb|dalton| run script after the \verb|-N| option (see
Section~\ref{sec:firstcalc}), or as stated in local
documantation. Note that the master/slave\index{master}\index{slave}
paradigm employed by 
\siraba\ will leave the master mainly doing sequential parts of the
calculation and distribution of tasks, thus very little computation
compared to the \verb|N-1| slaves, see
Ref.~\cite{pndjhapdkrthhkcpl253}.

In case of PVM runs, the program will spawn the requested number of
slaves (enabling you to create a slave on the same machine as the
master process in order to provide a more efficient use of the CPU
power on the master machine). The number of slaves is requested
through the keyword \Key{NODES} in the \Sec{PARALLEL} input module, as
indicated in the following example: 

\begin{verbatim}
**DALTON INPUT
.RUN PROPERTIES
.PARALLEL
*PARALLEL
.NODES
 4
**WAVE FUNCTIONS
.HF
**PROPERTIES
.SHIELD
**END OF INPUT
\end{verbatim}

Note that this input would correspond to an MPI run with 5 nodes, as
the master-process has to be added to the number of nodes.

By default the two-electron integrals will be screened, and an integral
threshold of $10^{-10}$ is the default threshold for whether an
integral will be calculated or not. This can be changed with the
keywords \Key{IFTHRS} and \Key{ICEDIF}\index{integral screening}.

%\chapter{Finite field calculations}\label{ch:finite}

Despite the large number of properties that in principle can be
calculated with \siraba , it is often of interest to study the
dependence of these properties under an external electric
perturbation. This can be easily achieved by adding static electric
fields\index{finite field}, and thus increase the number of properties
that can be 
calculated with \siraba .

In the next section we comment briefly on some important aspects of
finite electric field calculations, and the following section
describes the input for finite field calculations.

\section{General considerations}\label{sec:finitegeneral}

The presence of an external electric field can be modeled by adding a
term to our ordinary field-free, non-relativistic Hamiltonian
corresponding to the interaction between the dipole moment operator
and the external electric field:

\begin{equation}
\mathcal{H} = \mathcal{H}^{0} - \mathbf{Ed}_{e}
\end{equation}
where $\mathbf{d}_{e}$ is the electric dipole moment\index{dipole
moment} operator defined as 

\begin{equation}
\mathbf{d}_{e} = \sum_{i}\mathbf{r}_{i}
\end{equation}
and $\mathcal{H}^{0}$ is our ordinary field-free, non-relativistic
Hamiltonian operator. It is noteworthy that we do not include the
nuclear dipole moment\index{nuclear dipole moment} operator, and the
total electronic energy will 
thus depend on the position of the molecule in the Cartesian
coordinate frame.

The electric field dependence of different molecular properties are
obtainable by adding fields in different directions and with different
signs and then extract the information by numerical
differentiation\index{numerical differentiation}. Note that care has
to be taken to choose a field that 
is weak enough for the numeric differentiation to be valid,
yet large enough to give numerically significant changes in the
molecular properties. Note also that it may be necessary to increase
the convergence threshold for the solution of the response equations
if molecular properties are being evaluated.

Whereas the finite field approach may be combined with any property
that can be calculated with the \resp\ program, more care need to be
taken if the finite field method is used with the \aba\
program. Properties that involve perturbation-dependent basis
sets\index{perturbation-dependent basis set},
like nuclear shieldings and molecular Hessians, will often introduce
extra reorthonormalization terms due to the finite field operator, and
care has to be taken to ensure that these terms indeed have been
included in \siraba . 

{\em NOTE: In the current release, the only properties
calculated with perturbation dependent basis sets that may be
numerically differentiated using finite field, are the nuclear shieldings
and magnetizabilities using the implementation described in
Ref.~\cite{arthkrabmjpjjcp102}, and molecular gradients as described
in Ref.~\cite{}}. 

\section{Input description}\label{sec:finiteinput}

\begin{center}
\fbox{
\parbox[h][\height][l]{12cm}{
\small
\noindent
{\bf Reference literature:}
\begin{list}{}{}
\item Shielding and magnetizability polarizabilities: A.Rizzo, T.Helgaker, K.Ruud, A.Barszczewicz, M.Jaszu\'{n}ski and P.J{\o}rgensen. \newblock {\em J.Chem.Phys.}, {\bf 102},\hspace{0.25em}8953, (1995).
\end{list}
}}
\end{center}

The necessary input for a finite-field\index{finite field} calculation
is given in the 
\Sec{*INTEGRALS} and \Sec{*WAVE FUNCTIONS} input modules. A typical input file
for an finite field SCF calculation of the
magnetizability\index{magnetizability} of a molecule will be:

\begin{verbatim}
**DALTON INPUT
.RUN PROPERTIES
**INTEGRALS
.DIPLEN
**WAVE FUNCTIONS
.HARTREE-FOCK
*HAMILTONIAN
.FIELD
 0.003
 XDIPLEN
**PROPERTIES
.MAGNET
*END OF INPUT
\end{verbatim}

In the \Sec{*INTEGRALS} input module we request the evaluation of dipole
length\index{dipole length integral} integrals, as these correspond to
the electric dipole operator\index{dipole moment}, 
and will be used in \sir\ for evaluating the interactions between the
electric dipole and the external electric field. This is achieved in
the \Sec{HAMILTONIAN} input module, where the presence of an external
electric field\index{electric field!external} is signaled by the
keyword \Key{FIELD}. On the next line, the
strength of the electric field (in atomic units) is given, and on the following
line we give the direction of the applied electric field
(\verb|XDIPLEN|, \verb|YDIPLEN|, or \verb|ZDIPLEN|). Several fields may
of course be applied at the same time.


%\chapter{Solvent calculations}\label{ch:solvent}

This chapter describes the implementation of the Multiconfigurational
Self-Consistent Reaction Field (MCSCRF) model as implemented in \siraba
. The first section describes some considerations about the
implementation  and the range of
properties that may be evaluated with the present MCSCRF
implementation. The second section gives two input examples for MCSCRF 
calculations.

\section{General considerations}\label{sec:solventimpl}

\siraba\ has the possibility of modeling the effect of a surrounding
linear, homogeneous dielectric medium\index{dielectric medium} on a
variety of molecular 
properties using
SCF\index{SCF}\index{HF}\index{Hartree-Fock}\index{MCSCF} or MCSCF
wave functions. This is achieved by the 
Multiconfigurational Self-Consistent Reaction Field
(MCSCRF)\index{reaction field}\index{MCSCRF}
approach~\cite{kvmedpsjpc91,kvmhahjajthjcp89}, where the solute is
placed in a spherical cavity\index{cavity} and 
surrounded by the dielectric medium. The solvent response to the
the presence of the solute is modeled by a multipole
expansion\index{multipole expansion}, in \siraba\ in
principle to infinite order, but practical applications show that the
multipole expansion is usually converged at order $L=6$.

In \siraba\ the solvent model is implemented both for SCF and MCSCF wave
functions in a self-consistent manner as describes in
Ref.~\cite{kvmedpsjpc91,kvmhahjajthjcp89}. In MCSCF calculations where
MP2 orbitals is requested as starting orbitals for the MCSCF
optimization, the solvent model will not be added before entering the
MCSCF optimization stage, so MP2 gas-phase orbitals can be used as
starting guess even though the solvent model has not been implemented
for this wave function.

As regards molecular properties, the solvent model has so far been
extended to linear singlet and triplet response
properties\index{linear response}\index{triplet response} in the
\resp\ program. Thus a number of properties and excitation energies can
be calculated with the (MC)SCRF model, and several studies of such
properties have been presented, and we refer to these papers for an
overview of what can currently be calculated with the
approach~\cite{kvmpjhjajjcp100,kvmylhapjjcp100}, including ESR
hyperfine coupling constants~\cite{bfocobpjkvmjcp104}\index{hyperfine
coupling}. 

In addition, a non-equilibrium solvation\index{non-equilibrium
solvation} model has been implemented 
for molecular energies~\cite{kvmachahjajjcp103}. This model in needed
when studying processes where the charge distribution of the solute
cannot be expected to be in equilibrium with the charge distribution
of the solvent, e.g. when comapring with experiments where light has
been used as a perturbation.

In the \aba\ program, the solvent model has been implemented for
geometric distortions and nuclear shieldings and
magnetizabilities\index{nuclear shielding}\index{magnetizability},
and of course all the 
properties that do not use perturbation-dependent basis
sets\index{perturbation-dependent basis set}, like for
instance indirect spin-spin coupling constants\index{spin-spin
coupling}. This is noteworthy, as 
although the program will probably give results for most results
calculated using the solvent  model, these results will not
necessarily be theoretically correct, due to lack of reorthonormalization
contributions that have not been considered in the program. We
therefore give a fairly complete literature reference of works that
have been done with the
program~\cite{kvmpjkrthjcp106,poakvmkrthjpc100}. Properties not
included in this list are thus not trustworthy with the current
version of \siraba . 

\section{Input description}\label{sec:solventinp}

\begin{center}
\fbox{
\parbox[h][\height][l]{12cm}{
\small
\noindent
{\bf Reference literature:}
\begin{list}{}{}
\item General reference: K.V.Mikkelsen, E.Dalgaard,
P.Svanstr{\o}m. \newblock {\em J.Phys.Chem}, {\bf
91},\hspace{0.25em}3081, (1987).
\item General reference: K.V.Mikkelsen, H.{\AA}gren, H.J.Aa.Jensen,
and T.Helgaker. \newblock {\em J.Chem.Phys.}, {\bf
89},\hspace{0.25em}3086, (1988).
\item Non-equilibrium solvation: K.V.Mikkelsen, A.Cesar, H.{\AA}gren,
H.J.Aa.Jensen.\newblock {\em J.Chem.Phys.}, {\bf
103},\hspace{0.25em}9010, (1995).
\item Linear singlett response: K.V.Mikkelsen, P.J{\o}rgensen,
H.J.Aa.Jensen.\newblock {\em J.Chem.Phys.}, {\bf
100},\hspace{0.25em}6597, (1994).
\item Linear triplet response: P.-O.\AA strand, K.V.Mikkelsen, P.J{\o}rgensen,
K.Ruud and T.Helgaker. To be published.
\item Hyperfine couplings: B.Fernandez, O.Christensen, O.Bludsky,
P.J{\o}rgensen, 
K.V.Mikkelsen. \newblock {\em J.Chem.Phys.}, {\bf
104},\hspace{0.25em}629, (1996).
\item Magnetizabilities and nuclear shieldings: K.V.Mikkelsen,
P.J{\o}rgensen, K.Ruud, and T.Helgaker. \newblock {\em J.Chem.Phys.}, {\bf
107},\hspace{0.25em}1170, (1997).
\item Molecular Hessian: P.-O.\AA strand, K.V.Mikkelsen, K.Ruud and
T.Helgaker. \newblock {\em J.Phys.Chem.}, {\bf
100},\hspace{0.25em}19771, (1996).
\item Spin-spin couplings: P.-O.\AA strand, K.V.Mikkelsen, P.J{\o}rgensen,
K.Ruud and T.Helgaker. To be published.
\end{list}
}}
\end{center}


The necessary input for a solvent calculation is given in the
\Sec{*INTEGRALS} and \Sec{*WAVE FUNCTIONS} input modules. A typical input file
for an SCF calculation of the nuclear shielding constants of a
molecule in a dielectric medium will look like\index{nuclear
shielding}\index{dielectric medium}: 

\begin{verbatim}
**DALTON INPUT
.RUN PROPERTIES
**INTEGRALS
*ONEINT
.MAX L
 10
**WAVE FUNCTIONS
.HF
*SOLVENT
.DIELECTRIC
 78.5
.MAX L
 10
.CAVITY
 3.98
**PROPERTIES
.SHIELD
*END OF INPUT
\end{verbatim}

In \Sec{*INTEGRALS} we request the evaluation of the undifferentiated solvent
multipole integrals\index{multipole integral} as given in for instance
Ref.~\cite{kvmhahjajthjcp89} by the keyword \Key{MAX L} in the
\Sec{ONEINT} submodule. We request all 
integrals up to $L=10$ to be evaluated. This is needed if static or
dynamic (response) properties
calculations are to be done, but is not needed for a run of the
wave function only (\Sec{*WAVE FUNCTIONS}).

In \Sec{*WAVE FUNCTIONS} there is a separate input module for the
solvent input, 
headed by the name \Sec{SOLVENT}. We refer to Sec.~\ref{ref-solinp}
for a presentation of all possible keywords in this submodule. The
interaction between the solute and the dielectric
medium\index{dielectric medium} is
characterized by three parameters; the dielectric
constant\index{dielectric constant}, the cavity\index{cavity}
size, and the order of the multipole expansion\index{multipole
expansion}. In the above input we have 
requested a dielectric constant of 78.5 (corresponding to water\index{water})
through the keyword \Key{DIELECTRIC}, a cavity radius of 3.98 atomic
units with the keyword \Key{CAVITY}, and the multipole expansion is to
include all terms up to $L=10$, as can be seen from the keyword
\Key{MAX L}. Note that this number cannot be larger than the number given for
\Key{MAX L} in the \Sec{ONEINT} input module.

\subsection{Geometry optimization}\label{sec:solventgeoopt}

In the present release of the \siraba\ program, there are certain
limitations imposed on the optimizing geometries using the solvent
model. Only second-order geometry optimizations \index{geometry
optimization} are available, and 
only through the general \Sec{WALK} module. Furthermore, symmetry
cannot be used during the geometry optimization, and care must be
exercised in order to turn off automatic symmetry detection in case of
Hartree-Fock calculations. Thus the input for an SCF geometry
optimization with the solvent model would look like:

\begin{verbatim}
**DALTON INPUT
.WALK
**INTEGRALS
*ONEINT
.MAX L
 10
**WAVE FUNCTIONS
.HF
**FINAL
.VIBANA
.SHIELD
*END OF DALTON
\end{verbatim}


\subsection{Non-equilibrium solvation}

This example describes calculations for non-equilibrium
solvation\index{non-equilibrium solvation}. 
Ususally one starts with a calculation of a reference state
(most often the ground state) with equilibrium solvation, using
keyword \Key{INERSF}. The interface file is then
used (without user interference) for
a non-equilibrium excited state calculation; keyword
\Key{INERSI}. 

\begin{verbatim}
**DALTON INPUT
.RUN SIRIUS
**INTEGRALS
*ONEINT
.MAX L
 10
**WAVE FUNCTIONS
.TITLE
 2-RAS(2p2p') : on F+ (1^D) in Glycol 
 Widmark (5432)-ANO Basis set 
.MCSCF        
.NSYM
 8
*CONFIGURATION INPUT
.SPIN MULTIPLICITY
 1
.SYMMETRY
 1
.INACTIVE ORBITALS
 1  0  0  0  0  0  0  0 
.ELECTRONS
 6
.RAS1 SPACE
 0  0  0  0  0  0  0  0
.RAS2 SPACE
 1  2  2  0  2  0  0  0
.RAS3 SPACE
 8  4  4  3  4  3  3  1
.RAS1 ELECTRONS 
 0  0 
.RAS3 ELECTRONS 
 0  2 
*OPTIMIZATION
.NEO ALWAYS
.OPTIMAL ORBITAL TRIAL VECTORS
.MAX CI
 30
*ORBITAL INPUT
.NOSUPSYM
.MOSTART       | Note, we assume the existence of an fort.21 file
 NEWORB
*CI VECTOR
.STARTOLDCI
*SOLVENT
.CAVITY
 2.5133D0
.INERSI   | initial state inertial polarization
 37.7D0  2.050D0  | static and optic dielectric constants for Glycol
.MAX L
 10
*END OF INPUT
\end{verbatim}

%\chapter{Vibrational corrections}\label{ch:vibave}

\siraba\ provides an efficient automated procedure for calculating
rovibrationally averaged molecular $r_\alpha$ geometries, as well as an
automated procedure for calculating vibrational averages of a large
range of second-order molecular properties, for SCF and MCSCF wave
functions. In the current implementation, it is not possible to
exploit point-group symmetry, and one must ensure that the symmetry is
turned off in the calculation.

\begin{center}
\fbox{
\parbox[h][\height][l]{12cm}{
\small
\noindent
{\bf Reference literature:}
\begin{list}{}{}
\item Effective geometries: P.-O.~{\AA}strand, K.~Ruud and
P.~R.~Taylor.\newblock {\em J.Chem.Phys}, {\bf 112},\hspace{0.25em},
2655 (2000).
\item Vibrational averaged properties: K.~Ruud, P.-O.~{\AA}strand and
P.~R.~Taylor.\newblock {\em J.Chem.Phys.}, {\bf
112},\hspace{0.25em}2668, (2000). 
\item Temperature and isotope effects: K.~Ruud, J.~Lounila and
J.~Vaara. \newblock {\em J.Chem.Phys.}, to be published.
\end{list}
}}
\end{center}

\section{Effective geometries}\label{sec:effgeom}

The (ro)vibrationally averaged geometries\index{effective
geometries}\index{vibrationally averaged
geometries}\index{rovibrationally averaged geometries} can be
calculated from a knowledge of part of the cubic force field

\begin{equation}
\left<r_i\right> = r_{e,i} -
\frac{1}{4\omega_i^2}\sum_{j=1}^{3N-6}\frac{V^{\left(3\right)}_{ijj}}{\omega_j}
\end{equation}
where the summation runs over all normal modes in the molecule and
where $\omega_i$ is the harmonic frequency of normal mode $i$ and
$V^{\left(3\right)}_{ijj}$ is the cubic force field. A typical input
for determining (ro)vibrationally averages Hartree--Fock geometries
for different water isotopomers will look like

\begin{verbatim}
**DALTON INPUT
.WALK
*WALK
.ANHARM
.DISPLACEMENT
0.001
.TEMPERATURES
 4
 0.0 300.0 500.0 1000.0
**WAVE FUNCTIONS
.HF
*SCF INPUT
.THRESH
 1.0D-10
**START
*RESPONS
.THRESH
 1.0D-5
**PROPERTIES
*RESPONS
.THRESH
 1.0D-5
**FINAL
.VIBANA
*RESPONS
.THRESH
 1.0D-5
*VIBANA
.ISOTOP
 4 3
 1 2 1
 1 2 2
 2 1 1
*END OF INPUT
\end{verbatim}

The calculation of (ro)vibrationally averaged geometries are invoked
be the keyword \verb|.ANHARM| in the \verb|*WALK| input module. In
this example, the full cubic force field will be determined as 
first derivatives of analytical molecular Hessians. This will be done
in Cartesian coordinates, and the calculation will therefore require
the evaluation of $6K + 1$ analytical Hessians, where $K$ is the
number of atoms in the molecules. Although expensive, it allows to
calculate (ro)vibrationally for any isotopic spicies, in the above
example for H$_2\;^{16}$O, HD$_2\;^{16}$O, D$_2\;^{16}$O,
H$_2\;^{18}$O. This is directed by the keyword \verb|.ISOTOP|. We
note that the most abundant isotope will always be calculated, and are
therefore not included in the list above.

We have requested that rovibrationally averaged geometries be
calculated for 5 different temperatures. By default, these geometries
will include centrifugal distortions~\cite{krjljv}. This can be
turned by using the keyword \verb|.NO CENT| in the \verb|*WALK| input
module.

By default, the numerical differentiation will use a step length of
0.0001 bohr. Experience show this to be too short~\cite{poakrprtjcp112}, and we have
therefore changed this to be 0.001 bohr in the example above by the
use of the keyword \verb|.DISPLACMENT| in the \verb|*WALK| input
module.

If only one (or a few) isotopic species are of interest, we can significantly
speed up the calculation of the (ro)vibrationally averaged geometries
by doing the numberical differentiation in the normal coordinates of
the isotopic species of interest. This can be requested through the
keyword \verb|.NORMAL|. The relevant part of the cubic purce field is
then calculated as numerical second derivatives of analytical
gradients. We note that the suggested step length in this case should
be set to 0.0075~\cite{poakrprtjcp112}. We note that we will still need to calculate one
analytical Hessian in order to determine the normal coordinates.

As the default maximum number of iterations in the \verb|*WALK| is 25,
one will have to specify the appropriate number of iterations by using
the keyword \verb|.MAX IT| in the \verb|**DALTON INPUT| module if the
molecule being studied have more than 4-5 atoms.

\section{Vibrational averaged
geometries}\label{sec:vibavegeo}\index{vibrational
corrections}\index{zero-point vibrational
corrections}\index{temperature effects}

The change in the geometry accounts for part of the contribution to a
vibrationally averaged property, namely that due to the anharmonicity
of the potential~\cite{krpoaprtjacs123}. Although this term is important, we need to
include also the contribution from the averaging of the molecular
property over the harmonic oscillator wave function in order to get an
accurate estimate of the vibrational corrections to the molecular
property. 

At the effective geometry, this contribution to for instance the
nuclear shielding constants can be obtained from the following input
\begin{verbatim}
**DALTON INPUT
.WALK
*WALK
.VIBAVE
.DISPLACEMENT
0.05
.TEMPERATURES
 1
 300.0
**WAVE FUNCTIONS
.HF
*SCF INPUT
.THRESH
 1.0D-10
**START
.SHIELD
.ISOTOP
 3
 3 2 1
*LINRES
.THRESH
 1.0D-6
*RESPONS
.THRESH
 1.0D-5
**PROPERTIES
.SHIELD
.ISOTOP
 3
 3 2 1
*LINRES
.THRESH
 1.0D-6
*RESPONS
.THRESH
 1.0D-5
*END OF INPUT
\end{verbatim}

This input will calculate the harmonic contribution to the
(ro)vibrational average to the nuclear shielding constants at 300K for
$^{17}$ODH. It is important to realize that since each isotopic
species for each temperature will have its own unique
(ro)vibrationally averaged geometry, we will have to calculate the
harmonic contribution for each temperature and each isotopic species
separately. 

This calculation will always be doen in normal coordinates, and the
recommended step length is 0.05~\cite{krpoaprtjcp112}. As for the calculation of
(ro)vibrationally averaged geometries in normal coordinates, the
calculation requires the determination of one analytical Hessian in
order to determine the harmonic force field.

As the default maximum number of iterations in the \verb|*WALK| is 25,
one will have to specify the appropriate number of iterations by using
the keyword \verb|.MAX IT| in the \verb|**DALTON INPUT| module if the
molecule being studied have more than 4-5 atoms.


%\chapter{\label{chap:Relativity}Relativistic Effects}

The following approaches to treat relativistic effects are available in DALTON:

\begin{description}

\item[ECP]\index{Effective core potentials}\index{ECP}\index{basis set!ECP}
The Effective Core Potential approach of Pitzer and
Winter~\cite{rmpnmwijqc40} is available for single-point calculations
by asking for ECP as the basis set for the chosen element. So far,
only a limited set of elments is covered by the basis set library. See
the \verb|rsp_ecp| example in the test-suite. The corresponding spin-orbit
operators are not implemented.

\begin{center}
\fbox{
\parbox[h][\height][l]{12cm}{
\small
\noindent
{\bf Reference literature:}
\begin{list}{}{}
\item R.~M.~Pitzer and N.~M.~Winter. \newblock {\em Int.~J. of Quantum Chem.}, {\bf 40}, \hspace{0.25em} 773 (1991)
\item L.~E.~McMurchie and E.~R.~Davidson.  \newblock {\em J.~Comp.~Phys.}, {\bf 44},\hspace{0.25em} 289 (1981)
\end{list}
}}
\end{center}


\item[Douglas-Kroll]\index{Douglas--Kroll} The Douglas--Kroll scalar relatvistic one-electron integrals  
are available by adding the \verb|.DOUGLAS-KROLL|   keyword
\begin{verbatim}
**DALTON INPUT
.DOUGLAS-KROLL
.RUN WAVE FUNCTIONS
....
\end{verbatim}

See also the \verb|energy_douglaskroll| example in the test suite.

  NOTE: Exact analytical gradients and hessians are not available 
at the moment, the approximate gradient and Hessians does however give fairly accurate geometries.  For this approach, only basis sets should be used 
where the contraction coefficients were optimized including the Douglas-Kroll 
operators. DALTON currently provides:  DK-Pol (relativistic version of Sadlej's POL basis sets), raf-r for some heavy elements, and the relatvistically recontracted correlation-consistent basis sets of Dunning (cc-pVXZ-DK, X=D,T,Q,5). The combination with 
property operators should be done with care, {\it e.g.\/} the standard
magnetic property operators are not suitable in this case.

\begin{center}
\fbox{
\parbox[h][\height][l]{12cm}{
\small
\noindent
{\bf Reference literature:}
\begin{list}{}{}
\item M.~Douglas and N.~M.~Kroll. \newblock {\em Ann.~Phys.~(N.Y.)}, {\bf 82},\hspace{0.25em} 89 (1974)
\item B.~A.~Hess. \newblock {\em Phys.~Rev.~{\bf A}}, {\bf 33},\hspace{0.25em} 3742 (1986)
\end{list}
}}
\end{center}

\item[Spin-orbit Mean-Field]\index{AMFI}\index{spin-orbit mean-field} The spin-orbit mean-field approach can be used 
for either replacing the Breit-Pauli spin-orbit operator, or as an operator 
with suitable relativistic corrections in combination with the Douglas-Kroll 
approach. It is based on an effective one-electron operator, where the two-electron
terms are summed in a way comparable to the Fock operator. As all multi-center 
integrals are neglected, this scheme is very fast, avoids the storage of the 
two-electron spin-orbit integrals, and can therefore be used for large systems. 

\begin{verbatim}
.....
**INTEGRALS    
.MNF-SO    replaces     .SPIN-ORBIT             
.....
\end{verbatim}

For properties, the same substitution should be made, in the case of special 
components, \verb|X1SPNORB| labels are replaced by \verb|X1MNF-SO| and so on, whereas the 
two-electron terms will be skipped totally. For calculating phosphorescence with 
the quadratic response scheme, \verb|.PHOSPHORESENCE| should be just resplaced by 
\verb|.MNFPHO| which takes care of chosing the approriate integrals. 


\begin{center}
\fbox{
\parbox[h][\height][l]{12cm}{
\small
\noindent
{\bf Reference literature:}
\begin{list}{}{}
\item B.~A.~Hess, C.~M.~Marian, U.~Wahlgren and O.~Gropen. \newblock {\em Chem.~Phys.~Lett.}, {\bf 251},\hspace{0.25em} 365 (1996)
\end{list}
}}
\end{center}


NOTE:  

The choice between the Breit-Pauli or Douglas-Kroll mean-field operator 
is done by (not) providing the .DOUGLAS-KROLL keyword. It is therefore 
not possible to combine {\it e.g.\/} non-relatvistic wave-functions with the 
Douglas-Kroll spin-orbit integrals. 


In the present implementation, the mean-field approach works only for basis sets 
with a generalized contraction scheme such as the ANO basis sets, raf-r, or cc-pVXZ(-DK).
For other types of basis sets, the program might work without a crash, but it
will most likely provide erroneous results.  



\end{description}




%\chapter{SOPPA and SOPPA(CCSD) calculations}\label{ch:soppa}

The Dalton program system can also be used to perform Second-Order
Polarization Propagator Approximation (SOPPA) \index{SOPPA}
\index{polarization propagator} calculations
\cite{esnpjjodjcp73,mjpekdtehjajjojcp,spascpl260,tejospastcan100} or
Second-Order Polarization Propagator Approximation with Coupled Cluster
Singles and Doubles Amplitudes [SOPPA(CCSD)] \index{SOPPA(CCSD)}
calculations of optical properties like singlet or triplet excitation
energies\index{electronic   excitation} and oscillator
strengths\index{transition moment} as well as the 
following list of electric and magnetic properties
\begin{center}
\begin{list}{}{}
\item polarizability\index{polarizability}
\item magnetizability\index{magnetizability}
\item rotational {\em g} tensor\index{rotational g tensor}
\item nuclear magnetic shielding constant\index{nuclear shielding}
\item nuclear spin--rotation constant\index{spin-rotation constant}
\item indirect nuclear spin--spin coupling constant\index{spin-spin coupling}
\end{list}
\end{center}
as well as all the linear response functions described in chapter
\ref{ch:rspchap}.


\section{General considerations}\label{sec:soppageneral}

The Second-Order Polarization Propagator Approximation is a generalization of
the SCF linear response function
\cite{esnpjjodjcp73,jopjdycpr2,mjpekdtehjajjojcp}. In SOPPA, the SCF reference
wave function in the linear response function or polarization propagator is
replaced by a M{\o}ller-Plesset wave function and all matrix elements in the
response function are then evaluated through second order in the fluctuation
potential. This implies that electronic excitation energies and oscillator
strengths as well as linear response functions are correct through second
order. Although it is a second-order method like MP2, the SOPPA equations
differ significantly from the expressions for second derivatives of an MP2
energy. 

In the Second Order Polarization Propagator Approximation with Coupled Cluster
Singles and Doubles Amplitudes [SOPPA(CCSD)], the M{\o}ller-Plesset
correlation coefficients are replaced by the corresponding coupled cluster
singles and doubles amplitudes
\cite{soppaccsd,ekdspasjpca102,tejospastcan100,ctocd}. SOPPA(CCSD) is not a
coupled cluster linear response function as they are implemented in the CC
modules of Dalton program. However, the equations are essentially the
same as for SOPPA.  


\section{Input description}\label{sec:soppainput}

\begin{center}
\fbox{
\parbox[h][\height][l]{12cm}{
\small
\noindent
{\bf Reference literature:}
\begin{list}{}{}
\item General reference : E.~S. Nielsen, Poul J{\o}rgensen, and Jens
  Oddershede. 
\newblock {\em J.~Chem.~Phys.}, {\bf 73},\hspace{0.25em}6238, (1980)
\item General reference : J. Oddershede, Poul J{\o}rgensen and Danny Yeager,
\newblock {\em Comput. Phys. Rep.}, {\bf 2},\hspace{0.25em}33, (1984)
\item Excitation energy : Martin J. Packer, Erik K. Dalskov, Thomas
  Enevoldsen, Hans J{\o}rgen Aagaard Jensen and Jens Oddershede,  
\newblock {\em J. Chem. Phys.}, {\bf 105}, \hspace{0.25em}5886, (1996)
\item Rotational {\em g} tensor : Stephan P.~A. Sauer,
\newblock {\em Chem. Phys. Lett.}  {\bf 260},\hspace{0.25em}271, (1996)
\item Polarizability : Erik K. Dalskov and Stephan P.~A. Sauer.
\newblock {\em J.~Phys.~Chem.~{\bf A}}, {\bf 102},\hspace{0.25em}5269, (1998)
\item  Spin-Spin Coupling Constants : Thomas Enevoldsen, Jens
  Oddershede, 
and Stephan P.~A. Sauer.
\newblock {\em Theor.~Chem.~Acc.}, {\bf 100},\hspace{0.25em}275, (1998)
\item CTOCD-DZ nuclear shieldings: A.Ligabue, S.P.A.Sauer, P.Lazzeretti.
\newblock {\em J.Chem.Phys.}, {\bf 118},\hspace{0.25em}6830, (2003).\end{list}
}}
\end{center}

A prerequisite for any SOPPA calculation is that the calculation of
the MP2 energy and wavefunction is invoked by the keyword \Key{MP2} in
the \Sec{*WAVE FUNCTIONS} input module. Furthermore in the
\Sec{*PROPERTIES} or \Sec{*RESPONSE} input modules it has to be specified 
by the keyword \Key{SOPPA} that a SOPPA calculation of the properties
should be carried out.

A typical input file for a SOPPA calculation of the indirect nuclear
spin-spin coupling constants of a molecule will be:

\begin{verbatim}
**DALTON INPUT
.RUN PROPERTIES
**WAVE FUNCTIONS
.HF
.MP2
**PROPERTIES
.SOPPA
.SPIN-S
**END OF DALTON INPUT
\end{verbatim}
whereas as typical input file for the calculation of triplet
excitation energies\index{electronic excitation} with the
\Sec{*RESPONSE} module will be: 
\begin{verbatim}
**DALTON INPUT
.RUN RESPONSE
**WAVE FUNCTIONS
.HF
.MP2
**RESPONSE
.TRPFLG
.NOITRA
.SOPPA
*LINEAR
.SINGLE RESIDUE
.ROOTS
 4
**END OF DALTON INPUT
\end{verbatim}
The {\tt .NOITRA} keyword has the effect that the transformation of the two 
electron integrals necessary for a MP2 and SOPPA calculation is only performed
once in the \Sec{*WAVE FUNCTIONS} module.

A prerequisite for any SOPPA(CCSD) calculation is that the calculation of
the CCSD amplitudes for the SOPPA program is invoked by the keyword \Key{CC}
in the \Sec{*WAVE FUNCTIONS} input module together with the \Key{SOPPA(CCSD)}
option in the \Sec{CC INPUT} section. Furthermore, in the \Sec{*PROPERTIES} or
\Sec{*RESPONSE} input modules it has to specified by the keyword \Key{SOPPA(CCSD)}
that a SOPPA(CCSD) calculation of the properties should be carried out. 

A typical input file for a SOPPA(CCSD) calculation of the indirect nuclear
spin-spin coupling constants of a molecule will be:

\begin{verbatim}
**DALTON INPUT
.RUN PROPERTIES
**WAVE FUNCTIONS
.HF
.CC
*CC INPUT
.SOPPA(CCSD)
**PROPERTIES
.SOPPA(CCSD)
.SPIN-S
**END OF DALTON INPUT
\end{verbatim}
whereas as typical input file for the calculation of triplet
excitation energies\index{electronic excitation} with the
\Sec{*RESPONSE} module will be: 
\begin{verbatim}
**DALTON INPUT
.RUN RESPONSE
**WAVE FUNCTIONS
.HF
.CC
*CC INPUT
.SOPPA(CCSD)
**RESPONSE
.TRPFLG
.SOPPA(CCSD)
*LINEAR
.SINGLE RESIDUE
.ROOTS
 4
**END OF DALTON INPUT
\end{verbatim}


%\chapter{NEVPT2 calculations}\label{ch:nevpt2}

\section{General considerations}\label{sec:nevptgeneral}
NEVPT2 \index{NEVPT2} is a form of second order multireference
perturbation theory \index{multireference PT}
which can be applied to CAS--SCF wavefunctions or, more generally, to
CAS--CI wavefunctions. The term NEVPT is an acronym for
\textit{``n--electron valence state perturbation theory''}. While we
refer the reader to the pertinent
literature\cite{nevpt1,nevpt2,nevpt3,nevpt4}, we limit ourselves to
recalling here that the most relevant feature of NEVPT2 consists in
that the first order correction to the wavefunction is expanded over a
set of properly chosen \emph{multirefernce} functions which correctly
take into consideration the two--electron interactions occurring among
the active electrons. Among the properties ensured by NEVPT2 we quote:

\begin{itemize}
\item Strict separability (size consistence): the energy of a
  collection of non--interacting systems equals the sum of the
  energies of the isolated systems
\item Absence of intruder states: the zero order energies associated
  to the functions of the outer space are well separated from the zero
  order energy of the state under study, thus avoiding divergences in
  the perturbation summation
\item The first order correction to the wavefunction is an
  eigenfunction of the spin operators $S^2$ and $S_z$
\item Electronically excited states are dealt with at the same level
  of accuracy as the ground state
\item NEVPT2 energies are invariant under a unitary transformation of
  the active orbitals. Furthermore, the choice of canonical orbitals
  for the core and virtual orbitals (the default choice) ensure that
  the results coincide with those of an enlarged version of the theory
  fully invariant under rotations in the core and virtual orbital
  spaces, respectively\cite{nevpt4}
\item NEVPT2 coincides with MP2 in the case of a HF wavefunction
\end{itemize}

NEVPT2 has been implemented in two variants both of which are present
in \dalton, i.e. the \emph{strongly contracted} (SC) and the
\emph{partially contracted} (PC) variant. The two variants differ by
the number of perturber functions employed in the perturbation
summation. The PC--NEVPT2 uses a richer function space and is
supposedly more accurate than the SC--NEVPT2. The results of
SC--NEVPT2 and PC--NEVPT2 are anyway usually very close to one
another.

\begin{center}
\fbox{
\parbox[h][\height][l]{12cm}{
\small
\noindent
{\bf Reference literature:}
\begin{list}{}{}
\item General reference : C.~Angeli, R.~Cimiraglia, S.~Evangelisti,
  T.~Leininger and J.~P.~Malrieu,
\newblock {\em J.~Chem.~Phys.}, {\bf 114},\hspace{0.25em}10252, (2001)
\item General reference : C.~Angeli, R.~Cimiraglia and J.~P.~Malrieu,
\newblock {\em Chem. Phys. Lett.}, {\bf 350},\hspace{0.25em}297, (2001)
\item General reference : C.~Angeli, R.~Cimiraglia and J.~P.~Malrieu,
\newblock {\em J. Chem. Phys.}, {\bf 117}, \hspace{0.25em}9138, (2002)
\item  Excited states : C.~Angeli, S.~Borini and R.~Cimiraglia,
\newblock {\em Theor.~Chem.~Acc.}, {\em submitted.}
\end{list}
}}
\end{center}

\section{Input description}\label{sec:nevpt2input}

NEVPT2 must follow a CAS--SCF or CAS--CI calculation. The keyword
\Key{NEVPT2} has to be specified in the \Sec{*WAVE FUNCTIONS} data
section. Furthermore a small {*NEVPT2} data group
 can be specified preceding the few input data that can
optionally be provided by the user:
\Key{THRESH},
the threshold to discard small coefficients in the CAS wavefunction
(default = 0.0),
\Key{FROZEN},
a vector specifying for each symmetry the core orbitals which are
excluded from the correlation treatment (the default is no
freezing) and
\Key{STATE},
the state in a CASCI calculation. This keyword is unnecessary
(ignored) in the CASSCF case.

An example could be the following:

\begin{verbatim}
**DALTON INPUT
**WAVE FUNCTIONS
.MCSCF
.NEVPT2
*NEVPT2
.THRESH
1.0D-12
.FROZEN
 1 0 1 0
*END OF INPUT
\end{verbatim}

%%%%%%%%%%%%%%%%%%%%%%%%%%%%%%%%%%%%%%%%%%%%%%%%%%%%%%%%%%%%%%%%%%%%
\chapter{Examples of coupled cluster calculations}
\label{ch:ccexamples}
%%%%%%%%%%%%%%%%%%%%%%%%%%%%%%%%%%%%%%%%%%%%%%%%%%%%%%%%%%%%%%%%%%%

We collect in this Chapter a few examples of input files (\verb|DALTON.INP|) 
for calculations one might want to carry 
out with the \cc\ modules of the Dalton program.
Other examples may be found in the test suite. 

It should be stressed that %in difference to the \resp \ code,
all modules in \cc \ can be used 
simultaneously---assuming they are all implemented
for the chosen wavefunction model(s).
For instance,  linear, quadratic and cubic response functions 
can be obtained within the same calculation.

\section{Multiple model energy calculations}
%
\cc\  allows for the calculation of the ground state energy
of the given system using a variety of wavefunction models, listed in 
Section~\ref{sec:ccgeneral}. 
For any model specified the Hartree--Fock energy is always calculated. 
The following input describes the calculation of SCF, MP2, CCSD and 
CCSD(T) ground state energies:
%
\begin{verbatim}
**DALTON INPUT
.RUN WAVE FUNCTIONS
**WAVE FUNCTIONS
.CC
*CC INPUT
.MP2              
.CCSD
.CC(T)
**END OF DALTON INPUT
\end{verbatim}
Note that SCF, MP2 and CCSD energies are obtained by default if 
CCSD(T) (\Key{CC(T)}) is required. Therefore the 
keywords \Key{MP2} and \Key{CCSD}
may also be omitted in the previous example.

\section{First-order property calculation}

The following input exemplifies the calculation of all orbital-relaxed
first-order one-electron properties available in \cc, for the hierarchy of 
wavefunction models SCF (indirectly obtained through CCS), MP2, CCSD and CCSD(T).
For details, see Section~\ref{sec:ccfop}.
\begin{verbatim}
**DALTON INPUT
.RUN WAVE FUNCTIONS
**INTEGRALS
.DIPLEN
.SECMOM
.THETA
.EFGCAR
.DARWIN
.MASSVELO
**WAVE FUNCTIONS
.CC
*CC INPUT
.CCS              (gives SCF First order properties)
.MP2              (default if CCSD is calculated)
.CCSD
.CC(T)
*CCFOP
.ALLONE
**END OF DALTON INPUT
\end{verbatim}

\section{Static and frequency-dependent dipole polarizabilities and
            corresponding dispersion coefficients} 

The following input describes the calculation of the 
electric dipole polarizability component $\alpha_{zz}(\omega)$, 
for $\omega = 0.00$ and $\omega=0.072$ au, and its dispersion
coefficients up to order 6, in the hierarchy of CC models
CCS, CC2 and CCSD. 
The calculation of the electric dipole moment has been
included in the same run.

\begin{verbatim}
**DALTON INPUT
.RUN WAVE FUNCTIONS
**INTEGRALS
.DIPLEN
**WAVE FUNCTIONS
.CC
*CC INPUT
.CCS              
.CC2             
.CCSD
*CCFOP
.DIPMOM
*CCLR
.OPERATOR
ZDIPLEN ZDIPLEN
.FREQUENCIES
  2
0.00  0.072
.DISPCF
  6
**END OF DALTON INPUT
\end{verbatim}

\section{Static and frequency-dependent dipole hyperpolarizabilities 
            and corresponding dispersion coefficients} 

The previous input can be extended to include the 
calculation of the electric first and second hyperpolarizability components 
$\beta_{zzz}(\omega_1,\omega_2)$ and 
$\gamma_{zzzz}(\omega_1,\omega_2,\omega_3)$, 
for $\omega_1 = \omega_2 =(\omega_3 =) \ 0.00$ 
and $\omega_1 = \omega_2 =(\omega_3 =) \ 0.072$ au. 
The corresponding dispersion coefficients up to sixth order 
are also calculated.
For other specific cases, see Sections~\ref{sec:ccqr} and \ref{sec:cccr}

\begin{verbatim}
**DALTON INPUT
.RUN WAVE FUNCTIONS
**INTEGRALS
.DIPLEN
**WAVE FUNCTIONS
.CC
*CC INPUT
.CCS
.CC2
.CCSD
*CCFOP
.DIPMOM
*CCLR
.OPERATOR
ZDIPLEN ZDIPLEN
.FREQUENCIES
  2
0.00  0.072
.DISPCF
  6
*CCQR
.OPERATOR
ZDIPLEN ZDIPLEN ZDIPLEN
.MIXFRE
  2
0.00  0.072                      !omega_1
0.00  0.072                      !omega_2
*CCCR
.OPERATOR
ZDIPLEN ZDIPLEN ZDIPLEN ZDIPLEN
.MIXFRE
  2
0.00  0.072                      !omega_1
0.00  0.072                      !omega_2
0.00  0.072                      !omega_3
.DISPCF
 6
**END OF DALTON INPUT
\end{verbatim}
Obviously, linear, quadratic and cubic response modules can also be
run separately. 

\section{Excitation energies and oscillator strengths}
This is an example for the calculation of singlet excitation energy and
oscillator strength for a system with \verb+NSYM = 2+ (C$_s$).
\begin{verbatim}
**DALTON INPUT
.RUN WAVE FUNCTIONS
**INTEGRAL
.DIPLEN
**WAVE FUNCTIONS
.CC
*CC INPUT
.CCSD
.NSYM
 2
*CCEXCI
.NCCEXCI                !number of excited states
 2 1                    !2 states in symmetry 1 and 1 state in symmetry 2
*CCLRSD
.DIPOLE
**END OF DALTON INPUT
\end{verbatim}
Triplet excitation energies can be obtained adding
an extra line to \Key{NCCEXCI} specifying the number 
of required triplet excited states for each symmetry class.  
Note however that linear residues are not available
for triplet states (The \Sec{CCLRSD} sections should be removed).

\section{Gradient calculation, geometry optimization}
Available for CCS, CC2, MP2, CCSD and CCSD(T) using integral direct analytic gradient.

\noindent For a single integral-direct gradient calculation:
\begin{verbatim}
**DALTON INPUT
.DIRECT
.RUN WAVE FUNCTIONS
**INTEGRAL
.DIPLEN
.DEROVL
.DERHAM
**WAVE FUNCTIONS
.CC
*CC INP
.CCSD
*DERIVATIVES
**END OF DALTON INPUT
\end{verbatim}
Note that if several wavefunction models are specified, 
the gradient calculation is performed only for the ``lowest-level" 
model in the list.

\noindent For a geometry optimization:
\begin{verbatim}
**DALTON INPUT
.OPTIMIZE
**INTEGRAL
.DIPLEN
.DEROVL
.DERHAM
**WAVE FUNCTIONS
.CC
*CC INPUT
.CC(T)
**END OF DALTON INPUT
\end{verbatim}

%\section{Excited state geometry optimization}
\section{R12 methods}

\noindent At present available at the MP2 level. The input for
an MP2-R12/A calculation is as follows (the key \Key{AUXBAS} is used only if
an auxiliary basis is employed):
\begin{verbatim}
**DALTON INPUT
.DIRECT
.RUN WAVE FUNCTIONS
**INTEGRAL
.R12
.AUXBAS
**WAVE FUNCTIONS
.CC
*CC INPUT
.MP2
*R12
.NO A'
.NO B
**END OF DALTON INPUT
\end{verbatim}

\part{\lsdalton\ Reference Manual}

\chapter{List of \lsdalton\ keywords}\label{ch:keywords}

In this chapter, we describe all keywords for the different sections
of the LSDALTON.INP file. In general, LSDALTON.INP is divided into the following sections under
the headlines:
\begin{itemize}
\item **GENERAL contains general settings (optional)
\item **INTEGRAL contains settings for the calculation of integrals (optional)
\item **WAVE FUNCTION contains information about the wave function (e.g. HF/DFT) and settings
for the optimization of the wave function (mandatory)
\item **OPTIMIZE contains settings for geometry optimization (optional)
\item **DYNAMI contains settings for Born-Oppenheimer molecular dynamics (optional)
\item **LOCALIZE ORBITALS contains settings for orbital localization procedure (optional)
\item **RESPONS contains information about requested molecular properties for HF and DFT (optional)
\item **DEC \emph{or} **CC contains info about MP2 and coupled-cluster calculations (optional)
\item **PLT contains information about construction of *.plt files which may be used to visualize densities and orbitals using e.g. the Chimera program~\cite{chimera}.
\item **PLTGRID contains information about grid used for construction of *.plt files.
\end{itemize}
Each of these sections may contain subsections, indicated by a single asterisk. LSDALTON.INP
should always end with *END OF INPUT. 

\section{**GENERAL}\label{sec:general}

This input module defines the way the integrals should be calculated.
\begin{description}

\item[\Key{SCALAPACK}] Use SCALAPACK matrices. \newline
NB: Requires linking to the scalapack lib (provided by MKL).
\item[\Key{SCALAPACKBLOCKSIZE}] the blocksize used in SCALAPACK matrices
\item[\Key{CSR}] Use Compressed-Sparse Row matrices. NB: Requires linking to MKL library!
\item[\Key{TIME}] Activate the printout of timings.
\item[\Key{NOGCBASIS}] Deactivate the use of the Grand Canonical basis \cite{trilevel1, trilevel2}
\item[\Key{FORCEGCBASIS}] The Grand Canonical basis \cite{trilevel1, trilevel2} is deactivated when the program
detects the use of a Dunning basis set like (cc-pVXZ) but the calculation will force the use of the Grand Canonical basis using this keyword.

\end{description}
\section{**INTEGRAL basic keywords}\label{sec:integral}

This input module defines the way the integrals should be calculated.
\begin{description}

\item[\Key{ADMM}] Use the Auxiliary Density Matrix Method (ADMM)~\cite{ADMM:2010} to approximate 
the exact-exchange contribution. When using this keyword, an auxiliary basis set has to be 
specified in the \mol\ file such as:
\begin{verbatim}
1     BASIS
2     6-31+G* Aux=df-def2 CABS=STO-2G JK=3-21G
      ...
\end{verbatim} 
where the "JK" basis is used as ADMM auxiliary basis set. You also have to provide a valid 
"CABS" basis from the library (which is needed for technical reasons, but will not actually 
be used in the calculation). \emph{Note of caution: we do not recommend to use ADMM in combination 
with geometry optimization.}

\item[\Key{AOPRINT}] 
\verb| | \newline
\verb|<Print level>|\newline
Print the Atomic Orbital information. The higher the print level, the more information will be printed. Default value is 0.
\item[\Key{BASPRINT}] 
\verb| | \newline
\verb|<Print level>|\newline
Print the basis set information. The higher the print level, the more information will be printed.
Using a print level of 6 will print the input exponents and contraction coefficients read from file as well as the normalized basis used in LSDALTON.  
\item[\Key{DENSFIT}] Use density-fitting for the Coulomb contribution (no exact exchange density-fitting method have been introduced). 
When using this keyword, an auxiliary basis set has to be 
specified in the \mol\ file. For the Poples basis sets we recommend df-def2 and for
Dunnings basis sets use the corresponding cc-pV$X$Zdenfit basis sets, with $X=$T$,$Q$,5$.
\item[\Key{INTPRINT}] 
\verb| | \newline
\verb|<Print level>|\newline
Print information about the actual integral evaluation. The higher the print level, the more information will be printed. Default value is 0.
\item[\Key{LINSCAPRINT}] 
\verb| | \newline
\verb|<Print level>|\newline
Print level in the integral code. The higher the print level, the more information will be printed. This keyword corresponds to 
\begin{verbatim}
.AOPRINT
<Print level>
.BASPRINT
<Print level>
.MOLPRINT
<Print level>
.INTPRINT
<Print level>
\end{verbatim} 
See the individual keywords for more details. Default value is 0.
\item[\Key{MOLPRINT}] 
\verb| | \newline
\verb|<Print level>|\newline
Print information about the molecule. The higher the print level, the more information will be printed. Default value is 0.
\item[\Key{NOJENGINE}] Turn off J-engine algorithm (which is default) for the calculation of the Coulomb matrix 
(see Refs. \cite{shao:425} and \cite{shao:6572}). Compute instead Coulomb-like contributions from the explicitly calculated integrals.
\item[\Key{NOLINK}] Turn off the LinK algorithm (which is default) for the calculation of the exchange matrix (see Ref. \cite{ochsenfeld:1663}).
\item[\Key{NO SCREEN}] Deactivate the use of all screening methods.
\item[\Key{NOFAMILY}] Deactivate the exploitation of family type basis set, basis set sharing exponents for different angular momentums.  
\item[\Key{PARI}] Use the Pair-Atomic Resolution-of-the-Identity
approximation \cite{PARI:2013} for the Coulomb and exact Exchange contribution.
When using this keyword, an auxiliary basis set has to be 
specified in the \mol\ file.\newline
\emph{We do not recommend to use this method for any production runs, and have included it 
in case other method developers would like to explore it (see the reference for details).}
\item[\Key{RUNMM}] Use the (Fast) Multipole Method.
\item[\Key{THRESH}]
\verb| | \newline
\verb|<Threshold>|\newline
An overall screening threshold for integral evaluation. The default value is $10^{-8}$

The various integral-evaluation thresholds (below) are set according to this Threshold. 
\begin{itemize}
\item The Cauchy-Schwarz screening threshold is set equal to Threshold (can be set separately by \Key{THR\_CS})
\begin{itemize}
  \item The Screening threshold used for Coulomb is Threshold $\cdot 10^{-2}$ (Default: $10^{-10}$) 
  \item The Screening threshold used for Exchange is Threshold $\cdot 10^{-0}$ (Default: $10^{-8}$)
  \item The Screening threshold used for One-electron operators is Threshold $\cdot 10^{-7}$ (Default: $10^{-15}$)
\end{itemize}
\item The Primitve Cauchy-Schwarz screening threshold is set equal to Threshold$\cdot 10^{-1}$  (can be set separately by \Key{THR\_PS})
\item The Overlap-distribution distance-screening threshold is set equal to 
      Threshold $\cdot 10^{-1}$.
      When calculating overlap integrals, this threshold 
      is used for setting up AO extents. Overlap distributions (ODs) for which the 
      distance between the two AOs are larger than the sum of the extents are 
      screened away. %(OD screening)
\item The Overlap-distribution extent-screeing threshold is set equal to 
      Threshold $\cdot 10^{-1}$. When calculating overlap integrals, this threshold 
      is used for setting up OD extents. Overlap integrals between two ODs separated
      by more than the sum of the extents are screened away. %(OE screening)
\item The FMM threshold is used to distinguish the Coulomb repulsion into classical and non-classical interactions, based on
      the non-classical extent of the contineous charge distribution (orbitals or orbital products). By default this threshold 
      is set equal to Threshold $\cdot 10^{-1}$ (can be set separately by \Key{SCREEN} under *FMM)
\end{itemize}
\end{description}

\section{**INTEGRAL advanced keywords}\label{sec:integral}

This input module defines the way the integrals should be calculated. 
This section contains some advanced keywords which is mostly usefull 
for debugging or comparisons with other codes. 
\begin{description}

\item[\Key{NO CS}] Deactivate the use of Cauchy-Schwarz screening. 
Note that this does not deactivate the Primitive Cauchy-Schwarz screening. 
\item[\Key{NO PS}] Deactivate the use of primitive Cauchy-Schwarz screening.
\item[\Key{THR\_CS}] 
\verb| | \newline
\verb|<CS_Threshold>|\newline
Cauchy-Schwarz screening threshold. The default value is $10^{-8}$. Note that 
\begin{itemize}
  \item The Screening threshold used for Coulomb is CS\_Threshold $\cdot 10^{-2}$ (Default: $10^{-10}$) 
  \item The Screening threshold used for Exchange is CS\_Threshold $\cdot 10^{-0}$ (Default: $10^{-8}$)
  \item The Screening threshold used for One-electron operators is CS\_Threshold $\cdot 10^{-7}$ (Default: $10^{-15}$)
\end{itemize}
\item[\Key{THR\_PS}] 
\verb| | \newline
\verb|<PS_Threshold>|\newline
Primitive Cauchy-Schwarz screening threshold. The default value is $10^{-9}$
\item[\Key{CART-E}] Use cartesian E-coefficients instead of hermite E-coefficients for the McMurchie-Davidson
integral-evaluation scheme \cite{McMurchie1978} used in \lsdalton. The use of Cartesian E-coefficients is 
default in most integral programs. We however use hermite E-coefficients according to ref.~\cite{reine:4771}.
\item[\Key{FTUVMAXPRIM}] 
\verb| | \newline
\verb|<Maxprim>|\newline
Sets the maximum number of primitives used in the FTUV batches in the J-engine algorithm.
\item[\Key{LOW RJ000 ACCURACY}] Use a decreased accuracy for the calculation of the Boys function.
\item[\Key{MAXPASSES}] 
\verb| | \newline
\verb|<maxpass>|\newline
Change the maximum number of collected overlap distributions (default is 40).
\item[\Key{NO PASS}] Deactivate the use of Passes, the collection of overlap distributions which is used as default in order to increase efficiency. 
\item[\Key{NOSEGMENT}] Deactivate the use of segments. The use of segments is used to reduced the numer of primitive functions in the calculation of integrals. The basis is considered to be a general contracted basis where all primitives contribute to all contracted functions. 
\item[\Key{NSETUV}] Use non-spherical E-coefficients.
\item[\Key{OVERLAP-DF-J}] Use the Overlap density fitting algorithm for the calculation of the Coulomb matrix.
%\item[\Key{PARI}] Use pair-atomic-resolution-of-the-identity algorithm for the calculation of the Fock matrix
\item[\Key{UNCONT}] Treat the basis functions (given by input) as fully uncontracted basis fuctions (i.e. ignore
any contraction coefficients given in the basis and treat instead all the primitives as separate basis functions).
\end{description}

\subsection{*FMM}\label{subsec:denfit}

This input block is used to specify settings for the (continuous) fast multipole moment (FMM) treatment of 
classical Coulomb interactions.\newline
% comment by A. Krapp for FMM gradients. 
%
% Meant as a reminder to those who will rewrite/restructure/optimise the FMM code
%
For gradients OpenMP parallelisation is turned off. The reason for this is that a continous counter used 
to identify moments for orbital products has to be the same in both moments and derivative moments. 
For the same reason screening [only during writing (see printmm routines)!!] is turned off both for the serial 
and parallel cases. As a consequence of the screeening been turned off the moments have to be recalculated for the gradient and can not be reused from the preceeding energy evaluation. 
%
% a solution to these problems could be to calculate moments and derivatives moments at the same time as was
% done in the FCK3 code.
%
The gradient implementation has been tested both for regular and density fitting Coulomb interactions. 
It works for the combined N-e +  e-e + N-N interactions (default), 
and for the e-e interaction only (invoked by \Key{NOONE}).
It has however not been tested for speed and efficiency.
%
% end of comment 
\begin{description}
\item[\Key{LMAX}] \verb| |\newline
\verb|<Lmax>|\newline
The maximum multipole-moment expansion order of a given orbital product. Default Value is 8.
\item[\Key{TLMAX}] \verb| |\newline
\verb|<TLmax>|\newline
The maximum order used for translated multipole-moments. Default Value is 20.
\item[\Key{SCREEN}] \verb| |\newline
\verb|<Threshold>|\newline
The FMM threshold is used to distinguish the Coulomb repulsion into classical and non-classical interactions, based on the non-classical extent of the contineous charge distribution (orbitals or orbital products). 
By default this threshold is set equal to $\cdot 10^{-9}$ 
Defines the threshold used to determine if a contribution can be calculated classically or non-classically.
\item[\Key{NOONE}] Do not use FMM for the nuclear-electron attraction.
\item[\Key{NOMMBU}] Do not use io-buffers for interfacing multipole-moments to the FMM driver.
\end{description}

\section{**WAVE FUNCTION}\label{sec:wavefunc}
\begin{description}
\item[\Key{HF}] Hartree-Fock calculation.
\item[\Key{DFT}] \verb| | \newline
\verb|<FUNCTIONAL>|\newline
DFT calculation. The exchange---correlation functional is read from the following line.

Functionals in \lsdalton\ can be divided into two groups: generic and combined functionals. 
Combined functionals are a linear combination of generic ones. 
One can always create own combined functionals by using GGAKey general functional.
A number of standalone functionals are also included within \lsdalton. 

It should be noted that the input is not case sensitive, although the notation
employed in this manual makes use of case to exphasise exchange or correlation 
functional properties and reflect the original literature sources.

Supported functionals are: 
\subsubsection{Exchange Functionals}
\providecommand\exfn[1]{#1}

\begin{description}
\item[Slater] Dirac-Slater exchange functional \cite{dft:hohenberg,dft:kohn,dft:slater}.\index{Slater}

\item[Becke] 1988 Becke exchange GGA correction \cite{dft:becke88}. Note that the full Becke88 exchange functional is given as \exfn{Slater} + \exfn{Becke}.\index{Becke}

%\item[mBecke] 1998 modified \exfn{Becke} exchange correction presented in reference \cite{dft:edf1} for use in the EDF1 functional. The $\beta$ value of 0.0042 in \exfn{Becke} is changed to 0.0035.\index{mBecke}\index{EDF1}

%\item[B86] Becke 1986 exchange functional, a divergence free, semi-empirical gradient-corrected exchange functional~\cite{dft:b86,dft:b86r}.\index{B86} This functional corresponds to the B86R functional of the Molpro program.

%\item[B86mx] B86 exchange functional modified with a gradient correction for large density gradients~\cite{dft:b86mgc}.\index{B86mx}

%\item[DBx] Double-Becke exchange functional defined in 1998 by Gill et al.\cite{dft:edf1,dft:edf2} for use in the EDF1 functional. The full DBx functional is defined as 1.030952*\exfn{Slater} - 8.44793*\exfn{Becke} + 10.4017*\exfn{mBecke} \index{EDF1}\index{Double-Becke}\index{Becke}

%\item[DK87x] DePristo and Kress' 1987 rational function GGA exchange functional (equation 7) from Ref. \cite{dft:dk87}.\index{DK87} The full exchange functional is defined as \exfn{Slater} + \exfn{DK87x}.

%\item[G96x] Gill's 1996 GGA correction exchange functional~\cite{dft:g96}.\index{G96} The complete exchange functional is given by \exfn{Slater} + \exfn{G96x}.

%\item[LG93x] 1993 GGA exchange functional~\cite{dft:lg93,dft:g961lyp}.\index{LG93} The full LG93 exchange functional is given by \exfn{Slater} + \exfn{LG93x} 

%\item[LRC95x] 1995 GGA exchange functional with correct asymptotic behavior~\cite{dft:lrc95}.\index{LRC95} The LRC95x exchange functional includes the Slater exchange (Eq 6 from original reference).

\item[KTx] Keal and Tozer's 2003 GGA exchange functional (With x being 1,2 or 3). Note that the gradient correction pre-factor constant, $\gamma$, is not included in the KT exchange 
  definition, but rather in the KT1, KT2 and KT3 definitions. The full KT exchange is given by \cite{dft:kt12}\index{KT}, \exfn{Slater} + $\gamma$\exfn{KTx} ($\gamma$ is -0.006 for KT1,KT2 and -0.004 for KT3). 

\item[OPTX] Handy's 2001 exchange functional correction \cite{dft:optx}.\index{OPTX} The full OPTX exchange functional is given by 1.05151*\exfn{Slater} - 1.43169*\exfn{OPTX}.

\item[PBEx] Perdew, Burke and Ernzerhof 1996 exchange functional~\cite{dft:pbe}.\index{PBEx}

%\item[revPBEx] Zhang and Wang's 1998 revised PBEx exchange functional, with $\kappa$ of 1.245 \cite{dft:revpbe}.\index{revPBE}\index{PBE}

%\item[RPBEx] Hammer, Hansen and N{\o}rskov's 1999 revised PBEx exchange functional \cite{dft:revpbe}.\index{RPBE}\index{PBE}

%\item[mPBEx] Adamo and Barone's 2002 modified PBEx exchange functional~\cite{dft:mpbe}.\index{mPBE}\index{PBE}

\item[PW86x] Perdew and Wang 1986 exchange functional (the PWGGA-I functional)\cite{dft:pw86x}.\index{PW86x}

%\item[PW91x] Perdew and Wang 1991 exchange functional (the pwGGA-II functional) and includes Slater exchange \cite{dft:pw91}.\index{PW86x} This functional is also given in a separate parameterization in Refs.~\cite{dft:g96,dft:mpw}, which is labeled PW91x2, and is defined as \exfn{PW91x} = \exfn{Slater} + \exfn{PW91x2}.

%\item[mPW] Adamo and Barone's 1998 modified PW91x GGA correction exchange functional ~\cite{dft:pw91,dft:mpw}. The full exchange functional is given by \exfn{Slater} + \exfn{mPW}.\index{mPW}

\end{description}

\subsubsection{Correlation Functionals}
\providecommand\corfn[1]{#1}
\begin{description}

\item[VWN3] correlation functional of Vosko, Wilk and Nusair, 1980 (equation III) 
  \cite{dft:vwn}. This is the form used in the Gaussian program.\index{VWN3}

\item[VWN5] correlation functional of Vosko, Wilk and Nusair, 1980 (equation
  V -- the recommended one). The VWN keyword is a synonym for VWN5~\cite{dft:vwn}.\index{VWN5}

%\item[FT97c] Filatov and Thiel 1997 (FT97) correlation functional
%  \cite{dft:ft97}.\index{FT97}

\item[LYP] correlation functional by Lee, Yang and Parr, 1988
  \cite{dft:lyp1,dft:lyp2}.\index{LYP}

%\item[LYPr] 1998 modified \corfn{LYP} functional, which is the re-parameterized EDF1 version 
%  with modified parameters (0.055, 0.158, 0.25, 0.3505)
%  \cite{dft:lyp1,dft:lyp2,dft:edf1}.\index{LYPr}\index{EDF1}

\item[P86c] non-local part of the correlation functional of the Perdew 1986 correlation functional
  \cite{dft:p86}. PZ81 (1981 Perdew local) is usually used for the local part of the
  functional, with a total corelation functional of \index{P86}\index{PZ81}
  \corfn{P86c} + \corfn{PZ81}.

\item[PBEc] Perdew, Burke and Ernzerhof 1996 correlation functional, 
  defined as PW91c local and PBEc non-local correlation~\cite{dft:pbe}.\index{PBEc}

\item[PW91c] 1991 correlation functional of Perdew and Wang (the pwGGA-II functional)
  ~\cite{dft:pw91}.\index{PW91} This functional includes both the PW91c non-local and 
  PW91c local (ie PW92c) contributions. The non-local PW91c contribution may be determined
  as \corfn{PW91c} - \corfn{PW92c}.

%\item[PW92c] local correlation functional of Perdew and Wang, 1992~\cite{dft:pw91,dft:pw92}.\index{PW91}
%  This functional is the local contribution to the PW91c correlation functional.

%\item[PW92ac] gradient correction to the PW91c correlation functional of Perdew and Wang,
%  equation 16 from Ref.~\cite{dft:pw91,dft:pw92}.\index{PW91} The PWGGA-IIA functional
%  as defined in the original reference is \corfn{PW91c} + \corfn{PW92ac}. 

\item[PZ81] local correlation functional of Perdew and Zunger, 1981~\cite{dft:pz81}.\index{PZ81}

%\item[Wigner] original 1938 spin-polarized correlation functional~\cite{dft:wigner}.\index{Wigner}

%\item[WL90c] Wilson and Levy's 1990 non-local spin-dependent correlation functional 
%  (equation 15 from Ref.~\cite{dft:wl90}).\index{WL90}

\end{description}

\subsubsection{Standalone Functionals}
\providecommand\onefn[1]{#1}
\begin{description}
\item[LB94] asymptotically correct functional of Leeuwen and
  Baerends 1994~\cite{dft:lb94}. This functional improves description of the
  asymptotic density on the expense of core and inner valence.\index{LB94}

%\item[B97] Becke 1997 functional~\cite{dft:b97}.\index{B97}

%\item[B97-1] Hamprecht et al.'s 1998 re-parameterization of the 
%  B97 functional~\cite{dft:b97-1}.\index{B97}\index{B97-1}

%\item[B97-2] Modification of B97 functional in 2001 by Wilson, Bradley and Tozer 
%  \cite{dft:b97-2}.\index{B97}\index{B97-2}

%\item[B97-K] Boese and Martin 2004 re-parameterization of the 
%  B97-1 functional for kinetics~\cite{dft:b97-1}.\index{B97}\index{B97-K}

%\item[HCTH] is a synonym for the HCTH407 functional (detailed below).
%  \cite{dft:hcth407}.\index{HCTH}\index{HCTH407}

%\item[HCTH93] Original 1998 HCTH functional, parameterized on a set of 
%  93 training systems~\cite{dft:hcth93}.\index{HCTH}\index{HCTH93}

%\item[HCTH120] The HCTH functional parameterized on a set of 120 training systems 
%  in 2000~\cite{dft:hcth120}.\index{HCTH}\index{HCTH120}

%\item[HCTH147] The HCTH functional parameterized on a set of 147 training systems 
%  in 2000~\cite{dft:hcth120}.\index{HCTH}\index{HCTH147}

%\item[HCTH407] The HCTH functional parameterized on a set of 407 training systems 
%  in 2001~\cite{dft:hcth407}.\index{HCTH}\index{HCTH407}

%\item[HCTH407p] The HCTH407 functional re-parameterized in 2003 on a set of 407 
%  training systems and ammonia dimer to incorporate hydrogen bonding 
%  \cite{dft:hcth407p}.\index{HCTH}\index{HCTH407}\index{HCTH407p}

\end{description}

\subsubsection{Combined functionals}
\providecommand\funexample[1]{\\{\tt #1 }}
\begin{description}

\item[GGAKey] is a universal keyword allowing users to manually
  construct arbitrary linear combinations of exchange and correlation 
  functionals from the list above. Even fractional 
  Hartree--Fock exchange can be specified. This keyword is to be 
  followed by a string of functionals with associated weights. 
  The syntax is \verb|NAME=WEIGHT ...|. 
  As an example, B3LYP may be constructed as:
\begin{verbatim}
.DFT
 GGAKey HF=0.2 Slater=0.8 Becke=0.72 LYP=0.81 VWN=0.19
\end{verbatim}

The following GGA and hybrid functional aliases are defined within 
\lsdalton\ and provide further examples of the GGAKey keyword.

\item[SVWN5] is a sum of Slater functional and VWN (or VWN5) correlation
  functional. SVWN is a synonym for SVWN5. It is equivalent to
  \funexample{GGAKey Slater=1 VWN5=1}
  \index{SVWN}

\item[SVWN3] is a sum of the Slater exchange functional and VWN3 correlation
  functional. It is equivalent to the Gaussian program LSDA functional 
  and can alternatively be selected by following set of keywords
  \funexample{GGAKey Slater=1 VWN3=1}
  \index{SVWN3}

\item[LDA] A synonym for SVWN5 (or SVWN). \index{LDA}

%\item[BVWN] is a sum of the \exfn{Slater} functional, \exfn{Becke} correction and 
%  \corfn{VWN} correlation functional.  It is equivalent to 
%  \funexample{GGAKey Slater=1 Becke=1 VWN=1}
%  \index{BVWN}

\item[BLYP] is a sum of Slater functional, Becke88 correction and LYP
  correlation functional.  It is equivalent to 
  \funexample{GGAKey Slater=1 Becke=1 LYP=1}
  \index{BLYP}

\item[B3LYP] 3-parameter hybrid functional \cite{dft:b3lyp} equivalent to:
  \funexample{GGAKey HF=0.2 Slater=0.8 Becke=0.72 LYP=0.81 VWN=0.19}
  \index{B3LYP}

\item[B3LYP-G] hybrid functional with VWN3 form used for
  correlation---this is the form used by the Gaussian quantum chemistry
  program. Keyword B3LYPGauss is a synonym for B3LYPg.\index{B3LYPG} 
  This functional can be explicitely set up by
  \funexample{GGAKey HF=0.2 Slater=0.8 Becke=0.72 LYP=0.81 VWN3=0.19}
  \index{B3LYP, Gaussian version}

%\item[B1LYP] 1-parameter hybrid functional with 25\% exact exchange \cite{dft:b1lyp}. 
%  Equivalent to: \funexample{GGAKey HF=0.25 Slater=0.75 Becke=0.75 LYP=1}
%  \index{B3LYP}

\item[BP86] Becke88 exchange functional and Perdew86 correlation
  functional (with Perdew81 local correlation). The explicit form is:
  \funexample{GGAKey Slater=1 Becke=1 PZ81=1 P86c=1}
  \index{BP86}

\item[B3P86] variant of \verb|B3LYP| with VWN used for local
  correlation  and P86 for the nonlocal part.
  \funexample{GGAKey HF=0.2 Slater=0.8 Becke=0.72 P86c=0.81 VWN=1}
  \index{B3P86}

\item[B3P86-G] variant of \verb|B3LYP| with VWN3 used for local
  correlation and P86 for the nonlocal part.
  This is the form used by the Gaussian quantum chemistry program.
  \funexample{GGAKey HF=0.2 Slater=0.8 Becke=0.72 P86c=0.81 VWN3=1}
  \index{B3P86}\index{B3P86, Gaussian version}

\item[BPW91] Becke88 exchange functional and PW91 correlation
  functional. The explicit form is:
  \funexample{GGAKey Slater=1 Becke=1 PW91c=1}
  \index{BPW91}

%\item[B3PW91] 3-parameter Becke-PW91 functional, with PW91 correlation 
%  functional. Note that PW91c includes PW92c local correlation, thus only
%  excess PW92c local correlation is required (coefficient of 0.19).
%  \funexample{GGAKey HF=0.2 Slater=0.8 Becke=0.72 PW91c=0.81 PW92c=0.19}
%  \index{B3PW91}

%\item[B1PW91] 1-parameter hybrid functional \cite{dft:b1lyp} equivalent to:
%  \funexample{GGAKey HF=0.25 Slater=0.75 Becke=0.75 PW91c=1}
%  \index{B1PW91}

%\item[B86VWN] is a sum of \exfn{Slater} and \exfn{B86x} exchange functionals and 
%  the \corfn{VWN} correlation functional. It is equivalent to 
%  \funexample{GGAKey Slater=1 B86x=1 VWN=1}
%  \index{B86VWN}

%\item[B86LYP] is a sum of \exfn{Slater} and \exfn{B86x} exchange functionals and 
%  the \corfn{LYP} correlation functional. It is equivalent to 
%  \funexample{GGAKey Slater=1 B86x=1 LYP=1}
%  \index{B86LYP}

%\item[B86P86] is a sum of \exfn{Slater} and \exfn{B86x} exchange functionals and 
%  the \corfn{P86c} correlation functional. It is equivalent to 
%  \funexample{GGAKey Slater=1 B86x=1 P86c=1}
%  \index{B86P86}

%\item[B86PW91] is a sum of \exfn{Slater} and \exfn{B86x} exchange functionals and 
%  the \corfn{PW91c} correlation functional.  It is equivalent to 
%  \funexample{GGAKey Slater=1 B86x=1 PW91c=1}
%  \index{B86PW91}

%\item[BHandH] is an simple Half-and-half functional.
%  \funexample{GGAKey HF=0.5 Slater=0.5 LYP=1}
%  \index{BHandH}

%\item[BHandHLYP] is another simple Half-and-half functional.
%  \funexample{GGAKey HF=0.5 Slater=0.5 Becke=0.5 LYP=1}
%  \index{BHandH}

%\item[BW] is the sum of the Becke exchange and Wigner correlation
%  functionals \cite{dft:wigner,dft:bw}.\index{BW}
%  \funexample{GGAKey Slater=1 Becke=1 Wigner=1}

\item[CAMB3LYP] Coulomb Attenuated Method Functional of Yanai, Tew and
Handy \cite{dft:camb3lyp}. This functional accepts additional arguments
\verb|alpha|, \verb|beta| and \verb|mu| to modify the fraction of HF
exchange for short-range interactions, additional fraction of HF
exchange for long-range interaction and the interaction switching
factor $\mu$. This input can be specified as follows:
\begin{verbatim}
.DFT
 CAMB3LYP alpha=0.190 beta=0.460 mu=0.330
\end{verbatim}
\index{CAMB3LYP}

%\item[DBLYP] is a sum of the Double-Becke exchange functional and
%  the LYP correlation functional 
%  \cite{dft:becke88,dft:edf1,dft:lyp1,dft:lyp2}.\index{Double-Becke}
% \funexample{GGAKey Slater=1.030952 Becke=-8.44793 mBecke=10.4017 LYP=1}

%\item[DBP86] is the sum of the Double-Becke exchange functional and
%  the P86 correlation functional \cite{dft:becke88,dft:edf1,dft:p86}.\index{Double-Becke}
% \funexample{GGAKey Slater=1.030952 Becke=-8.44793 mBecke=10.4017 P86c=1 PZ81=1}

%\item[DBPW91] is a sum of the Double-Becke exchange functional and
%  the PW91 correlation functional \cite{dft:becke88,dft:edf1,dft:pw91}.\index{Double-Becke}
% \funexample{GGAKey Slater=1.030952 Becke=-8.44793 mBecke=10.4017 PW91c=1}

%\item[EDF1] is a fitted functional of Adamson, Gill and Pople \cite{dft:edf1}.
%  It is a linear combination of the Double-Becke exchange functional and the revised LYP
%  functional LYPr.\index{EDF1}
% \funexample{GGAKey Slater=1.030952 Becke=-8.44793 mBecke=10.4017 LYPr=1}

%\item[EDF2] is a linear combination of the Hartree-Fock exchange and the Double-Becke 
%  exchange, Slater exchange, LYP correlation, revised LYPr correlation and VWN 
%  correlation functionals \cite{dft:edf2}\index{EDF2}.
% \funexample{GGAKey HF=0.1695 Slater=0.2811 Becke=0.6227 mBecke=-0.0551 VWN=0.3029 LYP=0.5998 LYPr=-0.0053}

%\item[G96VWN] is the sum of the G96 exchange functional and the VWN 
%  correlation functional \cite{dft:g96}.
% \funexample{GGAKey Slater=1 G96x=1 VWN=1}

%\item[G96LYP] is the sum of the G96 exchange functional and the LYP
%  correlation functional \cite{dft:g96}.
% \funexample{GGAKey Slater=1 G96x=1 LYP=1}

%\item[G96P86] is the sum of the G96 exchange functional and the P86
%  correlation functional \cite{dft:g96}.
% \funexample{GGAKey Slater=1 G96x=1 P86c=1}

%\item[G96PW91] is the sum of the G96 exchange functional and the PW91
%  correlation functional \cite{dft:g96}.
% \funexample{GGAKey Slater=1 G96x=1 PW91c=1}

%\item[G961LYP] is a 1-parameter B1LYP type functional with the exchange gradient 
%  correction provided by the G96x functional \cite{dft:g961lyp}.
% \funexample{GGAKey HF=0.25 Slater=0.75 G96x=0.75 LYP=1}

%\item[KMLYP] Kang and Musgrave 2-parameter hybrid functional with a mixture of
%  Slater and Hartree--Fock exchange and VWN and LYP correlation functionals.
%  \cite{dft:kmlyp}.
% \funexample{GGAKey HF=0.557 Slater=0.443 VWN=0.552 LYP=0.448}

\item[KT1] Slater-VWN5 functional with the KT GGA exchange correction
  \cite{dft:kt12,dft:kt12a}.\index{KT1}
 \funexample{GGAKey Slater=1 VWN=1 KT=-0.006}

\item[KT2] differs from KT1 only in that the weights of the Slater and
  VWN5 functionals are from an empirical fit (not equal to 1.0)
  \cite{dft:kt12,dft:kt12a}.\index{KT2}
  \funexample{GGAKey Slater=1.07173 VWN=0.576727 KT=-0.006}

\item[KT3] a hybrid functional of Slater, OPTX and KT exchange with the
  LYP correlation functional \cite{dft:kt3}. The explicit form is
  \funexample{GGAKey Slater=1.092 KT=-0.004 LYP=0.864409 OPTX=-0.925452}
  \index{KT3}

%\item[LG1LYP] is a 1-parameter B1LYP type functional with the exchange gradient
%  correction provided by the LG93x functional \cite{dft:g961lyp}.
% \funexample{GGAKey HF=0.25 Slater=0.75 LG93x=0.75 LYP=1}

%\item[mPWVWN] is the combination of mPW exchange and VWN correlation functionals 
%   \cite{dft:mpw,dft:vwn}.\index{mPWVWN}
%  \funexample{GGAKey Slater=1 mPW=1 VWN=1}.

%\item[mPWLYP] is the combination of mPW exchange and LYP correlation functionals 
%   \cite{dft:mpw,dft:vwn}.\index{mPWLYP}
%  \funexample{GGAKey Slater=1 mPW=1 LYP=1}.

%\item[mPWP86] is the combination of mPW exchange and P86 correlation functionals 
%   \cite{dft:mpw,dft:vwn}.\index{mPWP86}
%  \funexample{GGAKey Slater=1 mPW=1 P86c=1 PZ81=1}.

%\item[mPWPW91] is the combination of mPW exchange and PW91 correlation functionals 
%   \cite{dft:mpw,dft:pw91}.\index{mPWPW91}
%  \funexample{GGAKey Slater=1 mPW=1 PW91c=1}.

%\item[mPW3PW91] is a 3-parameter combination of mPW exchange and PW91 correlation
%  functionals, with the PW91 (PW92c) local correlation~\cite{dft:mpw}.\index{mPW3PW91}
%  \funexample{GGAKey HF=0.2 Slater=0.8 mPW=0.72 PW91c=0.81 PW92c=0.19}.

%\item[mPW1PW91] is a 1-parameter combination mPW exchange and PW91 correlation
%  functionals with 25\% Hartree--Fock exchange \cite{dft:mpw}.\index{mPW1PW91}
%  \funexample{GGAKey HF=0.25 Slater=0.75 mPW=0.75 PW91c=1}.

%\item[mPW1K] optimizes mPW1PW91 for kinetics of H abstractions, with 42.8\% Hartree--Fock
%  exchange \cite{dft:mpw1k}.\index{mPW1PW91}
%  \funexample{GGAKey HF=0.428 Slater=0.572 mPW=0.572 PW91c=1}.

%\item[mPW1N] optimizes mPW1PW91 for kinetics of H abstractions, with 40.6\% Hartree--Fock
%  exchange \cite{dft:mpw1n}.\index{mPW1N}
%  \funexample{GGAKey HF=0.406 Slater=0.594 mPW=0.594 PW91c=1}.

%\item[mPW1S] optimizes mPW1PW91 for kinetics of H abstractions, with 6\% Hartree--Fock
%  exchange \cite{dft:mpw1s}.\index{mPW1S}
%  \funexample{GGAKey HF=0.06 Slater=0.94 mPW=0.94 PW91c=1}.

\item[OLYP] is the sum of the OPTX exchange functional with the
  LYP correlation functional \cite{dft:optx,dft:lyp1,dft:lyp2}.
  \funexample{GGAKey Slater=1.05151 OPTX=-1.43169 LYP=1}
  \index{OLYP}

%\item[OP86] is the sum of the OPTX exchange functional with the
%  P86 correlation functional \cite{dft:optx,dft:p86}.
%  \funexample{GGAKey Slater=1.05151 OPTX=-1.43169 P86c=1 PZ81=1}
%  \index{OP86}

%\item[OPW91] is the sum of the OPTX exchange functional with the
%  PW91 correlation functional \cite{dft:optx,dft:pw91}.
%  \funexample{GGAKey Slater=1.05151 OPTX=-1.43169 PW91c=1}
%  \index{OPW91}

\item[PBE0] a hybrid functional of Perdew, Burke and Ernzerhof with
  0.25 weight of exact exchange, 0.75 of \verb|PBEx| exchange functional and
  the \verb|PBEc| correlation functional \cite{dft:pbe0}.
  Alternative aliases are PBE1PBE or PBE0PBE.\index{PBE0}
  \funexample{GGAKey HF=0.25 PBEx=0.75 PBEc=1}

\item[PBE] same as above but with exchange estimated exclusively by
  \exfn{PBEx} functional \cite{dft:pbe}.\index{PBE} Alias of PBEPBE. 
  This is the form used by CADPAC and NWChem quantum chemistry programs.
  \funexample{GGAKey PBEx=1 PBEc=1}\\
  Note that the Molpro quantum chemistry program uses the \corfn{PW91c}
  non-local correlation functional instead of \corfn{PBEc}, which is 
  equivalent to the following:
  \funexample{GGAKey PBEx=1 PW91c=1}.

%\item[RPBE] is a revised PBE functional that employs the 
%  \exfn{RPBEx} exchange functional.
%  \funexample{GGAKey RPBEx=1 PBEc=1}

%\item[revPBE] is a revised PBE functional that employs the 
%  \exfn{revPBEx} exchange functional.
%  \funexample{GGAKey revPBEx=1 PBEc=1}

%\item[mPBE] is a revised PBE functional that employs the 
%  \exfn{mPBEx} exchange functional.
%  \funexample{GGAKey mPBEx=1 PBEc=1}

%\item[PW91VWN] is the combination of PW91 exchange and VWN correlation functionals 
%   \cite{dft:pw91,dft:vwn}.\index{PW91}
%  \funexample{GGAKey PW91x=1 VWN=1}.

%\item[PW91LYP] is the combination of PW91 exchange and LYP correlation functionals 
%   \cite{dft:pw91,dft:lyp1,dft:lyp2}.\index{PW91}
%  \funexample{GGAKey PW91x=1 LYP=1}.

%\item[PW91P86] is the combination of PW91 exchange and P86 (with Perdew 1981 local) 
%  correlation functionals \cite{dft:pw91,dft:pw86,dft:pz81}.
%  \funexample{GGAKey PW91x=1 P86c=1 PZ81=1}.

%\item[PW91PW91] is the combination of PW91 exchange and PW91 correlation functionals. 
%  Equivalent to PW91 keyword \cite{dft:pw91}.
%  \funexample{GGAKey PW91x=1 PW91c=1}.

%\item[XLYP] is a linear combination of \exfn{Slater}, \exfn{Becke} and \exfn{PW91x}
%  exchange and \corfn{LYP} correlation functionals \cite{dft:xlyp,dft:x3lyp}.\index{XLYP}
%  \funexample{GGAKey Slater=1 Becke=0.722 PW91x=0.347 LYP=1}.

%\item[X3LYP] is a linear combination of Hartree--Fock, \exfn{Slater}, \exfn{Becke} 
%  and \exfn{PW91x} exchange and \corfn{VWN} and \corfn{LYP} correlation functionals 
%  \cite{dft:xlyp,dft:x3lyp}.\index{X3LYP}
%  \funexample{GGAKey HF=0.218 Slater=0.782 Becke=0.542 PW91x=0.167 VWN=0.129 LYP=0.871}

\end{description}

Note that combinations of local and non-local correlation functionals
can also be generated with the GGAKey keyword. For example,
\verb|GGAKey P86c=1 PZ81=1| combines the PZ81 local and P86c non-local 
correlation functional, whereas \verb|GGAKey VWN=1 P86c=1| 
combines the VWN local and P86 non-local correlation functionals.


Linear combinations of all exchange and correlation functionals listed above
are possible with the \verb|GGAKey| keyword.


\subsubsection{XCFUN functionals}

Recently some support have been added for using the functionals from the XCFUN lib \textbf{fixme: insert proper reference}. Full 
support is expected in the next release, but unfortunetly time constraints did not allow for 
full support in this release. 

Currently only LDA and GGA type functionals are currently supported and only upto and including 
linear response. the XCFUN functionals are activated using the .XCFUN keyword under the **INTEGRALS section (see Section \ref{sec:integral})
\begin{verbatim}
**INTEGRALS
.XCFUN
\end{verbatim} 
Linear combinations XCFUN functionals are possible with the \verb|GGAKey| keyword.

\end{description}

\subsection{*DFT INPUT}\label{subsec:dftinput}
Controls the XC-integration grid. 
The grid is generated as follows: 
For each atom a radial grid is created (different radial grids available, see below), 
which is multiplied with an angular grid (Lebedev-grids).
The angular grids are by default pruned for radial points close to the nucleus,  
following Murray, Handy and Laming.~\cite{dft-int:pruning} 
Grid weights for the grid points are calculated following different schemes (see below). 
Grid points with weights below $10^{-20}$ are disregarded.  
Please note that the grid construction routines in the \lsdalton\
program changed compared to the \dalton\ program which may result in small deviations in the number of grid points etc.
\begin{description}
\item[\Key{ANGINT}] \verb||\newline
\verb|<ANGINT>|\newline
Determines the quality of the angular Lebedev grid -- the angular
integration of spherical harmonics will be exact up to the specified
order. Default value is $31$. Maximum value is $64$.
Note that the value of \verb|ANGINT| is changed by the following keywords:
\verb|COARSE|, \verb|NORMAL|, \verb|FINE|, \verb|ULTRAFINE|.
The \verb|GRIDX| keywords also imply specific \verb|ANGINT| values.
%%
%%\item[\Key{ANGMIN}] 
%% ANGMIN IS NOT ACTIVE ANYMORE IN LSDALTON 
%% 
\item[\Key{COARSE}]
Shortcut keyword for radial integration accuracy $10^{-11}$ and angular expansion order equal to $35$.
\item[\Key{DFTELS}] \verb||\newline
\verb|<DFTELS>|\newline
safety threshold -- stop if the charge integration error becomes larger then this threshold.
\item[\Key{DFTTHR}]\verb||\newline
\verb|<DFTHR0, DFTHRL, DFTHRI, RHOTHR>|
\begin{description}
\item[DFTHR0] is not used 
\item[DFTHRL] is not used 
\item[DFTHRI] threshold for screening product of gaussian atomic orbitals
\item[RHOTHR] threshold for screening of the electron density 
\end{description}
\item[\Key{DISPER}] 
Activates the addition of the empirical dispersion correction following S. Grimme.~\cite{dft-disp:Grimme1,dft-disp:Grimme2} The S6 factors and van der Waals radii of Ref.~\cite{dft-disp:Grimme2} are used.
Implemented for the functionals BP86, BLYP, PBE, B3LYP for energies and gradients. 
Does not work when functionals are defined with the \verb|combine| command.
\item[\Key{FINE}]
Shortcut keyword for radial integration accuracy $10^{-13}$ and angular expansion order equal to $42$.
\item[\Key{GRID TYPE}] \verb||\newline
\verb|<GC2 LMG TURBO BECKE BECKEORIG SSF BLOCK BLOCKSSF>|\newline
\verb|GC2|, \verb|LMG|, \verb|TURBO| define radial quadrature schemes, only one can be set.
\verb|BECKE|, \verb|BECKEORIG|, \verb|SSF|, \verb|BLOCK|, \verb|BLOCKSSF| define the grid partitioning schemes, only one can be set. 
\begin{description}
\item[GC2] Gauss-Chebyshev quadrature of second kind. 
\item[LMG] As proposed by Lindh, Malmqvist and Gagliardi (default).~\cite{dft-int:LMG}
\item[TURBO] Treutler-Ahlrichs M4-T2 scheme.~\cite{dft-int:treutler-ahlrichs} Implies also that the angular integration quality becomes Z-dependant (see also \verb|ANGINT|) and that pruning is used (see also \verb|NOPRUN|).
\item[BECKE] Becke partitioning scheme with atomic size correction.~\cite{dft-int:becke}
\item[BECKEORIG] Becke partitioning scheme without atomic size correction (default).~\cite{dft-int:becke}
\item[SSF] Stratmann-Scuseria-Frisch partitioning scheme.~\cite{dft-int:ssf}
\item[BLOCK] Becke partitioning scheme with atomic size correction~\cite{dft-int:becke}
combined with a blockwise handling of grid points~\cite{dft-int:dalton-blocked}. Useful for large molecules.
\item[BLOCKSSF] Stratmann-Scuseria-Frisch partitioning scheme~\cite{dft-int:ssf}
combined with a blockwise handling of grid points~\cite{dft-int:dalton-blocked}. Useful for large molecules.
%\item[CARTESIAN]
\end{description}
\item[\Key{GRID1}]
Shortcut for radial integration accuracy $1.0*10^{-5}$, 
angular expansion order equal to $17$, 
radial quadrature \verb|TURBO|, 
grid partitioning scheme \verb|BLOCK|.
\item[\Key{GRID2}]
Shortcut for radial integration accuracy $2.15447*10^{-7}$, 
angular expansion order equal to $23$, 
radial quadrature \verb|TURBO|, 
grid partitioning scheme \verb|BLOCK|.
\item[\Key{GRID3}]
Shortcut for radial integration accuracy $4.64159*10^{-9}$, 
angular expansion order equal to $29$, 
radial quadrature \verb|TURBO|, 
grid partitioning scheme \verb|BLOCK|.
\item[\Key{GRID4}]
Shortcut for radial integration accuracy $5.01187*10^{-14}$,
angular expansion order equal to $35$,
radial quadrature \verb|TURBO|,
grid partitioning scheme \verb|BLOCK|.
\item[\Key{GRID5}]
Shortcut for radial integration accuracy $2.15443*10^{-17}$,
angular expansion order equal to $47$,
radial quadrature \verb|TURBO|,
grid partitioning scheme \verb|BLOCK|.
%\item[\Key{HARTRE}]
%supresses the addition of the XC-contribution to the Fock-matrix/energy.
\item[\Key{HARDNESS}] \verb||\newline
\verb|<HARDNESS>|\newline
sets the hardness of the partitioning function in the Becke weighting scheme. Only positive integer values allowed. 
The higher the hardness the more stepfunction like the partitioning functions. 
Default value is $3$ as proposed by Becke.~\cite{dft-int:becke}
\item[\Key{NOPRUN}]
Supresses the default pruning of the atomic angular integration grids for radial points close to the nucleus. 
\item[\Key{NORMAL}]
Shortcut keyword for radial integration accuracy $10^{-13}$ and angular expansion order equal to $35$.
\item[\Key{RADINT}] \verb||\newline
\verb|<RADINT>|\newline
Determines the quality of the radial integration grid. Default radial integration accuracy is $10^{-11}$.
Note that the value of \verb|RADINT| is changed by the following keywords:
\verb|NORMAL|, \verb|FINE|, \verb|ULTRAFINE|.
The \verb|GRIDX| keywords also imply specific \verb|RADINT| values.
\item[\Key{ULTRAF}]
Shortcut keyword for radial integration accuracy $10^{-15}$ and angular expansion order equal to $64$.
\end{description}

\subsection{*DENSOPT}\label{subsec:densopt}
This section contains the different options for obtaining the SCF wave function
and energy. The default is diagonalization combined with the DIIS scheme for acceleration
of SCF convergence (no damping/level shifting). 
Should have a different name. DENSOPT?
\begin{description}
\item[\Key{2ND\_ALL}] Use second order optimization in all SCF iterations.
\item[\Key{2ND\_LOC}] Use second order optimization in local SCF iterations.
\item[\Key{ARH}] Use Augmented Roothaan-Hall scheme for density optimization 
(default)~\cite{ARH1,ARH2} where the CROP solver (\cite{crop}) is used for 
solving the (level-shifted) Newton equationw.\index{Augmented Roothaan-Hall}\index{ARH} 
\item[\Key{ARH FULL}] Use Augmented Roothaan-Hall for density optimization -
no truncation of the reduced space, keep all micro vectors for the CROP algorithm 
(see ref.~\cite{crop}).
\item[\Key{ARH DAVID}] Use Augmented Roothan-Hall for density optimization where
the (level-shifted) Newton equations are solved using a reduced space solver based 
Davidson's algorithm \cite{davidson:1975}. This option
is recommended for complicated optimizations, especially for large molecular systems
 with charges separated in space. This setting should be used if \Key{ARH} 
converges to a saddle point. 
 \item[\Key{ARH(LS) DAVID}] The same procedure as described for \Key{ARH DAVID} 
augmented with a line search to accelerate convergence for very large molecular 
systems, but is less efficient than \Key{ARH DAVID} for smaller molecular systems. 
\item[\Key{STABILITY}] Check stability of the optimized wave function by calculation
of the lowest Hessian eigenvalue. 
\item[\Key{STAB\_MAXIT}] \verb| | \newline
\verb|<Max. number of stability iterations>|\newline
Max. number of iterations in calculation of lowest Hessian eigenvalue (default is 40).
\item[\Key{CHOLESKY}] Do Cholesky decomposition of overlap matrix (for density optimization
in orthogonal AO basis) - default is L{\"o}wdin decomposition by diagonalization.
\item[\Key{CONTFAC}] \verb| | \newline
\verb|<Contraction factor>|\newline 
For update of trust-radius (used in ARH and second order optimization). If trust-radius
should be contracted,
contract it by this factor (default is 0.7).
\item[\Key{CONTRAC}] \verb| | \newline
\verb|<Contraction criterion>|\newline 
For update of trust-radius (used in ARH and second order optimization). Contract trust-radius
if trust-radius ratio is smaller than this criterion (default is 0.25).
\item[\Key{CONVDYN}] \verb| | \newline
\verb|<Option>|\newline 
Dynamic SCF convergence threshold. This is suitable for large calculations, since the standard SCF
convergence threshold is based on the Frobenius norm of the SCF gradient, which is not size-extensive.
Options are TIGHT, STANDARD, and SLOPPY.
\item[\Key{CONVTHR}] \verb| | \newline
\verb|<Threshold>|\newline
SCF convergence threshold - Frobenius norm of SCF gradient (default is 1.0d-4). 
Note that this convergence criterion
is not size-extensive. For large molecules, it is recommended to use .CONVDYN instead.
\item[\Key{LB94}] Uses the van Leeuwen-Baerends asymptotic correction. Note that the LB94 functional is the same as using the follwing input. 
\begin{verbatim}
**WAVE FUNCTIONS
.DFT
GGAKey SLATER=1.0 VWN=1.0
*DENSOPT
.LB94
**END OF INPUT
\end{verbatim}
However this correction can be combined with all functionals
\item[\Key{CS00}] Using the Casida-Salahub asymptotic correction\cite{CasidaSalahub} 
\begin{eqnarray}
V^{\prime}_{XC}(\textbf{r}) = \max \left(V_{XC} + \Delta , V_{XC} + V_{LB94,correction}\right),  
\end{eqnarray}
with a Zhan-Nichols-Dixon shift \cite{dixon,ZND}
\begin{eqnarray}
\Delta = NZD_{1} \cdot \epsilon_{HOMO} - NZD_{2}
\end{eqnarray}
Where $NZD_{1} = 0.2332$ and $NZD_{2} = 0.315$. Note that NWCHEM have introduced the same correction but uses a $NZD_{2} = 0.0116$. The parameters have been optimized with respect to a B3LYP functional so these parameters can be changed using the following keywords
\item[\Key{CS00 NZD1}] \verb| | \newline
\verb|<NZD1>|\newline
Changes the parameter $NZD_{1}$ (default $NZD_{1} = 0.2332$) see .CS00
\item[\Key{CS00 NZD1}] \verb| | \newline
\verb|<NZD2>|\newline
Changes the parameter $NZD_{2}$ (default $NZD_{1} = 0.315$) see .CS00
\item[\Key{CS00}] \verb| | \newline
\verb|<Delta>|\newline
Using the Casida-Salahub correction with a shift of Delta see .CS00

%\item[\Key{DIAGHESONLY}] Calculate only one SCF energy (i.e., no optimization) 
%and the lowest Hessian eigenvalue for a density constructed by weighted
%average of two input densities. Needs files D1 and D2 (densities). Lowest Hessian
%eigenvalue is found by setting up explicit Hessian and diagonalizing (expensive!).
\item[\Key{DIIS}] Pulay's DIIS scheme is used to speed up SCF convergence~\cite{diis1,diis2}.
\item[\Key{DISK}] Store densities and Fock/KS matrices from previous 
iterations on disk instead of in core when constructing the new Fock/KS matrix ({\it memory-saving option}).
\item[\Key{DISKSOLVER}] Store trial and sigma vectors on disk instead of in core
when solving linear equations ({\it memory-saving option}). Only referenced is density 
optimization method is ARH, Direct Density optimization or second order optimization.
\item[\Key{DORTH}] Do level shift by using the ratio $\vert\vert$Dorth$\vert\vert$/$\vert\vert$D$\vert\vert$ 
(for diagonalization only).
%\item[\Key{DSM}] The Density Subspace Minimization scheme is used to speed up SCF convergence~\cite{TOKJ,TOYJ}.
%\item[\Key{DSMONE}] The Density Subspace Minimization scheme is used to speed up SCF convergence, 
%but only one DSM step is taken in each SCF iteration.
%\item[\Key{DSMXTRA}] The Density Subspace Minimization scheme is used to speed up SCF convergence,
%including extra (more expensive) term.
\item[\Key{DUMPMAT}] Dump Fock/KS and density matrices from all iterations to disk for later investigation.
\item[\Key{EDIIS}] The energy-DIIS scheme is used to speed up SCF convergence~\cite{ediis}.
\item[\Key{EXPAND}] \verb| | \newline
\verb|<Expansion criterion>|\newline 
For update of trust-radius (used in ARH and second order optimization). Expand trust-radius
if trust-radius ratio is larger than this criterion (default is 0.75).
\item[\Key{EXPFAC}] \verb| | \newline
\verb|<Expansion factor>|\newline 
For update of trust-radius (used in ARH and second order optimization). If trust-radius
should be expanded,
expand it by this factor (default is 1.2).
\item[\Key{FIXSHIFT}] \verb| | \newline
\verb|<Shift>|\newline 
Use a fixed level shift in all SCF iterations.
%\item[\Key{HESONLY}] Calculate only one SCF energy (i.e., no optimization) 
%and the lowest Hessian eigenvalue for a density constructed by weighted
%average of two input densities. Needs files D1 and D2 (densities). Lowest Hessian
%eigenvalue is iteratively.
\item[\Key{HESVEC}] \verb| | \newline
\verb|<Number of Hessian eigenvalues>|\newline
Number of lowest Hessian eigenvalues to be calculated with keyword .STABILITY 
(default is 1). 
\item[\Key{LCV}] Compute the Least-Change Valence orbitals after valence density optimization step.
All subsequent calculations are computed using Least-Change Valence orbitals augmented with atomic virtual orbitals of the Atomic density as basis set.
Keyword only effective with TRILEVEL starting guess.
\item[\Key{LCM}]  Compute the Least-Change Molecular orbitals after the Full molecular density calculation.  All subsequent calculations are computed using these orbitals as basis set. .LCV is automatically included. Keyword only meaningful with TRILEVEL starting guess.
\item[\Key{LEVELSH}] \verb| | \newline
\verb|<N> <shift1> <shift2> ...<shiftN> |\newline 
Use custom level shifts in the first {\it N} SCF iterations.
%\item[\Key{LINCOMB}] \verb| | \newline
%\verb|<Weight for linear combination> |\newline
%Weight to use for D1 when linear combination of densities is used as starting
%guess (.START $\rightarrow$ LINCOMB)

\item[\Key{LOW ACCURACY START}] Use low accuracy settings in the first couple of 
iterations until the appropriate gradient threshold have been reached. 

%\item[\Key{LWITER}] Do L{\"o}wdin decomposition of overlap matrix (for density optimization
%in orthogonal AO basis) - iteratively (default is L{\"o}wdin by diagonalization)~\cite{shalf}.
%\item[\Key{LWQITER}] Do L{\"o}wdin decomposition of overlap matrix 
%in quadruple precision (for density optimization
%in orthogonal AO basis) - iteratively.
\item[\Key{MAXELM}] \verb| | \newline
\verb|<Trust radius (max. element)> |\newline
Absolute max. element of step allowed (default is 0.35).
Used in ARH and second order optimization.
\item[\Key{MAXIT}] \verb| | \newline
\verb|<Max. number of iterations> |\newline
Max. number of SCF iterations (default is 100).
\item[\Key{MAXRATIO}] \verb| | \newline
\verb|<Max. ratio for DORTH> |\newline
Largest accepted ratio when using level shift .DORTH.
\item[\Key{MAXSTEP}] \verb| | \newline
\verb|<Trust radius (Frobenius norm)>|\newline
Absolute max. Frobenius norm of step allowed (default is 0.6).
Used in ARH and second order optimization.
\item[\Key{MICTHRS}] \verb| | \newline
\verb|<Threshold for micro iterations>|\newline
The micro iterations are converged to gradient norm times this factor (default is 1.0d-2).
Only referenced if ARH, Direct Density optimization or second order optimization.
\item[\Key{MICROVECS}] \verb| | \newline
\verb|<Max. number of microvectors>|\newline
Max. number of microvectors to be kept in Conjugate Residual Optimal Vectors (CROP) scheme
(default is 2). 
Only referenced if ARH, Direct Density optimization or second order optimization.
\item[\Key{MINDAMP}] \verb| | \newline
\verb|<Minimum damping>|\newline
Never allow level shift to be smaller than this.
\item[\Key{MOCHANGE}] Level shifting is done by use of MO overlaps (diagonalization only).
\item[\Key{MUOPT}] Optimal level shift is chosen by doing a line search in the SCF energy - very expensive! For diagonalization only.
%\item[\Key{NALPHA}] \verb| | \newline
%\verb|<Number of alpha electrons>|\newline
%Explicitly specify number of alpha electrons (automatically set if not specified). If both NALPHA and NBETA are set, they must conform with the total number of electrons!
%\item[\Key{NBETA}] \verb| | \newline
%\verb|<Number of beta electrons>|\newline
%Explicitly specify number of alpha electrons (automatically set if not specified). If both NALPHA and NBETA are set, they must conform with the total number of electrons!
\item[\Key{NOAV}] Turn off Fock matrix averaging in SCF iterations (this is default since the default optimization scheme is ARH, which contains implicit averaging).
%\item[\Key{NEWDAMP}] New level shifting which should be better suited for ARH/CROP. CAREFUL! Still under development!
\item[\Key{NVEC}] \verb| | \newline
\verb|<Max. no of vectors for averaging>|\newline
Maximum number of previous Fock and density matrices to be stored and used for ARH and DIIS (default is 10 for HF and 7 for DFT).
%\item[\Key{NVECDSM}] \verb| | \newline
%\verb|<Max. no of vectors for DSM>|\newline
%\item[\Key{NVECDII}] \verb| | \newline
%\verb|<Max. no of vectors for DIIS>|\newline
\item[\Key{NOINCREM}] Turn off incremental scheme for Fock/KS matrix build.
\item[\Key{NOPREC}] Turn off preconditioning of linear equations (ARH, TrFD, and second order optimization).
\item[\Key{NOSHIFT}] Do no level shifting.
\item[\Key{OVERLAP}] \verb| | \newline
\verb|<Overlap>|\newline
Smallest accepted overlap when doing level shift by use of MO overlaps (.MOCHANGE).
%\item[\Key{PURIFY}] \verb| | \newline
%\verb|<Purification method>|\newline
%Use purification scheme for density optimization 
\item[\Key{RESTART}] The code automaticly saves the density matrix in each SCF iteration in a file called dens.restart. This option restart the calculation from this density matrix. If the program cannot find the dens.restart file the keyword have no effect. 
\item[\Key{RH}] Use standard Roothaan-Hall scheme (diagonalization) for density optimization.
%\item[\Key{SCALVIR}]
%\item[\Key{SOEO}]
%\item[\Key{SPARSE}]
\item[\Key{SOEO}]
Use second order ensemble optimization. Remember to specify active space and starting occupations. Use only in combination with \Key{NOGCBASIS}. 
Requires \Key{HF} or \Key{DFT} as wave function. After an ensemble optimization, the orbitals are written to file \verb|cmo.out| and the AO 
density matrix is written to \verb|dao.out|, which may be used to plot molecular orbitals and charge density plots.
\item[\Key{SOEOSPACE}] \verb| | \newline
\verb|<core>| \verb|<act>|\newline
Specification of the orbitals in the active space. \verb|<core>| being the number of core orbitals (occupations not optimized), and \verb|<act>| the number 
of active orbitals, where the orbital occupations are optimized.
\item[\Key{SOEOOCC}] \verb| | \newline
\verb|<fully occupied>|\newline 
\verb|<fractionally occupied>|\newline
\verb|<n1>| \verb|<n2>| $\cdots$ \newline
Specification of the distribution of electron pairs in the starting orbitals in a second order ensemble optimization. \verb|<fully occupied>| is the number of 
fully occupied orbitals and \verb|<fractionally occupied>| is the number of fractionally occupied orbitals. \verb|<n1>|, \verb|<n2>| etc. are the fractional 
occupations of the fractionally occupied spin-orbitals in the starting guess. The given occupations must lie between zero and one. If they are either zero 
or one, this orbital occupation is not optimized. Equal distribution may result in wrong results due to symmetry. The sum of the fully occupied orbitals and 
the given orbital occupations \verb|<n1>|, \verb|<n2>| etc., must equal the total number of electron pairs in the system. (NOTE that it is possible to 
specify an active space of zero orbitals, but still include fractional occupations. The optimization will thus optimize the shapes of the orbitals, without 
optimizing the orbital occupations.)
\item[\Key{SOEOSAVE}] 
Saves the AO Fock matrix (and the AO density matrix) to the file \verb|soeosave.out|. Used for restart.
\item[\Key{SOEORST}] 
Restart second order ensemble optimization from orbitals stored in \verb|soeosave.out| (so remember to copy this file to the working directory). 
The molecular geometry need not be the same as in the calculation 
where the stored orbitals were created. The active space and starting occupations still need to be specified using \Key{SOEOSPACE} and \Key{SOEOOCC}.
\item[\Key{SOEOGC}] 
Grand-canonical ensemble optimization. Removes the restriction on the total number of electrons from the second order ensemble optimization. The sum of the 
starting occupations still must be correct.
\item[\Key{SOEOMATHR}] \verb| | \newline
\verb|<macro threshold>|\newline 
Specifies the convergence threshold for the macro iterations in second order ensemble optimization.
\item[\Key{SOEOMITHR}] \verb| | \newline
\verb|<micro threshold>|\newline 
Specifies the convergence threshold for the micro iterations in second order ensemble optimization.
\item[\Key{SOEOTRUST}] \verb| | \newline
\verb|<trust-radius>|\newline 
Specifies the starting trust-radius in the ensemble optimization. Trust-radius will still be updated throughout the optimization.
\item[\Key{SOEODIPOLE}] Use the ensemble density matrix to determine electric dipole moment and static electric polarizability.
\item[\Key{START}] \verb| | \newline
\verb|<Option>|\newline
Starting guess for SCF optimization. Options are:
\begin{itemize}
\item {\bf H1DIAG} Obtain initial guess by diagonalizing one-electron Hamiltonian
\item {\bf ATOMS}  Obtain initial guess by atoms-in-molecules approach (default)
\item {\bf TRILEVEL} Optimization in three steps. 1. Atomic densities, 2. Valence molecular density, 3. Full molecular density~\cite{trilevel1, trilevel2}. 
\end{itemize}
Note that Huckel guess is not supported by \lsdalton.
%\item[\Key{TRSCF}] Use Trust-Region SCF scheme, i.e. diagonalization, .DORTH level shift, and DSM~\cite{TOKJ,TOYJ}.
%\item[\Key{TrFD}] Density optimization by minimization of the Roothaan-Hall energy E$_{\rm RH}$ = Tr{\bf FD}({\bf X})~\cite{TrFD}.
%\item[\Key{TrFD FULL}] As above, but keep the full subspace of trial vectors (instead of truncating) when solving linear equations.
%\item[\Key{UNREST}] Force unrestricted calculation (default for open shell systems).
\item[\Key{VanLenthe}] Use Van Lenthe's scheme for level shifting and averaging~\cite{VanLenthe}.

\subsection{\$INFO}\label{subsec:info}
For the different parts of the wave function optimization, it is possible
to get very detailed information.
This section describes these keywords. They must be put
under *LINSCA in a section beginning with
\$INFO and ending with \$END INFO. Note that these keywords
do NOT start with a dot!
\begin{description}
\item[\Keyinfo{INFO\_CROP}] Detailed info from Conjugated Residual OPtimal vectors scheme (Direct Density Optimization, ARH and second order optimization).
\item[\Keyinfo{INFO\_DIIS}] Detailed info from Direct Inversion in the Iterative Subspace algorithm.
%\item[\Keyinfo{INFO\_DSM}] Detailed info from Density Subspace Minimization algorithm.
%\item[\Keyinfo{INFO\_DSM\_DETAIL}] Even more detailed info from Density Subspace Minimization algorithm.
\item[\Keyinfo{INFO\_LEVELSHIFT}] Detailed info from determination of dynamic level shift (Direct Density Optimization, ARH and second order optimization). 
\item[\Keyinfo{INFO\_LINEQ}] Detailed info from solution of linear equations and trust-radius update (Direct Density Optimization, ARH and second order optimization).
\item[\Keyinfo{INFO\_RH}] Detailed info from Roothaan-Hall algorithm (diagonalization).
\item[\Keyinfo{INFO\_RH\_DETAIL}] Even more detailed info from Roothaan-Hall algorithm (diagonalization).
\item[\Keyinfo{INFO\_STABILITY}] Detailed info from calculation of lowest Hessian eigenvalue (stability analysis) and HOMO-LUMO gap.
\item[\Keyinfo{INFO\_STABILITY\_REDSPACE}] Reduced space info from calculation of lowest Hessian eigenvalue (stability analysis) and HOMO-LUMO gap.
%\item[\Keyinfo{INFO\_WEIGHT}] Detailed info about weights of the densities in DIIS and DSM algorithms.
\item[\Keyinfo{}]
\end{description}\end{description}

\section{**OPTIMIZE}\label{sec:optimize}
This input module describes the keywords needed to optimize a molecular geometry.
In the current release no second-order algorithms are available since the molecular Hessian is not implemented, but
a number of quasi-Newton trust-radius level-shifted techniques for finding
equilibrium geometries are available (\Key{BFGS}, \Key{PSB}, \Key{DFP}).
The methods differ in the way approximate Hessians are built and updated: \Key{INIMOD}, \Key{INIRED} and \Key{INITEV} provide different approximations for the initial Hessian, 
while \Key{MODHES} recomputes the approximated Hessian at every geometry.

The geometry optimization generates a sequence of geometry steps. The coordinates of each 
accepted step are stored in the files MOLECULE.\emph{XXX}, where \emph{XXX}
is the geometry step number, and the final converged geometry in MOLECULE.OUT.

Geometric structure can be optimized in redundant internal (\Key{REDINT}) or
Cartesian  (\Key{CARTES}) coordinates with two sets of convergence criteria (two for the gradient and two for the step).
Convergence is obtained when the following four quantities are respectively smaller than the four convergence criteria $\epsilon$, 5$\epsilon$, 3$\epsilon$ and 15$\epsilon$  \cite{NewConv}:
\begin{itemize}
\item the root-mean-square of the gradient, 
\item the maximum absolute value of the gradient, 
\item the root-mean square of the step vector,
\item and the maximum absolute element (in internal or Cartesian coordinates) of the step vector.
\end{itemize}


The default setting specifies the minimization in internal coordinates \cite{DaltonOpt}, with a model Hessian \cite{ModelHess} updated with BFGS formula. 
The atoms-in-molecule approach is used to set up the initial density-matrix at each new
geometry step.
However, it is possible to take as starting guess McWeeny-purified converged
density-matrix at the previous geometry. This is achieved by setting the \Key{RESTART} option under
the *DENSOPT subsection.
The default value of $\epsilon$ is $10^{-5}$
(Other predefined values for the convergence criteria can be set using the keywords \Key{VLOOSE}, \Key{LOOSE}, \Key{TIGHT} and \Key{VTIGHT}).
Alternatively the convergence criteria of Baker~\cite{Baker} can be set using \Key{BAKER}.


\vspace{1 cm}
\noindent
\textbf{Example of test cases for geometry optimization calculations} \newline
The test cases are located in LSDALTON/test/geomopt.
These test cases illustrate the main features of the geometry optimization:
\begin{itemize}
\item
\textit{geomopt/cartes\_bfgs\_min}:
performs a DFT/LDA geometry optimization in Cartesian coordinates,
with BFGS update of an initial Hessian equal to a unit matrix
and Baker convergence criteria.
\item
\textit{geomopt/redint\_bfgs}:
performs a DFT/BLYP geometry optimization in redundant internal coordinates
with a model Hessian updated with BFGS formula
and Baker convergence criteria.
\item
\textit{geomopt/Excited\_state\_opt}:
performs a DFT/B3LYP geometry optimization of an excited state in redundant internal coordinates
and Baker convergence criteria.
\item
\textit{geomopt/cartes\_psb\_min}:
performs a DFT/BLYP geometry optimization in Cartesian coordinates
with a diagonal initial Hessian updated by PSB formula,
McWeenie's purification used for constructing the initial guess
and Baker convergence criteria.
\item
\textit{geomopt/geoopt\_constrain}:
shows a constrained geometry optimization of ethane,
keeping the C-C bond distance fixed
(Internal coordinate \#1 as determined from the use of \Key{FINDRE}).
\item
\textit{geomopt/cartes\_trilevel\_newconv}:
performs a geometry optimization in Cartesian coordinates
with \Key{TRILEVEL} starting guess and a default convergence criteria,
 a diagonal initial Hessian updated by a loose DFP formula.
\end{itemize}

\begin{description}

\item[\Key{BAKER}]
Activates the convergence criteria of Baker \cite{Baker}. The minimum
is then said to be found when the largest element of the gradient
vector (in Cartesian or redundant internal coordinates) falls below
$3\cdot 10^{-4}$ and either the energy change from the last
iteration is less than $10^{-6}$ or the largest element of
the predicted step vector is less than $3\cdot 10^{-4}$.

\item[\Key{BFGS}]
Specifies the use of a first-order method\index{first-order optimization}
with the Broyden-Fletcher-Goldfarb-Shanno (BFGS)
update formula for optimization. This is the
preferred first-order method for minimizations, as this update is able
to maintain a positive definite Hessian.

\item[\Key{CARTES}]
Specifies that Cartesian coordinates\index{Cartesian coordinates}\index{coordinate system!Cartesian coordinates}
should be used in the optimization.

\item[\Key{CONSTR}]\verb| | \newline
\verb|<Number of contrained redundant internal coordinates>|\newline
\verb|<Indices of each contrained redundant internal coordinate, one per line>|\newline
Specifies a contrained geometry optimization. You will need to give
the total number of redundant internal coordinates to constrain and their
indices as defined using the \Key{FINDRE} keyword.
For example fixing redundant internal coordinates 4 and 9 would read like:
\begin{verbatim}
...
**OPTIMIZE
.CONSTR
2
4
9
...
**END OF INPUT
\end{verbatim}


\item[\Key{DFP}]
Specifies that a first-order\index{first-order optimization} method with the
Davidon-Fletcher-Powell (DFP) update formula should be used for optimization.

\item[\Key{FINDRE}]
Using this keyword no geometry optimization is carried out, but only the redundant internals coordinates are found.

\item[\Key{INIMOD}]
Use a simple model Hessian~\cite{ModelHess} diagonal in redundant
internal coordinates as the initial Hessian. All diagonal elements are
determined based on an extremely simplified molecular mechanics model,
yet this model provides Hessians that are good starting points for
most systems, thus avoiding any calculation of the exact Hessian. This
is the default for optimizations in redundant internal coordinates. Currently 
this is not an option for optimization in Cartesian coordinates.

\item[\Key{INIRED}]
Specifies that the initial Hessian should be diagonal in redundant internal 
coordinates. The different diagonal
elements are set equal to 0.5 for bonds, 0.2 for angles and 0.1 for
dihedral angles, unless \Key{INITEV} has been specified. If the
optimization is run in Cartesian coordinates, the diagonal internal
Hessian is transformed to Cartesians.

\item[\Key{INITEV}]\verb| | \newline
\verb|<Read EVLINI, the eigenvalues>|\newline
The default initial Hessian for first-order
minimizations is the identity matrix when Cartesian coordinates are used, and a diagonal
matrix when redundant internal coordinates are used. If \Key{INITEV}
is used, all the diagonal elements (and therefore the eigenvalues) are
set equal to the value EVLINI. 

\item[\Key{LOOSE}]
Default convergence criteria $\epsilon$ set to $1.0\cdot 10^{-4}$.

\item[\Key{MAX IT}]
Maximum number of iterations in geometry optimization scheme
(default set to 100).

\item[\Key{MODHES}]
Determine a new model Hessian (see \Key{INIMOD})
at every geometry without doing any updating. The model is thus used
in much the same manner as an exact Hessian, though it is obviously
only a relatively crude approximation to the analytical Hessian.

\item[\Key{PRINT}]\verb| | \newline
\verb|<Print level>|\newline
Print level for the output. The higher the print level, the more information
is printed to the output file.

\item[\Key{REDINT}]
Specifies that redundant internal coordinates\index{Redundant internal coordinates}\index{coordinate system!Redundant internal coordinates}
should be used in the optimization. This is the default.

\item[\Key{PSB}]
Specifies that a first-order method with the
Powell-Symmetric-Broyden (PSB) 
update formula should be used for optimization.

\item[\Key{TIGHT}]
Default convergence criteria $\epsilon$ set to $1.0\cdot 10^{-6}$.

\item[\Key{VLOOSE}]
Default convergence criteria $\epsilon$ set to $1.0\cdot 10^{-3}$.

\item[\Key{VTIGHT}]
Default convergence criteria $\epsilon$ set to $1.0\cdot 10^{-7}$.

\end{description}

\section{**DYNAMI}\label{sec:dynami}
This input module is devoted to molecular dynamics simulations. LSDALTON provides some tools for performing
direct Born-Oppenheimer molecular dynamics, i.e. trajectories are integrated with forces calculated on the fly.
Trajectory integration is available for SCF-type wavefunctions (with dispersion correction for DFT)
  and for DEC wavefunctions. Properties could be calculated along the trajectories and should be
 specified in the **RESPONSE section.
LSDALTON simulates microcanonical(NVE) ensembles. Currently no sampling is availabe and the starting coordinates
and velocities must be provided as input. Trajectories are integrated with gradient-based velocity-Verlet 
method \cite{Verlet}. 

LSDALTON provides several options for the choice of starting guess for SCF-type electronic theory methods. By
default, the guess is generated anew each time step according to the corresponding input. Alternatively, one can
restart from the density optimized at the previous time-step. This is achieved by setting the .RESTART option under
the *DENSOPT subsection. This reduces the number of SCF iterations per time-step and accelerates the integration. 
Two more efficient algorithms for starting guess propagation are implemented in LSDALTON: Fock matrix dynamics \cite{FMD} 
 and time-reversible propagator \cite{TimRev}.  
\newline
\vspace{1 cm}
\noindent
\textbf{Example test cases for **DYNAMI calculations}: \newline
These test cases illustrate the main features of direct dynamics. The test cases are located in LSDALTON/test/ddynam:
\begin{itemize}
\item
\textit{ddynam/H2O\_fmd}: free water molecule with Fock matrix dynamics; 
\item
\textit{ddynam/HCN\_ddyn}: HCN dissociation with restart density; 
\item
\textit{ddynam/HF\_vib}: a vibrating HF molecule with time-reversible propagator.
\end{itemize}

\begin{description}
\item[\Key{PRINT}]\verb| | \newline
\verb|<Print level>|\newline
Print level for the output. The higher the print level, the more information
is printed to the output file.
\item[\Key{TIMESTEP}]\verb| | \newline
Defines integration time step in femtoseconds which should be given in the next line (real).
\item[\Key{MAX ITER}]\verb| | \newline
Defines maximum number of time steps. Must be given in the next line (integer).
\item[\Key{MASSWE}]\verb| | \newline 
States that the integration is carried out in mass-weighted coordinates. The default is FALSE.
\item[\Key{MWVEL}]\verb| | \newline 
States that the input velocities are mass-weighted. The default is FALSE.
\item[\Key{VELOCI}]\verb| | \newline 
Begins the input of the velocities. In the next line the number of atoms should be given (integer),
then the velocities grouped atom-wise: three(x,y,z) in a line (real).
\item[\Key{FOCKMD}]\verb| | \newline 
Fock matrix dynamics. Below number of points for extrapolation $N$ and polynomial order $M$ must
be specified. Note that $M>N$ (both integer). Works only with .RESTART option under the *DENSOPT subsection
\item[\Key{TIMREV}]\verb| | \newline 
Time-reversible density propagation. Below the number of points for extrapolation $N$ (integer)  must be specified.
The only thoroughly tested option is $N=2$.

\end{description}


\section{**LOCALIZE ORBITALS}\label{subsec:orbloc}

This section describes keywords needed to perform localization of Hartree--Fock orbitals. The optimization of the
chosen localization function will be performed using a trust-region algorithm \cite{hoyvik:TRM}, and can be performed for  the occupied space and/or the virtual orbital space. For the occupied space the core and valence spaces will be localized separately.  

After running a Hartree-Fock calculation followed by an orbital localization two files with molecular orbital coefficients will be present; \textit{cmo\_orbitals.u} and \textit{lcm\_orbitals.u}. \textit{cmo\_orbitals.u} contains coefficients for the canonical molecular orbitals (CMO) and \textit{lcm\_orbitals.u} contains coefficients for the localized molecular orbitals (LMO). The coefficient matrix row index refers to the atomic orbital basis index, while the column index refers to the molecular orbital basis. The format of the files are as described in the Matrix File Format part of Section \ref{interfacing}.  

For the most efficient orbital localization for large molecular systems \cite{hoyvik:3L}, the .ARH, .START/TRILEVEL  and .LCM options are recommended for use as described in Section \ref{subsec:densopt}, and .PSM with powers 2 2 are recommended for the **LOCALIZE ORBITALS section.  

\begin{description}

\item[\Key{PSM}]\verb| | \newline
\verb|<m m>|\newline
Use powers m of the second moment (PSM) localization function \cite{jansik:2011}.  A power 0 specifies that no localization should be carried out (e.g., 0 1  to only localize the virtual space). m=1 is equivalent with the Boys localization function. For improved locality m$\ge$2 is recommended. 

\item[\Key{PFM}]\verb| | \newline
\verb|<m m>|\newline
Use powers m of the fourth moment (PFM) localization function \cite{hoyvik:PFM}.  A power 0 specifies that no localization should be carried out (e.g., 0 1  to only localize the virtual space). Using power 1 yields orbitals with good locality, but a power 2 is recommended if locality of the orbitals is essential. Powers higher than 3 are not recommended.  The PFM localization function should be chosen if basis sets augmented with diffuse functions are used.

\item[\Key{PipekMezey}]\verb| | \newline
Use Pipek-Mezey localization function using an implementation based on L{\"o}wdin population analysis 
rather than the traditional Mulliken population analysis (see Ref. \cite{hoyvik:PM}).  Both occupied and virtual orbitals will be localized. Please note, the Pipek-Mezey localization function suffers from system and basis set dependencies. For systems with complicated bonding structures or when using basis sets augmented with diffuse functions the PSM or PFM options should be used if good locality is essential.

\item[\Key{OccPipekMezey}]\verb| | \newline
Pipek-Mezey localization of only the occupied space.


\item[\Key{VirtPipekMezey}]\verb| | \newline
Pipek-Mezey localization of only the virtual space.


\item[\Key{PipekM(MULL)}]\verb| | \newline
Use Pipek-Mezey localization function using the Mulliken population analysis. A power 0 specifies that no localization should be carried out.  Otherwise the same considerations as for the \Key{PipekMezey} option.


\item[\Key{OccPipekM(MULL)}]\verb| | \newline
Pipek-Mezey localization (using Mulliken population analysis) of occcupied space.

\item[\Key{VirtPipekM(MULL)}]\verb| | \newline
Pipek-Mezey localization (using Mulliken population analysis) of virtual space.

\item[\Key{No Level2 Localization}]\verb| | \newline
Should only be used in combination with the .START/TRILEVEL and  .LCM options for the *DENSOPT section. The keyword is used to skip orbital localization after the second level in the three level procedure. Not recommended for general use, but can be used to save  computational time if only the virtual space is to be localized.  

\item[\Key{Only Loc}]\verb| | \newline
Provided the file with the molecular orbital coefficients (e.g., \textit{cmo\_orbitals.u} or \textit{lcm\_orbitals.u}) using this keyword the Hartree-Fock calculation will be skipped, and the orbitals will be read in and localized. Important: rename or copy the orbital coefficient file to \textit{orbitals\_in.u}, only this file will be read. After the localization procedure the molecular orbital coefficients corresponding to the localized basis is written to \textit{orbitals\_out.u}. Thus, for this keyword to be used, \textit{orbitals\_in.u} must exist and a localization function must be chosen.


\item[\Key{Orbital Locality}]\verb| | \newline
This keyword will make sure second and fourth moment orbital spreads are printed for all orbitals, not just the least local ones.


\item[\Key{Orbital Plot}]\verb| | \newline
\verb|<orbital specification>|\newline	
Generates .plt file for either the least local or the most local orbitals. To generate .plt files for the least local orbitals use orbital specification   LEASTL, whereas to generate .plt files for the most local orbitals use specification MOSTL. If both least local and most local are wanted, make two \Key{Orbital Plot} sections with each of the specifications. The .plt files may be plotted using for example the UCSF Chimera program package \cite{chimera}. Note: for plotting other orbitals, use the **PLT option described in section \ref{sec:plt}. Also note that the grid dimensions may be modified as described in Section~\ref{sec:pltgrid}.
\end{description}




\section{**RESPONS}\label{subsec:respons}
This section describes the keywords needed for obtaining molecular
properties for HF and DFT based on the atomic orbital based response formulation~\cite{thorvaldsen:214108}.
The response section begins with
**RESPONS. 
General response information related to more than one property
(e.g. how many excitation energies)
is put directly under **RESPONS.
After this,
each individual response property (e.g. the polarizability tensor)
is labeled by an asterisk, and the specific (optional) information related
to that property (e.g. which optical frequencies) follows 
below.
All input values are given in atomic units.
The general structure is thus:
\begin{description}
\item[\Keyinfo{**RESPONS}]
\item[\Keyinfo{.General response information}]
\item[\Keyinfo{*Label for property 1}]
\item[\Keyinfo{.Specific (optional) information for property 1}]
\item[\Keyinfo{*Label for property 2}]
\item[\Keyinfo{.Specific (optional) information for property 2}]
\end{description}
Each of the properties below have a test case.
The test cases may be found in the {\bf LSDALTON/test/LSresponse} directory.
To run a test case, go to the
{\bf test/} directory and type: 
\begin{description}
\item
\verb|./TEST  <test case>|
\end{description}
The test case files contains simple examples of input files.

Below, the list of general response input keywords follows.

\begin{description}
\item[\Key{NEXCIT}] \verb| | \newline
\verb|<Number of excitation energies>|\newline
Determines the number of excitation energies to be calculated.
If this keyword is set the excitation energies 
and the corresponding one-photon dipole absorption strengths
for excited states 1 to NEXCIT are always calculated.
\emph{\bf Important:} If properties involving excited states
are requested (e.g. two-photon absorption or the excited state gradient)
this keyword must be listed \emph{before}
any of these properties. \newline
{\bf Test case}: LSresponse/LSresponse\_HF\_opa
\end{description}

\subsection{*SOLVER}\label{subsec:responsesolver}
General information related to the way response equations are solved to obtained property.
By default standard and eigenvalue response equations are solved using the
algorithm with paired trial vectors\cite{coriani:2007} and damped
(complex) response equations are solved using the algorithm with
symmetrized trial vectors\cite{kauczor:2011}.\newline
\begin{description}
\item[\Key{SYM\_SOLVER}] \verb| | \newline
The solver with symmetrized trial vectors is used for solving standard
response equations.\newline
{\bf Test case}: LSresponse/LSresponse\_HF\_alpha\_astv
\item[\Key{PAIR\_SOLVER}] \verb| | \newline
The solver with paired trial vectors is used for solving damped (complex)
response equations.\newline
{\bf Test case}: LSresponse/LSresponse\_DFT\_alpha\_aptv
\item[\Key{CONVTHR}] \verb| | \newline
\verb|<Convergence threshold>|\newline
Convergence threshold to which response equations are solved.
Default value is set equal to $10^{-4}$.
\item[\Key{MAXIT}] \verb| | \newline
\verb|<Maximum number of iterations>|\newline
Maximum number of iterations in the iterative procedure.
Default value set equal to 100.
\item[\Key{MAXRED}] \verb| | \newline
\verb|<Maximum size of the reduced space>|\newline
Maximum size of the reduced space.
Default value set equal to 200.
\item[\Key{AOPREC}] \verb| | \newline
AO preconditioning is used (MO preconditioning by default).
AO preconditioning may only be used within the algorithm with paired trial vectors.
\item[\Key{NOPREC}] \verb| | \newline
No precondtioning is used.
\item[\Key{CONVDYN}] \verb| | \newline
\verb|<Option>|\newline 
Dynamic convergence threshold suitable for large calculations.
Options are TIGHT, STANDARD, and SLOPPY.
\item[\Key{RESTEXC}] \verb| | \newline
\verb|<Number Of excitation Vectors on file>|\newline 
This is a restart option, that allows the user to restart the calculation. 

Assume that you have already done one calculation where you requested 6 excitation 
energies but realize that you need 10, you can provide the file rsp\_eigenvecs and use

.RESTEXC\\
6 

option to restart the calculation from the previous. 
Naturally this option can also be used to bypass the excitation vector step if the previous calculation failed for some property evaluation which needed the excitation vectors.  
\item[\Key{DTHR}] \verb| | \newline
\verb|<Threshold>|\newline 
Threshold for when excited states is considered degenerate
\end{description}
Now follows a list of input keywords for the specific properties
available in \lsdalton.

\subsection{*ALPHA}\label{subsec:alpha}
Calculation of the (possibly complex) 
polarizability tensor $\alpha$ and its isotropic average.
The polarizability equals minus the linear response function
$<< \mu_A; \mu_B >>_{\omega_B}$, where
$\omega_B$ is in general allowed to be complex --
i.e. $\omega_B = \omega_B^R + i \omega_B^I$.
Frequencies not specified are set to zero by default,
i.e. if no frequencies are specified, the static polarizability is calculated.
\newline
{\bf Test case}: LSresponse/LSresponse\_HF\_alpha 
\begin{description}
\item[\Key{BFREQ}] \verb| | \newline
\verb|<Number of real frequencies for B operator>|\newline
\verb|<Freq1, freq2, ... , freqN>|\newline
Real frequencies $\omega_B^R$ in the corresponding linear response function.
First line after .BFREQ contains the number of frequencies
and the second line contains all frequencies on one line.
\item[\Key{IMBFREQ}] \verb| | \newline
\verb|<Number of imaginary frequencies for B operator>|\newline
\verb|<Freq1, freq2, ... , freqN>|\newline
Imaginary frequencies $\omega_B^I$ in the corresponding
linear response function.
The imaginary part of the complex polarizability describes a broadened one-photon
absorption spectrum.
If .IMBFREQ is not specified, only the real polarizability is calculated.
Input as for .BFREQ. 
\end{description}
For example, the following input is used 
to calculate complex polarizability tensors
for the frequencies $\omega_B = 0.1 + 0.05i$ and $\omega_B = 0.2$ 
(in a.u.):
\begin{description}
\item[\Keyinfo{**RESPONS}]
\item[\Keyinfo{*ALPHA}]
\item[\Keyinfo{.BFREQ}]
\item[\Keyinfo{2}]
\item[\Keyinfo{0.1   0.2}]
\item[\Keyinfo{.IMBFREQ}]
\item[\Keyinfo{2}]
\item[\Keyinfo{0.05   0.0}]
\end{description}


\subsection{*BETA}\label{subsec:beta}
Calculation of the (possibly complex) first hyperpolarizability tensor $\beta$ and 
the corresponding averages $\beta_{\parallel}$ and
$\beta_{\perp}$. 
The first hyperpolarizability tensor equals minus the quadratic response function
$<< \mu_A; \mu_B, \mu_C >>_{\omega_B, \omega_C}$, where
the frequencies are in general allowed to be complex --
i.e. $\omega_B = \omega_B^R + i \omega_B^I$ and
$\omega_C = \omega_C^R + i \omega_C^I$.
The input is analogous to the *ALPHA input.
Frequencies not specified are set to zero by default,
i.e. if no frequencies are specified, the static first hyperpolarizability is calculated. \newline
{\bf Test case}: LSresponse/LSresponse\_HF\_beta 
\begin{description}
\item[\Key{BFREQ}] \verb| | \newline
\verb|<Number of real frequencies for B operator>|\newline
\verb|<Freq1, freq2, ... , freqN>|\newline
Real frequencies $\omega_B^R$ in the corresponding quadratic response function.
Input as for .BFREQ under *ALPHA.
\item[\Key{IMBFREQ}] \verb| | \newline
\verb|<Number of imaginary frequencies for B operator>|\newline
\verb|<Freq1, freq2, ... , freqN>|\newline
Imaginary frequencies $\omega_B^I$ in the corresponding quadratic response function.
Input as for .BFREQ under *ALPHA.
\item[\Key{CFREQ}] \verb| | \newline
\verb|<Number of real frequencies for C operator>|\newline
\verb|<Freq1, freq2, ... , freqN>|\newline
Real frequencies $\omega_C^R$ in the corresponding quadratic response function.
Input as for .BFREQ under *ALPHA.
\item[\Key{IMCFREQ}] \verb| | \newline
\verb|<Number of imaginary frequencies for C operator>|\newline
\verb|<Freq1, freq2, ... , freqN>|\newline
Imaginary frequencies $\omega_C^I$ in the corresponding quadratic response function.
Input as for .BFREQ under *ALPHA.
\end{description}

\subsection{*DAMPED\_TPA}\label{subsec:dtpa}
Damped two-photon absorption~\cite{dampedtpa} where both individual photons
have the same energy (= half the excitation energy).
\newline
{\bf Test case}: LSresponse/LSresponse\_HF\_dtpa 
\begin{description}
\item[\Key{OPFREQ}] \verb| | \newline
\verb|<Number of one-photon frequencies>|\newline
\verb|<Freq1, freq2, ... , freqN>|\newline
The frequencies in the input are one-photon frequencies (=half the
excitation energy in case of resonance).
\item[\Key{GAMMA}] \verb| | \newline
\verb|<Damping parameter>|\newline
The damping parameter $\gamma$ is used in \emph{all} response equations,
i.e. $i \gamma$ is effectively added to all frequencies occuring in response equations.
Default value: 0.005 a.u.
\end{description}

\subsection{*QUASIMCD}\label{subsec:quasimcd}
Calculation of magnetic circular dichroism (MCD) $\mathcal{A}$ and $\mathcal{B}$ terms, based on the method of Ref \cite{KjaergaardMCD} but reformulated using Ref. \cite{thorvaldsen:214108}. The calculation of Damped MCD spectra according to Ref. \cite{KjaergaardDampedMCD} is also possible with this keyword.

{\bf Note that one or more of the 3 keywords \{ .DAMPEDXCOOR  , .MCDEXCIT, .DAMPEDRANGE \} must be specified}

If {\bf .MCDEXCIT} is specified the default is to proceed in the following way. 
\begin{enumerate}
\item Determine the number of excited states as specified. 
\item Determine $\mathcal{B}$ terms for all allowed transitions. Both London atomic orbitals LAO (also called Gauge including atomic orbitals GIAO) and non-LAOs are used to determine the $\mathcal{B}$ terms. LAOs are superior to standards AOs, but the difference can give an indication of how close the result is to the basis set limit. 
\item Simulate the MCD spectra based on the calculated $\mathcal{B}$ terms by associating a Lorentzian lineshape function (with lineshape parameter of 0.005 a.u.) with each $\mathcal{B}$ term (See Ref. \cite{KjaergaardMCD}). 
The simulated spectra is written to a file MCDspectraAU.dat (in atomic units). The first number is the frequency, the second the LAO ellipticity and the third is the non-LAO ellipticity. A corresponding file is generated call MCDspectra.dat using standard units ($cm^{-1}$ for frequencies and molar ellipticity for the ellipticity).
The individuel $\mathcal{B}$ term contribution to the MCD spectra is written to the files BtermAU.dat and Bterm.dat. Per default 5000 points are used for the simulated spectra. 
The standard units for the $\mathcal{B}$ term is given as $\frac{D^{2} \mu_{B}}{cm^{-1}}$, where $D$ is the unit Debye, $\mu_{B}$ is the Bohr magneton. The conversion factor between atomic units and the standard units are $5.88764 \dot 10^{-5}$.
The standard units for the $\mathcal{A}$ term is given as $D^{2} \mu_{B}$ with a conversion factor of $12.920947806163007$.

\item Perform damped calculations based on the Standard MCD calculation. A number of points around each of the MCD peaks (default value: 10) are determined for which to calculate the damped spectra. The result is printed to the files dampedMCDspectraRAU.dat (atomic units) and dampedMCDspectraR.dat.
\end{enumerate}
{\bf Test case}: LSresponse/LSresponse\_mcd\_calc, LSresponse/LSresponse\_mcd\newline
LSresponse/LSresponse\_mcd2, LSresponse/LSresponse\_mcd3  \newline
Note that only the LSresponse/LSresponse\_mcd\_calc test case is a good example for a general MCD calculation

\begin{description}
\item[\Key{MCDEXCIT}] \verb| | \newline
\verb|<Number of excited states to consider>|\newline
The $\mathcal{B}$ terms will be calculated for all states, unless they are zero due to selection rules (dipole forbidden)
\item[\Key{DEGENERATE}] \verb| | \newline
This keyword must be set if the molecule have a degenerate state so that $\mathcal{A}$ terms are possible, otherwise all states are assumed non-degenerate and no $\mathcal{A}$ terms will be calculated. When this keyword is set the $\mathcal{A}$ terms are calculated and a derivative Lorentzian lineshape function is used to simulate the individuel $\mathcal{A}$ term contributions to the MCD spectra (See Ref. \cite{KjaergaardMCD}). The individuel $\mathcal{A}$ terms contributions to the MCD spectra is written to the files AtermAU.dat and Aterm.dat   
\item[\Key{DAMPEDXCOOR}] \verb| | \newline
\verb|<Number of frequencies>|\newline
\verb|<Freq1, freq2, ... , freqN>|\newline
The frequencies for which to calculate the damped MCD spectra (in atomic units).
\item[\Key{DAMPEDRANGE}] \verb| | \newline
\verb|<Freq1>|\newline
\verb|<FreqN>|\newline
\verb|<Number of frequencies in range between Freq1 and Freq2>|\newline
The frequencies for which to calculate the damped MCD spectra (in atomic units), here specified as an interval.
\item[\Key{NO LONDON}] \verb| | \newline
Deactivates the use of LAOs
\item[\Key{NO LONDON}] \verb| | \newline
Deactivates the use of non-LAOs
\item[\Key{NO SIMULATE}] \verb| | \newline
Deactivates the simulated spectra
\item[\Key{NO ATERM}] \verb| | \newline
Deactivates the calculation of $\mathcal{A}$ terms
\item[\Key{NO BTERM}] \verb| | \newline
Deactivates the calculation of $\mathcal{B}$ terms
\item[\Key{NO DAMPED}] \verb| | \newline
Deactivates the performance of damped MCD spectra
\item[\Key{GAUSSIAN}] \verb| | \newline
Uses Gaussian lineshape functions (with a lineshape parameter of 0.0070851079363103793) instead of the default Lorentzian lineshape functions.
\item[\Key{LINESHAPEPARAM}] \verb| | \newline
\verb|<lineshape parameter>|\newline
Changes the lineshape parameter (default value 0.005 a.u. for lorentz, 0.0070851079363103793 a.u. for gaussian) used for the simulated spectra
\item[\Key{NVECFORPEAK}] \verb| | \newline
\verb|<number of frequencies used around each MCD peak>|\newline
Changes the number of frequencies used around each MCD peak (default 10).
\item[\Key{NSTEPS}] \verb| | \newline
\verb|<number of frequencies used for the simulated MCD spectra>|\newline
Changes the number of frequencies used for the simulated MCD spectra (default 5000).
\end{description}
Note that the keyword {\bf .DTHR} under {\bf *SOLVER} (see section \ref{subsec:responsesolver} ) can be used to define the threshold for when excited states is considered degenerate

\subsection{*SHIELD}\label{nmrshield}
Nuclear magnetic shielding tensors and chemical shift can be calculated with this keyword
{\bf Test case}: LSresponse/LSresponse\_NMR\_SHIELD

\subsection{*ESDIPOLE}\label{subsec:esd}
Permanent dipole moment for excited states.
{\bf Always requires specification of .NEXCIT!}
If .EXSTATES is not specified, the excited state dipole moments
for all excited states from 1 to NEXCIT
are calculated. \newline
{\bf Test case}: LSresponse/LSresponse\_HF\_esd
\begin{description}
\item[\Key{EXSTATES}] \verb| | \newline
\verb|<Number of specific excited states to consider>|\newline
\verb|<Specific excited states to consider>|\newline
Only calculate excited state gradient for selected excited states.
The first line after .EXSTATES contains the number
of excited states where the excited state gradient is to be calculated,
and the second line specifies which states.
{\bf Important:} The position of the highlest lying excited state
specified by .EXSTATES must be smaller than or equal to
the number of excited states specified by .NEXCIT!
\end{description}

\subsection{*ESGRAD}\label{subsec:esg}
Calculation of the molecular gradient for excited states.
{\bf Always requires specification of .NEXCIT!}
If .EXSTATES is not specified, excited state gradients
for all excited states from 1 to NEXCIT
are calculated. \newline
{\bf Test case}: LSresponse/LSresponse\_HF\_esg 
\begin{description}
\item[\Key{EXSTATES}] \verb| | \newline
\verb|<Number of specific excited states to consider>|\newline
\verb|<Specific excited states to consider>|\newline
Input is identical to .EXSTATES under *ESDIPOLE.
\end{description}
For example, the following input is used 
to calculate excited state gradients for excited states number 3 and 6:
\begin{description}
\item[\Keyinfo{**RESPONS}]
\item[\Keyinfo{.NEXCIT}]
\item[\Keyinfo{6}]
\item[\Keyinfo{*ESGRAD}]
\item[\Keyinfo{.EXSTATES}]
\item[\Keyinfo{2}]
\item[\Keyinfo{3   6}]
\end{description}


\subsection{*GAMMA}\label{subsec:gamma}
Calculation of the (possibly complex)  second hyperpolarizability tensor $\gamma$ and 
the corresponding averages $\gamma_{\parallel}$ and
$\gamma_{\perp}$. 
The second hyperpolarizability tensor equals minus the cubic response function
$<< \mu_A; \mu_B, \mu_C, \mu_D >>_{\omega_B, \omega_C, \omega_D}$, where
the frequencies are in general allowed to be complex --
i.e. $\omega_B = \omega_B^R + i \omega_B^I$,
$\omega_C = \omega_C^R + i \omega_C^I$, and
$\omega_D = \omega_D^R + i \omega_D^I$.
The input is analogous to the *ALPHA and *BETA inputs.
Frequencies not specified are set to zero by default,
i.e. if no frequencies are specified the static second hyperpolarizability is calculated. \newline
{\bf Test case}: LSresponse/LSresponse\_HF\_gamma 
\begin{description}
\item[\Key{BFREQ}] \verb| | \newline
\verb|<Number of real frequencies for B operator>|\newline
\verb|<Freq1, freq2, ... , freqN>|\newline
Real frequencies $\omega_B^R$ in the corresponding cubic response function.
Input as for .BFREQ under *ALPHA.
\item[\Key{IMBFREQ}] \verb| | \newline
\verb|<Number of imaginary frequencies for B operator>|\newline
\verb|<Freq1, freq2, ... , freqN>|\newline
Imaginary frequencies $\omega_B^I$ in the corresponding cubic response function.
Input as for .BFREQ under *ALPHA.
\item[\Key{CFREQ}] \verb| | \newline
\verb|<Number of real frequencies for C operator>|\newline
\verb|<Freq1, freq2, ... , freqN>|\newline
Real frequencies $\omega_C^R$ in the corresponding cubic response function.
Input as for .BFREQ under *ALPHA.
\item[\Key{IMCFREQ}] \verb| | \newline
\verb|<Number of imaginary frequencies for C operator>|\newline
\verb|<Freq1, freq2, ... , freqN>|\newline
Imaginary frequencies $\omega_C^I$ in the corresponding cubic response function.
Input as for .BFREQ under *ALPHA.
\item[\Key{DFREQ}] \verb| | \newline
\verb|<Number of real frequencies for D operator>|\newline
\verb|<Freq1, freq2, ... , freqN>|\newline
Real frequencies $\omega_D^R$ in the corresponding cubic response function.
Input as for .BFREQ under *ALPHA.
\item[\Key{IMDFREQ}] \verb| | \newline
\verb|<Number of imaginary frequencies for D operator>|\newline
\verb|<Freq1, freq2, ... , freqN>|\newline
Imaginary frequencies $\omega_D^I$ in the corresponding cubic response function.
Input as for .BFREQ under *ALPHA.
\end{description}


\subsection{*MOLGRA}\label{subsec:molgra}
Single point calculation of the molecular gradient. \newline
{\bf Test case}: LSresponse/LSresponse\_HF\_molgra 

\subsection{*TPA}\label{subsec:tpa}
Two-photon absorption where both individual photons
have the same energy (= half the excitation energy).
{\bf Always requires specification of .NEXCIT!}
If .EXSTATES is not specified, two-photon absorption
for all excited states from 1 to NEXCIT
are calculated. \newline
{\bf Test case}: LSresponse/LSresponse\_HF\_tpa 
\begin{description}
\item[\Key{EXSTATES}] \verb| | \newline
\verb|<Number of specific excited states to consider>|\newline
\verb|<Specific excited states to consider>|\newline
Input is identical to .EXSTATES under *ESDIPOLE.
\end{description}


\section{**DEC and **CC}\label{sec:dec}
Correlated coupled-cluster (CC) calculations using the Divide-Expand-Consolidate (DEC) scheme~\cite{dec1,dec2,dec3,dec4,dec5,dec6}, where
local orbitals are used to carry out the correlated calculation on a large molecular system in terms of many small local calculations.
Formally, the computational time for the DEC scheme scales linearly with system size.
Currently only second-order M{\o}ller-Plesset (MP2) calculations (energy, density, and geometry optimization) are available using the DEC scheme. 
DEC calculations are invoked using the **DEC keyword.

Furthermore, conventional coupled-cluster energies for the MP2, CC2, and CCSD models are available using the **CC keyword. The computational time for these implementations scale with the system size to the fifth, fifth, and sixth power for MP2, CC2, and CCSD, respectively.

The **DEC and **CC implementations employ massive parallelism and will be highly effective only if many computing nodes are available.

\textbf{Special note}: It is highly recommended that the memory available for the calculation is specified in the input, see \Key{MEMORY} keyword below.

The **DEC section uses local orbitals by default, while the **CC section uses canonical orbitals by default.
Below keywords for the **DEC section and **CC section and relevant test cases are given. 

\vspace{1 cm}
\noindent
\textbf{**DEC input keywords} (only MP2 model):

\begin{description}
\item[\Key{FOT}]\verb| | \newline
\verb|<FOT level>|\newline
Level for fragment optimization threshold (FOT). The FOT defines the precision of a DEC calculation compared
to a conventional calculation~\cite{dec1,dec2,dec3,dec4,dec5,dec6}.
The input is an integer which defines the negative power of the FOT; e.g. if FOT level=5, then the FOT=$10^{-5}$ a.u.
Thus, the higher the FOT, the smaller is the error of the DEC calculation compared to a conventional calculation (and of course, the more expensive is the calculation). Possible values: 1,2,3,4,5,6,7,8. Default value: 4.


\item[\Key{MEMORY}] \verb| | \newline
\verb|<Memory in gigabytes>| \newline
It is highly recommended to specify the memory (in gigabytes) available for the calculation using this keyword! For an MPI run this is the memory available for each MPI process.
If this keyword is not set, the program will try to estimate the available memory using a system call. While this will usually work fine it might fail on some architectures.
(This keyword can also be used in **CC section).


\item[\Key{PAIRTHR}] \verb| | \newline
\verb|<Pair distance threshold (in a.u.)>|\newline
Pair distance threshold (in a.u.) beyond which pair fragments are omitted from the calculation. By default this value is adapted to the FOT.

\item[\Key{PAIRTHRANGSTROM}]  \verb| | \newline
\verb|<Pair distance threshold (in Angstrom)>|\newline
Pair distance threshold (in {\AA}) beyond which pair fragments are omitted from the calculation. By default this value is adapted to the FOT.


\item[\Key{RESTART}] \verb| | \newline
Restart unfinished DEC calculation. Requires that HF files \emph{dens.restart}, \emph{fock.restart}, \emph{lcm\_orbitals.u}, and \emph{overlapmatrix} are present in the folder where the calculation is restarted. Additionally, if .info files are present it means that some of the fragment calculations have been
carried out and the .RESTART keyword will read information about the finished fragment calculations from file and calculate the missing fragments.
Specifically, if the files atomicfragments.info and atomicfragmentsdone.info are present, it means that some (or all) of the atomic fragment calculations are finished. If, in addition, the file fragenergies.info (energy calculation) \emph{or} mp2grad.info (MP2 density or gradient calculation) is also present,
it means that some (or all) of the pair fragments are also done.

\textbf{Note 1: This keyword should NOT be combined with .RESTART in the *DENSOPT section.}

\textbf{Note 2: This keyword cannot be used for a DEC-MP2 geometry optimization. The only way to "restart" a DEC-MP2 geometry optimization is use the current geometry in a new DEC-MP2 geometry optimization -- i.e., copy the most recently generated MOLECULE.xxx file to MOLECULE.INP and run the calculation again using the same LSDALTON.INP.}


\item[\Key{NOTABSORBH}] \verb| | \newline
By default orbitals originally assigned to hydrogen atoms are reassigned to the nearest heavy atom. This reassigning can be turned off by invoking this keyword.

\item[\Key{FROZENCORE}]  \verb| | \newline
Use the frozen core approximation (can also be used in **CC section).


\item[\Key{DENSITY}] \verb| | \newline
Calculate DEC-MP2 density matrix~\cite{dec5} (stored in MP2.dens). The MP2 electric dipole moment is also calculated if this keyword is invoked.

\item[\Key{GRADIENT}] \verb| | \newline
Calculate MP2 molecular gradient~\cite{dec5} at input geometry (for geometry optimization, see test case examples below). The MP2 density and electric dipole moment is also calculated if this keyword is invoked.

\item[\Key{KAPPATHR}] \verb| | \newline
\verb|<THR>|\newline
Threshold for $\bar{kappa}$ multiplier equations solved when DEC-MP2 molecular gradient is calculation. Default value is 10$^{-4}$.


\item[\Key{CANONICAL}] \verb| | \newline
Use canonical orbitals. This is only recommended for advanced users for testing purposes. In general, the DEC scheme is only meaningful for local orbitals.

\item[\Key{MPIGROUPSIZE}] \verb| | \newline
\verb|<Size>|\newline
Size of local MPI groups (integer) in massively parallel DEC scheme~\cite{dec6}. The default settings should suffice for most purposes, and use of the keyword is only recommended for advanced users.


\end{description}

\vspace{1 cm}
\noindent
\textbf{Example test cases for **DEC calculations}: \newline
These test cases illustrate the main features of the DEC scheme. The test cases are located in LSDALTON/test/dectests.

\begin{itemize}
\item
\textit{decmp2\_energy}: Calculate DEC-MP2 energy.
\item
\textit{decmp2\_density}: Calculate DEC-MP2 density matrix and electric dipole moment.
\item
\textit{decmp2\_gradient}: Calculate DEC-MP2 molecular gradient.
\item
\textit{decmp2\_geoopt}: Carry out DEC-MP2 geometry optization. Note that the geometry optimization is invoked by the **OPTIMIZE section, while the **DEC section specifies that the geometry optimization should use DEC-MP2 energies and molecular gradients. The DEC-MP2 geometry optimization will automatically estimate the intrinsic energy error of the DEC calculation. If this error is larger than the difference between two geometry, the FOT level will be increased to get a more accurate energy and gradient in the next iteration.
\end{itemize}


\vspace{1 cm}
\noindent
\textbf{**CC input keywords}:

\begin{description}

\item[\Key{MP2}]\verb| | \newline
Use MP2 model. This gives the canonical MP2 energy and is not optimized for parallel performance. It is only intended to be used for benchmarking DEC-MP2 calculations on relatively small molecular systems.

\item[\Key{CC2}]\verb| | \newline
Use CC2 model which is not an optimized code, but rather uses the CCSD optimized code with some contributions removed. See CCSD documentation for details.

\item[\Key{CCSD}]\verb| | \newline
Use CCSD model. The code employed is a massively parallel version and flexible with respect to memory. Due to the use of one-sided communication in these models it is recommended to run CCSD with an asynchronous progress engine switched on (how to do that can be found in the documentation of the respective MPI library you are using). When running a CCSD calculation make sure that at least one quantity of size $V^2O^2$ ($O$ = number of occupied orbitals, $V$ = number of virtual orbitals) fits into memory in double precision floating point numbers (with some additional space of course). Using this keyword automatically triggers checkpointing after each iteration. The calculation can be restarted with the usual \Key{.RESTART} keyword.

\item[\Key{FROZENCORE}]  \verb| | \newline
Use the frozen core approximation (can also be used in **DEC section).

\item[\Key{MEMORY}] \verb| | \newline
\verb|<Memory in gigabytes>| \newline
It is highly recommended to specify the memory (in gigabytes) available for the calculation using this keyword! For an MPI run this is the memory available for each MPI process.
If this keyword is not set, the program will try to estimate the available memory using a system call. While this will usually work fine it might fail on some architectures.
(This keyword can also be used in **DEC section).

\item[\Key{CCMAXITER}]  \verb| | \newline
Maximum number of iterations in CC solver.

\item[\Key{CCTHR}]  \verb| | \newline
Convergence threshold for residual norm in CC solver.


\item[\Key{SUBSIZE}]  \verb| | \newline
Number of previous vectors to store in CROP scheme~\cite{crop} for CC solver.

\item[\Key{CCSDNOSAFE}]  \verb| | \newline
Prevent saving of CCSD amplitudes as checkpoint files. 


\end{description}


\vspace{1 cm}
\noindent
\textbf{Example test cases for **CC calculations}: \newline
The test cases are located in LSDALTON/test/dectests.

\begin{itemize}
\item
\textit{fullmp2\_energy}: Calculate conventional canonical MP2 energy.
\item
\textit{fullccsd\_high}: Calculate conventional canonical CCSD energy.
\end{itemize}





\section{**PLT}\label{sec:plt}
Generation of .plt files which may be used to visualize densities, orbitals, and electrostatic potentials by calculating these at specific points in space.
The .plt files can be visualized e.g. using the Chimera program~\cite{chimera}.

\textbf{Important note}: This section assumes that an LSDALTON calculation has already been carried out to generate files containing
density matrix elements or molecular orbital coefficients. The .plt files can then be generated by running a new calculation where the LSDALTON.INP file has been modified as exemplified below.

The density matrix from HF and DFT calculation are saved in a file \emph{dens.restart}, while the MP2 density is saved in the file \emph{MP2.dens} (if requested, see **DEC section).
The canonical MO coefficients are saved in the file \emph{cmo\_orbitals.u}, while localized molecular orbitals are saved in a file \emph{lcm\_orbitals.u} (if requested).
The definition of the grid points is described in the **PLTGRID section.

Note that it is also possible to generate .plt files for orbitals on the run as described in Section \ref{subsec:orbloc}.

\begin{description}
\item[\Key{INPUT}] \verb| | \newline
\verb|<Name of input file>| \newline
The name of the input file containing density matrix elements or orbital coefficients. Possible input files include \emph{dens.restart}, \emph{MP2.dens}, \emph{cmo\_orbitals.u}, and \emph{lcm\_orbitals.u}.

\item[\Key{OUTPUT}] \verb| | \newline
\verb|<Name of output file>| \newline
The name of the output plt-file where the calculated values at each grid point are saved. Examples are given below.

\item[\Key{DENS}]\verb| | \newline
Construct .plt file for electron density at grid points.
For example to calculate electron density from the file  \emph{dens.restart} and save information in file  dens.plt,
insert the following section in the LSDALTON.INP file used to generate the  \emph{dens.restart} file in the first place:
 \newline
**PLT \newline
.INPUT \newline
dens.restart \newline
.OUTPUT \newline
dens.plt \newline
.DENS \newline

\item[\Key{EP}]\verb| | \newline
Construct .plt file for electrostatic potential grid points based on input density matrix.
For example to calculate electrostatic potential from the density in the the file  \emph{dens.restart} and save information in file  ep.plt,
insert the following section in the LSDALTON.INP file used to generate the  \emph{dens.restart} file in the first place:
 \newline
**PLT \newline
.INPUT \newline
dens.restart \newline
.OUTPUT \newline
ep.plt \newline
.EP \newline


\item[\Key{ORB}]\verb| | \newline
\verb|<Orbital index>|\newline
Construct .plt file for specific orbital at grid points. For example to calculate values of orbital number 8 at grid points from the file  \emph{lcm\_orbitals.u} and save information in file orb8.plt, insert the following section in the LSDALTON.INP file used to generate the  \emph{lcm\_orbitals.u} file in the first place: \newline
**PLT \newline
.INPUT \newline
lcm\_orbitals.u \newline
.OUTPUT \newline
orb8.plt \newline
.ORB \newline
8

\item[\Key{CHARGEDIST}]\verb| | \newline
\verb|<Orbital index 1     Orbital index 2>|\newline
Construct .plt file for charge distribution of two orbitals at grid points. 
For example to calculate charge distribution between orbitals 5 and 8 at grid points from the file  \emph{lcm\_orbitals.u} and save information in file cd\_5\_8.plt,
insert the following section in the LSDALTON.INP file used to generate the  \emph{lcm\_orbitals.u} file in the first place::
 \newline
**PLT \newline
.INPUT \newline
lcm\_orbitals.u \newline
.OUTPUT \newline
 cd\_5\_8.plt \newline
.CHARGEDIST \newline
5  8

\end{description}



\section{**PLTGRID}\label{sec:pltgrid}
Definition of 3-dimensinal grid used for generation of .plt files in **PLT section.
The default grid choice should suffice for most visualization purposes. This section is only for the advanced user who wishes to modify the default choice of grid.
This also modifies the grid used when .plt files for local orbitals are generated on the run, see Section \ref{subsec:orbloc}.

The grid box is defined in the following manner:
\begin{itemize}
\item
The first point in the grid box is (X1,Y1,Z1).                                                 
\item
The remaining grid points are then defined by going out in the X, Y, and Z directions with step sizes deltax, deltay, and deltaz, until there are nX, nY, and nZ points in the X,Y, and Z directions (giving a total number of gridpoints $nX\times nY \times nZ$).      
\end{itemize}

There are two options to define the grid box: (i) manual gridbox where X1, Y1, Z1, deltax, deltay, deltaz, nX, nY, nZ are defined explicitly in the input file; (ii)
molecule-specific gridbox where all atoms are contained within the grid box AND there is an additional buffer zone around the outermost atoms in the X, Y, and Z directions. Option (ii) thus requires that the deltax, deltay, deltaz, and buffer values are defined by the input, which implicitly defines X1, Y1, Z1, nX, nY, and nZ based on the molecular structure.

The default gridbox uses option (ii) with {deltax=deltay=deltaz=0.3 a.u.} and {buffer=6.0 a.u}. 
User-defined grid box based on options (i) and (ii) use the following inputs:


\begin{description}
\item[\Key{MANUAL}] \verb| | \newline
\verb|<X1   Y1   Z1>| \newline
\verb|<deltax    deltay    deltaz>| \newline
\verb|<nX    nY    nZ>| \newline
Manual definition of gridbox as described above. All input values are in a.u.

\item[\Key{MOLECULE}] \verb| | \newline
\verb|<buffer>| \newline
\verb|<deltax    deltay    deltaz>| \newline
Molecule-specifc definition of gridbox as described above. All input values are in a.u.

\end{description}


%\chapter{Integral evaluation, {\her}}\label{ch:hermit}

\section{General}\label{sec:herminp}

    {\her} is the integral evaluation part of the code. In ordinary
calculations there is no need to think about integral evaluation, as
this will be automatically taken care of by the program. However,
{\her} has an extensive set of atomic one- and two-electron
integrals\index{one-electron integral}\index{two-electron integral},
and some users may find it useful to generate explicit integrals using
{\her}. This is for instance necessary if the {\resp} program (dynamic
properties) is to be
used, as described in Chapter~\ref{ch:response}. Disk usage may also
be reduced by not calculating the
supermatrix\index{supermatrix}, and this is also controlled in the
\Sec{*INTEGRALS} input section.

It is worth noticing that the two-electron part of {\her} is actually
two integral programs. {\twoint} is the more general one and is invoked by
default in sequential calculations; {\eri} (Electron Repulsion
Integrals) is a highly vectorized two-electron integral code with
orientation towards integral distributions. {\eri} is
invoked by default in integral-direct coupled cluster calculations and
may in other cases be invoked by specifying the \Key{RUNERI} keyword
in the \Sec{*DALTON INPUT} input section, which ensures that {\eri}
rather than {\twoint} is called whenever {\eri} has the required
functionality. {\dalton} will, however, automatically revert to
{\twoint} for any two-electron integral not available in {\eri}:
{{\eri} cannot be used in parallel
calculations and it contains only first-derivatives (for
the Hartree--Fock gradient), thus it cannot be used to calculate
Hessians or to calculate MCSCF gradients.}

    Input to integral evaluation is
indicated by the keyword \Sec{*INTEGRALS}, and the section may be
ended with \Sec{END OF} or any keyword starting with two stars
(like {\it e.g.\/} \Sec{*WAVE FUNCTIONS}). The intermediate input is
divided into two sections: one general input section describing
what molecular integrals are to be evaluated, and then a set of
modules controlling the different parts of the calculation of
atomic integrals and the (possible) formation of
a supermatrix as defined in for instance Ref.~\cite{pemsjaahborjcp74}.


\section{\Sec{*INTEGRALS} directives}\label{sec:herinp}

          The following directives may be included in the input to
the integral evaluation.  They are organized according to the program
section names in which they can appear.

\subsection{End of input: \Sec{END OF}}

The last directive of the \Sec{*INTEGRALS} input may be \Sec{END OF}.

\subsection{General: \Sec{*INTEGRALS}}



General-purpose directives are given in the \Sec{*INTEGRALS}
section. This mainly includes requests for different atomic integrals,
as well as some
directives affecting the outcome of such an integral evaluation. Note
that although not explicitly stated, {\em none of the test options
work with symmetry}.

For all atomic integrals, the proper expression for the integral is
given, together with the labels written on the file
\texttt{AOPROPER}\index{AOPROPER}\index{integral label}, for
reference in later stages of a {\dalton} calculation (like for instance
in during the evaluation of dynamic response properties, or for
non-{\dalton} programs).

We also note that as long as any single atomic property
integral\index{property integral} is
requested in this module, the overlap integrals will also be
calculated. Note also, that unless the H\"{u}ckel starting guess is turned
off, this overlap matrix will not only be calculated for the requested
basis set, but also for a ``ghost'' ano-4 basis set appended to the
original set in order to do the H\"{u}ckel starting guess.

\begin{description}
\item[\Key{1ELPOT}] One-electron potential energy integrals.

\begin{list}{}{}
\item Integral:
$\sum_K\left<\chi_{\mu}\left|\frac{Z_K}{r_{K}}\right|\chi_{\nu}\right>$ 
\item Property label: \verb|POTENERG|
\end{list}

\item[\Key{AD2DAR}]\verb| |\newline
\verb|READ (LUCMD,*) DARFAC|

\begin{list}{}{}
\item Integral:
$\frac{\alpha^2}{4}\left<\chi_{\mu}\chi_{\nu}\left|\delta\left({\mathbf
r}_{12}\right)\right|\chi_{\rho}\chi_\sigma\right>$ 
\end{list}

Add two-electron Darwin integrals to the standard electron-repulsion
integrals with a perturbation factor \verb|DARFAC|.

\item[\Key{ANGLON}] Contribution to the one-electron contribution of
the magnetic moment\index{magnetic moment} using London orbitals
\index{London orbitals} arising from the differentiation of London-orbital transformed Hamiltonian, see Ref.~\cite{thpjjcp95}.

\begin{list}{}{}
\item Integral: $\left<\chi_{\mu}\left|{\bf
L}_N\right|\chi_{\nu}\right>$
\item Property labels: \verb|XANGLON |, \verb|YANGLON |, \verb|ZANGLON |
\end{list}

\item[\Key{ANGMOM}] Angular momentum\index{angular momentum} around the molecular origin.
This can be adjusted by changing the gauge origin through the use of
the \Key{GAUGEO} keyword.

\begin{list}{}{}
\item Integral: $\left<\chi_{\mu}\left|{\bf L}_{O}\right|\chi_{\nu}\right>$
\item Property labels: \verb|XANGMOM |, \verb|YANGMOM |, \verb|ZANGMOM |
\end{list}

\item[\Key{CARMOM}]\verb| |\newline
\verb|READ (LUCMD,*) IORCAR|

Cartesian multipole integrals\index{multipole integral} to order
\verb|IORCAR|. Read one more
line specifying order. See also the keyword \Key{SPHMOM}.

\begin{list}{}{}
\item Integral:
$\left<\chi_{\mu}\left|x^{i}y^{j}z^{k}\right|\chi_{\nu}\right>$
\item Property labels: \verb|CMiijjkk|
\end{list}
where $ii+jj+kk =$\verb|IORDER|, and where \verb|ii| = (i/10)*10+mod(i,10).

\item[\Key{CM-1}]\verb| |\newline
\verb|READ (LUCMD, '(A7)') FIELD1|

First derivative of the  electric dipole operator\index{electric dipole}
with respect to an external magnetic field\index{magnetic field}
due to differentiation of the London phase
factors, see Ref.~\cite{arthkrabmjpjjcp102}. Read one more line giving
the direction of the electric field~(A7)\index{electric field!external}. These 
include \verb|X-FIELD|, \verb|Y-FIELD|, and \verb|Z-FIELD|.

\begin{list}{}{}
\item Integral: $Q_{MN}\left<\chi_{\mu}\left|{\bf
r}D\right|\chi_{\nu}\right>$
\item Property labels: \verb|D-CM1 X |, \verb|D-CM1 Y |, \verb|D-CM1 Z |
\end{list}
where $D$ is the direction of the applied electric field as specified in
the input.

\item[\Key{CM-2}]\verb| |\newline
\verb|READ (LUCMD, '(A7)') FIELD2|

Second derivative electric dipole\index{electric dipole} operator
with respect to an external magnetic field\index{magnetic field} due
to differentiation of
the London phase factors, see Ref.~\cite{arthkrabmjpjjcp102}. Read one
more line giving the direction of the electric
field~(A7)\index{electric field!external}. These
include \verb|X-FIELD|, \verb|Y-FIELD|, and \verb|Z-FIELD|.

\begin{list}{}{}
\item Integral: $Q_{MN}\left<\chi_{\mu}\left|{\bf
rr}^{T}D\right|\chi_{\nu}\right>Q_{MN}$
\item Property labels: \verb|D-CM2 XX|, \verb|D-CM2 XY|, \verb|D-CM2 XZ|,
\verb|D-CM2 YY|, \verb|D-CM2 YZ|, \verb|D-CM2 ZZ|
\end{list}
where $D$ is the direction of the applied electric field as specified in
the input.

\item[\Key{DARWIN}] One-electron Darwin integrals\index{Darwin
integral}~\cite{skjohjajjcp103}.

\begin{list}{}{}
\item Integral: $\frac{\pi\alpha^{2}}{2}\left<\chi_{\mu}\left|\delta\left({\bf
r}\right)\right|\chi_{\nu}\right>$
\item Property label: \verb|DARWIN  |
\end{list}

\item[\Key{DCCR12}] Only required for interfaces to other
  implementations of the R12 approach. Obsolete, do not use.

\item[\Key{DEROVL}] Geometrical first derivatives of overlap
integrals.

\begin{list}{}{}
\item Integral: $\frac{\partial}{\partial {\mathbf
R}_K}\left<\chi_{\mu}\mid\chi_{\nu}\right>$ 
\item Property labels: \verb|1DOVLxyz|
\end{list}
where $xyz$ is the symmetry adapted nuclear coordinate.

\item[\Key{DERHAM}] Geometrical first derivatives of the one-electron
Hamiltonian matrix.

\begin{list}{}{}
\item Integral: $\frac{\partial}{\partial {\mathbf
R}_K}\left<\chi_{\mu}\left|\sum_K\frac{Z_K}{r_K}-\frac{1}{2}\nabla^2\right|\chi_{\nu}\right>$ 
\item Property labels: \verb|1DHAMxyz|
\end{list}
where $xyz$ is the symmetry adapted nuclear coordinate.

\item[\Key{DIASUS}] Diamagnetic magnetizability\index{diamagnetic
magnetizability} integrals, as
calculated with London atomic orbitals, see
Ref.~\cite{thpjjcp95}. It is calculated as the sum of the three
contributions \verb|DSUSLH|, \verb|DSUSLL|, and \verb|DSUSNL|.

\begin{list}{}{}
\item Integral: $\frac{1}{4}\left[\left<\chi_{\mu}\left|r^{2}_{N}I - {\bf r}_{N}
{\bf r}_{N}^{T}\right|\chi_{\nu}\right> +
\overline{Q_{MN}\left<\chi_{\mu}\left|{\bf r}{\bf
L}_{N}^{T}\right|\chi_{\nu}\right>} +
Q_{MN}\left<\chi_{\mu}\left|{\bf r}{\bf
r}^{T}h\right|\chi_{\nu}\right>Q_{MN}\right]$
\item Property label: \verb|XXdh/dB2|, \verb|XYdh/dB2|,
\verb|XZdh/dB2|, \verb|YYdh/dB2|, \verb|YZdh/dB2|, \verb|ZZdh/dB2|
\end{list}

\item[\Key{DIPGRA}]

Calculate dipole gradient integrals, that is, the geometrical first
derivatives of the dipole length integrals.

\begin{list}{}{}
\item Integral: $\frac{\partial}{\partial {\mathbf
R}_K}\left<\chi_{\mu}\left|{\mathbf r}\right|\chi_{\nu}\right>$ 
\item Property labels: \verb|abcDPG d|
\end{list}
where $abc$ is the symmetry adapted nuclear coordinate, and $d$ the
direction (x/y/z) of the dipole moment.

\item[\Key{DIPLEN}] Dipole length\index{dipole length} integrals.

\begin{list}{}{}
\item Integral: $\left<\chi_{\mu}\left|{\bf r}\right|\chi_{\nu}\right>$
\item Property labels: \verb|XDIPLEN |, \verb|YDIPLEN |, \verb|ZDIPLEN |
\end{list}

\item[\Key{DIPORG}]\verb| |\newline
\verb|READ (LUCMD, *) (DIPORG(I), I = 1, 3)|

Specify the dipole origin\index{dipole origin} to be used in the
calculation.
Read one more line containing the three Cartesian
components in bohrs (*). Default is~(0,0,0).

\item[\Key{DIPVEL}] Dipole velocity\index{dipole velocity} integrals.

\begin{list}{}{}
\item Integral: $\left<\chi_{\mu}\left|{\bf \nabla}\right|\chi_{\nu}\right>$
\item Property label: \verb|XDIPVEL |, \verb|YDIPVEL |, \verb|ZDIPVEL |
\end{list}

\item[\Key{DNS-KE}] Kinetic-energy correction to the diamagnetic
contribution to nuclear shielding constants with a common gauge
origin, see Ref.~\cite{pmpljvkrjcp119}.

\begin{list}{}{}
\item Integral: $\frac{3}{4}\left<\chi_{\mu}\left|\left[\nabla^2,\frac{{\bf r}_O^T{\bf r}_K -
{\bf r}_O{\bf r}_K^T}{r_{K}^3}\right]_+\right|\chi_{\nu}\right>$ 
\item Property label: \verb|abcNSKEd|, where \verb|abc| is the number
of the symmetry-adapted nuclear magnetic moment coordinate, and
\verb|d| refers to the x, y, or z component of the magnetic field. O
is the gauge origin.
\end{list}

\item[\Key{DPTOVL}] DPT (Direct Perturbation Theory) integrals: Small-component one-electron
overlap integrals.

\begin{list}{}{}
\item Integral: $\left<\chi_{\mu}\left|\frac{\partial^2}{\partial {\mathbf
r}^2}\right|\chi_{\nu}\right>$ 
\item Property labels: \verb|dd/dxdx|, \verb|dd/dxdy|,
\verb|dd/dxdz|, \verb|dd/dydy|, \verb|dd/dydz|, \verb|dd/dzdz|
\end{list}

\item[\Key{DPTPOT}] DPT  (Direct Perturbation Theory) integrals: Small-component one-electron
potential energy integrals.

\begin{list}{}{}
\item Integral:$\left<\chi_{\mu}\left|\frac{\partial}{\partial {\mathbf
r}}\frac{1}{\mathbf{R}_K}\frac{\partial}{\partial {\mathbf
r}}\right|\chi_{\nu}\right>$ 
\item Property labels: \verb|DERXXPVP|, \verb|DERXY+YX|,
\verb|DERXZ+ZX|, \verb|DERYY|, \verb|DERYZ+ZY|, \verb|DERZZ|
\end{list}

\item[\Key{DPTPXP}] DPT  (Direct Perturbation Theory) integrals: Small-component dipole length
  integrals for Direct Perturbation Theory.

\begin{list}{}{}
\item Integral:$\frac{1}{4}\left<\nabla\chi_{\mu}\left|{\mathbf
r}\right|\nabla\chi_{\nu}\right>$ 
\item Property labels: \verb|PXPDIPOL|, \verb|PYPDIPOL|,
\verb|PZPDIPOL|.
\end{list}

\item[\Key{DSO}] Diamagnetic spin-orbit\index{diamagnetic spin-orbit}
integrals. These are
calculated using Gaussian quadrature\index{Gaussian quadrature} as
described in
Ref.~\cite{dmtajcp73}. The number of quadrature point is controlled by
the keyword \Key{POINTS}.

\begin{list}{}{}
\item Integral: $\left<\chi_{\mu}\left|\frac{{\bf r}_K^T{\bf r}_LI - {\bf r}_K{\bf r}_L^T}{r_K^3r_L^3}\right|\chi_{\nu}\right>$
\item Property labels: \verb|DSO abcd| where \verb|ab| is the symmetry
coordinate of a given component for the symmetry-adapted nucleus K,
and \verb|cd| is in a similar fashion the symmetry coordinate for the
symmetry-adapted nucleus L.
\end{list}

\item[\Key{DSO-KE}] Kinetic energy correction to the diamagnetic
  spin-orbit\index{diamagnetic spin-orbit} integrals. These are
calculated using Gaussian quadrature\index{Gaussian quadrature} as
described in
Ref.~\cite{dmtajcp73}. The number of quadrature point is controlled by
the keyword \Key{POINTS}. Please note that this integral has not been
  extensively tested, and the use of this integral is at the risk of
  the user.

\begin{list}{}{}
\item Integral: $\left<\chi_{\mu}\left|\left[\nabla^2,\frac{{\bf r}_K^T{\bf r}_LI - {\bf r}_K{\bf r}_L^T}{r_K^3r_L^3}\right]_+\right|\chi_{\nu}\right>$
\item Property labels: \verb|DSOKabcd| where \verb|ab| is the symmetry
coordinate of a given component for the symmetry-adapted nucleus K,
and \verb|cd| is in a similar fashion the symmetry coordinate for the
symmetry-adapted nucleus L.
\end{list}

\item[\Key{DSUSLH}] The contribution to diamagnetic
magnetizability\index{diamagnetic magnetizability}
integrals from the differentiation of the London orbital\index{London
orbitals}
phase-factors, see Ref.~\cite{thpjjcp95}.

\begin{list}{}{}
\item Integral:$\frac{1}{4}Q_{MN}\left<\chi_{\mu}\left|{\bf
r}{\bf r}^{T}h\right|\chi_{\nu}\right>Q_{MN}$
\item Property labels: \verb|XXDSUSLH|, \verb|XYDSUSLH|,
\verb|XZDSUSLH|, \verb|YYDSUSLH|, \verb|YZDSUSLH|, \verb|ZZDSUSLH|
\end{list}

\item[\Key{DSUSLL}] The contribution to the diamagnetic
magnetizability\index{diamagnetic magnetizability} integrals from mixed differentiation on the Hamiltonian
and the London orbital phase factors\index{London orbitals}, see
Ref.~\cite{thpjjcp95}.

\begin{list}{}{}
\item Integral: $\frac{1}{4}\overline{Q_{MN}\left<\chi_{\mu}\left|{\bf
r}{\bf L}_{N}^{T}\right|\chi_{\nu}\right>}$
\item Property labels: \verb|XXDSUSLL|, \verb|XYDSUSLL|,
\verb|XZDSUSLL|, \verb|YYDSUSLL|, \verb|YZDSUSLL|, \verb|ZZDSUSLL|
\end{list}

\item[\Key{DSUSNL}] The contribution to the diamagnetic
magnetizability\index{diamagnetic magnetizability} integrals using
London orbitals\index{London orbitals} but with
contributions from the differentiation of the Hamiltonian only, see
Ref.~\cite{thpjjcp95}.

\begin{list}{}{}
\item Integral: $\frac{1}{4}\left<\chi_{\mu}\left|r^{2}_{N}I - {\bf r}_{N}
{\bf r}_{N}^{T}\right|\chi_{\nu}\right>$
\item Property labels: \verb|XXDSUSNL|, \verb|XYDSUSNL|,
\verb|XZDSUSNL|, \verb|YYDSUSNL|, \verb|YZDSUSNL|, \verb|ZZDSUSNL|
\end{list}

\item[\Key{DSUTST}] Test of the diamagnetic magnetizability
integrals\index{diamagnetic magnetizability} with London atomic
orbitals\index{London orbitals}. Mainly for debugging purposes.

\item[\Key{EFGCAR}] Cartesian electric field gradient
integrals\index{electric field!gradient}.

\begin{list}{}{}
\item Integral: $\frac{1}{3}\left<\chi_{\mu}\left|\frac{3{\bf
r}_K{\bf r}_K^T - {\bf r}_K^T{\bf r}_KI}{r_K^5}\right|\chi_{\nu}\right>$
\item Property labels: \verb|xyEFGabc|, where \verb|x| and \verb|y| are
the Cartesian directions, \verb|abc| the number of the symmetry
independent center, and \verb|c| that centers c'th symmetry-generated
atom.
\end{list}

\item[\Key{EFGSPH}] Spherical electric field gradient
\index{electric field!gradient} integrals. Obtained by transforming
the Cartesian electric-field gradient integrals (see \Key{EFGCAR}) to
spherical basis.

\item[\Key{ELGDIA}] Diamagnetic one-electron spin-orbit integrals
without London orbitals.

\begin{list}{}{}
\item Integral: 
\item Property labels: \verb|D1-SO XX|, \verb|D1-SO XY|, \verb|D1-SO XZ|, \verb|D1-SO YX|, \verb|D1-SO YY|, \verb|D1-SO YZ|, \verb|D1-SO ZX|, \verb|D1-SO ZY|, \verb|D1-SO ZZ|
\end{list}

\item[\Key{ELGDIL}] Diamagnetic one-electron spin-orbit integrals
with London orbitals.

\begin{list}{}{}
\item Integral: 
\item Property labels: \verb|D1-SOLXX|, \verb|D1-SOLXY|, \verb|D1-SOLXZ|, \verb|D1-SOLYX|, \verb|D1-SOLYY|, \verb|D1-SOLYZ|, \verb|D1-SOLZX|, \verb|D1-SOLZY|, \verb|D1-SOLZZ|
\end{list}

\item[\Key{EXPIKR}]\verb| |\newline
\verb|READ (LUCMD, *) (EXPKR(I), I = 1, 3)|

Cosine and sine\index{cosine integral}\index{sine integral} integrals.
Read one more line containing the wave numbers in the three Cartesian
directions. The center of expansion is always~(0,0,0).

\begin{list}{}{}
\item Integral: 
\item Property labels: \verb|COS KX/K|, \verb|COS KY/K|,
  \verb|COS KZ/K|, \verb|SIN KX/K|, \verb|SIN KY/K|, \verb|SIN KZ/K|.
\end{list}

\item[\Key{FC}] Fermi contact integrals,
\index{Fermi contact integrals}\index{integrals!Fermi contact}
see Ref.~\cite{ovhapjhjajsbpthjcp96}.

\begin{list}{}{}
\item Integral: $\frac{4\pi g_e}{3}\left<\chi_{\mu}\left|\delta\left({\bf
r}_K\right)\right|\chi_{\nu}\right>$
\item Property labels: \verb|FC NAMab|, where \verb|NAM| is the three
first letters in the name of this atom, as given in the
\verb|MOLECULE.INP| file, and \verb|ab| is the number of the
symmetry-adapted nucleus.
\end{list}

\item[\Key{FC-KE}] Kinetic energy correction to Fermi-contact integrals,
\index{Fermi contact integrals!kinetic energy correction}
see Ref.~\cite{pmpljvkrjcp119}.

\begin{list}{}{}
\item Integral: $\frac{2\pi g_e}{3}\left<\chi_{\mu}\left|\left[\nabla^2,\delta\left({\bf
r}_K\right)\right]_+\right|\chi_{\nu}\right>$
\item Property labels: \verb|FCKEnacd|, where \verb|na| is the two
first letters in the name of this atom, as given in the
\verb|MOLECULE.INP| file, and \verb|cd| is the number of the
symmetry-adapted nucleus.
\end{list}

\item[\Key{FINDPT}]\verb| |\newline
\verb|READ (LUCMD, *) DPTFAC|

A direct relativistic perturbation is added to the
  Hamiltonian and metric with the perturbation parameter DPTFAC, where
  the actually applied perturbation is DPTFAC*$\alpha_{fs}^2$.

\item[\Key{GAUGEO}]\verb| |\newline
\verb|READ (LUCMD, *) (GAGORG(I), I = 1, 3)|

Specify the gauge origin\index{gauge origin} to be used in the
calculation. Read one more line containing the three Cartesian
components (*). Default is~(0,0,0).

\item[\Key{HBDO}] Symmetric combination of half-differentiated
  overlap matrix with respect to an external magnetic field
  perturbation when London orbitals are used.

\begin{list}{}{}
\item Integral: $-\frac{1}{4}\left(Q_{MO} +
Q_{NO}\right)\left<\chi_{\mu}\left|{\mathbf r}\right|\chi_{\nu}\right>$
\item Property labels: \verb| HBDO X |, \verb| HBDO Y |, \verb| HBDO Z |.
\end{list}

\item[\Key{HDO}] Symmetrized, half-differentiated
overlap integrals with respect to geometric
distortions\index{overlap!half-differentiated}, see
Ref.~\cite{klbpjhjajjothjcp97}. Differentiation on the ket-vector.

\begin{list}{}{}
\item Integral: $\left<\frac{\partial \chi_{\mu}}{\partial
R_{ab}}\mid\chi_{\nu}\right> -
\left<\chi_{\mu}\mid\frac{\partial\chi_{\nu}}{\partial R_{ab}}\right>$
\item Property label: \verb|HDO abc |, where \verb|abc| is the number
of the symmetry-adapted coordinate being differentiated.
\end{list}

\item[\Key{HDOBR}] Geometric half-differentiated overlap
matrix\index{overlap!half-differentiated}
differentiated once more on the ket-vector with respect to an external
magnetic field, see Ref.~\cite{klbpjthkrhjajjcp98}.

\begin{list}{}{}
\item Integral: $-\frac{1}{2}Q_{NO}\left<\frac{\partial
\chi_{\mu}}{\partial R_K}\left|{\mathbf r}\right|\chi_{\nu}\right>$
\item Property labels: \verb|abcHBD d|, where \verb|abc| is the
symmetry coordinate of the nuclear coordinate being differentiations,
and \verb|d| is the coordinate of the external magnetic field.
\end{list}

\item[\Key{HDOBRT}] Test the calculation of the \Key{HDOBR}
integral. Mainly for debugging purposes.

\item[\Key{INPTES}] Test the correctness of the \Sec{*INTEGRALS}-input. Mainly
for debugging purposes, but also a good option to check if the \mol\ input
has been typed in correctly.

\item[\Key{KINENE}] Kinetic energy integrals\index{kinetic
energy}. Note however, that the kinetic energy integrals used in the
wave function optimization is generated in the \Sec{ONEINT} section.

\begin{list}{}{}
\item Integral:
$\frac{1}{2}\left<\chi_{\mu}\left|\nabla^{2}\right|\chi_{\nu}\right>$
\item Property label: \verb|KINENERG|.
\end{list}

\item[\Key{LONMOM}] Contribution to the London magnetic
moment\index{magnetic moment} from
the differentiation with respect to magnetic field on the London
orbital\index{London orbitals} phase factors, see Ref.~\cite{thpjjcp95}.

\begin{list}{}{}
\item Integral:
$\frac{1}{4}Q_{MN}\left<\chi_{\mu}\left|{\bf r}h\right|\chi_{\nu}\right>$
\item Property labels: \verb|XLONMOM |, \verb|YLONMOM |, \verb|ZLONMOM |.
\end{list}

\item[\Key{MAGMOM}] One-electron contribution to the magnetic
moment\index{magnetic moment}
around the nuclei to which the
atomic orbitals are attached. This is the London atomic
orbital\index{London orbitals}
magnetic moment\index{magnetic moment} as defined in Eq.~(35) of
Ref.~\cite{krthklbpjhjajjcp99}. The integral is calculated as the sum of \Key{LONMOM} and \Key{ANGLON}.

\begin{list}{}{}
\item Integral:
$\left<\chi_{\mu}\left|{\bf L}_N + \frac{1}{4}Q_{MN}{\bf
r}h\right|\chi_{\nu}\right>$
\item Property label: \verb|dh/dBX  |, \verb|dh/dBY  |, \verb|dh/dBZ  |.
\end{list}

\item[\Key{MASSVE}] Mass-velocity\index{mass-velocity} integrals.

\begin{list}{}{}
\item Integral:
$\frac{\alpha^2}{8}\left<\chi_{\mu}\left|\nabla^{2}\cdot\nabla^2\right|\chi_{\nu}\right>$
\item Property label: \verb|MASSVELO|.
\end{list}

\item[\Key{MGMO2T}] Test of two-electron integral contribution to
magnetic moment.

\item[\Key{MGMOMT}] Test the calculation of the \Key{MAGMOM}
integrals.

\item[\Key{MGMTHR}]\verb| |\newline
\verb|READ (LUCMD, *) PRTHRS|

Set the threshold for which two-electron integrals should be tested
 with the keyword \Key{MGMO2T}. Default is~10$^{-10}$.

\item[\Key{MNF-SO}] Calculates the atomic mean-field spin-orbit
 integrals as described in Ref.~\cite{bahcmmuwogcpl251}. As the
 calculation of these 
 integrals require a proper description of the atomic states, reliable
 results can only be expected for generally contracted basis sets such
 as the ANO sets, and in some cases also the correlation-consistent
 basis sets ((aug-)cc-p(C)VXZ)).

\item[\Key{NELFLD}] Nuclear electric field integrals\index{electric
 field at nucleus}.

\begin{list}{}{}
\item Integral:
$\left<\chi_{\mu}\left|\frac{{\bf r}_K}{r_{K}^3}\right|\chi_{\nu}\right>$
where $K$ is the nucleus of interest.
\item Property labels: \verb|NEF abc |, where \verb|abc| is the number
of the symmetry-adapted nuclear coordinate.
\end{list}

\item[\Key{NO HAM}] Do not calculate ordinary one- and two-electron
  Hamiltonian integrals.

\item[\Key{NO2SO}] Do not calculate two-electron contribution to
  spin--orbit integrals.

\item[\Key{NOPICH}] Do not add direct perturbation theory correction
  to Hamiltonian integral, see keyword~\Key{FINDPT}.

\item[\Key{NOSUP}] Do not calculate the supermatrix\index{supermatrix}
integral file.
This may be required in order to reduce the amount of disc space used
in the calculation (to approximately one-third before entering the
evaluation of molecular properties).
Note however, that this will increase the time used for the evaluation
of the wave function
significantly in ordinary Hartree--Fock runs. It is default for direct and
parallel calculations.

\item[\Key{NOTV12}] Obsolete keyword, do not use.

\item[\Key{NOTWO}] Only calculate the one-electron part of the
Hamiltonian integrals. It is default for direct and parallel calculations.

\item[\Key{NPOTST}] Test of the nuclear potential integrals
calculated with the keyword \Key{NUCPOT}. Mainly for debugging
purposes.

\item[\Key{NSLTST}] Test of the integrals calculated with the
keyword \Key{NSTLON}. Mainly for debugging purposes.

\item[\Key{NSNLTS}] Test of the integrals calculated with the
keyword \Key{NSTNOL}. Mainly for debugging purposes.

\item[\Key{NST}] Calculate the one-electron contribution to the
diamagnetic nuclear shielding\index{diamagnetic nuclear shielding}
tensor integrals using London atomic
orbitals\index{London orbitals}, see Ref.~\cite{thpjjcp95}. It is
calculated as the sum of \verb|NSTLON| and \verb|NSTNOL|.

\begin{list}{}{}
\item Integral:
$\frac{1}{2}\left<\chi_{\mu}\left|\frac{{\bf r}_N^T{\bf r}_K -
{\bf r}_N{\bf r}_K^T}{r_{K}^3} + Q_{MN}\frac{{\bf r}_N^T{\bf
l}_K}{r_{K}^3}\right|\chi_{\nu}\right>$
where $K$ is the nucleus of interest.
\item Property label: \verb|abcNST d|, where \verb|abc| is the number
of the symmetry-adapted nuclear magnetic moment coordinate, and
\verb|d| refers to the x, y, or z component of the magnetic field.
\end{list}

\item[\Key{NSTCGO}] Calculate the diamagnetic nuclear shielding\index{diamagnetic nuclear shielding}
tensor integrals without using London atomic orbitals\index{London
orbitals}. Note that the gauge origin is controlled by the keyword
\Key{GAUGEO}.

\begin{list}{}{}
\item Integral:
$\frac{1}{2}\left<\chi_{\mu}\left|\frac{{\bf r}_O^T{\bf r}_K -
{\bf r}_O{\bf r}_K^T}{r_{K}^3}\right|\chi_{\nu}\right>$
where $K$ is the nucleus of interest.
\item Property label: \verb|abcNSCOd|, where \verb|abc| is the number
of the symmetry-adapted nuclear magnetic moment coordinate, and
\verb|d| refers to the x, y, or z component of the magnetic field. O
is the gauge origin.
\end{list}

\item[\Key{NSTLON}] Calculate the contribution to the London orbital nuclear
shielding tensor\index{diamagnetic nuclear shielding} from the differentiation of the London orbital
phase-factors\index{London orbitals}, see Ref.~\cite{thpjjcp95}.

\begin{list}{}{}
\item Integral:
$\frac{1}{2}Q_{MN}\left<\chi_{\mu}\left|\frac{{\bf r}_N^T{\bf
l}_K}{r_{K}^3}\right|\chi_{\nu}\right>$
where $K$ is the nucleus of interest.
\item Property labels: \verb|abcNSLOd|, where \verb|abc| is the number
of the symmetry-adapted nuclear magnetic moment coordinate, and
\verb|d| refers to the x, y, or z component of the magnetic field.
\end{list}

\item[\Key{NSTNOL}] Calculate the contribution to the nuclear
shielding tensor when using London atomic orbitals\index{diamagnetic
nuclear shielding} from the 
differentiation of the Hamiltonian alone\index{London orbitals}, see Ref.~\cite{thpjjcp95}.

\begin{list}{}{}
\item Integral:
$\frac{1}{2}\left<\chi_{\mu}\left|\frac{{\bf r}_N^T{\bf r}_K -
{\bf r}_N{\bf r}_K^T}{r_{K}^3}\right|\chi_{\nu}\right>$
where $K$ is the nucleus of interest.
\item Property label: \verb|abcNSNLd|, where \verb|abc| is the number
of the symmetry-adapted nuclear magnetic moment coordinate, and
\verb|d| refers to the x, y, or z component of the magnetic field.
\end{list}

\item[\Key{NSTTST}] Test the calculation of the one-electron
diamagnetic  nuclear shielding\index{diamagnetic nuclear shielding}
tensor using London atomic orbitals\index{London orbitals}.

\item[\Key{NUCMOD}]\verb| |\newline
\verb|READ (LUCMD, *) INUC|

Choose nuclear model. A 1 corresponds to a point nucleus (which is the
default), and 2 corresponds to a Gaussian distribution model.

\item[\Key{NUCPOT}] Calculate the nuclear potential energy.
Currently this keyword can only be used in calculations not employing
symmetry.\index{potential energy!at nucleus}

\begin{list}{}{}
\item Integral:
$\left<\chi_{\mu}\left|\frac{Z_K}{r_{K}}\right|\chi_{\nu}\right>$
where $K$ is the nucleus of interest.
\item Property labels: \verb|POT.E ab|, where \verb|ab| are the two
first letters in the name of this nucleus. Thus note that in order
to distinguish between integrals, the first two letters in an
atom's name must be unique.
\end{list}


\item[\Key{OCTGRA}]
Calculate octupole gradient integrals, that is, the geometrical first
derivatives of the third moment integrals (\Key{THIRDM}).
(note: it is NOT the gradient of the \Key{OCTUPO} integrals).

\begin{list}{}{}
\item Integral: $\frac{\partial}{\partial {\mathbf
R}_K}\left<\chi_{\mu}\left|{\mathbf r}^3\right|\chi_{\nu}\right>$ 
\item Property labels: \verb|abODGcde|
\end{list}
where \verb|ab| is the symmetry adapted nuclear coordinate, and \verb|cde| the
component (x/y/z) of the third moment tensor. Currently, this integral
does not work with symmetry.

\item[\Key{OZ-KE}] Calculates the kinetic energy correction to the
  orbital Zeeman operator, see Ref.~\cite{pmpljvkrjcp119}.

\begin{list}{}{}
\item Integral: $\left<\chi_{\mu}\left|\left[\nabla^2,{\mathbf
    l}_O\right]_+\right|\chi_{\nu}\right>$  
\item Property labels: \verb|XOZKE   |, \verb|YOZKE   |, \verb|ZOZKE   |.
\end{list}

\item[\Key{PHASEO}]\verb| |\newline
\verb|READ (LUCMD, *) (ORIGIN(I), I = 1, 3)|

Set the origin appearing in the London atomic orbital phase-factors.
Read one more line containing the Cartesian components of this origin (*).
Default is~(0,0,0).

\item[\Key{POINTS}]\verb| |\newline
\verb|READ (LUCMD,*) NPQUAD|

Read the number of quadrature points\index{Gaussian quadrature} to be
used in the evaluation of
the diamagnetic spin-orbit\index{diamagnetic spin-orbit} integrals, as
requested by the keyword
\Key{DSO}. Read one more line containing the number of quadrature
points. Default is~40.

\item[\Key{PRINT}]\verb| |\newline
\verb|READ (LUCMD,*) IPRDEF|

Set default print level during the integral evaluation.  Read one more line
containing print level. Default is the value of \verb|IPRDEF|
from the general input module for {\dalton}.

\item[\Key{PROPRI}] Print all one-electron property integrals requested.

\item[\Key{PSO}] Paramagnetic spin-orbit integrals\index{paramagnetic
spin-orbit}, see
Ref.~\cite{ovhapjhjajsbpthjcp96}.

\begin{list}{}{}
\item Integral:
$\left<\chi_{\mu}\left|\frac{{\bf l}_K}{r_{K}^{3}}\right|\chi_{\nu}\right>$
where $K$ is the nucleus of interest.
\item Property label: \verb|PSO abc |, where \verb|abc| is the number
of the symmetry-adapted nuclear magnetic moment coordinate.
\end{list}

\item[\Key{PSO-KE}] Kinetic energy correction to the paramagnetic
  spin-orbit integrals\index{paramagnetic spin-orbit}, see
Ref.~\cite{pmpljvkrjcp119}.

\begin{list}{}{}
\item Integral:
$\left<\chi_{\mu}\left|\left[\nabla^2,\frac{{\bf
      l}_K}{r_{K}^{3}}\right]_+\right|\chi_{\nu}\right>$ 
where $K$ is the nucleus of interest.
\item Property label: \verb|PSOKEabc|, where \verb|abc| is the number
of the symmetry-adapted nuclear magnetic moment coordinate.
\end{list}

\item[\Key{PSO-OZ}] Orbital-Zeeman correction to the paramagnetic
  spin-orbit integrals\index{paramagnetic spin-orbit}, see
Ref.~\cite{pmpljvkrjcp119}.

\begin{list}{}{}
\item Integral:
$\left<\chi_{\mu}\left|\left[{\mathbf l}_O,\frac{{\bf
      l}_K}{r_{K}^{3}}\right]_+\right|\chi_{\nu}\right>$ 
where $K$ is the nucleus of interest.
\item Property label: \verb|abcPSOZd|, where \verb|abc| is the number
of the symmetry-adapted nuclear magnetic moment coordinate, and $d$ is
the direction (x/y/z) of the external magnetic field (corresponding to
the component of the orbital Zeeman operator).
\end{list}

\item[\Key{PVP}] Calculate the $pVp$ integrals that appear in the
Douglas--Kroll--He\ss\ transformation~\cite{bahpra33}.

\begin{list}{}{}
\item Integral:
$\left<\chi_{\mu}\left|\nabla\left(\sum_K\frac{Z_K}{r_iK}\right)\nabla\right|\chi_{\nu}\right>$ 
\item Property labels: \verb|pVpINTEG|.
\end{list}

\item[\Key{PVIOLA}] Parity-violating electroweak interaction.

\begin{list}{}{}
\item Integral:
\item Property labels: \verb|PVIOLA X|, \verb|PVIOLA Y|, \verb|PVIOLA Z|.
\end{list}

\item[\Key{QDBINT}]\verb| |\newline
\verb|READ (LUCMD,'(A7)') FIELD3|

London orbital corrections arising from the second-moment of charge
operator in finite-perturbation calculations involving an external
electric field gradient. Possible values for the perturbation (FIELD3)
may be XX/XY/XZ/YY/YZ/ZZ-FGRD.

\begin{list}{}{}
\item Integral: 
\item Property labels: \verb|ab-QDB X|, \verb|ab-QDB Y|,
  \verb|ab-QDB Z|, where $ab$ is the component of the electric field
  gradient operator read in the variable FIELD3.
\end{list}

\item[\Key{QDBTST}] Test of the \Key{QDBINT} integrals, mainly for
  debugging purposes.

\item[\Key{QUADRU}] Quadrupole moment integrals.
\index{quadrupole moment integrals}\index{integrals!quadrupole moment}
For traceless quadrupole moment integrals as
defined by Buckingham~\cite{adbacp12}, see the keyword \Key{THETA}.

\begin{list}{}{}
\item Integral: $\frac{1}{4}\left<\chi_{\mu}\left|
r^{2}_{O}I_3 - {\bf r}_{O}{\bf r}_{O}^{T}
%HJAAJ July 2001: acc. to hermit code
\right|\chi_{\nu}\right>$
\item Property label: \verb|XXQUADRU|, \verb|XYQUADRU|,
\verb|XZQUADRU|, \verb|YYQUADRU|, \verb|YZQUADRU|, \verb|ZZQUADRU|
\end{list}

\item[\Key{QUAGRA}]\verb| |\newline
Calculate quadrupole gradient integrals, that is, the geometrical first
derivatives of the second moment integrals
({\it i.e.\/} \Key{SECMOM}, note: it is NOT the gradient of the \Key{QUADRU} integrals).

\begin{list}{}{}
\item Integral: $\frac{\partial}{\partial {\mathbf
R}_K}\left<\chi_{\mu}\left|{\mathbf r}{\mathbf r}^T\right|\chi_{\nu}\right>$ 
\item Property labels: \verb|abcQDGde|
\end{list}
where \verb|abc| is the symmetry adapted nuclear coordinate, and \verb|de| the
component (xx/xy/xz/yy/yz/zz) of the second moment tensor. Currently
symmetry can not be used with these integrals.

\item[\Key{QUASUM}] Calculate all atomic integrals as square
matrices, irrespective of their inherent Hermiticity or
anti-Hermiticity.

\item[\Key{RANGMO}] Calculate the diamagnetic magnetizability integrals using the CTOCD-DZ method,
\index{CTOCD-DZ, CTOCD-DZ diamagnetic magnetizability} see Ref.~\cite{paololazz1,paololazz2}.
The gauge origin is, as default, in the center of mass. 

\begin{list}{}{}
\item Integral: $\left<\chi_{\mu}\left|{\bf
r}_O{\bf L}_{N}^{T}\right|\chi_{\nu}\right>$ 
\item Property label: \verb|XXRANG|, \verb|XYRANG|,
\verb|XZRANG|, \verb|YXRANG|, \verb|YYRANG|, \verb|YZRANG|, 
\verb|ZXRANG|, \verb|ZYRANG|, \verb|ZZRANG|
\end{list}

\item[\Key{R12}] Perform integral evaluation as required by the R12 method.
One-electron integrals for the R12 method (Cartesian multipole integrals up to order 2)
are precomputed and stored on the file \texttt{AOPROPER}. Two-electron integrals are
computed in direct mode.

\item[\Key{R12EXP}]\verb| |\newline
\verb|READ (LUCMD,*) GAMMAC|

Same as \Key{R12} but with Gaussian-damped linear $r_{12}$ terms of
the form $r_{12}\exp{-\gamma r_{12}^2}$. The value of $\gamma$ is read from the input line.

\item[\Key{R12INT}] Calculation of two-electron integrals over r12.

\item[\Key{ROTSTR}] Rotational strength integrals in the mixed
representation~\cite{tbphkkrjcp110}.

\begin{list}{}{}
\item Integral: $\left<\chi_{\mu}\left|\nabla\;{\mathbf r}^T +
{\mathbf r}\nabla^T\right|\chi_{\nu}\right>$
\item Property labels : \verb|XXROTSTR|, \verb|XYROTSTR|, \verb|XZROTSTR|, \verb|YYROTSTR|, \verb|YZROTSTR|, \verb|ZZROTSTR|.
\end{list}

\item[\Key{RPSO}] Calculate the diamagnetic nuclear shielding tensor 
integrals using the CTOCD-DZ method,
\index{CTOCD-DZ,CTOCD-DZ diamagnetic nuclear shieldings}  
see Ref.~\cite{paololazz1,paololazz2,ctocd}. 
The gauge origin is, as default, at the center of mass. 
Setting the gauge origin somewhere else will give wrong results in calculations using symmetry.

\begin{list}{}{}
\item Integral:
$\left<\chi_{\mu}\left|\frac{{\bf r}_O^T{\bf l}_K}{r_{K}^{3}}\right|\chi_{\nu}\right>$
where $K$ is the nucleus of interest.
\item Property label: \verb|abcRPSOd|, where \verb|abc| is the number
of the symmetry-adapted nuclear magnetic moment coordinate and \verb|d| refers to the
x, y, z component of the magnetic field
\end{list}


\item[\Key{S1MAG}] Calculate the first derivative overlap
matrix\index{overlap!magnetic field derivative} with respect to an
external magnetic field by differentiation
of the London phase factors\index{London orbitals}, see Ref.~\cite{thpjjcp95}.

\begin{list}{}{}
\item Integral: $\frac{1}{2}Q_{MN}\left<\chi_{\mu}\left|{\bf
r}\right|\chi_{\nu}\right>$
\item Property labels: \verb|dS/dBX  |, \verb|dS/dBY  |,
\verb|dS/dBZ  |
\end{list}

\item[\Key{S1MAGL}] Calculate the first magnetic half-differentiated overlap
matrix\index{overlap!magnetic field derivative} with respect to an
external magnetic field as needed with the
natural connection\index{natural connection}, see
Ref.~\cite{krthklbpjjocp195}. Differentiated
on the bra-vector.

\begin{list}{}{}
\item Integral: $\frac{1}{2}Q_{MO}\left<\chi_{\mu}\left|{\bf
r}\right|\chi_{\nu}\right>$
\item Property label: \verb%d<S|/dBX%, \verb%d<S|/dBY%,
\verb%d<S|/dBZ%
\end{list}

\item[\Key{S1MAGR}] Calculate the first magnetic half-differentiated overlap
matrix\index{overlap!magnetic field derivative} with respect to an external magnetic field as needed with the
natural connection\index{natural connection}, see Ref.~\cite{krthklbpjjocp195}. Differentiated on
the ket-vector.

\begin{list}{}{}
\item Integral: $\frac{1}{2}Q_{ON}\left<\chi_{\mu}\left|{\bf
r}\right|\chi_{\nu}\right>$
\item Property labels: \verb%d|S>/dBX%, \verb%d|S>/dBZ%,
\verb%d|S>/dBZ%
\end{list}

\item[\Key{S1MAGT}] Test the integrals calculated with the keyword
\Key{S1MAG}. Mainly for debugging purposes.

\item[\Key{S1MLT}] Test the integrals calculated with the keyword
\Key{S1MAGL}. Mainly for debugging purposes.

\item[\Key{S1MRT}] Test the integrals calculated with the keyword
\Key{S1MAGR}. Mainly for debugging purposes.

\item[\Key{S2MAG}] Calculate the second derivative of the overlap
matrix\index{overlap!magnetic field derivative} with respect to an external magnetic field by differentiation
of the London phase factors\index{London orbitals}, see Ref.~\cite{thpjjcp95}.

\begin{list}{}{}
\item Integral: $\frac{1}{4}Q_{MN}\left<\chi_{\mu}\left|{\bf r}{\bf
r}^{T}\right|\chi_{\nu}\right>Q_{MN}$
\item Property labels: \verb|dS/dB2XX|, \verb|dS/dB2XY|,
\verb|dS/dB2XZ|, \verb|dS/dB2YY|, \verb|dS/dB2YZ|, \verb|dS/dB2ZZ|
\end{list}

\item[\Key{S2MAGT}] Test the integrals calculated with the keyword
\Key{S2MAG}. Mainly for debugging purposes.

\item[\Key{SD}] Spin-dipole integrals,
\index{spin-dipole integrals}\index{integrals!spin-dipole},
see Ref.~\cite{ovhapjhjajsbpthjcp96}.

\begin{list}{}{}
\item Integral: $\frac{g_e}{2}\left<\chi_{\mu}\left|\frac{3{\bf
r}_K{\bf r}_K^T - r_K^2}{r_{K}^5}\right|\chi_{\nu}\right>$
\item Property label: \verb|SD abc d|, where \verb|abc| is the number
of the first symmetry-adapted coordinate (corresponding to
symmetry-adapted nuclear magnetic moments) and \verb|d| is the x, y,
or z component of the magnetic moment with respect to spin coordinates.
\end{list}

\item[\Key{SD+FC}] Calculate the sum of the spin-dipole\index{spin-dipole integrals} and
Fermi-contact integrals\index{Fermi contact integrals}
\index{integrals!spin-dipole plus Fermi-contact}.

\begin{list}{}{}
\item Integral: $\frac{g_e}{2}\left<\chi_{\mu}\left|\frac{3{\bf
r}_K{\bf r}_K^T - r_K^2}{r_{K}^5}\right|\chi_{\nu}\right> + \frac{4\pi
g_e}{3}\left<\chi_{\mu}\left|\delta\left({\bf
r}_K\right)\right|\chi_{\nu}\right>$
\item Property label: \verb|SDCabc d|, where \verb|abc| is the number
of the first symmetry-adapted coordinate (corresponding to
symmetry-adapted nuclear magnetic moments) and \verb|d| is the x, y,
or z component of the magnetic moment with respect to spin coordinates.
\end{list}

\item[\Key{SD-KE}] Kinetic energy correction to spin--dipole integrals
\index{spin-dipole integrals!kinetic energy correction}
\index{integrals!kinetic energy correction to spin-dipole},
see Ref.~\cite{pmpljvkrjcp119}.

\begin{list}{}{}
\item Integral: $\frac{g_e}{4}\left<\chi_{\mu}\left|\left[\nabla^2,\frac{3{\bf
r}_K{\bf r}_K^T - r_K^2}{r_{K}^5}\right]_+\right|\chi_{\nu}\right>$
\item Property label: \verb|SDKEab c|, where \verb|ab| is the number
of the first symmetry-adapted coordinate (corresponding to
symmetry-adapted nuclear magnetic moments) and \verb|c| is the x, y,
or z component of the magnetic moment with respect to spin coordinates.
\end{list}

\item[\Key{SECMOM}] Second moment integrals
\index{integrals!second moment}\index{second moment integrals}.

\begin{list}{}{}
\item Integral: $\left<\chi_{\mu}\left|{\bf r}{\bf
r}^{T}\right|\chi_{\nu}\right>$
\item Property labels: \verb|XXSECMOM|, \verb|XYSECMOM|,
\verb|XZSECMOM|, \verb|YYSECMOM|, \verb|YZSECMOM|, \verb|ZZSECMOM|
\end{list}

\item[\Key{SELECT}]\verb| |\newline
\verb|READ (LUCMD, *) NPATOM|\newline
\verb|READ (LUCMD, *) (IPATOM(I), I = 1, NPATOM|

Select which atoms for which a given atomic integral is to be
calculated. This applies mainly to property integrals for which
there exist a set of integrals for each nucleus. Read one more line
containing the number of atoms selected, and then another line
containing the numbers of the atoms selected. Most useful when
calculating diamagnetic spin-orbit\index{diamagnetic spin-orbit}
integrals, as this is a rather time-consuming calculation. The
numbering is of symmetry-independent nuclei.

\item[\Key{SOFIEL}] External magnetic-field dependence of the spin--orbit
operator integrals~\cite{jvkrovjcp111}.

\begin{list}{}{}
\item Integral:
$\frac{1}{2}\sum_KZ_K\left<\chi_{\mu}\left|\frac{{\bf r}_O^T{\bf r}_K -
{\bf r}_O{\bf r}_K^T}{r_{K}^3}\right|\chi_{\nu}\right>$
\item Property labels: \verb|SOMF  XX|, \verb|SOMF  XY|, \verb|SOMF XZ|, 
\verb|SOMF  YX|, \verb|SOMF  YY|, \verb|SOMF  YZ|, \verb|SOMF ZX|, 
\verb|SOMF  ZY|, \verb|SOMF  ZZ|. 
\end{list}

\item[\Key{SOMAGM}] Nuclear magnetic moment dependence of the spin--orbit
operator integrals~\cite{jvkrovjcc20}.

\begin{list}{}{}
\item Integral: $\sum_KZ_K\left<\chi_{\mu}\left|\frac{{\bf r}_K^T{\bf r}_LI - {\bf r}_K{\bf r}_L^T}{r_K^3r_L^3}\right|\chi_{\nu}\right>$
\item Property label: \verb|abcSOMMd|, where \verb|abc| is the number
of the symmetry-adapted nuclear magnetic moment coordinate, and
\verb|d| refers to the x, y, or z component of the spin--orbit operator.
\end{list}

\item[\Key{SORT I}]\index{integral sort} Requests that the
two-electron integrals should be
sorted for later use in \sir . See also keywords \Key{PRESORT} in the
\Sec{*DALTON} and \Sec{TRANSFORMATION} input sections.

\item[\Key{SOTEST}] Test the calculation of spin-orbit integrals as
requested by the keyword \Key{SPIN-O}.

\item[\Key{SPHMOM}]\verb| |\newline
\verb|READ (LUCMD,*) IORSPH|

Spherical multipole\index{multipole integral} integrals to order
\verb|IORSPH|. Read one more
line specifying order. See also the keyword \Key{CARMOM}.

\begin{list}{}{}
\item Property label: \verb|CMiijjkk|
\end{list}
where $i+j+k =$\verb|IORDER|, and where \verb|ii| = (i/10)*10+mod(i,10).

\item[\Key{SPIN-O}] Spatial spin-orbit\index{spatial spin-orbit}
integrals, see Ref.~\cite{ovhapjhjajthjojcp96}. Both the one- and the
two-electron integrals are calculated, the latter is stored on the file
\verb|AO2SOINT|.

\begin{list}{}{}
\item One-electron Integral:
$\sum_AZ_A\left<\chi_{\mu}\left|\frac{{\bf l}_A}{r_{A}^{3}}\right|\chi_{\nu}\right>$
where $Z_A$ is the charge of nucleus $A$ and the summation runs over
all nuclei of the molecule.
\item Property labels: \verb|X1SPNORB|,  \verb|Y1SPNORB|,  \verb|Z1SPNORB|
\item Two-electron Integral:
$\left<\chi_{\mu}\chi_{\nu}\left|\frac{{\bf l}_{12}}{r_{12}^{3}}\right|\chi_{\rho}\chi_{\sigma}\right>$
\item Property labels: \verb|X2SPNORB|,  \verb|Y2SPNORB|,  \verb|Z2SPNORB|
\end{list}

%\item[\Key{SQHDOL}] Square, non-symmetrized half differentiated
%overlap integrals\index{overlap!half-differentiated} with respect to
%geometric distortions, see
%Ref.~\cite{klbpjhjajjothjcp97}. Differentiation on the bra-vector.
%
%\begin{list}{}{}
%\item Integral: $\left<\frac{\partial \chi_{\mu}}{\partial
%R_{ab}}\mid\chi_{\nu}\right>$
%\item Property label: \verb|SQHDOLab|, where \verb|ab| is the number
%of the symmetry-adapted coordinate being differentiated.
%\end{list}

\item[\Key{SQHDOR}] Square, non-symmetrized half-differentiated
overlap integrals with respect to geometric distortions\index{overlap!half-differentiated}, see
Ref.~\cite{klbpjhjajjothjcp97}. Differentiation on the ket-vector.

\begin{list}{}{}
\item Integral: $\left<\chi_{\mu}\mid\frac{\partial\chi_{\nu}}{\partial
R_{ab}}\right>$
\item Property label: \verb|SQHDRabc|, where \verb|abc| is the number
of the symmetry-adapted coordinate being differentiated.
\end{list}

\item[\Key{SUPONL}] Only calculate the supermatrix. Requires the
presence of the two-electron integral file\index{two-electron
integral}\index{supermatrix}.

\item[\Key{SUSCGO}] Diamagnetic magnetizability\index{diamagnetic
magnetizability integrals} integrals calculated
without the use of London atomic orbitals. The choice of gauge
origin\index{gauge origin}
can be controlled by the keyword \Key{GAUGEO}.

\begin{list}{}{}
\item Integral: $\frac{1}{4}\left<\chi_{\mu}\left|r^{2}_{O}I - {\bf r}_{O}
{\bf r}_{O}^{T}\right|\chi_{\nu}\right>$
\item Property labels: \verb|XXSUSCGO|, \verb|XYSUSCGO|,
\verb|XZSUSCGO|, \verb|YYSUSCGO|, \verb|YZSUSCGO|, \verb|ZZSUSCGO|
\end{list}

\item[\Key{THETA}] Traceless quadrupole moment\index{quadrupole
moment integrals!traceless} integrals as defined by Buckingham~\cite{adbacp12}.

\begin{list}{}{}
\item Integral: $\frac{1}{2}\left<\chi_{\mu}\left|
3{\bf r} {\bf r}^{T} - r^{2}I_3
\right|\chi_{\nu}\right>$
%HJAAJ July 2001: acc. to hermit code
\item Property labels: \verb|XXTHETA |, \verb|XYTHETA |,
\verb|XZTHETA |, \verb|YYTHETA |, \verb|YZTHETA |, \verb|ZZTHETA |
\end{list}

\item[\Key{THIRDM}] Third moment integrals\index{integrals!third moment}
\index{third moment integrals}.

\begin{list}{}{}
\item Integral: $\left<\chi_{\mu}\left|{\bf r}^3\right|\chi_{\nu}\right>$
\item Property labels: \verb|XXX 3MOM|, \verb|XXY 3MOM|, \verb|XXZ 3MOM|, 
\verb|XYY 3MOM|, \verb|XYZ 3MOM|, \verb|XZZ 3MOM|, \verb|YYY 3MOM|, 
\verb|YYZ 3MOM|, \verb|YYZ 3MOM|, \verb|ZZZ 3MOM|. 
\end{list}

\item[\Key{U12INT}] Calculation of two-electron integrals over
  $\left[T1,r_{12}\right]$.

\item[\Key{U21INT}] Calculation of two-electron integrals over
  $\left[T2,r_{12}\right]$.

\item[\Key{WEINBG}]\verb| |\newline
\verb|READ (LUCMD,*) BGWEIN|\newline
Read in the square of the sin of the Weinberg angle appearing in the
definition of parity-violating integrals, see \Key{PVIOLA}. The
Weinberg angle factor will if this keyword is used be set to
\verb|[1-4*BGWEIN]|. 

\item[\Key{XDDXR3}] Direct perturbation theory paramagnetic
  spin--orbit like integrals. 

\begin{list}{}{}
\item Integral:
\item Property labels: \verb|ALF abcd|, where $a$ is ????.
\end{list}



\end{description}

\subsection{One-electron integrals: \Sec{ONEINT}}
\label{sec:oneinp}

Directives affecting the one-electron undifferentiated Hamiltonian
integral calculation appear in the \Sec{ONEINT} section.
\begin{description}
\item[\Key{CAVORG}]\verb| |\newline
\verb|READ (LUCMD,*) (CAVORG(I), I = 1, 3|

Read one more line containing the origin to be used for the origin of
the cavity\index{cavity!origin} in self-consistent reaction field
calculations. The default
is that this origin is chosen to be the center of mass\index{center of
mass} of the molecule.

\item[\Key{NOT ALLRLM}] Save only the totally symmetric multipole integrals
calculated in the solvent run on disc. Default is that multipole
integrals of all symmetries are written disc. May be used in
calculations of the energy alone in order to save disc space.

\item[\Key{PRINT}]\verb| |\newline
\verb|READ (LUCMD,*) IPRONE|

Set print level during the calculation of one-electron Hamiltonian
integrals.  Read one more line containing print level. Default is
the value of \verb|IPRDEF| from the \Sec{*INTEGRALS} input module.

\item[\Key{SKIP}] Skip the the calculation of one-electron Hamiltonian
integrals. Mainly for debugging purposes.

\item[\Key{SOLVEN}]\verb| |\newline
\verb|READ (LUCMD,*) LMAX|

\begin{list}{}{}
\item Integral:
$\left<\chi_{\mu}\left|x^iy^jz^k\right|\chi_{\nu}\right>$ for all
integrals where $i+j+k\leq$\verb|LMAX|.
\item Property label: Not extractable. Integrals written to file
\verb|AOSOLINT|
\end{list}

Calculate the necessary integrals needed to model the effects of a
dielectric medium\index{dielectric medium} by a
reaction-field\index{reaction field} method as
described in
Ref.~\cite{kvmhahjajthjcp89}.  Read one more line containing maximum
angular quantum
number for the multipole integrals\index{multipole integral} used for
the reaction field.
\end{description}

\subsection{General: \Sec{READIN}}\label{sec:herrdn}

Directives to control the reading of the molecule input appear in the
\Sec{READIN} section.
\begin{description}
\item[\Key{CM FUN}]\verb| |\newline
\verb|READ (LUCMD,*) LCMMAX, CMSTR, CMEND|

Use Rydberg basis functions\index{Rydberg basis function}\index{basis function!Rydberg}
 (center of mass functions\index{center of mass function}) as suggested by
Kaufman {\it et al.\/}~\cite{kkwbmjjpbamop22}. \verb|LCMMAX| denoted
the maximum quantum number of the Rydberg functions, basis functions
for all quantum up to and including \verb|LCMMAX| will be generated
(s=0, p=1 etc.) \verb|CMSTR| and \verb|CMEND| are the half-integer
start- and ending quantum number for the Rydberg basis functions. The
basis functions will be placed at the position of a dummy center
indicated as \verb|X| in the \verb|MOLECULE.INP| file. The charge of
the ion-core is determined by the keyword \Key{ZCMVAL}. If no center
named \verb|X| is present in the \verb|MOLECULE.INP| file, this input
will be ignored.

\item[\Key{MAXPRI}]\verb| |\newline
\verb|READ (LUCMD,*) MAXPRI|

Set maximum number of primitives\index{primitive orbitals} in any
contraction.  Read one more line containing number.  Default
is~25, except for the Cray-T3D/E, where the default is 14.

%\item[\Key{MOLINP}] Indicates that the molecular input comes at the
%end of the current file. 
%Default is that the molecular input is to be
%read from unit~9.

\item[\Key{OLDNORM}] Use the normalization scheme for spherical AO's used in {\dalton} 1.0.
From {\dalton} 1.1, all components of spherical AO's are normalized to 1.
This was not the case in {\dalton} 1.0, and this option is needed to read 
MO's correctly from an old file generated by {\dalton} 1.0 if spherical AO's were used.
(Cartesian AO's are still not all normalized to 1 for $d$-orbitals and higher $l$'s.)

\item[\Key{PRINT}]\verb| |\newline
\verb|READ (LUCMD,*) IPREAD|

Set print level for input processing.  Read one more line containing
print level. Default is the \verb|IPRDEF| from the \Sec{*INTEGRALS} input
module.

\item[\Key{R12AUX}] An auxiliary basis is used.
Basis sets must be identified as either orbital basis or
auxiliary basis in the \verb|MOLECULE.INP| file (line 5).
See Sec.~\ref{sec:r12aux} on page~\pageref{sec:r12aux}.

\item[\Key{SYMTHR}]\verb| |\newline
\verb|READ (LUCMD,*) TOLLRN|

Read threshold for considering atoms to be related by symmetry. Used
in the automatic symmetry detection routines. Default is $5.0\cdot
10^{-6}$.

\item[\Key{UNCONT}] Force the program to use the basis set as a primitive (completely decontracted) set.

\item[\Key{WRTLIN}] Write out the lines read in during the input
  processing of the \verb|MOLECULE.INP| file. Primarily for debugging
  purposes or for analyzing input errors in the \verb|MOLECULE.INP|
  file.

\item[\Key{ZCMVAL}]\verb| |\newline
\verb|READ (LUCMD,*) ZCMVAL|

Read the charge of the center of the Rydberg basis functions specified
by the \Key{CM FUN} keyword. Default is a charge of one.

\end{description}

\subsection{Integral sorting: \Sec{SORINT}}

Affects the sorting of the two-electron
integrals\index{integral sort}. Note that in order for \sir\ to use
the sorted integrals
generated with the keyword \Key{SORT I}, the keyword
\Key{PRESORT} has to used in the \Sec{TRANSFORMATION} input
module of the \Sec{*WAVE FUNCTIONS} input section. For standard
calculation you will in general not need to change any of the
default settings.

\begin{description}

\item[\Key{DELAO}]

Delete AOTWOINT file from Hermit after the integrals have been sorted.

\item[\Key{INTSYM}]\verb| |\newline
\verb|READ (LUCMD, *) ISNTSYM|

Symmetry of the two-electron integrals that are to be sorted.
Default is totally-symmetric two-electron integrals
(\verb|ISNTSYM| = 1).

\item[\Key{IO PRI}]\verb| |\newline
\verb|READ (LUCMD, *) ISPRFIO|

Set the print level in the fast-I/O routines. Default is a print level of 0.

\item[\Key{KEEP}]\verb| |\newline
\verb|READ (LUCMD, *) (ISKEEP(I),I=1,8)|\newline
Allowed values: 0 and 1.
A value of "1" indicates that the basis functions of this symmetry will
not be used and the integrals with these basis functions are omitted.


\item[\Key{PRINT}]\verb| |\newline
\verb|READ (LUCMD, *) ISPRINT|

Print level in the integral sorting routines.

\item[\Key{THRQ}]\verb| |\newline
\verb|READ (LUCMD, *) THRQ2|

Threshold for setting an integral to zero. By default this threshold
is 1.0D-15.

\end{description}

\subsection{Construction of the supermatrix file:
\Sec{SUPINT}}\label{sec:supint}

Directives affecting the construction of the
supermatrix\index{supermatrix} file is given
in the \Sec{SUPINT} section.

\begin{description}
\item[\Key{NOSYMM}] No advantage is taken of integral symmetry in the
construction of the supermatrix file. This may increase the disc space
requirements as well as  CPU time. Since \aba\ do not use the
supermatrix file (which it in the current version does not), this
keyword is mainly for debugging purposes.

\item[\Key{PRINT}]\verb| |\newline
\verb|READ (LUCMD,*) IPRSUP|

Set the print level during the construction of the supermatrix file.
Read one more line containing the print level. Default is the
value of \verb|IPRDEF| in the \Sec{*INTEGRALS} input module.

\item[\Key{SKIP}] Skip the construction of the supermatrix file.
An alternative keyword is \Key{NOSUP} in the \Sec{*INTEGRALS} input
module.

\item[\Key{THRESH}]\verb| |\newline
\verb|READ (LUCMD,*) THRSUP|

Threshold for the supermatrix integrals. Read one more line containing
the threshold. Default is the same as the threshold for
discarding the two-electron integrals (see the chapter describing the
\mol\ input format, Ch.~\ref{ch:molinp}).
\end{description}

\subsection{Two-electron integrals using {\twoint}: \Sec{TWOINT}}

Directives affecting the two-electron\index{two-electron integral}
undifferentiated Hamiltonian
integral calculation appear in the \Sec{TWOINT} section.
\begin{description}
\item[\Key{ICEDIF}]\verb| |\newline
\verb|READ (LUCMD,*) ICDIFF,IEDIFF|

Screening\index{integral screening} threshold for Coulomb and exchange
contributions to the Fock
matrix in direct and parallel calculations. The thresholds for the
integrals are ten to the negative power of these numbers. By default the same
screening threshold will be used for Coulomb and exchange
contribution which will change dynamically as the wave function
converges more and more tightly.

\item[\Key{IFTHRS}]\verb| |\newline
\verb|READ (LUCMD,*) IFTHRS|

Screening threshold\index{integral screening} used in direct and
parallel calculations. The integral threshold will be ten to the
negative power of this number. The default is that this value will change dynamically as the wave function converges more and more tightly.

\item[\Key{PANAS}] Calculates scaled two-electron integrals as
proposed by Panas as a simple way of introducing electron
correlation in calculations of molecular energies~\cite{ipcpl245}.

\item[\Key{PRINT}]\verb| |\newline
\verb|READ (LUCMD, *) IPRINT, IPRNTA, IPRNTB, IPRNTC, IPRNTD|

Set print level for the derivative integral calculation of a particular shell
quadruplet.  Read one more line containing print level and the four
shell indices.  The print level is changed from the default
for this quadruplet only.

\item[\Key{RETURN}] Stop after the shell quadruplet specified
under \Key{PRINT} above. Mainly for debugging purposes.

\item[\Key{SKIP}] Skip the calculation of two-electron Hamiltonian
integrals. An alternative keyword is \Key{NOTWO} in the \Sec{*INTEGRALS}
input module.

\item[\Key{SOFOCK}] Construct the Fock matrix in symmetry-orbital
basis during a direct or parallel calculation. Currently not active.

\item[\Key{THRFAC}] Not used in \siraba .

\item[\Key{TIME}] Provide detailed timing breakdown for the
two-electron integral calculation.
\end{description}


\subsection{Two-electron integrals using {\eri}: \Sec{ER2INT}}
\label{ch:eri}

Directives controlling the two-electron integral calculation with {\eri}
are specified in the \Sec{ER2INT} section.
By default {\eri} is only used for integral-direct coupled cluster
calculations. However, {\eri} can also be invoked by specifying the
\Key{RUNERI} keyword in the \Sec{*DALTON INPUT} input section. Note that
{\dalton} will automatically use {\twoint} for all integrals not
available in {\eri}.

\begin{description}
\item[\Key{AOBTCH}]\verb| |\newline
\verb|READ (LUCMD,*) IAOBCH| 

Only integrals with the first integral index belonging to AO batch
number \verb|IAOBCH| will be calculated.

\item[\Key{BUFFER}]\verb| |\newline
\verb|READ (LUCMD,*) LBFINP|

This option may be used to set the buffer length for integrals written
to disk. For compatibility with {\twoint}, the default buffer length
is 600. Longer buffer lengths may give more efficient I/O.

\item[\Key{DISTRI}] Use distributions for electron 1. Must be used in
  connection with the keyword \Key{SELCT1}.

\item[\Key{DISTST}] Test the calculation of two-electron integrals
  using distributions.

\item[\Key{DOERIP}] Use the ERI integral program for the calculation
  of two-electron integrals instead of TWOINT.

\item[\Key{EXTPRI}]\verb| |\newline
\verb|READ (LUCMD,*) IPROD1, IPROD2|

Full print for overlap distribution (OD) classes \verb|IPROD1| and \verb|IPROD2|.

\item[\Key{GENCON}] Treat all AOs as generally contracted during
integral evaluation.

\item[\Key{GRDZER}] During evaluation of the molecular gradient, the
gradient is set to zero upon each entry into \verb|ERIAVE| (for
debugging).

\item[\Key{INTPRI}] Force the printing of calculated two-electron integrals.

\item[\Key{INTSKI}] Skip the calculation of two-electron integrals in
  the ERI calculation.

\item[\Key{MAXDIS}]\verb| |\newline
\verb|READ (LUCMD,*) MAXDST|  

Read in the maximum number of integral distributions calculated in
each call to the integral code. Default value is 40.

\item[\Key{MXBCH}]\verb| |\newline
\verb|READ (LUCMD,*) MXBCH|

Read in the maximum number of integral batches to be treated
simultaneously, and thus determines the vector lengths. Default value
is 1000000000.

\item[\Key{NCLERI}] Calculate only integrals with non-classical
contributions.

\item[\Key{NEWCR1}] Use an old transformation routine for the
  generation of the Cartesian integrals for electron 1.

\item[\Key{NOLOCS}] Do not use local symmetry during evaluation.

\item[\Key{NONCAN}] Do not sort integral indices in canonical
order. This option is applicable only for undifferentiated integrals
written fully to disk (without the use of distributions). This option
may save some time but will lead to incorrect results whenever
canonical ordering is assumed.

\item[\Key{NO12GS}] During sorting of OD batches, treat batches
containing one and two distinct AOs as equivalent.

\item[\Key{NOPS12}] Do not assume permutational symmetry between the
two electrons.

\item[\Key{NOPSAB}] Do not assume permutational symmetry between
orbitals of electron 1.

\item[\Key{NOPSCD}] Do not assume permutational symmetry between
orbitals of electron 2.
 
\item[\Key{NOSCRE}] Do not do integral screening before integrals are
  written to disk.

\item[\Key{NOWRIT}] Do not write integrals to disk.

\item[\Key{NSPMAX}]\verb| |\newline
\verb|READ (LUCMD,*) NSPMAX| 

Allows for the reduction of the number of symmetry generations for
each basis function in order to reduce memory requirements. Mainly for
debugging purposes, do not use.


\item[\Key{OFFCNT}] During sorting of OD batches, treat batches
containing one and two distinct AO centers as equivalent.

\item[\Key{PRINT}]\verb| |\newline
\verb|READ (LUCMD,*) IPRERI, IPRNT1, IPRNT2|

Print level for {\eri}. By giving numbers different from zero for
\verb|IPRNT1| and \verb|IPRNT2|, extra print information may be given
for these overlap distributions, after which the program will exit.

\item[\Key{RETURN}] Stop after the shell quadruplet specified
under \Key{PRINT} above. Mainly for debugging purposes.

\item[\Key{SELCT1}]\verb| |\newline
\verb|READ (LUCMD,*) NSELCT(1)|\newline
\verb|READ (LUCMD,*) (NACTAO(I,1),I=1,NSELCT(1))|

Calculate only integrals containing indices \verb|NACTAO(I,1)| for the
first AO.

\item[\Key{SELCT2}] Same as \Key{SELCT1} but for the second AO.

\item[\Key{SELCT3}] Same as \Key{SELCT1} but for the third AO.

\item[\Key{SELCT4}] Same as \Key{SELCT1} but for the fourth AO.

\item[\Key{SKIP}] Skip the calculation of two-electron Hamiltonian
integrals. An alternative keyword is \Key{NOTWO} in the \Sec{*INTEGRALS}
input module.

\item[\Key{TIME}] Provide detailed timings for the two-electron
  integral calculation in ERI.

\item[\Key{WRITEA}] Write all integrals to disk without any screening.

\end{description}


%\chapter{\mol\ input style}\label{ch:molinp}
\index{geometry!Cartesian coordinate input}
\index{Cartesian coordinate input}
\index{geometry!Z-matrix input}
\index{Z-matrix input}

The {\mol}-style input is originally based on the input for the \mol\
integral program by Alml\"{o}f~\cite{moleculeref}. However, there are
few remnants of the original input structure. For users of earlier
releases of the Dalton program, note should be taken of the fact that
the input structure for the \mol\ file has undergone major changes,
however, backward compatibility has in most cases been
retained\footnote{The two exceptions to the backward are that when
using the {\tt ATOMBASIS} keyword, the basis set name has to be
preceded by ``Basis=''. When specifying the Cartesian
coordinates of the atoms, it is no longer required that the
coordinates start at position 5. However, blanks are no longer
allowed in the names of atoms, and the atom names are still
restricted to 4 characters.}.
The program supports both Cartesian\index{Cartesian coordinate input}
and Z-matrix input\index{Z-matrix input} of the
molecular coordinates. However, the Z-matrix input provided is only a
convenient way of describing the molecular geometry, as the Z-matrix is
converted to Cartesian coordinates, which are then used in the
subsequent calculations. Note that the Z-matrix input can only be
used together with the basis set library (\quotekw{BASIS} in first line).

The program includes an extensive basis set
library\index{basis set!library}, which are described below. 
There are additional possibilities for choosing the number of primitive
and contracted orbitals to be used with the Atomic Natural Orbital
(ANO) basis sets\index{ANO basis set}\index{basis set!ANO}
and the ``Not Quite van Duijneveldt'' (NQvD) basis
sets\index{NQvD basis set}\index{basis set!NQvD}. The large number of
different basis sets provided is in part
related to the variety of molecular properties with very different
basis set requirements that can be calculated with \dalton.
Most of the basis sets have been downloaded from the EMSL basis set
library service\index{EMSL basis set library service}\index{basis set!EMSL library}\footnote{http://www.emsl.pnl.gov:2080/forms/basisform.html~\cite{emslref}} Only a small number of the basis sets were obtained from different sources:

\begin{list}{--}{}
\item The ano-1, ano-2, ano-3, ano-4 and Sadlej-pVTZ basis sets which where
downloaded from the MOLCAS home-page\index{NQvD basis set}\index{basis set!NQvD}\index{ANO basis set}
\index{basis set!ANO}\index{sadlej basis set}\index{basis set!sadlej}
(http://www.teokem.lu.se/molcas/).
\item The NQvD (The Not Quite van Duijneveldt) basis sets were constructed
by Knut F\ae gri~\cite{nqvdref}.
\item The Turbomole-X {X = SV,SVP,DZ,DZP,TZ,TZP,TZV,TZVP,TZVPP,TZVPPP} basis sets have
been downloaded from Turbomole\index{Turbomole basis set}\index{basis set!Turbomole}(http://www.turbomole.com)
\item The pc-$n$ and apc-$n$ polarization-consistent basis sets were provided to us by Frank Jensen.
\end{list}

We note that each file containing a given basis set in the BASIS
directory contains the proper reference to be used when doing a 
calculation with a given basis set. For convenience we also list 
these references with the basis sets in Section~\ref{sec:basislist}.

The description of the \mol\ input is divided into five parts.
Section~\ref{sec:molgeneral} describes the general section of the
molecule input,
section~\ref{sec:molcart} describes the Cartesian coordinate
input\index{Cartesian coordinate input},
section~\ref{sec:molzmat} describes the Z-matrix
input\index{Z-matrix input}, and finally
Section~\ref{sec:molbasis} describes the basis set
library\index{basis set!library}.
Section~\ref{sec:basislist} lists the basis sets (including references and supported elements) 
in the basis set library\index{basis set!library}.

\section{General \mol\ input}\label{sec:molgeneral}

In the general input section of the \mol\ input file, we will consider
such information as molecular symmetry\index{symmetry}, number of
symmetry distinct atoms\index{symmetry-distinct atom}, generators of a
given molecular point group\index{symmetry!generator}\index{symmetry!group}, and so on.
This information usually constitutes the four/five first lines of the
input.

The input is best described by an example.
The following is the first lines of an input for
tetrahedrane\index{tetrahedrane}, treated in
$C_{2v}$~symmetry, with a 4-31G** basis.  The line numbers are for
convenience in the subsequent input description and should {\em
not} appear in the actual input.  Note also that in order to fit
the example across the page some liberties have been taken with
column spacings.
\begin{verbatim}
 1:INTGRL
 2:        Tetrahedrane, Td_symmetric geometry
 3:                 4-31G** basis
 4:Atomtypes=2 Generators=2 X Y Integrals=1.00D-15
\end{verbatim}

We now define the input line-by-line.  The {\tt FORMAT} is given
in parenthesis.
\begin{description}
\item[1] The word \verb|INTGRL|\index{INTGRL} {\tt (A6)}.
\item[2-3] Two arbitrary title lines {\tt (A72)}.
\item[4] General instructions about the molecule.

This line is keyword-driven. The general structure of the input is
\verb|Keyword=|. The input is case sensitive, but \dalton\ will recognize
the keywords whether specified with only three characters (minimum) or
the full name (or any intermediate option). The order of the keywords
is arbitrary. The following keywords are recognized for this line:
\begin{description}
\item[Angstrom] Indicates that the atomic Cartesian coordinates are
  given in \AA ngstr\"{o}m, and not in bohr (atomic units) which is
  the default.
\item[Atomtypes] \verb|(Integer)|. {\em This keyword is
  required\/}. Number of atom types (number of atoms specified 
in separate blocks). For a Z-matrix input this will be the total
number of atoms in the molecule, the Z-matrix module will then
extract the number of atom types.
\item[Cartesian] Indicates that a Cartesian Gaussian basis set will be
  used in the calculations.\index{Cartesian basis function}\index{basis function!Cartesian}
\item[Charge] \verb|(Integer)|. The charge of the molecule\index{charge
  of molecule}. Will be used  by the program to determine the Hartree--Fock
occupation\index{HF occupation}\index{Hartree--Fock occupation}.
\item[Generators] \verb|(Integer|+\verb|Character)|. Number of symmetry
  generators\index{symmetry!generator}. If this keyword is not
  specified (and \verb|Nosymmetry| not invoked)
  the automatic symmetry detection routines of the program will be
invoked. Symmetry can be turned off (needed for instance if starting a
walk at a highly symmetric structure which one knows will break
symmetry) using the keyword \verb|Nosymmetry|. \dalton\ is restricted
  to the Abelian subgroups of D$_{2h}$, and thus there can be 1 to 3
  generating elements.

The number of generators\index{symmetry!generator} is followed the
equally many blocks of characters specifying which Cartesian axis
change sign during each of the generators. {\tt X}
is reflection\index{reflection}
in the \mbox{$yz$-plane}, {\tt XY} is rotation\index{rotation} about
the \mbox{$z$-axis},
and {\tt XYZ} denotes inversion\index{inversion}. Due to the handling
of symmetry in
the program, it is recommended to use mirror planes as symmetry
generating elements if possible.
\item[Integrals] \verb|(Real)|. Indicates the threshold for which
  integrals smaller than this will be considered to be zero. If not
  specified, a threshold of 1.0D-15 will be used. A threshold
of 1.0D-15 will give integrals correct to approximately 1.0D-13.
\item[Nosymmetry] Indicates that the calculation is to be run without
  the use of point-group symmetry. Automatic symmetry detection will
  also be disabled.\index{symmetry!automatic detection}
\item[Own] Indicates that a user-supplied scheme for generating
  transformed angular momentum basis functions will be used.
\item[Spherical] Default. Indicates that a spherical Gaussian basis
  set will be used in the calculations.\index{spherical basis function}\index{basis function!spherical}
\end{description}
\end{description}

Note that if one wants to use a basis set
library\index{basis set!library}, there are two
options. One option is to use a common basis set for the entire
molecule in which the first line should be replaced by two lines,
which for a calculation using the 4-31G** basis would look like:
\begin{verbatim}
 1:BASIS
 2:4-31G**
\end{verbatim}
This option will not be active with customizable basis sets like the
ANO or NQvD sets.


Alternatively you may specify different basis sets for different
atoms, in which case the first line should read
\begin{verbatim}
 1:ATOMBASIS
\end{verbatim}

The fourth line (fifth in a calculation using the basis set library with "BASIS" in line 1)
looks a bit devastating. However, for ordinary
Hartree--Fock\index{HF}\index{MP2}\index{SCF}\index{Hartree--Fock}\index{M{\o}ller-Plesset!second-order}
or MP2 calculations, only the number of different atom types and the charge
need to be given (if the molecule is charged), as symmetry and
Hartree--Fock occupation\index{HF occupation}\index{Hartree--Fock occupation}
will be taken care of by the program. Thus
this line could in the above example be reduced to
\begin{verbatim}
 4:Atomtypes=2
\end{verbatim}
or even more concisely (though not more readable) as
\begin{verbatim}
 4:Ato=2
\end{verbatim}


Let us finally give some remarks about the symmetry
detection\index{symmetry!automatic detection}
routines. These routines will detect any symmetry of a molecule
by explicit testing for the occurrence of rotation axes, mirror planes
and center of inversion. The occurrence of a symmetry element is
tested in the program against a threshold which may be adjusted by the
keyword \Key{SYMTHR} in the \Sec{MOLBAS} input section. By default,
the program will require
geometries that are correct to the sixth decimal place in order to
detect all symmetry elements.

The program will translate and rotate the molecule into a suitable
reference geometry before testing for the occurrence of symmetry
operations. The program will not, due to the handling of symmetry
in the program, transform  the molecule back to original input
coordinates. Furthermore, if there are symmetry equivalent nuclei,
these will be removed from the input, and a new, standardized
molecule input file will be generated and used in subsequent
iterations of for instance a geometry optimization. This
standardized input file (including basis set) is printed to the file
\verb|DALTON.BAS|, which is among the files copied back after the end
of a calculation.

\dalton\ can only take advantage of point groups that are subgroups
of D$_{2h}$. If symmetry higher than that is detected, the program
will use the highest common subgroup of  the symmetry group
detected and D$_{2h}$.

We recommend that the automatic symmetry detection feature is not
used when doing MCSCF\index{MCSCF} calculations, as symmetry
generators\index{symmetry!generator} and their order in the input
determines the order of the irreducible representations
needed when specifying active spaces. Thus, for MCSCF calculations
we recommend that the symmetry is  explicitly specified through
the appropriate symmetry generators, as well as the explicit
Hartree--Fock occupation numbers.

\section{Cartesian geometry input}\label{sec:molcart}
\index{Cartesian coordinate input}

Assuming that we have given the general input as indicated above, we
now want to specify the spatial arrangements of the atoms in a
Cartesian coordinate system. We will also for sake of illustration
assume that we have given explicitly the generators of the point group
to be used in the calculation (in this case C$_{2v}$, with the yz- and
xz-planes as mirror planes).

In tetrahedrane we will have two different kinds of atoms, carbon
and hydrogen, as indicated by the number 2 on the fourth line of the input.
We will also assume that we enter the basis set ourselves,
in order to present the input format for the basis set.

For tetrahedrane, the input would then look like
\begin{verbatim}
 1:INTGRL
 2:        Tetrahedrane, Td_symmetric geometry
 3:                 4-31G** basis
 4:Atomtypes=2 Generators=2 X Y Integrals=1.00D-15
 5:Charge=6.0 Atoms=2 Blocks=3 1 1 1
 6:C1    1.379495419          .0                 0.975450565
 7:C2     .0                 1.379495419         -.975450565
 8:    8    3
 9:486.9669   .01772582
10:73.37109   .1234779
11:16.41346   .4338754
12:4.344984   .5615042
13:8.673525            -.1213837
14:2.096619            -.2273385
15:.6046513             1.185174
16:.1835578                      1.00000
17:    4    2
18:8.673525   .06354538
19:2.096619   .2982678
20:.6046513   .7621032
21:.1835578             1.000000
22:    1    1
23:0.8        1.0
24:Charge=1.0 Atoms=2 Blocks=2 1 1
25:H1    3.020386510         .0                  2.1357357837
26:H2     .0                 3.020386510         -2.1357357837
27:    4    2
28:18.73113   .03349460
29:2.825394   .2347270
30:.6401218   .8137573
31:.1612778             1.000000
32:    1    1
33:0.75       1.0
\end{verbatim}

Lines 1-4 are already described. The different new types of lines are:
\begin{description}
\item[5] This line is keyword-driven. The general structure of the input is
\verb|Keyword=|. The input is case sensitive, but it will recognize
the keywords whether specified with only three characters (minimum) or
the full name (or any intermediate option). The order of the keywords
is arbitrary. The following keywords are
recognized for this line:
\begin{description}
\item[Atoms] \verb|(Integer)|. Number of {\em
  symmetry-distinct}\index{symmetry-distinct atom} atoms of 
this type (or, if the symmetry detection routines are being used, all
atoms of this kind).
\item[Basis] \verb|(Character)|. If \verb|ATOMBASIS| has been specified,
  the keyword is required, and have to be followed by the name of the
  basis set that is to be used for this group of atoms, {\it e.g.\/}
  \verb|Basis=6-31G**|. By specifying
  \verb|Basis=pointcharge|\index{point charge}, the
  atoms in this block will be treated as point charges, that is,
  having only a charge but no basis functions attached to
  them.

  Different effective core potentials (ECP) could be used when \verb|ATOMBASIS|
  is specified. For instance, Stuttgart ECPs with corresponding
  Stuttgart double zeta basis sets\index{effective core
  potentials}\index{ECP} can be used by specifying
  \verb|Basis=ecp-sdd-DZ| \verb|ECP=ecp-sdd-DZ| (see \verb|test/rsp_ecp| for example).
\item[Blocks] \verb|(Integers)|. Maximum angular quantum number + 1 used in the
basis set for this atom type ($s=1$, $p=2$, etc.).
Ignored if library basis sets are being used (\verb|BASIS| or
\verb|ATOMBASIS| in first line). This number is followed by one
integer for each angular momentum used in the basis, indicating 
the number of groups (blocks) of generally contracted
functions of angular quantum number~{\tt I-1}.
Ignored if the basis set library is used. \newline
It is noteworthy that
\dalton\ collects all basis functions into one such shell, and
evaluates all integrals arising from that 
shell\index{shell of basis functions} simultaneously, and
the memory requirements grow rapidly with the number of basis
functions in a shell (note for instance that four $g$ functions
actually are 36 basis functions,
as there are 9 components of each $g$ function).
Memory requirements\index{memory} can therefore be reduced by splitting
basis functions of the quantum number into different blocks. However,
this will decrease the performance of the
integral\index{performance of integral program} calculation.
\item[Charge] \verb|(Real)|. {\em This keyword is required\/}.
  Charge of this atom\index{charge of atom} or point charge. 
\item[Pol] \verb|(Integer+real)|. This keyword adds single, primitive
  basis function of a given quantum number (quantum number + 1 given
  in the input) and a given exponent. An arbitrary number of
  polarization functions can be given. For instance, we can add a $p$
  function with exponent $0.05$ and a $d$ function with exponents
  $0.6$ we can write \verb|Pol 2 0.05 3 0.6|.
\item[Set] \verb|(Integer)|. Indicates whether the basis set specified
  is the ordinary orbital basis or the auxiliary basis set needed for
  instance in certain r12 calculations, see
  Sec.~\ref{sec:r12aux}. The keyword is only active when the keyword
  \Key{R12AUX} has been specified in the \Sec{MOLBAS} input section.
\end{description}
\item[6] \verb|NAME X Y Z Isotope=18|
\begin{description}
\item[NAME] Atom name.  A different name should be used for
each atom of the same type, although this is not required. Note that
only the first four characters of the atom name will be used by the
program.
\item[X] $x$-coordinate (in atomic units, unless \AA ngstr\"{o}m
has been requested on line 4 of the input).
\item[Y] $y$-coordinate.
\item[Z] $z$-coordinate.
\item[Isotope=] \index{Isotope=} Specify the atomic mass of the nucleus (closest
  integer number). By default the mass of the most abundant isotope
  of the element will be used. When automatic symmetry detection is
  used, the program will distinguish between different nuclei if they
  have different atomic mass number. A calculation of HDO would thus
  be run in $C_s$ symmetry. 
\end{description}
The Cartesian coordinates may
be given in free format.
\item[7] This is the other symmetry-distinct center of this type.
\item[8] \verb|FRMT, NPRIM, NCONT, NOINT| {\tt (A1,I4,2I5)}.
\begin{description}
\item[FRMT] A single character describing the input format of the
basis set in this block. The default format is {\tt (8F10.4)} which
will be used if {\tt FRMT} is left blank. In this format
the first column is the orbital exponent and the seven last columns
are contraction coefficients. If no numbers are given, a zero is
assumed. If more than 7 contracted functions occur in a given block,
the contraction coefficients may be continued on the next line, but
the first column (where the orbital exponents are given) must then be
left blank.

An {\tt F} or {\tt f} in the first position will indicate that the
input is in free format. This will of course require that all
contraction coefficients need to be typed in, as all numbers need
to be present on each line. However, note that this options is
particularly handy together with completely decontracted basis
sets, as described below. Note that the program reads the free
format input from an internal file that is 80 characters long, and
no line can therefore exceed 80 characters.

One may also give the format {\tt H} or {\tt h}. This corresponds to
high precision format {\tt (4F20.8)}, where the first column again is
reserved for the orbital exponents, and the three next columns are
designated to the contraction coefficients. If no number is given, a zero
is assumed. If there are more than three contracted orbitals in a
given block, the contraction coefficients may be continued on the next
line, though keeping the column of the orbital exponents blank.

\item[NPRIM] Number of primitive\index{primitive basis function}
Gaussians in this block.
\item[NCONT] Number of contracted\index{contracted basis function}
Gaussians in this block. If a zero
is given, an uncontracted basis set will be assumed, and only orbital
exponents need to be given.
\end{description}
\item[9] \verb|EXP, (CONT(I), I=1,NCONT)|
\begin{description}
\item[EXP] Exponent of this primitive.
\item[CONT(I)] Coefficient of this primitive in contracted
function~{\tt I}.
\end{description}
We note that the format of the orbital exponents\index{orbital exponent} 
and the contraction\index{contraction coefficient}
coefficients are determined from the value of {\tt FRMT} defined on
line~8.
\item[10-16] These lines complete the specification of this
contraction block: the $s$~basis here.
\item[17-21] New contraction block (see lines~8 and~9 above).
\item[22-23] New contraction block.
\item[24-33] Specifies a new atom type: coordinates and basis set.
\end{description}

\section{Z-matrix input}\label{sec:molzmat}
\index{geometry!Z-matrix input}
\index{Z-matrix input}

The Z-matrix input provided with \dalton\ is quite rudimentary, and
common options like parameter representations of bond length and
angles as well as dummy atoms are not provided. Furthermore,
the Z-matrix input is not used in the program, but instead immediately
converted to Cartesian coordinates which are then used in the
subsequent calculation.
Another restriction is that if Z-matrix input is used, one cannot
punch ones own basis set, but must instead resort to one of the basis sets
provided with the basis set library (i.e. {\tt BASIS} in first line).
Finally, you cannot explicitly specify higher symmetry in input line 4
when using Z-matrix input,
you can only request no symmetry ($C_1$) with \verb|Nosymmetry|
or allow \dalton\ to detect symmetry automatically.

The input format is free, with
the restriction that the name of each atom must be given a space
of 4 characters, and none of the other input variables needed must be
placed in these positions.

The program will use Z-matrix input\index{Z-matrix input} if there is
the word \quotekw{ZMAT} 
in the first four position of line 6 in the molecule input.
The following {\tt NONTYP} lines contain the Z-matrix specification
for the {\tt NONTYP} atoms.

A typical Z-matrix input could be:
\begin{verbatim}
BASIS
6-31G**
   Test Z-matrix input of ammonia
   6-31G** basis set
Atomtypes=4
ZMAT
N   1 7.0
H1  2 1 1.0116 1.0
H2  3 1 1.0116 2 106.7 1.0
H3  4 1 1.0116 2 106.7 3 106.7 1 1.0
\end{verbatim}
The five first lines should be familiar by now, and will be discussed
no further here. 
The special 6'th line tells that this is Z-matrix input.
The Z-matrix input starts on line 7, and on this first Z-matrix
line only the atom name, a running number and the charge of the
atom\index{charge of atom}
is given. The running number is only for
ease of reference to a given atom, and is actually not used within the
program, where any reference to an atom, is the number of the
atom consecutively in the input list.

The second Z-matrix line consists of the atom name, a running number, the
number of the atom to which this atom is bonded with a given bond
length in {\AA}ngstr{\"o}ms, and then finally the charge of this atom.

The third Z-matrix line is identical to the second, except that an extra atom
number, to which the two first atoms on this line is bonded to with a
given bond angle in degrees.

On the fourth Z-matrix line yet another atom has been added, and the position
of this atom relative to the three previous ones on this line is
dependent upon on an extra number inserted just before the nuclear
charge of this atom. If the next to last number is a 0, the
position of this atom is given by the dihedral angle {\tt
(A1,A2,A3,A4)} in degrees, where {\tt Ai} denotes atom {\tt i}. If, on the other
hand, this next to last number is $\pm 1$, the position of the fourth
atom is given with respect to two angles, namely {\tt (A1,A2,A3)} and
{\tt (A2,A3,A4)}. The sign is to be $+ 1$ if the triple product
$\overrightarrow{\left(A_{2}A_{1}\right)}\cdot\left[\overrightarrow{\left(A_{2}A_{3}\right)}\times\overrightarrow{\left(A_{2}A_{4}\right)}\right]$
is positive.

\section{Using basis set libraries}\label{sec:molbasis}
\index{basis set!library}
\index{BASIS}
\index{ATOMBASIS}

The use of predefined basis sets is indicated
by the word {\tt BASIS} or {\tt ATOMBASIS} on the first line of the
molecular input.
(If you want Z-matrix input you must use {\tt BASIS}.)

The specified basis set(s) are searched for in the following directories:
\begin{itemize}
\item all user specified basis set directories (with {\tt dalton -b dir1 -b dir2 ...}
\item the job directory
\item the basis set library supplied with \dalton.
\end{itemize}

If {\tt BASIS} is used, a common basis set is used for all atoms in
the molecule, and the name of this basis set is given on the second line.
If we want to use one basis set for all the
atoms in a molecule, the molecule input file can be significantly
simplified, as we may delete all the input information regarding the
basis set. Thus, the input in the previous section for tetrahedrane
with the 6-31G** basis will, if the basis set library is used,  be reduced
to:

\begin{verbatim}
 1:BASIS
 2:6-31G**
 3:        Tetrahedrane, Td_symmetric geometry
 4:                 4-31G** basis
 5:Atomtypes=2 Generators=2 X Y Integrals=1.00D-15
 6:Charge=6.0 Atoms=2
 7:C1    1.379495419          .0                 0.975450565
 8:C2     .0                 1.379495419         -.975450565
 9:Charge=1.0 Atoms=2
10:H1    3.020386510         .0                  2.1357357837
11:H2     .0                 3.020386510         -2.1357357837
\end{verbatim}

The use of the basis set library is indicated by the presence of the
\verb|BASIS| word in the beginning of \mol-file instead of
\verb|INTGRL|.

An alternative approach would be to use different basis sets for
different atoms, {\it e.g.\/} the concept of locally
dense\index{locally dense basis set} basis sets
introduced in NMR calculations by Chesnut {\it et
al.\/}~\cite{dbcberkdmdaejcc14}. This is for instance also required
when using the ANO\index{ANO basis set}\index{basis set!ANO} or
NQvD\index{NQvD basis set}\index{basis set!NQvD}
basis sets. Another option is to use
standards basis sets from the basis set 
library\index{basis set!library} and add your own sets
of diffuse, tight or polarizing\index{polarization function} basis
functions. Returning to
tetrahedrane, we could for instance use the 6-31G* basis set for
carbon and the 4-31G** basis set for hydrogen. This could be achieved
as

\begin{verbatim}
 1:ATOMBASIS
 2:        Tetrahedrane, Td_symmetric geometry
 3:    Mixed basis (6-31G* on C and 4-31G** on H)
 4:Atomtypes=2 Generators=2 X Y Integrals=1.00D-15
 5:Charge=6.0 Atoms=2 Basis=6-31G*
 6:C1    1.379495419          .0                 0.975450565
 7:C2     .0                 1.379495419         -.975450565
 8:Charge=1.0 Atoms=2 Basis=4-31G Pol 2 0.75D0
 9:H1    3.020386510         .0                  2.1357357837
10:H2     .0                 3.020386510         -2.1357357837
\end{verbatim}


Thus, when using {\tt ATOMBASIS}\index{ATOMBASIS} the name of the
basis set for a given
set of identical atoms is given on the same line as the nuclear
charge, indicated by the keyword ``Basis=''.

The string {\tt Pol} denotes that the rest of the line specifies
diffuse, tight or polarizing\index{polarization function}
functions, all which will be added as segmented basis functions.
For each basis function, its ``angular momentum'' ($l+1$) and its
exponent must be given. Thus, in the above input we indicate that
we add a $p$ function with exponent 0.75 to the hydrogen basis
set. The order of these functions are arbitrary (that is, a $p$
function can be given before an $s$ function and so on).

Augmenting\index{basis!augmented} the correlation-consistent\index{basis!correlation-consistent}\index{correlation-consistent basis set} 
sets of Dunning~\cite{dewthdjcp100} is a straight-forward process in \dalton. 
The aug-cc-pVXZ and aug-cc-pCVXZ basis sets are extended in an even-tempered manner 
(in the manner of Dunning~\cite{dewthdjcp100})
by including a 'd-', 't-' or 'q-' prefix to give doubly, triply or quadruply 
augmented basis sets, respectively. For example, specifying t-aug-cc-pVDZ will
produce a triply augmented cc-pVDZ basis.
%hjaaj: aug-ecp does not exist in basis/ directory, thus commented here. /Aug-2005 hjaaj.
%The aug-ecp basis sets may be augmented 
%in the same manner by using the same prefixes, eg d-aug-ecp. 
Note that these basis sets are not listed 
explicitly in the basis library directory\index{basis set!library}, 
but are automatically generated within \dalton\ from the respective 
aug-cc-pVXZ
% and aug-ecp basis sets. 
basis set.

The ANO basis sets require that you give the number of contracted
functions you would like to use for each of the primitive sets defined
in the basis sets. Thus, assuming we would like to simulate the
6-31G** basis set input using an ANO basis set but with the
polarization functions of the 6-31G** set, this could be achieved
through an input like

\begin{verbatim}
 1:ATOMBASIS
 2:        Tetrahedrane, Td_symmetric geometry
 3:    Mixed basis (6-31G* on C and 4-31G** on H)
 4:Atomtypes=2 Generators=2 X Y Integrals=1.00D-15
 5:Charge=6.0 Atoms=2 Basis=ano-1 3 2 0 0 Pol 3 0.8
 6:C1    1.379495419          .0                 0.975450565
 7:C2     .0                 1.379495419         -.975450565
 8:Charge=1.0 Atoms=2 Basis=ano-1 2 0 0 Pol 2 0.75D0
 9:H1    3.020386510         .0                  2.1357357837
10:H2     .0                 3.020386510         -2.1357357837
\end{verbatim}

This input will give a [3s2p0d0f] ANO basis set\index{ANO basis set}\index{basis set!ANO}
on carbon, with a
polarizing $d$ function with exponent 0.8, and a [2s0p0d] ANO basis
set on hydrogen with a polarizing $p$ function with exponent 0.75 as
above.

Note that the number of contracted functions in the ANO\index{ANO basis set}\index{basis set!ANO}
set has to be
given for all primitive blocks, even though you do not want any
contracted functions of a given quantum number. Here also, {\tt Pol}
separates the number of contracted functions from polarization
functions.

The NQvD basis set~\cite{nqvdref} was constructed in order to provide,
in electronic form, a basis set compilation very similar to original
set of van Duijneveldt~\cite{fbvdibmrap}\index{NQvD basis set}\index{basis set!NQvD}.
The sets are in general as good, or slightly better, than the original
van Duijneveldt basis, with only minor changes in the orbital
exponents.

In the NQvD basis set, you need not only to pick the number of
contracted functions, but also your primitive set. The contracted
basis set will be constructed contracting the (NPRIM-NCONT + 1)
tightest functions with contraction coefficients based on the
eigenvectors from the atomic optimization, keeping the outermost
orbitals uncontracted.

NOTE: As is customary, the orbital exponents of all hydrogen basis
functions are automatically multiplied by a factor of 1.44.

Thus, an input for tetrahedrane employing the NQvD basis set might
look like

\begin{verbatim}
 1:ATOMBASIS
 2:        Tetrahedrane, Td_symmetric geometry
 3:    Mixed basis (6-31G* on C and 4-31G** on H)
 4:Atomtypes=2 Generators=2 X Y Integrals=1.00D-15
 5:Charge=6.0 Atoms=2 Basis=NQvD 8 4 3 2 Pol 3 0.8D0
 6:C1    1.379495419          .0                 0.975450565
 7:C2     .0                 1.379495419         -.975450565
 8:Charge=1.0 Atoms=2 Basis=NQvD 4 2 Pol 2 0.75D0
 9:H1    3.020386510         .0                  2.1357357837
10:H2     .0                 3.020386510         -2.1357357837
\end{verbatim}

This input will use an (8s4p/4s) primitive basis set on carbon and
hydrogen respectively, contracting it to a [3s2p/2s] set. The
polarization functions should not require further explanation at this
stage.

The only limitations to the use of polarization functions when {\tt
ATOMBASIS} is used, is that the length of the line must note exceed 80
characters. If that happens, we recommend collecting a standard basis set
from the file \verb|DALTON.BAS|, and then adding
functions to this set.

\section{Auxiliary basis sets}
\label{sec:r12aux}

It is possible to specify more than one basis set. For example, by typing
\begin{verbatim}
 1:BASIS
 2:4-31G** 6-311++G(3df,3pd)
\end{verbatim}
the basis set 6-311++G(3df,3pd) will be used as an auxiliary basis.
When using an auxiliary basis, each atom line must contain a basis-set
identifier, which at present may take the values \verb|Set=1| (orbital basis) or 
\verb|Set=2| (auxiliary basis). Basis sets with \verb|Set=1| must be read first,
basis sets with \verb|Set=2| thereafter.
The above also applies to the \verb|ATOMBASIS| and \verb|INTGRL| input modes.
Examples are provided by the following input files:
\begin{verbatim}
 1:BASIS
 2:cc-pVDZ cc-pCVQZ
 3:Direct MP2-R12/cc-pVDZ calculation on H2O
 4:Auxiliary basis: cc-pCVQZ
 5:Atomtypes=1 Generators=1 X
 6:Charge=8.0 Atoms=1 Set=1
 7:O       .000000000000000    .000000000000000  -0.124309000000000       *
 8:Charge=1.0 Atoms=1 Set=1
 9:H      1.427450200000000    .000000000000000   0.986437000000000       *
10:Charge=8.0 Atoms=1 Set=2
11:O       .000000000000000    .000000000000000  -0.124309000000000       *
12:Charge=1.0 Atoms=1 Set=2
13:H      1.427450200000000    .000000000000000   0.986437000000000       *
\end{verbatim}
\begin{verbatim}
 1:ATOMBASIS
 2:Direct MP2-R12/cc-pVDZ calculation on H2O
 3:Auxiliary basis: cc-pCVQZ
 4:Atomtypes=4 Generators=1 X
 5:Charge=8.0 Atoms=1 Set=1 Basis=cc-pVDZ
 6:O       .000000000000000    .000000000000000  -0.124309000000000       *
 7:Charge=1.0 Atoms=1 Set=1 Basis=cc-pVDZ
 8:H      1.427450200000000    .000000000000000   0.986437000000000       *
 9:Charge=8.0 Atoms=1 Set=2 Basis=cc-pCVQZ
10:O       .000000000000000    .000000000000000  -0.124309000000000       *
11:Charge=1.0 Atoms=1 Set=2 Basis=cc-pCVQZ
12:H      1.427450200000000    .000000000000000   0.986437000000000       *
\end{verbatim}
Note that the keyword \Key{R12AUX} must be specified in the 
\Sec{MOLBAS} input section to be able to read the above inputs.

\section{The basis sets supplied with \dalton }
\label{sec:basislist}
\index{basis set!library}

As was mentioned above, all the basis sets supplied with this release
of the \dalton\ program --- with a few exceptions --- have been obtained 
from the EMSL basis set library~\cite{emslref}. 
Supplied basis sets include the STO-$n$G and Pople style basis sets,
Dunning's correlation-consistent basis 
sets\index{basis!correlation-consistent}\index{correlation-consistent basis set}, 
Ahlrichs Turbomole basis sets and Huzinaga basis sets. 
In a very few cases we have corrected the
files as obtained from EMSL, however we take no responsibility
for any errors inherent in these files.

The ANO and Sadlej-pVTZ polarization basis sets have been obtained from the
MOLCAS homepage without any further processing, and should therefore
be free of errors. The NQvD basis provided to us by Knut F\ae gri 
along with the Turbomole basis sets have been slightly reformatted for more 
convenient processing of the file, hopefully without having introduced any errors. 
The pc-$n$ and apc-$n$ basis sets have been provided to us by Frank Jensen.

We have included several Turbomole basis sets,
although we have retained the Ahlrichs basis sets
from the EMSL basis set library. We recommend the Turbomole
basis sets rather than the Ahlrichs basis sets from EMSL, which
contain some errors. The Ahlrichs-VDZ basis
is similar to the Turbomole-SV basis, while the Ahlrichs-VTZ is a
combination of the Turbomole-TZ (H--Ar) and Turbomole-DZ (K--Kr).

Below we give a comprehensive list of all basis sets included in 
the basis set library\index{basis set!library}, together with 
a list of the elements supported, and the complete reference to be 
cited when employing a given basis set in a calculation. 

\setlongtables
\begin{longtable}{lll}
\multicolumn{3}{}{\bf Basis Sets included in \dalton\ distribution.} \\ 
\hline \hline
\bf{Basis set name} & \bf{Elements} & \bf{References}\\
\hline\hline
\endfirsthead
\multicolumn{3}{}{\bf{Basis Sets included in \dalton\ distribution.}} \\
\hline \hline
\bf{Basis set name} & \bf{Elements} & \bf{References}\\
\hline \hline
\endhead
\endfoot \endlastfoot
\bf{\emph{STO-$n$G}} & & \\
STO-2G & H--Ca, Sr & \cite{wjhrfsjapjcp51,wjhrdrfsjapjcp52} \\
STO-3G & H--Cd & \cite{wjhrfsjapjcp51,wjhrdrfsjapjcp52,
   wjpbalwjhrfsic19,wjpwjhjcc4} \\
STO-6G & H--Ar & \cite{wjhrfsjapjcp51,wjhrdrfsjapjcp52} \\
\hline
\bf{\emph{Pople-style basis sets}} & & \\
3-21G & H--Cs & \cite{jsbjapwjhjacs102,msgjsbjapwjpwjhjacs104,
   kddwjhjcc7,kddwjhjcc8-1,kddwjhjcc8-2,edgdfjpc99} \\
3-21G* & H--Cl & 3-21G, with polariz. functions from
  \cite{wjpmmfwjhdjdjapjsbjacs104}.\\
  & & \emph{Note}: Polariz. functions only on Na--Cl \\
3-21++G & H, Li--Ar & 3-21G, with diffuse functions from
  \cite{tcjcgwsprsjcc4}.\\
3-21++G* & H--Cl & 3-21++G and 3-21G*\\
4-31G & H--Ar & \cite{rdwjhjapjcp54,msgjsbjapwjpwjhjacs104} 
   He,Ne from Gaussian 90 \\
6-31G & H--Ar, Zn & \cite{wjhrdjapjcp56,jddjapjcp62,
   mmfwjpwjhjsbmsgdjdjapjcp77,vrjapmrtlwjcp109} 
   He,Ne from Gaussian 90 \\
6-31G* & H--Ar & 6-31G, with polariz. functions from
  \cite{pchjaptca28,mmfwjpwjhjsbmsgdjdjapjcp77}\\
6-31G** & H--Ar  & 6-31G*, with polariz. functions from
  \cite{pchjaptca28}\\
6-31+G & H--Ar & 6-31G, with diffuse functions from
  \cite{tcjcgwsprsjcc4}.\\
6-31++G & H--Ar & 6-31G, with diffuse functions from
  \cite{tcjcgwsprsjcc4}.\\
6-31+G*   & H--Ar & 6-31+G and 6-31G*\\
6-31++G*  & H--Ar & 6-31++G and 6-31G*\\
6-31++G** & H--Ar & 6-31++G and 6-31G**\\
6-31G(3df,3pd) & H--Ar & 6-31G, with polariz. functions from
  \cite{mjfjapjsbjcp80}.\\
6-311G & H--Ar, Br, I & \cite{rkjsbrsjapjcp72,admgscjcp72,
   lacmpmjpbnedrcblrjcp103,mngapmpmlrjcp103}\\
6-311G* & H--Ar, Br, I & 6-311G, with polariz. functions from
  \cite{rkjsbrsjapjcp72,lacmpmjpbnedrcblrjcp103}.\\
6-311G** & H--Ar, Br, I & 6-311G, with polariz. functions from
  \cite{rkjsbrsjapjcp72,lacmpmjpbnedrcblrjcp103}.\\
6-311+G* & H--Ne & 6-311G* with diffuse functions from 
  \cite{tcjcgwsprsjcc4}.\\
6-311++G** & H--Ne & 6-311G** with diffuse functions from
  \cite{tcjcgwsprsjcc4}.\\
6-311G(2df,2pd) & H--Ne & 6-311G, with polariz. functions from
  \cite{mjfjapjsbjcp80}.\\
6-311++G(2d,2p) & H--Ne & 6-311++G, with polariz. functions from
  \cite{mjfjapjsbjcp80}.\\
6-311++G(3df,3pd) & H--Ar & 6-311++G, with polariz. functions from
  \cite{mjfjapjsbjcp80}.\\
\hline
%
Huckel & H--Cd & \\
MINI(Huzinaga) & H--Ca & \cite{huzinagabasis} \\
MINI(Scaled) & H--Ca & \cite{huzinagabasis,htshjcc1} \\
\hline
%
\bf{\emph{Dunning-Hay basis sets}} & & \\
SV(Dunning-Hay) & H, Li--Ne & \cite{thdpjhhfs1977} \\
SVP(Dunning-Hay) & H, Li--Ne & SV, with polariz. functions from
  \cite{thdpjhhfs1977,emhfsjcp83}.\\
SVP+Diffuse(Dunning-Hay) & H, Li--Ne & SVP with diffuse functions from
  \cite{thdpjhhfs1977,emhfsjcp83}.\\
SV+Rydberg(Dunning-Hay) & H, Li--Ne & SV, with polariz. functions from
  \cite{thdpjhhfs1977-2}.\\
\small{SV+DoubleRydberg(Dunning-Hay)} & H, Li--Ne & SV, with polariz. functions from
  \cite{thdpjhhfs1977-2}.\\
DZ(Dunning) & H, B--Ne, Al--Cl & \cite{thdjcp53,thdpjhhfs1977} \\
DZP(Dunning) & H, B--Ne, Al--Cl & DZ, with polariz. functions from 
  \cite{thdpjhhfs1977,emhfsjcp83}.\\
DZP+Diffuse(Dunning) & H, B--Ne & DZP with diffuse functions from
  \cite{thdpjhhfs1977,emhfsjcp83}.\\
DZ+Rydberg(Dunning) & H, B--Ne, Al--Cl & DZ, with polariz. functions from
  \cite{thdpjhhfs1977-2}.\\
DZP+Rydberg(Dunning) & H, B--Ne, Al--Cl & DZP, with polariz. functions from 
  \cite{thdpjhhfs1977-2}.\\
TZ(Dunning) & H, Li--Ne & \cite{thdjcp55}\\
\hline
\multicolumn{2}{l}{\bf{\emph{Dunning's correlation-consistent basis 
sets\index{basis set!correlation-consistent}\index{correlation-consistent basis set}}}} \\
\multicolumn{3}{l}{\emph{Note: H,He (valence only) are included in all core-valence basis sets 
  for convenience.}} \\
cc-pVXZ (X = D,T,Q,5,6) & H--Ne, Al--Ar, & \cite{thdjcp90,dewthdjcp100,dewthdjcp98,
  jkkapjpca106,akwdewkapthdjcp110} \\
  & Ca, Ga--Kr & \emph{Note}: 6Z only includes H,C--O\\
cc-pCVXZ (X = D,T,Q,5) & H, He, B--Ne, & cc-pVXZ, with core functions from
  \cite{thdjcp90,jkkapjpca106,dewthdjcp103}. \\
  & Na--Ar & \emph{Note}: 5Z only includes H, He, B--Ne. \\
cc-pwCVXZ (X = D,T,Q,5) & H, He, C--Ne,  & cc-pCVXZ with core functions from 
  \cite{thdjcp90,kapthdjcp117} \\
  & Al--Ar & \\
aug-cc-pVXZ (X = D,T,Q,5) & H, He, B--Ne, & cc-pVXZ, with aug. functions from 
  \cite{thdjcp90,rakthdrjhjcp96,dewthdjcp98,dewthdjcp100}.\\
  & Al--Ar, Ga--Kr & \\
aug-cc-pV6Z & H, He, B--Ne, & \cite{akwtvmthdjsm388,tmakwtdhmp96} with aug. functions from 
  \cite{akwtvmthdjsm388,tmakwtdhmp96,tmthdijqc76}.\\
  & Al--Ar & \\
aug-cc-pCVXZ (X = D,T,Q,5) & H, He, B--F, & aug-cc-pVXZ, cc-pCVXZ and \cite{dewthdjcp103,kapthdjcp117}.\\
  & Ne, Al-Ar & \emph{Note}: Ne only available for TZ and QZ \\
$n$-aug-cc-pVXZ ($n$ = d,t,q) & as aug-cc-pVXZ & aug-cc-pVXZ. See Sec.~\ref{sec:molbasis} \\
$n$-aug-cc-pCVXZ ($n$ = d,t,q) & as aug-cc-pCVXZ & aug-cc-pCVXZ. See Sec.~\ref{sec:molbasis} \\
\hline
\newpage
cc-pVXZdenfit (X=T,Q,5) & H, B--F, Al--Cl & Turbomole program \\
cc-pVXZ-DK (X = D,T,Q,5) & H,He,B--Ne, & \cite{thdjcp90,dewthdjcp100,dewthdjcp98,akwdewkapthdjcp110} 
  cc-pVXZ re-contracted for \\
  & Al--Ar, Ga--Kr & Douglas-Kroll calculations.\\
\hline
\multicolumn{2}{l}{\bf{\emph{Frank Jensen's polarization-consistent basis 
sets\index{basis set!polarization-consistent}\index{polarization-consistent basis set}}}} \\
pc-$n$ ($n$ = 0,1,2,3,4)  & H, C--F & \cite{fjjcp115,fjjcp116} \\
                      & Si--Cl & \cite{fjthjcp121} \\
apc-$n$ ($n$ = 0,1,2,3,4) & H, C--F & pc-$n$, with aug. functions from \cite{fjjcp117} \\
                      & Si--Cl & \cite{fjthjcp121} \\
\hline
\bf{\emph{Ahlrich's Turbomole basis sets}} & & \\
\multicolumn{3}{l}{\emph{Note}: See text above -- Turbomole basis sets are preferred over EMSL sets} \\
Turbomole-SV & H--Kr & \cite{ashhrajcp97} Turbomole SV basis\\
Turbomole-XZ (X=D,T) & H--Kr & \cite{ashhrajcp97} Turbomole DZ,TZ basis\\
Turbomole-TZV & H--Kr & \cite{aschrajcp100} Turbomole TZV basis \\
Turbomole-XP & H--Kr & \cite{ashhrajcp97,aschrajcp100} Turbomole polariz. basis \\
 ~~~~(X=SV,DZ,TZ,TZV) & & \\
Turbomole-TZVPP & H--Kr & \cite{aschrajcp100} Turbomole TZVPP basis \\
Turbomole-TZVPPP & H--He,B--Ne,Al--Ar & \cite{aschrajcp100} Turbomole TZVPPP basis \\
Ahlrichs-VXZ (X=D,T) & H--Kr & \cite{ashhrajcp97} From EMSL \\
Ahlrichs-pVDZ & H--Kr & \cite{ashhrajcp97} From EMSL: polariz. functions unpublished.\\
\hline
%
\bf{\emph{Huzinaga basis sets}} & & \\
Huz-II & H, C--F, P, S & \cite{wkijc19,mswkjcp76,huzinagaintern} All the Huz basis sets are of\\
Huz-IIsu2 & H, C--F, P, S & \cite{wkijc19,mswkjcp76,huzinagaintern} approximate valence TZ quality\\
Huz-III & H, C--F, P, S & \cite{wkijc19,mswkjcp76,huzinagaintern} \\
Huz-IIIsu3 & H, C--F, P, S & \cite{wkijc19,mswkjcp76,huzinagaintern}\\
Huz-IV & H, C--F, P, S & \cite{wkijc19,mswkjcp76,huzinagaintern}\\
Huz-IVsu4 & H, C--F, P, S & \cite{wkijc19,mswkjcp76,huzinagaintern}\\
\hline
GAMESS-VTZ & H, Be--Ne, Na--Ar & \cite{thdjcp55,admgscjcp72,ajhwjcp52} From GAMESS program \\
GAMESS-PVTZ & H, Be--Ne & \cite{thdjcp55,admgscjcp72,ajhwjcp52} From GAMESS program \\
%
McLean-Chandler-VTZ & Na--Ar & \cite{admgscjcp72} \\
Wachtersa+f & Sc--Cu & \cite{ajhwjcp52,wachterintern1969} with $f$ functions from \cite{cwbsrllabjcp91} \\
Sadlej-pVTZ & H--I & \cite{ajstca79,ajsmujmst234,ajstca81,ajstca81-2} \\
  &  & \emph{Note}: No rare-gas, B, Al, Ga or In basis sets\\
%
Sadlej-pVTZ-J & H, C--O, S & \cite{pfpgaaspasjcp115} Sadlej-pVTZ optimized for NMR calcs.\\
aug-cc-pVTZ-J & H, C--F, S & \cite{pfpgaaspasjcp115} aug-cc-pVTZ optimized for NMR calcs. \\
\hline
\multicolumn{2}{l}{\bf{\emph{ANO basis sets\index{basis!ANO} -- see Sec.~\ref{sec:molbasis}.}}} & \\
NQvD & & \cite{nqvdref} \\
Almlof-Taylor-ANO & H--Ne & \cite{japrtjcp86} Almlof and Taylor ANO \\
NASA-Ames-ANO & H, B--Ne, & \cite{japrtjcp86,cwbsrlaktca77} \\
  & Al, P, Ti, Fe, Ni & \\
ano-1 & H--Ne & \cite{powpambortca77} Roos Augmented ANO basis sets\\
ano-2 & Na--Ar & \cite{powbjpbortca79} \\
ano-3 & Sc--Zn & \cite{rpamminpowbortca92} \\
ano-4 & H--Kr & \cite{kpbdpowbortca90} Roos ANO basis sets\\
\hline
raf-r & O, Y--Pd, Hf--Tl, & Wahlgren/Faegri Relavistic Basis Set\\
 & Po, Th, U & \\
\hline
\multicolumn{3}{l}{\bf{\emph{Effective core potential (ECP) basis sets}}} \\
\multicolumn{3}{l}{\emph{Note}: ecp-sdd-DZ is Stuttgart ECP valence basis sets included in previous Dalton releases,}\\
\multicolumn{3}{l}{see http://www.theochem.uni-stuttgart.de/pseudopotentials/index.en.html.}\\
\multicolumn{3}{l}{Others are from EMSL with the EMSL name given in the third column,}\\
\multicolumn{3}{l}{please check EMSL for complete reference.}\\
aug\_cc\_pvdz\_pp & Cu-Kr, Y-Xe, & aug-cc-pVDZ-PP\\
                  & Hf-Rn &\\
aug\_cc\_pvtz\_pp & Cu-Kr, Y-Xe, & aug-cc-pVTZ-PP\\
                  & Hf-Rn &\\
aug\_cc\_pvqz\_pp & Cu-Kr, Y-Xe, & aug-cc-pVQZ-PP\\
                  & Hf-Rn &\\
aug\_cc\_pv5z\_pp & Cu-Kr, Y-Xe, & aug-cc-pV5Z-PP\\
                  & Hf-Rn &\\
cc\_pvdz\_pp & Cu-Kr, Y-Xe, & cc-pVDZ-PP\\
             & Hf-Rn &\\
cc\_pvtz\_pp & Cu-Kr, Y-Xe, & cc-pVTZ-PP\\
             & Hf-Rn &\\
cc\_pvqz\_pp & Cu-Kr, Y-Xe, & cc-pVQZ-PP\\
             & Hf-Rn &\\
cc\_pv5z\_pp & Cu-Kr, Y-Xe, & cc-pV5Z-PP\\
             & Hf-Rn &\\
cc\_pwcvdz\_pp & Cu, Zn, Y-Cd, & cc-pwCVDZ-PP\\
               & I, Hf-Hg &\\
cc\_pwcvtz\_pp & Cu, Zn, Y-Cd, & cc-pwCVTZ-PP\\
               & I, Hf-Hg &\\
cc\_pwcvqz\_pp & Cu, Zn, Y-Cd, & cc-pwCVQZ-PP\\
               & I, Hf-Hg &\\
cc\_pwcv5z\_pp & Cu, Zn, Y-Cd, & cc-pwCV5Z-PP\\
               & Hf-Hg &\\
crenbl\_ecp & H, Li-Uus & CRENBL ECP\\
crenbs\_ecp & Sc-Zn, Y-Cd, La, & CRENBS ECP\\
            & Hf-Rn, Rf-Uus &\\
def2\_qzvp & H-La, Hf-Rn & Def2-QZVP\\
def2\_qzvpp & H-La, Hf-Rn & Def2-QZVPP\\
def2\_sv\_p & H-La, Hf-Rn & Def2-SV(P)\\
def2\_svp & H-La, Hf-Rn & Def2-SVP\\
def2\_tzvp & H-La, Hf-Rn & Def2-TZVP\\
def2\_tzvpp & H-La, Hf-Rn & Def2-TZVPP\\
dzq & Y-Ag & DZQ\\
ecp-sdd-DZ & Li, B-F, Na-Cl, & Stuttgart ECP valence basis sets\\
           & K-Sc, Cr-Kr, Sr, &\\
           & Zr-Ba, Hf-Bi &\\
hay\_wadt\_mb\_n1\_ecp & K-Cu, Rb-Ag, & Hay-Wadt MB (n+1) ECP\\
                       & Cs-Au &\\
hay\_wadt\_vdz\_n1\_ecp & K-Cu, Rb-Ag, & Hay-Wadt VDZ (n+1) ECP\\
                        & Cs-Au &\\
lanl08 & Na-La, Hf-Bi & LANL08\\
lanl08\_f & Sc-Cu, Y-Ag, & LANL08(f)\\
          & La, Hf-Au &\\
lanl08\_p & Sc-Zn & LANL08+\\
lanl08d & Si-Cl, Ge-Br, & LANL08d\\
        & Sn-I, Pb-Bi &\\
lanl2dz\_ecp & H, Li-La, Hf-Au, & LANL2DZ ECP\\
             & Pb-Bi, U-Pu &\\
lanl2dzdp\_ecp & C-F, Si-Cl, & LANL2DZdp ECP\\
               & Ge-Br, Sn-I, &\\
               & Pb-Bi &\\
lanl2tz & Sc-Zn, Y-Cd, & LANL2TZ\\
        & La, Hf-Hg &\\
lanl2tz\_f & Sc-Cu, Y-Ag, & LANL2TZ(f)\\
           & La, Hf-Au &\\
lanl2tz\_p & Sc-Zn & LANL2TZ+\\
modified\_lanl2dz & Sc-Cu, Y-Ag, & modified LANL2DZ\\
                  & La, Hf-Au &\\
sbkjc\_polarized\_p\_2d\_lfk & H-Ca, Ge-Sr, & SBKJC Polarized (p,2d) - LFK\\
                             & Sn-Ba, Pb-Rn &\\
sbkjc\_vdz\_ecp & H-Ce, Hf-Rn & SBKJC VDZ ECP\\
sdb\_aug\_cc\_pvqz & Ga-Br, In-I & SDB-aug-cc-pVQZ\\
sdb\_aug\_cc\_pvtz & Ga-Br, In-I  & SDB-aug-cc-pVTZ\\
sdb\_cc\_pvqz & Ga-Kr, In-Xe & SDB-cc-pVQZ\\
sdb\_cc\_pvtz & Ga-Kr, In-Xe & SDB-cc-pVTZ\\
stuttgart\_rlc\_ecp & Li-Ca, Zn-Sr, & Stuttgart RLC ECP\\
                    & In-Ba, Hg-Rn, &\\
                    & Ac-Lr &\\
stuttgart\_rsc\_1997\_ecp & K-Zn, Rb-Cd, & Stuttgart RSC 1997 ECP\\
                          & Cs-Ba, Ce-Yb, &\\
                          & Hf-Hg, Ac-Lr, &\\
                          & Db &\\
stuttgart\_rsc\_ano\_ecp & La-Lu & Stuttgart RSC ANO/ECP\\
stuttgart\_rsc\_segmented\_ecp & La-Lu & Stuttgart RSC Segmented/ECP\\
\hline
\end{longtable}

In the following, we give the comprehensive list of all ECPs included in the current release,
together with a list of the elements supported. The ecp-sdd-DZ is Stuttgart
ECPs included in previous Dalton releases~(see http://www.theochem.uni-stuttgart.de/pseudopotentials/index.en.html),
while others are from EMSL, please check EMSL for the complete reference to be cited.

\setlongtables
\begin{longtable}{lll}
\multicolumn{3}{}{\bf ECPs included in \dalton\ distribution.} \\ 
\hline \hline
\bf{ECP name} & \bf{Elements} & \bf{EMSL name}\\
\hline\hline
\endfirsthead
\multicolumn{3}{}{\bf{ECPs included in \dalton\ distribution.}} \\
\hline \hline
\bf{ECP name} & \bf{Elements} & \bf{EMSL name}\\
\hline \hline
\endhead
\endfoot \endlastfoot
aug\_cc\_pvdz\_pp & Cu-Kr, Y-Xe, Hf-Rn & aug-cc-pVDZ-PP\\
aug\_cc\_pvtz\_pp & Cu-Kr, Y-Xe, Hf-Rn & aug-cc-pVTZ-PP\\
aug\_cc\_pvqz\_pp & Cu-Kr, Y-Xe, Hf-Rn & aug-cc-pVQZ-PP\\
aug\_cc\_pv5z\_pp & Cu-Kr, Y-Xe, Hf-Rn & aug-cc-pV5Z-PP\\
cc\_pvdz\_pp & Cu-Kr, Y-Xe, Hf-Rn & cc-pVDZ-PP\\
cc\_pvtz\_pp & Cu-Kr, Y-Xe, Hf-Rn & cc-pVTZ-PP\\
cc\_pvqz\_pp & Cu-Kr, Y-Xe, Hf-Rn & cc-pVQZ-PP\\
cc\_pv5z\_pp & Cu-Kr, Y-Xe, Hf-Rn & cc-pV5Z-PP\\
cc\_pwcvdz\_pp & Cu, Zn, Y-Cd, I, Hf-Hg & cc-pwCVDZ-PP\\
cc\_pwcvtz\_pp & Cu, Zn, Y-Cd, I, Hf-Hg & cc-pwCVTZ-PP\\
cc\_pwcvqz\_pp & Cu, Zn, Y-Cd, I, Hf-Hg & cc-pwCVQZ-PP\\
cc\_pwcv5z\_pp & Cu, Zn, Y-Cd, Hf-Hg & cc-pwCV5Z-PP\\
crenbl\_ecp & Li-Uus & CRENBL ECP\\
crenbs\_ecp & Sc-Zn, Y-Cd, La, Hf-Rn, Rf-Uus & CRENBS ECP\\
def2\_qzvp & Rb-La, Hf-Rn & Def2-QZVP\\
def2\_qzvpp & Rb-La, Hf-Rn & Def2-QZVPP\\
def2\_sv\_p & Rb-La, Hf-Rn & Def2-SV(P)\\
def2\_svp & Rb-La, Hf-Rn & Def2-SVP\\
def2\_tzvp & Rb-La, Hf-Rn & Def2-TZVP\\
def2\_tzvpp & Rb-La, Hf-Rn & Def2-TZVPP\\
dzq & Y-Ag & DZQ\\
ecp-sdd-DZ & Li-Mg, Si-Ce, Nd, Sm-Tb, & Stuttgart ECPs\\
           & Ho, Yb-Bi, Rn &\\
hay\_wadt\_mb\_n1\_ecp & K-Cu, Rb-Ag, Cs-La, Hf-Au & Hay-Wadt MB (n+1) ECP\\
hay\_wadt\_vdz\_n1\_ecp & K-Cu, Rb-Ag, Cs-La, Ta-Au & Hay-Wadt VDZ (n+1) ECP\\
lanl08 & Na-La, Hf-Bi & LANL08\\
lanl08\_f & Sc-Cu, Y-Ag, La, Hf-Au & LANL08(f)\\
lanl08\_p & Sc-Zn & LANL08+\\
lanl08d & Si-Cl, Ge-Br, Sn-I, Pb-Bi & LANL08d\\
lanl2dz\_ecp & Na-La, Hf-Au, Pb-Bi, U-Pu & LANL2DZ ECP\\
lanl2dzdp\_ecp & Si-Cl, Ge-Br, Sn-I, Pb-Bi & LANL2DZdp ECP\\
lanl2tz & Sc-Zn, Y-Cd, La, Hf-Au & LANL2TZ\\
lanl2tz\_f & Sc-Cu, Y-Ag, La, Hf-Au & LANL2TZ(f)\\
lanl2tz\_p & Sc-Zn & LANL2TZ+\\
modified\_lanl2dz & Sc-Cu, Y-Ag, La, Hf-Au & modified LANL2DZ\\
sbkjc\_polarized\_p\_2d\_lfk & Li-Ca, Ge-Sr, Sn-Ba, Pb-Rn & SBKJC Polarized (p,2d) - LFK\\
sbkjc\_vdz\_ecp & Li-Ce, Hf-Rn & SBKJC VDZ ECP\\
sdb\_aug\_cc\_pvqz & Ga-Br, In-I & SDB-aug-cc-pVQZ\\
sdb\_aug\_cc\_pvtz & Ga-Br, In-I & SDB-aug-cc-pVTZ\\
sdb\_cc\_pvqz & Ga-Kr, In-Xe & SDB-cc-pVQZ\\
sdb\_cc\_pvtz & Ga-Kr, In-Xe & SDB-cc-pVTZ\\
stuttgart\_rlc\_ecp & Li-Ca, Zn-Sr, In-Ba, Hg-Rn, Ac-Lr & Stuttgart RLC ECP\\
stuttgart\_rsc\_1997\_ecp & K-Zn, Rb-Cd, Cs-Ba, Cs-Yb, & Stuttgart RSC 1997 ECP\\
                          & Hf-Hg, Ac-Lr, Db &\\
stuttgart\_rsc\_ano\_ecp & La-Lu & Stuttgart RSC ANO/ECP\\
stuttgart\_rsc\_segmented\_ecp & La-Lu & Stuttgart RSC Segmented/ECP\\
\hline
\end{longtable}

%
% checked with "ispell" sep.2003 /hjaaj
\chapter{\label{chap:sirius-inpref}Molecular wave functions, {\sir}}

\section{\label{sec:sirius-ref-notes} General notes for the {\sir} input reference
manual}

{\sir} is the part of the code that computes the wave function/density.

The following sections contain a list of all generally relevant keywords to
{\sir}, only currently inactive keywords and some special debug
options are omitted.

\begin{enumerate}
\item {The input for the wave function section must begin with

\begin{inputex} \begin{verbatim}
**WAVE FUNCTIONS
\end{verbatim} \end{inputex}
   with no leading blanks.  The preceding lines in the input file may
   contain arbitrary information.
}
\item{ Input is directed by keywords written in upper case.
   Only the first 7 characters including the prompt are significant.
   The keywords are divided in a number of main input groups. Each main
   input group is initiated by a {\starkey}. For example

\begin{inputex} \begin{verbatim}
*ORBITAL INPUT
\end{verbatim} \end{inputex}
   marks the beginning of the input group for orbital input.
}
\item { The keywords belonging to one of the main input groups begin with
   the prompt {\dotkey}.
}
\item { Keywords that are necessary to specify are marked by "Required".
   For other keywords the default values can be used in ordinary runs.
}
\item {Any keyword line beginning with a \quotekw{!} or
   \quotekw{\#} will be treated as a
   comment line.  An illegal keyword will cause a dump of all keywords
   for the current input section.
}
\item{A dump of keywords can be obtained in any input section by
specifying the keyword \quotekw{\Key{OPTIONS}}.  For example, the input

\begin{inputex} \begin{verbatim}
**WAVE FUNCTIONS
.OPTIONS
**END OF DALTON INPUT
\end{verbatim} \end{inputex}
   will cause a dump of the labels for the main input groups in {\sir},
   while

\begin{inputex} \begin{verbatim}
**WAVE FUNCTIONS
*ORBITAL INPUT
.OPTIONS
**END OF DALTON INPUT
\end{verbatim} \end{inputex}
   will cause a dump of the labels for the \quotekw{*ORBITAL INPUT} input group
   in {\sir}.
}
\item{ The {\sir} input is finished with a line beginning with two stars,
   {\it e.g.\/}

\begin{inputex} \begin{verbatim}
**END OF DALTON INPUT
\end{verbatim} \end{inputex}
}
\end{enumerate}


\pagebreak[3]
\section{\label{sec:ref-newinp}
   Main input groups in the **WAVE FUNCTIONS input module}

\noindent
The main input groups (those with the {\starkey} prompt) are listed here and
the full descriptions are given in the designated sections.

\noindent
The first input group is always required in order to specify the type of
calculation, and follows immediately after the \Sec{*WAVE FUNCTIONS}
keyword.

%Section~\ref{ref-geninp} \Sec{GENERAL INPUT}

\noindent The remaining input groups may be specified in any
order. In this chapter they are grouped alphabetically, although
the short presentation below gather them according to purpose.

%\ifsolvent
The following two input groups are used to modify the
molecular environment by adding field-dependent
terms in the Hamiltonian and by invoking
the self-consistent reaction field model for solvent
effects, respectively:

Section~\ref{ref-haminp} \Sec{HAMILTONIAN}

Section~\ref{ref-solinp} \Sec{SOLVENT}
%\else
%The following input group describes additional field-dependent
%terms in the Hamiltonian :
%
%Section~\ref{ref-haminp} \Sec{HAMILTONIAN}
%\fi

\noindent
The next input group specifies the configurations included
in the MCSCF and CI wave functions:

Section~\ref{ref-wavinp} \Sec{CONFIGURATION INPUT}

\noindent
The two next groups are used to specify initial orbitals and initial
guess for the CI vector:

Section~\ref{ref-orbinp} \Sec{ORBITAL INPUT}

Section~\ref{ref-civinp} \Sec{CI VECTOR}

\noindent
The two following input groups control the second-order MCSCF
optimization:

Section~\ref{ref-optinp} \Sec{OPTIMIZATION}

Section~\ref{ref-stpinp} \Sec{STEP CONTROL}

\noindent
The next five groups have special input only relevant for the
respective calculation types:

Section~\ref{ref-rhfinp} \Sec{SCF INPUT}

Section~\ref{ref-dftinp} \Sec{DFT INPUT}

Section~\ref{ref-mp2inp} \Sec{MP2 INPUT}

Section~\ref{ref-nevpt2inp} \Sec{NEVPT2 INPUT}

Section~\ref{ref-cicinp} \Sec{CI INPUT}

\noindent
The next section is used to select some types of analysis of the final
Hartree--Fock, DFT, MCSCF, or CI wave function:

Section~\ref{ref-popinp} \Sec{POPULATION ANALYSIS}

\noindent
The next section is used to change the default integral transformation
and specify any final integral transformation after convergence (a
program following {\sir} may need a higher transformation level):

Section~\ref{ref-trainp} \Sec{TRANSFORMATION}

\noindent
The next two input groups control the amount of printed output and
collect options not fitting in any of the other groups:

Section~\ref{ref-priinp} \Sec{PRINT LEVELS}

Section~\ref{ref-auxinp} \Sec{AUXILIARY INPUT}

\noindent
Finally we note that there is an input module controlling the
calculation of coupled cluster wave functions. This is treated in a
separate chapter:

Chapter~\ref{ch:CC} \Sec{CC INPUT}

\bigskip
\noindent
The wave function input is finished when a line is encountered beginning
with two stars, for example

\begin{inputex} \begin{verbatim}
   **END OF DALTON INPUT
\end{verbatim} \end{inputex}
or

\begin{inputex} \begin{verbatim}
   **MOLORB
   ... formatted molecular orbitals coefficients
   **END OF DALTON INPUT
\end{verbatim} \end{inputex}

\noindent
The \Sec{*MOLORB} keyword or the \Sec{*NATORB} keyword
must be somewhere on the input file and be
followed by molecular orbital coefficients if the option for formatted
input of molecular orbitals has been specified.  Apart from this
requirement, arbitrary information can be written to the following lines
of the input file.



\pagebreak[3]
\subsection{\label{ref-geninp}\Sec{*WAVE FUNCTIONS}}

{\bf Purpose:}

Specification of which wave function calculation is to be performed.

{\bf Primary keywords, listed in the order the corresponding modules
 will be executed by the program (if the keyword is set): }

\begin{description}

\item[\Key{HF}]
  Restricted closed-shell or one open-shell 
  Hartree--Fock\index{HF}\index{SCF}\index{Hartree--Fock}\index{open shell!HF} calculation.
  The occupied orbitals, optimization control etc. are specified in the
  \quotekw{*SCF INPUT} submodule.
  Note: you cannot specify both \quotekw{.HF} and \quotekw{.DFT} keywords.

\item[\Key{DFT}]
  \verb"READ (LUINP,'(A80)) LINE" \\ Restricted closed-shell, one
  open-shell or high-spin spin-restricted Kohn-Sham density functional
  theory\index{DFT}\index{Kohn--Sham}\index{open shell!DFT}
  calculation.  On the following line you must specify which
  functional to use.  The occupied orbitals, optimization control
  etc. are specified in the \quotekw{*SCF INPUT} submodule shared with
  the \quotekw{.HF} option.  The DFT specific input options are
  collected in the \quotekw{*DFT INPUT} input submodule.  Note: you
  cannot specify both \quotekw{.HF} and \quotekw{.DFT} keywords.

\item[\Key{MP2}]
 \index{M{\o}ller-Plesset!second-order}\index{MP2}
  M{\o}ller-Plesset second-order perturbation theory calculation.
  Requires \quotekw{.HF} or previously calculated canonical Hartree--Fock orbitals.

\item[(\quotekw{.FC MVO} in \quotekw{*SCF INPUT}]
  Calculation of modified virtual SCF orbitals based on the
  potential determined by the keyword (see comments below).
  The occupied SCF orbitals are not modified.
  Note that this keyword is not located in this module but in the
  \quotekw{*SCF INPUT} submodule. It is mentioned here to make clear
  at what point this transformation will be performed, if requested.

\item[\Key{CI}]
  \index{CI}\index{Configuration Interaction}
  Configuration interaction calculation.

\item[\Key{MCSCF}]
  Multiconfiguration self-consistent field (MCSCF) calculation.\index{MCSCF}

\item[\Key{NEVPT2}]
  Multireference second-order perturbation theory calculation.\index{NEVPT2}

\item[\Key{CC}]
  \index{Coupled Cluster}\index{CC}
  Coupled cluster calculation. Automatically activates Hartree--Fock (\quotekw{.HF}).
  After the Hartree--Fock calculation,
  the \cc\ module is called to do a coupled cluster (response) calculation.
  For further input options for the \cc\ module see Section~\ref{sec:ccgeneral}. 

\item[\Key{CC ONLY}] Skip the calculation of a Hartree--Fock wave
  function, and start directly in the coupled cluster part. Convenient
  for restarts in the coupled cluster module.

\end{description}

{\bf Secondary keywords (in alphabetical order): }

\begin{description}

\item[\Key{FLAGS}]
  \verb"READ (LUINP,NMLSIR)" \\
  Read namelist \verb"$NMLSIR ... $END" \\
  Only for debugging. Set internal control flags directly.
  Usage is not documented.

\item[\Key{INTERFACE}]
  Write the "\verb|SIRIFC|" interface file\index{interface file} for post-processing programs.

%hjaaj oct 2003: obsolete ...
%\item[\Key{NSYM}]
%%  Required, no defaults. \\
%  Default: specified by integral program \\
%  \verb"READ (LUINP,*) NSYM" \\
%  Number of spatial Abelian symmetries (1, 2, 4, or 8), corresponding
%  to $D_{2h}$ or one of its subgroups.

\item[\Key{PRINT}]
  \verb"READ (LUINP,*) IPRSIR" \\
  General {\sir} print level and default for all other print parameters in this module.

\item[\Key{RESTART}]
  restart {\sir}\index{restart!wave function},
  the {\sir} restart file (\verb|SIRIUS.RST|) must be available

%hjaaj Oct 2003: obsolete, we want the specific .HF or .DFT
%\item[\Key{SCF}]
%    Alias for the \quotekw{\Key{HF}} keyword.

\item[\Key{STOP}]
  \kw{READ (LUINP,'(A20)') REWORD} \\
  Terminate {\sir} according to the instruction given on the following line.
  Three stop points are defined:
\begin{enumerate}

\item \hspace{2em} \quotekw{ AFTER INPUT}

\item \hspace{2em} \quotekw{ AFTER MO-ORTHONORMALIZATION}

\item \hspace{2em} \quotekw{ AFTER GRADIENT} (only for MCSCF and 2nd-order HF or DFT)
\end{enumerate}

\item[\Key{TITLE}] 
  \verb"READ (LUINP,'(A)') TITLE(NTIT)" 
  Any number of title lines (until next line beginning with
  \quotekw{.} or \quotekw{*} prompt).
  Up to 6 title lines will be saved and used in the output, additional
  lines will be discarded.

\item[\Key{WESTA}]
  Write the "\verb|SIRIFC|" interface file for the WESTA post-processing program.

%hjaaj Oct 2003: obsolete
%\item[\Key{BASIS SET}]
%   Default: specified by integral program \\
%   \verb"READ (LUINP,*) (NBAS(I), I = 1,NSYM)" \\
%   Number of basis functions per symmetry.
\end{description}

%hj%\noindent{\bf Comments:}

%\ifabacus ABACUS: 
%If the full molecular Hessian is calculated in
%ABACUS and the number of symmetries (\verb|NSYM|) is greater than
%one, then the MCSCF wave function will be automatically calculated
%in determinants\index{determinants} and, if singlet,
%\quotekw{.PLUS COMBINATIONS}).  This is so because the CSFs can
%only have one spatial symmetry, and it is generally necessary to
%solve linear response equations of several symmetries to get the
%full molecular Hessian. 
%\fi

%hjaaj Oct 2003: obsolete
%BASIS SET is provided such that the number of basis functions in each
%symmetry may be specified if {\sir} is modified to interface to an
%integral program which doesn't write this information to the integral
%file.

\pagebreak[3]
\subsection{\label{ref-auxinp}\Sec{AUXILIARY INPUT}}

{\bf Purpose:}

Input which does not naturally fit into any of the other
categories.

\begin{description}
\item[\Key{NOSUPMAT}]
  Do not use P-SUPERMATRIX integrals, but calculate Fock matrices
  from AO integrals (slower, but requires less disk space). The
  default is to use the supermatrix file if it exists. See option
  \Key{NOSUP} in Chapter~\ref{sec:herminp}.

%\item[\Key{ONESUP}]
%  Use same unit for P-SUPERMATRIX\index{supermatrix} and ONE-ELECTRON
%  integrals \index{one-electron integral}
%  (LUSUPM=LUONEL, default is different units).
\end{description}

\pagebreak[3]
\subsection{\label{ref-cicinp}\Sec{CI INPUT}}

{\bf Purpose:}

Options for a CI calculation.

\begin{description}
\item[\Key{CIDENSITY}]
  Calculate CI one-electron density matrix and natural
  orbital\index{natural orbital}
  occupations after convergence.

\item[\Key{CINO}]
  Generate CI\index{CI}\index{Configuration Interaction} natural
  orbitals\index{natural orbital} for CI root
  number \kw{ISTACI},
  clear the \verb|SIRIUS.RST| file and write the orbitals with label \quotekw{NEWORB  }.
  The \quotekw{\Key{STATE}} option must be specified.

\item[\Key{CIROOTS}]
  Default: NROOCI = 1\\
  \kw{READ (LUINP,*) NROOCI} \\
  Converge the lowest \kw{NROOCI} CI roots\index{root!CI} to threshold.

\item[\Key{DISKH2}]
  Active two-electron MO integrals on disk (see comments below).

\item[\Key{MAX ITERATIONS}]
  \kw{READ (LUINP,*) MXCIMA} \\
  Max iterations in iterative diagonalization of CI matrix (default = 20).

\item[\Key{STATE}]
  Default: not specified\\
  \kw{READ (LUINP,*) ISTACI} \\
  Alternative to \quotekw{\Key{CIROOTS}}.  Converge root number \kw{ISTACI}
  to threshold, converge all lower roots only to THQMIN
  (from the \quotekw{\Sec{OPTIMIZATION}} input group, see
  p.~\pageref{ref-optinp}).

\item[\Key{THRESH}]
  Default = 1.D-05\\
  \kw{READ (LUINP,*) THRCI} \\
  Threshold for CI energy gradient.  The CI energy will be converged to
  approximately the square of this number.

\item[\Key{THRPWF}]
  Default = 0.05D0 for electronic ground states, and 0.10D0 for
  excited states\\
  \kw{READ (LUINP,*) THRPWF} \\
  Only CI coefficients greater than threshold are printed
  (PWF: print wave function).

\item[\Key{WEIGHTED RESIDUALS}]
  Use energy weighted residuals\index{residual} (see comments below).

\item[\Key{ZEROELEMENTS}]
  Zero small elements in CI trial vectors (see comments below).
\end{description}


\noindent{\bf Comments:}

DISKH2: By default the CI module will attempt to place the two-electron
integrals with four active indices in memory for more efficient
calculation of CI sigma vectors, if memory is insufficient for this
the
integrals will automatically be placed on disk.  The DISKH2 keyword
forces the integrals always to be on disk.

WEIGHTED RESIDUALS:  Normally the CI states will be converged to a
residual less than the specified threshold, and this will give
approximately the squared number of decimal places in the energy.
Depending on the value of the energy, the eigenvectors will be converged
to different accuracy. If the eigenvectors are wanted with, for instance at
least 6 decimal places for property calculations, specify a threshold of
1.0D-6 and weighted residuals.

ZEROELEMENTS: an experimental option that might save time (if the CI
module can use sparseness) by zeroing all elements less than 1.0D-3
times the largest element in the CI trial vector before
orthonormalization against previous trial vectors.
See also \quotekw{.SYM CHECK} under \quotekw{*OPTIMIZATION}
(p.~\pageref{ref-optinp}).


\pagebreak[3]
\subsection{\label{ref-civinp}\Sec{CI VECTOR}}

{\bf Purpose:}

To obtain initial guess for CI vector(s).

\begin{description}
\item[\Key{PLUS COMBINATIONS}]
  Use with \quotekw{\Key{STARTHDIAGONAL}} to choose plus combination
  of degenerate   diagonal elements ({\bf STRONGLY RECOMMENDED} for
  calculation of singlet states  with \quotekw{\Key{DETERMINANTS}}).

\item[\Key{SELECT}]
  \kw{READ (LUINP,*) ICONF} \\
  Select CSF (or determinant if \quotekw{\Key{DETERMINANTS}}) no.
  ICONF as start configuration.\index{configuration!start}

\item[\Key{STARTHDIAGONAL}]
  Select configurations corresponding to the lowest diagonal elements in
  the configuration part of the Hessian (this is the default option).

\item[\Key{STARTOLDCI}]
  Start from old CI-vector stored saved on the \verb|SIRIUS.RST| file.

%\ifabacus
%\item[.ABACUS]
%  Geometry walk, use CI vector and "GEOSAVE" information saved by
%  ABACUS at previous geometry.  The "GEOSAVE" information is used
%  to decide as early as possible in the wave function optimization
%  if the step should be rejected, thus saving CPU time if the step
%  is rejected.
%\fi

\end{description}

\pagebreak[3]
\subsection{\label{ref-wavinp}\Sec{CONFIGURATION INPUT}}

{\bf Purpose:}

To specify the configuration part in MCSCF and CI calculations.

\begin{description}
\item[\Key{CAS SPACE}]
  \kw{READ (LUINP,*) (NASH(I),I=1,NSYM)} \\
  CAS calculation: Active orbitals\index{active orbital} in each symmetry.

\item[\Key{ELECTRONS}]
  Required.\\
  \kw{READ (LUINP,*) NACTEL} \\
  Number of active electrons\index{active electrons} (the number of
  electrons to be distributed in the active orbitals).
  The total number of electrons is this number
  plus two times the total number of inactive orbitals.

\item[\Key{INACTIVE ORBITALS}]
  Required.\\
  \kw{READ (LUINP,*) (NISH(I),I=1,NSYM)} \\
  Number of inactive orbitals\index{inactive orbital} each symmetry.

\item[\Key{RAS1 ELECTRONS}]
   \kw{READ (LUINP,*) NEL1MN,NEL1MX} \\
   Minimum and maximum number of RAS1 electrons; this can alternatively
   be specified with \quotekw{\Key{RAS1 HOLES}}

\item[\Key{RAS1 HOLES}]
  \kw{READ (LUINP,*) NHL1MN,NHL1MX} \\
  Minimum and maximum number of holes\index{electron hole} in RAS1; alternative
  to \quotekw{\Key{RAS1 ELECTRONS}}

\item[\Key{RAS1 SPACE}]
   \kw{READ (LUINP,*) (NAS1(I),I=1,NSYM)} \\
   RAS calculation: RAS1 orbital space\index{RAS1 orbital space}

\item[\Key{RAS2 SPACE}]
   \kw{READ (LUINP,*) (NAS2(I),I=1,NSYM)} \\
   RAS calculation: RAS2 orbital space\index{RAS2 orbital space}

\item[\Key{RAS3 ELECTRONS}]
   \kw{READ (LUINP,*) NEL3MN, NEL3MX} \\
   Minimum and maximum number of RAS3 electrons

\item[\Key{RAS3 SPACE}]
   \kw{READ (LUINP,*) (NAS3(I),I=1,NSYM)} \\
   RAS calculation: RAS3 orbital space\index{RAS3 orbital space}

\item[\Key{SPIN MULTIPLICITY}]
  Required for MCSCF and CI wave
  functions\index{MCSCF}\index{CI}\index{Configuration Interaction}.\\
  \kw{READ (LUINP,*) ISPIN}\\
  For CSF basis: state spin multiplicity\index{spin multiplicity} = $2S + 1$,
  where $S$ is the spin quantum number. \\
  For determinant basis this option determines the minimum spin
  multiplicity.  The $M_S$ value is determined as (ISPIN-1)/2.

\item[\Key{SYMMETRY}]
  Required for MCSCF and CI wave functions.\\
  \kw{READ (LUINP,*) LSYM} \\
  Spatial symmetry\index{symmetry} of CI and/or MCSCF wave function

\end{description}

\noindent{\bf Comments:}

\noindent SYMMETRY   Specifies total spatial symmetry of the wave
function in $D_{2h}$ symmetry or one of its subgroups: $C_{2v}$, $C_2h$,
$D_2$, $C_s$, $C_i$, $C_2$, $C_1$. The  symmetry number of wave
function follows MOLECULE output ordering of symmetries ($D_{2h}$
subgroup irreps).

\noindent
CAS and RAS\index{RASSCF}\index{CASSCF}\index{MCSCF} are exclusive and
both cannot be specified in the same
MCSCF or CI\index{MCSCF}\index{CI}\index{Configuration Interaction}
calculation. One of them {\em must} be specified.

\pagebreak[3]
\subsection{\label{ref-dftinp}\Sec{DFT INPUT}}

{\bf Purpose:}

To specify the parameters of the DFT integration and the optional use of empirical corrections.

\begin{description}

\item[\Key{DFTAC}]
  \kw{READ (LUINP,*) RTYPE} \\
  \kw{READ (LUINP,*) CTYPE} \\
  \kw{READ (LUINP,*) DFTIPTA, DFTIPTB, DFTBR1, DFTBR2} \\
  Switches on the asymptotic correction of the exchange correlation potential. This correction is a pointwise manipulation of the 
  exchange--correlation potential. This implies activation of the .DFTVXC keyword in the SCF stage. RTYPE defines the
  potential to be used to replace the asymptotic GGA potential, possible options are MULTPOLE (a simple multipole model)
  and LB94 (the potential from the LB94 model potential~\cite{dft:lb94}). CTYPE defines how the potential of the parent
  functional is connected to the asymptotic model, possible options are LINEAR (as used in the Tozer-Handy correction~\cite{dft:th}), 
  TANH (a modified connection by Tozer~\cite{dft:tanh}, which removes discontinuities associated with linear interpolation) and 
  GRAC (the gradient-regulated asymptotic correction of Gr\"uning \emph{et. al.}~\cite{dft:grac}). 
  
  Four numerical input parameters are then input
  the first two are the $\alpha$ and $\beta$ ionization potentials (either calculated or experimental). If GRAC is chosen for the 
  connection type then the last two value specify the parameters $\alpha$ and $\beta$ (see Ref.~\cite{dft:grac} for details). 
  Recommended values are 0.5 and 40.0. Otherwise the last two parameters specify multiples of the 
  Bragg Radii and are used to define the core, interpolation and asymptotic regions. For grid points within DFTBR1 Bragg Radii of 
  each atom the potential is unmodified, for points outside DFTBR2 Bragg Radii the potential is replaced with its asymptotic model.
  In between the interpolation model is used. Recommended values in this case are 3.5 and 4.7. Care should be taken when 
  choosing alternative values for the final two parameters in each scheme, inappropriate values can make SCF convergence difficult.


\item[\Key{DFTD2}]
  switches on Grimmes DFT-D2 empirical dispersion correction~\cite{dft:dftd2}. The code will attempt to assign the correct functional dependent 
  parameters based on the chosen DFT functional. Analytic gradient contributions are available.
 
\item[\Key{D2PAR}]
  \kw{READ (LUINP,*) D2\_s6\_inp, D2\_alp\_inp, D2\_rs6\_inp} \\
  using this keyword user input values of the $s_6$, $\alpha$ and $s_{r,6}$ DFT-D2 parameters may be specified. If supplied these values override
  any values defined within the code.

\item[\Key{DFTD3}]
  switches on Grimmes DFT-D3 empirical dispersion correction~\cite{dft:dftd3}. The code will attempt to assign the correct functional dependent 
  parameters based on the chosen DFT functional. Analytic gradient contributions are available.
  
\item[\Key{DFD3BJ}]
  switches on Grimmes DFT-D3 empirical dispersion correction with Becke-Johnson damping~\cite{dft:dftd3bj}. 
  The code will attempt to assign the correct functional dependent parameters based on the chosen DFT functional. 
  Analytic gradient contributions are available. This is the presently recommended version.
  
\item[\Key{3BODY}]
  keyword for adding 3-body terms to the DFT-D3 dispersion energy. Note that gradients are not implemented for these corrections 

\item[\Key{D3PAR}]
  \kw{READ (LUINP,*) D3\_s6\_inp, D3\_alp\_inp, D3\_rs6\_inp, D3\_rs18\_inp, D3\_s18\_inp} \\
  keyword for specifying the $s_6$, $\alpha$, $s_{r,6}$, $s_{r,8}$ and $s_8$ parameters of the DFT-D3 methods. Note, take care to 
  match the parameter values to the correct version of the DFT-D3 correction.

\item[\Key{DFTELS}]
  \kw{READ (LUINP,*) DFTELS} \\
  safety threshold -- stop if the charge integration gives too large
  error.

\item[\Key{DFTTHR}]
  \kw{READ (LUINP,*) DFTHRO, DFTHRI} \\
  Thresholds determining accuracy of the numerical integration. The
  first number determines the density threshold (contributions to a
  property from places where the density is below the threshold will
  be skipped) and the second one -- orbital threshold (orbitals are
  assumed to be exactly 0 if they are below the threshold). The
  default value for DFTHR0 is $1.0D-9$ and for DFTRHI is $1.0D-13$.

\item[\Key{DFTVXC}]
  keyword to specify explicit construction of the exchange--correlation 
  potential for GGA forms. This is automatically invoked when .DFTAC is selected 
  and not recommended for use otherwise.

\item[\Key{RADINT}]
  \kw{READ (LUINP,*) RADINT} \\

  Determines the quality of the radial part of the grid and
  corresponds to the upper limit of the error in case of an
  integration on an atom. Default value is $1.0D-13$.

\item[\Key{ANGINT}]
  \kw{READ (LUINP,*) ANGINT} \\
  Determines the quality of the angular Lebedev grid -- the angular
  integration of spherical harmonics will be exact up to the specified
  order. Default value is 35.
\item[\Key{GRID TYPE}]
  \kw{READ (LUINP,*) LINE} \\
  Allows specification of different partitioning methods and radial
  schemes. \verb|BECKE| is Becke partitioning scheme with correction
  for atomic sizes using Bragg sizes, \verb|BECKEORIG| is the same
  Becke partitioning scheme but without correction. \verb|SSF| is a
  partitioning scheme for large molecules designed to reduce the grid
  generation time. \verb|LMG| select LMG radial scheme adjusted to
  currently used basis set. Gauss-Chebychev radial scheme of second
  order is provided for reference and can be selected by keyword
  \verb|GC2|.

% The default is \verb|BECKE LMG| which is optimal for an
%  overwhelming number of cases.
\item[\Key{COARSE}]
  Shortcut keyword for radial integration accuracy $10^{-11}$ and
  angular expansion order equal to $35$.
\item[\Key{NORMAL}]
  Shortcut keyword for radial integration accuracy $10^{-13}$ and
  angular expansion order equal to $35$.
\item[\Key{FINE}]
  Shortcut keyword for radial integration accuracy $10^{-13}$ and
  angular expansion order equal to $42$.
\item[\Key{ULTRAF}]
  Shortcut keyword for radial integration accuracy $10^{-15}$ and
  angular expansion order equal to $65$.

\end{description}

\subsection{\label{ref-dft}DFT functionals}

In general, functionals in \dalton\ can be divided into two groups: 
generic exchange and correlation functionals 
and combined functionals. Combined functionals are a linear combination of
generic ones. There are a large number of combined functionals defined below, 
however the user can also create their own combined functionals with 
the \verb|Combine| keyword. 
A number of standalone functionals are also included within \dalton. 
In addition a number of double-hybrid functionals (energies only) are available,
which include a post-SCF second-order perturbation theory contribution. 

It should be noted that the input is not case sensitive, although the notation
employed in this manual makes use of case to emphasize exchange or correlation 
functional properties and reflect the original literature sources.


\subsubsection{Exchange Functionals}
\providecommand\exfn[1]{#1}
\begin{description}

\item[Slater] Dirac-Slater exchange functional
\cite{dft:hohenberg,dft:kohn,dft:slater}.\index{Slater}

\item[Becke] 1988 Becke exchange GGA correction \cite{dft:becke88}. 
  Note that the full Becke88 exchange functional is given as
  \exfn{Slater} + \exfn{Becke}.\index{Becke}

\item[mBecke] 1998 modified \exfn{Becke} exchange correction presented in reference
  \cite{dft:edf1} for use in the EDF1 functional. The $\beta$ value
  of 0.0042 in \exfn{Becke} is changed to 0.0035.\index{mBecke}\index{EDF1}

\item[B86] Becke 1986 exchange functional, a divergence free, semi-empirical 
    gradient-corrected exchange functional~\cite{dft:b86,dft:b86r}.\index{B86} This 
    functional corresponds to the B86R functional of the Molpro program.

\item[B86mx] B86 exchange functional modified with a gradient correction 
  for large density gradients~\cite{dft:b86mgc}.\index{B86mx}

\item[DBx] Double-Becke exchange functional defined in 1998 by 
  Gill et al.\cite{dft:edf1,dft:edf2} for use in the EDF1 functional. 
  The full DBx functional is defined as

  1.030952*\exfn{Slater} - 8.44793*\exfn{Becke} + 10.4017*\exfn{mBecke}
  \index{EDF1}\index{Double-Becke}\index{Becke}

\item[DK87x] DePristo and Kress' 1987 rational function GGA exchange functional 
  (equation 7) from Ref. \cite{dft:dk87}.\index{DK87} The full exchange
  functional is defined as \exfn{Slater} + \exfn{DK87x}.

%\item[FT97ax] Filatov and Thiel 1997 (FT97) exchange functional GGA correction, 
%  variant A.~\cite{dft:ft97}.\index{FT97}
%  The complete exchange functional is given by \exfn{Slater} + \exfn{FT97ax}.
%
%\item[FT97bx] Filatov and Thiel 1997 (FT97) exchange functional GGA correction,
%  variant B.~\cite{dft:ft97}.\index{FT97} In this variant, the $\beta$ parameter 
%  is a switching function dependent on the density gradient, $\nabla n_{\sigma}$ 
%  and only significantly varies from variant A calculated molecular properties 
%  if core electron effects are significant. This is the default exchange functional 
%  in the combined FT97 exchange-correlation functional.
%  The complete exchange functional is given by \exfn{Slater} + \exfn{FT97bx}.

\item[G96x] Gill's 1996 GGA correction exchange functional~\cite{dft:g96}.\index{G96}
  The complete exchange functional is given by \exfn{Slater} + \exfn{G96x}.

\item[LG93x] 1993 GGA exchange functional~\cite{dft:lg93,dft:g961lyp}.\index{LG93}
  The full LG93 exchange functional is given by \exfn{Slater} + \exfn{LG93x} 

\item[LRC95x] 1995 GGA exchange functional with correct asymptotic behavior~\cite{dft:lrc95}.\index{LRC95}
  The LRC95x exchange functional includes the Slater exchange (Eq 6 from original reference).

\item[KTx] Keal and Tozer's 2003 GGA exchange functional. Note that the gradient correction 
  pre-factor constant, $\gamma$, is not included in the KT exchange 
  definition, but rather in the KT1, KT2 and KT3 definitions. The full KT exchange is given by 
  \cite{dft:kt12}\index{KT},

  \exfn{Slater} + $\gamma$\exfn{KTx} ($\gamma$ is -0.006 for KT1,KT2 and -0.004 for KT3). 

\item[OPTX] Handy's 2001 exchange functional correction \cite{dft:optx}.\index{OPTX}
  The full OPTX exchange functional is given by 
  1.05151*\exfn{Slater} - 1.43169*\exfn{OPTX}.

\item[PBEx] Perdew, Burke and Ernzerhof 1996 exchange functional~\cite{dft:pbe}.\index{PBEx}

\item[revPBEx] Zhang and Wang's 1998 revised PBEx exchange functional, with $\kappa$ of 1.245
  \cite{dft:revpbe}.\index{revPBE}\index{PBE}

\item[RPBEx] Hammer, Hansen and N{\o}rskov's 1999 revised PBEx exchange functional 
  \cite{dft:revpbe}.\index{RPBE}\index{PBE}

\item[mPBEx] Adamo and Barone's 2002 modified PBEx exchange functional~\cite{dft:mpbe}.
  \index{mPBE}\index{PBE}

\item[PW86x] Perdew and Wang 1986 exchange functional (the PWGGA-I functional)
  ~\cite{dft:pw86x}.\index{PW86x}

\item[PW91x] Perdew and Wang 1991 exchange functional (the pwGGA-II functional)
  and includes Slater exchange \cite{dft:pw91}.\index{PW86x} This functional is 
  also given in a separate parameterization in Refs.~\cite{dft:g96,dft:mpw},
  which is labeled PW91x2, and is defined as 
  \exfn{PW91x} = \exfn{Slater} + \exfn{PW91x2}.

\item[mPW] Adamo and Barone's 1998 modified PW91x GGA correction exchange functional
  ~\cite{dft:pw91,dft:mpw}. The full exchange functional is given by
  \exfn{Slater} + \exfn{mPW}.\index{mPW}

\end{description}

\subsubsection{Correlation Functionals}
\providecommand\corfn[1]{#1}
\begin{description}

\item[VWN3] correlation functional of Vosko, Wilk and Nusair, 1980 (equation III) 
  \cite{dft:vwn}. This is the form used in the Gaussian program.\index{VWN3}

\item[VWN5] correlation functional of Vosko, Wilk and Nusair, 1980 (equation
  V -- the recommended one). The VWN keyword is a synonym for VWN5~\cite{dft:vwn}.\index{VWN5}

%\item[FT97c] Filatov and Thiel 1997 (FT97) correlation functional
%  \cite{dft:ft97}.\index{FT97}

\item[LYP] correlation functional by Lee, Yang and Parr, 1988
  \cite{dft:lyp1,dft:lyp2}.\index{LYP}

\item[LYPr] 1998 modified \corfn{LYP} functional, which is the re-parameterized EDF1 version 
  with modified parameters (0.055, 0.158, 0.25, 0.3505)
  \cite{dft:lyp1,dft:lyp2,dft:edf1}.\index{LYPr}\index{EDF1}

\item[P86c] non-local part of the correlation functional of the Perdew 1986 correlation functional
  \cite{dft:p86}. PZ81 (1981 Perdew local) is usually used for the local part of the
  functional, with a total corelation functional of \index{P86}\index{PZ81}
  \corfn{P86c} + \corfn{PZ81}.

\item[PBEc] Perdew, Burke and Ernzerhof 1996 correlation functional, 
  defined as PW91c local and PBEc non-local correlation~\cite{dft:pbe}.\index{PBEc}

\item[PW91c] 1991 correlation functional of Perdew and Wang (the pwGGA-II functional)
  ~\cite{dft:pw91}.\index{PW91} This functional includes both the PW91c non-local and 
  PW91c local (ie PW92c) contributions. The non-local PW91c contribution may be determined
  as \corfn{PW91c} - \corfn{PW92c}.

\item[PW92c] local correlation functional of Perdew and Wang, 1992~\cite{dft:pw91,dft:pw92}.\index{PW91}
  This functional is the local contribution to the PW91c correlation functional.

\item[PW92ac] gradient correction to the PW91c correlation functional of Perdew and Wang,
  equation 16 from Ref.~\cite{dft:pw91,dft:pw92}.\index{PW91} The PWGGA-IIA functional
  as defined in the original reference is \corfn{PW91c} + \corfn{PW92ac}. 

\item[PZ81] local correlation functional of Perdew and Zunger, 1981~\cite{dft:pz81}.\index{PZ81}

\item[Wigner] original 1938 spin-polarized correlation functional~\cite{dft:wigner}.\index{Wigner}

\item[WL90c] Wilson and Levy's 1990 non-local spin-dependent correlation functional 
  (equation 15 from Ref.~\cite{dft:wl90}).\index{WL90}

\end{description}

\subsubsection{Standalone Functionals}
\providecommand\onefn[1]{#1}
\begin{description}

\item[LB94] asymptotically correct functional of Leeuwen and
  Baerends 1994~\cite{dft:lb94}. This functional improves description of the
  asymptotic density on the expense of core and inner valence.\index{LB94}

\item[B97] Becke 1997 functional~\cite{dft:b97}.\index{B97}

\item[B97-1] Hamprecht et al.'s 1998 re-parameterization of the 
  B97 functional~\cite{dft:b97-1}.\index{B97}\index{B97-1}

\item[B97-2] Modification of B97 functional in 2001 by Wilson, Bradley and Tozer 
  \cite{dft:b97-2}.\index{B97}\index{B97-2}

\item[B97-D] Grimme's re-parameterization of the B97-1 functional for use with 
empirical dispersion correction~\cite{dft:b97-d}.\index{B97}\index{B97-D}

\item[B97-K] Boese and Martin 2004 re-parameterization of the 
  B97-1 functional for kinetics~\cite{dft:b97-1}.\index{B97}\index{B97-K}

\item[HCTH] is a synonym for the HCTH407 functional (detailed below).
  \cite{dft:hcth407}.\index{HCTH}\index{HCTH407}

\item[1-4] The ``quarter'' functional of Menconi, Wilson and Tozer 
  \cite{dft:14}.\index{1-4}\index{1-4}

\item[HCTH93] Original 1998 HCTH functional, parameterized on a set of 
  93 training systems~\cite{dft:hcth93}.\index{HCTH}\index{HCTH93}

\item[HCTH120] The HCTH functional parameterized on a set of 120 training systems 
  in 2000~\cite{dft:hcth120}.\index{HCTH}\index{HCTH120}

\item[HCTH147] The HCTH functional parameterized on a set of 147 training systems 
  in 2000~\cite{dft:hcth120}.\index{HCTH}\index{HCTH147}

\item[HCTH407] The HCTH functional parameterized on a set of 407 training systems 
  in 2001~\cite{dft:hcth407}.\index{HCTH}\index{HCTH407}

\item[HCTH407p] The HCTH407 functional re-parameterized in 2003 on a set of 407 
  training systems and ammonia dimer to incorporate hydrogen bonding 
  \cite{dft:hcth407p}.\index{HCTH}\index{HCTH407}\index{HCTH407p}

\end{description}

\subsubsection{Combined functionals}
\providecommand\funexample[1]{\\{\tt #1 }}
\begin{description}

\item[Combine] is a universal keyword allowing users to manually
  construct arbitrary linear combinations of exchange and correlation 
  functionals from the list above. Even\index{Combine} fractional 
  Hartree--Fock exchange can be specified. This keyword is to be 
  followed by a string of functionals with associated weights. 
  The syntax is \verb|NAME=WEIGHT ...|. 
  As an example, B3LYP may be constructed as:
\begin{verbatim}
.DFT
 Combine HF=0.2 Slater=0.8 Becke=0.72 LYP=0.81 VWN=0.19
\end{verbatim}

The following GGA and hybrid functional aliases are defined within 
\dalton\ and provide further examples of the Combine keyword.

\item[SVWN5] is a sum of Slater functional and VWN (or VWN5) correlation
  functional. SVWN is a synonym for SVWN5. It is equivalent to
  \funexample{Combine Slater=1 VWN5=1}
  \index{SVWN}

\item[SVWN3] is a sum of the Slater exchange functional and VWN3 correlation
  functional. It is equivalent to the Gaussian program LSDA functional 
  and can alternatively be selected by following set of keywords
  \funexample{Combine Slater=1 VWN3=1}
  \index{SVWN3}

\item[LDA] A synonym for SVWN5 (or SVWN). \index{LDA}

\item[BVWN] is a sum of the \exfn{Slater} functional, \exfn{Becke} correction and 
  \corfn{VWN} correlation functional.  It is equivalent to 
  \funexample{Combine Slater=1 Becke=1 VWN=1}
  \index{BVWN}

\item[BLYP] is a sum of Slater functional, Becke88 correction and LYP
  correlation functional.  It is equivalent to 
  \funexample{Combine Slater=1 Becke=1 LYP=1}
  \index{BLYP}

\item[B3LYP] 3-parameter hybrid functional \cite{dft:b3lyp} equivalent to:
  \funexample{Combine HF=0.2 Slater=0.8 Becke=0.72 LYP=0.81 VWN=0.19}
  \index{B3LYP}

\item[B3LYPg] hybrid functional with VWN3 form used for
  correlation---this is the form used by the Gaussian quantum chemistry
  program. Keyword B3LYPGauss is a synonym for B3LYPg.\index{B3LYPG} 
  This functional can be explicitely set up by
  \funexample{Combine HF=0.2 Slater=0.8 Becke=0.72 LYP=0.81 VWN3=0.19}
  \index{B3LYP, Gaussian version}

\item[B1LYP] 1-parameter hybrid functional with 25\% exact exchange \cite{dft:b1lyp}. 
  Equivalent to: \funexample{Combine HF=0.25 Slater=0.75 Becke=0.75 LYP=1}
  \index{B3LYP}

\item[BP86] Becke88 exchange functional and Perdew86 correlation
  functional (with Perdew81 local correlation). The explicit form is:
  \funexample{Combine Slater=1 Becke=1 PZ81=1 P86c=1}
  \index{BP86}

\item[B3P86] variant of \verb|B3LYP| with VWN used for local
  correlation  and P86 for the nonlocal part.
  \funexample{Combine HF=0.2 Slater=0.8 Becke=0.72 P86c=0.81 VWN=1}
  \index{B3P86}

\item[B3P86g] variant of \verb|B3LYP| with VWN3 used for local
  correlation and P86 for the nonlocal part.
  This is the form used by the Gaussian quantum chemistry program.
  \funexample{Combine HF=0.2 Slater=0.8 Becke=0.72 P86c=0.81 VWN3=1}
  \index{B3P86}\index{B3P86, Gaussian version}

\item[BPW91] Becke88 exchange functional and PW91 correlation
  functional. The explicit form is:
  \funexample{Combine Slater=1 Becke=1 PW91c=1}
  \index{BPW91}

\item[B3PW91] 3-parameter Becke-PW91 functional, with PW91 correlation 
  functional. Note that PW91c includes PW92c local correlation, thus only
  excess PW92c local correlation is required (coefficient of 0.19).
  \funexample{Combine HF=0.2 Slater=0.8 Becke=0.72 PW91c=0.81 PW92c=0.19}
  \index{B3PW91}

\item[B1PW91] 1-parameter hybrid functional \cite{dft:b1lyp} equivalent to:
  \funexample{Combine HF=0.25 Slater=0.75 Becke=0.75 PW91c=1}
  \index{B1PW91}

\item[B86VWN] is a sum of \exfn{Slater} and \exfn{B86x} exchange functionals and 
  the \corfn{VWN} correlation functional. It is equivalent to 
  \funexample{Combine Slater=1 B86x=1 VWN=1}
  \index{B86VWN}

\item[B86LYP] is a sum of \exfn{Slater} and \exfn{B86x} exchange functionals and 
  the \corfn{LYP} correlation functional. It is equivalent to 
  \funexample{Combine Slater=1 B86x=1 LYP=1}
  \index{B86LYP}

\item[B86P86] is a sum of \exfn{Slater} and \exfn{B86x} exchange functionals and 
  the \corfn{P86c} correlation functional. It is equivalent to 
  \funexample{Combine Slater=1 B86x=1 P86c=1}
  \index{B86P86}

\item[B86PW91] is a sum of \exfn{Slater} and \exfn{B86x} exchange functionals and 
  the \corfn{PW91c} correlation functional.  It is equivalent to 
  \funexample{Combine Slater=1 B86x=1 PW91c=1}
  \index{B86PW91}

\item[BHandH] is an simple Half-and-half functional.
  \funexample{Combine HF=0.5 Slater=0.5 LYP=1}
  \index{BHandH}

\item[BHandHLYP] is another simple Half-and-half functional.
  \funexample{Combine HF=0.5 Slater=0.5 Becke=0.5 LYP=1}
  \index{BHandH}

\item[BW] is the sum of the Becke exchange and Wigner correlation
  functionals \cite{dft:wigner,dft:bw}.\index{BW}
  \funexample{Combine Slater=1 Becke=1 Wigner=1}

\item[CAMB3LYP] Coulomb Attenuated Method Functional of Yanai, Tew and
Handy \cite{dft:camb3lyp}. This functional accepts additional arguments
\verb|alpha|, \verb|beta| and \verb|mu| to modify the fraction of HF
exchange for short-range interactions, additional fraction of HF
exchange for long-range interaction and the interaction switching
factor $\mu$. This input can be specified as follows:
\begin{verbatim}
.DFT
 CAMB3LYP alpha=0.190 beta=0.460 mu=0.330
\end{verbatim}
\index{CAMB3LYP}


\item[rCAMB3LYP] Revised version of the CAMB3LYP Functional \cite{dft:rcamb3lyp} 
designed to give near piecewise linear behaviour of the energy vs. particle number. 
This functional accepts additional arguments \verb|alpha|, \verb|beta| and \verb|mu| 
with the same meanings and syntax as for the CAMB3LYP functional.\index{rCAMB3LYP}

\item[DBLYP] is a sum of the Double-Becke exchange functional and
  the LYP correlation functional 
  \cite{dft:becke88,dft:edf1,dft:lyp1,dft:lyp2}.\index{Double-Becke}
 \funexample{Combine Slater=1.030952 Becke=-8.44793 mBecke=10.4017 LYP=1}

\item[DBP86] is the sum of the Double-Becke exchange functional and
  the P86 correlation functional \cite{dft:becke88,dft:edf1,dft:p86}.\index{Double-Becke}
 \funexample{Combine Slater=1.030952 Becke=-8.44793 mBecke=10.4017 P86c=1 PZ81=1}

\item[DBPW91] is a sum of the Double-Becke exchange functional and
  the PW91 correlation functional \cite{dft:becke88,dft:edf1,dft:pw91}.\index{Double-Becke}
 \funexample{Combine Slater=1.030952 Becke=-8.44793 mBecke=10.4017 PW91c=1}

\item[EDF1] is a fitted functional of Adamson, Gill and Pople \cite{dft:edf1}.
  It is a linear combination of the Double-Becke exchange functional and the revised LYP
  functional LYPr.\index{EDF1}
 \funexample{Combine Slater=1.030952 Becke=-8.44793 mBecke=10.4017 LYPr=1}

\item[EDF2] is a linear combination of the Hartree-Fock exchange and the Double-Becke 
  exchange, Slater exchange, LYP correlation, revised LYPr correlation and VWN 
  correlation functionals \cite{dft:edf2}\index{EDF2}.
 \funexample{Combine HF=0.1695 Slater=0.2811 Becke=0.6227 mBecke=-0.0551 VWN=0.3029 LYP=0.5998 LYPr=-0.0053}

\item[G96VWN] is the sum of the G96 exchange functional and the VWN 
  correlation functional \cite{dft:g96}.
 \funexample{Combine Slater=1 G96x=1 VWN=1}

\item[G96LYP] is the sum of the G96 exchange functional and the LYP
  correlation functional \cite{dft:g96}.
 \funexample{Combine Slater=1 G96x=1 LYP=1}

\item[G96P86] is the sum of the G96 exchange functional and the P86
  correlation functional \cite{dft:g96}.
 \funexample{Combine Slater=1 G96x=1 P86c=1}

\item[G96PW91] is the sum of the G96 exchange functional and the PW91
  correlation functional \cite{dft:g96}.
 \funexample{Combine Slater=1 G96x=1 PW91c=1}

\item[G961LYP] is a 1-parameter B1LYP type functional with the exchange gradient 
  correction provided by the G96x functional \cite{dft:g961lyp}.
 \funexample{Combine HF=0.25 Slater=0.75 G96x=0.75 LYP=1}

\item[KMLYP] Kang and Musgrave 2-parameter hybrid functional with a mixture of
  Slater and Hartree--Fock exchange and VWN and LYP correlation functionals.
  \cite{dft:kmlyp}.
 \funexample{Combine HF=0.557 Slater=0.443 VWN=0.552 LYP=0.448}

\item[KT1] Slater-VWN5 functional with the KT GGA exchange correction
  \cite{dft:kt12,dft:kt12a}.\index{KT1}
 \funexample{Combine Slater=1 VWN=1 KT=-0.006}

\item[KT2] differs from KT1 only in that the weights of the Slater and
  VWN5 functionals are from an empirical fit (not equal to 1.0)
  \cite{dft:kt12,dft:kt12a}.\index{KT2}
  \funexample{Combine Slater=1.07173 VWN=0.576727 KT=-0.006}

\item[KT3] a hybrid functional of Slater, OPTX and KT exchange with the
  LYP correlation functional \cite{dft:kt3}. The explicit form is
  \funexample{Combine Slater=1.092 KT=-0.004 LYP=0.864409 OPTX=-0.925452}
  \index{KT3}

\item[LG1LYP] is a 1-parameter B1LYP type functional with the exchange gradient
  correction provided by the LG93x functional \cite{dft:g961lyp}.
 \funexample{Combine HF=0.25 Slater=0.75 LG93x=0.75 LYP=1}

\item[mPWVWN] is the combination of mPW exchange and VWN correlation functionals 
   \cite{dft:mpw,dft:vwn}.\index{mPWVWN}
  \funexample{Combine Slater=1 mPW=1 VWN=1}.

\item[mPWLYP] is the combination of mPW exchange and LYP correlation functionals 
   \cite{dft:mpw,dft:vwn}.\index{mPWLYP}
  \funexample{Combine Slater=1 mPW=1 LYP=1}.

\item[mPWP86] is the combination of mPW exchange and P86 correlation functionals 
   \cite{dft:mpw,dft:vwn}.\index{mPWP86}
  \funexample{Combine Slater=1 mPW=1 P86c=1 PZ81=1}.

\item[mPWPW91] is the combination of mPW exchange and PW91 correlation functionals 
   \cite{dft:mpw,dft:pw91}.\index{mPWPW91}
  \funexample{Combine Slater=1 mPW=1 PW91c=1}.

\item[mPW3PW91] is a 3-parameter combination of mPW exchange and PW91 correlation
  functionals, with the PW91 (PW92c) local correlation~\cite{dft:mpw}.\index{mPW3PW91}
  \funexample{Combine HF=0.2 Slater=0.8 mPW=0.72 PW91c=0.81 PW92c=0.19}.

\item[mPW1PW91] is a 1-parameter combination mPW exchange and PW91 correlation
  functionals with 25\% Hartree--Fock exchange \cite{dft:mpw}.\index{mPW1PW91}
  \funexample{Combine HF=0.25 Slater=0.75 mPW=0.75 PW91c=1}.

\item[mPW1K] optimizes mPW1PW91 for kinetics of H abstractions, with 42.8\% Hartree--Fock
  exchange \cite{dft:mpw1k}.\index{mPW1PW91}
  \funexample{Combine HF=0.428 Slater=0.572 mPW=0.572 PW91c=1}.

\item[mPW1N] optimizes mPW1PW91 for kinetics of H abstractions, with 40.6\% Hartree--Fock
  exchange \cite{dft:mpw1n}.\index{mPW1N}
  \funexample{Combine HF=0.406 Slater=0.594 mPW=0.594 PW91c=1}.

\item[mPW1S] optimizes mPW1PW91 for kinetics of H abstractions, with 6\% Hartree--Fock
  exchange \cite{dft:mpw1s}.\index{mPW1S}
  \funexample{Combine HF=0.06 Slater=0.94 mPW=0.94 PW91c=1}.

\item[OLYP] is the sum of the OPTX exchange functional with the
  LYP correlation functional \cite{dft:optx,dft:lyp1,dft:lyp2}.
  \funexample{Combine Slater=1.05151 OPTX=-1.43169 LYP=1}
  \index{OLYP}

\item[OP86] is the sum of the OPTX exchange functional with the
  P86 correlation functional \cite{dft:optx,dft:p86}.
  \funexample{Combine Slater=1.05151 OPTX=-1.43169 P86c=1 PZ81=1}
  \index{OP86}

\item[OPW91] is the sum of the OPTX exchange functional with the
  PW91 correlation functional \cite{dft:optx,dft:pw91}.
  \funexample{Combine Slater=1.05151 OPTX=-1.43169 PW91c=1}
  \index{OPW91}

\item[PBE0] a hybrid functional of Perdew, Burke and Ernzerhof with
  0.25 weight of exact exchange, 0.75 of \verb|PBEx| exchange functional and
  the \verb|PBEc| correlation functional \cite{dft:pbe0}.
  Alternative aliases are PBE1PBE or PBE0PBE.\index{PBE0}
  \funexample{Combine HF=0.25 PBEx=0.75 PBEc=1}

\item[PBE] same as above but with exchange estimated exclusively by
  \exfn{PBEx} functional \cite{dft:pbe}.\index{PBE} Alias of PBEPBE. 
  This is the form used by CADPAC and NWChem quantum chemistry programs.
  \funexample{Combine PBEx=1 PBEc=1}\\
  Note that the Molpro quantum chemistry program uses the \corfn{PW91c}
  non-local correlation functional instead of \corfn{PBEc}, which is 
  equivalent to the following:
  \funexample{Combine PBEx=1 PW91c=1}.

\item[RPBE] is a revised PBE functional that employs the 
  \exfn{RPBEx} exchange functional.
  \funexample{Combine RPBEx=1 PBEc=1}

\item[revPBE] is a revised PBE functional that employs the 
  \exfn{revPBEx} exchange functional.
  \funexample{Combine revPBEx=1 PBEc=1}

\item[mPBE] is a revised PBE functional that employs the 
  \exfn{mPBEx} exchange functional.
  \funexample{Combine mPBEx=1 PBEc=1}

\item[PW91VWN] is the combination of PW91 exchange and VWN correlation functionals 
   \cite{dft:pw91,dft:vwn}.\index{PW91}
  \funexample{Combine PW91x=1 VWN=1}.

\item[PW91LYP] is the combination of PW91 exchange and LYP correlation functionals 
   \cite{dft:pw91,dft:lyp1,dft:lyp2}.\index{PW91}
  \funexample{Combine PW91x=1 LYP=1}.

\item[PW91P86] is the combination of PW91 exchange and P86 (with Perdew 1981 local) 
  correlation functionals \cite{dft:pw91,dft:pw86,dft:pz81}.
  \funexample{Combine PW91x=1 P86c=1 PZ81=1}.

\item[PW91PW91] is the combination of PW91 exchange and PW91 correlation functionals. 
  Equivalent to PW91 keyword \cite{dft:pw91}.
  \funexample{Combine PW91x=1 PW91c=1}.

\item[XLYP] is a linear combination of \exfn{Slater}, \exfn{Becke} and \exfn{PW91x}
  exchange and \corfn{LYP} correlation functionals \cite{dft:xlyp,dft:x3lyp}.\index{XLYP}
  \funexample{Combine Slater=1 Becke=0.722 PW91x=0.347 LYP=1}.

\item[X3LYP] is a linear combination of Hartree--Fock, \exfn{Slater}, \exfn{Becke} 
  and \exfn{PW91x} exchange and \corfn{VWN} and \corfn{LYP} correlation functionals 
  \cite{dft:xlyp,dft:x3lyp}.\index{X3LYP}
  \funexample{Combine HF=0.218 Slater=0.782 Becke=0.542 PW91x=0.167 VWN=0.129 LYP=0.871}

\end{description}


Note that combinations of local and non-local correlation functionals
can also be generated with the Combine keyword. For example,
\verb|Combine P86c=1 PZ81=1| combines the PZ81 local and P86c non-local 
correlation functional, whereas \verb|Combine VWN=1 P86c=1| 
combines the VWN local and P86 non-local correlation functionals.


Linear combinations of all exchange and correlation functionals listed above
are possible with the \verb|Combine| keyword.

\subsubsection{Double-hybrid functionals}
\begin{description}
\item[B2PLYP] is the double hybrid of Ref.~\cite{dft:b2plyp}

\item[B2TPLYP] is a modification of the B2PLYP functional for thermodynamics~\cite{dft:b2tplyp}

\item[mPW2PLYP] a double hybrid using an alternative GGA exchange contribution and tested on the G3/05 benchmark dataset~\cite{dft:mpw2plyp}

\item[B2GPLYP] is a modification of the B2PLYP functional for general purpose calculations~\cite{dft:b2tplyp}

\item[B2PIPLYP] is a form related to B2PLYP but designed to give better performance for sterically crowed or stacked aromatic ring systems~\cite{dft:b2piplyp}

\item[PBE0DH] a theoretically derived double-hybrid parameterization~\cite{dft:pbe0dh}

\end{description}

Note that at present double-hybrid functionals are implemented for energies only, analytic gradient contributions are not implemented.

\pagebreak[3]
\subsection{\label{ref-haminp}\Sec{HAMILTONIAN}}

{\bf Purpose:}

Add extra terms to the Hamiltonian (for finite field\index{finite field} calculations).

\begin{description}
\item[\Key{FIELD TERM}]
  Default = no fields. \\
  \verb"READ (LUINP,  *  ) EFIELD(NFIELD)" \\
  \verb"READ (LUINP,'(A)') LFIELD(NFIELD)" \\
  Enter field strength (in atomic units) and property label on separate lines
  where label is a \molecule-style property label on file \verb|AOPROPER|
  produced by the property module, see Chapter~\ref{ch:hermit}.
  The calculation of the necessary property integral(s) must be requested
  in the \quotekw{**INTEGRALS} input module. \\
  NOTE: This keyword may be repeated several times for adding more than
  one finite field (max 20 fields).

\item[\Key{PRINT}]
  Default = 0.\\
  \verb"READ (LUINP,*) IPRH1" \\
  If greater than zero:
  print the one-electron Hamiltonian matrix, including
  specified field-dependent terms, in AO basis.
\end{description}

\noindent{\bf Comments:}

Up to \mxfelt simultaneous fields may be specified by repeating the
\quotekw{\Key{FIELD TERM}} keyword.
The field integrals are read from file \verb|AOPROPER| with the specified label.

\pagebreak[3]
\subsection{\label{ref-mp2inp}\Sec{MP2 INPUT}}

{\bf Purpose:}

\index{MP2}\index{M{\o}ller-Plesset!second-order}
To direct MP2 calculation. Note that MP2 energies as well as
properties also are available through the coupled cluster module, see
Chapter~\ref{ch:CC}.
For open--shell SCF, the singly occupied orbitals are frozen in the MP2 section.

\begin{description}

\item[\Key{MP2 FROZEN}]
  Default = no frozen orbitals\\
  \kw{READ (LUINP,*) (NFRMP2(I),I=1,NSYM)} \\
  Occupied SCF orbitals frozen in MP2 calculation.

\item[\Key{PRINT}]
  \kw{READ (LUINP,*) IPRMP2} \\
  Print level for MP2 calculation.

\item[\Key{SAVE WF1}]
  Save first order wave function on SIRIFC.
  Default is only to save first order wave function if MP2 is the last wave function level
  \emph{and} at least one of the PROPERTIES (ABACUS) or RESPONSE modules have been
  requested (presumably then for a SOPPA calculation).

\end{description}

\subsubsection{Modifications of MP2 model.}

\begin{description}

\item[\Key{SCSMP2}]
  Grimme's spin-component scaled MP2 ($p_S = 1.2$, $p_T = 1/3$)

\item[\Key{SOSMP2}]
  Head-Gordon's scaled opposite spin MP2 ($p_S = 1.3$, $p_T = 0$)

\item[\Key{MP2 SCALED}]
  \kw{READ (LUINP,*) p_S, p_T} \\
  Your own scaling factors for a scaled MP2 model.

\item[\Key{LEVELSHIFT}]
  \kw{READ (LUINP,*) MP2_LSHIFT} \\
  Level shift of MP2 denominators.
\end{description}

\noindent{\bf Comments:}

The MP2 module expects canonical Hartree--Fock orbitals. The MP2 module will
check the orbitals and it exits if the Fock matrix has off-diagonal non-negligible
elements.
If starting from saved canonical Hartree--Fock orbitals from a previous calculations,
although no Hartree--Fock calculation will be done
the number of occupied Hartree--Fock orbitals in each symmetry must anyway be
specified with the \quotekw{.DOUBLY OCCUPIED} under \quotekw{*SCF INPUT}.

The MP2 calculation will produce the MP2 energy and the natural orbitals
for the density matrix through second order.  The primary purpose of
this option is to generate good starting orbitals for CI or MCSCF wave
functions, but it
may of course also be used to obtain the MP2 energy, perhaps with frozen
core orbitals. {\em For valence MCSCF calculations it is recommended that the
\quotekw{\Key{MP2 FROZEN}} option is used in order to obtain the appropriate
correlating orbitals\index{correlating orbitals}\index{MCSCF} as start
for an MCSCF calculation.\/}  As the commonly
used basis sets do not contain correlating orbitals for the core
orbitals and as the core correlation energy therefore becomes arbitrary,
the \quotekw{\Key{MP2 FROZEN}} option can also be of benefit in MP2 energy
calculations.

\pagebreak[3]
\subsection{\label{ref-nevpt2inp}\Sec{NEVPT2 INPUT}}

{\bf Purpose:}

\index{NEVPT2}\index{multireference PT!second-order}
Calculation of the second order correction to the energy for a
CAS--SCF or CAS--CI zero order wavefunction.
The user is referred to Chapter~\ref{ch:nevpt2} on
page~\pageref{ch:nevpt2}  for a brief
introduction to the $n$--electron valence state second order
perturbation theory (NEVPT2).

\begin{description}
\item[\Key{THRESH}]
 Default = 0.0D0\\
  \kw{READ (LUINP,*) THRNEVPT} \\
  Threshold to discard small coefficients in the CAS wavefunction


\item[\Key{FROZEN}]
  Default = no frozen orbitals\\
  \kw{READ (LUINP,*) (NFRNEVPT2(I),I=1,NSYM)} \\
  Orbitals frozen in NEVPT2 calculation

\item[\Key{STATE}]
 No default provided\\
\kw{READ (LUINP,*) ISTNEVCI} \\
Root number in a CASCI calculation. This keyword is unnecessary
(ignored) in the CASSCF case.
\end{description}


\noindent{\bf Comments:}


%The present version of the NEVPT2 module requires the
%\quotekw{\Key{DETERMINANTS}} option  to be set.

The use of canonical orbitals for the core and virtual orbitals is
strongly recommended since this choice guarantees compliance of the
results with a totally invariant form of NEVPT2 (see page~\pageref{ch:nevpt2}).

At present the NEVPT2 module can deal with active spaces of dimension
not higher than 14.

\pagebreak[3]
\subsection{\label{ref-optinp}\Sec{OPTIMIZATION}}

{\bf Purpose:}

To change defaults for optimization of an MCSCF\index{MCSCF} wave function.
Some of the options also affect a QC-HF optimization.

\begin{description}
\item[\Key{ABSORPTION}]
  \kw{READ (LUINP,'(A8)') RWORD} \\
  RWORD = ` LEVEL 1', ` LEVEL 2', or ` LEVEL 3'\\
  Orbital absorption\index{orbital absorption} in MCSCF optimization
  at level 1, 2, or 3, as specified
  (normally level 3, see comments below).  This keyword may be repeated to
  specify more than one absorption level, the program will then begin with
  the lowest level requested and, when that level is converged,
  disable the lower level and shift to the next level.
% 940816-hjaaj: The following may not be true for RAS ????
% Absorption at several levels are only useful in
% first macro iteration, therefore the lower levels are disabled after
% convergence.

\item[\Key{ACTROT}]
  include specified active-active rotations
\begin{verbatim}
  READ (LUINP,*) NWOPT
  DO I = 1,NWOPT
    READ (LUINP,*) JWOP(1,I),JWOP(2,I)
  END DO
\end{verbatim}
  JWOP(1:2,I) denotes normal molecular orbital numbers (not the active
  orbital numbers).

\item[\Key{ALWAYS ABSORPTION}]
  Absorption\index{orbital absorption} in all MCSCF macro iterations
  (default is to disable absorption in
  local region or after \quotekw{\Key{MAXABS}} macro iterations, whichever comes first).
  Absorption is always disabled after Newton-Raphson algorithm has been used,
  and thus also when doing \quotekw{\Key{CORERELAX}},
  because absorption may cause variational collapse if the desired state is excited.

\item[\Key{CI PHP MATRIX}]
  Default : MAXPHP = 1 (Davidson's algorithm)\\
  \kw{READ (LUINP,*) MAXPHP} \\
  PHP is a subblock of the CI matrix which is calculated explicitly
  in order to obtain improved CI trial vectors compared to the
  straight Davidson\index{Davidson algorithm} algorithm.  The
  configurations corresponding to
  the lowest diagonal elements are selected, unless
  \quotekw{\Key{PHPRESIDUAL}} is specified.
  \kw{MAXPHP} is the maximum dimension of PHP, the actual dimension
  will be less if \kw{MAXPHP} will split degenerate configurations.

\item[\Key{COREHOLE}]
  \kw{READ (LUINP,*) JCHSYM,JCHORB} \\
  JCHSYM = symmetry of core orbital\\
  JCHORB = the orbital in symmetry JCHSYM with a single core hole\\
  Single core hole\index{core hole} MCSCF calculation. The calculation must be of RAS type
  with only the single core-hole orbital in RAS1, the state specified with
  \quotekw{\Key{STATE}} is optimized with the core-hole orbital
  frozen\index{frozen core hole}.
  The specified core hole orbital must be either inactive or
  the one RAS1 orbital, if it is inactive then it will switch places with
  the RAS1 orbital and it will not be possible to also
  specify \quotekw{\Key{REORDER}}. If explicit reordering is required you must reorder
  the core orbital yourself and let \kw{JCHORB} point to the one RAS1 orbital.
  Orbital absorption is activated at level 2. See comments below for more information.

\item[\Key{CORERELAX}]
  (ignored if \quotekw{\Key{COREHOLE}} isn't also specified)\\
  Optimize state with relaxed core orbital\index{relaxed core hole} (using Newton-Raphson algorithm,
  it is not necessary to explicitly specify \quotekw{\Key{NR ALWAYS}}).
  It is assumed that this calculation follows an optimization
  with frozen core orbital and that the orbital has already been
  moved to the RAS1 space ({\it i.e.\/}, the specific value of
  \quotekw{JCHORB} under \quotekw{\Key{COREHOLE}} is ignored). Any
  orbital absorption   will be ignored.

\item[\Key{DETERMINANTS}]
  Use determinant\index{determinants} basis instead of CSF basis (see comments).

\item[\Key{EXACTDIAGONAL}]
  Default for RAS calculations.\\
  Use the exact orbital Hessian\index{orbital Hessian} diagonal.

\item[\Key{FOCKDIAGONAL}]
  Default for CAS calculations.\\
  Use an approximate orbital Hessian diagonal which only uses Fock
  contributions.

\item[\Key{FOCKONLY}]
  Activate TRACI option (default : program decides).\\
  Modified TRACI option where all orbitals, also active orbitals, are
  transformed to Fock type orbitals in each iteration.

\item[\Key{FROZEN CORE ORBITALS}]
  \kw{READ (LUINP,*) (NFRO(I),I=1,NSYM)} \\
  Frozen orbitals : Number of inactive (doubly occupied) orbitals to be frozen
  in each symmetry (the first NFRO(I) in symmetry I) in MCSCF.\index{frozen orbitals!MCSCF}
  Active orbitals and specific inactive orbitals can be frozen with \quotekw{.FREEZE}
  under \Sec{ORBITAL INPUT}.
  Frozen orbitals in SCF are specified in the \Sec{SCF INPUT} input module.

\item[\Key{MAX CI}]
  \kw{READ (LUINP,*) MAXCIT} \\
  maximum number of CI iterations before MCSCF (default = 3).

\item[\Key{MAX MACRO ITERATIONS}]
  \kw{READ (LUINP,*) MAXMAC} \\
  maximum number of macro iterations in MCSCF optimization (default = 25).
\index{iteration number!MCSCF macro, max}

\item[\Key{MAX MICRO ITERATIONS}]
  \kw{READ (LUINP,*) MAXJT} \\
  maximum number of micro iterations per macro iteration in MCSCF optimization
  (default = 24).

\item[\Key{MAXABS}]
  \kw{READ (LUINP,*) MAXABS} \\
  maximum number of macro iterations with 
  absorption\index{orbital absorption} (default = 3).

\item[\Key{MAXAPM}]
  \kw{READ (LUINP,*) MAXAPM} \\
  maximum number orbital absorptions\index{orbital absorption} within
  a macro iteration
  (APM : Absorptions Per Macro iteration; default = 5)

\item[\Key{NATONLY}]
  Activate TRACI option (default : program decides).\\
  Modified TRACI option where the inactive and secondary orbitals are not
  touched (these two types of orbitals are already natural orbitals).

\item[\Key{NEO ALWAYS}]
  Always norm-extended optimization (never switch to New\-ton-Raph\-son).
  Note: \quotekw{\Key{NR ALWAYS}} and \quotekw{\Key{CORERELAX}}
  takes precedence over \quotekw{\Key{NEO ALWAYS}}.

\item[\Key{NO ABSORPTION}]
  Never orbital absorption\index{orbital absorption} (default settings removed)

\item[\Key{NO ACTIVE-ACTIVE ROTATIONS}]
  No active-active rotations in RAS optimization.

\item[\Key{NOTRACI}]
  Disable TRACI option (default : program decides).

\item[\Key{NR ALWAYS}]
  Always Newton-Raphson optimization (never NEO optimization).
  Note: \quotekw{\Key{NR ALWAYS}} takes precedence over
  \quotekw{\Key{NEO ALWAYS}}.

\item[\Key{OLSEN}]
  Use Jeppe Olsen's generalization of the Davidson
  algorithm\index{Davidson algorithm}.

\item[\Key{OPTIMAL ORBITAL TRIAL VECTORS}]
  Generate "optimal" orbital trial
  vectors~\cite{hjajpjhajcp87}.\index{optimal orbital trial vector} 

\item[\Key{ORB\_TRIAL VECTORS}]
  Use also orbital trial vectors as start vectors for auxiliary roots
  in each macro iteration (CI trial vectors are always generated).

\item[\Key{PHPRESIDUAL}]
  Select configurations for PHP matrix based on largest residual
  rather than lowest diagonal elements.

\item[\Key{SIMULTANEOUS ROOTS}]
  Default : NROOTS = ISTATE, LROOTS = NROOTS\\
  \kw{READ (LUINP,*) NROOTS, LROOTS} \\
  NROOTS = Number of simultaneous roots in NEO\\
  LROOTS = Number of simultaneous roots in NEO at start

\item[\Key{STATE}]
  Default = 1\\
  \kw{READ (LUINP,*) ISTATE} \\
  Index of MCSCF Hessian\index{MCSCF Hessian} at convergence (1 for
  lowest state, 2 for first
  excited state, etc. within the spatial symmetry\index{symmetry} and
  spin symmetry\index{spin symmetry}
  specified under \Sec{CONFIGURATION INPUT}).

\item[\Key{SYM CHECK}]
  Default: ICHECK = 2 when NROOTS $>$ 1, else ICHECK = -1.\\
  \kw{READ (LUINP,*) ICHECK} \\
  Check symmetry of the LROOTS start CI-vectors and remove those which
  have wrong symmetry ({\it e.g.\/} vectors of delta symmetry in a sigma
  symmetry calculation).
\begin{verbatim}
  ICHECK < 0  : No symmetry check.
  ICHECK = 1  : Remove those vectors which do not have the same
                symmetry as the ISTATE vector, reassign ISTATE.
  ICHECK = 2  : Remove those vectors which do not have the same
                symmetry as the lowest state vector before selecting
                the ISTATE vector.
  other values: check symmetry, do not remove any CI vectors.
\end{verbatim}
  The \quotekw{\Key{SIMULTANEOUS ROOTS}} input will automatically be
  updated if CI vectors are removed.

\item[\Key{THRCGR}]
  \kw{READ (LUINP,*) THRCGR} \\
  Threshold for print of CI gradient. Default is 0.1D0.

\item[\Key{THRESH}]
  Default = 1.0D-05\\
  \kw{READ (LUINP,*) THRMC} \\
  Convergence threshold for energy gradient in MCSCF optimization.
  The convergence of the energy will be approximately the square of this
  number.

\item[\Key{TRACI}]
  Activate TRACI option (default : program decides).\\
  Active orbitals are transformed to natural orbitals and the CI-vectors
  are counter-rotated such that the CI states do not change.  The
  inactive and secondary orbitals are transformed to Fock type orbitals
  (corresponding to canonical orbitals for closed shell Hartree--Fock).
  For RAS wave functions the active orbitals are only transformed
  within their own class (RAS1, RAS2, or RAS3) as the wave function is
  not invariant to orbital rotations between the classes.  For RAS, the
  orbitals are thus not true natural orbitals, the density matrix is
  only block diagonalized.  Use \quotekw{\Key{IPRCNO}} (see
  p.~\pageref{ref-priinp})   to control output from this
  transformation.

\end{description}


\noindent{\bf Comments:}

COREHOLE: Single core-hole\index{core hole} calculations are
performed as RAS calculations where the opened core orbital is in
the RAS1 space.  The RAS1 space must therefore contain one and
only one orbital when the COREHOLE option is used, and the
occupation must be restricted to be exactly one electron. The
orbital identified as the core orbital must be either inactive or
the one RAS1 orbital, if it is inactive it will switch places with
the one RAS1 orbital. The core orbital (now in RAS1) will be
frozen in the following optimization. After this calculation has
converged, the CORERELAX option may be added and the core orbital
will be relaxed\index{relaxed core hole}.  When CORERELAX is
specified it is assumed that the calculation was preceded by a
frozen core\index{frozen core hole} calculation, and that the
orbital has already been moved to the RAS1 space. Default
corresponds to the main peak, shake-up energies may be obtained by
specifying \quotekw{\Key{STATE}} larger than one. Absorption is
very beneficial in core hole calculations because of the large
orbital relaxation following the opening of the core hole.

ABSORPTION: Absorption\index{orbital absorption} level 1 includes occupied - occupied rotations
only (including active-active rotations); level 2 adds inactive -
secondary rotations and only active - secondary rotations are excluded
at this level; and finally level 3 includes all non-redundant rotation
for the frozen CI vector.  Levels 1 and 2 require the same integral
transformation (because the inactive - secondary rotations are
performed using the P-supermatrix integrals) and level 1 is therefore
usually not used. Level 3 is the normal and full level, but it can be
advantageous to activate level 2 together with level 3 if big
inactive-active or occupied-occupied rotations are expected.

ORB\_TRIAL: Orbital trial\index{orbital trial vector}\ vectors as
start vectors can be used for
excited states and other calculations with more than one simultaneous
roots.  The orbital start trial vectors are based on the eigenvectors of
the NEO matrix in the previous macro iterations.  However, they are
probably not cost-effective for multiconfiguration calculations where
optimal orbital trial\index{optimal orbital trial vector} vectors are
used and they are therefore not used
by default.

SYM CHECK: The symmetry check is performed on the matrix element
$\langle VEC1 \mid oper \mid VEC2\rangle$, where "oper" is
the CI-diagonal.
It is recommended and the default to use \quotekw{\Key{SYM CHECK}}
for excited states, including
CI vectors of undesired symmetries is a waste of CPU time.

DETERMINANTS: The kernels of the CI sigma routines and density matrix
routines are always performed in determinant\index{determinants}
basis.  However, this
keyword specifies that the external representation is Slater
determinants as well.  The default is that the external representation
is in CSF\index{CSF}\index{configuration state function} basis as
described in chapter 8 of MOTECC-90.  The external
CSF\index{CSF}\index{configuration state function} basis is
generally to be preferred to be sure that the converged
state(s) have pure and correct spin symmetry\index{spin symmetry}, and
to save disk space.
It is recommended to specify \quotekw{\Key{PLUS COMBINATIONS}} under
\quotekw{\Sec{CI VECTOR}} for
calculations on singlet states\index{singlet state} with
determinants\index{determinants},
in particular for
excited singlet\index{excited state} states which often have lower lying triplet states.


\pagebreak[3]
\subsection{\label{ref-orbinp}\Sec{ORBITAL INPUT}}

{\bf Purpose:}

To define an initial set of molecular orbitals\index{molecular orbital!initial set}
and to control the use of super symmetry\index{supersymmetry}, frozen
orbitals\index{frozen orbitals}, deletion of orbitals\index{delete orbitals},
reordering and punching of orbitals.

\begin{description}
\item[\Key{5D7F9G}]
  Delete unwanted components in Cartesian d, f, and g orbitals.
  (s in d; p in f; s and d in g). By default, \her\ provides atomic
  integrals in spherical basis, and this option should therefore not
  be needed.

\item[\Key{AO DELETE}]
  \kw{READ (LUINP,*) THROVL } \\
  Delete MO's based on canonical orthonormalization using eigenvalues
  and eigenvectors of the AO overlap matrix.\index{linear dependence} \\
  THROVL: limit for basis
  set numerical linear dependence (eigenvectors with eigenvalue less
  than THROVL are excluded). Default is 1.0$\cdot$10$^{-6}$.

\item[\Key{CMOMAX}]
  \kw{READ (LUINP,*) CMAXMO} \\
  Abort calculation if the absolute value of any initial MO coefficient is
  greater than CMAXMO (default : CMAXMO = $10^3$).  Large MO coefficients
  can cause significant loss of accuracy in the two-electron integral
  transformation.

\item[\Key{DELETE}]
  \kw{READ (LUINP,*) (NDEL(I),I = 1,NSYM) } \\
  Delete orbitals\index{deleted orbitals}, {\it i.e.\/} number of molecular orbitals
  in symmetry \quotekw{I} is number of basis functions in symmetry \quotekw{I} minus
  \quotekw{NDEL(I)}. \\
  Only for use with \quotekw{.MOSTART} options \quotekw{FORM12} or \quotekw{FORM18}, 
  it cannot be used with \quotekw{H1DIAG}, \quotekw{EWMO}, or \quotekw{HUCKEL},
  and the other restart options as \quotekw{NEWORB} reads this information from file
  and this will overwrite what ever was specified here.

\item[\Key{FREEZE}]
  Default: no frozen orbitals.
\begin{verbatim}
  READ (LUINP,*) (NNOR(ISYM), ISYM = 1,NSYM)
  DO ISYM = 1,NSYM
    IF (NNOR(ISYM) .GT. 0) THEN
      READ (LUINP,*) (INOROT(I), I = 1,NNOR(ISYM))
      ...
    END IF
  END DO
\end{verbatim}
  where \kw{INOROT} = orbital numbers of the orbitals to be
          frozen\index{frozen orbitals!MCSCF and SCF} (not rotated)
          in symmetry \quotekw{ISYM} both in SCF and MCSCF
          after any reordering (counting from 1 in each symmetry).\\
  Must be specified after all options reducing the number of orbitals.
  Frozen occupied orbitals in SCF can only be specified in the \Sec{SCF INPUT} input module
  and frozen inactive orbitals in MCSCF can only be specified in the \Sec{OPTIMIZATION}
  input module.

\item[\Key{GRAM-SCHMIDT ORTHONORMALIZATION}]
  Default.\\
  Gram--Schmidt orthonormalization\index{orthonormalization!Gram--Schmidt} of input orbitals.

\item[\Key{LOCALIZATION}]
  \kw{READ (LUINP,*) REWORD} \\
  Specify that the doubly occupied (inactive) orbitals should be localized after SCF 
  or MCSCF is converged.
  Two options for localization of the orbitals are currently available:
  \begin{description}
  \item[{\tt BOYLOC\ }] Use the Boys localization scheme~\cite{Boyloc}.
  %\item[{\tt PIPLOC\ }] Use the Pipek-Mezey localization scheme~\cite{}.
  % aug 04: PIPLOC is not implemented yet.
  \item[{\tt SELECT\ }] Select a subset of the orbitals to be localized. The first
  line following this option contains the number orbitals to localize per symmetry,
  and the following lines contain which orbitals to localize within each symmetry,
  one line per symmetry with orbitals to localize.
  This option is typically used for localizing degenerate 
  core orbitals, leaving all other orbitals intact.
  \begin{verbatim}
         READ(LUCMD,*)(NBOYS(I),I=1,NSYM)
         DO I=1,NSYM
            IF (NBOYS(I).GT.0) THEN
               READ(LUCMD,*)(BOYSORB(J,I),J=1,NBOYS(I))
            END IF
         END DO
   \end{verbatim}
  \end{description}

\item[\Key{MOSTART}]
   Molecular orbital input\index{molecular orbital}\\
   \kw{READ (LUINP,'(1X,A6)') RWORD} \\
   where RWORD is one of the following:
   \begin{description}
   \item[{\tt FORM12\ }] Formatted input (6F12.8)  supplied after
        \Sec{*MOLORB} or \Sec{*NATORB} keyword. Use also \quotekw{.DELETE}
        if orbitals were deleted.
   \item[{\tt FORM18\ }] Formatted input (4F18.14) supplied after
        \Sec{*MOLORB} or \Sec{*NATORB} keyword. Use also \quotekw{.DELETE}
        if orbitals were deleted.
   \item[{\tt EWMO\ }] Start orbitals generated by projecting the EWMO
        H{\"u}ckel eigenvectors in a good generally contracted ANO basis set
        onto the present basis set.
        The EWMO model generally works better than the EHT model.
        Default initial guess for molecules in which all atoms have a nuclear charge
        less than or equal to 36.
        Note: EWMO/HUCKEL is not implemented yet if any element has a
        charge larger than 36).
        The start density will thus be close to one generated from atomic densities,
        but with molecular valence interaction in the EWMO model.
        This works a lot better than using a minimal basis set for EWMO.
   \item[{\tt HUCKEL\ }] Start orbitals generated by projecting the EHT
        H{\"u}ckel eigenvectors in a good generally contracted ANO basis set
        onto the present basis set.
        Note: EWMO/HUCKEL is not implemented yet if any element has a
        charge larger than 36.
        The start density will thus be close to one generated from atomic densities,
        but with molecular valence interaction in the H{\"u}ckel model.
        This works a lot better than using a minimal basis set for H{\"u}ckel.
   \item[{\tt H1DIAG\ }] Start orbitals that diagonalize
        one-electron Hamiltonian matrix (default
        for molecules containing elements with a nuclear larger than 36).
   \item[{\tt NEWORB\ }] Input from {\sir} restart file
                            (\verb|SIRIUS.RST| file) with label \quotekw{NEWORB  }
   \item[{\tt OLDORB\ }] Input from {\sir} restart file
                            (\verb|SIRIUS.RST| file) with label \quotekw{OLDORB  }
   \item[{\tt SIRIFC\ }] Input from {\sir} interface file ("\verb|SIRIFC|")
\end{description}

%Oct 2003 hjaaj: now NOSUPSYM always default,
% SUPSYM must be activated explicitly with .SUPSYM
%\item[\Key{NOSUPSYM}]
%  Deactivate automatic identification of "super
%  symmetry"\index{supersymmetry} (see
%  comments). This is automatically enforced in case of \aba\ or
%  \resp\ calculations.

\item[\Key{PUNCHINPUTORBITALS}]
  Punch input orbitals with label \Sec{*MOLORB}, Format (4F18.14).
  These orbitals may {\it e.g.\/} be transferred to another computer and
  read there with \quotekw{.MOSTART} followed by \quotekw{ FORM18} on
  next line from this input section.

\item[\Key{PUNCHOUTPUTORBITALS}]
  Punch final orbitals with label \Sec{*MOLORB}, Format (4F18.14).
  These orbitals may {\it e.g.\/} be transferred to another computer and
  read there with \quotekw{.MOSTART} followed by \quotekw{ FORM18} on
  next line from this input section.

\item[\Key{REORDER}]
Default: no reordering.
\begin{verbatim}
  READ (LUINP,*) (NREOR(I), I = 1,NSYM)
  DO I = 1,NSYM
     IF (NREOR(I) .GT. 0) THEN
        READ (LUINP,*) (IMONEW(J,I), IMOOLD(J,I), J = 1,NREOR(I))
     END IF
  END DO
  NREOR(I) = number of orbitals to be reordered in symmetry I
  IMONEW(J,I), IMOOLD(J,I) are orbital numbers in symmetry I.

For example if orbitals 1 and 5 in symmetry 1 should change place, specify
.REORDER
 2 0 0 0
 1 5 5 1
\end{verbatim}
  Reordering of molecular orbitals (see comments).

\item[\Key{SUPSYM}]
  Default is NOSUPSYM.\\
  Enforce automatic identification of "super
  symmetry"\index{supersymmetry} (see comments).

\item[\Key{SYMMETRIC ORTHONORMALIZATION}]
  Default: Gram-Schmidt orthonormalization\\
  Symmetric orthonormalization of input
  orbitals\index{orthonormalization!symmetric}.

\item[\Key{THRSSY}]
  \kw{READ (LUINP,*) THRSSY} \\
  Threshold for identification of "super
  symmetry"\index{supersymmetry} and degeneracies among
  "super symmetries" from matrix elements of the kinetic energy matrix
  (default: 5.0D-8).

\end{description}


\noindent{\bf Comments:}

If \quotekw{\Key{SUPSYM}} is specified, then
{\sir} automatically identifies "super symmetry"\index{super symmetry!orbitals},
{\it i.e.\/} it assigns orbitals to the irreps of the true point
group of the molecule\index{symmetry!group} which is a
"super group" of the Abelian group used in the calculation.
Degenerate orbitals will be averaged and the "super symmetry"
will be enforced in the orbitals.
Note that "super symmetry" can only be used
in the RHF, MP2, MCSCF, and RESPONS modules, and should
not be invoked if other modules are used,
for example, if \Sec{*PROPERTIES} (\aba) is invoked.
%hj aug 04: it should be OK for closed shell cases, also for CC ???
% it is only for spatially degenerate states that elements
% of the orbital gradient may be non-zero, right ???
Also, it cannot be used
in finite field calculations where the field lowers the symmetry.
The initial orbitals must be symmetry orbitals, and the super symmetry
analysis is performed on the kinetic energy matrix in this basis.
The \quotekw{.THRSSY} option is used to define when the kinetic
energy matrix element between two orbitals is considered to be
zero and when two diagonal matrix elements are degenerate. In the
first case the orbitals can belong to different irreps of the
supergroup and in the second case the two orbitals are considered
to be degenerate. The analysis will fail if there are accidental
degeneracies in diagonal elements.  This can happen if the nuclear
geometry deviates slightly from a higher symmetry point group, for
example because too few digits has been used in the input of the
nuclear geometry. If the program stops because the super symmetry
analysis fails with a degeneracy error, you might consider to use
more digits in the nuclear coordinates, to change \kw{THRSSY}, or
to disable super symmetry by not using \quotekw{.SUPSYM}.  The value of
\kw{THRSSY} should be sufficiently small to avoid accidental
degeneracies and sufficiently large to ignore small errors in
geometry and numerical round-off errors.


\Key{REORDER}\index{orbital reordering} can for instance be used for
linear molecules to interchange
undesired delta orbitals among the active orbitals in symmetry 1 with
sigma orbitals.  Another example is movement of the core orbital to the
RAS1 space for core hole calculation.  In general, use of this option
necessitates a pre-calculation with STOP AFTER MO-ORTHONORMALIZATION and
identification of the various orbitals by inspection of the output.


\pagebreak[3]
\subsection{\label{ref-popinp}\Sec{POPULATION ANALYSIS}}

{\bf Purpose:}

To direct population analysis\index{population analysis} of the wave function.
Requires a set of natural orbitals\index{natural orbital} and their occupation.

\begin{description}
\item[\Key{ALL}]
  Do all options.

%\item[\Key{DIPMOM}]
%  Calculate dipole moments. Note that this requires that the dipole
%  length integrals are available on the file \verb|AOONEINT|.\index{dipole moment}
%Aug 04: not working as far as I know /hjaaj

\item[\Key{GROSSALL}]
  Do all gross population analysis. Note that this requires that the dipole
  length integrals are available on the file \verb|AOPROPER|\index{population analysis}

\item[\Key{GROSSMO}]
  Do gross MO population analysis.\index{population analysis}

\item[\Key{MULLIKEN}]
  Do Mulliken population analysis\index{population analysis}\index{population analysis!Mulliken}\index{Mulliken population analysis}

\item[\Key{NETALL}]
  Do all net population analysis.\index{population analysis}

\item[\Key{NETMO}]
  Do net MO population analysis.\index{population analysis}

\item[\Key{PRINT}]
  Default = 1\\
  \kw{READ (LUINP,*) IPRMUL} \\
  Print level for population analysis.

%\item[\Key{QUADRP}]
%  Calculate quadrupole moments. Note that this requires that the quadrupole
%  integrals are available on the file \verb|AOONEINT|\index{quadrupole moment}
%Aug 04: not working as far as I know /hjaaj

\item[\Key{VIRIAL}]\index{virial analysis}
  Do virial analysis.
\end{description}

\pagebreak[3]
\subsection{\label{ref-priinp}\Sec{PRINT LEVELS}}

{\bf Purpose:}

To control the printing of output.

\begin{description}
\item[\Key{CANONI}]
  Generate canonical/natural orbitals if the wave function has
  converged\index{canonical orbital}\index{natural orbital}.

\item[\Key{IPRAVE}]
  \kw{READ (LUINP,*) IPRAVE} \\
  Sets print level for routines used in "super symmetry" averaging
  (default = 0).

\item[\Key{IPRCIX}]
  \kw{READ (LUINP,*) IPRCIX} \\
  Sets print level for setup of determinant/CSF index information (default = 0).

\item[\Key{IPRCNO}]
  \kw{READ (LUINP,*) IPRCNO} \\
  Sets print level for \quotekw{.TRACI} option (default = 1,
  to print the natural orbital occupations in each iteration set
  IPRCNO = 1, higher values will give more print).

\item[\Key{IPRDIA}]
  \kw{READ (LUINP,*) IPRDIA} \\
  Sets print level for calculation of CI diagonal (default = 0)

\item[\Key{IPRDNS}]
  \kw{READ (LUINP,*) IPRDNS} \\
  Sets print level for calculation of CI density matrices (default = 0)

%\item[\Key{IPRERR}]
%  \kw{READ (LUINP,*) IPRERR} \\
%  Sets print level for statistics in error file, LUERR (default = 1)

\item[\Key{IPRFCK}]
  \kw{READ (LUINP,*) IPRFCK} \\
  Sets print level in the supersymmetry section (default=0).

\item[\Key{IPRKAP}]
  \kw{READ (LUINP,*) IPRKAP} \\
  Sets print level in routines for calculation of optimal orbital trial
  vectors (default = 0)

\item[\Key{IPRSIG}]
  \kw{READ (LUINP,*) IPRSIG} \\
  Sets print level for calculation of CI sigma vectors (default = 0)

\item[\Key{IPRSOL}]
  \kw{READ (LUINP,*) IPRSOL} \\
  Sets print level in the solvent contribution parts of the
  calculation (default = 5).

\item[\Key{NOSUMMARY}]
  No final summary of calculation.

\item[\Key{POPPRI}]
  \kw{READ (LUINP,*) LIM\_POPPRI} \\
  Print Mulliken occupation of the first LIM\_POPPRI atoms in
  each SCF iteration. Useful for understanding convergence.
  (Default = 16, corresponding to two lines of output).

\item[\Key{PRINTFLAGS}]
 Default: flags set by general levels in \quotekw{\Key{PRINTLEVELS}}
\begin{verbatim}
  READ (LUINP,*) NUM6, NUM4
  IF (NUM6 .GT. 0) READ (LUINP,*) (NP6PTH(I), I=1,NUM6)
  IF (NUM4 .GT. 0) READ (LUINP,*) (NP4PTH(I), I=1,NUM4)
\end{verbatim}
  Individual print flag settings (debug option).

\item[\Key{PRINTLEVELS}]
  Default: IPRI6 = 0 and IPRI4 = 5 \\
  \kw{READ (LUINP,*) IPRI6,IPRI4 } \\
  Print levels on units LUW6 and LUW4, respectively.
%
%\item[\Key{PRINTUNITS}]
%  \kw{READ (LUINP,*) LUW6,LUW4 } \\
%  Unit numbers for general output and summary output, respectively
%  (default: LUW4 = 6 and LUW6 = 6).
%
\item[\Key{THRPWF}]
  \kw{READ (LUINP,*) THRPWF} \\
  Threshold for printout of wave function CI coefficients (default = 0.05).
 \end{description}



%\ifsolvent
\pagebreak[3]
\subsection{\label{ref-rhfinp}\Sec{SCF INPUT}}

{\bf Purpose:}

This section deals with the closed shell, one open shell and
high--spin spin-restricted 
Hartree--Fock cases\index{SCF}\index{HF}\index{Hartree--Fock}
and Kohn-Sham DFT\index{DFT}. 
The input here will usually only be used if either
\quotekw{\Key{DFT}} or \quotekw{\Key{HF}}
has been specified under \quotekw{\Sec{*WAVE FUNCTIONS}}
(though it is also needed for MP2 calculations based on saved closed-shell HF
orbitals).
High--spin spin-restricted open-shell Hartree--Fock or Kohn--Sham DFT calculations are activated by
using the \quotekw{.SINGLY OCCUPIED} described here.
Other single configuration cases with more than one open shell\index{open shell!SCF}
can be handled by the general \quotekw{\Key{MCSCF}} option, by appropriate specifications
in the \Sec{CONFIGURATION INPUT} section.

\begin{description}
\item[\Key{AUTOCCUPATION}]
  Default for SCF calculations starting from extended H\"{u}ckel, EWMO, or H1DIAG
  starting orbitals.

  Allow the distribution of the Hartree--Fock/DFT occupation numbers over
  symmetries\index{Hartree--Fock occupation}\index{HF occupation} to
  change based on changes in orbital ordering during DIIS\index{DIIS} optimization.
  This keyword is incompatible with \quotekw{.SINGLY OCCUPIED} and \quotekw{.COREHOLE}, or
  if the HF calculation is followed by CI or MCSCF.

%\item[\Key{EDIIS}]
%  Use a from Kudin {\it et al.} slightly modified (E)DIIS-scheme.
%  Keys associated with DIIS are also valid for EDIIS (e.g. MXDIIS etc.)

\item[\Key{C2DIIS}]
  Use Harrell Sellers' C2-DIIS algorithm instead of Pulay's C1-DIIS algorithm
  (see comments).

\item[\Key{COREHOLE}]
  \kw{READ (LUINP,*) JCHSYM,JCHORB} \\
  JCHSYM = symmetry of core orbital\\
  JCHORB = the orbital in symmetry JCHSYM with a single core hole\\
  Single core hole\index{core hole} open shell RHF calculation, \quotekw{\Key{OPEN
  SHELL}} must not
  be specified.  The specified core orbital must be
  inactive\index{inactive orbital}.
  The number of doubly occupied orbitals in symmetry \kw{JCHSYM} will be reduced with one
  and instead an open shell orbital will be added for the core hole orbital.
  If the specified core orbital is not the last occupied orbital in symmetry
  \kw{JCHSYM} it will switch places with that orbital and user-defined reordering
  is not possible.
  If explicit reordering is required you must also reorder
  the core orbital yourself and let \kw{JCHORB} point to the last occupied orbital
  of symmetry \kw{JCHSYM}.  See comments below.

\item[\Key{CORERELAX}]
  (ignored if \quotekw{\Key{COREHOLE}} isn't also specified)\\
  Optimize core hole\index{core hole} state with relaxed
  core\index{relaxed core} orbital using Newton-Raphson algorithm.
  It is assumed that this calculation follows an optimization
  with frozen core orbital and the specific value of
  \quotekw{JCHORB} under \quotekw{\Key{COREHOLE}} is ignored (no
  reordering will take place).

%\item[\Key{DIRFOCK}]
%  Direct Fock matrix constructions (recalculate integrals when needed).
%  Default: AO integrals or P-supermatrix integrals read from disk.
%\fi

\item[\Key{DOUBLY OCCUPIED}]
    \kw{READ (LUINP,*) (NRHF(I),I=1,NSYM)} \\
  \index{HF}\index{SCF}\index{Hartree--Fock}\index{MP2}\index{M{\o}ller-Plesset!second-order}
  Explicit specification of number of doubly occupied orbitals in each symmetry
  for DFT, RHF and MP2 calculations. This keyword
  is required when Hartree--Fock or MP2 is part of a multistep
  calculation which includes an MCSCF wave function. 
  Otherwise the program by default will try to guess the occupation,
  corresponding to the  \quotekw{.AUTOCC} keyword.

\item[\Key{ELECTRONS}]
  \kw{READ (LUINP,*) NRHFEL} \\
  Number of electrons in the molecule\index{electrons in molecule}.
  By default, this number will be determined on the basis of the nuclear
  charges and the total charge of the molecule\index{charge of molecule}
  as specified in the \molinp\ file.
  The keyword is incompatible with the keywords \quotekw{.DOUBLY OCCUPIED},
  \quotekw{.OPEN SHELL}, and \quotekw{.SINGLY OCCUPIED}.

\item[\Key{FC MVO}]
  \kw{READ (LUINP,*) (NMVO(I), I = 1,NSYM)} \\
  Modified virtual orbitals using Bauschlichers suggestion
  (see Ref.~\cite{cwbjcp72})
  for CI or for start guess for MCSCF. The modified virtual orbitals
  are obtained by  diagonalizing the virtual-virtual
  block of the Fock matrix constructed from NMVO(1:NSYM) doubly
  occupied orbitals.
  The occupied SCF orbitals (i.e those specified with
  \quotekw{.DOUBLY OCCUPIED} and \quotekw{.OPEN SHELL}
  or by automatic occupation) are not modified.
  The construction of modified virtual orbitals
  will follow any SCF and MP2 calculations.
  See comments below.

\item[\Key{FOCK ITERATIONS}]
  \kw{READ (LUINP,*) MAXFCK} \\
  Maximum number of closed-shell Roothaan\index{Roothaan iteration}
  Fock iterations (default = 0).

\item[\Key{FROZEN CORE ORBITALS}]
  \kw{READ (LUINP,*) (NFRRHF(I),I=1,NSYM)} \\
  Frozen orbitals per symmetry (if MP2 follows then at least these orbitals
  must be frozen in the MP2 calculation).
  NOTE: no Roothaan Fock iterations allowed if frozen orbitals.

\item[\Key{H1VIRT}] Use the virtual orbitals that diagonalize the
  one-electron Hamiltonian operator.

\item[\Key{MAX DIIS ITERATIONS}]
  \kw{READ (LUINP,*) MXDIIS} \\
  Maximum number of DIIS iterations\index{iteration number!DIIS, max}\index{DIIS!max iterations} (default = 60).

\item[\Key{MAX ERROR VECTORS}]
  \kw{READ (LUINP,*) MXEVC} \\
  Maximum number of DIIS error vectors\index{DIIS!error vectors, max}
  (default = 10, if there is sufficient memory available to hold these
  vectors in memory).

\item[\Key{MAX MACRO ITERATIONS}]
  \kw{READ (LUINP,*) MXHFMA} \\
  Maximum number of QCSCF macro\index{iteration number!QCSCF macro, max}
 iterations (default = 15).


\item[\Key{MAX MICRO ITERATIONS}]
  \kw{READ (LUINP,*) MXHFMI} \\
  Maximum number of QCSCF\index{SCF!quadratic convergent} micro iterations per macro iteration (default = 12).

\item[\Key{NODIIS}]
  Do not use DIIS algorithms\index{DIIS} (default: use DIIS algorithm).

\item[\Key{NONCANONICAL}]
  No transformation to canonical orbitals\index{canonical orbital}.

\item[\Key{NOQCSCF}]
  No quadratically convergent SCF\index{SCF!no quadratically convergent} iterations.
  Default is to switch to QCSCF if DIIS doesn't converge.

\item[\Key{OPEN SHELL}]
  Default = no open shell\\
  \kw{READ (LUINP,*) IOPRHF} \\
  Symmetry of the open shell in a one open shell\index{open shell!HF}\index{HF!open shell}
  calculation. See also \quotekw{.SINGLY OCCUPIED} for high-spin ROHF with more than one
  singly occupied orbital.

\item[\Key{PRINT}]
  \kw{READ (LUINP,*) IPRRHF} \\
  Resets general print level to \verb|IPRRHF| in Hartree--Fock/DFT calculations
  (if not specified, global print levels will be used).

\item[\Key{SHIFT}] 
  \kw{READ (LUINP,*) SHFTLVL} \\
  Initial value of level-shift parameter in DIIS iterations. 
  The default value is 0.0D0 (no level shift).
  May be tried if convergence problems in DIIS. The value is added
  to the diagonal of the occupied part of the Fock matrix before
  Roothaan diagonalization, reducing the mixing of occupied and
  virtual orbitals (step restriction).
  NOTE that the value should thus be negative.  The DIIS routines
  will automatically invoke level-shifting (step restriction) if
  DIIS seems to be stalling.

\item[\Key{SINGLY OCCUPIED}] Default = no singly occupied orbitals \\
    \kw{READ (LUINP,*) (NROHF(I),I=1,NSYM)} \\
  High--spin spin-restricted open-shell Hartree--Fock (aka HSROHF, ROHF)
  or Kohn-Sham DFT (aka HSROKS, HSRODFT, RODFT, ROKS).
 \index{HSROHF}\index{HSRODFT}\index{HSROKS}\index{ROHF!high spin}\index{RODFT!high spin}\index{ROKS!high spin}
  Specify the number of singly occupied orbitals in each irreducible representation
  of the molecular point group. Only the high-spin state of these
  singly-occupied orbitals will be made and used in the calculations.
  We recommend to always run high-spin open-shell geometry optimizations as direct calculations
  (\quotekw{\Key{DIRECT}} under \quotekw{\Sec{DALTON}}),
  because analytical molecular gradients are only implemented for direct calculations
  (numerical gradients will be used for non-direct calculations).

\item[\Key{THRESH}]
  Default = 1.0D-05 (1.0D-06 if MP2)\\
  \kw{READ (LUINP,*) THRRHF} \\
  Hartree--Fock/DFT convergence threshold for energy gradient.  The convergence
  of the energy will be approximately the square of this number.

\end{description}


\noindent{\bf Comments:}

By default, the RHF/DFT part of a calculation will consist of :
\begin{enumerate}
\item {MAXFCK Roothaan Fock iterations (early exit if convergence
    or oscillations). However, the default is that no Roothaan Fock
iterations are done unless explicitly requested through the keyword
\quotekw{.FOCK I}.
}
\item {MXDIIS DIIS iterations (exit if convergence, {\it i.e.\/} gradient norm
    less than THRRHF, and if convergence rate too slow or even diverging).
}
\item {Unless NOQCSCF, quadratically convergent Hartree--Fock/DFT until
    gradient norm less than THRRHF.
}
\item{If \quotekw{.FC MVO} has been specified
    then the virtual SCF orbitals will be modified by diagonalizing
    the virtual-virtual block of
    a modified Fock matrix: the Fock matrix
    based on the occupied orbitals specified after the keyword, a
    good choice is the inactive (doubly occupied) orbitals in the
    following CI or MCSCF.
    The occupied SCF orbitals will not be modified.
    If the RHF calculation is followed by a CI or an MCSCF calculation,
    \quotekw{.FC MVO} will usually provide much
    better start orbitals than the canonical orbitals (canonical
    orbitals will usually put diffuse, non-correlating orbitals in the
    active space). \\
    WARNING: if both \quotekw{.MP2} and \quotekw{.FC MVO} are specified,
    then the MP2 orbitals will be destroyed and replaced with \quotekw{.FC MVO}
    orbitals.
}
\end{enumerate}

In general \quotekw{.DOUBLY OCCUPIED} should be specified for CI or MCSCF
\index{HF occupation}\index{Hartree--Fock occupation}\index{CI}\index{MCSCF}
\index{Configuration Interaction}
wave function calculations -- you anyway need to know the distribution
of orbitals over symmetries to specify the \quotekw{*CI INPUT} input.
For RHF\index{RHF}\index{SCF}\index{Hartree--Fock}
or MP2\index{MP2}\index{M{\o}ller-Plesset!second-order}
calculations the orbital occupation will be determined on the
basis of the nuclear charges and molecular charge of the molecule as
specified in the \molinp\ file.

By default, starting orbitals and initial orbital occupation will
be determined automatically on the basis of a H\"{u}ckel\index{H\"{u}ckel}
calculation (for molecules where all nuclear charges are
less than or equal to 36), corresponding to the \quotekw{.AUTOCC} keyword.
\index{starting orbitals!SCF}\index{H\"{u}ckel!starting orbitals}.
If problems is experienced due to the
H\"{u}ckel starting guess, it can be avoided by requiring another set of
starting orbitals ({\it e.g.\/} \verb|H1DIAG|).

%The default convergence threshold is quite sharp (compare with the
%default for MCSCF), this is done in order to have good orbitals
%for MP2 calculations.  For Hartree--Fock
%calculations with many basis functions
%which are not to be followed by MP2 or used for finite difference
%property calculations, some CPU time may be save by lowering the
%threshold to the minimum acceptable accuracy.

It is our experience that
it is usually most efficient not to perform any Roothaan Fock iterations
before DIIS is activated, therefore, MAXFCK = 0 as default.
The algorithm described in
Harrell Sellers, Int. J. Quant. Chem. {\bf 45}, 31-41 (1993) is
also implemented, and may be selected with \quotekw{\Key{C2DIIS}}.


FC MVO: This option can be used without a Hartree--Fock calculation
to obtain compact virtual orbitals, but \quotekw{.DOUBLY OCCUPIED} must be
specified anyway in order to identify the virtual orbitals to be transformed.

COREHOLE: Enable SCF
single core-hole\index{core hole} calculations. To perform
an SCF core hole calculation just add the \quotekw{\Key{COREHOLE}}
keyword to the input for the closed-shell RHF ground state
calculation, specifying from which orbital to remove an electron,
and provide the program with the ground state orbitals using the
appropriate \quotekw{\Key{MOSTART}} option (normally \kw{NEWORB}).
Note that this is different from the MCSCF version of
\quotekw{\Key{COREHOLE}} under \quotekw{\Sec{OPTIMIZATION}}
(p.~\pageref{ref-optinp}); in the MCSCF case the user must
explicitly move the core hole orbital from the inactive class to
RAS1 by modifying the \quotekw{\Sec{CONFIGURATION INPUT}}
(p.~\pageref{ref-wavinp}) specifications between the initial
calculation with filled core orbitals and the core hole
calculation. The core hole\index{core hole} orbital will be
frozen\index{frozen core hole} in the following optimization.
After this calculation has converged, the CORERELAX option may be
added and the core orbital will be relaxed\index{relaxed core hole}.  
When CORERELAX is specified it is assumed that the
calculation was preceded by a frozen core calculation, and that
the orbital has already been moved to the open shell orbital. Only
the main peak can be obtained in SCF calculations, for shake-up
energies MCSCF must be used.

\pagebreak[3]
\subsection{\label{ref-solinp}\Sec{SOLVENT}}

{\bf Purpose:}

Model solvent effects with the self-consistent
reaction\index{reaction field} field model.
Any specification of dielectric constant(s)\index{dielectric constant}
will activate this model.

\begin{description}
\item[\Key{CAVITY}]
  Required, no defaults.\\
  \kw{READ (LUINP,*) RSOLAV}\\
  Enter radius of spherical cavity\index{cavity!radius} in atomic units (\bohr{}).

\item[\Key{DIELECTRIC CONSTANT}]
  \kw{READ (LUINP,*) EPSOL}\\
  Enter relevant dielectric constant\index{dielectric constant} of solvent.

\item[\Key{INERSINITIAL}]
  \kw{READ (LUINP,*) EPSOL, EPPN}\\
  Enter static and optical dielectric constant\index{dielectric constant} of solvent for calculation
  of the initial state defining inertial polarization\index{inertial
  polarization}. Note that the optical dielectric constant specified here
  only will be used in case there is a calculation of response
  properties, for which this is an alternative input to the use of the
  keyword \Key{INERSFINAL}.

\item[\Key{INERSFINAL}]
  \kw{READ (LUINP,*) EPSTAT,EPSOL}\\
  Enter static and optical dielectric\index{dielectric constant} constants of solvent
  for calculation of the final state with inertial polarization
  from a previous calculation with \quotekw{\Key{INERSINITIAL}}\index{final polarization}. 
  This can for example be used to optimize an excited electronic state
  with inertial polarization from a previous ground state calculation.
  This keyword an  also be used to specify the static and optical dielectric constants
  for non-equilibrium solvation linear, quadratic, or cubic response functions,
  see also Sec.~\ref{sec:solvnoneqrsp}, but this is usually easier done with
  \quotekw{.INERSINITIAL} (requires only one \dalton\ calculation instead of two).

\item[\Key{MAX L}]
  Required, no defaults.\\
  \kw{READ (LUINP,*) LSOLMX}\\
  Enter maximum L quantum number in multipole expansion of charge
  distribution in cavity.

\item[\Key{PRINT}]
  \kw{READ (LUINP,*) IPRSOL} \\
  Print level in solvent module routines (default = 0).
\end{description}

\noindent{\bf Comments:}

One and only one of \quotekw{\Key{DIELECTRIC CONSTANT}},
\quotekw{\Key{INERSINITIAL}}, and \quotekw{\Key{INERSFINAL}} must be
specified.
%\fi % end of \ifsolvent



\pagebreak[3]
\subsection{\label{ref-stpinp}\Sec{STEP CONTROL}}

{\bf Purpose:}

User control of the NEO restricted step optimization.

\begin{description}
\item[\Key{DAMPING FACTOR}]
  Default = 1.0D0\\
  \kw{READ (LUINP,*) BETA} \\
  Initial value of damping (BETA).\index{damping}

\item[\Key{DECREMENT FACTOR}]
  Default = 0.67D0\\
  \kw{READ (LUINP,*) STPRED} \\
  Decrement factor on trust radius\index{trust radius}

\item[\Key{GOOD RATIO}]
  Default = 0.8D0 \\
  \kw{READ (LUINP,*) RATGOD} \\
  Threshold ratio for good second order agreement: the trust radius can
  be increased if ratio is better than RATGOD.

\item[\Key{INCREMENT FACTOR}]
  Default = 1.2D0\\
  \kw{READ (LUINP,*) STPINC} \\
  Increment factor on trust radius.\index{trust radius}

\item[\Key{MAX DAMPING}]
  Default = 1.0D6\\
  \kw{READ (LUINP,*) BETMAX} \\
  Maximum damping value.\index{damping}

\item[\Key{MAX STEP LENGTH}]
  Default = 0.7D0\\
  \kw{READ (LUINP,*) STPMAX} \\
  Maximum acceptable step length, trust radius will never be larger than
  STPMAX even if the ratio is good as defined by GOOD RATIO.

\item[\Key{MIN DAMPING}]
  Default = 0.2D0\\
  \kw{READ (LUINP,*) BETMIN} \\
  Minimum damping value

\item[\Key{MIN RATIO}]
  Default = 0.4D0 for ground state, 0.6 for excited states\\
  \kw{READ (LUINP,*) RATMIN} \\
  Threshold ratio for bad second order agreement: the trust radius is
  to be decreased if ratio is worse than RATMIN.

\item[\Key{NO EXTRA TERMINATION TESTS}]
  Skip extra termination tests and converge micro iterations to
  threshold.   Normally the micro iterations are terminated if the
  reduced NEO matrix has more negative eigenvalues than corresponding
  to the desired state, because then we are in a "superglobal" region
  and we just want to step as quickly as possible to the region where
  the Hessian (and NEO matrix) has the correct structure.  Further
  convergence is usually wasted.

\item[\Key{REJECT THRESHOLD}]
  Default = 0.25 for ground state, 0.4 for excited states\\
  \kw{READ (LUINP,*) RATREJ} \\
  Threshold ratio for unacceptable second order agreement: the step
  must be rejected if ratio is worse than RATREJ.

\item[\Key{THQKVA}]
  Default: 8.0 for MCSCF; 0.8 for QCSCF\\
  \kw{READ (LUINP,*) THQKVA} \\
  Convergence factor for micro iterations in local (quadratic) region:
  THQKVA*(norm of gradient)**2

\item[\Key{THQLIN}]
  Default: 0.2D0\\
  \kw{READ (LUINP,*) THQLIN} \\
  Convergence factor for micro iterations in global (linear) region: \\
  THQLIN*(norm of gradient)

\item[\Key{THQMIN}]
  Default: 0.1D0\\
  \kw{READ (LUINP,*) THQMIN} \\
  Convergence threshold for auxiliary roots in NEO MCSCF optimization.

\item[\Key{TIGHT STEP CONTROL}]
  Tight step control also for ground state calculations
  (tight step control is always enforced for excited states)

\item[\Key{TOLERANCE}]
  Default = 1.1D0\\
  \kw{READ (LUINP,*) RTTOL} \\
  Acceptable tolerance in deviation of actual step from trust radius
  (the default value of 1.1 corresponds to a maximum of 10\% deviation).

\item[\Key{TRUST RADIUS}]
  Default = STPMAX=0.7D0 or, if restart, trust radius determined by previous
            iteration.\index{trust radius}\\
  \kw{READ (LUINP,*) RTRUST} \\
  Initial trust radius.

\end{description}


\pagebreak[3]
\subsection{\label{ref-trainp}\Sec{TRANSFORMATION}}

{\bf Purpose:}

Transformation\index{integral transformation} of two-electron
integrals\index{two-electron integral} to molecular orbital
basis\index{molecular orbital}.

\begin{description}
\item[\Key{FINAL LEVEL}]
  \kw{READ (LUINP,*) ITRFIN} \\
  Final integral transformation\index{integral transformation} level (only active if the keyword
  \quotekw{\Key{INTERFACE}} has been specified, or this is an \aba\ or
  \resp\ calculation.

\item[\Key{LEVEL}]
  \kw{READ (LUINP,*) ITRLVL} \\
  Integral transformation level (see comments).

\item[\Key{OLD TRANSFORMATION}]
  Use existing transformed integrals

\item[\Key{PRINT}]
  \kw{READ (LUINP,*) IPRTRA} \\
  Print level in integral transformation module

\item[\Key{RESIDENT MEMORY}]
  \kw{READ (LUINP,*) MWORK} \\
  On virtual memory computers, the transformation will run more
  efficiently if it can be kept within the possible resident memory
  size: the real memory size.  {\sir} will attempt to only use MWORK
  double precision words in the transformation.
\end{description}


\noindent{\bf Comments:}

There are several types of integral transformations which may be
specified by the two transformation level keywords.
\begin{itemize}
   \item[0:] CI calculations, MCSCF gradient (default if CI, but
             no MCSCF specified).
             One index all orbitals, three indices only active
             orbitals.

   \item[1:] Obsolete, do not use.

   \item[2:] Default for MCSCF optimization. All integrals needed for {\sir}
             second-order MCSCF optimization, including the integrals
             needed to explicitly construct the diagonal of the orbital
             Hessian. Two indices occupied orbitals, two indices all
             orbitals, with some reduction for inactive indices.
             Both (cd/ab) and (ab/cd) are stored.

   \item[3:] Same integrals as 2, including also the (ii/aa) and
             (ia/ia) integrals for exact inactive-secondary diagonal elements
             of the orbitals Hessian.

   \item[4:] All integrals with minimum two occupied indices.

   \item[5:] 3 general indices, one occupied index.  Required for MP2
             natural orbital analysis (the MP2 module automatically
             performs an integral transformation of this level).

  \item[10:] Full transformation.
\end{itemize}


\pagebreak[3]
\subsection{\label{ref-cube}\Sec{CUBE}}

{\bf Purpose:}

Generates cube file\index{cube file} of total SCF electron density and/or
molecular orbitals after SCF calculations. The keyword \quotekw{\Key{INTERFACE}}
must be specified.

\begin{description}
\item[\Key{DENSITY}]
  Generates cube file ``\kw{density.cube}'' with total SCF electron density.

\item[\Key{HOMO}]
  Generates cube file ``\kw{homo.cube}'' with the information of the highest
occupied molecular orbitals.

\item[\Key{LUMO}]
  Generates cube file ``\kw{lumo.cube}'' with the information of the lowest
unoccupied molecular orbitals.

\item[\Key{MO}]
  \kw{READ (LUINP,*) IDX\_MO} \\
  Generates cube file ``\kw{mo.cube}'' with specified indices of molecular orbitals
by ``\kw{IDX\_MO}''. For instance, valid format is like ``1-6,7,10-12'' only including
digits, minus sign and comma.

\item[\Key{FORMAT}]
  \kw{READ (LUINP,*) CUBE\_FORMAT} \\
  Specifies cube file format, only ``\kw{GAUSSIAN}'' (Gaussian cube file format,
see\linebreak \verb|http://www.gaussian.com/g_tech/g_ur/u_cubegen.htm|) for the
time being.

\item[\Key{ORIGIN}]
  \kw{READ (LUINP,*) CUBE\_ORIGIN} \\
  Reads the coordinates (a.u.) of origin/initial point.

\item[\Key{INCREMENT}]
  \kw{READ (LUINP,*) N1, X1, Y1, Z1} \\
  \kw{READ (LUINP,*) N2, X2, Y2, Z2} \\
  \kw{READ (LUINP,*) N3, X3, Y3, Z3} \\
  Reads the number of increments and increments (a.u.) along three running directions,
in which ``\kw{(X1,Y1,Z1)}'' is the slowest running direction, and ``\kw{(X3,Y3,Z3)}''
is the fastest running direction.

As described at \verb|http://www.gaussian.com/g_tech/g_ur/u_cubegen.htm|, if the
origin/initial point is (X0,Y0,Z0), then the point at (I1,I2,I3) has coordinates:

X-coordinate: X0+(I1-1)*X1+(I2-1)*X2+(I3-1)*X3\\
Y-coordinate: Y0+(I1-1)*Y1+(I2-1)*Y2+(I3-1)*Y3\\
Z-coordinate: Z0+(I1-1)*Z1+(I2-1)*Z2+(I3-1)*Z3
\end{description}


\pagebreak[3]
\section{\label{sec:ref-molorbinp} \Sec{*MOLORB} input module}

If formatted input of the molecular orbitals has been specified in
the \Sec{ORBITAL INPUT} section, then {\sir} will attempt to find
the two-star label "\verb|**MOLORB|" in the input file and read
the orbital coefficients from the lines following this label.

%\chapter{HF, SOPPA, and MCSCF molecular properties, \abacus}\label{ch:abacus}

\section{Directives for evaluation of HF, SOPPA, and MCSCF molecular properties}
\label{sec:abainp}

  The following directives may be included in the input to \aba.
They are organized according to the program section (module) names
in which they can appear.

\subsection{General: \Sec{*PROPERTIES}}\label{subsec:abacus}

This module controls the main features of the HF, SOPPA, and MCSCF property calculations,
that is, which properties is to be calculated.
In addition it includes
directives affecting the performance of several of the program
sections.
This includes HF and MCSCF molecular gradients and Hessians.
It should be noted, however, that the specification of what
kind of walk (minimization\index{geometry optimization}, location of
transition states\index{transition state}, dynamical
walks\index{dynamics}) is given in the \Sec{WALK} or \Sec{OPTIMIZE}
submodules in the general input  module. See also Chapter~\ref{ch:geometrywalks}.

See Chapter~\ref{ch:CC} for specification of CC property calculations.

Note that \resp\ (Chapter~\ref{ch:response})
is the most general part of the code for calculating
many different electronic linear, quadratic, or cubic molecular
response properties based on SCF, MCSCF, or CI wave functions or Kohn--Sham DFT.
Some of these SCF/MCSCF properties can also be requested
from the \Sec{*PROPERTIES} input modules described here.
NOTE: for such properties you should request them either here or
in \Sec{*RESPONSE}, otherwise you will calculate them twice!
Usually the output is nicest here in
the \Sec{*PROPERTIES} module ({\it e.g.\/} collected in tables and in
often used units, most properties are only given in atomic
units in \resp), and nuclear contributions are included if relevant.
No nuclear contributions are added in \resp .
Some specific properties, especially those involving nuclear derivatives,
can only be calculated via \Sec{*PROPERTIES}.
Other properties, for example quadratic and cubic molecular response functions
can only be calculated in the \Sec{*RESPONSE} module.


\begin{description}

\item[\Key{ALPHA}] Invokes the calculation of frequency dependent
polarizabilities\index{frequency}\index{polarizability}.
Combined with the keyword \Key{SOPPA} or \Key{SOPPA(CCSD)}
it invokes a SOPPA\index{SOPPA} or SOPPA(CCSD)
\index{SOPPA(CCSD)} calculation of the frequency dependent polarizability.

\item[\Key{CAVORG}]\verb| |\newline
\verb|READ (LUCMD,*) (CAVORG(ICOOR), ICOOR = 1, 3)|

Reads the origin to be used for the cavity\index{cavity!origin}\index{reaction field}
during a solvent calculation. By default this is chosen to be the
center of mass\index{center of mass}. Should by used with care, as it
has to correspond to the center used in the evaluation of the
undifferentiated solvent integrals in the {\her} section, see Chapter~\ref{sec:oneinp}.

\item[\Key{CTOCD}] Starts the calculation of the magnetic properties with the
CTOCD-DZ\index{CTOCD-DZ} method (Ref.~\cite{paololazz1,paololazz2,ctocd}). This sets also
automatically the .NOLOND option since the CTOCD-DZ formalism is gauge
independent for Nuclear Magnetic Shieldings. The default gauge origin is chosen to be
the center of mass.  \Key{CTOCD} and \Key{SOPPA} or \Key{SOPPA(CCSD)} can be combined to
perform CTOCD calculations at the SOPPA\index{SOPPA} or SOPPA(CCSD)\index{SOPPA(CCSD)} level.

\item[\Key{DIPGRA}] Invokes the calculation of dipole moment
gradients\index{dipole gradient}\index{atomic polar tensor}\index{APT}
(commonly known also as Atomic Polar Tensors
(APTs)) as described in Ref.~\cite{tuhhjajpjjcp84}. If combined with a
request for \Key{VIBANA} this will generate IR intensities\index{IR
intensity}.

\item[\Key{DIPORG}]\verb| |\newline
\verb|READ (LUCMD, *) (DIPORG(ICOOR), ICOOR = 1, 3)|

Reads in a user defined dipole origin\index{dipole origin}\index{origin!dipole, second order, quadrupole, third order}
in \bohr{}. It is also used for second order (.SECMOM), quadrupole (.QUADRU),
and third order (.THIRDM) moments.  This may affect properties in
which changes in the dipole origin\index{dipole origin} is canceled by
similar changes in the nuclear part.
It should also be used with care, as the same dipole
origin must be used during the integral evaluation sections, in
particular if one is doing numerical
differentiation with respect to electric field perturbations. For such
finite-field calculations\index{finite field}, we refer to Chapter
\ref{ch:finite}, which deals with finite field calculations. It is
primarily used for debugging.

\item[\Key{ECD}] Invokes the calculation of Electronic Circular
Dichroism (ECD)\index{electronic circular dichroism}\index{ECD} as
described in Ref.~\cite{klbaehkrthjopjtca90,mpkrthcpl388}. This
necessitates the specification of the number electronic
excitations\index{electronic excitation} in
each symmetry, given in the \Sec{EXCITA} module. The reader is
referred to the section where the calculation of ECD is described in
more detail (Sec.~\ref{sec:ecd}).

\item[\Key{EXCITA}] Invokes the calculation of electronic
excitation\index{electronic excitation}\index{excitation energy}
energies as residues of linear response functions\index{linear response}\index{response!linear}\index{single residue}
as described by Olsen and J\o rgensen \cite{jopjjcp82}. It also
calculates closely related properties like transition
moments\index{transition moments}\index{rotatory strength} and
rotatory strengths. Combined with the
keyword \Key{SOPPA} or \Key{SOPPA(CCSD)} it invokes a SOPPA\index{SOPPA}
or SOPPA(CCSD)\index{SOPPA(CCSD)} calculation of electronic
excitation energies and transition moments.


\item[\Key{EXPFCK}] Invokes the simultaneous calculation of
two-electron expectation values and derivative Fock-matrices. This is
default in direct and parallel runs in order to save memory. In
ordinary calculations the total CPU time will increase as a result of
invoking this option.

\item[\Key{EXPGRA}] Calculates the gradient of the orbital
  exponents. This can be used to optimize the exponents in an
  uncontracted basis set, if combined with a suitable script for
  predicting new orbital exponents based on this gradient.
  It has been used for the optimization of polarization consistent basis
  sets~\cite{fjjcp115}.

\item[\Key{GAUGEO}]\verb| |\newline
\verb|READ (LUCMD, *) (GAGORG(ICOOR), ICOOR = 1, 3)|

Reads in a user defined gauge origin\index{gauge origin} and overwrites
both the \Key{NOCMC} option, as well as the default value of
center-of-mass coordinates. Note that an unsymmetric position of the
gauge origin will lead to wrong results in calculations employing
symmetry, as the program will not be able to detect that such a choice
of gauge origin breaks the symmetry of the molecule.
(NOTE: a specification of \verb|.GAUGEO| in the \verb|**INTEGRALS| section
is \emph{not} used in this section, the \verb|**PROPERTIES| section.)

\item[\Key{HELLMA}]
Tells the program to use the Hellman--Feynman approximation when
calculating the molecular gradients and
Hessians~\cite{hhbook,rpfpr56,vbthwkkrmp96}---that is, all
contributions to the molecular gradient and Hessians from
differentiation of the orbitals are ignored. Requires large basis sets
in order to give reliable results, but does not require any
differentiated two-electron integrals.

%\item[\Key{HYPER}]\verb| |\newline
%
%Indicates that a quadratic response calculation is to be performed.

\item[\Key{INPTES}] Checks the input in the \Sec{*PROPERTIES} input
  section and then stops.


\item[\Key{LINEAR}] Invokes the linear coupling model for estimation
of Franck-Condon factors~\cite{optphavckrcp260}. In this model, the
gradient of an excited state is combined with the ground-state
vibrational frequencies and normal modes to provide vibronic coupling
constants. Requires that the DALTON.HES for the ground electronic
state is available, and that the keyword \verb|.VIBANA| also is
activated.

\item[\Key{LOCALI}] Invokes the generation of localized molecular
orbitals, which are then used in the analysis of second order
properties / linear response functions in terms of localized occupied
and virtual molecular orbitals. Currently only Mulliken localized
occupied orbitals or Foster-Boys~\cite{Boyloc} localized occupied and
virtual orbitals can be generated. Naturally, the generation of
localized molecular orbitals requires that the use of point group
symmetry is turned off.


\item[\Key{MAGNET}] Invokes the calculation of the molecular
magnetizability\index{magnetizability} (commonly known as magnetic
susceptibility) as
described in Ref.~\cite{krthklbpjhjajjcp99} and the rotational {\em g}
tensor (see keyword \Key{MOLGFA})\index{rotational g tensor}.  By
default this is done
using London orbitals\index{London orbitals} in order to
ensure fast basis set convergence as shown in
Ref.~\cite{krthklbpjhjajjcp99}. The use of London
orbitals can be disabled by the keyword \Key{NOLOND}.

Furthermore, the natural connection\index{natural connection}
(Ref.~\cite{joklbkrthpjtca90,krthjopjklbcpl235}) is the default in order to ensure
numerically stable results. The natural
connection can be turned off by the keyword \Key{NODIFC}, in which case
the symmetric connection\index{symmetric connection} will be used.

The gauge origin\index{gauge origin} is chosen to be the center of
mass\index{center of mass} of the molecule.
This origin can be changed by the two keywords \Key{GAUGEO} and
\Key{NOCMC}. This will of course not affect the total magnetizability,
only the magnitude of the dia- and paramagnetic terms.

Combined with the keyword \Key{SOPPA} or \Key{SOPPA(CCSD)} it invokes a SOPPA\index{SOPPA}
or SOPPA(CCSD)\index{SOPPA(CCSD)} calculation of the magnetizability and the rotational
{\em g} tensor (Ref.~\cite{spascpl260}). London orbitals are automatically disabled in
SOPPA\index{SOPPA} or SOPPA(CCSD)\index{SOPPA(CCSD)} calculations.

\Key{MAGNET} in combination with the keyword \Key{CTOCD}\index{CTOCD-DZ} invokes a
calculation without the use of London orbitals both with the CTOCD-DZ method
(Ref~\cite{paololazz1,paololazz2}) and with the common origin method.
Changing the default value of the gauge origin could give wrong results!

%\item[\Key{MCD}] Requests the calculation of Magnetic Circular
%Dichroism~\cite{}.

\item[\Key{MOLGFA}] Invokes the calculation of the rotational
{\em g} tensor\index{rotational g tensor} as described in
Ref.~\cite{jgkrthjcp105} and the molecular
magnetizability\index{magnetizability} (see keyword \Key{MAGNET}). By
default this is done
using London orbitals\index{London orbitals}  and the
natural connection\index{natural connection}. The use of London
orbitals can be turned off by the keyword \Key{NOLOND}.

By definition the gauge origin\index{gauge origin} of the molecular
g-factor is to be the
center of mass\index{center of mass} of the molecule, and although the
gauge origin can be
changed through the keywords \verb|.NOCMC | and \verb|.GAUGEO|, this
is not recommended, and may give erroneous results.

Note that if the isotopic constitution of the molecule is such that
the vibrational wave function has lower symmetry than the electronic
wave function, care must be taken to ensure the symmetry corresponds
to the symmetry of the nuclear framework. The automatic symmetry
detection routines will in general ensure that this is the case.

Combined with the keyword \Key{SOPPA} or \Key{SOPPA(CCSD)} it invokes a SOPPA\index{SOPPA}
or SOPPA(CCSD)\index{SOPPA(CCSD)} calculation of the magnetizability and the
rotational {\em g} tensor (Ref.~\cite{spascpl260}). London orbitals are automatically
disabled in SOPPA\index{SOPPA} or SOPPA(CCSD)\index{SOPPA(CCSD)} calculations.

\item[\Key{MOLGRA}] Invokes the calculation of the analytical
molecular gradient as \index{molecular gradient}described in Ref.~\cite{tuhjahjajpjjcp84}.

\item[\Key{MOLHES}] Invokes the calculation of the analytical
molecular Hessian\index{molecular Hessian} and gradient\index{molecular gradient} as
described in Ref.~\cite{tuhjahjajpjjcp84}.

\item[\Key{NACME}] Invokes the calculation of first-order non-adiabatic
coupling\index{non-adiabatic coupling matrix element} matrix elements as
described in Ref.~\cite{klbpjhjajjothjcp97}.
The keyword \Key{EXCITA} in this section,
the keywords \Key{FNAC} and \Key{NEXCIT} in section \Sec{EXCITA},
and the keyword \Key{SKIP} in section \Sec{TROINV}
must also be specified to get the first-order non-adiabatic coupling matrix elements.

% -- Feb. 2014 hjaaj: the NACMEs can still be calculated in the RESPONS module
% I think (not tested), but it is much easier to do it under **PROPERTIES,
% where it has been implemented to calculate vibrational g factors (.VIB_G)

%Presently, complete non-adiabatic
% coupling matrix elements cannot be obtained from this keyword alone,
% but has to be combined with subsequent calculations in the
% \resp\ program.

\item[\Key{NMR}] Invokes the calculation of both parameters
entering the NMR spin-Hamiltonian, that is nuclear
shieldings\index{nuclear shielding} and
indirect nuclear spin-spin coupling\index{spin-spin coupling}
constants. The reader is referred to the
description of the two keywords \Key{SHIELD} and \Key{SPIN-S}.

\item[\Key{NOCMC}] This keyword sets the gauge
origin\index{gauge origin}  to the origin of the Cartesian Coordinate system,
that is (0,0,0). This keyword is automatically invoked in case of VCD
and OECD calculations.

\item[\Key{NODARW}] Turns off the calculation of the Darwin
correction\index{Darwin correction}. By default the two major relativistic
corrections to the energy in the Breit-Pauli approximation, the
mass-velocity\index{mass-velocity correction} and Darwin
corrections, are calculated
perturbatively.

\item[\Key{NODIFC}] Disables the use of the natural
connection\index{natural connection}, and
the symmetric connection\index{symmetric connection} is used instead. The
natural connection and
its differences as compared to the symmetric connection is described
in Ref.~\cite{joklbkrthpjtca90,krthjopjklbcpl235}.

As the symmetric connection may give numerically inaccurate results,
it's use is not recommended for other than comparisons with other
programs.

\item[\Key{NOHESS}] Turns off the calculation of the analytical
molecular Hessian\index{Hessian}. This option overrides any request for the
calculation of molecular Hessians.

\item[\Key{NOLOND}] Turns off the use of London atomic
orbitals\index{London orbitals} in
the calculation of molecular magnetic properties. The gauge origin is
by default then chosen to be the center of mass. This can be altered
by the keywords \Key{NOCMC} and \Key{GAUGEO}.

\item[\Key{NOMASV}] Turns off the calculation of the
mass-velocity\index{mass-velocity correction}
correction. By default the two major relativistic corrections to the
energy  in the Breit-Pauli approximation, the mass-velocity and Darwin
corrections\index{Darwin correction}, are calculated
perturbatively.

\item[\Key{NQCC}] Calculates the nuclear quadrupole moment
coupling constants\index{nuclear quadrupole coupling}\index{NQCC}.

\item[\Key{NUMHES}] In VROA or Raman intensity calculations, use the
  numerical molecular Hessian calculated from the analytical molecular gradients instead
  of a fully analytical molecular Hessian calculation in the final
  geometry.

\item[\Key{OECD}] Invokes the calculation of Oriented Electronic Circular Dichroism
(OECD)\index{ECD!oriented}\index{electronic circular dichroism!oriented}\index{OECD}\index{oriented electronic circular dichroism}
as described in Ref.~\cite{tbpaehcpl246}. This
necessitates the specification of the number of electronic
excitations\index{electronic excitation} in
each symmetry, given in the \Sec{EXCITA} module.
Note that OECD can only be calculated at the mathematical origin
and the \Key{NOCMC} option is automatically turned on.
The reader is referred to Sec.~\ref{sec:ecd} for more details.

\item[\Key{OPTROT}] Requests the calculation of the optical rotation
of a molecule~\cite{thkrklbpjjofd99,plpmp91}.
By default the optical rotation is calculated
both with and without the use of London orbitals
(using the length gauge formulation).
Note that in the
formalism used in \dalton , this quantity vanishes in the static
limit, and frequencies need to be set in the \verb|*ABALNR| input
module. See also the description in Chapter~\ref{sec:optrot}.

\item[\Key{OR}] Requests the calculation of the optical rotation
of a molecule using the manifestly origin invariant ``modified''
velocity gauge formulation\cite{Pedersen:ORMVE}.
See also the description in Chapter~\ref{sec:optrot}.

\item[\Key{PHASEO}]\verb| |\newline
\verb|READ (LUCMD, *) (ORIGIN(ICOOR), ICOOR = 1, 3)|

Changes the origin of the phase-factors entering the London atomic orbitals.
This will change the value of all of the contributions to
the different magnetic field dependent properties when using London
atomic orbitals, but the total magnetic properties will remain
unchanged. To be used for debugging purposes only.

\item[\Key{POLARI}] Invokes the calculation of frequency-independent
polarizabilities\index{polarizability}. See the keyword \Key{ALPHA} in
this input section for the calculation of frequency-dependent polarizabilities.

\item[\Key{POPANA}] Invokes a population analysis\index{population analysis}\index{dipole gradient} based on the
dipole gradient as first introduced by Cioslowski \cite{jcjacs111}.
This flag also invokes the \Key{DIPGRA} flag and the \Key{POLARI} flags.
Note that the charges obtained in this approach is not without conceptual problems (as are the Mulliken charges)~\cite{hskrpoajcp120}.

\item[\Key{PRINT}]\verb| |
\newline
\verb|READ (LUCMD, *) IPRDEF|

Set default print level for the calculation.  Read one
more line containing print level. Default print level is the
value of \verb|IPRDEF| from the general input module.

\item[\Key{QUADRU}] Calculates the molecular quadrupole
moment\index{quadrupole moments}\index{moments!total quadrupole}.
This includes both the electronic and nuclear contributions to the
quadrupole moments. These will printed separately only if a print
level of 2 or higher has been chosen. Note that the quadrupole moment is
defined according to Buckingham \cite{adbacp12}. The quadrupole moment
is printed in the molecular input orientation as well as being
transformed to the principal moments of inertia coordinate system.

\item[\Key{RAMAN}] Calculates Raman intensities\index{Raman intensity}, as described in
Ref.~\cite{thkrklbpjjofd99}. This property needs a lot of settings
in order to perform correctly, and the reader is therefore referred to
Section~\ref{sec:vroa}, where the calculation of this property is described in more detail.

\item[\Key{REPS}] \verb| |\newline
\verb|READ (LUCMD, *) NREPS|\newline
\verb|READ (LUCMD, *) (IDOSYM(I),I = 1, NREPS)|

Consider perturbations of selected symmetries only.  Read one more
line specifying how many symmetries, then one line listing the
desired symmetries. This option is currently only implemented
for geometric perturbations.

%\item[\Key{RESTART}] Restart in the property evaluation section. This
%keyword is currently disabled.

\item[\Key{SELECT}]\verb| |
\newline
\verb|READ (LUCMD,*) NPERT|\newline
\verb|READ (LUCMD, *) (IPOINT(I),I=1,NPERT)|

Select which nuclear geometric perturbations are to be considered.
Read one more line specifying how many perturbations, then on a
new line the list of perturbations to be considered. By default,
all perturbations are to be considered, but by invoking this keyword,
only those perturbations specified in the sequence will be considered.

The perturbation ordering follows the ordering of the symmetrized
nuclear coordinates. This ordering can be obtained by setting the
print level in the \verb|*MOLBAS| module to 11 or higher.

\item[\Key{SECMOM}] Calculates the 9 cartesian molecular second order
moments\index{second order moments}\index{moments!total second order}.
This includes both the electronic and nuclear contribution to the
second order moment. These will printed separately only if a print
level of 2 or higher has been chosen.

\item[\Key{SHIELD}] Invokes the calculation of nuclear
shielding\index{nuclear shielding} constants. By default this is done
using London orbitals\index{London orbitals} in order to
ensure fast basis set convergence as shown in
Ref.~\cite{kwjfhppjacs112,krthrkpjklbhjajjcp100}. The use of London
orbitals can be disabled by the keyword \verb|.NOLOND|.

Furthermore, the natural connection\index{natural connection}
(Ref.~\cite{joklbkrthpjtca90,krthjopjklbcpl235}) is the default in order to ensure
numerically stable results as well as physically interpretable
results for the paramagnetic and diamagnetic terms. The natural
connection can be turned off by the keyword \verb|.NODIFC| in which
case the symmetric connection\index{symmetric connection} is used instead.

The gauge origin\index{gauge origin} is by default chosen to be the center of
mass\index{center of mass} of the molecule.
This origin can be changed by the two keywords \verb|.NOCMC | and \verb|.GAUGEO|
(NOTE: specification of \verb|.GAUGEO| in the \verb|**INTEGRALS| section
is \emph{not} used in this section, the \verb|**PROPERTIES| section).
This choice of gauge origin will not affect
the final shieldings if London orbitals are used, only the size of the
dia- and paramagnetic contributions.

Combined with the keyword \Key{SOPPA} or \Key{SOPPA(CCSD)} it invokes a SOPPA\index{SOPPA}
or SOPPA(CCSD)\index{SOPPA(CCSD)} calculation of the Nuclear Magnetic Shieldings
(Ref.~\cite{paololazz1,paololazz2,ctocd}). London orbitals are automatically disabled in
SOPPA\index{SOPPA} or SOPPA(CCSD)\index{SOPPA(CCSD)}
calculations. Gauge origin independent SOPPA or SOPPA(CCSD) calculations of Nuclear
Magnetic Shieldings can be carried out with the CTOCD-DZ method\index{CTOCD-DZ}
(see Refs.~\cite{paololazz1,paololazz2,ctocd}) using the keyword \Key{CTOCD}.

In combination with the keyword \Key{CTOCD}\index{CTOCD-DZ} this invokes a calculation of the
Nuclear Magnetic Shieldings without the use of London orbitals but with both
the CTOCD-DZ method (Ref.~\cite{paololazz1,paololazz2,ctocd}) and with the common
origin method. For the CTOCD-DZ method the Nuclear Magnetic Shieldings are given in
the output file for both the origin at the center of mass and at the respective atoms.
Changing the default value of the gauge origin could give wrong results!

\item[\Key{SOPPA}] Indicates that the requested molecular properties
be calculated using the second-order polarization-propagator
approximation~\cite{mjpekdtehjajjojcp}.\index{SOPPA} This requires that
the MP2 energy and wave function have been calculated. London orbitals
can not be used together with the SOPPA approximation. For details on
how to invoke an atomic integral direct SOPPA calculation
\cite{spas037} see chapters \ref{sec:AOsoppa} and \ref{sec:soppa}.

\item[\Key{SOPPA(CCSD)}] Indicates that the requested molecular properties
be calculated using the Second-Order Polarization-Propagator
Approximation with Coupled Cluster Singles and Doubles
Amplitudes~\cite{soppaccsd,tejospastcan100,ekdspasjpca102,ctocd}.\index{SOPPA(CCSD)}
This requires that the CCSD energy and wave function have been
calculated. London orbitals can not be used together with the
SOPPA(CCSD) approximation. For details on how to invoke an atomic
integral direct SOPPA(CCSD) calculation \cite{spas037, spas089} see
chapters \ref{sec:AOsoppa}
 and \ref{sec:soppa}.

\item[\Key{SPIN-R}] Invokes the calculation of
spin-rotation\index{spin-rotation constant}
constants as described in Ref.~\cite{jgkrthjcp105}. By default this is
done using London orbitals\index{London orbitals}  and the
natural connection\index{natural connection}. The use of London
orbitals can be turned off by the keyword \verb|.NOLOND|.

By definition the gauge origin\index{gauge origin} of the
spin-rotation constant is to be the
center of mass\index{center of mass} of the molecule, and although the
gauge origin can be
changed through the keywords \verb|.NOCMC | and \verb|.GAUGEO|, this
is not recommended, and may give erroneous results.

In the current implementation, symmetry dependent nuclei cannot be
used during the calculation of spin-rotation constants.

\item[\Key{SPIN-S}] Invokes the calculation of indirect nuclear
spin-spin coupling\index{spin-spin coupling} constants. By default all
spin-spin couplings
between nuclei with naturally occurring isotopes with abundance more
than 1\% and non-zero spin will be calculated, as well as all the different
contributions (Fermi contact, dia- and paramagnetic spin-orbit and
spin-dipole)\index{Fermi contact}\index{spin-dipole}\index{paramagnetic spin-orbit}\index{diamagnetic spin-orbit}. The implementation is described in
Ref.~\cite{ovhapjhjajsbpthjcp96}.

As this is a very time consuming property, it is recommended to
consult the chapter describing the calculation of NMR-parameters
(Ch.~\ref{ch:magnetic}). The main control of which
contributions and which nuclei to calculate spin-spin couplings
between is done in the \verb|*SPIN-S| module.

\item[\Key{THIRDM}] Calculates the 27 cartesian molecular third order
moments\index{third order moments}\index{moments!third order}.
This includes both the electronic and nuclear contribution to the
third order moments. These will printed separately only if a print
level of 2 or higher has been chosen.

\item[\Key{VCD}] Invokes the calculation of Vibrational Circular
Dichroism (VCD)\index{VCD}\index{vibrational circular dichroism}
according to the implementation described in
Ref.~\cite{klbpjthkrhjajjcp98}.  By default this is done using London
orbitals\index{London orbitals} in order to
ensure fast basis set convergence as shown in
Ref.~\cite{klbpjthkrhjajjcp100}. The use of London
orbitals can be disabled by the keyword \verb|.NOLOND|.

Furthermore, the natural connection\index{natural connection}
(Ref.~\cite{joklbkrthpjtca90,krthjopjklbcpl235}) is default in order to ensure
numerically stable results. The natural
connection can be turned off by the keyword \verb|.NODIFC| in which
case the symmetric connection\index{symmetric connection} will be used.

%The gauge origin\index{gauge origin} is chosen to be the center of
%mass\index{center of mass} of the molecule.
%This origin can be changed by the two keywords \verb|.GAUGEO| and
%\verb|.NOCMC |. This will of course not affect the final VCD results,
%only the size of the contributing terms.

In the current implementation, the keyword \Key{NOCMC} will be set
true in calculations of Vibrational Circular Dichroism, that is, the
coordinate system origin will be used as gauge origin. Changing this
default value will give incorrect results for VCD.

Note that in the current release, VCD is not implemented for Density
functional theory calculations, and the program will stop if VCD is
requested for a DFT calculation.

%\item[\Key{VERDET}] Requests the calculation of Verdet
%constants~\cite{mjpjarkrthcpl222}. London orbitals can not be used in
%these calculations.

\item[\Key{VIB\_G}] Invokes the calculation of the vibrational g factor,\index{vibrational g factor}
i.e. the non-adiabatic correction to the moment of inertia tensor for
molecular vibrations.\index{non-adiabatic corrections}\index{moment of
inertia tensor!non-adiabatic corrections}
This keyword has to be combined with the keyword \Key{SKIP} in the section \Sec{TROINV}.

\item[\Key{VIBANA}] Invokes a vibrational analysis\index{vibrational analysis} in the current
geometry. This will generate the vibrational frequencies in the
current point. If combined with \verb|.DIPGRA| the IR intensities
will be calculated as well\index{IR intensity}.

\item[\Key{VROA}] Invokes the calculation of Vibrational Raman
Optical Activity\index{Raman optical activity}\index{ROA}, as
described in Ref.~\cite{thkrklbpjjofd99}. This
property needs a lot of settings in order to perform correctly, and
the reader is therefore referred to Section~\ref{sec:vroa}, where the
calculation of this property is described in more detail.

\item[\Key{WRTINT}] Forces the magnetic first-derivate two-electron
integrals to be written to disc. This is default in MCSCF
calculations, but not for SCF runs. This file can be very large, and
it is not recommended to use this option for ordinary SCF runs.

\end{description}

\subsection{Calculation of Atomic Axial Tensors (AATs):
\Sec{AAT}}\label{sec:aat}

Directives for controlling the calculation of Atomic Axial
Tensors\index{atomic axial tensor}\index{AAT},
needed when calculating Vibrational Circular Dichroism
(VCD)\index{vibrational circular dichroism}\index{VCD}.
\begin{description}

\item[\Key{INTPRI}]\verb| |\newline
\verb|READ (LUCMD,*) INTPRI|

Set the print level in the calculation of the necessary differentiated
integrals when calculating Atomic Axial Tensors\index{atomic axial tensor}\index{AAT}. Read one more line
containing print level. Default value is value of \verb|IPRDEF|
from the general input module. The print level of the rest of the
calculation of Atomic Axial Tensors are controlled by the keyword
\verb|.PRINT |.

\item[\Key{NODBDR}] Skip contributions originating from first
half-differentiated overlap\index{overlap!half-differentiated}
integrals with respect to both nuclear
distortions as well as magnetic field. This will give wrong results
for VCD\index{VCD}\index{vibrational circular dichroism}. Mainly for debugging purposes.

\item[\Key{NODDY}] Checks the calculation of the electronic part of
the Atomic Axial Tensors\index{atomic axial tensor}\index{AAT} by calculating these both in the ordinary
fashion as well as by a noddy routine. The program will not
perform a comparison, and will not abort if differences is found.
Mainly for debugging purposes.

\item[\Key{NOELC}] Skip the calculation of the pure electronic
contribution to the Atomic Axial Tensors\index{atomic axial tensor}\index{AAT}. This will give wrong results
for VCD\index{VCD}\index{vibrational circular dichroism}. Mainly for debugging purposes.

\item[\Key{NONUC}] Skip the calculation of the pure nuclear
contribution to the Atomic Axial Tensors\index{atomic axial tensor}\index{AAT}. This will give wrong results
for VCD. Mainly for debugging purposes.

\item[\Key{NOSEC}] Skip the calculation of second order orbital
contributions to the Atomic Axial Tensors\index{atomic axial tensor}\index{AAT}. This will give wrong
results for VCD\index{VCD}\index{vibrational circular dichroism}. Mainly for debugging purposes.

\item[\Key{PRINT}]\verb| |\newline
\verb|READ (LUCMD,*) IPRINT|

Set print level in the calculation of Atomic Axial Tensors\index{atomic axial tensor}\index{AAT} (this does
not include the print level in the integral calculation, which are
controlled by the keyword \verb|.INTPRI|). Read one
more line containing print level. Default value is the value of
\verb|IPRDEF| from the general input module.

\item[\Key{SKIP}] Skips the calculation of Atomic Axial Tensors\index{atomic axial tensor}\index{AAT}.
This will give wrong results for VCD\index{VCD}\index{vibrational circular dichroism}, but may be of interest for
debugging purposes.

\item[\Key{STOP}] Stops the entire calculation after finishing the
calculation of the Atomic Axial Tensors\index{atomic axial tensor}\index{AAT}. Mainly for debugging purposes.
\end{description}

\subsection{Freguency-dependent linear response calculations: \Sec{ABALNR}}\label{sec:abalnr}

Directives to control the calculation of frequency dependent linear
response\index{linear response}\index{response!linear}
functions.
%At present these directives only affect the
%calculation of frequency dependent linear response functions appearing
%in connection with Vibrational Raman Optical Activity
%(ROA)\index{Raman optical activity}\index{ROA}.

\begin{description}

\item[\Key{FREQUE}]\verb| |\newline
\verb|READ (LUCMD,*) NFRVAL|\newline
\verb|READ (LUCMD,*) (FRVAL(I), I = 1, NFRVAL)|

Set the number of frequencies as well as the
frequency\index{frequency!linear response} at which the
frequency-dependent linear response equations are to be evaluated.
Read one more line containing the number of frequencies to be
calculated, and another line reading these frequencies. The
frequencies are to be entered in atomic units. By default only the
static case is evaluated.
The \Key{FREQUE} keyword may be combined with the wave length input
\Key{WAVELE} (see below).

\item[\Key{DAMPING}]\verb| |\newline
\verb|READ (LUCMD,*) ABS_DAMP|

Sets the lifetime of the excited states if absorption is also included
in the calculation of the linear response functions as described in
Ref.~\cite{pndmbhjajjojcp115,pnkrthjcp120}. The default is that no
absorption is included in the calculation. The lifetime is given in
atomic units. By default the algorithm with symmetrized trial vectors is
used \cite{kauczor:2011}.

\item[\Key{OLDCPP}]\verb| |\newline
If absorption is included in the calculation of the linear response
functions, the complex polarization propagator
solver \cite{pndmbhjajjojcp123,pndmbhjajjojcp115} 
is used to solve damped response equations. \Key{OLDCPP} requires that
\Key{DAMPING} is specified.

\item[\Key{MAX IT}]\verb| |\newline
\verb|READ (LUCMD,*) MAXITE|

Set the maximum number of micro iterations in the iterative solution of
the frequency-dependent linear response functions. Read one more line
containing maximum number of micro iterations. Default value is
60.

\item[\Key{MAXPHP}]\verb| |\newline
\verb|READ (LUCMD,*) MXPHP|

Set the maximum dimension for the sub-block of the configuration
Hessian that will be explicitly inverted. Read one more line
containing maximum dimension. Default value is~0.

\item[\Key{MAXRED}]\verb| |\newline
\verb|READ (LUCMD,*) MXRM|

Set the maximum dimension of the reduced space to which new basis
vectors are added as described in Ref.~\cite{tuhjahjajpjjcp84}. Read
one more line containing maximum dimension. Default value is~400.

\item[\Key{OPTORB}] Use optimal orbital trial vectors\index{optimal orbital trial vector} in the
iterative solution of the frequency-depen\-dent linear
response\index{linear response}\index{response!linear}
equations. These are generated as described in
Ref.~\cite{tuhjahjajpjjcp84} by solving the orbital response equation
exact, keeping the configuration part fixed.

\item[\Key{PRINT}]\verb| |\newline
\verb|READ (LUCMD,*) IPRLNR|

Set the print level in the calculation of frequency-dependent linear
response properties. Read one more line containing the print level.
The default value is the value of \verb|IPRDEF| from the general input
module.

\item[\Key{SKIP}] Skip the calculation of the frequency-dependent
response functions. This will give wrong results for ROA. Mainly for
debugging purposes.

\item[\Key{STOP}] Stops the program after finishing the
calculation of the frequency-dependent linear response equations. Mainly
for debugging purposes.

\item[\Key{THRESH}]\verb| |\newline
\verb|READ (LUCMD,*) THCLNR|

Set the convergence threshold for the solution
of the frequency dependent response equations. Read one more line
containing the convergence threshold~(D12.6). The default value is
$5.0\cdot10^{-5}$.

\item[\Key{WAVELE}]\verb| |\newline
\verb|READ (LUCMD,*) NWVLEN|\newline
\verb|READ (LUCMD,*) (WVLEN(I), I = 1, NWVLEN)|

Set the number of wave lengths as well as the wave
lengths\index{wave lengths!linear response} at which the
frequency-dependent linear response equations are to be evaluated.
Read one more line containing the number of wave lengths to be
calculated, and another line reading these wave lengths. The
wave lengths are to be entered in units of nanometers (nm).
By default only the
static case (infinite wavelength, zero frequency) is evaluated.
The \Key{WAVELE} keyword may be combined with the frequency input
\Key{FREQUE} (see above).
\end{description}

\subsection{Dipole moment and dipole gradient contributions:
\Sec{DIPCTL}}\label{sec:dipctl}

Directives controlling the calculation of contributions to the
dipole gradient\index{dipole gradient} appear in the \verb|*DIPCTL| section.

\begin{description}
\item[\Key{NODC}] Neglect contributions to traces from
inactive one-electron density matrix. This will give wrong results for
the dipole gradient. Mainly for debugging purposes.

\item[\Key{NODV}] Neglect contributions to traces from
active one-electron density matrix. This will give wrong results for the
dipole gradient. Mainly for debugging purposes.

\item[\Key{PRINT}]\verb| |\newline
\verb|READ (LUCMD,*) IPRINT|

Set print level in the calculation of the dipole gradient\index{dipole gradient}.  Read one more
line containing print level. The default
is the variable \verb|IPRDEF| from the general input module.

\item[\Key{SKIP}] Skip the calculation of dipole gradient.

%\item[\Key{TEST}] Test dipole moments and dipole
%reorthonormalization(?) through a dummy routine. This test routine is
%currently only implemented for the symmetric connection, and must thus
%only be used together with the \verb|.NODIFC| keyword.

\item[\Key{STOP}] Stop the program after finishing the calculation
of the dipole gradient. Mainly for debugging purposes.
\end{description}

%hjaaj: do not document *END OF as this is an obsolete keyword,
%       only kept for backwards compatibility /July 2005
%\subsection{End of input: \Sec{END OF}}
%
%The last directive in the input can be \verb|*END OF|.

\subsection{Calculation of excitation energies: \Sec{EXCITA}}
\label{sec:excita}

Directives to control the calculations of electronic
transition\index{electronic excitation}
properties and excitation energies\index{excitation energy} appear in
the \verb|*EXCITA| input module.
For SCF\index{SCF}\index{HF}\index{Hartree--Fock} wave
functions the properties are calculated using the
random phase approximation (RPA) and for MCSCF\index{MCSCF}
wave functions the multiconfigurational (MC-RPA) is used.
In the case of Kohn--Sham DFT, time-dependent linear response theory
is used in the adiabatic approximation to the functional kernel.

Implemented electronic transition properties are at the moment:

\begin{enumerate}
\item Excitation Energies
\index{electronic excitation}\index{excitation energy!electronic}.
These are always calculated when
invoking the \verb|.EXCITA| keyword in the general input module.
\item Oscillator Strengths\index{oscillator strength} which determine
intensities in visible and UV absorption.
\item Rotatory Strengths\index{rotatory strength} which determine
Electronic Circular Dichroism\index{electronic circular dichroism}\index{ECD}
(ECD).
\end{enumerate}

\begin{description}
\item[\Key{DIPSTR}] Calculates the dipole strengths\index{dipole strength},
that is, the dipole oscillator strengths\index{oscillator strength}
which determine the visible and UV absorption, using the dipole length form.

\item[\Key{FNAC}] Calculate first-order non-adiabatic coupling
matrix\index{non-adiabatic coupling matrix element} elements
from the reference state to the states requested with \Key{NEXCIT}.
The keyword \Key{NACME} in the parent section to \Sec{EXCITA}
and the keyword \Key{SKIP} in section \Sec{TROINV}
must also be specified to get the coupling elements.


\item[\Key{INTPRI}]\verb| |\newline
\verb|READ (LUCMD, *) IPRINT|

Set the print level in the calculation of the necessary differentiated
integrals when calculating the linear response functions. Read one
more line containing print level. Default value is the value of
\verb|IPRDEF| from the general input module. The print level of the
rest of the calculation of electronic excitation energies are
controlled by the keyword \verb|.PRINT |.

\item[\Key{MAX IT}]\verb| |\newline
\verb|READ (LUCMD,*) MAXITE|

Set the maximum number of micro iterations in the iterative
solution of the linear response equations. Read
one more line containing maximum number of micro iterations.
Default value is 60.

\item[\Key{MAXPHP}]\verb| |\newline
\verb|READ (LUCMD,*) MXPHP|

Set the maximum dimension for the sub-block of the configuration
Hessian that will be explicitly inverted. Read one more line
containing maximum dimension. Default value is~0.

\item[\Key{MAXRED}]\verb| |\newline
\verb|READ (LUCMD,*) MXRM|

Set the maximum dimension of the reduced space to which new basis
vectors are added as described in Ref.~\cite{tuhjahjajpjjcp84}. Read
one more line containing maximum dimension. Default value is~400.

\item[\Key{NEXCIT}]\verb| |\newline
\verb|READ (LUCMD,*) (NEXCIT(I), I= 1,NSYM)|

Set the number of excitation energies\index{excitation energy} to be
calculated in each
symmetry. Read one more line containing the number of excitations in
each of the irreducible representations of the molecular point group.
The default is not to calculate one excitation energy in each of the
irreducible representations.

\item[\Key{OPTORB}] Use optimal orbital trial vectors\index{optimal
orbital trial vector} in the
iterative solution of the eigenvalue equations
in order to speed up the calculation.
Only relevant for MCSCF.
These are generated by solving the orbital response equation
exact, keeping the configuration part fixed as described in
Ref.~\cite{tuhjahjajpjjcp84}.

\item[\Key{PRINT}]\verb| |\newline
\verb|READ (LUCMD,*) IPREXE|

Set the print level in the calculation of electronic excitation
energies. Read one more line containing the print level.
The default value is the \verb|IPRDEF| from the general input module.

\item[\Key{ROTVEL}] Calculate rotational strengths\index{electronic
circular dichroism}\index{ECD} in Electronic
Circular Dichroism (ECD) without using London orbitals.

\item[\Key{SKIP}] Skip the calculation of electronic excitation
energies. This will give wrong results for ECD.
Mainly for debugging purposes.

\item[\Key{STOP}] Stops the program after finishing the
calculation of the eigenvalue equations.
Mainly for debugging purposes.

\item[\Key{SUMRUL}] Calculate oscillator strength sum rules from
the calculated excitation energies and dipole oscillator strengths.
Accurate results require to calculate all excitation energies supported
by the one-electron basis set.

\item[\Key{THRESH}]\verb| |\newline
\verb|READ (LUCMD,*) THREXC|

Set the convergence threshold for the solution
of the linear response equations. Read one more line
containing the convergence threshold. The default value is
$1\cdot10^{-4}$.

\item[\Key{TRIPLET}]
\index{excitation energies!triplet}
Indicates that it is triplet excitation energies that is to be
calculated instead of the default singlet excitation energies.
\end{description}

\subsection{One-electron expectation values:
\Sec{EXPECT}}\label{sec:expect}

Directive that control the calculation of one-electron expectation
values appear in the \verb|*EXPECT| input module. Notice, however,
that the directives controlling the calculation of one-electron
expectation values needed for the molecular gradient and Hessian
appear in the \verb|*ONEINT| section.

\begin{description}
\item[\Key{ALL CO}] Indicates that all components of the expectation
  value contributions to the nuclear
shielding\index{nuclear shielding} or indirect spin--spin
  coupling\index{diamagnetic spin-orbit}
  tensors are to be calculated at the
same time. This is the
default for ordinary calculations. However, in direct and parallel
calculations on large molecules this may give too large memory
requirements, and instead only the components of one symmetry-independent
nucleus are calculated at a time. However, by invoking
this keyword, all components are calculated simultaneously even in
direct/parallel calculations.

\item[\Key{DIASUS}] Invokes the calculation of the one-electron
contribution to the magnetizability\index{magnetizability} expectation
value. By default this
is done using London atomic\index{London orbitals} orbitals. Default
value is \verb|TRUE| if
magnetizability has been requested in the general input module,
otherwise \verb|FALSE|.

\item[\Key{ELFGRA}] Invokes the calculation of the electronic
contribution to the nuclear quadrupole moment coupling
tensor\index{nuclear quadrupole coupling}\index{NQCC} (that
is, the electric field
\index{electric field!gradient}
gradient). Default value is \verb|TRUE| if
nuclear quadrupole coupling constants have been requested in the
general input module, otherwise \verb|FALSE|.

\item[\Key{NODC}] Do not calculate contributions from the inactive
one-electron density matrix. This will give wrong results for the
one-electron expectation values. Mainly for debugging purposes.

\item[\Key{NODV}] Do not calculate contributions from the active
one-electron density matrix. This will give wrong results for the
one-electron expectation values. Mainly for debugging purposes.

\item[\Key{NEFIEL}] Invokes the evaluation of the electric
field at the individual nuclei\index{electric field!at nucleus}. Default
value is \verb|TRUE| if
spin-rotation\index{spin-rotation constant} constants have been
requested in the general input
module, otherwise \verb|FALSE|. In the current implementation,
symmetry dependent nuclei cannot be used when calculating this property.

\item[\Key{POINTS}]\verb| |\newline
\verb|READ (LUCMD,*) NPOINT|

Set the number of integration points to be used in the Gaussian
quadrature\index{Gaussian quadrature}
when evaluating  the diamagnetic spin-orbit\index{diamagnetic
spin-orbit} integrals. Default value is 40.

\item[\Key{PRINT}]\verb| |\newline
\verb|READ (LUCMD,*) MPRINT|

Set print level in the calculation of one-electron expectation values.
Read one more line containing print level. Default value is the
value of \verb|IPRDEF| from the general input module.

\item[\Key{QUADRU}] Calculates the electronic contribution to the
molecular (traceless) quadrupole moments\index{quadrupole
moments}\index{moments!electronic quadrupole}. Default value is \verb|TRUE|
if molecular quadrupole moment has been requested in the general input
module, otherwise \verb|FALSE|.

\item[\Key{SHIELD}] Invokes the calculation of the one-electron
contribution to the nuclear shielding\index{nuclear shielding}
expectation values. By default
this is done using London atomic\index{London orbitals}
orbitals. Default value is
\verb|TRUE| if nuclear shieldings have been requested in the general
input module, otherwise \verb|FALSE|.

\item[\Key{SKIP}] Skip the calculation of one-electron expectation
values. This may give wrong final results for some properties. Mainly
for debugging purposes.

\item[\Key{SPIN-S}] Invokes the calculation of the diamagnetic
spin-orbital\index{diamagnetic spin-orbit} integral, which is the
diamagnetic contribution to
indirect nuclear spin-spin coupling\index{spin-spin coupling}
constants. Default value is
\verb|TRUE| if spin-spin couplings have been requested in the general
input module, otherwise \verb|FALSE|.

\item[\Key{STOP}] Stop the entire calculation after finishing the
calculation of one-electron expectation values. Mainly for debugging
purposes.

\end{description}

%\subsection{Floating orbitals: \Sec{FLOAT}}\label{sec:float}
%
%Directives that control the calculation when using floating orbitals
%as described by Helgaker and Alml{\"o}f \cite{thjajcp89} appear in
%the \verb|*FLOAT| input module.
%
%\begin{description}
%\item{\verb|.PRINT |}\verb| |\newline
%\verb|READ (LUCMD,'(I5)') IPRINT|
%
%Set the print level in the calculation of floating orbital
%contributions. Read one more line containing the print level~(I5).
%Default print level is the \verb|IPRDEF| variable from the general
%input module.
%
%\item{\verb|.RESPON|} Print the response contributions from
%the floating orbitals.
%
%\item{\verb|.SKIP  |} Skip the calculation of specific terms
%contributing when using floating orbitals. This may give wrong results
%when using floating orbitals. Mainly for debugging purposes.
%
%\item{\verb|.STOP  |} Stop the calculation after calculating the
%response contributions from the floating orbitals. Mainly for
%debugging purposes.
%\end{description}

\subsection{Geometry analysis: \Sec{GEOANA}}

Directives controlling the calculation and printing of bond angles\index{bond distance}\index{bond angle}\index{dihedral angle}\index{geometry!bond distance}\index{geometry!bond angle}\index{geometry!dihedral angle}
and dihedral angles appear in the \verb|*GEOANA| section. The program will also define atoms
to be bonded to each other depending on their bond distance. For all atoms
defined to be bonded to each other, the bond distance and bond angles
will be printed.

\begin{description}
\item[\Key{ANGLES}]\verb| |\newline
\verb|READ (LUCMD,*) NANG|\newline
\verb|DO  I = 1, NANG|\newline
\verb|   READ (LUCMD,*) (IANG(J,I), J = 1,3)|\newline
\verb|END DO|

Calculate and print
bond angles\index{bond angle}\index{geometry!bond angle}. Read one
more line specifying the number of angles, and then read \verb|NANG|
lines containing triplets $A,B,C$ of atom labels, each
specifying a particular bond angle~$\angle ABC$. Notice that in the
current version of the program there is an upper limit of 20 bond
angles that will be printed. The rest will be ignored. We also note
that program always will print the angles between atoms defined to be
bonded to each other on the basis of the van der Waals\index{van der
Waals radius} radii of the atoms.

\item[\Key{DIHEDR}]\verb| |\newline
\verb|READ (LUCMD,*) NDIHED|\newline
\verb|DO  I = 1, NDIHED|\newline
\verb|   READ (LUCMD,*) (IDIHED(J,I), J = 1,4)|\newline
\verb|END DO|

Calculate and print dihedral\index{dihedral angle}\index{geometry!dihedral angle}
(torsional) angles.  Read one more line specifying the number of angles,
and then read \verb|NDIHED| lines containing quadruplets $A,B,C,D$ of atom
labels.  The angle computed is that between the planes~$ABC$
and $BCD$. Notice that in the current version of the program there is
an upper limit of 20 dihedral angles that will be printed. The rest
will be ignored.

\item[\Key{SKIP}] Skip the geometry analysis, with the exceptions
mentioned in the introduction to this section. This is the default
value, but it is overwritten by the keywords \verb|.ANGLES| and
\verb|.DIHEDR|.
\end{description}

\subsection{Right-hand sides for geometry response equations: \Sec{RHSIDE}}

Directives affecting the construction of the right-hand
sides~(RHS)\index{property gradient}\index{right-hand side}---that is,
wave function gradient terms---for the geometric derivative response
calculations as well as some matrices needed for reorthonormalization
contributions appear in the \verb|*RHSIDE| section.

\begin{description}
\item[\Key{ALLCOM}] Requests that all paramagnetic
spin-orbit\index{paramagnetic spin-orbit}
right-hand sides are to be calculated in one batch, and not for each
symmetry-independent center at a time which is the default. This will
slightly speed up the calculation, at the cost of significantly larger
memory requirements.

\item[\Key{FCKPRI}]\verb| |\newline
\verb|READ (LUCMD,*) IPRFCK|

Set print level for the calculation of derivative Fock matrices.  Read
one more line specifying print level. The default  is the value of
\verb|IPRDEF| in the general input module.

\item[\Key{FCKSKI}] Skip the derivative Fock matrix contributions
to the right-hand sides. This will give wrong results for all
properties depending on right hand sides. Mainly for debugging purposes.

\item[\Key{FCKTES}] Test the Fock matrices. Mainly for debugging
purposes.

\item[\Key{FSTTES}] Test one-index transformation of derivative
Fock matrices.

\item[\Key{GDHAM}] Write out differentiated Hamiltonian and
differentiated Fock matrices to file for use in post-\dalton\ programs.

\item[\Key{GDYPRI}]\verb| |\newline
\verb|READ (LUCMD,*) IPRGDY|

Set print level for the calculation of the Y-matrix appearing in the
reorthonormalization terms, as for instance in
Ref.~\cite{tuhjahjajpjjcp84}. Default  is the value of \verb|IPRALL|
defined by the \verb|.PRINT | keyword. If
this has not been specified, the default is the value of \verb|IPRDEF|
from the general input section.

\item[\Key{GDYSKI}] Skip the calculation of the lowest-order
reorthonormalization contributions to the second-order molecular
properties. This will give wrong results for these properties. Mainly
for debugging purposes.

\item[\Key{INTPRI}]\verb| |\newline
\verb|READ (LUCMD, *) IPRINT, IPRNTA, IPRNTB, IPRNTC, IPRNTD|

Set print level for the derivative integral calculation for a particular shell
quadruplet.  Read one more line containing print level and the four
shell indices.  The print level is changed from the default
for this quadruplet only. Default value is the value of \verb|IPRDEF|
from the general input module. Note that the print level of all shell
quadruplets can be changed by the keyword \verb|.PRINT |.

\item[\Key{INTSKI}] Skip the calculation of derivative integrals.
This will give wrong results for the total molecular Hessian. Mainly
for debugging purposes.

\item[\Key{NODC}] Do not calculate contributions from the inactive
one-electron density matrix. This will give wrong results for the
total molecular property. Mainly for debugging purposes.

\item[\Key{NODDY}] Test the orbital part of the right-hand side.
The run will not be aborted. Mainly for debugging purposes.

\item[\Key{NODPTR}] The transformation of the two-electron density
matrix is back-transformed to atomic orbital basis using a
noddy-routine for comparison.

\item[\Key{NODV}] Do not calculate contributions from the active
one-electron density matrix. This will give wrong results for the
molecular property. Mainly for debugging purposes.

\item[\Key{NOFD}] Do not calculate the contribution from the
differentiated Fock-matrices to the total right-hand side. This will
give wrong results for the requested molecular property. Mainly for
debugging purposes.

\item[\Key{NOFS}] Do not calculate the contribution to the total
right-hand side from the one-index transformed Fock-matrices with the
differentiated connection matrix. This will give wrong results for
the requested molecular property. Mainly for debugging purposes.

\item[\Key{NOH1}] Do not calculate the contribution from the
one-electron terms to the total right-hand side. This will give wrong
results for the requested property. Mainly for debugging purposes.

\item[\Key{NOH2}] Do not calculate the contribution from the
two-electron terms to the total right-hand side. This will give wrong
results for the requested molecular property. Mainly for debugging
purposes.

\item[\Key{NOORTH}] Do not calculate the orbital reorthonormalization
contribution (the one-index transformed contributions) to the total
right-hand side. This will give wrong results for the requested
molecular property. Mainly for debugging purposes.

\item[\Key{NOPV}] Do not calculate contributions from the two-electron
density matrix. This will give wrong results for the requested
molecular property. Mainly for debugging purposes.

\item[\Key{NOSSF}] Do not calculate the contribution to the total
right-hand side from the double-one-index
transformation between the differentiated connection matrix and the
Fock-matrix. This option will only affect the calculation of the molecular
Hessian, and will give a wrong result for this. Mainly for debugging
purposes.

\item[\Key{PRINT}]\verb| |\newline
\verb|READ (LUCMD,) IPRALL|

Set print levels.  Read one more line containing the print level for
this part of the calculation.  This will be the default print
level in the calculation of differentiated two-electron integrals,
differentiated Fock-matrices, derivative
overlap matrices, two-electron density and derivative integral
transformation, as well as in the construction of the right-hand sides.
To set the print level in each of these parts individually, see the
keywords \verb|.FCKPRI|, \verb|.GDYPRI|, \verb|.INTPRI|,
\verb|.PTRPRI| and \verb|.TRAPRI|.

\item[\Key{PTRPRI}]\verb| |\newline
\verb|READ (LUCMD,) IPRTRA|

Set print level for the  two-electron densities transformation. Read
one more line containing print level.
Default value is the value of  \verb|IPRDEF| from the general input
module. Note also that this print level is also controlled by the keyword
\verb|.PRINT |.

\item[\Key{PTRSKI}] Skip transformation of active two-electron
density matrix. This will give wrong results for the total
second-order molecular property. Mainly for debugging purposes.

\item[\Key{RETURN}] Stop after the shell quadruplet specified
under \verb|.INTPRI| above. Mainly for debugging purposes.

\item[\Key{SDRPRI}]\verb| |\newline
\verb|READ (LUCMD,) IPRSDR|

Set the print level in the calculation of the differentiated connection
matrix. Read one more line containing the print level. Default
value is the value given by the keyword \verb|.PRINT |. If this
keyword has not been given, the default is the value of \verb|IPRDEF|
given in the general input module.

\item[\Key{SDRSKI}] Do not calculate the differentiated connection
matrices. This will give wrong results for properties calculated with
perturbation dependent basis sets. Mainly for debugging purposes.

\item[\Key{SDRTES}] The differentiated connection matrices will be
transformed and printed in atomic orbital basis. Mainly for debugging
purposes.

\item[\Key{SIRPR4}]\verb| |\newline
\verb|READ (LUCMD, *) IPRI4|

\sir\  ``output unit~4'' print level.  Read one more line specifying
print level. Default is~0.

\item[\Key{SIRPR6}]\verb| |\newline
\verb|READ (LUCMD, *) IPRI6|

\sir\ ``output unit~6'' print level.  Read one more line specifying
print level. Default is~0.

\item[\Key{SKIP}] Skip the calculation of right-hand sides. This
will give wrong values for the requested second-order properties.
Mainly for debugging purposes.

\item[\Key{SORPRI}]\verb| |\newline
\verb|READ (LUCMD,*) IPRSOR|

Set print level for the two-electron density matrix sorting. Read one
more line containing print level. Default value is the value of
\verb|IPRDEF| from the general input module.

\item[\Key{STOP}] Stop the entire calculation after finishing
the construction of the right-hand side. Mainly for debugging purposes.

\item[\Key{TIME}] Provide detailed timing breakdown for the
two-electron integral calculation.

\item[\Key{TRAPRI}]\verb| |\newline
\verb|READ (LUCMD,*) IPRTRA|

Set print level for the derivative integrals transformation.  Read one more
line specifying print level. Default is the value of
\verb|IPRDEF| from the general input module. Notice that the default print
level is also affect by the keyword \verb|.PRINT |.

\item[\Key{TRASKI}] Skip transformation of derivative integrals.
Mainly for debugging purposes.

\item[\Key{TRATES}] Testing of derivative integral
transformation. The calculation will not be aborted. Mainly for
debugging purposes.
\end{description}

\subsection{Linear response for static singlet property operators:
\Sec{LINRES}}\label{sec:linres}

Directives to control the calculation of frequency-independent linear
response functions\index{linear response}\index{response!linear}. At
present these directives only affect the
calculation of frequency-independent linear response functions appearing
in connection with singlet, magnetic imaginary perturbations.

\begin{description}
\item[\Key{MAX IT}]\verb| |\newline
\verb|READ (LUCMD,*) MAXITE|

Set the maximum number of micro iterations in the iterative solution of
the frequency independent linear response functions. Read one more line
containing maximum number of micro iterations. Default value is
60.

\item[\Key{MAXPHP}]\verb| |\newline
\verb|READ (LUCMD,*) MXPHP|

Set the maximum dimension of the sub-block of the configuration
Hessian that will be explicitly inverted. Read one more line
containing maximum dimension. Default value is~0.

\item[\Key{MAXRED}]\verb| |\newline
\verb|READ (LUCMD,*) MXRM|

Set the maximum dimension of the reduced space to which new basis
vectors are added as described in Ref.~\cite{tuhjahjajpjjcp84}. Read
one more line containing maximum dimension. Default value is~400.

\item[\Key{OPTORB}] Use optimal orbital trial vectors in
the\index{optimal orbital trial vector}
iterative solution of the frequency-inde\-pen\-dent linear response
equations. These are generate by solving the orbital response equation
exact, keeping the configuration part fixed as described in
Ref.~\cite{tuhjahjajpjjcp84}.

\item[\Key{PRINT}]\verb| |\newline
\verb|READ (LUCMD,*) IPRCLC|

Set the print level in the solution of the magnetic
frequency-independent linear response equations. Read one more line
containing print level. Default is the value of \verb|IPRDEF| in
the general input module.

\item[\Key{SKIP}] Skip the calculation of the frequency-independent
response functions. This will give wrong results for shielding,
magnetizabilities, optical rotation, VCD, VROA and spin-spin coupling constants\index{nuclear
shielding}\index{spin-spin coupling}\index{magnetizability}\index{optical rotation}\index{vcd}\index{vroa}. Mainly for
debugging purposes.

\item[\Key{STOP}] Stops the program after finishing the
calculation of the frequency-independent linear response equations. Mainly
for debugging purposes.

\item[\Key{THRESH}]\verb| |\newline
\verb|READ (LUCMD,*) THRCLC|

Set the convergence threshold for the solution
of the frequency-independent response equations. Read one more line
containing the convergence threshold. The default value is
$1.0\cdot10^{-4}$ for calculations which cannot take advantage of Sellers
formula for quadratic errors in the response
property~\cite{hsijqc30}, and $2.0\cdot10^{-3}$ for those calculations
that can.
\end{description}

\subsection{Localization of molecular orbitals: \Sec{LOCALI}}\label{sec:locali}

Directives to control the generation of localized orbitals for the use
in the analysis of second order / linear response properties in
localized molecular orbitals. At present these directives only affect
the calculation of spin-spin coupling constants. Naturally, the
generation of localized molecular orbitals requires that the use of
point group symmetry is turned off.

\begin{description}
\item [\Key{FOSBOY}] The occupied molecular orbitals are localized with
the Foster-Boys localization procedure~\cite{Boyloc}. It requires the
\Key{SOSOCC} keyword in the \Sec{SPIN-S} section.

\item [\Key{FBOCIN}]\verb| |\newline
 \verb|READ (LUCMD, * ) NO2LOC|\newline
 \verb|READ (LUCMD, * ) ( NTOC2L(I), I = 1, NO2LOC)|\newline
All occupied molecular orbitals are localized with the Foster-Boys
localization procedure~\cite{Boyloc}. Afterwards \verb|NO2LOC| occupied
orbitals are delocalized again. \verb|NTOC2L|  are the indices of the
occupied orbitals which are delocalized again.

\item [\Key{FBOOCC}]\verb| |\newline
 \verb|READ (LUCMD, * )NO2LOC|\newline
 \verb|READ (LUCMD, * ) ( NTOC2L(I), I = 1, NO2LOC )|\newline
A subset of occupied molecular orbitals are localized with the
Foster-Boys localization procedure~\cite{Boyloc}. \verb|NO2LOC|
occupied molecular orbitals are not localized. \verb|NTOC2L|  are the
indices of the occupied orbitals which are not localized, but remain in
canonical form.

\item [\Key{FBOVIR}] The whole set of virtual molecular orbitals is localized with
the Foster-Boys localization procedure~\cite{Boyloc}. The virtual
orbitals are paired with occupied orbitals. First one virtual orbital
is paired with each occupied orbital. Afterwards additional sets of
localized virtual orbitals are generated which are again paired with
one occupied orbital each and which are orthogonalized to the already
existing localized virtual orbitals. This is repeated until all virtual
orbital are localized and paired to occupied orbitals. It requires the
\Key{SOSOCC} keyword in the \Sec{SPIN-S} section and the \Key{FOSBOY},
\Key{FBOCIN} or \Key{FBOOCC} keyword in the \Sec{LOCALI} section.


\item [\Key{FBSETV}]\verb| |\newline
 \verb|READ (LUCMD, *) NFBSET|\newline
\verb|NFBSET| sets of virtual orbitals are localized with the
Foster-Boys localization procedure~\cite{Boyloc}. A set of virtual
orbitals consists of as many virtual orbitals as there are occupied
orbitals. It requires the \Key{SOSOCC} keyword in the \Sec{SPIN-S}
section and the \Key{FOSBOY}, \Key{FBOCIN} or \Key{FBOOCC} keyword in
the \Sec{LOCALI} section.

\item [\Key{FBSTVO}]\verb| |\newline
 \verb|READ(LUCMD, * ) NFBSET, NV2LOC|\newline
 \verb|READ(LUCMD, * ) ( NOCVI(I), I = 1, NV2LOC ) |\newline
Similar to \Key{FBSETV}, but localizes only \verb|NFBSET| sets of
virtual orbitals for a subset of \verb|NV2LOC| occupied orbitals. In
total \verb|NFBSET*NV2LOC| localized virtual orbitals will be
generated. \verb|NOCVI| are the indices of the occupied orbitals with
which the virtual orbitals are paired. It requires the \Key{SOSOCC}
keyword in the \Sec{SPIN-S} section and the \Key{FOSBOY}, \Key{FBOCIN}
or \Key{FBOOCC} keyword in the \Sec{LOCALI} section.

\item [\Key{LABOCC}]\verb| |\newline
 \verb|READ (LUCMD, * ) NOCLAB|\newline
 \verb|READ (LUCMD, * ) (TABOCL(I), I = 1, NOCLAB )|\newline
Allows one to add some labels to the occupied orbitals which are
localized. Up to 20 labels of up to 8 characters can be added. It
requires the \Key{FOSBOY}, \Key{FBOCIN} or \Key{FBOOCC} 
keyword in the \Sec{LOCALI} section.

\item [\Key{LABVIR}]\verb| |\newline
 \verb|READ (LUCMD, * ) NVILAB|\newline
 \verb|READ (LUCMD, * ) ( TABVIL(I), I = 1, NVILAB )|\newline
Allows one to add some labels to the virtual orbitals which are
localized. Up to 20 labels of up to 8 characters can be added. It
requires the \Key{FBOVIR}, \Key{FBSETV} or \Key{FBSTVO} keyword in the
\Sec{LOCALI} section.

\end{description}



\subsection{Nuclear contributions: \Sec{NUCREP}}

Directives affecting the nuclear contribution to the molecular
gradient\index{molecular gradient} and molecular Hessian\index{molecular Hessian}
calculation appear in the
\verb|*NUCREP| section.
\begin{description}
\item[\Key{PRINT}]\verb| |\newline
\verb|READ (LUCMD,*) IPRINT|

Set the print level in the calculation of the nuclear contributions.
Read one more line containing print level. Default value is the
value of \verb|IPRDEF| from the general input module.

\item[\Key{SKIP}] Skip the calculation of the nuclear
contribution. This will give wrong
results for the total molecular gradient and Hessian. Mainly for
debugging purposes.

\item[\Key{STOP}] Stop the program after finishing the calculation
of the nuclear contributions. Mainly for debugging purposes.
\end{description}

\subsection{One-electron integrals: \Sec{ONEINT}}

Directives affecting the calculation of one-electron integral contributions in the
calculation of molecular gradients\index{molecular gradient} and molecular
Hessians\index{molecular Hessian} appear in the \verb|*ONEINT| section.

\begin{description}
\item[\Key{NCLONE}] Calculate only the classical contributions to the
nuclear-attraction integrals.

\item[\Key{NODC}] Do not calculate contributions from the inactive
one-electron density matrix. This will give wrong results for the
total molecular gradient and
Hessian\index{molecular gradient}\index{molecular Hessian}. Mainly for debugging
purposes.

\item[\Key{NODV}] Do not calculate contributions from the active
one-electron density matrix. This will give wrong results for the
total molecular gradient and Hessian\index{molecular gradient}\index{molecular Hessian}. Mainly for debugging purposes.

\item[\Key{PRINT}]\verb| |\newline
\verb|READ (LUCMD,*) IPRINT|

Set print level in the calculation of one-electron contributions to
the molecular gradient and Hessian\index{molecular gradient}\index{molecular Hessian}.  Read one more line containing
print level. Default value is the value of \verb|IPRDEF| from the
general input module.

\item[\Key{SKIP}] Skip the calculation of one-electron integral
contributions to the molecular gradient and Hessian\index{molecular gradient}\index{molecular Hessian}. This will give
wrong total results for these properties. Mainly for debugging
purposes.

\item[\Key{STOP}] Stop the entire calculation after the
one-electron integral contributions to the molecular gradients and
Hessians has been evaluated\index{molecular gradient}\index{molecular Hessian}. Mainly for debugging purposes.
\end{description}

\subsection{Relaxation contribution to geometric Hessian: \Sec{RELAX}}

Directives controlling the calculation of the relaxation
contributions ({\it i.e.\/} those from the response terms) to the different
second-order molecular properties, appear in the \verb|*RELAX| section.
\begin{description}
\item[\Key{NOSELL}] Do not use Sellers' method \cite{hsijqc30}. This method
ensures that the error in the relaxation Hessian is quadratic in the
error of the response equation solution, rather than linear. Mainly
for debugging purposes.

\item[\Key{PRINT}]\verb| |\newline
\verb|READ (LUCMD,*) IPRINT|

Set the print level in the calculation of the relaxation
contributions.  Read one more line containing print level.
Default value is the value of \verb|IPRDEF| from the general input
module.

\item[\Key{SKIP}] Skip the calculation of the relaxation
contributions.  This does not skip the solution of the response
equations. This will give wrong results for a large number of
second-order molecular properties. Mainly for debugging purposes.

\item[\Key{STOP}] Stop the entire calculation after the
calculation of the relaxation contributions to the requested
properties. Mainly for debugging purposes.

\item[\Key{SYMTES}] Calculate both the $ij$ and $ji$ elements of
the relaxation Hessian to verify its Hermiticity or anti-Hermiticity
(depending on the property being calculated). Mainly for debugging
purposes.
\end{description}

\subsection{Reorthonormalization contributions to geometric Hessian: \Sec{REORT}}

Directives affecting the calculation of reorthonormalization
contributions to the geometric Hessian appear in the \verb|*REORT |
section.
\begin{description}
\item[\Key{PRINT}]\verb| |\newline
\verb|READ (LUCMD,*) IPRINT|

Set print level in the calculation of the lowest-order
reorthonormalization contributions to the molecular Hessian.  Read one
more line containing print level. Default value is the value of
\verb|IPRDEF| from the general input module.

\item[\Key{SKIP}] Skip the calculation of the reorthonormalization
contributions to the molecular Hessian. This will give wrong results
for this property. Mainly for debugging purposes.

\item[\Key{STOP}] Stop the entire calculation after finishing the
calculation of the reorthonormalization contributions to the molecular
Hessian. Mainly for debugging purposes.
\end{description}

\subsection{Response calculations for geometric Hessian: \Sec{RESPON}}
\label{sec:abares}

Directives affecting the response (coupled-perturbed MCSCF)
calculation of geometric perturbations appear in the \verb|*RESPON| section.
\begin{description}
\item[\Key{D1DIAG}] Neglect diagonal elements of the orbital
Hessian when generating trial vectors. Mainly for debugging
purposes.

\item[\Key{DONEXT}] Force the use of optimal orbital
trial\index{optimal orbital trial vector} vectors in
the solution of the geometric response equations as described in
Ref.~\cite{tuhjahjajpjjcp84}. This is done by solving the orbital part
exact while keeping the configuration part fixed.

\item[\Key{MAX IT}]\verb| |\newline
\verb|READ (LUCMD,*) MAXNR|

Maximum number of iterations to be used when solving the geometric
response equations.  Read one more line specifying value.
Default value is~60.

\item[\Key{MAXRED}]\verb| |\newline
\verb|READ (LUCMD,*) MAXRED|

Set the maximum dimension of the reduced space to which new basis
vectors are added as described in Ref.~\cite{tuhjahjajpjjcp84}. Read
one more line containing maximum dimension. Default value is the
maximum of 400 and 25 times the number of symmetry-independent nuclei.

\item[\Key{MAXSIM}]\verb| |\newline
\verb|READ (LUCMD,*) MAXSIM|

Maximum number of geometric perturbations to solve simultaneously in a
given symmetry.  Read one more line specifying value.  Default
is~15.

\item[\Key{MCHESS}] Explicitly calculate electronic Hessian and
test its symmetry. Does not abort the calculation. Mainly for
debugging purposes.

\item[\Key{NEWRD}] Forces the solution vectors to be written to a
new file. This will also imply that \verb|.NOTRIA| will be set to
\verb|TRUE|, that is, that no previous solution vectors will be used
as trial vectors.

\item[\Key{NOAVER}] Use an approximation to the orbital Hessian
diagonal when generating trial vectors.

\item[\Key{NONEXT}] Do not use optimal orbital
trial\index{optimal orbital trial vector} vectors.

\item[\Key{NOTRIA}] Do not use old solutions as trial vectors, even
though they may exist.

\item[\Key{NRREST}] Restart geometric response calculation using old
solution vectors.

\item[\Key{PRINT}]\verb| |\newline
\verb|READ (LUCMD,*) IPRINT|

Set the print level during the solution of the geometric response
equations.  Read one more line containing print level. Default
value is the value of \verb|IPRDEF| in the general input module.

\item[\Key{RDVECS}]\verb| |\newline
\verb|READ (LUCMD, *) NRDT|\newline
\verb|READ (LUCMD, *) (NRDCO(I), I = 1, NRDT)|

Solve for specific geometric perturbations only.  Read
one more line specifying number to solve for and then another
line specifying their sequence numbers. This may give wrong results
for some components of the molecular Hessian. Mainly for debugging
purposes.

\item[\Key{SKIP}] Skip the solution of the geometric response
equations. This will give wrong results for the geometric Hessian.
Mainly for debugging purposes.

\item[\Key{THRESH}]\verb| |\newline
\verb|READ (LUCMD,*) THRNR|

Threshold for convergence of the geometric response
equations.  Read one more line specifying value.  Default
value is~10$^{-3}$.

\item[\Key{STOP}] Stop the entire calculation after solving all
the geometric  response equations. Mainly for debugging purposes.
\end{description}

\subsection{Second-order polarization propagator approximation:
\Sec{SOPPA}}\label{sec:soppa}

Directives controling the calculation of molecular properties
using the second-order polarization propagator approximation, and
whether the MO (default) or AO based implementation is used.
The two implementations are desribed in Chapter~\ref{ch:soppa}, 
as well as some additional requirements for the AO based approach, 
see Section~\ref{sec:AOsoppa}.

\begin{description}
\item[\Key{HIRPA}] Use the higher-order RPA Polarization Propagator
  Approximation.

\item[\Key{SOPW4}] Requests that the W4 term in the SOPPA expressions
  are calculated explicitly.

\item[\Key{DIRECT}] Requests that the specified SOPPA or
SOPPA(CCSD) calculation is run using the AO-based implementation.
This is synonymous to \Key{AOSOP} or \Key{AOSOC}, depending on whether
\Key{SOPPA} or \Key{SOPPA(CCSD)} was used in \Sec{*PROPERTIES}.
This is possible for the
calculation of electronic singlet excitation energies with corresponding
oscillator and rotatory strengths, triplet excitation energies and
singlet type linear response functions. 

\item[\Key{DCRPA}] Requests an atomic orbital based RPA(D)
calculation. An RPA calculation will also be performed.
RPA(D) is currently available for the calculation of
electronic singlet and triplet excitation energies.
The necessary M{\o}ller-Plesset correlation
coefficients have to be requested by the \Key{CC} keyword in the
\Sec{*WAVE FUNCTIONS} input module combined with the \Key{MP2} and
\Key{AO-SOPPA} keywords in the \Sec{CC INPUT} section of the \Sec{*WAVE
FUNCTIONS} input module.

\item[\Key{AOSOP}] Requests an atomic orbital based SOPPA
calculation. This is possible for the calculation of
electronic singlet and triplet excitation energies and singlet type linear 
response functions. 
The necessary M{\o}ller-Plesset correlation
coefficients have to be requested by the \Key{CC} keyword in the
\Sec{*WAVE FUNCTIONS} input module combined with the \Key{MP2} and
\Key{AO-SOPPA} keywords in the \Sec{CC INPUT} section of the \Sec{*WAVE
FUNCTIONS} input module.

\item[\Key{AOSOC}] Requests an atomic orbital based SOPPA(CCSD)
calculation. This is possible for the calculation of
electronic singlet and triplet excitation energies and singlet type 
linear response functions.
The necessary CCSD amplitudes have to be requested by the \Key{CC} keyword 
in the \Sec{*WAVE FUNCTIONS} input module combined with the \Key{CCSD} and
\Key{AO-SOPPA} keywords in the \Sec{CC INPUT} section of the \Sec{*WAVE
FUNCTIONS} input module.

\item[\Key{AOCC2}] Requests an atomic orbital based SOPPA(CC2)
calculation. This is possible for the calculation of
electronic singlet and triplet excitation energies and singlet type 
linear response functions.
The necessary CC2 amplitudes have to be requested by the \Key{CC} keyword 
in the \Sec{*WAVE FUNCTIONS} input module combined with the \Key{CC2} and
\Key{AO-SOPPA} keywords in the \Sec{CC INPUT} section of the \Sec{*WAVE
FUNCTIONS} input module.

\item[\Key{AOHRP}] Requests an atomic orbital based higher-order RPA
calculation. This is possible for the calculation of
electronic singlet and triplet excitation energies and singlet type 
linear response functions. 

\item[\Key{AORPA}] Requests a RPA calculating run using the AO-based 
SOPPA implementation. This can be useful for calculation of excitation 
energies, since a subsequent calculation at a higher level of theory will 
use the converged RPA vectors as a starting guess.

\item[\Key{SOPCHK}] Request that the $E[2]$ and $S[2]$ matrices are
calculated explicitly and written to the output. This is only for
debugging purposes of the atomic integral direct implementation.

\item[\Key{NSAVMX}]\verb| |\newline
\verb|READ (LUCMD,*) NSAVMX|

Number of optimal trial vectors, which are kept in the solution of the
eigenvalue problem in the atomic integral direct implementation. The
default is 3. Increasing the number might reduce the number of
iterations necessary for solving the eigenvalue problem, but increases
the disk space requirements.

\item[\Key{NEXCI2}]\verb| |\newline
\verb|READ (LUCMD,*) (NEXCI2(I),I=1,NSYM)|

Allows in the atomic integral direct implementation to converge the
highest \verb|NEXCI2(I)| excitation energies in symmetry \verb|I| with
a larger threshold than the other excitation energies. The larger
threshold is given with the keyword \Key{THREX2}.

\item[\Key{THREX2}]\verb| |\newline
\verb|READ (LUCMD,*) THREX2|

Specifies in the atomic integral direct implementation the threshold to
which the highest excitation energies are to be converged. The number
of excitation energies for which this applies is chosen with the
keyword \Key{NEXCI2}.

\end{description}

\subsection{Indirect nuclear spin-spin couplings:
\Sec{SPIN-S}}\label{sec:spin-s}

This input module controls the calculation of which indirect nuclear
spin-spin coupling constants and what contributions to the total
spin-spin coupling constants that are to be calculated.

\begin{description}
\item[\Key{ABUNDA}]\verb| |\newline
\verb|READ (LUCMD,*) ABUND|

Set the natural abundance\index{abundance!spin-spin} threshold in percent
for discarding couplings between certain nuclei.
By default all isotopes in the molecule with a natural
abundance above this limit will be included in the list of nuclei for which
spin-spin coupling constants will be calculated. Read one more line
containing the abundance threshold in percent. The default value is~1.0 ({\it i.e.\/} 1\%),
which includes both protons and $^{13}$C nuclei.

\item[\Key{COUPLING NUCLEUS}]\verb| |\newline
\verb|READ (LUCMD,*) NUCSPI|\newline
\verb|READ (LUCMD,*) (IPOINT(IS), IS=1,NUCSPI)|

Calculates all coupling constants in a molecule to a selected number
of nuclei only. The first number \verb|NUCSPI| is the number of nuclei
to which couplings shall be calculated, and the next line reads in
the number of the symmetry-independent nucleus as given in the
\molinp\ file.

\item[\Key{ISOTOP}]\verb| |\newline
\verb|READ (LUCMD,*) (ISOTPS(IS), IS=1, NATOMS)|

Calculate the indirect spin--spin coupling constants for a given
isotopic constitution of the molecule. The next line reads the isotope
number for each of the atoms in the molecule (including also
symmetry-dependent molecules). The isotopic number for each atom is
given in terms of the occurrence in the list of natural abundance of
the isotopes for the given atom, {\it i.e.\/} the most abundant
isotope is number 1, the second-most abundant is number 2 and so on.

\item[\Key{NODSO}] Do not calculate diamagnetic
spin-orbit\index{diamagnetic spin-orbit}
contributions to the total indirect spin-spin
coupling\index{spin-spin coupling} constants. This
will give wrong results for the total spin-spin couplings.

\item[\Key{NOFC}] Do not calculate the Fermi
contact\index{Fermi contact} contribution
to the total indirect spin-spin coupling\index{spin-spin coupling}
constants. This will give
wrong results for the total spin-spin couplings.

\item[\Key{NOPSO}] Do not calculate the paramagnetic
spin-orbit\index{paramagnetic spin-orbit} contribution to the indirect
spin-spin coupling\index{spin-spin coupling} constants. This will
give wrong results for the total spin-spin couplings.

\item[\Key{NOSD}] Do not calculate the spin-dipole\index{spin-dipole}
contribution to
the total indirect spin-spin coupling\index{spin-spin coupling}
constants. This will give wrong
results for the total spin-spin couplings.

\item[\Key{PRINT}]\verb| |\newline
\verb|READ (LUCMD,*) ISPPRI|

Set the print level in the output of the final results from the
spin-spin coupling constants. In order to get all individual tensor
components (in a.u.), a print level of at least~5 is needed. Read one
more line containing the print level. Default value is the value
of \verb|IPRDEF| from the general input module.

\item[\Key{SD+FC}] Do not split the
spin-dipole\index{spin-dipole}\index{Fermi contact} and Fermi
contact contributions in the calculations.

\item[\Key{SDxFC ONLY}]

Will only calculate the spin
dipole--Fermi\index{spin-dipole}\index{Fermi contact} contact cross
term, and the
Fermi contact--Fermi contact contribution for the triplet
responses. The first of these two terms only contribute to the
anisotropy, and one may in this way obtain the most important triplet
contributions to the isotropic and
anisotropic\index{spin-spin coupling}\index{spin-spin anisotropy}
spin-spin couplings by only solving one instead of seven response
equations for each nucleus.

\item[\Key{SELECT}]\verb| |\newline
\verb|READ (LUCMD,*) NPERT|\newline
\verb|READ (LUCMD, *) (IPOINT(I), I = 1, NPERT)|

Select which symmetry-independent nuclei for which
indirect nuclear spin-spin couplings is to be calculated. This option
will override any
selection based on natural abundance (the \verb|.ABUNDA| keyword), and
at least one isotope of the nuclei requested will be evaluated (even
though the most abundant isotope with a non-zero spin has a lower
natural abundance
than the abundance threshold). Read one more line containing the
number of nuclei selected, and another line with their number (sorted after
the input order). By default, all nuclei with an isotope with non-zero spin
and with a natural abundance larger than the threshold will be included in
the list of nuclei for which indirect spin-spin couplings will be
calculated.

\item [\Key{SOS}]
Analysis of the spin-spin coupling constants in terms of pairs of
occupied and virtual orbitals \cite{spas079,spas086}. This implies that
the coupling constants are calculated as sum over all excited states,
which means that it is only possible in combination with Hartree-Fock
wavefunctions (RPA) or with density functional theory. The occupied and
virtual orbitals can be canonical Hartree-Fock or Kohn-Sham orbitals or
can be localized with the \Key{LOCALI} keyword in the \Sec{*PROPERTIES}
section.

\item [\Key{SOSOCC}]
Analysis of the spin-spin coupling constants in terms of pairs of
occupied orbitals \cite{spas079,spas086}. This implies that the
coupling constants are calculated as sum over all excited states, which
means that it is only possible in combination with Hartree-Fock
wavefunctions (RPA) or with density functional theory. The occupied
orbitals can be canonical Hartree-Fock or Kohn-Sham orbitals or can be
localized with the \Key{LOCALI} keyword in the \Sec{*PROPERTIES}
section.

\item [\Key{SOSOCS}]\verb| |\newline
  \verb|READ (LUCMD,*) NSTATI, NSTATF, NITRST|\newline
Similar to \Key{SOSOCC} but here only a window of states is included in
the sum over all excited states. The first and last state to be
included are specified by \verb|NSTATI| and \verb|NSTATF|, while one
can specifies with \verb|NITRST| for how many states at a time the
accumulated coupling constants will be printed. The occupied orbitals
can be canonical Hartree-Fock or Kohn-Sham orbitals or can be localized
with the \Key{LOCALI} keyword in the \Sec{*PROPERTIES} section.

\item [\Key{SINGST}]\verb| |\newline
 \verb|READ (LUCMD, *) NSTATS|\newline
Only the contributions from the \verb|NSTATS| lowest singlet states are
included in the analysis of spin-spin coupling constants in terms of
pairs of occupied orbitals \cite{spas079,spas086}.

\item [\Key{TRIPST}]\verb| |\newline
 \verb|READ (LUCMD, *) NSTATT|\newline
 Only the contributions from the \verb|NSTATT| lowest triplet states are
 included in the analysis of spin-spin coupling constants in terms of
pairs of occupied orbitals \cite{spas079,spas086}.
\end{description}

\subsection{Translational and rotational invariance:
\Sec{TROINV}}\label{sec:abatro}

Directives affecting the use of translational and rotational
invariance\index{translational invariance}\index{rotational invariance}~\cite{trarot}
appear in the \verb|*TROINV| section.
\begin{description}
\item[\Key{COMPAR}] Use both translational and
 rotational\index{translational invariance}\index{rotational invariance} symmetry
 and check the molecular Hessian against the Hessian obtained
without the use of translational and rotational invariance. This is
default in a calculation of vibrational circular dichroism
(VCD)\index{vibrational circular dichroism}\index{VCD}.

\item[\Key{PRINT}]\verb| |\newline
\verb|READ (LUCMD,*) IPRINT|

Set print level for the setting up and use of translational and
rotational invariance.  Read one more line containing print
level. Default value is the value of \verb|IPRDEF| from the
general input module.

\item[\Key{SKIP}] Skip the setting up and use of translational
and rotational invariance\index{translational invariance}\index{rotational invariance}.

\item[\Key{STOP}] Stop the entire calculation after the setup of
translational and rotational invariance\index{translational invariance}\index{rotational invariance}. Mainly for debugging purposes.

\item[\Key{THRESH}]\verb| |\newline
\verb|READ (LUCMD,*) THRESH|

Threshold defining linear dependence among
supposedly independent coordinates.  Read one more line specifying
value.  Default is~0.1.
\end{description}

\subsection{Linear response for static triplet property operators:
\Sec{TRPRSP}}\label{sec:trprsp}

Directives controlling the set-up of right-hand sides for triplet
perturbing operators (for instance the Fermi contact and spin-dipole
operators entering the nuclear spin-spin coupling
constants)\index{spin-dipole}\index{Fermi contact}\index{spin-spin coupling},
as well as when solving the triplet response equations appear in the
\verb|*TRPRSP| input module.

\begin{description}
\item[\Key{INTPRI}]\verb| |\newline
\verb|READ (LUCMD ,*) INTPRI|

Set the print level in the calculation of the atomic integrals
contributing to the different triplet operator right-hand sides. Read
one more line containing the print level. Default is the value
of \verb|IPRDEF| from the general input module.

\item[\Key{MAX IT}]\verb| |\newline
\verb|READ (LUCMD,*) MAXTRP|

Set the maximum number of micro iterations in the iterative solution of
the triplet response equations. Read one more line containing the
maximum number of iterations. Default is~60.

\item[\Key{MAXPHP}]\verb| |\newline
\verb|READ (LUCMD,*) MXPHP|

Set the maximum dimension for the sub-block of the configuration
Hessian that will be explicitly inverted. Read one more line
containing maximum dimension. Default value is~0.

\item[\Key{MAXRED}]\verb| |\newline
\verb|READ (LUCMD,*) MXRM|

Set the maximum dimension of the reduced space to which new basis
vectors are added as described in Ref.~\cite{tuhjahjajpjjcp84}. Read
one more line containing maximum dimension. Default value is~400.

\item[\Key{NORHS}] Skip the construction of the right-hand sides
for triplet perturbations. As this by necessity implies that all
right-hand sides and solution vectors are zero, this option is
equivalent to \verb|.SKIP  |. This will furthermore give wrong results
for the total spin-spin\index{spin-spin coupling} couplings. Mainly for debugging purposes.

\item[\Key{NORSP}] Skip the solution of the triplet response
equations. This will give wrong results for the total spin-spin
couplings\index{spin-spin coupling}. Mainly for debugging purposes.

\item[\Key{OPTORB}] Optimal orbital
trial\index{optimal orbital trial vector} vectors used in the
solution of the triplet response equations. These are generate by
solving the orbital response equation
exact, keeping the configuration part fixed as described in
Ref.~\cite{tuhjahjajpjjcp84}.

\item[\Key{PRINT}]\verb| |\newline
\verb|READ (LUCMD,*) IPRTRP|

Set the print level during the setting up of triplet operator
right-hand sides and in the solution of the response equations for
the triplet perturbation operators. Read one more line containing the
print level. Default is the value of \verb|IPRDEF| from the
general input module.

\item[\Key{SKIP}] Skip the construction of triplet right-hand
sides as well as the solution of the response equations for
the triplet perturbation operators. This will give wrong results for
the indirect nuclear spin-spin\index{spin-spin coupling}
couplings. Mainly for debugging
purposes.

\item[\Key{STOP}] Stop the entire calculation after generating the
triplet right-hand sides, and solution of the triplet response
equations. Mainly for debugging purposes.

\item[\Key{THRESH}]\verb| |\newline
\verb|READ (LUCMD,*) THRTRP|

Set the threshold for convergence in the solution of the triplet
response equations. Read one more line containing the
threshold. Default is~$1\cdot10^{-4}$.
\end{description}

\subsection{Two-electron contributions: \Sec{TWOEXP}}

Directives affecting the calculation of two-electron derivative
integral contributions to the molecular gradient\index{molecular gradient} and
Hessian\index{molecular Hessian} appear in
the \verb|*TWOEXP| section.

\begin{description}
\item[\Key{DIRTST}] Test the direct calculation of Fock matrices and
integral distributions. Mainly for debugging purposes.

\item[\Key{FIRST}] Compute first derivative integrals but not
second derivatives. This is default if only molecular gradients and
not the molecular Hessian has been requested.

\item[\Key{INTPRI}]\verb| |\newline
\verb|READ (LUCMD,*) IPRINT, IPRNTA, IPRNTB, IPRNTC, IPRNTD|

Set print level for the derivative integral calculation for a particular shell
quadruplet.  Read one more line containing print level and the four
shell indices.  The print level is changed from the default
for this quadruplet only. Default value is the value of \verb|IPRDEF|
from the general input module. Note that the print level of all shell
quadruplets can be changed by the keyword \verb|.PRINT |.

\item[\Key{INTSKI}] Skip the calculation of derivative integrals.
This will give wrong results for the total molecular gradients and
Hessians. Mainly for debugging purposes.

\item[\Key{NOCONT}] Do not contract derivative integrals
(program back-transforms density matrices to the primitive Gaussian
basis instead).

\item[\Key{NODC}] Do not calculate contributions from the inactive
one-electron density matrix. This will give wrong results for the
total molecular gradient and Hessian. Mainly for debugging purposes.

\item[\Key{NODV}] Do not calculate contributions from the active
one-electron density matrix. This will give wrong results for the
total molecular gradient and Hessian. Mainly for debugging purposes.

\item[\Key{NOPV}] Do not calculate contributions from the two-electron
density matrix. This will give wrong results for the total molecular
gradient and Hessian. Mainly for debugging purposes.

\item[\Key{PRINT}]\verb| |\newline
\verb|READ (LUCMD,*) IPRALL|

Set print levels.  Read one more line containing the print level for
this part of the calculation.  This will be the default print
level in the two-electron density matrix transformation, the
symmetry-orbital two-electron density matrix sorting, as well as the
print level in the integral derivative evaluation. To set the print
level in each of these parts individually, see the keywords
\verb|.INTPRI|, \verb|.PTRPRI|, \verb|.SORPRI|.

\item[\Key{PTRNOD}] The transformation of the two-electron density
matrix is back-transformed to the atomic orbital basis using a
noddy-routine for comparison.

\item[\Key{PTRPRI}]\verb| |\newline
\verb|READ (LUCMD,*) IPRPRT|

Set print level for the two-electron density matrix transformation.
Read one more line containing print level. Default value is the
value of  \verb|IPRDEF| from the general input module. Note also that
this print level is controlled by the keyword \verb|.PRINT |.

\item[\Key{PTRSKI}] Skip transformation of active two-electron
density matrix. This will give wrong results for the total molecular
Hessian. Mainly for debugging purposes.

\item[\Key{RETURN}] Stop after the shell quadruplet specified
under \verb|.INTPRI| above. Mainly for debugging purposes.

\item[\Key{SORPRI}]\verb| |\newline
\verb|READ (LUCMD,*) IPRSOR|

Set print level for the two-electron density matrix sorting. Read one
more line containing print level. Default value is the value of
\verb|IPRDEF| from the general input module. Note also that this print
level is controlled by the keyword \verb|.PRINT |.

\item[\Key{SORSKI}] Skip sorting of symmetry-orbital two-electron
density matrix. This will give wrong results for the total molecular
Hessian. Mainly for debugging purposes.

\item[\Key{SECOND}] Compute both first and second derivative
integrals. This is default when calculating molecular Hessians.

\item[\Key{SKIP}] Skip all two-electron derivative integral
and two-electron density matrix processing.

\item[\Key{STOP}] Stop the the entire calculation after finishing
the calculation of the two-electron derivative integrals. Mainly for
debugging purposes.

\item[\Key{TIME}] Provide detailed timing breakdown for the
two-electron integral calculation.
\end{description}

\subsection{Vibrational analysis: \Sec{VIBANA}}
\label{sec:abavib}

Directives controlling the calculation of harmonic
vibrational\index{vibrational analysis}
frequencies appear in the \verb|*VIBANA| section, as well as properties
depending on a normal coordinate analysis or vibrational frequencies.
Such properties include in the present version of the program:
Vibrational Circular Dichroism (VCD), Raman intensities, Raman
Optical Activity (ROA), and vibrational averaging.\index{vibrational circular
dichroism}\index{VCD}\index{Raman intensity}\index{IR
intensity}\index{Raman optical activity}\index{ROA}\index{effective
geometries}\index{r$_e$ geometries}\index{vibrationally averaged properties}

\begin{description}
%\item[\Key{INTERN}] Use internal coordinates for the vibrational
%analysis.  This has no effect on vibrational analyses performed at
%stationary points, but is asserted to provide more reliable
%results at non-stationary points.  Read more lines (see below) to
%specify the internal coordinates, variables are \verb|TYPE|,
%\verb|A|, \verb|B|, \verb|C|, \verb|D|, \verb|COEF|,
%\verb|SCAL|~(1X,A4,4I5,2F10.5). Currently this option is not bug-free,
%and it's use is not to be recommended.
%\begin{description}
%\item[\bf TYPE] This has the value \verb*|    | if this line continues
%a linear combination of primitive internal coordinates.  Otherwise
%the possible values are
%\begin{description}
%\item[\bf STRE] Bond stretch
%\item[\bf INVR] Reciprocal of bond stretch
%\item[\bf BEND] Angle bend
%\item[\bf OUT ] Angle between bond and plane
%\item[\bf TORS] Torsional angle
%\item[\bf LIN1] First of a collinear bending pair
%\item[\bf LIN2] Second of a collinear bend
%\end{description}
%\item[\bf A-D ] Atom numbers specifying the internal coordinates.
%Note that for an angle bend the angle is $\angle ACB$, including
%collinear bends!  Also, the out-of-plane mode is the angle between
%$AC$ and $BCD$??  Finally , in the collinear bend case the
%collinear angle is $\angle ACB$, and $D$ is used to specify a
%plane such that the first bend is in the plane~$ABD$.
%\item[\bf COEF] Coefficient of this primitive internal coordinate in
%the current linear combination.  Default is~1.
%\item[\bf SCAL] Overall scale factor for this linear combination
%(specify on first line only).  Default is~1.
%\end{description}

\item[\Key{HESFIL}] Read the molecular Hessian\index{Hessian} from the file
\verb|DALTON.HES|. This file may have been made in an earlier
calculation using the keyword \Key{HESPUN}, or constructed from a
calculation with the GaussianXX program and converted to \dalton\
format using the \verb|FChk2HES.f| program. Useful in VCD and VROA
analyses\index{VCD}\index{ROA}\index{vibrational circular
dichroism}\index{Raman optical activity}.

\item[\Key{HESPUN}] Write the molecular Hessian\index{Hessian} to the file
\verb|DALTON.HES| for use as a starting Hessian in
first-order\index{first-order optimization} geometry
optimizations (see keyword \Key{HESFIL} in the \Sec{OPTIMIZE} input
module), or for later use in a vibrational analysis (see keyword
\Key{HESFIL} in this input module)\index{vibrational analysis}.

%\item[\Key{NOCID}] Do not calculate the Circular Intensity
%Differentials\index{circular intensity differential}\index{CID} (CIDs)
%as defined by Barron~\cite{barronbook}, but
%instead print the chirality\index{chirality number} numbers defined by
%Hug~\cite{whc48} in
%calculation of vibrational Raman Optical Activity
%(VROA)\index{ROA}\index{Raman optical activity}.

\item[\Key{ISOTOP}]

\begin{verbatim}
READ (LUCMD,*) NISOTP, NATM
DO 305 ICOUNT = 1, NISOTP
   READ (LUCMD,*) (ISOTP(ICOUNT,N), N = 1, NATM)
END DO
\end{verbatim}

Read in the number of different isotopically\index{isotopic
constitution} substituted species
\verb|NISOTP| for which we are to do a vibrational analysis. The
isotopic species containing only the most abundant isotopes is always
calculated.

\verb|NATM| is the total number of atoms in the molecules (see
discussion in Section~\ref{sec:vibfreq}). For each isotopic species,
the isotope for each atom in the molecule is read in. A 1 denotes the
most abundant isotope, a 2 the second-most abundant isotope and so on.

\item[\Key{PRINT}]\verb| |\newline
\verb|READ (LUCMD,*) PRINT|

Set the print level in the vibrational\index{vibrational analysis}
analysis of the molecule.  Read
one more line containing print level. Default value is the value
of \verb|IPRDEF| from the general input module.

\item[\Key{SKIP}] Skip the analysis of the vibrational frequencies
and normal modes of the molecule.
\end{description}

%%% Local Variables:
%%% mode: latex
%%% TeX-master: "Master"
%%% End:

%\chapter{Linear and non-linear response functions, RESPONSE}
\label{ch:response}

\section{Directives for evaluation of molecular response functions}\label{sec:rspinp}

The directives in the following subsections may be included in the input to \resp.
They are organized according to the program section names in which they
appear.

\resp\ is the most general part of the code for calculating 
many different electronic linear, quadratic, or cubic molecular
response properties based on SCF or MCSCF or CI wave functions.
No nuclear contributions are added.

If the final wave function from \Sec{*WAVE FUNCTIONS} was \Key{CI}, then
a configuration interaction\index{Configuration Interaction!response}\index{CI!response}
response calculation will be performed. 
This is equivalent to a CI sum-over-states\index{response!CI sum-over-states}
calculation of response properties,
but of course calculated directly without diagonalization of the full
CI Hamiltonian matrix.

Some of the SCF/MCSCF response properties can also be requested
from \Sec{*PROPERTIES} input modules.
NOTE: for such properties you should request them either here or
in \Sec{*PROPERTIES}, otherwise you will calculate them twice!
Usually the output is nicest in
the \Sec{*PROPERTIES} module (e.g. collected in tables and in
often used units, most properties are only given in atomic
units in \resp), and nuclear contributions are included if relevant.
Some specific properties, especially those involving nuclear derivatives,
can only be calculated via \Sec{*PROPERTIES}.

Calculations of coupled cluster response properties are performed
by different modules and are described
in Chapter~\ref{ch:CC} on coupled cluster calculations.

In addition, SOPPA\index{SOPPA}\index{polarization propagator}
(Second-Order Polarization Propagator 
approximation) or SOPPA(CCSD)\index{SOPPA(CCSD)} (Second Order Polarization
Propagator Approximation with Coupled Cluster Singles and Doubles Amplitudes) 
for the calculation of linear response
\index{linear response}\index{response!linear}
properties and excitation energies with transition moments
may be requested in this input section. The current
implementation of the SOPPA method is described 
in Ref.~\cite{mjpekdtehjajjojcp} and of the SOPPA(CCSD) method in 
Ref.~\cite{soppaccsd}. Note that a SOPPA calculation
requires the keyword \Key{SOPPA}, whereas a SOPPA(CCSD) calculation requires
the keyword \Key{SOPPA(CCSD)}.
%M. J. Packer, E. K. Dalskov, T. Enevoldsen, H. J. Aa. Jensen, and
%J. Oddershede, J. Chem. Phys., submitted. 


\subsection{General: \Sec{*RESPONSE}}

General-purpose directives are given in the \Sec{*RESPONSE} section.

After the last directive of the \Sec{*RESPONSE} input group
should follow another {\ttfamily{**<something>}} input group
(or \Sec{*END OF DALTON INPUT} if this was the last input to \dalton).

\begin{description}

\item{\Key{SOPPA}}
Requests the second order polarization propagator approximation 
in the linear response module.
The SOPPA\index{SOPPA}\index{polarization propagator!SOPPA}\index{response!SOPPA}
flag requires that
the preceding {\sir} calculation has generated the MP2 correlation
coefficients and written them to disk (set \Key{RUN RESPONSE} in \Sec{*DALTON}
input as well as \Key{HF} and \Key{MP2} in \Sec{*WAVE FUNCTIONS}). See
example input in Chapter~\ref{ch:starting}.

\item{\Key{SOPPA(CCSD)}}
Requests the second order polarization propagator approximation with coupled 
cluster singles and doubles amplitudes in the linear response module.
The SOPPA(CCSD)\index{SOPPA(CCSD)}\index{polarization propagator!SOPPA(CCSD)}
\index{response!SOPPA(CCSD)} flag requires that the preceding Coupled Cluster 
calculation has generated the CCSD amplitudes and written them to disk 
(set \Key{RUN RESPONSE} in 
\Sec{*DALTON} input, \Key{HF} and \Key{CC} in \Sec{*WAVE FUNCTIONS}  and 
\Key{SOPPA(CCSD)} in \Sec{CC INPUT}). See example input in 
Chapter~\ref{ch:starting}.

\item{\Key{SOPW4}}
Calculate explicitly the W4 term described by Oddershede {\it et
al.\/}~\cite{jopjdycpr2}. This term is already included in the normal
SOPPA\index{SOPPA}\index{polarization propagator!SOPPA} or 
SOPPA(CCSD)\index{SOPPA(CCSD)} result, and used mostly for comparing to older
calculations. Note that this keyword requires that \Key{SOPPA} or 
\Key{SOPPA(CCSD)}is set.

\item{\Key{HIRPA}}
Invoke the higher RPA approximation for the calculation of linear
response properties\index{linear response!higher RPA}.
This approximation is identical to that of McKoy
and coworkers~\cite{jrtsvmjcp58,tsjrvmjcp58}. The requirements to the
preceding wave function 
calculation is the same as for the \Key{SOPPA} keyword.
This keyword overrides a simultaneous specification of \Key{SOPPA}.

\item{\Key{TRPFLG}}
Triplet flag. This option is set whenever triplet
(spin-dependent)\index{triplet response}
operators must be used in a response calculation
\cite{jodlypjjcp91,ovhapjhjajthjojcp97}.
This flag forces triplet linear response for \Sec{LINEAR},
both for second order properties and electronic excitations
(without and with \Key{SINGLE RESIDUE}).

\item{\Key{NOAVDI}}
Do not use Fock type decoupling of the two-electron density matrix.
Add $F^ID$ instead of $(F^I+F^A)D$ to $E^{[2]}$ approximate
orbital diagonal. Not recommended as the approximate orbital diagonal
normally will become more different from the exact orbital diagonal.

\item{\Key{INPTEST}}
Input test. For debugging purposes only. The program stops after the
input section.

\item{\Key{MAXPHP}}\\
\verb|READ *, MAXPHP|\\
Change the maximum dimension of $H_0$ subspace.   Default is 100.
PHP is a subblock of the CI matrix which is calculated explicitly
in order to obtain improved CI trial vectors compared to the
straight Davidson algorithm\cite{erdjcp17}.  The configurations
corresponding to 
the lowest diagonal elements are selected, unless \Key{PHPRESIDUAL} is
specified. MAXPHP is the maximum dimension of PHP, the 
actual dimension will be less if MAXPHP will split degenerate configurations.
 
\item{\Key{MAXRM}}\\
\verb|READ *, MAXRM |\\
Change the maximum dimension of the reduced space. Default is 200.
When solving a linear system of equations or an eigenvalue equation,
the reduced space is increased by the number of
frequencies/excitations in each iteration. For single root
calculations this should exceed the number of iterations required.
MAXRM should be increased if many frequencies or excitation energies
are to be calculated.
Sharp convergence thresholds also requires
more iterations and thus larger dimension of reduced space.

\item{\Key{NODOIT}}
Turns off direct one-index transformation \cite{ovhahjajjcc15}. 
In this way all one-index transformed integrals are stored on disk.

\item{\Key{NOITRA}}
No integral transformation. Normally the two-electron integrals are 
transformed in the beginning of a response calculation. In some cases
this is not desirable, e.g., if the response part is only used for
calculating average values of operator, or if the transformed two-electron
integrals already exist from a previous response calculation. 

\item{\Key{S0MIX}}
Sum rule is calculated in mixed representation, that is, calculate
$N_e=\langle0\mid [r,p] \mid0\rangle$ provided that dipole length and
velocity integrals are available on the property integral file 
(calculated with \Sec{*HERMIT} options \Key{DIPLEN} and \Key{DIPVEL}).
The calculated quantity gives a measure of the quality of the basis
set\index{basis set!quality}.

\item{\Key{OPTORB}}
Orbital trial vectors are calculated with optimal orbital
trial\index{optimal orbital trial vector} vector
algorithm \cite{tuhjahjajpjjcp84}.

\item{\Key{ORBSFT}}\\
\verb|READ *, ORBSFT|\\
Change the amount for shifting the orbital
diagonal\index{orbital diagonal Hessian} of the MCSCF Hessian.
May be used if there is a large number of negative eigenvalues.
Default is $10^{-4}$. 

\item{\Key{ORBSPC}}
Calculation with only orbital operators. 

\item{\Key{PHPRESIDUAL}}
Select configurations for PHP matrix based on largest residual
rather than lowest diagonal elements.

\item{\Key{PROPAV}} \\
\verb|READ '(A)', LABEL|\\
Property average\index{property average}. The average value of an
operator is calculated. 
The line following this option must contain the
label of the operator given in the integral property file.
(See section \ref{ch:hermit}.)

\item{\Key{QRREST}}
Restart quadratic response\index{quadratic response}\index{response!quadratic}
calculation.
Requires that all needed linear response solutions and excitation vectors
are available on the RSPVEC file.
\end{description}

\subsection{Linear response calculation: \Sec{LINEAR}}

A\index{linear response}\index{response!linear} linear response
\cite{jodlypjjcp91,pjhjajjojcp89} calculation is performed for a given
choice of operators,
-$\langle\!\langle A; B \rangle\!\rangle_{\omega}$.
(Note that {\em minus} the linear response properties are written to output.)

In the same \resp\ calculation these linear response properties can be calculated
together with excitation energies
and with long range dispersion coefficients, but not 
together with quadratic or cubic response.

\begin{description}

\item{\Key{TRIPLET}} Defines $A$ and $B$ to be triplet operators.
Will also make a simultaneous \Sec{LINEAR} \Key{SINGLE RESIDUE} calculation to
a calculation of triplet excitation energies and transition moments.

\item{\Key{DIPLEN}}
Sets $A$ and $B$ to dipole operators\index{dipole length}.

\item{\Key{DIPVEL}}
Sets $A$ and $B$ to velocity operators\index{dipole velocity}.

\item{\Key{DIPMAG}}
Sets $A$ and $B$ to angular momentum operators\index{angular momentum}.

\item{\Key{QUADMOM}}
Sets $A$ and $B$ to the quadrupole\index{quadrupole operator} operators.

\item{\Key{SPIN-O}}
Sets $A$ and $B$ to spin-orbit operators\index{spin-orbit}.

\item{\Key{PROPRT}}\\
\verb|READ '(A)', LABEL|\\
Sets $A$ and $B$ to a given operator with label; LABEL.
(The calculation of the operator must be specified to the integral
module, see section \ref{ch:hermit}.)

\item{\Key{FREQUE}}\\
\verb|READ *, NFREQ|\\
\verb|READ *, FREQ(1:NFREQ)|\\
Response equations are evaluated at given
frequencies\index{frequency}. Two lines following 
this option must contain 1) The number of frequencies, 2) Frequencies.
Remember to increase \Key{MAXRM} if many frequencies are specified.

\item{\Key{THCLR}}\\
\verb|READ *, THCLR|\\
Relative convergence threshold for all requested linear response functions.
Default is 1.0D-3; note that this number should be at least 10 times
bigger than the final gradient norm in the SCF/MCSCF
wave function optimization. The accuracy of the linear response 
properties will be quadratic in this threshold; thus the default
corresponds to convergence to approximately 6 digits.

\item{\Key{PRINT}}\\
\verb|READ *,IPRLR|\\
Sets print level for linear response module. Default is 2.
 
\item{\Key{RESTLR}}
Restart\index{restart!linear response} of response calculation. This
can only be used if the  
operator specified is the same which was used \textit{last} in the previous
response calculation.

\item{\Key{MAXIT}}\\
\verb|READ (LUCMD,*) MAXITL|\\
Maximum number of iterations for solving a linear response 
equation. Default is 60.

\item{\Key{MAXITO}}\\
\verb|READ (LUCMD,*) MAXITO|\\
Maximum number of iterations in the optimal orbital
algorithm\index{optimal orbital trial vector}
\cite{tuhjahjajpjjcp84}. 
Default is 5.

%\item{\Key{OLSEN}}
%CI trial vectors are obtained with Olsen algorithm (for the single
%residue calculation only).

\end{description}

\subsection{Linear response excitation energies calculation: \Sec{LINEAR} with \Key{SINGLE RESIDUE}}

Single residues\index{single residue!linear response} of the linear
response\index{linear response!single residue}\index{response!excitations} function is
computed. Residues of a linear response function correspond to
transition moments\index{transition moment!linear response} and the associated poles
correspond to vertical electronic excitation energies.
%\cite{jodlypjjcp91,pjhjajjojcp89}

In the same \resp\ calculation these excitation properties can be calculated
together with linear response properties
and with long range dispersion coefficients, but not 
together with quadratic or cubic response.

\begin{description}

\item{\Key{SINGLE RESIDUE}} Required to get excitation energies, without
this keyword the linear response function will be evaluated.

\item{\Key{TRIPLET}} Calculate triplet excitation energies and transition moments.
Will also make a simultaneous linear response calculation of triplet symmetry.

\item{\Key{ROOTS}}\\
\verb|READ '*',(ROOTS(I) I=1,NSYM)|\\
Number of roots.  The line following this option contains the number
of excited states\index{excited state} per symmetry. Excitation
energies\index{excitation energy} are calculated for each state and if
any operators are given, 
symmetry-allowed transition moments\index{transition moment} are
calculated between the 
reference state and the excited states.
Remember to increase \Key{MAXRM} if many frequencies are specified.

\item{\Key{DIPLEN}}
Calculate transition moments for all dipole operators\index{dipole length}.

\item{\Key{DIPVEL}}
Calculate transition moments for all velocity operators\index{dipole velocity}.

\item{\Key{DIPMAG}}
Calculate transition moments for all angular momentum operators\index{angular momentum}.

\item{\Key{QUADMOM}}
Calculate transition moments for all quadrupole\index{quadrupole operator} operators.

\item{\Key{SPIN-O}}
Calculate transition moments for all spin-orbit operators\index{spin-orbit}.
Warning: this option implies \Key{TRIPLET} and
forces the excitations to be of triplet symmetry,
and all operators -- including
e.g. \Key{DIPLEN} -- will be assumed by the program to be of triplet symmetry!!

\item{\Key{PROPRT}}\\
\verb|READ '(A)', LABEL|\\
Calculate either singlet or triplet transition moments for a given operator with label; LABEL.
(The calculation of the operator must be specified to the integral
module, see section \ref{ch:hermit}.)

\item{\Key{PRINT}}\\
\verb|READ *,IPRPP|\\
Sets print level for single residue linear response module. Default is 2.

\item{\Key{MAXIT}}\\
\verb|READ *, MAXITP|\\
Maximum number of iterations for solving the linear response eigenvalue
equation. Default is 60.

\item{\Key{MAXITO}}\\
\verb|READ *, MAXITO|\\
Maximum number of iterations in the optimal orbital
algorithm\index{optimal orbital trial vector}
\cite{tuhjahjajpjjcp84}. 
Default is 5.

\item{\Key{RESTPP}}
Restart\index{restart!excitation energy} of response
calculation. This can only be used if the root which is 
specified is the same which was used \textit{last} in the previous
response calculation.

\item{\Key{THCPP}}\\
\verb|READ *, THCPP|\\
Threshold for solving the single residue linear response eigenvalue equation. 
Default is 1.0D-3; note that this number should be at least 10 times
bigger than the final gradient norm in the SCF/MCSCF
wave function optimization.
The accuracy of the pole (excitation energy) will be 
quadratic in this threshold, thus the default corresponds to approximately
6 digits. The accuracy of transition moments will be linear in this threshold.

\item{\Key{OLSEN}}
CI trial vectors are obtained with Olsen algorithm.

\end{description}

\subsection{Quadratic response calculation: \Sec{QUADRA}}

Calculation of third order properties\index{properties!third order}
 as quadratic response
functions\index{quadratic response}\index{response!quadratic}.
$A$, $B$, and $C$-named options refer to the operators in the quadratic
response function 
$\langle\!\langle A;B,C \rangle\!\rangle_{\omega_b,\omega_c}$
\cite{ovhapjhjajthjojcp97,hhhjajpjjojcp97,haovhkpjthjcp98}

The second order properties from the linear response functions
$\langle\!\langle A;B,\rangle\!\rangle_{\omega_b}$ are also printed
(if $A$ and $B$ operators have the same spin symmetry),
as they can be obtained at no extra computational cost.

\begin{description}

\item{\Key{SHG}}
Only response functions connected with second harmonic
generation\index{second harmonic generation}\index{response!second harmonic generation}\index{quadratic response!second harmonic generation}
are computed, $\beta(-2\omega,\omega,\omega)$ .
Can be specified together with \Key{POCKEL}.
Frequencies must be specified with \Key{FREQUE}.
Remember to specify operators as well, e.g. \Key{DIPLEN}.

\item{\Key{POCKEL}}
Only response functions connected with electro-optical
Pockels effect\index{Pockels effect}\index{response!Pockels effect}\index{quadratic response!Pockels effect}
$\beta(-\omega; \omega,0)$, are computed.
Can be specified together with \Key{SHG}.
Frequencies must be specified with \Key{FREQUE}.
Remember to specify operators as well, e.g. \Key{DIPLEN}.

\item{\Key{DIPLEN}}
Sets $A$, $B$, and $C$ to dipole operators\index{dipole length}.

\item{\Key{DIPLNX}}
Sets $A$, $B$, and $C$ to the $x$ dipole operator\index{dipole length}.

\item{\Key{DIPLNY}}
Sets $A$, $B$, and $C$ to the $y$ dipole operator\index{dipole length}.

\item{\Key{DIPLNZ}}
Sets $A$, $B$, and $C$ to the $z$ dipole operator\index{dipole length}.

\item[\Key{APROP}, \Key{BPROP}, \Key{CPROP}]
Specify the operators $A$, $B$, and $C$. The line following this
option should be the label of the operator as it appears in the file
AOPROPER.

\item[\Key{ASPIN}, \Key{BSPIN}, \Key{CSPIN}]
Spin information for quadratic response calculations.
The line following these options contains the spin
rank\index{spin rank} of the operators 
$A$, $B$, and $C$, respectively, 0 for singlet operators and 1 for triplet
operators.
In a triplet response calculations two of these operators are of rank one,
and the remaining operator of rank zero.

\item{\Key{FREQUE}}\\
\verb|READ *, NFREQ|\\
\verb|READ *, FREQ(1:NFREQ)|\\
Response equations are evaluated at given
frequencies\index{frequency!quadratic response}. Two lines 
following this option must contain 1) The number of frequencies, 2)
Frequencies.
For the Kerr effect only the $B$-frequency is set,
and in other cases both $B$ and $C$-frequencies are set.
May not be used together with \Key{BFREQ} or \Key{CFREQ}.
Default is one frequency of each type: zero (static).

\item[\Key{BFREQ}, \Key{CFREQ}]
Individual specification of the frequencies $\omega_b$ and $\omega_c$.
Input as in \Key{FREQUE} above.
May not be used for \Key{SHG} and \Key{POCKEL}.
May not be used together with \Key{FREQUE}.
Default is one frequency of each type: zero (static).

\item{\Key{MAXIT}}
Maximum number of iterations for solving a linear response equation.
Default is 60.

\item{\Key{MAXITO}}
Maximum number of iterations in the optimal
orbital\index{optimal orbital trial vector} algorithm 
\cite{tuhjahjajpjjcp84}. 
Default is 5.

\item{\Key{PRINT}}\\
\verb|READ *,IPRHYP|\\
Print level.

\item{\Key{THCLR}}
Threshold for solving the linear response equations.
Default is $10^{-3}$.

\end{description}

\subsection{Quadratic response calculation of second order transition moments:
\Sec{QUADRA} with \Key{SINGLE RESIDUE}}

%Calculation of third order properties as quadratic response
%functions\index{quadratic response}\index{response!quadratic}.
%$A, B$, and $C$-named options refer to the operators in the quadratic
%response function 
%$\langle\!\langle A;B,C \rangle\!\rangle_{\omega_b,\omega_c}$
%\cite{ovhapjhjajthjojcp97,hhhjajpjjojcp97,haovhkpjthjcp98}

\begin{description}

\item{\Key{SINGLE RESIDUE}}
Required to
compute the single residue\index{single residue!quadratic response} of the quadratic
response function\index{quadratic response!single residue}\index{response!quadratic, single residue}.
For the case of dipole operators this corresponds to two-photon
transition
moments\index{two-photon!transition moment}\index{transition moment}\index{transition moment!two-photon}.

\item{\Key{ROOTS}}\\
\verb|READ '*',(ROOTS(I) I=1,NSYM)|\\
Number of roots.  The line following this option contains the number
of excited states\index{excited state!second order moment} per symmetry. Excitation
energies\index{excitation energy!second order moment} are calculated for each state and if
any operators are given, 
symmetry-allowed second order transition moments\index{transition moment!second order} are
calculated between the 
reference state and the excited states.
Remember to increase \Key{MAXRM} if many frequencies are specified.

\item{\Key{PHOSPHORESCENCE}}
Specifies a phosphorescence\index{phosphorescence} calculation, i.e.,
the spin-orbit\index{spin-orbit} 
induced singlet-triplet transition\index{singlet-triplet transition}. This keyword sets up the  
calculation so that no further response input is required except \Key{ROOTS}; the
$A$ operator is set to the dipole operators\index{dipole length} and
the $B$ operator  
is set to the spin-orbit\index{spin-orbit}
operators. \cite{ovhapjhjajthjojcp97,haovbmaqc27} 
The reference state {\em must} be a singlet spin state.

\item{\Key{MCDBTERM}}
Specifies the calculation of all individual components to the 
${\cal{B}}(0\to f)$ term of magnetic circular dichroism
(MCD)\index{magnetic circular dichroism}\index{B-term}\index{MCD}.
This keyword sets up the calculation so that no further response input is required except \Key{ROOTS}. 
The $A$ operator is set equal to the $\alpha$ component of dipole 
operator\index{dipole length} and
the $B$ operator to the $\beta$ component of the angular momentum\index{angular momentum}
operator. The resulting "mixed" two-photon transition moment to state $f$ 
is then multiplied for the dipole-allowed one-photon transition moment 
from state $f$ (for the $\gamma$ component, with $\alpha \neq \beta \neq \gamma$).
\cite{Coriani:MCDRSP} 

\item{\Key{DIPLEN}}
Sets $A$ and $B$ to $x, y, z$ dipole operators\index{dipole length}.

\item{\Key{DIPLNX}}
Sets $A$ and $B$ to the $x$ dipole operator\index{dipole length}.

\item{\Key{DIPLNY}}
Sets $A$ and $B$ to the $y$ dipole operator\index{dipole length}.

\item{\Key{DIPLNZ}}
Sets $A$ and $B$ to the $z$ dipole operator\index{dipole length}.

\item{\Key{BFREQ}, \Key{FREQUE}}\\
\verb|READ *, NFREQ|\\
\verb|READ *, FREQ(1:NFREQ)|\\
The frequencies $\omega_b$ in atomic units.
Response equations are evaluated at given
frequencies\index{frequency}. Two lines 
following this option must contain 1) The number of frequencies, 2)
Frequencies.

\item{\Key{ISPABC}}\\
\verb|READ *, ISPINA,ISPINB,ISPINC|\\
Spin symmetry of $A$-operators (ISPINA), $B$-operators (ISPINB),
and the excitation operator (ISPINC): "0" for singlet and "1" for triplet.
Default is "0,0,0", i.e. all of singlet spin symmetry.
{\bf Note: triplet operators are only implemented for singlet reference states.}
%hjaaj June 2001: .ASPIN etc. should be defined for .SINGLE
%\item[\Key{ASPIN}, \Key{BSPIN}, \Key{CSPIN}]
%\index{quadratic response}\index{response!quadratic}
%Spin information for quadratic response calculations.
%The line following these options contains the spin
%rank\index{spin rank} of the operators 
%$A$, $B$, and $C$, respectively, 0 for singlet operators and 1 for triplet
%operators. If \Key{SINGLE} is specified, \Key{CSPIN} denotes the
%spin of the excited state. If \Key{DOUBLE} is specified,
%both \Key{BSPIN} and \Key{CSPIN} denote excited state spins.
%In a triplet response calculations two of these operators are of rank one,
%and the remaining operator of rank zero.

\item[\Key{APROP}, \Key{BPROP}]
Specify the operators $A$ and $B$, respectively. The line following this
option should be the label of the operator as it appears in the file
AOPROPER.

\item{\Key{PRINT}}\\
\verb|READ *,IPRSMO|\\
Print level.

\item{\Key{MAXITL}}
Maximum number of iterations for linear equations in this section.
Default is 60.

\item{\Key{MAXITP}}
Maximum number of iterations in solving the linear
response\index{linear response}\index{response!linear} eigenvalue 
equations.
Default is 60.

\item{\Key{MAXITO}}
Maximum number of iterations in the optimal
orbital\index{optimal orbital trial vector} algorithm 
\cite{tuhjahjajpjjcp84}. 
Default is 5.

\item{\Key{THCLR}}
\verb|READ *, THCLR|\\
Threshold for solving the linear response equations.
Default is $10^{-3}$.

\item{\Key{THCPP}}\\
\verb|READ *, THCPP|\\
Threshold for solving the linear response
\index{linear response}\index{response!linear}
eigenvalue equation. Default is $10^{-3}$.

\end{description}


\subsection{Quadratic response calculation of transition moments between excited states:
\Sec{QUADRA} with \Key{DOUBLE RESIDUE}}

Calculation of third-order properties as quadratic response
functions\index{quadratic response}\index{response!quadratic}.
$A,B$, and $C$-named options refer to the operators in the quadratic
response function 
$\langle\!\langle A;B,C \rangle\!\rangle_{\omega_b,\omega_c}$
\cite{ovhapjhjajthjojcp97,hhhjajpjjojcp97,haovhkpjthjcp98}

\begin{description}

\item{\Key{DOUBLE RESIDUE}}
Computes the double residue\index{quadratic response!double residue}
of the quadratic
response function\index{double residue!quadratic response}\index{response!quadratic, double residue}.
Double residues of the quadratic response function correspond to transition
moments between excited states\index{transition moment!between excited states},
$\langle B \mid A \mid C \rangle$. 

\item{\Key{DIPLEN}}
Sets $A$ to dipole operators\index{dipole length}.

\item{\Key{DIPLNX}}
Sets $A$ to the $x$ dipole operator\index{dipole length}.

\item{\Key{DIPLNY}}
Sets $A$ to the $y$ dipole operator\index{dipole length}.

\item{\Key{DIPLNZ}}
Sets $A$ to the $z$ dipole operator\index{dipole length}.

\item[\Key{PROPRT}]
%hjaaj June 2001, ought to define as well: \item[\Key{APROP}]
Specify another $A$ operator. The line following this
option should be the label of the operator as it appears in the file
AOPROPER. This option may be repeated for different property operators.

\item{\Key{ISPABC}}\\
\verb|READ *, ISPINA,ISPINB,ISPINC|\\
Spin symmetry of $A$-operators (ISPINA)
and the left and right excitation operators (ISPINB and ISPINC):
"0" for singlet and "1" for triplet.
Default is "0,0,0", i.e. all of singlet spin symmetry.
{\bf Note: triplet operators are only implemented for singlet reference states.}
%hjaaj June 2001: .ASPIN etc. should be defined for .DOUBLE
%\item[\Key{ASPIN}, \Key{BSPIN}, \Key{CSPIN}]
%\index{quadratic response}\index{response!quadratic}
%Spin information for quadratic response calculations.
%The line following these options contains the spin
%rank\index{spin rank} of the operators 
%$A$, $B$, and $C$, respectively, 0 for singlet operators and 1 for triplet
%operators. If \Key{SINGLE} is specified, \Key{CSPIN} denotes the
%spin of the excited state. If \Key{DOUBLE} is specified,
%both \Key{BSPIN} and \Key{CSPIN} denote excited state spins.
%In a triplet response calculations two of these operators are of rank one,
%and the remaining operator of rank zero.


\item{\Key{PRINT}}\\
\verb|READ *,IPRPP|\\
Print level for solving linear response eigenvalue equations.

\item{\Key{IPREXM}}\\
\verb|READ *,IPREXM|\\
Print level for special excited state transition moment routines.


\item{\Key{THCPP}}\\
\verb|READ *, THCPP|\\
Threshold for solving the linear response
eigenvalue equation. Default is $10^{-3}$.

\item{\Key{MAXIT}}
Maximum number of iterations for solving linear response
eigenvalue equation in this section.

\item{\Key{MAXITO}}
Maximum number of iterations in the optimal
orbital\index{optimal orbital trial vector} algorithm 
\cite{tuhjahjajpjjcp84}. 
Default is 5.

\end{description}


\subsection{Cubic response calculation: \Sec{CUBIC}}
Calculation of fourth-order properties as cubic response functions\index{cubic response}\index{response!cubic}
\cite{pndjovhacpl242,djpnhajcp105,pndjhapdkrthhkcpl253}.
$A,B$,$C$, and $D$-named options refer to the operators in the cubic
response function 
$\langle\!\langle A;B,C,D \rangle\!\rangle_{\omega_b,\omega_c,\omega_d}$

\begin{description}

\item[\Key{APROP}, \Key{BPROP}, \Key{CPROP}, \Key{DPROP}]
Specify the operators $A$, $B$, $C$, and $D$. The line following this
option should be the label of the operator as it appears in the file
AOPROPER.

\item[\Key{BFREQ}, \Key{CFREQ}, \Key{DFREQ}]
The frequencies\index{frequency!cubic response}
$\omega_b$, $\omega_c$, and $\omega_d$, respectively. Input as in
\Key{FREQUE}.

\item{\Key{FREQUE}}\\
\verb|READ *, NFREQ|\\
\verb|READ *, FREQ(1:NFREQ)|\\
Sets the frequencies\index{frequency!cubic response} whenever a optical process is specified.
Can also be used for the residue calculation and it does then set 
both $\omega_b$ and $\omega_c$ for the single residue and only
$\omega_b$ for the double residue.

\item{\Key{THG   }}
Only response functions connected to the third harmonic
generation\index{third harmonic generation} are
computed, $\gamma(-3\omega;\omega,\omega,\omega)$ \cite{djpnylhajcp105}.

\item{\Key{DC-SHG}}
Only response functions connected to the static electric field-induced
second harmonic generation\index{electric field!induced SHG} are computed,
$\gamma(-2\omega;\omega,\omega,0)$.

\item{\Key{DC-KERR}}
Only response functions connected to the static electric field induced
Kerr effect\index{electric field!induced Kerr} are computed,
$\gamma(-\omega;\omega,0,0)$.

\item{\Key{IDRI  }}
Only response functions connected to the intensity dependent 
refractive\index{refractive index!intensity dependent} index are computed,
$\gamma(-\omega;\omega,-\omega,\omega)$.

\item{\Key{DIPLEN}}
Sets $A$, $B$, $C$, and $D$ to dipole operators\index{dipole length}.

\item{\Key{DIPLNX}}
Sets $A$, $B$, $C$, and $D$ to the $x$ dipole operator\index{dipole length}.

\item{\Key{DIPLNY}}
Sets $A$, $B$, $C$, and $D$ to the $y$ dipole operator\index{dipole length}.

\item{\Key{DIPLNZ}}
Sets $A$, $B$, $C$ and $D$ to the $z$ dipole operator\index{dipole length}.

\item{\Key{MAXIT }}
Maximum number of iterations for solving linear equations, default value is 20.

\item{\Key{MAXITO}}
Maximum number of ORPCTL-microiterations, default value is 10.

\item{\Key{PRINT}}
Print flag for output, default value is 2. Timer information is printed
out if print flag greater than 5. Response vectors printed out if
print flag greater than 10.

\item{\Key{THCLR}}
Threshold for convergence of response vectors, default value is $10^{-4}$.

\item{\Key{THRNRM}}
Threshold for norm of property vector $X^{[1]}$ in order to solve the linear
equation \\
$\left( E^{[2]} - S^{[2]} \right)N^{X} = X^{[1]}$, default
value is $10^{-9}$. 

%hjaaj June 2001: is triplet tested ?? (was not listed in dalton1.1 manual)
%item{\Key{ISABCD}}\\
%verb|READ *, ISPINA,ISPINB,ISPINC,ISPIND|\\

%hjaaj June 2001
%\Key{INVEXP} is a programmers test option 

\end{description}

\subsection{Cubic response calculation of third-order transition moments:
\Sec{CUBIC} with \Key{SINGLE RESIDUE}}
Calculation of single residues\index{single residue!cubic response} of
cubic response functions\index{cubic response!single residue}\index{response!cubic, single residue}
\cite{pndjovhacpl242,djpnhajcp105,pndjhapdkrthhkcpl253}.
$A,B$,$C$, and $D$-named options refer to the operators in the cubic
response function 
$\langle\!\langle A;B,C,D \rangle\!\rangle_{\omega_b,\omega_c,\omega_d}$

\begin{description}

\item{\Key{SINGLE}}
Computes the single residue\index{single residue!cubic response} of the cubic
response function\index{cubic response!single residue}.
In the case of dipole operators this corresponds to
three-photon absorption\index{three-photon!absorption}.

\item[\Key{APROP}, \Key{BPROP}, \Key{CPROP}]
Specify the operators $A$, $B$, and $C$, respectively.
The line following this
option should be the label of the operator as it appears in the file
AOPROPER.

\item[\Key{BFREQ}, \Key{CFREQ}]
The frequencies\index{frequency!cubic response single residue}
$\omega_b$ and $\omega_c$, respectively. Input as in
\Key{FREQUE}.

\item{\Key{FREQUE}}\\
\verb|READ *, NFREQ|\\
\verb|READ *, FREQ(1:NFREQ)|\\
Sets the frequencies\index{frequency!cubic response} whenever a optical process is specified.
Can also be used for the residue calculation and it does then set 
both $\omega_b$ and $\omega_c$ for the single residue and only
$\omega_b$ for the double residue.

\item{\Key{DIPLEN}}
Sets $A$, $B$, $C$, and $D$ to dipole operators\index{dipole length}.

\item{\Key{DIPLNX}}
Sets $A$, $B$, $C$, and $D$ to the $x$ dipole operator\index{dipole length}.

\item{\Key{DIPLNY}}
Sets $A$, $B$, $C$, and $D$ to the $y$ dipole operator\index{dipole length}.

\item{\Key{DIPLNZ}}
Sets $A$, $B$, $C$ and $D$ to the $z$ dipole operator\index{dipole length}.

\item{\Key{MAXIT}}
Maximum number of iterations for solving linear equations, default value is 20.

\item{\Key{MAXITO}}
Maximum number of ORPCTL-microiterations, default value is 10.

\item{\Key{PRINT}}
Print flag for output, default value is 2. Timer information is printed
out if print flag greater than 5. Response vectors printed out if
print flag greater than 10.

\item{\Key{THCLR}}
Threshold for convergence of response vectors, default value is $10^{-4}$.

\item{\Key{MAXITP}}
Maximum number of iteration for solving eigenvalue equation, default
value is 20.

\item{\Key{ROOTS}}
Number of roots (excited states) to converge. \\
\verb|READ (LUCMD,*) (NTMCNV(J),J=1,NSYM)|\\

\item{\Key{THCPP}}
Threshold for convergence of eigenvector, default value is $10^{-6}$.

\end{description}


\subsection{Cubic response calculation of second order moments 
between excited states and excited state polarizabilities:
\Sec{CUBIC} with \Key{DOUBLE RESIDUE}}
Calculation of double residues\index{double residue!cubic response} of
cubic response functions\index{cubic response!double residue}\index{response!cubic, double residue}
\cite{pndjovhacpl242,djpnhajcp105,pndjhapdkrthhkcpl253}.
$A,B$,$C$, and $D$-named options refer to the operators in the cubic
response function 
$\langle\!\langle A;B,C,D \rangle\!\rangle_{\omega_b,\omega_c,\omega_d}$.
$C$ and $D$ refer to the left hand state and right hand state
after the double residue has been taken.

Excited state polarizabilites are only calculated if one or more of the keywords
\Key{DIPLEN}, \Key{DIPLNX}, \Key{DIPLNY}, and \Key{DIPLNZ}
are specified.
Only singlet excitations and singlet property operators are implemented.

\begin{description}

\item{\Key{DOUBLE}}
REQUIRED.
Computes the double\index{double residue} residue of the cubic
response function\index{cubic response}\index{response!cubic}.
In the case of dipole operators this corresponds to excited
state polarizabilities and two-photon transition
moments\index{two-photon!transition moment!excited states}\index{excited state!polarizability} 
between excited states \cite{djpnylhajcp105}.

\item{\Key{ROOTS}}
Number of roots (excited states) to converge for each spatial symmetry. \\
\verb|READ (LUCMD,*) (NTPCNV(J),J=1,NSYM)|\\
Default: one of each symmetry.

\item[\Key{APROP}, \Key{BPROP}]
Specify the operators $A$ and $B$, respectively. The line following this
option should be the label of the operator as it appears in the file
AOPROPER. These two keywords can be repeated for different properties.

\item[\Key{BFREQ}]
The frequencies\index{frequency!cubic response}
$\omega_b$. Input as in \Key{FREQUE}.
Default only zero frequency (static).

\item{\Key{FREQUE}}\\
\verb|READ *, NFREQ|\\
\verb|READ *, FREQ(1:NFREQ)|\\
Sets the frequencies\index{frequency!cubic response} whenever a optical process is specified.
Can also be used for the residue calculation and it does then set 
both $\omega_b$ and $\omega_c$ for the single residue and only
$\omega_b$ for the double residue.
Default only zero frequency (static).

\item{\Key{DIPLEN}}
Sets $A$ and  $B$ to dipole operators\index{dipole length}.

\item{\Key{DIPLNX}}
Sets $A$ and $B$ to the $x$ dipole operator\index{dipole length}.

\item{\Key{DIPLNY}}
Sets $A$ and $B$ to the $y$ dipole operator\index{dipole length}.

\item{\Key{DIPLNZ}}
Sets $A$ and $B$ to the $z$ dipole operator\index{dipole length}.

\item{\Key{MAXIT}}
Maximum number of iterations for solving linear equations, default value is 60.

\item{\Key{MAXITO}}
Maximum number of ORPCTL-microiterations, default value is 10.

\item{\Key{PRINT}}
Print flag for output, default value is 2. Timer information is printed
out if print flag greater than 5. Response vectors printed out if
print flag greater than 10.

\item{\Key{THCLR}}
Threshold for convergence of response vectors, default value is $10^{-4}$.

\item{\Key{MAXITP}}
Maximum number of iteration for solving eigenvalue equation, default
value is 20.

\item{\Key{THCPP}}
Threshold for convergence of eigenvector, default value is $10^{-6}$.

\end{description}

\subsection{Module for C6, C8, C10 coefficients and more\Sec{C6}}


\begin{description}

\item{\Key{C6SPH}, \Key{C8SPH}, \Key{C10SPH}}
Specification of one of \Key{C6SPH}, \Key{C8SPH}, \Key{C10SPH}
calculates and writes to a formatted interface file (RESPONSE.C8) the spherical multipole
moments in the specified/default grid points needed for C6, C8, and C10
coefficients, respectively ($L=1$, $L=1,2,3$, or $L=1,2,3,4,5$;
all for $M_L = -L,\ldots,0,\ldots,L$).

\item{\Key{C6ATM}, \Key{C8ATM}, \Key{C10ATM}}
\Key{C6ATM}, \Key{C8ATM}, \Key{C10ATM} do the same as \Key{C6SPH} etc. for
atoms. Only $M_L=0$ is
calculated and written to file (all $M_L$ values give same multipole moment 
for atoms).

\item{\Key{C6LMO}, \Key{C8LMO}, \Key{C10LMO}}
\Key{C6LMO}, \Key{C8LMO}, \Key{C10LMO} is \Key{C6SPH} etc. for linear
molecules\index{linear molecule}. Only
multipole moments\index{multipole moment} with zero or positive $M_L$
are calculated and written to file.

\end{description}


\noindent{\bf Comments:}

You must tell the integral module to calculate the necessary one-electron integrals.
For \Key{C8SPH}, \Key{C8ATM}, or \Key{C8LMO} you will need

\begin{verbatim}
**INTEGRALS
.SPHMOM
   3
\end{verbatim}

which calculate spherical moments for $L = 0, \ldots, 3$.
For the \Key{C6xxx} and the \Key{C10xxx} options
you will need $L = 0, 1$ and $L = 0, \ldots, 5$, respectively.

\subsection{Electron Spin Resonance: \Sec{ESR}}

Calculation of ESR parameters\index{ESR}

\subsubsection{Hyperfine coupling}

Default\index{hyperfine coupling}: hyperfine coupling tensors using 
the Restricted-Unrestricted Approach\index{restricted-unrestricted method}.

\begin{description}
\item{\Key{FCCALC}} \\
Calculate the isotropic Fermi-contact contributions to hyperfine coupling tensors
\item{\Key{SDCALC}} \\
Calculate the spin-dipole contributions to the hyperfine coupling tensor
\item{\Key{ATOMS}} \\
\verb|READ (LUCMD,'(I5)'),ESRNUC |\\
\verb|READ (LUCMD,*) (NUCINF(IG), IG = 1, ESRNUC) | \\
Select atoms for which to calculate hyperfine coupling constants.
The first line contains  the number of atoms and the second line the
index of each atom (ordered as in the molecule input file \texttt{MOLECULE.INP})
\item{\Key{MAXIT}}      \\
\verb|READ (LUCMD,'(I5)'),MAXESR |\\
   The line following gives the maximum number of iterations.  (Default = 20)

\item{\Key{PRINT}}     \\
\verb|READ (LUCMD,'(I5)'),IPRESR |\\
   The line following gives the print level for ESR routines.

\item{\Key{THCESR}}     \\
\verb|READ (LUCMD,*),THCESR|\\
   The line following is the threshold for convergence (Default = 1.0D-5)

\end{description}
The following options are obsolete but are kept for backward compatibility.
They are replaced by \Key{FCCALC} and \Key{SDCALC} above which also
enables the printing of the tensors of the most important isotopes of the
atoms is established units.
\begin{description}

\item{\Key{SNGPRP}}    \\
\verb|READ (LUCMD,'(A)'), LABEL|\\
   Singlet Operator. The line following is the label in the AOPROPER file.

\item{\Key{TRPPRP}}    \\
\verb|READ (LUCMD,'(A)'), LABEL |\\
   Triplet Operator. The line following is the label in the AOPROPER file.


\end{description}

\subsubsection{Zero-field splitting: \Key{ZFS}}

Calculation of the spin-spin contribution to the zero-field splitting
tensor:

\begin{description}
  \item{\Key{ZFS}} \\
\end{description}

\subsubsection{Electronic g-tensors:  \Key{G-TENSOR}}
\label{sec:g-tensor}
Calculation of the electronic g-tensor:

\begin{description}
  \item{\Key{G-TENSOR}} \\
   Initializes input block for g-tensor related options
\end{description}
The default is to calculate all contributions. The following
options selects individual contributions
\begin{description}
   \item{\Key{RMC}}
   Relativistic mass correction
   \item{\Key{OZSO1}}
   Second-order (paramagnetic) orbital-Zeeman + 1-electron spin-orbit contributions
   \item{\Key{OZSO2}}
   Second-order (paramagnetic) orbital-Zeeman + 2-electron spin-orbit contributions
   \item{\Key{GC1}}
   1-electron gauge correction (diamagnetic) contributions
   \item{\Key{GC2}}
   2-electron gauge correction (diamagnetic) contributions
   \item{\Key{ECC}}
   Choose electron center of charge (ECC) as gauge origin.
\end{description}
The following are utility options for modifying the default calculational
procedure.
\begin{description}
%  \item{\Key{ZERO}}
%  \verb|READ(LUCMD,'(A80)')G_LINE |\\
%  This option is mainly for linear molecules in a $\Sigma$ state.
%  Specifying e.g. "ZZ" on the input line instructs the program
%  to skip the calculation of the paramagnetic contribution to $g_{zz}$.
   \item{\Key{MEAN-FIELD}}
   Uses the approximate atomic mean field (AMFI) spin-orbit operator
   for evaluating the paramagnetic contributions.
   \item{\Key{SCALED}}
   Uses the approximate 1-electron spin-orbit operator with scaled nuclear
   charges for evaluating the paramagnetic contributions.
%  \item{\Key{ADD-SO}}
%  Adds the 1- and 2-electron spin-orbit operators. This 
%  option may be used when one is not interested in the individual
%  1- and 2-electron contribution to the paramagnetic g-tensor,
%  since it reduces the number of response equations to be solved.
\end{description}



%%%%%%% CC %%%%%%%
%\chapter{Coupled-cluster calculations, CC}\label{ch:CC}
\index{Coupled Cluster}
\index{CCS}
\index{CC2}
\index{CCD}
\index{CCSD}
\index{CCSD(T)}
\index{CC3}

The coupled cluster module {\cc} is designed for large-scale
correlated calculations of energies and properties using a
hierarchy of coupled cluster models: CCS, CC2, CCSD, and CC3, as well as standard methods such as MP2 and CCSD(T).
%\typeout{SONIA: Mention CCSD(T) more explicitly as well???}
The module offers almost the same options as the other
parts of the \dalton\ program for SCF and MCSCF wave functions.
It thus contains a wave function optimization section and 
a response function section
where linear, quadratic and cubic response functions and electronic
transition properties are calculated.
At the moment, molecular gradients are available 
up to the CCSD(T) level\index{gradient} for ground states.  
London orbitals have so far not been implemented 
for the calculation of magnetic
properties\index{magnetic properties}\index{London orbitals}.
However, gauge-invariant magnetic properties can be calculated using the
CTOCD-DZ method~\cite{ctocd,pccpcctocd}.
In the manual, more details can be found about the specific implementations.

An additional feature of the module is that all levels of correlation
treatment have been implemented using integral-direct techniques making
it possible to run calculations using large basis
sets~\cite{directCC}\index{integral direct}.

In this chapter the general structure of the input for the
coupled cluster module is described.
The complete input for the coupled cluster module appears as
sections in the input for the \sir\ module, with the general
input in the input section \Sec{CC INPUT}. In order to get into the \cc\ module
one has to specify the \Key{CC} keyword in the general input
section of \sir\ \Sec{*WAVE FUNCTIONS}. For instance, a minimal
input file for a CCSD(T) energy calculation would be:
\begin{verbatim}
**DALTON INPUT
.RUN WAVE FUNCTIONS
**WAVE FUNCTIONS
.CC
*CC INPUT
.CC(T)
**END OF DALTON INPUT
\end{verbatim}

Several models can be calculated at the same time by specifying more models
in the CC input section. 
The models supported in the \cc\ module are 
CCS\cite{Christiansen:CPL243},
MP2\cite{Moller34},
CC2\cite{Christiansen:CPL243},
the CIS(D) excitation energy approximation \cite{Head-Gordon:94},
CCSD\cite{Purvis82},
the CCSDR(3) excitation energy approximation\cite{Christiansen:PERTURBATIVE_TRIPLES}, 
CCSD(T)\cite{Raghavachari89}, and CC3\cite{Christiansen:JCP103,Koch:JCP106}.
Several electronic properties can also 
be calculated in one calculation by specifying simultaneously 
the various input sections in the *CC INPUT section
as detailed in the following sections.

There is also a possibility for performing cavity coupled cluster
self-consistent-reaction field calculations for solvent modeling.

%%%%%%%%%%%%%%%%%%%%%%%%%%%%%%%%%%%%%%%%%%%%%%%%%%%%%%%%%%%%%%%%%%%%
\section{General input for CC: \Sec{CC INPUT}}\label{sec:ccgeneral}
%%%%%%%%%%%%%%%%%%%%%%%%%%%%%%%%%%%%%%%%%%%%%%%%%%%%%%%%%%%%%%%%%%%

In this section keywords for the coupled
cluster program are defined. In particular the coupled cluster 
model(s) and other parameters common to all submodules are specified.

\begin{description}
\item[\Key{B0SKIP}] 
   Skip the calculation of the $B0$-vectors 
   in a restarted run. See comments under \verb|.IMSKIP|.
%
\item[\Key{CC(2)}]  
        Run calculation for non-iterative CC2 excitation energy model 
        (which is actually the well-known CIS(D) model).
 
\item[\Key{CC(3)}]  
       Run calculation for CC(3) ground state energy model.

\item[\Key{CC(T)}]  
        Run calculation for CCSD(T) (Coupled Cluster Singles and 
        Doubles with perturbational treatment of triples) model.
        \index{CCSD(T)}
%
\item[\Key{CC2}]    
        Run calculation for CC2 model. 
        \index{CC2}
 
\item[\Key{CC3}]     
        Run calculation for CC3 model.
        \index{CC3}
%
\item[\Key{CCD}]    
        Run calculation for Coupled Cluster Doubles
        (CCD) model. 
        \index{Coupled Cluster!Doubles}
        \index{CCD}
%
\item[\Key{CIS}]    
        Run calculation for CI Singles (CIS) method. 
        \index{Configuration Interaction!Singles}
        \index{CI!Singles}
        \index{CIS}
%
\item[\Key{CHO(T)}]  
        Run calculation for CCSD(T) (Coupled Cluster Singles and 
        Doubles with perturbational treatment of triples) model
        using Cholesky-decomposed orbital energies denominators.
        Directives to control the calculation can be specifed in the
        \Sec{CHO(T)} input module.
        \index{CCSD(T)}
%
\item[\Key{CCR(3)}] 
        Run calculation for CCSDR(3) excitation energy model.
%
\item[\Key{CCS}] 
        Run calculation for Coupled Cluster Singles
        (CCS) model. 
        \index{Coupled Cluster!Singles}
        \index{CCS}
%
\item[\Key{CCSD}]   
        Run calculation for Coupled Cluster Singles and Doubles
        (CCSD) model. 
        \index{Coupled Cluster!Singles and Doubles}
        \index{CCSD}
%
%
\item[\Key{MTRIP}]   
        Run calculation of modified triples corrections for (MCCSD(T) for CCSD(T) and MCC(3) for CC(3))
        meaning that if the ground state lagrangian multipliers are calculated 
        they are used instead of amplitudes. Specialists option. 
        Do not use with analytical gradients. 
%
\item[\Key{CCSTST}] 
   Test option which runs the CCS finite field calculation as a pseudo CC2
   calculation. \verb+CCSTST+ disables all terms which depend on the
   double excitation amplitudes or multipliers. This flag must be
   set for CCS finite field calculations.
   \index{CCS}
   \index{finite field}
%
\item[\Key{DEBUG}]  
   Test option: print additional debug output.
%
\item[\Key{E0SKIP}] 
   Skip the calculation of the $E0$-vectors 
   in a restarted run. See comment under \verb|.IMSKIP|.
%
\item[\Key{F1SKIP}] 
   Skip the calculation of F-matrix transformed first-order
   cluster amplitude responses in a restarted run. See comment under \verb|.IMSKIP|.
%
\item[\Key{F2SKIP}] 
   Skip the calculation of F-matrix transformed second-order
   cluster amplitude responses in a restarted run. See comment under \verb|.IMSKIP|.
%
\item[\Key{FIELD}] \verb| |\newline
    \verb|READ (LUCMD,*) EFIELD|\newline
    \verb|READ (LUCMD,'(1X,A8)') LFIELD|

    Add an external finite field (operator label \verb+LFIELD+)
    of strength \verb+EFIELD+ to the Hamiltonian
    (same input format as in Sec.~\ref{ref-haminp}).
    These fields are only included in the CC calculation and 
    not in the calculation of the SCF reference state 
    ({\it i.e.\/} the orbitals). 
    Using this option in the calculation of numerical derivatives 
    thus gives so-called orbital-unrelaxed energy derivatives, 
    which is standard in coupled cluster response function theory.
    Note that this way of adding an external field does not work for 
    models including triples excitations (most notably CCSD(T) and CC3).
    For finite field calculations of orbital-relaxed energy 
    derivatives the field must be included in the SCF calculation.
    For such calculations, use the input section \Sec{HAMILTONIAN} (Sec.~\ref{ref-haminp}).
    \index{finite field}
    \index{external field}
 
\item[\Key{FREEZE}] \verb| |\newline
      \verb|READ (LUCMD,*) NFC, NFV|

     Specify the number of frozen core orbitals and number of frozen virtuals (0 is recommended for the latter).
     The program will automatically freeze the NFC(NFV) orbitals of lowest (highest) orbital energy
     among the canonical Hartree--Fock orbitals.
     For benzene a relevant choice is for example 6 0.

\item[\Key{FROIMP}] \verb| |\newline
      \verb|READ (LUCMD,*) (NRHFFR(I),I=1,MSYM)|\newline
      \verb|READ (LUCMD,*) (NVIRFR(I),I=1,MSYM)|

      Specify for each irreducible representation how
      many orbitals should be frozen (deleted) for the coupled
      cluster calculation. In calculations, the first \verb+NRHFFR(I)+
      orbitals will be kept frozen in symmetry class \verb+I+ and
      the last \verb+NVIRFR(I)+ orbitals will be deleted from the 
      orbital list.
 
\item[\Key{FROEXP}]  \verb| |\newline
    \verb|READ (LUCMD,*) (NRHFFR(I),I=1,MSYM)|\newline
    \verb|DO ISYM = 1, MSYM|\newline
    \verb|  IF (NRHFFR(ISYM.NE.0) THEN|\newline
    \verb|    READ (LUCMD,*) (KFRRHF(J,ISYM),J=1,NRHFFR(ISYM))|\newline
    \verb|  END IF|\newline
    \verb|END DO|\newline
    \verb|READ (LUCMD,*) (NVIRFR(I),I=1,MSYM)|\newline
    \verb|DO ISYM = 1, MSYM|\newline
    \verb|  IF (NVIRFR(ISYM.NE.0) THEN|\newline
    \verb|    READ (LUCMD,*) (KFRVIR(J,ISYM),J=1,NVIRFR(ISYM))|\newline
    \verb|  END IF|\newline
    \verb|END DO|

    Specify explicitly for each irreducible representation the
    orbitals that should be frozen (deleted) in the coupled cluster
    calculation.
 
\item[\Key{FRSKIP}] 
   Skip the calculation of the F-matrix transformed right eigenvectors
   in a restarted run. See comment under \verb|.IMSKIP|.
%
\item[\Key{HERDIR}] 
       Run coupled cluster program AO-direct\index{integral direct}
       using {\her}
       (Default is to run the program only AO integral direct
       if the \verb+DIRECT+ keyword was set in the general
       input section of \dalton . If \verb+HERDIR+ is not specified the \eri\
       program is used as integral generator.) 
%
\item[\Key{IMSKIP}] 
   Skip the calculation of some response intermediates in a restarted run.
   This options and the following skip options is primarily for very experienced users 
   and programmers who want to save a little bit of CPU time in restarts on very large 
   calculations. 
   {\em It is assumed that the user knows what he is doing when using these options, and
   the program does not test if the intermediates are there or are correct. 
   The relevant quantities are assumed to be at the correct directory and files.
   These comments also applies for all the other skip options following! }
%
\item[\Key{L0SKIP}]  
   Skip the calculation of the zeroth-order ground-state Lagrange
   multipliers in a restarted run. See comment under \verb|.IMSKIP|.
%
\item[\Key{L1SKIP}] 
   Skip the calculation of the first-order responses of the 
   ground-state Lagrange multipliers in a restarted run. See comment under \verb|.IMSKIP|.
%
\item[\Key{L2SKIP}]  
   Skip the calculation of the second-order responses of the 
   ground-state Lagrange multipliers in a restarted run. See comment under \verb|.IMSKIP|.
%
\item[\Key{LESKIP}]  
   Skip the calculation of left eigenvectors
   in a restarted run. See comment under \verb|.IMSKIP|.
%
\item[\Key{LISKIP}] 
   Skip the calculation of (ia$\mid$jb) integrals in a restarted run. See comment under \verb|.IMSKIP|.
%
\item[\Key{M1SKIP}] 
   Skip the calculation of the special zeroth-order Lagrange 
   multipliers for ground-excited state transition moments,
   the so-called $M$-vectors, in a restarted run. See comment under \verb|.IMSKIP|.
%
\item[\Key{MAX IT}] \verb| |\newline
  \verb|READ (LUCMD,'(I5)') MAXITE|

  Maximum number of iterations for wave function optimization 
  (default is \verb+MAXITE = 40+).
 
\item[\Key{MAXRED}] \verb| |\newline 
  \verb|READ (LUCMD,*) MAXRED|

  Maximum dimension of the reduced space for the 
  solution of linear equations (default is \verb+MAXRED = 200+).
 
%\item[\Key{CC1A}]   
%        Run calculation for CCSDT-1A model.  Not recommended for general use.
%
%\item[\Key{CC1B}]    
%        Run calculation for CCSDT-1B model.  Not recommended for general use.
%
\item[\Key{MP2}]    
Run calculation for second-order M{\o}ller-Plesset perturbation theory
(MP2)\index{M{\o}ller-Plesset!second-order}\index{MP2} method.

\item[\Key{MXDIIS}] \verb| |\newline
  \verb|READ (LUCMD,*) MXDIIS|

  Maximum number of iterations for DIIS algorithm
  before wave function information is discarded and a new DIIS 
  sequence is started
  (default is \verb+MXDIIS = 8+).
 
\item[\Key{MXLRV}] \verb| |\newline
  \verb|READ (LUCMD, *) MXLRV|

  Maximum number of trial vectors in the solution of 
  linear equations. If the number of trial vectors reaches this
  value, all trial vectors and their transformations with the
  Jacobian are skipped and the iterative procedure for the solution of the
  linear (i.e. the response) equations is restarted from the current 
  optimal solution. 
 
\item[\Key{NOCCIT}]
   No iterations in the wave function optimization is carried out.

\item[\Key{NSIMLE}] \verb| |\newline
  \verb|READ (LUCMD, *) NSIMLE|

  Set the maximum number of response equations that should be 
  solved simultaneously. Default is 0 which means that all
  compatible equations (same equation type and symmetry class) 
  are solved simultaneously.
 
\item[\Key{NSYM}] \verb| |\newline
       \verb|READ (LUCMD,*) MSYM2|

       Set number of irreducible representations. 
%       Due to difficulties to parse the number of irrep's from
%       the integral program to the coupled cluster input section,
%       this information must be specified in the coupled cluster
%       input section, if one of the options \verb+.FROIMP+ or
%       \verb+.NCCEXCI+ (see \verb+*CCEXCI+ subsection) are to be used.
%       Note that \verb+.NSYM+ has to be set before these keywords.
 
\item[\Key{O2SKIP}] 
   Skip the calculation of right-hand side vectors for the 
   second-order cluster amplitude equations
   in a restarted run. See comment under \verb|.IMSKIP|.
%
%\item[\Key{CCR(A)}] 
%        Run calculation for CCSDR(A) model. Not recommended for general use.
%
%\item[\Key{CCR(B)}]  
%        Run calculation for CCSDR(B) model. Not recommended for general use.

%
%\item[\Key{CCR(T)}] 
%        Run calculation for CCSDR(T) model. Not recommended for general use.
%
\item[\Key{PAIRS}]
         Decompose the singles and/or doubles coupled cluster correlation energy
         into contributions from singlet- and triplet-coupled
         pairs of occupied orbitals. Print those pair energies.

\item[\Key{PRINT}]  \verb| |\newline
\verb|READ (LUCMD,'(I5)') IPRINT|

       Set print parameter for coupled cluster program
       (default is to take the value \verb+IPRUSR+, set in the general
       input section of Dalton).
%
\item[\Key{R1SKIP}] 
   Skip the calculation of the first-order amplitude responses
   in a restarted run. See comment under \verb|.IMSKIP|.
%
\item[\Key{R2SKIP}] 
   Skip the calculation of the 
   second-order cluster amplitude equations
   in a restarted run. See comment under \verb|.IMSKIP|.
%
\item[\Key{RESKIP}] 
   Skip the calculation of the first-order responses of the 
   ground-state Lagrange multipliers in a restarted run. See comment under \verb|.IMSKIP|.
%
\item[\Key{RESTART}] 
       Try to restart the calculation from the cluster amplitudes,
       Lagrange multipliers, response amplitudes etc.\ stored on
       disk.
%
\item[\Key{SOPPA(CCSD)}] 
       Write the CCSD singles and doubles amplitudes on the Sirius interface
       for later use in a SOPPA(CCSD)\index{SOPPA(CCSD)} calculation. This
       chooses automatically the Coupled Cluster Singles and Doubles (CCSD) model.
%
\item[\Key{THRENR}] \verb| |\newline
       \verb|READ (LUCMD,*) THRENR|

       Set threshold for convergence of the ground state energy.
 
\item[\Key{THRLEQ}] \verb| |\newline
       \verb|READ (LUCMD,*) THRLEQ|

       Set threshold for convergence of the response equations.
 
\item[\Key{THRVEC}] \verb| |\newline
       \verb|READ (LUCMD,*) THRVEC|

       Set threshold for convergence of the ground state CC vector function.
 
%
%\item[\Key{NEWCAU}] % unfinished  option !
%   Solve Cauchy equations with different Cauchy order simultaneous. 
%   (The algorithm behind this is still in the test phase...)
%
\item[\Key{X2SKIP}] 
   Skip the calculation of the $\eta^{(2)}$ intermediates (needed
   to build the right-hand side vectors for the second-order 
   ground state Lagrange multiplier response equations) 
   in a restarted run. See comment under \verb|.IMSKIP|.
%
\end{description}

%All the special skip options for the restart of response calculations
%would better be placed in a separate input subsection.
%(And the corresponding logical variables on a separate 
%common block.)

%%%%%%%%%%%%%%%%%%%%%%%%%%%%%%%%%%%%%%%%%%%%%%%%%%%%%%%%%%%%%%%%%%%%
\section{Ground state first-order properties: \Sec{CCFOP}}
\label{sec:ccfop}
%%%%%%%%%%%%%%%%%%%%%%%%%%%%%%%%%%%%%%%%%%%%%%%%%%%%%%%%%%%%%%%%%%%
\index{expectation values}
\index{one-electron properties}
\index{first-order properties}

In this Section, the calculation of ground-state first-order
(one-electron) properties is described. The calculation
is evoked with the \Sec{CCFOP} flag followed by the appropriate
keywords as described in the list below. Note that \Sec{CCFOP}
assumes that the proper integrals are written on the
AOPROPER file, and one therefore has to set the correct property
integral keyword(s) in the \Sec{*INTEGRAL} input Section. For properties
that have both an electronic and a nuclear contribution, these will
be printed separately with a print level of 10 or above.
\index{AOPROPER}

The calculation of first order properties is implemented for the 
coupled cluster models CCS (which gives SCF first order properties),
CC2, MP2 and CCSD.  
By default, the chosen properties include orbital relaxation
contributions \index{orbital relaxation},
i.e. they are calculated from the relaxed CC (or MP) densities. 
To disabilitate orbital relaxation for the CC2 and CCSD models, 
see \Key{NONREL} below. Relaxation is always included for MP2. 
Note also that the present implementation does not allow for 
CC2 relaxed first-order
properties in the frozen core approximation 
(\Key{FROIMP}, see Sec.~\ref{sec:ccgeneral}).

For details on the implementation, see 
Refs.~\cite{Halkier:CCFOP,Halkier:CC2RLXFOP}
Publications that report results obtained with this module
should cite Ref.\ \cite{Halkier:CCFOP}.

\begin{description}
\item[\Key{DIPMOM}] 
        Calculate the permanent molecular electric dipole moment
        (\verb+DIPLEN+ integrals).
        \index{electric dipole}
        \index{dipole moment}
%
\item[\Key{QUADRU}] 
        Calculate the permanent traceless molecular electric
        quadrupole moment (\verb+THETA+ integrals). Note that the
        origin is the origin of the coordinate system specified
        in the MOLECULE.INP file.
        \index{electric quadrupole}
        \index{quadrupole moment}
%
\item[\Key{NQCC  }] 
        Calculate the electric field gradients at the nuclei
        (\verb+EFGCAR+ integrals).
        \index{electric field gradient}
%
\item[\Key{SECMOM}] 
        Calculate the electronic second moment of charge
        (\verb+SECMOM+ integrals).
        \index{second moment of charge}
%
\item[\Key{RELCOR}] 
        Calculate scalar-relativistic one-electron
        corrections to the ground-state
        energy (\verb+DARWIN+ and \verb+MASSVELO+ integrals).
        \index{relativistic corrections, one-electron}
        \index{Darwin term, one-electron}
        \index{mass-velocity term}
%
\item[\Key{ALLONE}] 
        Calculate all of the above-mentioned properties (all the
        above-mentioned property integrals are needed).
%
\item[\Key{DAR2EL}] 
        Calculate relativistic two-electron Darwin term.
        \index{Darwin term, two-electron}
%
\item[\Key{OPERAT}] \verb| |\newline
\verb|READ (LUCMD,'(1X,A8)') LABPROP|\newline
        Calculate the electronic contribution to the property defined
        by the operator label \verb+LABPROP+ (corresponding 
        \verb+LABPROP+ integrals needed).
%
\item[\Key{NONREL}] 
        Compute the properties using the unrelaxed CC densities instead
        of the default relaxed densities.
%
\item[\Key{TSTDEN}] 
        Calculate the CC energy using the two-electron CC density.
        Programmers keyword used for debugging purposes---Do not use.
%
\end{description}

%
%%%%%%%%%%%%%%%%%%%%%%%%%%%%%%%%%%%%%%%%%%%%%%%%%%%%%%%%%%%%%%%%%%%
\section{Linear response functions: \Sec{CCLR}}\label{sec:cclr}
%%%%%%%%%%%%%%%%%%%%%%%%%%%%%%%%%%%%%%%%%%%%%%%%%%%%%%%%%%%%%%%%%%%
\index{linear response}
\index{polarizabilities, frequency-dependent}
\index{polarizabilities, coupled cluster}
\index{polarizabilities, static}
\index{dipole polarizability}
\index{dispersion coefficients}
\index{Cauchy moments}

In the \Sec{CCLR  } section the input that is
specific for coupled cluster linear response properties is read in. 
This section includes presently 
\begin{itemize}
\item frequency-dependent linear response properties 
      $\alpha_{AB}(\omega)  = - \langle\langle A; B \rangle\rangle_\omega$
      where $A$ and $B$ can be any of the one-electron
      operators for which integrals are available in the 
      \Sec{*INTEGRALS} input part.
\item dispersion coefficients $D_{AB}(n)$ for $\alpha_{AB}(\omega)$
      which for $n \ge 0$ are defined by the expansion
      $$ \alpha_{AB}(\omega) = \sum_{n=0}^{\infty} \omega^n \, D_{AB}(n) $$
      In addition to the dispersion coefficients for $n \ge 0$
      there are also coefficients available for $ n = -1, \ldots, -4$,
      which are related to the Cauchy moments
       by $ D_{AB}(n) = S_{AB}(-n-2)$.
      \\
      Note, that for real response functions only even moments
      $D_{AB}(2n) = S_{AB}(-2n-2)$ with $n \ge -2$ are available,
      while for imaginary response functions only odd moments
      $D_{AB}(2n+1) = S_{AB}(-2n-3)$ with $n \ge -2$ are available.
\end{itemize}
Coupled cluster linear response functions and dispersion coefficients
are implemented for the models CCS, CC2 and CCSD. 
%Publications that report results obtained with CC linear response
%calculations should cite Ref.\ \cite{Christiansen:CCLR}. 
The theoretical background for the implementation is detailed in Ref.\ \cite{Christiansen:CCLR,Christiansen:QEL,Haettig:CAUCHY}.
%For dispersion coefficients also a citation of Ref.\ \cite{Haettig:CAUCHY} 
%should be included.
The properties calculated are in the approach now generally known as coupled cluster 
linear response---in the frequency-independent limit this coincides with the so-called 
orbital-unrelaxed energy derivatives (and thus the orbital-unrelaxed finite field result).

\begin{center}
\fbox{
\parbox[h][\height][l]{12cm}{
\small
\noindent
{\bf Reference literature:}
\begin{list}{}{}
\item Linear response:  O.~Christiansen, A.~Halkier, H.~Koch, P.~J{\o}rgensen, and T.~Helgaker \newblock {\em J.~Chem.~Phys.}, {\bf 108},\hspace{0.25em}2801, (1998).
\item Dispersion coefficients: C.~H\"{a}ttig, O.~Christiansen, and P.~J{\o}rgensen \newblock {\em J.~Chem.~Phys.}, {\bf 107},\hspace{0.25em}10592, (1997).
\end{list}
}}
\end{center}

\begin{description}
\item[\Key{ASYMSD}] 
Use an asymmetric formulation of the linear response function which
does not require the solution of response equations for the operators $A$, 
but solves two sets of response equations for the operators $B$.
%
% \item[\Key{RELAXE}] 
%
% \item[\Key{UNRELA}] 
%
\item[\Key{AVERAG}] \verb| |\newline
   \verb|READ (LUCMD,'(A)') AVERAGE|\newline
   \verb|READ (LUCMD,'(A)') SYMMETRY|

Evaluate special tensor averages of linear response functions.
Presently implemented are the isotropic average of the dipole polarizability
$\bar{\alpha}$ and the dipole polarizability anisotropy $\alpha_{ani}$.
Specify \verb+ALPHA_ISO+ for \verb+AVERAGE+ to obtain $\bar{\alpha}$ and
\verb+ALPHA_ANI+ to obtain $\alpha_{ani}$ and $\bar{\alpha}$.
The \verb+SYMMETRY+ input defines the selection rules that can be
exploited to reduce the number of tensor elements that have to be
evaluated. Available options are
\verb+ATOM+, \verb+SPHTOP+ (spherical top), \verb+LINEAR+,
\verb+XYDEGN+ ($x$- and $y$-axis equivalent, i.e.\ a $C_z^n$
symmetry axis with $n \ge 3$),  and \verb+GENER+ (use point
group symmetry from geometry input).
\index{dipole polarizability}
 
\item[\Key{DIPOLE}] 
Evaluate all symmetry allowed elements of the dipole polarizability
(max. 6 components).
\index{dipole polarizability}

\item[\Key{DISPCF}] \verb| |\newline
   \verb|READ (LUCMD,*) NLRDSPE|

   Calculate the dispersion coefficients 
   $D_{AB}(n)$ up to $n = $ \verb+NLRDSPE+.
%
\item[\Key{FREQUE}] \verb| |\newline
   \verb|READ (LUCMD,*) NBLRFR |\newline
   \verb|READ (LUCMD,*) (BLRFR(I),I=1,NBLRFR)|

Frequency input for $\langle\langle A;B \rangle\rangle_{\omega}$.
 
%
%\item[\Key{ALLDSP}] 
%
% \item[\Key{FTST  }] 
%
\item[\Key{OPERAT}] \verb| |\newline
   \verb|READ (LUCMD,'(2A)') LABELA, LABELB|\newline
   \verb|DO WHILE (LABELA(1:1).NE.'.' .AND. LABELA(1:1).NE.'*')|\newline
   \verb|  READ (LUCMD,'(2A)') LABELA, LABELB|\newline
   \verb|END DO|

Read pairs of operator labels. 
 
\item[\Key{PRINT }] \verb| |\newline
   \verb|READ (LUCMD,*) IPRSOP|

   Set print level for linear response output.
 
\end{description}

%
%%%%%%%%%%%%%%%%%%%%%%%%%%%%%%%%%%%%%%%%%%%%%%%%%%%%%%%%%%%%%%%%%%%
\section{Quadratic response functions: \Sec{CCQR}}
\label{sec:ccqr}
%%%%%%%%%%%%%%%%%%%%%%%%%%%%%%%%%%%%%%%%%%%%%%%%%%%%%%%%%%%%%%%%%%%
\index{quadrict response}
\index{third-order properties}
\index{hyperpolarizabilities, first}
\index{hyperpolarizabilities, dipole}
\index{dispersion coefficients}

In the \Sec{CCQR} section you have to specify the input for
coupled cluster quadratic response calculations. This section
includes:
\begin{itemize}
\item frequency-dependent third-order properties
      $$\beta_{ABC}(\omega_A;\omega_B,\omega_C) = -
        \langle\langle A; B, C\rangle\rangle_{\omega_B,\omega_C} 
        \qquad \mbox{with~} \omega_A = -\omega_B - \omega_C
       $$
      where $A$, $B$ and $C$ can be any of the one-electron operators
      for which integrals are available in the \Sec{*INTEGRALS} 
      input part.
\item dispersion coefficients $D_{ABC}(n,m)$ for third-order properties,
      which for $n\ge 0$ are defined by the expansion
      $$ \beta_{ABC}(-\omega_B-\omega_C;\omega_B,\omega_C)  = 
        \sum_{n,m=0}^{\infty} \omega_{B}^n \, \omega_{C}^m \, D_{ABC}(n,m) 
      $$
\end{itemize}
The coupled cluster quadratic response function is at present
implemented for the coupled cluster models CCS, CC2 and CCSD.
Publications that report results obtained by CC quadratic response
calculations should cite Ref.\ \cite{Haettig:CCQR}.
For dispersion coefficients also a citation of 
Ref.\ \cite{Haettig:DISPBETA} should be included.

The response functions are evaluated for a number of 
operator triples (given using the
\Key{OPERAT}, \Key{DIPOLE}, or \Key{AVERAG} keywords) 
which are combined with pairs of frequency arguments specified using the 
keywords \Key{MIXFRE}, \Key{SHGFRE}, \Key{ORFREQ}, \Key{EOPEFR}
or \Key{STATIC}. 
The different frequency keywords are 
compatible and might be arbitrarily combined or repeated.
For dispersion coefficients use the keyword \Key{DISPCF}.

\begin{description}
\item[\Key{OPERAT}] \verb| |\newline
\verb|READ (LUCMD'(3A)') LABELA, LABELB, LABELC|\newline
\verb|DO WHILE (LABELA(1:1).NE.'.' .AND. LABELA(1:1).NE.'*')|\newline
\verb|  READ (LUCMD'(3A)') LABELA, LABELB, LABELC|\newline
\verb|END DO|

Read triples of operator labels. 
For each of these operator triples the quadratic response
function will be evaluated at all frequency pairs.
Operator triples which do not correspond to symmetry allowed
combination will be ignored during the calculation.
 
\item[\Key{DIPOLE}] 
Evaluate all symmetry allowed elements of the first
dipole hyperpolarizability (max. 27 components).
 
\item[\Key{PRINT }] \verb| |\newline
     \verb|READ (LUCMD,*) IPRINT|

     Set print parameter for the quadratic response section.
     % \begin{itemize}
     %   \item[0]  ??
     %   \item[5]  ??
     %   \item[10]  ??
     % \end{itemize}
 
%\\  % Experts only option !
% \item[\Key{EXPCOF}] 
% \latex{\begin{minipage}[t]{13cm}}
% \begin{verbatim}
% READ (LUCMD'(A)') LINE
% DO WHILE (LABELA(1:1).NE.'.' .AND. LABELA(1:1).NE.'*')
%   READ (LINE,*) ICA, ICB, ICC
%   READ (LUCMD'(A)') LINE
% END DO
% \end{verbatim}
% \latex{\end{minipage} \\ [2ex]}
% Read triples of order parameters for the expansion coefficients
% $d_{ABC}(l,m,n)$ of the quadratric response function
% \cite{Haettig:DISPBETA}.
 
\item[\Key{AVERAG}] \verb| |\newline
   \verb|READ (LUCMD,'(A)') LINE|

   Evaluate special tensor averages of quadratic response properties.
   Presently implemented are only the vector averages of the first
   dipole hyperpolarizability $\beta_{||}$, $\beta_{\bot}$ and 
   $\beta_K$. All three of these averages are obtained if 
   \verb+HYPERPOL+ is specified on the input line that follows
   \Key{AVERAG}.
   The \Key{AVERAG} keyword should be used before any \Key{OPERAT} 
   or \Key{DIPOLE} input in the \Sec{CCQR} section.
 
\item[\Key{MIXFRE}]  \verb| |\newline
   \verb|READ (LUCMD,*) MFREQ|\newline
   \verb|READ (LUCMD,*) (BQRFR(IDX),IDX=NQRFREQ+1,NQRFREQ+MFREQ)|\newline
   \verb|READ (LUCMD,*) (CQRFR(IDX),IDX=NQRFREQ+1,NQRFREQ+MFREQ)|

   Input for general frequency mixing 
   $\beta_{ABC}(-\omega_B-\omega_C;\omega_B,\omega_C)$: on the first line 
   following \Key{MIXFRE} the number of differenct frequency
   (for this keyword) is read, from the second and third line
   the frequency arguments $\omega_B$ and $\omega_C$ are read
   ($\omega_A$ is set to $-\omega_B-\omega_C$).

\item[\Key{SHGFRE}]  \verb| |\newline
   \verb|READ (LUCMD,*) MFREQ|\newline
   \verb|READ (LUCMD,*) (BQRFR(IDX),IDX=NQRFREQ+1,NQRFREQ+MFREQ)|

   Input for second harmonic generation
   $\beta_{ABC}(-2\omega;\omega,\omega)$:
   on the first line following \Key{SHGFRE} the number of different
   frequencies is read, from the second line the input for 
   $\omega_B = \omega$ is read. $\omega_C$ is set to $\omega$ 
   and $\omega_A$ to $-2\omega$.
   \index{second harmonic generation}
   \index{SHG}

\item[\Key{ORFREQ}]  \verb| |\newline
   \verb|READ (LUCMD,*) MFREQ|\newline
   \verb|READ (LUCMD,*) (BQRFR(IDX),IDX=NQRFREQ+1,NQRFREQ+MFREQ)|

   Input for optical rectification $\beta_{ABC}(0;\omega,-\omega)$:
   on the first line following \Key{SHGFRE} the number of different
   frequencies is read, from the second line the input for 
   $\omega_B = \omega$ is read. $\omega_C$ is set to $\omega_C = -\omega$ and
   $\omega_A$ to $0$.
   \index{optical rectification}
   \index{OR}

\item[\Key{EOPEFR}]  \verb| |\newline
   \verb|READ (LUCMD,*) MFREQ|\newline
   \verb|READ (LUCMD,*) (BQRFR(IDX),IDX=NQRFREQ+1,NQRFREQ+MFREQ)|

   Input for the electro optical Pockels effect
   $\beta_{ABC}(-\omega;\omega,0)$:
   on the first line following \Key{SHGFRE} the number of different
   frequencies is read, from the second line the input for 
   $\omega_B = \omega$ is read. $\omega_C$ is set to $0$ and
   $\omega_A$ to $\omega_A = -\omega$.
   \index{Pockels effect, electro optical}
   \index{EOPE}

\item[\Key{STATIC}] 
   Add $\omega_A = \omega_B = \omega_C = 0$ to the frequency list.

\item[\Key{DISPCF}]  \verb| |\newline
   \verb|READ (LUCMD,*) NQRDSPE|

   Calculate the dispersion coefficients
   $D_{ABC}(n,m)$ up to order $n+m =$\verb+NQRDSPE+. 
   \index{dispersion coefficients}
 
% \item[\Key{ALLDSP}]  debug option only!

\item[\Key{XYDEGE}] 
Assume X and Y directions as degenerate in the calculation
of the hyperpolarizability averages (this will prevent
the program to use the components $\beta_{zyy}$, $\beta_{yzy}$
$\beta_{yyz}$ for the computation of the vector averages).

\item[\Key{NOBMAT}] 
Test option:
Do not use B matrix transformations but pseudo F matrix 
transformations (with the zeroth-order Lagrange multipliers 
exchanged by first-order responses) to compute the terms
$\bar{t}^A {\bf B} t^{B} t^{C}$. This is usually less
efficient.

\item[\Key{USE R2}] 
Test option: use second-order response vectors instead of
first-order Lagrange multiplier responses.
 
\end{description}

%
%%%%%%%%%%%%%%%%%%%%%%%%%%%%%%%%%%%%%%%%%%%%%%%%%%%%%%%%%%%%%%%%%%%
\section{Cubic response functions: \Sec{CCCR}}\label{sec:cccr}
%%%%%%%%%%%%%%%%%%%%%%%%%%%%%%%%%%%%%%%%%%%%%%%%%%%%%%%%%%%%%%%%%%%
\index{cubic response}
\index{response!cubic}
\index{Coupled Cluster!cubic response}
\index{fourth-order properties}
\index{properties!fourth-order}
\index{hyperpolarizabilities}
\index{second hyperpolarizability}
\index{hyperpolarizabilities!second}

In the \Sec{CCCR} section the input that is  specific for 
coupled cluster cubic response properties is read in.
This section includes:
\begin{itemize}
\item frequency-dependent fourth-order properties
      $$ \gamma_{ABCD}(\omega_A;\omega_B,\omega_C,\omega_D) = -
           \langle\langle A;B,C,D\rangle\rangle_{\omega_B,\omega_C,\omega_D}
         \quad \mbox{with~} \omega_A = -\omega_B -\omega_C -\omega_D
      $$
      where $A$, $B$, $C$ and $D$ can be any of the one-electron
      operators for which integrals are available in the 
      \Sec{*INTEGRALS} input part.
      \index{fourth-order properties}
\item dispersion coefficients $D_{ABCD}(l,m,n)$ for 
      $ \gamma_{ABCD}(\omega_A;\omega_B,\omega_C,\omega_D) $
      which for $n \ge 0$ are defined by the expansion 
      $$ \gamma_{ABCD}(\omega_A;\omega_B,\omega_C,\omega_D) = 
          \sum_{l,m,n=0}^{\infty} \omega_B^l \, \omega_C^m \, \omega_D^n
            D_{ABCD}(l,m,n) $$
      \index{dispersion coefficients}
\end{itemize}
Coupled cluster cubic response functions and dispersion coefficients
are implemented for the models CCS, CC2 and CCSD.
%Publications that report results obtained with CC cubic response
%calculations should cite Ref.\ \cite{Haettig:CCCR}.
%For dispersion coefficients also a citation of Ref.\
%\cite{Haettig:DISPGAMMA} should be included.

The response functions are evaluated for a number of operator quadruples
(specified with the keywords \Key{OPERAT}, \Key{DIPOLE}, or \Key{AVERAG})
which are combined with triples of frequency arguments specified
using the keywords \Key{MIXFRE}, \Key{THGFRE}, \Key{ESHGFR}, \Key{DFWMFR},
\Key{DCKERR}, or \Key{STATIC}. The different frequency keywords are 
compatible and might be arbitrarely combined or repeated.
For dispersion coefficients use the keyword \Key{DISPCF}.

\begin{center}
\fbox{
\parbox[h][\height][l]{12cm}{
\small
\noindent
{\bf Reference literature:}
\begin{list}{}{}
\item Cubic response: C.~H\"{a}ttig, O.~Christiansen, and P.~J{\o}rgensen \newblock {\em Chem.~Phys.~Lett.}, {\bf 282},\hspace{0.25em}139, (1998).
\item Dispersion coefficients: C.~H\"{a}ttig, and P.~J{\o}rgensen \newblock {\em Adv.~Quantum Chem.}, {\bf 35},\hspace{0.25em}111, (1999).
\end{list}
}}
\end{center}

\begin{description}
\item[\Key{AVERAG}] \verb| |\newline
\verb|READ (LUCMD,'(A)') AVERAGE|\newline
\verb|READ (LUCMD,'(A)') SYMMETRY|

Evaluate special tensor averages of cubic response functions.
Presently implemented are the isotropic averages of the second
dipole hyperpolarizability
$\gamma_{||}$ and $\gamma_{\bot}$.
Set \verb+AVERAGE+ to \verb+GAMMA_PAR+ 
to obtain $\gamma_{||}$ and to
\verb+GAMMA_ISO+ to obtain $\gamma_{||}$ and $\gamma_{\bot}$.
The \verb+SYMMETRY+ input defines the selection rules 
exploited to reduce the number of tensor elements that have to be
evaluated. Available options are
\verb+ATOM+, \verb+SPHTOP+ (spherical top), \verb+LINEAR+,
and \verb+GENER+ (use point group symmetry from geometry input).
Note that the \Key{AVERAG} option should be specified in the \Sec{CCCR}
section before any \Key{OPERAT} or \Key{DIPOLE} input.
 
\item[\Key{DCKERR}] \verb| |\newline
\verb|READ (LUCMD,*) MFREQ|\newline
\verb|READ (LUCMD,*) (DCRFR(IDX),IDX=NCRFREQ+1,NCRFREQ+MFREQ)|

Input for dc-Kerr effect $\gamma_{ABCD}(-\omega;0,0,\omega)$:
on the first line following \Key{DCKERR} the number of different
frequencies are read, from the second line the input for
$\omega_D = \omega$ is read. $\omega_C$ and $\omega_D$ to $0$
and $\omega_A$ to $-\omega$. 
\index{Kerr effect, dc}\index{dc-Kerr effect}
 
\item[\Key{DFWMFR}] \verb| |\newline
\verb|READ (LUCMD,*) MFREQ|\newline
\verb|READ (LUCMD,*) (BCRFR(IDX),IDX=NCRFREQ+1,NCRFREQ+MFREQ)|

Input for degenerate four wave mixing
$\gamma_{ABCD}(-\omega;\omega,\omega,-\omega)$:
on the first line following \Key{DFWMFR} the number of different
frequencies are read, from the second line the input for
$\omega_B = \omega$ is read. $\omega_C$ is set to $\omega$,
$\omega_D$ and $\omega_A$ to $-\omega$. 
\index{degenerate four wave mixing}\index{DFWM}
 
\item[\Key{DIPOLE}] 
Evaluate all symmetry allowed elements of the second dipole
hyperpolarizability (max. 81 components per frequency).

\item[\Key{DISPCF}] \verb| |\newline
\verb|READ (LUCMD,*) NCRDSPE| 

Calculate the dispersion coefficients $D_{ABCD}(l,m,n)$ up  to
$l+m+n = $ \verb+NCRDSPE+.
Note that dispersion coefficients presently are only available for
real fourth-order properties.
\index{dispersion coefficients}
 
% \item[\Key{ODDISP}]  % not yet implemented...
 
\item[\Key{ESHGFR}] \verb| |\newline
\verb|READ (LUCMD,*) MFREQ|\newline
\verb|READ (LUCMD,*) (BCRFR(IDX),IDX=NCRFREQ+1,NCRFREQ+MFREQ)|

Input for electric field induced second harmonic generation\index{ESHG}
$\gamma_{ABCD}(-2\omega;\omega,\omega,0)$:
on the first line following \Key{ESHGFR} the number of different
frequencies are read, from the second line the input for
$\omega_B = \omega$ is read. $\omega_C$ is set to $\omega$,
$\omega_D$ to $0$ and $\omega_A$ to $-2\omega$. 
\index{second harmonic generation}\index{second harmonic generation!electric field induced}\index{ESHG}
 
\item[\Key{L2 BC}] solve response equations for the second-order
Lagrangian multipliers $\bar{t}^{BC}$ instead of the equations for 
the second-order amplitudes $t^{AD}$.
 
\item[\Key{L2 BCD}] solve response equations for the second-order
Lagrangian multipliers $\bar{t}^{BC}$, $\bar{t}^{BD}$, $\bar{t}^{CD}$
instead of the equations for the second-order amplitudes
$t^{AD}$, $t^{AC}$, $t^{AB}$.
 
\item[\Key{MIXFRE}] \verb| |\newline
\verb|READ (LUCMD,*) MFREQ|\newline
\verb|READ (LUCMD,*) (BCRFR(IDX),IDX=NCRFREQ+1,NCRFREQ+MFREQ)|\newline
\verb|READ (LUCMD,*) (CCRFR(IDX),IDX=NCRFREQ+1,NCRFREQ+MFREQ)|\newline
\verb|READ (LUCMD,*) (DCRFR(IDX),IDX=NCRFREQ+1,NCRFREQ+MFREQ)|

Input for general frequency mixing
$\gamma_{ABCD}(\omega_A;\omega_B,\omega_C,\omega_D)$: on the first line
following \Key{MIXFRE} the number of differenct frequencies
is read and from the next three lines the frequency arguments 
$\omega_B$, $\omega_C$, and $\omega_D$ are read
($\omega_A$ is set to $-\omega_B-\omega_C$).
\index{general frequency mixing}
                                                           
\item[\Key{NO2NP1}] test option: switch off $2n+1$-rule for second-order
                    Cauchy vector equations.
 
\item[\Key{OPERAT}] \verb| |\newline
\verb|READ (LUCMD,'(4A)') LABELA, LABELB, LABELC, LABELD|\newline
\verb|DO WHILE (LABELA(1:1).NE.'.' .AND. LABELA(1:1).NE.'*')|\newline
\verb|  READ (LUCMD,'(4A)') LABELA, LABELB, LABELC, LABELD|\newline
\verb|END DO|

Read quadruples of operator labels.
For each of these operator quadruples the cubic response
function will be evaluated at all frequency triples.
Operator quadruples which do not correspond to symmetry allowed
combination will be ignored during the calculation. 

\item[\Key{PRINT}] \verb| |\newline
\verb|READ (LUCMD,*) IPRINT|

Set print parameter for the cubic reponse section.

\item[\Key{STATIC}] 
Add $\omega_A = \omega_B = \omega_C = \omega_D = 0$ to the frequency list.

\item[\Key{THGFRE}] \verb| |\newline
\verb|READ (LUCMD,*) MFREQ|\newline
\verb|READ (LUCMD,*) (BCRFR(IDX),IDX=NCRFREQ+1,NCRFREQ+MFREQ)|

Input for third harmonic generation
$\gamma_{ABCD}(-3\omega;\omega,\omega,\omega)$:
on the first line following \Key{THGFRE} the number of different
frequencies is read, from the second line the input for
$\omega_B = \omega$ is read. $\omega_C$ and $\omega_D$ are set to 
$\omega$ and $\omega_A$ to $-3\omega$. 
\index{third harmonic generation}\index{THG}
 
\item[\Key{USECHI}]
test option: use second-order $\chi$-vectors as intermediates
 
\item[\Key{USEXKS}] 
test option: use third-order $\xi$-vectors as intermediates
 
%\item[\Key{EXPCOF}]  % for expert use only !
%

\end{description}

%
%%%%%%%%%%%%%%%%%%%%%%%%%%%%%%%%%%%%%%%%%%%%%%%%%%%%%%%%%%%%%%%%%%%
\section{Calculation of excitation energies: \Sec{CCEXCI}}\label{sec:ccexci}
%%%%%%%%%%%%%%%%%%%%%%%%%%%%%%%%%%%%%%%%%%%%%%%%%%%%%%%%%%%%%%%%%%%
\index{linear response}
\index{response!linear}
\index{excitation energy!Coupled Cluster}
\index{excited state!Coupled Cluster}

In the \Sec{CCEXCI} section the input that is
specific for coupled cluster linear response calculation of
electronic excitation energies is given. 
Coupled cluster linear response excitation energies 
are implemented for the iterative CC models CCS, CC2, CCSD, and CC3 for 
both singlet and triplet excited states.
For singlet excited states the non-iterative models CC(2)(=CIS(D)) and CCSDR(3)
are also available.
%Publications that report results obtained with CC excitation energies
%should cite Ref.\ \cite{Christiansen:JCP105} for singlet excitation energies
%and \cite{Hald:JCP113} for triplet excitation energies.
For understanding the theoretical background for some
aspects of the CC3 calculations consult also Ref.\ \cite{Christiansen:JCP105,Hald:JCP113,Christiansen:JCP103,Hald:JCP115}.

\begin{center}
\fbox{
\parbox[h][\height][l]{12cm}{
\small
\noindent
{\bf Reference literature:}
\begin{list}{}{}
\item Singlet excitation energies: O.~Christiansen, H.~Koch, A.~Halkier, P.~J{\o}rgensen, T.~Helgaker, and A.~M.~Sanchez de Meras \newblock {\em J.~Chem.~Phys.}, {\bf 105},\hspace{0.25em}6921, (1996).
\item Triplet excitation energies: K.~Hald, C.~H\"{a}ttig, and P.~J{\o}rgensen \newblock {\em J.~Chem.~Phys.}, {\bf 113},\hspace{0.25em}7765, (2000).
\end{list}
}}
\end{center}



\begin{description}
\item[\Key{CC2PIC}] 
Functions as CCSPIC but the picking is based on CC2 excitation energies.
\item[\Key{CCSDPI}] 
Functions as CCSPIC but the picking is based on CCSD excitation energies.
%
\item[\Key{CCSPIC}] \verb| |\newline 
\verb|READ(LUCMD,*) OMPCCS| \newline
Keyword for picking a state with a given CCS excitation energy.
Optimize a number of states in CCS with different symmetries and this option
will pick the one in closest correspondence with the given input excitation
energy, and skip the states at higher energies and other symmetries in 
the following calculation.
Useful for example in numerical Hessian calculations on excited states.

\item[\Key{MARGIN}] 
Specifies the maximum allowed deviation of the actual excitation energy
from the input excitation energy when using CCSPIC, CC2PIC and CCSDPI. 

\item[\Key{NCCEXC}] \verb| |\newline
\verb|READ (LUCMD,*) (NCCEXCI(ISYM,1),ISYM=1,MSYM)|

Give the number of states desired.
For singlet states only, one single line is required with
the number of excitation energies for each symmetry class (max. 8). \\ 
If also triplet states are desired an additional line is given in same format. 
%
\item[\Key{NOSCOM}] 
%
For CC3 calculations only: indicates that no self-consistent solution
should be seeked in the partitioned CC3 algorithm.

\item[\Key{OMEINP}]\verb| |\newline
%
\verb|READ (LUCMD,*) (NOMINP(ISYM,1),ISYM=1,MSYM)|\newline
\verb|DO ISYM = 1, MSYM|\newline
\verb|  DO IOM = 1, NOMINP(ISYM,1)|\newline
\verb|     READ (LUCMD,*) IOMINP(IOM,ISYM,1),EOMINP(IOM,ISYM,1)|\newline
\verb|  ENDDO|\newline
\verb|ENDDO|

A way to provide an input omega for the partitioned CC3 algorithm or restrict
the self-consistent solution to specific states.
If OMEINP is not specified the program uses the best choice available to it at that
moment based on previous levels of approximations (CCSD or even better CCSDR(3)) 
and calculates all states as given by NCCEXC.
IOMPINP is 1 for the lowest excited state of a given symmetry, 2 for the second lowest etc. \\
By giving an 0.0 input excitation energy (as EOMINP) the program takes the best previous
approximation found in this run - otherwise the user can specify a qualified guess 
(perhaps from a previous calculation which is now restarted).

\item[\Key{THREXC}] 

The threshold for the solution of the excitation energies and
corresponding response eigenvectors. 
The threshold is the norm of the residual for the eigenvalue equation.
(Default: 1.0D-04). 

%
%\item[\Key{FDJAC}] 
%
%\item[\Key{FDEXCI}] 
%
%\item[\Key{JACEXP}] 
%
%\item[\Key{JACTST}] 
%
%\item[\Key{LHTR}] 
%
%\item[\Key{STSD}] 
%
\item[\Key{TOLSC}] 
For CC3 calculations only: 
Set tolerance for excitation energies for obtaining a self-consistent
solution to the partitioned CC3 algorithm.
Tolerance refers to the eigenvalue itself in the self-consistency iterations
of the default solver for CC3. Not used in DIIS solver (see \Key{R3DIIS}).
(Default: 1.0D-04).

\item[\Key{R3DIIS}]
Use DIIS solver for CC3.
% and for undocumented ccsdt-1a and ccsdt-1b
This solver only scales linearly with the number of excited states
in comparison to the default solver which scales quadratically.
However this solver might fail in cases with states dominated by
double or higher excited determinants. 
(Default: OFF).

%\item[\Key{STVEC}] 
%
%\item[\Key{STOLD}] 
%
\end{description}

%
%%%%%%%%%%%%%%%%%%%%%%%%%%%%%%%%%%%%%%%%%%%%%%%%%%%%%%%%%%%%%%%%%%%
\section{Ground state--excited state transition moments: \Sec{CCLRSD}}
\label{sec:cclrsd}
%%%%%%%%%%%%%%%%%%%%%%%%%%%%%%%%%%%%%%%%%%%%%%%%%%%%%%%%%%%%%%%%%%%
\index{linear response, coupled cluster}
\index{oscillator strength, coupled cluster}
\index{transition moment, coupled cluster}
\index{transition strength, coupled cluster}
\index{absorption strength, coupled cluster}


In the \Sec{CCLRSD} section the input that is
specific for coupled cluster response calculation of ground state--excited state
electronic transition properties is read in.
This section includes for example calculation of oscillator strength etc.
The transition properties are implemented for the models CCS, CC2 and CCSD
(singlet states only).
%Publications that report results obtained with this CC module should cite 
%Ref.\ \cite{Christiansen:CCLR}.
The theoretical background for the implementation is detailed in Ref.\ \cite{Christiansen:CCLR,Christiansen:QEL}.
This section \Sec{CCLRSD} has to be used in connection with \Sec{CCEXCI} 
for the calculation of excited states.

\begin{center}
\fbox{
\parbox[h][\height][l]{12cm}{
\small
\noindent
{\bf Reference literature:}
\begin{list}{}{}
\item Ground-state transition moments: O.~Christiansen, A.~Halkier, H.~Koch, P.~J{\o}rgensen, and T.~Helgaker \newblock {\em J.~Chem.~Phys.}, {\bf 108},\hspace{0.25em}2801, (1998).
\end{list}
}}
\end{center}

\begin{description}
\item[\Key{DIPOLE}] 

Calculate the ground state--excited state dipole (length) transition properties including
the oscillator strength.

\item[\Key{DIPVEL}] 

Calculate the ground state--excited state dipole-velocity  transition properties including
the oscillator strength in the dipole-velocity form. (The dipole length form is recommended
for standard calculations).

\item[\Key{NO2N+1}] 
 
Use an alternative, and normally less efficient, formulation for calculation
the transition matrix elements (involving solution of response equations for 
all operators instead of solving for the so-called $M$ vectors which is the default).

\item[\Key{OPERAT}] 

\noindent\verb|READ (LUCMD,'(2A8)') LABELA, LABELB|\newline

Read pairs of operator labels for which the residue of the linear response function is desired.
Can be used to calculate the transition property for a given operator
by specifying that operator twice. The operator can be any of the one-electron
operators for which integrals are available in the \Sec{*INTEGRALS} input part.




\item[\Key{SELEXC}] 

\noindent\verb|READ(LUCMD,*) IXSYM,IXST|
 
Select which excited states the calculation of transition properties
are carried out for. The default is all states according to the CCEXCI input section
(the program takes into account symmetry). For calculating selected states only,
provide a list of symmetry and state numbers (order after increasing energy in 
each symmetry class). This list is read until next input label is found.



\end{description}

%
%%%%%%%%%%%%%%%%%%%%%%%%%%%%%%%%%%%%%%%%%%%%%%%%%%%%%%%%%%%%%%%%%%%
\section{Ground state--excited state two-photon transition moments:
\Sec{CCSM}} \label{sec:ccsm}
%%%%%%%%%%%%%%%%%%%%%%%%%%%%%%%%%%%%%%%%%%%%%%%%%%%%%%%%%%%%%%%%%%%
\index{transition strength, two-photon}
\index{two-photon transition moment}
\index{transition moment, two-photon}
\index{transition moment, second-order}

This section describes the calculation of 
%second order 
two-photon transition strengths 
(two-photon dipole is a special case) 
within the Coupled Cluster response code.
The two-photon transition strength is defined
\[
S^{of}_{AB,CD}(\omega) = \frac{1}{2} \{ M^{AB}_{of}(-\omega) M^{CD}_{fo}(\omega)
                         +[M^{CD}_{of}(-\omega) M^{AB}_{fo}(\omega)]^\ast\}
\]
\Sec{CCSM} drives the calculation of the left ($M^{XY}_{of}(\omega)$)
and right ($M^{XY}_{fo}(\omega)$) transition moments, and of the transition 
strength $S^{of}_{AB,CD}(\omega)$.
The methodology is implemented for the CCS, CC2 and CCSD models.
Results obtained using this functionality should cite 
\cite{Haettig:MULTIPHOTON,Haettig:TWOPHOTON}.
\begin{description}
\item[\Key{OPERAT}] \verb| |\newline
\verb|READ (LUCMD,'(4A)') LABELA, LABELB, LABELC, LABELD|\newline
\verb|DO WHILE (LABELA(1:1).NE.'.' .AND. LABELA(1:1).NE.'*')|\newline
\verb|  READ (LUCMD,'(4A)') LABELA, LABELB, LABELC, LABELD|\newline
\verb|END DO|\newline
Read quadruples of operator labels.
For each of these operator quadruples the first residues of the quadratic response
function will be evaluated at all frequencies.
Operator pairs which do not correspond to symmetry allowed
combinations will be ignored during the calculation.
%
\item[\Key{DIPOLE}] 
Evaluate all symmetry allowed elements of the all--dipole tensor
of the first residues of quadratic response function
(81 components). In other words, all four operator labels 
are set equal to all possible cartesian components of 
the electric dipole moment operator (\verb+DIPLEN+ in Sec.~\Sec{*INTGRL}).
%
\item[\Key{PRINT }] \verb| |\newline
\verb|READ (LUCMD,*) IPRSM|\newline
Read print level. Default is 0.
%
\item[\Key{SELSTA}] \verb| | \newline
\verb|READ (LUCMD,'(A70)') LABHELP|\newline
\verb|DO WHILE (LABHELP(1:1).NE.'.' .AND. LABHELP(1:1).NE.'*')|\newline
\verb|  READ (LUCMD,'(3A)') IXSYM, IXST, SMFREQ|\newline
\verb|END DO| \newline
Select one or more excited states $f$ (among those specified
in \Sec{CCEXCI}), and the laser frequency $\omega$.
The symmetry (\verb+IXSYM+) and state number (\verb+IXST+)
within that symmetry are then given,
one pair (\verb|IXSYM,IXST|) per line, together with the
laser frequency \verb+SMFREQ+ (in atomic units).
%\verb+SMFREQ+ specifies the frequency of the two-photon 
%transition moment (in atomic units).

Default is all states specified in \Sec{CCEXCI}, and for each state 
a laser frequency equal to half the excitation energy.

%
\item[\Key{HALFFR}] 
Set the frequency argument for the two-photon transition moments
equal to  half the excitation energy to the final state $f$. Default,
if \verb+SMFREQ+ is not specified in \Key{SELSTA}.
%
\end{description}

%
%%%%%%%%%%%%%%%%%%%%%%%%%%%%%%%%%%%%%%%%%%%%%%%%%%%%%%%%%%%%%%%%%%%
\section{Ground state--excited state three-photon 
transition moments: \Sec{CCTM}}\label{sec:cctm}
%%%%%%%%%%%%%%%%%%%%%%%%%%%%%%%%%%%%%%%%%%%%%%%%%%%%%%%%%%%%%%%%%%%
\index{transition moment!three-photon}
\index{transition moment!third-order}
\index{transition strength!three-photon}
\index{transition strength!third-order}
\index{three-photon!transition moment}

This section describes the calculation of third-order transition
moments and strengths. Three photon transition strengths are defined as
{\small \[
S^{of}_{ABC,DEF}(\omega_1,\omega_2) = \frac{1}{2} 
       \{ M^{ABC}_{of}(-\omega_1,-\omega_2) M^{DEF}_{fo}(\omega_1,\omega_2)
        +[M^{DEF}_{of}(-\omega_1,-\omega_2) M^{ABC}_{fo}(\omega_1,\omega_2)]^\ast\}
\] }
where $M^{ABC}_{of}(\omega_1,\omega_2)$ and $M^{ABC}_{fo}(\omega_1,\omega_2)$
are the left and right three photon transition moments, respectively.
The methodology is implemented for the CC models CCS, CC2 and CCSD.
%Publications reporting results obtained with this module should 
%cite~\cite{Haettig:MULTIPHOTON}.

\begin{center}
\fbox{
\parbox[h][\height][l]{12cm}{
\small
\noindent
{\bf Reference literature:}
\begin{list}{}{}
\item C.~H\"{a}ttig, O.~Christiansen, and P.~J{\o}rgensen \newblock {\em J.~Chem.~Phys.}, {\bf 108},\hspace{0.25em}8331, (1998).
\end{list}
}}
\end{center}

\begin{description}
\item[\Key{DIPOLE}] \verb| |\newline
Calculate the three-photon moments and strengths for all possible
combinations of Cartesian 
components of the electric dipole moment operator (729 combinations).
\item[\Key{OPERAT}] \verb| |\newline
\verb|READ (LUCMD,'(6A8)') LABELA, LABELB, LABELC, LABELD, LABELE, LABELF|\newline
\verb|DO WHILE (LABELA(1:1).NE.'.' .AND. LABELA(1:1).NE.'*')|\newline
\verb|   READ (LUCMD,'(6A8)') LABELA, LABELB, LABELC, LABELD, LABELE, LABELF|\newline
\verb|ENDDO|\newline   
Select the sextuples of operator labels for which to calculate the three-photon transition
moments. Operator sextuples which do not correspond to symmetry-allowed combinations will be
ignored during the calculation.
Use exactly 8 characters for each label,
this means that if you should want e.g LABELA='XXROTSTR', LABELB='YYROTSTR', LABELC='XANGMOM ', LABELD='XYROTSTR',
LABELE=' YDIPLEN', and LABELF='ZZROTSTR'
you must enter\newline
\verb|XXROTSTRYYROTSTRXANGMOM XYROTSTRYDIPLEN ZZROTSTR|

\item[\Key{PRINT}] \verb| |\newline
\verb|READ (LUCMD,*) IPRTM|\newline
Read print level. Default is 0.
\item[\Key{SELSTA}] \verb| | \newline
\verb|READ (LUCMD,'(A70)') LABHELP|\newline
\verb|DO WHILE (LABHELP(1:1).NE.'.' .AND. LABHELP(1:1).NE.'*')|\newline
\verb|  READ (LUCMD,*) IXSYM, IXST, FREQB, FREQC|\newline
\verb|END DO| \newline
Select one or more excited states $f$ (among those specified
in \Sec{CCEXCI}), and the laser frequencies.
The symmetry (\verb+IXSYM+) and state number (\verb+IXST+)
within that symmetry are then given,
one pair (\verb|IXSYM,IXST|) per line.
\verb+FREQB+ and \verb+FREQC+ specify the 
laser frequencies $\omega_1$ and $\omega_2$ (in atomic units).


Default is all states specified in \Sec{CCEXCI}, and for each state
both laser frequencies equal to one third of the excitation energy.

%
\item[\Key{THIRDF}]
Set the frequency arguments for the three-photon transition moments
equal to one third of the excitation energy of the (chosen) 
final state $f$. Default, if \verb+FREQB,FREQC+ are not specified 
in \Key{SELSTA}.
%
\end{description}

%
%%%%%%%%%%%%%%%%%%%%%%%%%%%%%%%%%%%%%%%%%%%%%%%%%%%%%%%%%%%%%%%%%%%
\section{Magnetic circular dichroism: \Sec{CCMCD}}\label{sec:ccmcd}
%%%%%%%%%%%%%%%%%%%%%%%%%%%%%%%%%%%%%%%%%%%%%%%%%%%%%%%%%%%%%%%%%%%
\index{magnetic circular dichroism}
\index{MCD}
\index{transition strengths}
\index{transition moments, one-photon}
\index{transition moments, two-photon}
\index{B-term}

This Section deals with the calculation of the 
${\cal{B}}(o\!\to\!f)$ term of magnetic circular dichroism,
for a dipole-allowed transition between the ground state $n$ and 
the excited state $f$.
The ${\cal{B}}(o\to f)$ term results from an appropriate combination of
products of electric dipole one-photon transition moments
and mixed electric dipole-magnetic dipole
two-photon transition moments,
or, more precisely, a combination of transition strengths
$S_{AB,C}^{of}(0)$.
These are related to the single residues of the (linear and) 
quadratic response functions~\cite{Coriani:MCDRSP,Coriani:MOACC,Coriani:PHD}:
\[
{\cal{B}}(o\to f) = \varepsilon_{\alpha\beta\gamma} ETC
\]
%
%re corresponds to an appropriate 
%combination of single residues of quadratic response functions, 
%or, more precisely, a combination of transition strengths
%$S_{AB,C}^{of}(0.0)$.
\noindent
Within \Sec{CCMCD} we specify the keywords needed in order
to compute the individual 
contributions for each chosen transition $(o\to f)$ to the
${\cal{B}}(o\to f)$ term. \Sec{CCMCD} has to be used in 
connection with \Sec{CCEXCI} for the calculation of 
excited states.
The calculation is implemented for the models CCS, CC2 and CCSD.

%\noindent Publications that report results obtained by this module
%should cite Ref.\ \cite{Coriani:MOACC}.

\begin{center}
\fbox{
\parbox[h][\height][l]{12cm}{
\small
\noindent
{\bf Reference literature:}
\begin{list}{}{}
\item S.~Coriani, C.~H\"{a}ttig, P.~J{\o}rgensen, and T.~Helgaker \newblock {\em J.~Chem.~Phys.}, {\bf 113},\hspace{0.25em}3561, (2000).
\end{list}
}}
\end{center}

\begin{description}
\item[\Key{MCD   }] 
All six triples of operators obtained from the
permutations of 3 simultaneously different 
cartesian components of the 3 operators 
\verb+LABELA,LABELB,LABELC+ are automatically specified.
Operator triples which do not correspond to symmetry allowed
combination will be ignored during the calculation.
Default.

\item[\Key{OPERAT}] \verb| |\newline
\verb|READ (LUCMD,'(3A)') LABELA, LABELB, LABELC|\newline
\verb|DO WHILE (LABELA(1:1).NE.'.' .AND. LABELA(1:1).NE.'*')|\newline
\verb|  READ (LUCMD,'(3A)') LABELA, LABELB, LABELC|\newline
\verb|END DO|
%

Manually select the operator label triplets defining the operator
cartesian components involved in the ${\cal B}$ term. 
%, or, in general, in the $S^of_AB,C(0)$ strength.
The first two operators \verb+(LABELA,LABELB)+
are those that enter the two-photon moment.
The third operator \verb+(LABELC)+ 
is the one that enters the one-photon 
moment. They could be any of the (one-electron)
operators for which integrals are available in 
\Sec{*INTEGRALS}.
Specifically for the ${\cal{B}}$ term, \verb+LABELA+
corresponds to any component of the electric dipole 
moment operator, \verb+LABELB+ a component of the
angular momentum (magnetic dipole) and
\verb+LABELC+ to a component of the electric dipole.
Operator triples which do not correspond to symmetry allowed
combination will be ignored during the calculation.

%In case we need to specify relaxed/derivative operators
%the \Key{OPERAT} input is slightly different:
%\item[\Key{OPERAT}] \verb| |\newline
%\verb|READ (LUCMD'(3A)') LABELA, LABELB, LABELC|\newline
%\verb|DO WHILE (LABELA(1:1).NE.'.' .AND. LABELA(1:1).NE.'*')|\newline
%\verb|  READ (LUCMD'(3A)') (UNREL), (RELAX), (UNREL)|\newline
%\verb|  READ (LUCMD'(3A)') LABELA,  LABELB,  LABELC|\newline
%\verb|END DO|
%where \verb+(UNREL)+ refers to unrelaxed operators and 
%      \verb+(RELAX)+to relaxed operators.
%
%\item[\Key{MCDLAO}] 
%
% This keyword turns on the calculation of the B term components
% using London atomic orbitals or the relaxed angular momentum 
% operator. Unfinished.
%
%\item[\Key{NO2N+1}] 
%
%Disable the use of the $\bar{M}^f$ vectors in the one-photon transition
%moment part (does not use $2n+1$ rule).
%Disabled
\item[\Key{PRINT }] \verb| |\newline
\verb|READ (LUCMD,*) IPRINT |\newline
Sets the print level in the \Sec{MCDCAL} section. Default is \verb+IPRINT=0+.
%
\item[\Key{SELSTA}] \verb| |\newline 
\verb|READ (LUCMD,'(A70)') LABHELP|\newline
\verb|DO WHILE (LABHELP(1:1).NE.'.' .AND. LABHELP(1:1).NE.'*')|\newline
\verb|   READ(LABHELP,*) IXSYM,IXST|\newline
\verb|END DO|

Manually select one or more given excited states $f$ among those specified
in \Sec{CCEXCI}. 
%The maximum number of states that can be selected is
%set in the ccmcd.h common block.
The symmetry (\verb+IXSYM+) and state number (\verb+IXST+)
within that symmetry are then given,
one pair (\verb|IXSYM,IXST|) per line.
Default is all dipole allowed states for all symmetries 
specified in \Sec{CCEXCI}.
%
%\item[\Key{RELAXE}] 
%
% Disabled
%
%\item[\Key{UNRELA}] 
%
% Disabled
%
%\item[\Key{USEPL1}] 
%
%To be used in connection with LAO-relaxed angular momentum operators.
%
\end{description}

%
%%%%%%%%%%%%%%%%%%%%%%%%%%%%%%%%%%%%%%%%%%%%%%%%%%%%%%%%%%%%%%%%%%%
\section{Transition moments between two excited states: \Sec{CCQR2R}}
\label{sec:ccqr2r}
\index{Coupled Cluster!quadratic response}
\index{quadratic response}
\index{response!quadratic}
\index{oscillator strength!excited states!Coupled Cluster}
\index{transition moment!excited states!Coupled Cluster}
\index{transition strength!excited states!Coupled Cluster}
%%%%%%%%%%%%%%%%%%%%%%%%%%%%%%%%%%%%%%%%%%%%%%%%%%%%%%%%%%%%%%%%%%%

In the \Sec{CCQR2R} section the input that is
specific for coupled cluster response calculation of excited state--excited state
electronic transition properties is read in.
This section includes presently for example calculation of excited state
oscillator strength.
The transition properties are available for the models CCS, CC2 and CCSD,
but only for singlet states.
%Publications that report results obtained with this CC module should cite Ref.\ \cite{Christiansen:CCLR}.
The theoretical background for the implementation is 
detailed in Ref.\ \cite{Christiansen:CCLR,Christiansen:QEL}.
This section has to be used in connection with \Sec{CCEXCI} for the 
calculation of excited states.

\begin{center}
\fbox{
\parbox[h][\height][l]{12cm}{
\small
\noindent
{\bf Reference literature:}
\begin{list}{}{}
\item O.~Christiansen, A.~Halkier, H.~Koch, P.~J{\o}rgensen, and T.~Helgaker \newblock {\em J.~Chem.~Phys.}, {\bf 108},\hspace{0.25em}2801, (1998).
\end{list}
}}
\end{center}

\begin{description}

\item[\Key{DIPOLE}]
%
Calculate the ground state--excited state dipole (length) transition properties including
the oscillator strength.
%
\item[\Key{DIPVEL}]
%
Calculate the excited state--excited state dipole-velocity  transition properties including
the oscillator strength in the dipole-velocity form. (The dipole length form is recommended
for standard calculations).
%
\item[\Key{NO2N+1}]
%
Use an alternative, and normally less efficient, formulation for calculating
the transition matrix elements (involving solution of response equations for
all operators instead of solving for the so-called $N$ vectors which is the default).
%
\item[\Key{OPERAT}]\verb| |  \newline
\verb|READ (LUCMD,'(2A8)') LABELA, LABELB|\newline
%
Read pairs of operator labels for which the residue of the linear response
function is desired.
Can be used to calculate the transition property for a given operator
by specifying that operator twice. The operator can be any of the one-electron
operators for which integrals are available in the \Sec{*INTEGRALS} input part.
%
\item[\Key{SELEXC}] \verb| | \newline
\verb|READ(LUCMD,*) IXSYM1,IXST1,IXSYM2,IXST2|\newline 
%
Select which excited states the calculation of transition properties
are carried out for. The default is all states according to the CCEXCI input section
(the program takes into account symmetry). For calculating selected states only,
provide a list of symmetry and state numbers (order after increasing energy in
each symmetry class) for state one and for state two.
This list is read until next input label is found.

\end{description}


%
%%%%%%%%%%%%%%%%%%%%%%%%%%%%%%%%%%%%%%%%%%%%%%%%%%%%%%%%%%%%%%%%%%%
\section{Excited state first-order properties: \Sec{CCEXGR}}\label{sec:ccexgr}
%%%%%%%%%%%%%%%%%%%%%%%%%%%%%%%%%%%%%%%%%%%%%%%%%%%%%%%%%%%%%%%%%%%
\index{expectation values, excited states, coupled cluster}
\index{one-electron properties, excited states, coupled cluster}
\index{first-order properties, excited states, coupled cluster}

In the \Sec{CCEXGR} section the input that is specific for coupled cluster 
response calculation of excited-state first-order properties is read in.
This section includes presently for example calculation of excited state 
dipole moments and second moments of the electronic charge distribution.
In many cases the information generated in this way is helpful 
for making qualitative assignments of the electronic states, for example
in conjunction with the oscillator strengths and the 
orbital analysis of the response eigenvectors presented in the output.

The excited state properties are available for the models CCS, CC2 and CCSD,
but only for singlet states.
%Publications that report results obtained this module should cite Ref.\ \cite{Christiansen:CCLR}.
The theoretical background for the implementation is detailed in Ref.\ \cite{Christiansen:CCLR,Christiansen:QEL}.
This section has to be used in connection with \Sec{CCEXCI} for the calculation of excited states.

The properties calculated are in the approach now generally known as coupled cluster
response - for these frequency independent properties this coincides with the so-called
orbital-unrelaxed energy derivatives (and thus the orbital-unrelaxed finite-field result)
for the excited-state total energies as obtained by the sum of the CC ground state energy
and the CC response excitation energy.

{\bf Note of caution:}
Default in this section is therefore orbital {\it unrelaxed}, while for the ground state 
first order properties \Sec{CCFOP} default is {\it relaxed}. 
To find results in a consistent approximation turn orbital relaxation off
for the ground state (for CCS, CC2, CCSD) calculation.

\begin{center}
\fbox{
\parbox[h][\height][l]{12cm}{
\small
\noindent
{\bf Reference literature:}
\begin{list}{}{}
\item O.~Christiansen, A.~Halkier, H.~Koch, P.~J{\o}rgensen, and T.~Helgaker \newblock {\em J.~Chem.~Phys.}, {\bf 108},\hspace{0.25em}2801, (1998).
\end{list}
}}
\end{center}

\begin{description}
\item[\Key{ALLONE}]
        Calculate all of the above-mentioned excited state properties (all the
        above-mentioned property integrals are needed).
%
\item[\Key{DIPMOM}]
        Calculate the excited state permanent molecular electric dipole moment
        (\verb+DIPLEN+ integrals).
        \index{electric dipole}
        \index{dipole moment}
%
\item[\Key{NQCC  }]
        Calculate the excited state electric field gradients at the nuclei
        (\verb+EFGCAR+ integrals).
        \index{electric field gradient}
%
\item[\Key{OPERAT}] \verb| |\newline
\verb|READ (LUCMD,'(1X,A8)') LABPROP|\newline
        Calculate the excited state electronic contribution to the property defined
        by the operator label \verb+LABPROP+ (corresponding
        \verb+LABPROP+ integrals needed).

\item[\Key{QUADRU}]
        Calculate the excited state permanent traceless molecular electric
        quadrupole moment (\verb+THETA+ integrals). Note that the
        origin is the origin of the coordinate system specified
        in the MOLECULE.INP file.
        \index{electric quadrupole}
        \index{quadrupole moment}
%
\item[\Key{RELCOR}]
        Calculate excited state scalar-relativistic one-electron
        corrections to the 
        energy (\verb+DARWIN+ and \verb+MASSVELO+ integrals).
        \index{relativistic corrections, one-electron}
        \index{Darwin term, one-electron}
        \index{mass-velocity term}
%
\item[\Key{SECMOM}]
        Calculate the excited state electronic second moment of charge
        (\verb+SECMOM+ integrals).
        \index{second moment of charge}
%

\item[\Key{SELEXC}] 

\noindent\verb|READ(LUCMD,*) IXSYM,IXST|

Select which excited states the calculation of excited state properties 
are carried out for. The default is all states according to the CCEXCI input section.
When calculating selected states only,
provide a list of symmetry and state numbers (order after increasing energy in
each symmetry class).
This list is read until next input label is found.


%\item[\Key{SELXST}] 
%Effect unknown at this moment... - in fact it is obsolete and without any effect.

\end{description}


%
%%%%%%%%%%%%%%%%%%%%%%%%%%%%%%%%%%%%%%%%%%%%%%%%%%%%%%%%%%%%%%%%%%%
\section{Excited state linear response functions and
         two-photon transition moments between two excited states:
         \Sec{CCEXLR}}
\label{sec:ccexlr}
%%%%%%%%%%%%%%%%%%%%%%%%%%%%%%%%%%%%%%%%%%%%%%%%%%%%%%%%%%%%%%%%%%%

In the \Sec{CCEXLR} section input that is specific for 
second residues of coupled cluster cubic response functions 
is read in.
This section includes:
\begin{itemize}
\item frequency-depend second-order properties of excited states
      $$ \alpha^{(i)}_{AB}(\omega) = 
         -\langle\langle A; B\rangle\rangle^{(i)}_\omega $$
      where $A$ and $B$ can be any of the one-electron operators
      for which integrals  are available in the \Sec{*INTEGRALS}
      input part.
\item two-photon transition moments between two excited states.
\end{itemize}
Second residues of coupled cluster cubic response functions are
implemented for the models CCS, CC2, and CCSD.
Publications that report results obtained with second residues
of CC cubic response functions should cite Ref.\ \cite{Haettig:EXLR}.

\begin{description}
\item[\Key{OPERAT}] \verb| |\newline
\verb|READ (LUCMD,'(2A)') LABELA, LABELB|\newline
\verb|DO WHILE (LABELA(1:1).NE.'.' .AND. LABELA(1:1).NE.'*')|\newline
\verb|  READ (LUCMD,'(2A)') LABELA, LABELB|\newline
\verb|END DO|

Read pairs of operator labels.
For each of these operator pairs the second residues of the cubic response
function will be evaluated at all frequencies.
Operator pairs which do not correspond to symmetry allowed
combination will be ignored during the calculation.
 
\item[\Key{DIPOLE}] 
Evaluate all symmetry allowed elements of the dipole--dipole tensor
of the second residues of cubic response function 
(max. 6 components for second-order properties, 
 max. 9 for two-photon transition moments).
 
\item[\Key{SELSTA}] \verb| |\newline
\verb|READ (LUCMD,'(A80)') LABHELP|\newline
\verb|DO WHILE(LABHELP(1:1).NE.'.' .AND. LABHELP(1:1).NE.'*')|\newline
\verb|  READ(LUCMD,*) ISYMS(1), IDXS(1), ISYMS(2), IDXS(2)|\newline
\verb|END DO|

Read symmetry and index of the initial state and the final state.
If initial and final state coincide one obtains excited state
second-order properties, if they are different one obtains the
two-photon transition moments between the two excited states.
 
\item[\Key{PRINT }] \verb| |\newline
\verb|READ (LUCMD,*) IPRINT|

Set print parameter for the \Sec{CCEXLR} section.
 
\item[\Key{ALLSTA}] 
calculate polarizabilities for all excited states.
 
\item[\Key{HALFFR}] 
Use half the excitation energy as frequency argument for two-photon
transition moments.
Note, that the \Key{HALFFR} keyword is incompatible with a 
user-specified list of frequencies. \\
For excited state second-order properties the \Key{HALFFR} keyword is
equivalent to the \Key{STATIC} keyword.
 
\item[\Key{USELEF}] 
Use left excited state response vectors instead of the right excited
state response vectors (default is to use the right excited state
response vectors).
 
\item[\Key{FREQ  }]  \verb| |\newline
%  or \Key{FREQUE}
\verb|READ (LUCMD,*) MFREQ|\newline
\verb|READ (LUCMD,*) (BEXLRFR(IDX),IDX=NEXLRFR+1,NEXLRFR+MFREQ)|

Frequency input for $\alpha^{(i)}_{AB}(\omega)$.
 
\item[\Key{STATIC}] 
Add $\omega = 0$ to the frequency list.
 
\end{description}

%%\include{ccslv}
%%
%%%%%%%%%%%%%%%%%%%%%%%%%%%%%%%%%%%%%%%%%%%%%%%%%%%%%%%%%%%%%%%%%%
\section{Numerical Gradients}\label{sec:ccgr}
%%%%%%%%%%%%%%%%%%%%%%%%%%%%%%%%%%%%%%%%%%%%%%%%%%%%%%%%%%%%%%%%%%%
%
\index{numerical derivatives, coupled cluster}

This section is used in the calculation of numerical derivatives
of the CC energy. Since it is numerical it can be used for all models
and both ground and excited states.

For excited states there is the problem of specifying 
which excited state is to be studied - and keeping track of this.
One can specify the excited state by symmetry and number, using the keywords below.
This works fine for gradients (though ordering may change in course of the
optimization, but excited state optimization will inevitably be less black box
than ground state optimizations), but can fail when there is symmetry lowering
in the calculation of numerical hessians. 
For this purpose one can give the excitation energy for the appropriate state at a lower level 
(for example CCS) and from that find symmetry and number used in the real higher level calculation.
%Solves partially the problem in (i). 
It is implemented by the keywords in the CCEXCI section.

\begin{description}

\item[\Key{NUMGD }] 

Specify the the calculation of gradients is to be done numerically.
All that is required for coupled cluster (in addition to appropriate minimization
keywords see Chapter~\ref{ch:geometrywalks}).

\item[\Key{XSTNUM}] 

\verb|READ (LUCMD,*) IXSTAT|\newline

The number of the excited state for which the  gradient is to be calculated 
(counted in terms of increasing energy).

\item[\Key{XSTSYM}] 

\verb|READ (LUCMD,*) IXSTSY|\newline

Symmetry for excited state for which gradient is to be calculated.


\end{description}

%%
%%%%%%%%%%%%%%%%%%%%%%%%%%%%%%%%%%%%%%%%%%%%%%%%%%%%%%%%%%%%%%%%%%
\section{R12 methods: \Sec{R12}}\label{sec:r12}
%%%%%%%%%%%%%%%%%%%%%%%%%%%%%%%%%%%%%%%%%%%%%%%%%%%%%%%%%%%%%%%%%%%
%
\index{R12 theory!MP2-R12 method!cusp correction!coupled cluster}

The calculation of MP2-R12 energy corrections is requested. Note that an
integral-direct calculation must be carried out and that the
key \Key{R12} must be specified in the \Sec{*INTEGRALS} section.

\begin{center}
\fbox{ \parbox[h][\height][l]{12cm}{\small\noindent
{\bf Reference literature:}
\begin{list}{}{}
\item W.~Klopper and C.~C.~M.~Samson,
\newblock {\em J.~Chem.~Phys.\/} {\bf 116}, 6397 (2002).
\item C.~C.~M.~Samson, W.~Klopper and T.~Helgaker,
\newblock {\em Comp.~Phys.~Commun.\/} {\bf 149}, 1 (2002).
\end{list} }} \end{center}

\begin{description}

\item[\Key{NO 1}]

Results for Ansatz 1 of the MP2-R12 method are not computed.

\item[\Key{NO 2}]

Results for Ansatz 2 of the MP2-R12 method are not computed.

\item[\Key{NO A}]

Results for approximation A of the MP2-R12 method are not printed.

\item[\Key{NO A'}]

Results for approximation A$^\prime$ of the MP2-R12 method are not printed.

\item[\Key{NO B}]

Results for approximation B are not computed and not printed.

\item[\Key{NO RXR}]

Those extra terms are ignored, which occur in approximation B when an auxiliary basis set
is invoked for the resolution-of-identity (RI) approximation. If they are included, 
the results are marked as B$^\prime$.

\item[\Key{NO HYB}]

The default MP2-R12 calculation implemented in the Dalton program
avoids two-electron integrals that involve
two or more basis functions of the auxiliary basis (of course, only
if such a basis is employed). This approach is denoted
as hybrid scheme between approximations A and B. To obtain the
full MP2-R12 energy in approximation B, the keyword \verb|.NO HYB| must
be specified. Then, two-electron integrals with up to two
auxiliary basis functions are calculated but the calculation becomes 
more time-consuming.

\item[\Key{R12DIA}]
The MP2-R12 equations are solved by diagonalizing the matrix representation 
of the Fock operator in the basis of R12 double replacements. If 
negative eigenvalues occur,
a warning is issued. When this happens, the results should not be trusted,
since the RI approximation appears to be insufficiently accurate.
This diagonalizing is the default.

\item[\Key{R12SVD}]
The MP2-R12 equations are solved by single value decomposition
(the use of this keyword is not recommended).

\item[\Key{R12XXL}]

All possible output from the MP2-R12 approach is generated (the 
use of this keyword is not recommended).

\item[\Key{SVDTHR}]  \verb| | \newline
\verb|READ (LUCMD,*) SVDTHR|\newline
Threshold for singular value decomposition (default = $10^{-12}$).

\item[\Key{VCLTHR}]  \verb| | \newline
\verb|READ (LUCMD,*) SVDTHR|\newline
Threshold for neglect of R12 terms (default = 0, neglecting nothing).

\end{description}

%%%%%%% end of CC %%%%%%%
%\part{Appendix: \lsdalton\ Tool box}
%\section*{Appendix: DALTON Tool box}

This appendix describes the pre- and postprocessing programs
supplied to us by various users and authors. The programs are designed
to create files directly useable by the \siraba\ program, or do
various final analyses of one or more \siraba\ output files. We
strongly encourage users to supply us with any such programs they have
written in connection with their own use of the \siraba\ program,
which we will be glad to distribute along with the \siraba\
program.

This appendix gives a short description of the programs supplied with
the current distribution of the program, with proper references and a
short description of the use of the program.

Executable versions of the programs in this directory will
automatically be produced during installation of the \siraba\ program,
but they will by default not be installed in the installation
directory of the \siraba\ program, but rather placed in the
\verb|tools| directory.

\subsection*{A1: FChk2HES}

\noindent
{\large\bf Author: \normalsize\large Gilbert Hangartner,
Universit\"{a}t Freiburg, Switzerland}

\smallskip

\noindent 
{\bf Purpose:} Reads the Formatted Checkpoint file of Gaussian  
     and creates a \verb|DALTON.HES| file with Hessian and Atomic coordinates
     as well as a dummy \verb|MOLECULE.INP| file.

\smallskip
\noindent
{\bf Commandline:} FChk2HES [-n] [filename]

-n does not read and write geometry; filename is optional, default is
"\verb|Test.FChk|" .

\smallskip
\noindent
{\bf Comments:}     The Gaussian checkpointfile "Test.FChk" is created
     with the keyword "FormCheck=ForceCart", and where the keyword "NoSymm"
     has  been specified

If symmetry is used: Hessian\index{Hessian} will be in standard(Gaussian) orientation,
        but geometry will be in input orientation. So do not use the
        \verb|MOLECULE.INP| file from this utility, 
        neither the coordinates in the end of \verb|DALTON.HES|.
        Use either the standard-orientation from Gaussian outputfile, or
        use the converted checkpointfile (created with the command
        "formchk checkpointfilename convertedfilename") as an input for
        this program. Anyway, this will not lead to the geometry you 
        specified in the input, and the Hessian will be incompatible with the
        calculation you want to do in \siraba. (If this was already carried
        out at the input orientation ...)

The only way to get the geometry orientation you want is to turn
        off symmetry. 
        Then, both the Hessian and the geometry in the Test.FChk file are
        in the input orientation, and everthing is fine.

The program has been made for the purpose of being used in VCD or
VROA\index{VCD}\index{ROA}\index{vibrational circular
dichroism}\index{Raman optical activity} 
calculations where it may be of interest to compare predicted spectra
obtained from SCF London orbital invariants/AATs\index{atomic axial
tensor}\index{AAT} with a MP2 or DFT force field.

\subsection*{A2: labread}

\noindent
{\large\bf Author: \normalsize\large Hans J\o rgen Aa.Jensen,
Odense University, Denmark}

\smallskip

\noindent 
{\bf Purpose:} Reads unformatted AO-integral files and prints the
INTEGRAL labels found on the file

\smallskip
\noindent
{\bf Commandline:} labread $<$ infile $>$ outfile

\smallskip
\noindent
{\bf Comments:}  A convenient tool in connection with \siraba\ errors
connected to missing labels on a given file in order to check that the
given integrals do indeed, or do not, exist on a given file (usually
\verb|AOPROPER|). 

\subsection*{A3: ODCPROG}

\noindent
{\large\bf Author: \normalsize\large Antonio Rizzo (Istituto di
Chimica Quantistica ed Energetica Molecolare, Pisa, Italy), and
Sonia Coriani (University of Trieste, Italy)}

\smallskip

\noindent 
{\bf Purpose:} Analyze the magnetizability and nuclear shielding
polarizabilities from a set of finite-field magnetizability and
nuclear shielding calculations.

\smallskip
\noindent
{\bf Commandline:} odcprog

Requires the existence of a readmg.dat file, and $n$ \verb|DALTON.CM| files,
where $n$ is the number of finite field output files.

\smallskip
\noindent
{\bf readmg.dat:} 

\begin{verbatim}
TITLE (1 line)
N ZPRT IPG ISSYM NIST NLAST NSTEP
FILE1
FILE2
... 
FILEN 
\end{verbatim}
where
\begin{itemize}
\item[N] Number of output DALTON.CM files for the FF calculations
\item[ZPRT]  Sets the print level (T for maximum print level, F otherwise)
\item[IPG] Point group (1=Td, 2=Civ, 3=D2h, 4=C2v, 5=C3v, 6=Dih)
\item[ISSYM] Site symmetry of the atom for which the shielding
               polarizabilities are required
\item[NIST] First atom of which shielding polarizabilities must be computed
               (of those in  the \verb|DALTON.CM| files)
\item[NLAST] Last atom of which shielding polarizabilities must be computed
               (of those in  the \verb|DALTON.CM| files)
\item[NSTEP] Step to go from nist to nlast in the do-loop for
               shielding polarizabilities
\item[FILEn] $n$th \verb|DALTON.CM| file (current name and location)
\end{itemize}

\smallskip
\noindent
{\bf Comments:} 
    Only 6 point groups are presently implemented, of which
    C3v only for shielding-polarizabilities and 
    Dih only for hyperpolarizabilities 

    When \verb|issym.ne.ipg| and both
    hypermagnetizabilities\index{magnetizability
    polarizability}\index{shielding polarizability} and shielding 
    polarizabilities are required, the number of field set-ups should 
    be equal to the one for shieldings

    Different field set-ups are needed according to
    molecular  and/or nuclear site symmetries
    See for reference Raynes and Ratcliffe~\cite{wtrrrmp37}.
    Linear molecules along the Z axis, planar molecules
    on XZ plane. In general, follow standard point
    symmetry arrangements

\begin{itemize}
\item[T$_d$]   Eg. CH$_4$ - No symmetry - 3 calculations

\begin{verbatim}
Need     0  0  0
Need     0  0  z
Need     x  0  z
\end{verbatim}

\item[C$_{\infty v}$] Eg. CO - No symmetry - 5 calculations

\begin{verbatim}
Need     0  0  0
Need     x  0  0
Need     0  0  z
Need     0  0 -z
Need     x  0  z
\end{verbatim}

\item[D$_{2h}$] Eg. C$_2$H$_4$ -  No symmetry - 7 calculations

\begin{verbatim}
Need     0  0  0
Need     0  0  z
Need     x  0  0
Need     0  y  0
Need     x  y  0
Need     x  0  z
Need     0  y  z
\end{verbatim}

\item[C$_{2v}$] Eg. H$_2$O - No symmetry - 8 calculations

\begin{verbatim}
Need     0  0  0
Need     0  0  z
Need     0  0 -z
Need     x  0  0
Need     0  y  0
Need     x  y  0
Need     x  0  z
Need     0  y  z
\end{verbatim}

\item[C$_{3v}$] Eq. Shielding H1 in CH$_4$

\begin{verbatim}
Need     0  0  0
Need     0  0  z
Need     0  0 -z
Need     x  0  0
Need    -x  0  0
Need     x  0  z
\end{verbatim}

\item[D$_{\infty h}$] Eg. N$_2$ -  4 calculations

\begin{verbatim}
Need     0  0  0
Need     0  0  z
Need     x  0  0
Need     x  0  z
\end{verbatim}
\end{itemize}

%\subsection*{A4: arturo}
%\noindent
%{\large\bf Author: \normalsize\large Asger Halkier (Aarhus University, Denmark)
%and Sonia Coriani (University of Trieste, Italy)}
%
%\smallskip
%
%\noindent
%{\bf Purpose:} Performs central difference analysis of (finite
%field) energy calculations with \cc.
%
%\smallskip
%\noindent
%{\bf Commandline:} arturo
%
%Requires the existence of 
%
%\smallskip
%\noindent

%\part{References}
\bibliography{dalton}
%
\part{Index}
\printindex
\end{document}
